%# -*- coding:utf-8 -*-
%%%%%%%%%%%%%%%%%%%%%%%%%%%%%%%%%%%%%%%%%%%%%%%%%%%%%%%%%%%%%%%%%%%%%%%%%%%%%%%%%%%%%
%%  `MAPreface-I.tex'

\begin{preface}
这套教材分三册出版。第一、二册讲述一元函数微积分,它的内容包括:实数、极限理论,一元微积分,级数和
广义积分。前六章和第九章的实数空间由我撰写,一元的其余部分由沈燮昌教授撰写。第三册讲述~$m$~维~Euclid~空
间中的微积分,由廖可人副教授和李正元副教授撰写。

教材内容的选取,基本上没有超出经数学、力学、天文学教材编审委员会~1980~年审订的教学大纲的范围。但由于对内容
的理解和要求不同,在讲述上也会有所差别。我们对内容的要求比我校六十年代教材要高些。如一元部分较系统地讲述凸
函数和上、下极限,多元部分严格地讲述~$m$~元微积分和微分形式。个别可以不讲的内容用星号标出。

本册教材的内容属于传统的内容,但希望在系统性、严格性、逻辑性上能处理得更好些。如讲述初等函数的连续性时,为
了严格处理指数函数的连续性,有的教材把它放在定积分之后讲,有的教材虽把它放在微积分之前讲,但讲得较麻烦。本
书对指数函数的处理,是为了解决上述的矛盾,而又保持逻辑上严格性的一个尝试。又如利用导数研究函数性质时,先讲
~Cauchy~中值定理的两个应用,然后讲~Lagrange~中值定理的两个应用(函数的升降与凹凸),最后讲函数升降与凹凸的
两个应用,这样逻辑上一环扣一环,使问题讲述显得干净、利索。

特别希望在内容讲法上有所突破,使它能帮助教师怎样去讲授这些内容,能指导学生应该怎样去理解和掌握这些内容。

在讲授概念时,不仅讲概念的实际背景,还要求通过实际背景,把概念的每一个符号、每一个式子的涵义揭示出来。如讲
序列极限时,通过求曲边三角形的面积,分析为什么要“存在~$N$”,为什么要“任给~$\varepsilon$”,和极限定义中
“四句话”的意义。又如讲微分时,通过求瞬时速度,分析为什么要求改变量的线性主部及其系数,并指出线性主部和高
阶无穷小项的意义,这样做更有利于学生理解概念的实质。

在讲定理的证明和公式的推导时,不仅要逻辑上清楚和注意表达的艺术,还要求讲出内容之间的有机联系、分析证明的想
法、揭露问题的本质。如当极限四则运算和幂指运算条件不满足时极限怎么求,引出~L'H\^opital~法则;证无穷与无穷之
比的不定型时,先分析证明困难所在和指出解决问题的办法;又如由一次逼近(微分)的充要条件和逼近式的唯一性,引
出高次逼近(Taylor~公式)无充要条件和逼近式的唯一性;对~Peano~余项公式证明的分析,引出~Lagrange~余项公式
证明的方法。这种从学生原有基础出发,引出不断地提出问题、分析问题的讲法,使学生能更好地掌握证明的思想和方法。

一元微积分的一些概念和方法,差不多都可以从几何上给予解释,这样做不仅使概念讲解得更活,也使学生的思维更加活
跃。如讲一致连续与不一致连续时,结合曲线图形来看,如果曲线有一处坡度最陡则一致连续,如果曲线无限地变陡,且
没有坡度最陡的地方则不一致连续。这种“看图识字”的讲法,可以使学生记得牢,学得活。

教材中配有大量例题,既有几何、物理方面的应用题,也有相当数量的计算和推理题;既注意了演算的数量,也注意了解
题的基本方法和特殊的技巧。在前面各章、节附有一些思考题,这是考虑到学生初学微积分时,理解概念不深,这样做有
利于培养学生独立思考的能力。

对实数与极限的处理,我是分为两步教学的。在一元微积分之前,严格讲述极限定义、性质、运算;在一元微积分之后,%
再实数定义、确界和极限存在性、连续函数性质证明。这时可以从一般空间观点来讲,即从空间的连续性、紧性、完备性
的观点来讲述。如用连通性引入无理数,用连通的全序域定义实数空间。根据几年来的教学实践,分两步教学的效果还是
比较满意的。教材中也为另一种讲法作了安排,可以把第二章的确界与第九章的实数公里系统作为预备知识,把区间套和
连续函数中间值定理的证明放到第三章,第九章只保留紧性、完备性及其应用,和上、下极限,这种讲法也是可取的。如
果把紧性及其应用也放到第三章,从逻辑顺序上看是完整些,同学接受来说也不会有什么困难,但对训练来说可能难以保
证。

本人二十几年来一直从事数学分析课的教学工作,曾与许多同事一起工作过,从他们那里学到不少有益的东西。特别在七
十年代,教研室组织过多次极限和微分概念的讨论,这些讨论使我受益匪浅。在这里向他们表示感谢。

本书由李正元副教授初审,沈燮昌教授统一全书。他们对本书提出了一些修改的意见,书稿送出版社后,又经欧阳光中副
教授与董延闿教授对本书作了认真细致的审阅,提出了许多宝贵的修改意见,对他们的宝贵意见,我谨表示深深地谢意。
\Sign{1985~年}
\end{preface}

\endinput
%%
%% End of file `MAPreface-I.tex'.