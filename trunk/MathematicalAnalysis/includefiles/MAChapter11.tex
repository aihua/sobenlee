%# -*- coding:utf-8 -*-
%%%%%%%%%%%%%%%%%%%%%%%%%%%%%%%%%%%%%%%%%%%%%%%%%%%%%%%%%%%%%%%%%%%%%%%%%%%%%%%%%%%%%
%%  MAChapter11.tex'


\chapter{数值级数}\label{ch:11}

\section{数值级数的基本概念与简单性质}
\subsection{数值级数的收敛与发散}
\subsection{级数的基本性质}
\begin{exercise}
\item 求下列级数的和。
\begin{exlistcols}
  \item $\sum_{k=1}^\sinf\dfrac1{4k^2-1}$;
  \item $\sum_{k=1}^\sinf\dfrac1{(3k-2)(3k+1)}$;
  \item $\sum_{k=1}^\sinf\dfrac1{k(k+1)(k+2)}$;
  \item $\sum_{k=1}^\sinf\dfrac{(-1)^{k-1}}{k(k+2)}$;
  \item $\sum_{k=1}^\sinf\dfrac1{\smbsqrt{k(k+1)}\mparenb{\sqrt k+\sqrt{k+1}}}$;
  \item $\sum_{k=1}^\sinf\arctan\dfrac1{2k^2}$。
\end{exlistcols}
\item 证明,
\[
  \sum_{n=1}^\sinf\frac1{n(n+m)}=\frac1m\mparenB{1+\frac12+\dotsb+\frac1m},
\]
其中~$m\geq1$~为固定的自然数。
\item 证明,若级数~$\sum_{n=1}^\sinf a_n$~各项是正的,而把级数的项经过组合而得到的级数~$\sum_{n=1}^\sinf A_n$~收敛,则原来的级数
也收敛。
\item 设~$\lim_\ntoinf na_n=a\neq0$。证明,级数~$\sum_{n=1}^\sinf a_n$~发散。
\item 证明,若级数~$\sum_{n=1}^\sinf a_n\mcond{a_n\geq0}$~收敛,则级数~$\sum_{n=1}^\sinf a_n^2$~也收敛。请举出例子说明,反之并
不成立。
\item 求下列级数的和。
\begin{exlistcols}
  \item $\sum_{n=1}^\sinf\dfrac{(-1)^{n-1}}{2^{n-1}}$;
  \item $\sum_{n=1}^\sinf\dfrac{2n-1}{2^n}$;
  \item $\sum_{n=1}^\sinf r^n\sin n\theta\mcond{\mabs r<1}$;
  \item $\dfrac12+\sum_{n=1}^\sinf r^n\cos n\theta\mcond{\mabs r<1}$。
\end{exlistcols}
\item 研究级数~$\sum_{n=1}^\sinf\sin nx$~在~$x\in\mintc 0{2\pi}$~上的敛散性。
\item 讨论下列级数的敛散性。
\begin{exlistcols}
  \item $\sum_{n=1}^\sinf\dfrac1{\smbsqrt{n(n+1)}}$;
  \item $\sum_{n=1}^\sinf\dfrac1{\sqrt{n^3+1}}$。
\end{exlistcols}
\end{exercise}

\section{正项级数}
\subsection{正项级数收敛的充要条件}
\subsection{比较判别法}
\subsection{d'Alembert~判别法及~Cauchy~判别法}
\subsection{Raabe~判别法}
\subsection{Cauchy~积分判别法}
\begin{exercise}
\item 判断下列级数的敛散性。
\begin{exlistcols}
  \item $\sum_{n=1}^\sinf\dfrac1{n^{\ln n}}$;
  \item $\sum_{n=1}^\sinf\dfrac1{(\ln n)^{\ln n}}$。
\end{exlistcols}
\item 判断下列级数的敛散性。
\begin{exlistcols}[3]
  \item $\sum_{n=1}^\sinf\dfrac{n\ln n}{2^n}$;
  \item $\sum_{n=1}^\sinf\dfrac{n^{n-1}}{(2n^2+n+1)^{\frac{n-1}2}}$;
  \item $\sum_{n=1}^\sinf\dfrac{n!\,2^n}{n^n}$;
  \item $\sum_{n=1}^\sinf\dfrac{n!\,3^n}{n^n}$;
  \item $\sum_{n=1}^\sinf\dfrac{n^2}{\mparenB{n+\dfrac1n}^n}$;
  \item $\sum_{n=1}^\sinf\dfrac{a^n}{1+a^{2n}}$;
  \item $\dfrac31+\dfrac{3\times 5}{1\times 4}+\dfrac{3\times5\times7}{1\times4\times7}+
         \dfrac{3\times5\times7\times9}{1\times4\times7\times10}+\dotsb$。
\end{exlistcols}
\item 用~Cauchy~判别法证明~$\sum_{n=1}^\sinf\dfrac1{2^{n+(-1)^n}}$~收敛,并说明~d'Alembert~判别法对此级数失效。
\item 设~$a_n>0$,且~$\lim_\ntoinf\dfrac{a_{n+1}}{a_n}=\ell$。证明~$\lim_\ntoinf\sqrt[n]{a_n}=\ell$。并讨论反之是否成立。
\item 设~$a_n>0$,且~$\dfrac{a_{n+1}}{a_n}\leq r<1$。证明,用~$\sum_{k=1}^na_k$~近似代替~$\sum_{n=1}^\sinf a_n$~的误差
小于~$\dfrac r{1-r}a_n$。
\item 用~Cauchy~积分判别法判别下列级数的敛散性。
\begin{exlistcols}[4]
  \item $\sum_{n=1}^\sinf\dfrac1{3^{\sqrt n}}$;
  \item $\sum_{n=1}^\sinf\dfrac n{3^{\sqrt n}}$;
  \item $\sum_{n=1}^\sinf\dfrac1{2^{\ln n}}$;
  \item $\sum_{n=1}^\sinf\dfrac1{3^{\ln n}}$。
\end{exlistcols}
\item 利用~Taylor~级数估计无穷小量~$a_n$~的阶,从而判别下列级数的敛散性。
\begin{exlistcols}
  \item $\sum_{n=1}^\sinf2^n\sin\dfrac\pi{3^n}$;
  \item $\sum_{n=1}^\sinf\dfrac1{\ln(n+1)}\sin\dfrac1n$;
  \item $\sum_{n=1}^\sinf\dfrac1{\sqrt{n^3+1}}$;
  \item $\sum_{n=1}^\sinf\mparenb{\sqrt{n+1}-\sqrt n}^p\ln\dfrac{n-1}{n+1}$;
  \item $\sum_{n=3}^\sinf\ln^p\cos\dfrac\pi n$;
  \item $\sum_{n=1}^\sinf\mparenb{\sqrt{n+a}-\sqrt[4]{n^2+n+b}}$;
  \item $\sum_{n=1}^\sinf\mparenbb{\me-\mparenB{1+\dfrac1n}^n}^p$。
\end{exlistcols}
\item 若正项级数~$\sum_{n=1}^\sinf a_n$~发散,而~$S_n$~表示级数的第~$n$~部分和。证明级数~$\sum_{n=1}^\sinf\dfrac{a_n}{S_n}$~也
发散。
\item 若正项级数~$\sum_{n=1}^\sinf a_n$~收敛,且~$a_{n+1}\leq a_n$。证明~$\lim_\ntoinf n\cdot a_n=0$。\label{exer-11.2.9}
\item 设
\[%\LEFTRIGHT
  \Bigggl\lbrace\begin{alignedat}{2}
    a_n     & =\dfrac1{n^2}, &\quad & n\neq k^2,k=1,2,\dotsc;\\[2pt]
    a_{k^2} & =\dfrac1{k^2}, &\quad & k=1,2,\dotsc 。
  \end{alignedat}
\]
证明,
\begin{exlistcols}
  \item 级数~$\sum_{n=1}^\sinf a_n$~收敛;
  \item 极限~$\lim_\ntoinf na_n\neq0$。
\end{exlistcols}
\item\label{exer-11.2.11} 设~$0<p_1<p_2<\dotsb<p_n<\dotsb$。证明~$\sum_{n=1}^\sinf\dfrac1{p_n}$~收敛当且仅
当~$\sum_{n=1}^\sinf\dfrac n{p_1+p_2+\dotsb+p_n}$~收敛。
\item 利用~Raabe~判别法判别法研究下列级数的敛散性。
\begin{exlistcols}
  \item $\sum_{n=1}^\sinf\mparenbb{\dfrac{(2n-1)!!}{(2n)!!}}^p\mcond{p\in\mR}$;
  \item $\sum_{n=1}^\sinf\dfrac{\alpha(\alpha+1)\dotsm(\alpha+n-1)}{n!}\dfrac1{n^\beta}\mcond{\alpha,\beta>0}$。
\end{exlistcols}
\item 设~$a_n>0$。证明~$\limsup_\ntoinf n\mparenB{\dfrac{1+a_{n+1}}{a_n}-1}\geq1$。
\item 设两个正项级数~$\sum_{n=1}^\sinf a_n$~与~$\sum_{n=1}^\sinf b_n$~都发散,讨论下列级数的敛散性。
\begin{exlistcols}
  \item $\sum_{n=1}^\sinf\min\mrangeb{a_n}{b_n}$;
  \item $\sum_{n=1}^\sinf\max\mrangeb{a_n}{b_n}$。
\end{exlistcols}
\item 设正项级数~$\sum_{n=1}^\sinf a_n$~收敛。证明~$\lim_\ntoinf\dfrac1n\sum_{k=1}^nka_k=0$,并由此推证\ref{exer-11.2.9}。
\end{exercise}

\section{任意项级数}
\subsection{交错级数收敛判别法}
\subsection{绝对收敛与条件收敛}
\subsection{Dirichlet~判别法}
\begin{exercise}
\item 不用~Cauchy~准则,证明~$\sum_{n=1}^\sinf\mabsb{a_n}$~收敛蕴涵~$\sum_{n=1}^\sinf a_n$~收敛。
\item 若级数~$\sum_{n=1}^\sinf a_n$~收敛,且~$\lim_\ntoinf\dfrac{b_n}{a_n}=1$,讨论~$\sum_{n=1}^\sinf b_n$~的敛散性。可以研究
以下例子,
\[
  a_n=\frac{(-1)^n}{\sqrt n},\qquad b_n=a_n+\frac1n 。
\]
\item 证明,若级数~$A\coloneq\sum_{n=1}^\sinf a_n$~及~$B\coloneq\sum_{n=1}^\sinf b_n$~皆收敛,且
\[
  a_n\leq c_n\leq b_n\mcond*{n=1,2,\dotsc}。
\]
证明级数~$C\coloneq\sum_{n=1}^\sinf c_n$~也收敛。若级数~$A$~与~$B$~皆发散,讨论级数~$C$~的敛散性。
\item 级数~$\sum_{n=1}^\sinf a_n$~满足
\begin{exlistcols}
  \item $\lim_\ntoinf a_n=0$;
  \item $\sum_{n=1}^\sinf\mparenb{a_{2n-1}+a_{2n}}$~收敛。
\end{exlistcols}
证明~$\sum_{n=1}^\sinf a_n$~收敛。
\item 证明,若级数~$\sum_{n=1}^\sinf a_n^2$~及~$\sum_{n=1}^\sinf b_n^2$~收敛,则级数
\[
  \sum_{n=1}^\sinf\mabsb{a_nb_n},\quad\sum_{n=1}^\sinf (a_n+b_n)^2,\quad \sum_{n=1}^\sinf \dfrac{\mabs{a_n}}n
\]
也收敛。
\item 下列是非题,正确的请给予证明,错误的请举出反例。
\begin{exlistcols}
  \item 若~$a_n>0$,则~$a_1-a_1+a_2-a_2+a_3-a_3+\dotsb$~收敛;
  \item 若~$a_n\to0$,则~$a_1-a_1+a_2-a_2+a_3-a_3+\dotsb$~收敛;
  \item 若~$\sum_{n=1}^\sinf a_n$~收敛,则~$\sum_{n=1}^\sinf(-1)^na_n$~收敛;
  \item 若~$\sum_{n=1}^\sinf a_n$~收敛,则~$\sum_{n=1}^\sinf a_n^2$~收敛;
  \item 若~$\sum_{n=1}^\sinf a_n^2$~收敛,则~$\sum_{n=1}^\sinf a_n^3$~收敛;
  \item 若~$\sum_{n=1}^\sinf a_n$~发散,则~$a_n\nto0$;
  \item 若~$\sum_{n=1}^\sinf a_n$~收敛,且~$b_n\to1$,则~$\sum_{n=1}^\sinf a_nb_n$~收敛;
  \item 若~$\sum_{n=1}^\sinf\mabsb{a_n}$~收敛,且~$b_n\to1$,则~$\sum_{n=1}^\sinf a_nb_n$~收敛;
  \item 若~$\sum_{n=1}^\sinf a_n$~收敛,且~$a_n>0$,则~$\lim_\ntoinf na_n=0$。
\end{exlistcols}
\item 判别下列级数的敛散性。
\begin{exlistcols}
  \item $\sum_{n=1}^\sinf(-1)^n\dfrac{\sqrt n}{n+100}$;
  \item $\sum_{n=1}^\sinf\dfrac{\ln n}n\sin\dfrac n2\pi$;
  \item $\sum_{n=1}^\sinf(-1)^n\dfrac{1+\dfrac12+\dotsb+\dfrac1n}n$;
  \item $\sum_{n=1}^\sinf\dfrac{(-1)^n}{\sqrt n+(-1)^n}$;
  \item $\sum_{n=1}^\sinf\sin\mparenb{\pi\sqrt{n^2+1}}$;
  \item $\sum_{n=1}^\sinf\dfrac{(-1)^{\frac{n(n-1)}2}}{3^n}$;
  \item $\sum_{n=1}^\sinf\dfrac{(-1)^n}{n^p}\mcond{p>0}$;
  \item $\sum_{n=1}^\sinf\dfrac1{3^n}\sin\dfrac n2\pi$;
  \item $\sum_{n=1}^\sinf(-1)^n\dfrac{\cos2n}n$;
  \item $\sum_{n=1}^\sinf(-1)^n\dfrac{\sin^2n}n$;
  \item $\sum_{n=2}^\sinf\mparenbb{\dfrac1{\sqrt n-1}-\dfrac1{\sqrt n+1}}$。
\end{exlistcols}
\item\label{exer-11.3.9}设~$b_n>0$,且~$\lim_\ntoinf n\mparenB{\dfrac{b_n}{b_{n+1}}-1}>0$。证明,级
数~$\sum_{n=1}^\sinf(-1)^{n+1}b_n$~收敛。
\item 利用\ref{exer-11.3.9}~讨论下列级数的敛散性。
\begin{exlistcols}
  \item $\sum_{n=1}^\sinf(-1)^{n+1}\mparenbb{\dfrac{(2n-1)!!}{(2n)!!}}^p$;
  \item $1+\sum_{n=1}^\sinf\dfrac{\alpha(\alpha-1)\dotsm(\alpha-n+1)}{n!}$。
\end{exlistcols}
\item 设~$p>0$。证明级数~$\sum_{n=1}^\sinf\dfrac{(-1)^{n+1}}{n^p}$~的和介于~$\dfrac12$~与~$1$~之间。
\item 讨论下列级数的敛散性。
\begin{exlistcols}
  \item $\sum_{n=1}^\sinf(-1)^n\dfrac{\sin n}n$;
  \item $\sum_{n=1}^\sinf\dfrac{(-1)^{n+1}}n\cdot\dfrac a{1+a^n}\mcond{a>0}$;
  \item $\sum_{n=1}^\sinf\dfrac{\sin n\sin n^2}n$;
  \item $\sum_{n=1}^\sinf\dfrac{\sin\mparenB{n+\dfrac1n}}n$。
\end{exlistcols}
\item 设~$\sum_{n=1}^\sinf a_n$~收敛,且~$\lim_\ntoinf na_n=0$。证明~$\sum_{n=1}^\sinf n\mparenb{a_n-a_{n+1}}$~收敛,并且
\[
  \sum_{n=1}^\sinf n\mparenb{a_n-a_{n+1}}=\sum_{n=1}^\sinf a_n 。
\]
\item 对序列~$\mbrace{a_k}$~与~$\mbrace{b_k}$~分别定义~$S_n=\sum_{k=1}^na_k$~与~$\Delta b_k=b_{k+1}-b_k$。证明,
\begin{exlist}
  \item 如果~$\mbrace{S_n}$~有界,$\sum_{k=1}^\sinf\mabsb{\Delta b_k}$~收敛,且~$\lim_\ntoinf b_n=0$,%
  则~$\sum_{k=1}^\sinf a_kb_k$~收敛,且有
  \[
    \sum_{k=1}^\sinf a_kb_k=-\sum_{k=1}^\sinf S_k\Delta b_k;
  \]
  \item 如果~$\sum_{k=1}^\sinf a_k$~与~$\sum_{k=1}^\sinf\mabsb{\Delta b_k}$~都收敛,则~$\sum_{k=1}^\sinf a_kb_k$~收敛。
\end{exlist}
\item 设正项数列~$\mbrace{x_n}$~单调上升且有界。证明~$\sum_{n=1}^\sinf\mparenB{1-\dfrac{x_n}{x_{n+1}}}$~收敛。
\item 用~Taylor~公式讨论下列级数的敛散性与绝对收敛性。
\begin{exlistcols}
  \item $\sum_{n=1}^\sinf\dfrac{(-1)^{n-1}}{\mparenb{\sqrt n+(-1)^{n-1}}^p}$;
  \item $\sum_{n=1}^\sinf\ln\mparenbb{1+\dfrac{(-1)^n}{n^p}}$;
  \item $\sum_{n=1}^\sinf\dfrac{(-1)^n}{\mparenb{n+(-1)^n}^p}$;
  \item $\sum_{n=1}^\sinf\dfrac{\sin\dfrac n4\pi}{n^p+\sin\dfrac n4\pi}$。
\end{exlistcols}
\item 研究下列级数的敛散性与绝对收敛性(其中~$p>0$)。
\begin{exlistcols}
  \item $\sum_{n=1}^\sinf\dfrac{(-1)^n}{n^p}$;
  \item $\sum_{n=1}^\sinf\dfrac{(-1)^n}{n^p+\dfrac1n}$;
  \item $\sum_{n=1}^\sinf\dfrac{(-1)^n}{\mparenb{n+(-1)^n}^p}$;
  \item $\sum_{n=1}^\sinf\dfrac{(-1)^{\mfloor{\sqrt n}}}{n^p}$;
  \item $\sum_{n=1}^\sinf(-1)^{n-1}\dfrac{2^n\sin^{2n}x}n$;
  \item $\sum_{n=1}^\sinf\mparenbb{\dfrac x{a_n}}^{\msp n}$,这里~$\lim_\ntoinf a_n=a>0$;
  \item $1+\sum_{n=1}^\sinf\dfrac{(2n-1)!!}{(2n)!!}\cdot\dfrac{(-1)^n}{2n+1}$;
  \item $\sum_{n=1}^\sinf\dfrac{(-1)^{n-1}n!}{(x+1)(x+2)\dotsm(x+n)}\mcond{x>0}$;
  \item $\sum_{n=1}^\sinf(-1)^nr^{n+\sqrt n}\mcond{r>0}$;
  \item $\sum_{n=1}^\sinf n!\,\mparenbb{\dfrac xn}^n$。
\end{exlistcols}
\end{exercise}

\section{收敛级数的性质}
\subsection{无穷级数的可结合性}
\subsection{无穷级数的可交换性问题}
\subsection{级数的乘法}
\begin{exercise}
\item 设~$\sum_{n=1}^\sinf a_n$~收敛。证明,将相邻奇偶项交换后所成的级数收敛,且具有相同的和数。
\item 证明,若将收敛级数的各项重新排列,而使每一项离开原有的位置不超过~$m$~个位置($m$~为预先给定的数),则其和不变。
\item%%% 证明,
\begin{exlist}\FixExHead
  \item 若级数~$\sum_{n=1}^\sinf a_n$~收敛,则把该级数的项经过组合而不变更其先后次序所得的级数
  \[
    \sum_{n=1}^\sinf A_n,\quad A_n\coloneq\sum_{i=p_n}^{p_{n+1}-1}a_i\mcond*{p_1=1<p_2<\dotsb}
  \]
  也收敛,且有相同的和。并讨论反之是否成立,即~$\sum_{n=1}^\sinf A_n$~收敛是否蕴涵~$\sum_{n=1}^\sinf a_n$~收敛;
  \item 若~$A_n$~中各项~$a_i$~同号,且~$A_n$~与~$A_{n+1}$~异号,则~$\sum_{n=1}^\sinf A_n$~收敛蕴涵~$\sum_{n=1}^\sinf a_n$~收敛。
\end{exlist}
\item 设~$p_n>0$,级数~$\sum_{n=1}^\sinf\dfrac1{p_n}$~收敛。证明,
\begin{exlist}
  \item $\sum_{n=1}^\sinf\dfrac n{p_1+p_2+\dotsb+p_n}$~收敛;(参看\ref{exer-11.2.11})
  \item $\lim_\ntoinf\dfrac{n^2}{p_1+p_2+\dotsb+p_n}=0$。
\end{exlist}
\item 证明,由级数~$\sum_{n=1}^\sinf\dfrac{(-1)^{n-1}}{\sqrt n}$~的项组合成的级数
\[
  1+\frac1{\sqrt 3}-\frac1{\sqrt 2}+\frac1{\sqrt 5}+\frac1{\sqrt 7}-\frac1{\sqrt 4}+\dotsb
\]
发散。
\item 已知
\[
  H_n=\sum_{k=1}^n\frac1k=\gamma+\ln n+r_n,
\]
这里~$\gamma$~为~Euler~常数,而~$\lim_\ntoinf r_n=0$。证明,
\begin{exlist}
  \item $\dfrac12+\dfrac14+\dotsb+\dfrac1{2m}=\dfrac12\ln m+\dfrac12\gamma+\dfrac12r_m$;
  \item 若把级数~$\sum_{n=1}^\sinf\dfrac{(-1)^{n+1}}n$~的各项重新安排,而使挨次~$p$~个正项的一组与挨次~$q$~个负项的一组相
  交替,则新级数的和为
  \[
    \ln 2+\frac12\ln\frac pq。
  \]
\end{exlist}
\item 证明,级数~$\sum_{n=1}^\sinf\dfrac{(-1)^{n+1}}n$~的平方(Cauchy~乘积)是收敛的。
\item 设~$x_n>0$,而~$\lim_\ntoinf x_n=0$,并且~$\lim_\ntoinf\dfrac{x_{n+1}}{x_n}=b$。证明~$b\leq1$。
\item 通过序列转化为级数的方法,证明下面两个序列收敛。
\begin{exlistcols}
  \item $x_n=\sum_{k=1}^n\dfrac1{\sqrt k}-2\sqrt n$;
  \item $x_n=\sum_{k=1}^n\dfrac{\ln k}k-\dfrac{\ln^2n}2$。
\end{exlistcols}
\item 求级数
\[
  \mparenbb{\sum_{n=1}^\sinf x^n}^{\msp 3}\mcond*{\mabs x<1}
\]
的和。
\item 令~$\me^x=\sum_{n=0}^\sinf\dfrac{x^n}{n!}$。证明~$\me^{x+y}=\me^x\cdot\me^y$。
\end{exercise}

\section{广义积分与级数的联系}

\begin{exercise*}
\item 设~$m\in\mN$。证明,
\begin{exlist}
  \item $\dfrac1m=\mbinom m1-\mbinom m2\mparenB{1+\dfrac12}+\mbinom m3\mparenB{1+\dfrac12+\dfrac13}-\dotsb
          +(-1)^{m-1}\mbinom mm\mparenB{1+\dfrac12+\dotsb+\dfrac1m}$;
  \item $\sum_{k=1}^\sinf\dfrac1{k(k+1)(k+2)\dotsm(k+m)}=\dfrac1{m!\,m}$。
\end{exlist}
\item 设~$\sum_{n=0}^\sinf\dfrac{n^m}{n!}=\me\cdot p_m$。证明,
\begin{exlistcols}
  \item $p_{m+1}=1+mp_1+\dfrac{m(m-1)}{2!}p_2+\dotsb+p_m$;
  \item $p_m$~都是正整数。
\end{exlistcols}
\item 设~$\lim_\ntoinf a_n=\ell$。证明,
\begin{exlistcols}
  \item 当~$\ell>1$~时,级数~$\sum_{n=1}^\sinf\dfrac1{n^{a_n}}$~收敛;
  \item 当~$\ell<1$~时,级数~$\sum_{n=1}^\sinf\dfrac1{n^{a_n}}$~发散。
\end{exlistcols}
并讨论~$\ell=1$~时会有什么结论。
\item 设正项级数~$\sum_{n=1}^\sinf a_n$~发散,且~$S_n=\sum_{k=1}^na_k$。证明,级数~$\sum_{n=1}^\sinf\dfrac{a_n}{S_n^p}$~
当~$p>1$~时收敛,而当~$p\leq 1$~时发散。
\item 设正项级数~$\sum_{n=1}^\sinf a_n$~收敛,且~$r_n=\sum_{k=n}^\sinf a_k$。证明,级数~$\sum_{n=1}^\sinf\dfrac{a_n}{r_n^p}$~
当~$p<1$~时收敛,而当~$p\geq 1$~时发散。
\item 设~$p>1$。证明,级数~$\sum_{n=1}^\sinf\dfrac1{(n+1)n^{\sfrac1p}}<p$。
\item 设~$0<x_1<\pi$,而~$x_{n+1}=\sin x_n$。讨论级数~$\sum_{n=1}^\sinf x_n^p$~的敛散性。
\item 设~$x_{n+1}=x_n(1-qx_n)$,其中~$0<q<1$~且~$0<x_1<\dfrac1q$。讨论级数~$\sum_{n=1}^\sinf x_n$~与~$\sum_{n=1}^\sinf x_n^2$~的
敛散性。
\item 设~$p_n>0$,而~$\sum_{n=1}^\sinf\dfrac1{p_n}$~收敛。证明,级
数~$\sum_{n=1}^\sinf\dfrac{n^2}{(p_1+p_2+\dotsb+p_n)^2}p_n$~收敛。
\item 设~$b_n$~单调下降收敛于零,且级数~$\sum_{n=1}^\sinf a_nb_n$~收敛。令~$S_n=\sum_{k=1}^na_k$。证明,
\begin{exlist}
  \item $\lim_\ntoinf S_n\cdot b_n=0$;
  \item 级数~$\sum_{n=1}^\sinf(b_n-b_{n+1})$~收敛,且~$\sum_{n=1}^\sinf a_nb_n=\sum_{n=1}^\sinf(b_n-b_{n+1})$。
\end{exlist}
\end{exercise*}




\endinput
%%
%% End of file `MAChapter11.tex'.