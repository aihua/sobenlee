%# -*- coding:utf-8 -*-
%%%%%%%%%%%%%%%%%%%%%%%%%%%%%%%%%%%%%%%%%%%%%%%%%%%%%%%%%%%%%%%%%%%%%%%%%%%%%%%%%%%%%


\chapter{寄法名官哥穿道服\KG 散生日敬济拜冤家}


诗曰:

\[
汉武清斋夜筑坛,自斟明水醮仙官。
殿前玉女移香案,云际金人捧露盘。
绛节几时还入梦?碧桃何处更骖鸾?
茂陵烟雨埋弓剑,石马无声蔓草寒。
\]

话说当日西门庆在潘金莲房中歇了一夜。那妇人恨不的钻入他腹中,在枕畔千般贴恋,万种牢笼,泪揾鲛鮹,语言温顺,实指望买住汉子心。不料西门庆外边又刮剌上了王六儿,替他狮子街石桥东边,使了一百二十两银子,买了一所房屋居住。门面两间,到底四层,一层做客位,一层供养佛像祖先,一层做住房,一层做厨房。自从搬过来,那街坊邻舍知他是西门庆伙计,不敢怠慢,都送茶盒与他,又出人情庆贺。那中等人家称他做韩大哥、韩大嫂。以下者赶着以叔婶称之。西门庆但来他家,韩道国就在铺子里上宿,教老婆陪他自在顽耍。朝来暮往,街坊人家也都知道这件事,惧怕西门庆有钱有势,谁敢惹他!见一月之间,西门庆也来行走三四次,与王六儿打的一似火炭般热。

看看腊月时分,西门庆在家乱着送东京并府县、军卫、本卫衙门中节礼。有玉皇庙吴道官使徒弟送了四盒礼物,并天地疏、新春符、谢灶诰。西门庆正在上房吃饭,玳安儿拿进帖来,上写着:“王皇庙小道吴宗哲顿首拜。”西门庆看了说道:“出家人,又教他费心。”吩咐玳安,叫书童儿封一两银子拿回帖与他。月娘在旁,因话题起道:“一个出家人,你要便年头节尾受他的礼物,到把前日你为李大姐生孩儿许的愿醮,就叫他打了罢。”西门庆道:“早是你题起来,我许下一百二十分醮,我就忘死了。”月娘道:“原来你是个大诌答子货!谁家愿心是忘记的?你便有口无心许下,神明都记着。嗔道孩儿成日恁啾啾唧唧的,想就是这愿心未还压的他。”西门庆道:“既恁说,正月里就把这醮愿,在吴道官庙里还了罢。”月娘道:“昨日李大姐说,这孩子有些病痛儿的,要问那里讨个外名。”西门庆道:“又往那里讨外名?就寄名在吴道官庙里就是了。”因问玳安:“他庙里有谁在这里?”玳安道:“是他第二个徒弟应春跟礼来的。”西门庆一面走出外边来,那应春连忙磕头说道:“家师父多拜上老爹,没什么孝顺,使小徒弟来送这天地疏并些微礼儿,与老爹赏人。”西门庆止还了半礼,说道:“多谢你师父厚礼。”一面让他坐。应春道:“小道怎么敢坐!”西门庆道:“你坐了,我有话和你说。”那道士头戴小帽,身穿青布直裰,谦逊数次,方才把椅儿挪到旁边坐下,问道:“老爹有甚钧语吩咐?”西门庆道:“正月里,我有些醮愿,要烦你师父替我还还儿,就要送小儿寄名,不知你师父闲不闲?”徒弟连忙立起身来说道:“老爹吩咐,随问有甚经事,不敢应承。请问老爹,订在正月几时?”西门庆道:“就订在初九,爷旦日罢。”徒弟道:“此日正是天诞。又《玉匣记》上我请律爷交庆,五福骈臻,修斋建醮甚好。请问老爹多少醮款?”西门庆道:“今岁七月,为生小儿许了一百二十分清醮。”徒弟又问:“那日延请多少道众?”西门庆道:“请十六众罢。”说毕,左右放桌儿待茶。先封十五两经钱,另外又是一两酬答他的节礼,又说:“道众的衬施,你师父不消备办,我这里连阡张香烛一事带去。”喜欢的道士屁滚尿流,临出门谢了又谢,磕了头儿又磕。

到正月初八日,先使玳安儿送了一石白米、一担阡张、十斤官烛、五斤沉檀马牙香、十六匹生眼布做衬施,又送了一对京段、两坛南酒、四只鲜鹅、四只鲜鸡、一对豚蹄、一脚羊肉、十两银子,与官哥儿寄名之礼。西门庆预先发帖儿,请下吴大舅、花大舅、应伯爵、谢希大四位相陪。陈敬济骑头口,先到庙中替西门庆瞻拜。到初九日,西门庆也没往衙门中去,绝早冠带,骑大白马,仆从跟随,前呼后拥,竟出东门往玉皇庙来。远远望见结彩宝幡,过街榜棚。须臾至山门前下马,睁眼观看,果然好座庙宇。但见:

\[
青松郁郁,翠柏森森。金钉朱户,玉桥低影轩官;碧瓦雕檐,绣幕高悬宝槛。七间大殿,中悬敕额金书;两庑长廊,彩画天神帅将。三天门外,离娄与师旷狰狞,左右阶前,自虎与青龙猛勇。八宝殿前,侍立是长生玉女,九龙床上,坐着个不坏金身。金钟撞处,三千世界尽皈依;玉磬鸣时,万象森罗皆拱极。朝天阁上,天风吹下步虚声;演法坛中,夜月常闻仙佩响。自此便为真紫府,更于何处觅蓬莱?
\]
西门庆由正门而入,见头一座流星门上,七尺高朱红牌架,列着两行门对,大书:

\[
黄道天开,祥启九天之阊阖,迓金舆翠盖以延恩;
玄坛日丽,光临万圣之幡幢,诵宝笈瑶章而阐化。
\]
到了宝殿上,悬着二十四字斋题,大书着:“灵宝答天谢地,报国酬恩,九转玉枢,酬盟寄名,吉祥普满斋坛。”两边一联:

\[
先天立极,仰大道之巍巍,庸申至悃;
昊帝尊居,鉴清修之翼翼,上报洪恩。
\]
西门庆进入坛中香案前,旁边一小童捧盆中盥手毕,铺排跪请上香。西门庆行礼叩坛毕,只见吴道官头戴玉环九阳雷巾,身披天青二十八宿大袖鹤氅,腰系丝带,忙下经筵来,与西门庆稽首道:“小道蒙老爹错爱,迭受重礼,使小道却之不恭,受之有愧。就是哥儿寄名,小道礼当叩祝,增延寿命,何以有叨老爹厚赏,诚有愧赧。经衬又且过厚,令小道愈不安。”西门庆道:“厚劳费心辛苦,无物可酬,薄礼表情而已。”叙礼毕,两边道众齐来稽首。一面请去外方丈,三间厂厅名曰松鹤轩,那里待茶。西门庆刚坐下,就令棋童儿:“拿马接你应二爹去。只怕他没马,如何这咱还没来?”玳安道:“有姐夫骑的驴子还在这里。”西门庆道:“也罢,快骑接去。”棋童应诺去了。吴道官诵毕经,下来递茶,陪西门庆坐,叙话:“老爹敬神一点诚心,小道都从四更就起来,到坛讽诵诸品仙经,今日三朝九转玉枢法事,都是整做。又将官哥儿的生日八字,另具一文书,奏名于三宝面前,起名叫做吴应元。永保富贵遐昌。小道这里,又添了二十四分答谢天地,十二分庆赞上帝,二十四分荐亡,共列一百八十分醮款。”西门庆道:“多有费心.”不一时,打动法鼓,请西门庆到坛看文书。西门庆从新换了大红五彩狮补吉服,腰系蒙金犀角带,到坛,有绛衣表白在旁,先宣念斋意:

\[
大宋国山东清河县县牌坊居住,奉道祈恩,酬醮保安,信官西门庆,本命丙寅年七月廿八日子时建生,同妻吴氏,本命戊辰年八月十五日子时建生。
\]
表白道:“还有宝眷,小道未曾添上。”西门庆道:“你只添上个李氏,辛未年正月十五日卯时建生,同男官哥儿,丙申年七月廿三日申时建生罢。”表白文宣过一遍,接念道:

\[
领家眷等,即日投诚,拜干洪造。伏念庆一介微生,三才未品。出入起居,每感龙天之护佑;迭迁寒暑,常蒙神圣以匡扶。职列武班,叨承禁卫,沐恩光之宠渥,享符禄之丰盈。是以修设清醮,共二十四分位,答报天地之洪恩,酬祝皇王之巨泽。又修清醮十二分位,兹逢天诞,庆赞帝真。介五福以遐昌,迓诸天而下迈。庆又于去岁七月二十三日,因为侧室李氏生男官哥儿,要祈坐蓐无虞,临盆有庆。又愿将男官哥儿寄于三宝殿下,赐名吴应元,告许清醮一百二十分位,续箕裘之\textuni{38E7}嗣,保寿命之延长。附荐西门氏门中三代宗亲等魂:祖西门京良,祖妣李氏;先考西门达,妣夏氏;故室人陈氏,及前亡后化,升坠罔知。是以修设清醮十二分位,恩资道力,均证生方。共列仙醮一百八十分位,仰干化单,俯赐勾销。谨以宣和三年正月初九日天诞良辰,特就大慈玉皇殿,仗延官道,修建灵宝,答天谢地,报国酬盟,庆神保安,寄名转经,吉祥普满大斋一昼夜。延三境之司尊,迓万天之帝驾。一门长叨均安,四序公和迪吉。统资道力,介福方来。谨意。
\]
宣毕斋意,铺设下许多文书符命、表白,一一请看,共有一百八九十道,甚是齐整详细。又是官哥儿三宝荫下寄名许多文书、符索、牒札,不暇细览。西门庆见吴道官十分费心,于是向案前炷了香,画了文书,叫左右捧一匹尺头,与吴道官画字。吴道官固辞再三,方令小童收了。然后一个道士向殿角头咕碌碌擂动法鼓,有若春雷相似。合堂道众,一派音乐响起。吴道官身披大红五彩法氅,脚穿朱履,手执牙笏,关发文书,登坛召将。两边鸣起钟来。铺排引西门庆进坛里,向三宝案左右两边上香。西门庆睁眼观看,果然铺设斋坛齐整。但见:

\[
位按五方,坛分八级。上供三请四御,旁分八极九霄,中列山川岳渎,下设幽府冥官。香腾瑞霭,千枝画烛流光;花簇锦筵,百盏银灯散彩。天地亭,高张羽盖;玉帝堂,密布幢幡。金钟撞处,高功蹑步奏虚皇;玉佩鸣时,都讲登坛朝玉帝。绛绡衣,星辰灿烂;美蒙冠,金碧交加。监坛神将狰狞,直日功曹猛勇。青龙隐隐来黄道,白鹤翩翩下紫宸。
\]

西门庆刚绕坛拈香下来,被左右就请到松鹤轩阁儿里,地铺锦毯,炉焚兽炭,那里坐去了。不一时,应伯爵、谢希大来到。唱毕喏,每人封了一星折茶银子,说道:“实告要送些茶儿来,路远。这些微意,权为一茶之需。”西门庆也不接,说道:“奈烦!自恁请你来陪我坐坐,又干这营生做什么?吴亲家这里点茶,我一总都有了。”应伯爵连忙又唱喏,说:“哥,真个?俺每还收了罢。”因望着谢希大说道:“都是你干这营生!我说哥不受,拿出来,倒惹他讪两句好的。”良久,吴大舅、花子由都到了。每人两盒细茶食来点茶,西门庆都令吴道官收了。吃毕茶,一同摆斋,咸食斋馔,点心汤饭,甚是丰洁。西门庆同吃了早斋。原来吴道官叫了个说书的,说西汉评话《鸿门会》。吴道官发了文书,走来陪坐,问:“哥儿今日来不来?”西门庆道,“正是,小顽还小哩,房下恐怕路远唬着他,来不的。到午间,拿他穿的衣服来,三宝面前,摄受过就是一般。”吴道官道:“小道也是这般计较,最好。”西门庆道:“别的倒也罢了,他只是有些小胆儿。家里三四个丫鬟连养娘轮流看视,只是害怕。猫狗都不敢到他跟前。”吴大舅道:“孩儿们好容易养活大——”正说着,只见玳安进来说:“里边桂姨、银姨使了李铭、吴惠送茶来了。”西门庆道:“叫他进来。”李铭、吴惠两个拿着两个盒子跪下,揭开都是顶皮饼、松花饼、白糖万寿糕、玫瑰搽穰卷儿。西门庆俱令吴道官收了,因问李铭:“你每怎得知道?”李铭道:“小的早晨路见陈姑夫骑头口,问来,才知道爹今日在此做好事。归家告诉桂姐、三妈说,旋约了吴银姐,才来了。多上复爹,本当亲来,不好来得,这粗茶儿与爹赏人罢了。”西门庆吩咐:“你两个等着吃斋。”吴道官一面让他二人下去,自有坐处,连手下人都饱食一顿。

话休饶舌。到了午朝,拜表毕,吴道官预备了一张大插桌,又是一坛金华酒,又是哥儿的一顶青缎子绡金道髻,一件玄色纻丝道衣,一件绿云缎小衬衣,一双白绫小袜,一双青潞绸衲脸小履鞋,一根黄绒线绦,一道三宝位下的黄线索,一道子孙娘娘面前紫线索,一付银项圈条脱,刻着“金玉满堂,长命富贵”,一道朱书辟非黄绫符,上书着“太乙司命,桃延合康”八字,就扎在黄线索上,都用方盘盛着,又是四盘羹果,摆在桌上。差小童经袱内包着宛红纸经疏,将三朝做过法事,一一开载节次,请西门庆过了目,方才装入盒担内。共约八抬,送到西门庆家。西门庆甚是欢喜,快使棋童儿家去,叫赏道童两方手帕、一两银子。

且说那日是潘金莲生日,有吴大妗子、潘姥姥、杨姑娘、郁大姐,都在月娘上房坐的。见庙里送了斋来,又是许多羹果插卓礼物,摆了四张桌子,还摆不下,都乱出来观看。金莲便道:“李大姐,你还不快出来看哩!你家儿子师父庙里送礼来了,又有他的小道冠髻,道衣儿。噫,你看,又是小履鞋儿!”孟玉楼走向前,拿起来手中看,说道:“大姐姐,你看道士家也恁精细,这小履鞋,白绫底儿,都是倒扣针儿方胜儿,锁的这云儿又且是好。我说他敢有老婆!不然,怎的扣捺的恁好针脚儿?”吴月娘道:“没的说。他出家人,那里有老婆!想必是雇人做的。”潘金莲接过来说:“道士有老婆,相王师父和大师父会挑的好汗巾儿,莫不是也有汉子?”王姑子道:“道士家,掩上个帽子,那里不去了!似俺这僧家,行动就认出来。”金莲说道:“我听得说,你住的观音寺背后就是玄明观。常言道:男僧寺对着女僧寺,没事也有事。”月娘道:“这六姐,好恁罗说白道的!”金莲道:“这个是他师父与他娘娘寄名的紫线锁。又是这个银脖项符牌儿,上面银打的八个字,带着且是好看。背面坠着他名字,吴什么元?”棋童道:“此是他师父起的法名吴应元。”金莲道:“这是个‘应’字。”叫道:“大姐姐,道士无礼,怎的把孩子改了他的姓?”月娘道:“你看不知礼!”因使李瓶儿:“你去抱了你儿子来,穿上这道衣,俺每瞧瞧好不好?”李瓶儿道:“他才睡下,又抱他出来?”金莲道:“不妨事,你揉醒他。”那李瓶儿真个去了。

这潘金莲识字,取过红纸袋儿,扯出送来的经疏,看见上面西门庆底下同室人吴氏,旁边只有李氏,再没别人,心中就有几分不忿,拿与众人瞧:“你说贼三等儿九格的强人!你说他偏心不偏心?这上头只写着生孩子的,把俺每都是不在数的,都打到赘字号里去了。”孟玉楼问道:“可有大姐姐没有?”金莲道:“没有大姐姐倒好笑。”月娘道:“也罢了,有了一个,也就是一般。莫不你家有一队伍人,也都写上,惹的道士不笑话么?”金莲道:“俺每都是刘湛儿鬼儿么?比那个不出材的,那个不是十个月养的哩!”正说着,李瓶儿从前边抱了官哥儿来。孟玉楼道:“拿过衣服来,等我替哥哥穿。”李瓶儿抱着,孟玉楼替他戴上道髻儿,套上项牌和两道索,唬的那孩子只把眼儿闭着,半日不敢出气儿。玉楼把道衣替他穿上。吴月娘吩咐李瓶儿:“你把这经疏,拿个阡张头儿,亲往后边佛堂中,自家烧了罢。”那李瓶儿去了。玉楼抱弄孩子说道:“穿着这衣服,就是个小道士儿。”金莲接过来说道:“什么小道士儿,倒好相个小太乙儿!”被月娘正色说了两句道:“六姐,你这个什么话,孩儿们面上,快休恁的。”那金莲讪讪的不言了。一回,那孩子穿着衣服害怕,就哭起来。李瓶儿走来,连忙接过来,替他脱衣裳时,就拉了一抱裙奶屎。孟玉楼笑道:“好个吴应元,原来拉屎也有一托盘。”月娘连忙叫小玉拿草纸替他抹。不一时,那孩子就磕伏在李瓶儿怀里睡着了。李瓶儿道:“小大哥原来困了,妈妈送你到前边睡去罢。”吴月娘一面把桌面都散了,请大妗子、杨娘、潘姥姥众人出来吃斋。

看看晚来。原来初八日西门庆因打醮,不用荤酒。潘金莲晚夕就没曾上的寿,直等到今晚来家与他递酒,来到大门站立。不想等到日落时分,只陈敬济和玳安自骑头口来家。潘金莲问:“你爹来了?”敬济道:“爹怕来不成了,我来时,醮事还未了,才拜忏,怕不弄到起更!道士有个轻饶素放的,还要谢将吃酒。”金莲听了,一声儿没言语,使性子回到上房里,对月娘说:“贾瞎子传操——干起了个五更!隔墙掠肝肠——死心塌地,兜肚断了带子——没得绊了!刚才在门首站了一回,见陈姐夫骑头口来了,说爹不来了,醮事还没了,先打发他来家。”月娘道:“他不来罢,咱每自在,晚夕听大师父、王师父说因果、唱佛曲儿。”正说着,只见陈敬济掀帘进来,已带半酣儿,说:“我来与五娘磕头。”问大姐:“有锺儿,寻个儿筛酒,与五娘递一锺儿。”大姐道:“那里寻锺儿去?只恁与五娘磕个头儿。到住回,等我递罢。你看他醉的腔儿,恰好今日打醮,只好了你,吃的恁憨憨的来家。”月娘便问道:“你爹真个不来了?玳安那奴才没来?”陈敬济道:“爹见醮事还没了,恐怕家里没人,先打发我来了,留下玳安在那里答应哩。吴道士再三不肯放我,强死强活拉着吃了两三大锺酒,才了。”月娘问:“今日有那几个在那里?”敬济道:“今日有大舅和门外花大舅、应三叔、谢三叔,又有李铭、吴惠两个小优儿。不知缠到多咱晚。只吴大舅来了。门外花大舅叫爹留住了,也是过夜的数。”金莲没见李瓶儿在跟前,便道:“陈姐夫,你也叫起花大舅来?是那门儿亲,死了的知道罢了。你叫他李大舅才是。”敬济道:“五娘,你老人家乡里姐姐嫁郑恩——睁着个眼儿,闭着个眼儿罢了。”大姐道:“贼囚根子,快磕了头,趁早与我外头挺去!又口里恁汗邪胡说了!”敬济于是请金莲转上,踉踉跄跄磕了四个头,往前边去了。

不一时,掌上灯烛,放桌儿,摆上菜儿,请潘姥姥、杨姑娘、大妗子与众人来。金莲递了酒,打发坐下,吃了面。吃到酒阑,收了家活,抬了桌出去。月娘吩咐小玉把仪门关了,炕上放下小桌儿,众人围定两个姑子,正在中间焚下香,秉着一对蜡烛,听着他说因果。先是大师父讲说,讲说的乃是西天第三十二祖下界降生东土,传佛心印的佛法因果,直从张员外家豪大富说起,漫漫一程一节,直说到员外感悟佛法难闻,弃了家园富贵,竟到黄梅寺修行去。说了一回,王姑子又接念偈言。

念了一回,吴月娘道:“师父饿了,且把经请过,吃些甚么。”一面令小玉安排了四碟儿素菜咸食,又四碟薄脆、蒸酥糕饼,请大妗子、杨姑娘、潘姥姥陪二位师父吃。大妗子说:“俺每都刚吃的饱了,教杨姑娘陪个儿罢,他老人家又吃着个斋。”月娘连忙用小描金碟儿,每样拣了点心,放在碟儿里,先递与两位师父,然后递与杨姑娘,说道:“你老人家陪二位请些儿。”婆子道:“我的佛爷,老身吃的勾了。”又道:“这碟儿里是烧骨朵,姐姐你拿过去,只怕错拣到口里。”把众人笑的了不得。月娘道:“奶奶,这个是庙上送来托荤咸食。你老人家只顾用,不妨事。”杨姑娘道:“既是素的,等老身吃。老身干净眼花了,只当做荤的来。”正吃着,只见来兴儿媳妇子惠香走来。月娘道:“贼臭肉,你也来什么?”惠香道:“我也来听唱曲儿。”月娘道:“仪门关着,你打那里进来了?”玉箫道:“他厨房封火来。”月娘道:“嗔道恁鼻儿乌嘴儿黑的,成精鼓捣,来听什么经!”

当下众丫鬟妇女围定两个姑子,吃了茶食,收过家活去,搽抹经桌干净。月娘从新剔起灯烛来,炷了香。两个姑子打动击子儿,又高念起来。从张员外在黄梅山寺中修行,白日长跪听经,夜夜参禅打坐。四祖禅师见他不凡,收留做了徒弟,与了他三桩宝贝,教他往浊河边投胎夺舍,直说到千金小姐在浊河边洗濯衣裳,见一僧人借房儿住,不合答了他一声,那老人就跳下河去了。潘金莲熬的磕困上来,就往房里睡去了。少顷,李瓶儿房中绣春来叫,说官哥儿醒了,也去了。只剩下李娇儿、孟玉楼、潘姥姥、孙雪娥、杨姑娘、大妗子守着。又听到河中漂过一个大鳞桃来,小姐不合吃了,归家有孕,怀胎十月。王姑子又接唱了一个《耍孩儿》。唱完,大师父又念了四偈言:

\[
五祖一佛性,投胎在腹中,
权住十个月,转凡度众生。
\]
念到此处,月娘见大姐也睡去了,大妗子\textuni{22C49}在月娘里间床上睡着了,杨姑娘也打起欠呵来,桌上蜡烛也点尽了两根,问小玉:“这天有多少晚了?”小玉道:“已是四更天气,鸡叫了。”月娘方令两位师父收拾经卷。杨姑娘便往玉楼房里去了。郁大姐在后边雪娥房里宿歇。月娘打发大师父和李娇儿一处睡去了。王姑子和月娘在炕上睡。两个还等着小玉顿了一瓶子茶,吃了才睡。大妗子在里间床上和玉箫睡。月娘因问王姑子:“后来这五祖长大了,怎生成正果?”王姑子复从爹娘怎的把千金小姐赶出,小姐怎的逃生,来到仙人庄;又怎的降生五祖,落后五祖养活到六岁;又怎的一直走到浊河边,取了三桩宝贝,迳往黄梅寺听四祖说法;又怎的遂成正果,后来还度脱母亲生天;直说完了才罢。月娘听了,越发好信佛法了。有诗为证:

\[
听法闻经怕无常,红莲舌上放毫光。
何人留下禅空话?留取尼僧化饭粮!
\]
