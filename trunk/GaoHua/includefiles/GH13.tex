%# -*- coding:utf-8 -*-
%%%%%%%%%%%%%%%%%%%%%%%%%%%%%%%%%%%%%%%%%%%%%%%%%%%%%%%%%%%%%%%%%%%%%%%%%%%%%%%%%%%%%

\chapter{“抢救”风暴下的延安和各根据地}

\section{“抢救”的策略和手段}

“抢救”是在审干和反奸的基础上发展起来的,或是与审干、反奸交叉进行的,在运动的方法和策略上,既有相似处,也有差异。无论是审干、反奸,或是“抢救”,都有一个事先设定的主观判断,这就是凡知识分子干部和做白区工作的干部大多都有问题,而他们一般不愿主动向党作出完全、彻底的坦白。这样就必须首先研究他作自己交代的材料,按图索骥,步步深入,从中发现疑点,继而取得证据。但“抢救”在此基础上还要向前发展,即在获取证词的过程中,更多地诉诸暴力和恐吓的手段。

审干甫始,所有人员均需交待历史,此谓“写自传”。凡属从国统区来延安的知识分子,则需反复写,交领导机关检查,再“填月表”,交代传主每个月的历史。康生曾对社会部工作人员面授“斗争策略”。指示他们从受审人员写的“自传”中发现问题,康生启发道:

\begin{quoting}
刚来边区时写的自传,整风时写的自传,到反省机关写的自传,把这三部自传一对照就矛盾百出。有怕出毛病而挨斗争的,先留一个大纲,有个大纲写一百遍都不会错。你就叫他当面写自传,在我们这里写自传,他的大纲在家里,在这里写就有了漏洞\footnote{师哲:《峰与谷——师哲回忆录》,页~202。}。
\end{quoting}

依照这种策略,审查人员勒令被怀疑对象三番五次写自传,写月表,从中找出前后矛盾之处,再顺藤摸瓜,提“一万个为什么”,使受审人永远答不完。接着,又勒令受审人员揭发同伙,将与受审人员有联系的其他人一网收来,此谓“老鼠战略”——即以一人为突破口,逼其咬出其他人。

从审查人员的交待材料发现“敌人”固然是一种行之有效的方法,但是这种方法也有缺点,这就是既费时又费事,且不能大面积地发现“敌人”。针对这种情况,康生又采取另一谋略,这就是暗中布置特工在各单位可疑人群中故意散布“反动言论”,以钓出“反革命”。然而这种方法的效果也不太明显,因为在审干、反奸的紧张气氛中,绝大多数干部都谨言慎行,提高了警惕性,一般不会主动上钩。

1943~年~4~月~3~日,中共中央发出第二个“四三决定”,正式号召参加整风的一切同志大胆说话,互相批评,以大民主的方式,来批评领导,揭露错误。此项决定的真正意图在于“引蛇出洞”,“暴露敌人”。

遵照第二个“四三决定”,延安各机关、学校纷纷召开“民主检查大会”,刘少奇和其他中央领导干部还亲自参加了中央党校召开的民主大会。\footnote{张鼎丞:《整风在延安中央党校》,载《星火燎原》,第~6~集(北京:人民文学出版社,1962~年),页~8。}一些干部受到中央决定的鼓舞,居然忘了一年前王实味事件的教训,他们在“民主大会”上,慷慨激昂,情绪激动,言辞激烈地批评起领导的“官僚主义”、“压制民主”和“特权思想”;这些上台发言的人大多为知识分子干部,有的人在会场上甚至声泪俱下,泣不成声,于是一个个都跌入了早已为他们设计好的陷阱,成为“反革命”或“特务分子”。

通过检查自传发现了一批“反革命”,又通过召开“民主大会”再钓出一批“反革命”,但是“反革命”、“特务”的数目离上层领导头脑中的敌情估计还相差很远。这时,群众运动就派上了用场,各机关、学校普遍提高了敌情观念,大反右倾麻痹思想,纷纷以比赛的精神来清查特务分子。上级部门则以明确的指示或暗示来具体指导下级的审干小组成员:凡出身剥削阶级家庭,抗战后来延安的知识分子大多有嫌疑;曾被国民党逮捕、从事白区工作的同志也是重点审查人群;至于经常散扩不满言论的人更可能是特务分子,这样,就产生了对各单位“敌人”比例指标的要求问题。中央书记处秘密制定反奸策略,指示通过召开坦白大会,“形成坦白运动的潮流,造成群众的压力与群众的清查运动”,认为如此做了,“即可清出大批特务与各种有政治问题的人”。对于怎样召开坦白大会中央也有具体的布置:第一、“必须在各机关、学校、农村组织自卫军,实行放哨戒严,禁止会客及出入的自由,规定严格的生活起居制度”。第二、“在坦白大会之前,必须有慎重的严密的准备,除开调查一切嫌疑分子的材料,定出嫌疑名单之外,并须动员一批积极分子,秘密监视嫌疑分子;注意他每日的言论行动,在大会上察言观色;在他恐慌动摇到极度时,即抓紧对他劝说,督促他坦白并坦白别人。”\footnote{中央书记处:《发动华中反特运动指示》(1943~年~11~月~15~日);载中国人民解放军国防大学党史党建政工教研室编:《中共党史教学参考资料》,第~17~册,页~385;另参见蒋南翔:《关于抢救运动的意见书》(1945~年~3~月),载《中共党史研究》;1988~年第~4~期。}在上级机关的精心指导下,利用群众运动的方式,“依靠群众的喉咙,依靠群众的拳头,依靠群众声势汹汹的态度”,大批“反革命”和“特务”分子终于被挖出。可是这些被指称为“特务”、“反革命”的人员只承认自己有缺点和错误,却死不肯承认自己是“反革命”或“特务”。

下一步就需要进行政策攻心,逼迫受审人员承认自己是“特务”,这一步是全部过程中最困难的阶段。

提问者要求受审人员回答各种层出不穷、离奇古怪的问题:“你平日经常散布反动言论,勾结对党不满的分子,你必须老实交待,国民党派你来延安搞破坏的具体任务是什么?”

“某一次国民党大逮捕,其他同志被捕牺牲了,为什么只有你没被捕?”

“你在教会学校读书,一定参加了英国特务机关!”

“你的舅舅~1936~年从东北逃往北平,住在你的家,他是日本特务机关的特务,你也经他介绍,加入日特机关,成为日特机关的情报员,你在延安为日特提供了多少情报?”

“你说你是坐火车从上海到西安的,一定是国民党派你来的,没有国民党开的介绍信,你可以坐火车吗”?这个提问称得上是审干、反奸、抢救中的“经典提问”,许多被审人员都被问到这个问题。在绥德师范礼堂的斗争大会上,一个嫌疑对象被追问:“你没有特务关系,怎么能从上海到北平坐得上火车?”\footnote{韦君宜:《思痛录》,页~12、13。}

“你的父亲现在还在北平,又有钱,不是汉奸才怪!你和他什么关系?”\footnote{韦君宜:《思痛录》,页~12、13。}

“你家里又不缺吃又不少穿,你来延安干什么?”\footnote{李锐:《直言:李锐六十年的忧与思》,页~48。}

“你一贯积极工作,是为了取得组织信任,便于长期潜伏,不然的话,为什么放弃在国统区现成的正规学校不上,偏偏来边区吃苦?”\footnote{刘晓:《最美好的时光》,载《延安马列学院回忆录》,页~257。}

“你的同伙已经向党坦白自己是特务,他也揭发你是特务,你为什么还执迷不悟?拒绝党的挽救呢?”

如此荒诞不经的问题,不一而足,被审问者即使有一百张嘴,也难于回答清楚这类提问。这些审干领导小组成员,多数世代居住在偏远山沟,从未去过大城市,更没见过火车,加之头脑中已经形成的习惯思维,他们很难相信,世界上竟然有人不是为吃饱肚子,而是为了所谓信仰来投奔共产党。于是,为了敲开被审问者的嘴,只能诉诸于“强硬手段”了。

手段之一:疲劳战、车轮战。

逼迫受审人员几天几夜不合眼,审问者轮番休息,以连续作战的方式,利用受审者神智昏迷,精神崩溃,取得口供。1943~年~4~月至~1944~年~6~月,李锐被关押在边区保安处受审,曾经“五天五夜不准睡觉,不准瞌一下眼皮(有哨兵日夜持短枪监守,威胁)”,在保安处有人还受到长达十五天十五夜的疲劳审讯,“受审时,通常是长时间立正站着(以致腿肿)和坐矮板凳;有时加带手铐,时间长短不定”\footnote{李锐:《直言:李锐六十年的忧与思》,页~45。}。

手段之二:捆绑吊打、刑讯逼供。

据师哲披露,1943~年~5~—~6~月间,社会部讨论起草《审讯条例》。会上围绕是否用刑的问题,产生两种尖锐对立的意见,康生坚持主张用刑,说“不用刑,那怎么审讯”?\footnote{师哲:《峰与谷——师哲回忆录》,页~200~—~201。}于是,刑讯逼供成为审查中的主要手段。普遍的刑罚有将受审人员吊在梁上,施以鞭打,或加之拳打脚踢。仅据关中分区一个县的统计,在运动中就曾采用压杠子、打耳光、举空甩地等二十四种肉刑。据延安地区一个县委扩大会议记录记载,县委书记、区委书记都曾亲自上阵殴打受审人员,县委书记打入约十七人次,挨过县区领导打的有九十一人,被县委领导人私自关押的有二十九人\footnote{王素园:《陕甘宁边区“抢救运动”始末》,载中共中央党史研究室编:《中共党史资料》,第~37~辑,页~215~—~18。}。

保安处常用的刑罚有:老虎凳、鞭打、长时间带手铐、绑在十字架上抽打受审者,保安处处长“周兴即亲自这样打入”\footnote{李锐:《直言:李锐六十年的忧与思》,页~45。}。

手段之三:饿饭。

据李锐回忆,在边区保安处,专设“特字号”监房,收押“顽固分子”。“每人每餐只给半碗饭,有的人曾饿过一个多月”\footnote{李锐:《直言:李锐六十年的忧与思》,页~45~—~46、46。}。

手段之四,假枪毙。

在审干、反奸、抢救运动中,假枪毙是一种常见的斗争和惩罚方式。经过种种酷刑拷打,如果被审查者仍拒不交待,这时审查机关负责审讯的干部就会想到利用假枪毙的方式再作一次榨取口供的努力。选择假枪毙的时间一般在月黑风高之夜,将嫌疑分子五花大绑押往野地,嗖嗖几声枪响,子弹从耳边飞过,给受刑者造成极大的心理与肉体伤害,许多人甚至会长时间精神失常。原中共地下河南省委书记张维桢在中央党校受审期间,就曾被拖出去假枪毙。

种种刑罚、肉体折磨,以后被一句“逼、供、信”轻描淡写地遮盖过去。在统称“逼、供、信”的审讯方式中,精神和肉体上的折磨一般都是同时并进。精神上的折磨是反复不断地向被审查者进行“劝说”,恐吓、引诱受审对象进入早已布置好的圈套。精神施压若不奏效,就辅之以肉刑,或五花大绑,被施之以拳打脚踢,或被拖出去受“假枪毙”的巨大精神和肉体的折磨。有的人因多年监禁,成为“满头白发的青年”。\footnote{李锐:《直言:李锐六十年的忧与思》,页~45~—~46、46。}许多人因遭受这种双重折磨而导致精神和心理上的创伤,在经过许多年以后仍难以抚平。一个受审者当年曾被五花大绑,整整四十八个小时才给解开绳子,以至双手和指尖,全成暗紫色,两只手腕被绳子紧勒过的地方开始腐烂,绳痕一直到~1949~年后还没有蜕尽\footnote{王素园:《陕甘宁边区“抢救运动”始末》,载中共中央党史研究室编:《中共党史资料》,第~37~辑,页~215~—~18。}。

延安地区关押受审人员的机关共有四个:社会部看守所、西北公学、陕甘宁保安处和西北行政学院。

中社部看守所设于枣园后沟,在里面关押的都是重犯,王实味从~1943~年~4~月~2~日被押解到此,一直关押到~1947~年~3~月才被转移至晋西北的兴县。

西北公学也位于枣园后沟,距社会部看守所不远,是中央社会部集中受审人员的特别机构,1942~至~1944~年,共拘押有较“严重”问题人员五百余人,其中~480~人被打成“特务”、“叛徒”、“日特”。

陕甘宁保安处为公开的镇压机关,整风开始后,关押各类被挖出的“特务”五、六百人。到“抢救”高潮时,保安处所在地“凤凰山坡上一层层一排排的窑洞,已经挤满了人,又新挖了一些窑洞。原来一个坑睡四人,后来睡五、六人”,“挤得翻不了身”\footnote{王素园:《陕甘宁边区“抢救运动”始末》,载中共中央党史研究室编:《中共党史资料》,第~37~辑,页~215~—~18。}。

西北行政学院为西北局和边区系统关押一般嫌疑人员的临时集中营,共押有~908~个受审人员,其中包括三类人员:1942~年~4~月后,在秘密审干中“坦白”的边区工农出身的干部;被怀疑为“特务”的外来干部;以及“抢救”中送来的边区各厅、院、局、银行的知识分子出身的干部。于光远和以后长期担任邓颖超秘书的陈楚平(此人原为南京中央大学学生,也被打成“特嫌”)当时即被集中在此接受审查,并参加开荒劳动\footnote{于光远:《文革中的我》(上海:上海远东出版社,1995~年),页~53。}。

至于更大数目的各机关、学校的被抢救人员,则全部拘押在各自的单位,接受隔离状态下的审查。一经隔离,受审人员就失去了人身自由,不准回家,也不许通信。

在康生和各单位审干小组施行的精神、肉体双重折磨下,大批“特务”被制造出来,人们互相“揭发”,甚至许多夫妻也互相“咬”对方是“特务”。各单位、学校的“日特”、“国特”、“叛徒”鱼贯上台自首,有的还被树为“坦白”典型,胸佩大红花,骑在马上,风尘仆仆地巡回各地现身说法。1943~年夏秋之后,各机关、学校大门紧闭,门口由警卫把守,延安的人们已中断互相往来,“谁也不敢理谁”(王德芬语),在偌大的延安城,也需持介绍信才能办事。入夜,延安万籁俱寂,听不到一点声音,陷入一片恐怖、沉寂之中。

\section{中直机关的“抢救”}

1943~年~4~月~3~日以后,延安的审干运动走向高潮,原先由各单位整风审干领导小组对所在单位党员、干部秘密进行的“排队”、“摸底”,现在已发展到公开号召干部向党“坦白”。此时正式的口号是审干、反奸、坦白,还没有使用“抢救”一词,但从斗争的性质、内容和方式看,与稍后的“抢救”别无二致。

1943~年~2~月,两年前遵照中央指示,和一百余名干部一同撤退来延安的河南省委交通科长杜征远被送往延安中央组织部接受隔离审查(此时陈云已不过问中组部工作,由彭真代理中组部部长一职)。审干人员诱导启发杜征远:河南省委王志杰(1942~年初被中共中央任命为河南省委书记)、郭晓棠(河南省委宣传部长)、张维桢(前河南省委书记)、危拱之(河南省委组织部长)都是“特务”,“就你交通科长不是特务”?杜征远因为从未被捕过,实在交待不出来,审干人员就拿一条麻绳放在杜的面前,威胁杜:“不承认非勒死你不可。”杜征远在压力下被迫承认自己是“日本特务”、“国民党特务”,审干人员乘胜追击,又让社交待“是谁介绍的,在什么地方参加的”,“别人谁是特务,和谁联系”等问题。

中组部审干人员在威迫杜征远时点出河南省委主要负责人都是“特务”,意图给杜征远造成精神上的巨大震撼,尽管当时中组部、中社部还没有触动王志杰等河南省委负责人。1943~年~3~月,中央通知王志杰、危拱之、郭晓棠进中央党校参加整风学习,同时帮助中央审查河南干部,这时王志杰等一点也不知道,有关部门已经根据康生的逻辑,事先判定他们都是“特务”,攻下杜征远,就是为了以杜为突破口,榨出杜征远的口供,一举将王志杰等全部打成“特务”\footnote{张文杰:《河南党组织被康生诬陷为“红旗党”的历史真相》,载《河南党史通讯》第~1~期,引自廖盖隆主编:《中共党史文摘年刊》(1985)(北京:中共党史资料出版社,1987~年),页料~344~—~45。}。

1943~年春,审干、反奸运动已获得重大进展,不仅“张克勤案”已经包装完毕,通过攻下杜征远,河南“红旗党案”也初见眉目。4~月~9~日至\footnote{陈永发:《延安的阴影》,页~135。}12~日,延安连续召开坦白动员大会。4~月~12~日,由康生一手导演,将“张克勤反革命特务案”的主角张克勤,拉到八路军大礼堂,在千人参加的延安反奸坦白大会上正式亮相。康生在大会上作动员报告,说:这几个月我很忙,白天开会,晚上捉鬼。说到“鬼”字,顺手指看站在右侧的四个人,第一个即是张克勤。现在,张克勤已被树为“坦白典型”,从此不断地被带到各机关、学校去“现身说法”。在康生的直接指导下,各机关、学校纷纷掀起“坦白”高潮,通过开大会、小会、规劝会、斗争会和控诉会等五花八门的形式,压迫延安的党员干部进行坦白,至~7~月~9~日,已有四百五十人坦白。康生见状,极为振奋,于~1943~年~7~月~15~日,在延安中央大礼堂召开的中共中央直属机关干部大会上作《抢救失足者》报告。康生宣布延安已逮捕了二百多人,他并公开点出杜征远等的名字,说“破坏河南党的杜征远”是“敌探兼国特的兼差特务分子”。康生在报告中代表“共产党中央”,号召所有“为敌人服务”的内奸、特务迅速坦白。康生并解释“抢救”的涵义:“自然界的失足者,主要是外边的人来救,而政治上的失足者,遇到要抢救的时机,却主要是依靠自己”。在这个开启恐怖镇压大闸的会议上,彭真也发表了主题类似的讲话。朱德虽参加了大会,但在简短的讲话中,强调保护干部,与康生、彭真的演说大相径庭。由于朱德在延安只具象征意义而毫无实权,他的发言并不能冲淡会场的肃杀气氛。在~7~月~15~日干部大会上,被康生机关驯化的十二个人上台依次作了坦白。会场气氛恐怖,“一片沉重压迫的沉寂”,使许多与会者当场“吓得面色苍白,茫然失神”\footnote{康生:《抢救失足者》(1943~年~7~月~15~日),载中国人民解放军国防大学党史党建政工教研室编:《中共党史教学参考资料》,第~17~册,页~380~—~84;另参见师哲:《我所了解的康生》,载《峰与谷——师哲回忆录》,页~197。}。

紧接着~8~月~15~日,中共中央颁布《关于审查干部的决定》,开宗明义宣布“特务之多,原不足为怪”,并宣称“特务是一个世界性的问题”,文件的口气和文辞与毛的一贯风格如出一辙。这样,在毛泽东、康生的领导下,“抢救”、“肃奸”斗争迅速在陕甘宁边区展开。而中共中央直属机关,首当其冲成为“抢救”的重点。

中共中央直属机关属于任弼时、李富春的领导范围。1940~年任弼时自苏联返延安后,开始在党内负责组织和中直机关工作,1943~年~3~月,陈云休养后,原先由陈云领导的中央组织部已改由彭真掌握。延安整风后,实际负责中直机关整风审干运动的领导人是中组部副部长、中央办公厅主任李富春,及其副手中央办公厅秘书处处长王首道。任弼时作为中央分管组织及中直机关的领导人,也可能过问中直机关的运动情况。

据现有资料反映,在整风转入审干、抢救阶段后,任弼时的态度比较冷静,他在中直机关作动员报告时只是一般号召,并无个人创造。任弼时自二十年代后期以来,历经中共党内多次斗争,阅历、经验比较丰富,他的个人作风也比较公道。1942~年后,任弼时把工作重点放在领导西北局方面。实际领导中直机关审干、抢救的是总学委副主任康生和毛泽东的故旧李富春。

康生、李富春挑选的抢救对象即是以后在中共党内大名鼎鼎、当时担任中央统战部副部长、前中国女子大学副校长的柯庆施。事情表面上的起由是中央大礼堂外的墙上出现了一条标语,揭发柯庆施是坏人。而柯庆施在三十年代领导北平地下党时,党组织曾遭到国民党破坏,柯因出差绥远躲过国民党逮捕,故而被认为有叛徒嫌疑。事实上揪斗柯庆施有更复杂的背景。

有关柯庆施在延安被整肃的材料,目前所能见到的只有两则。即王明的《中共五十年》和师哲的《峰与谷——师哲回忆录》中的少量记载。柯庆施自五十年代末到六十年代初,紧密追随毛泽东,是毛氏预谋打倒刘少奇的极少数知情者之一,因而文革后柯庆施受到批判,他在延安被迫害的情况就被完全隐去了。

柯庆施在延安抢救运动中被整肃的时间大的是在~1943~年下半年,幕前指挥者是李富春。

根据王明和师哲的记载:

\begin{quoting}
中直机关为批斗柯庆施夫妇连续在中央大礼堂召开群众斗争大会,斗争会由李富春主持,李勒令柯庆施交待问题(王明和师哲均未谈及逼问柯庆施夫妇交待问题的内容)。一次斗争会从下午一直开到午夜,因柯庆施拒不承认自己有问题,李富春宣布柯庆施是反革命分子,随即柯庆施就被捆绑起来。在批斗高潮中,柯庆施的妻子因不堪迫害跳井身亡。虽然柯庆施没有被押往社会部,但对他的监视并没放松,柯一直被软禁在家中\footnote{王明:《中共五十年》,页~148;师哲:《峰与谷》,页~2~—~3。}。
\end{quoting}

柯庆施长期在国民党统治区做地下工作,与李富春谈不上有嫌隙,很显然,李富春是奉命办事,那谁是批斗柯庆施的幕后策划者呢?

可以得出的答案是:康生和刘少奇,而康生、刘少奇的活动又得到毛泽东的默许。

柯庆施与刘少奇有历史积怨。柯庆施是原中共北方局组织部长,1936~年~3~月,刘少奇赴天津担任中共北方局书记,对原北方局进行大改组,任命自己的老部下彭真取代了柯庆施担任北方局组织部长,并在党内展开了对柯庆施等人“左的关门主义错误”的批判。1939~年柯庆施被任命为中央统战部副部长,成为统战部部长王明的副手。现在打击柯庆施,已名正言顺:即柯庆施在北方局执行了王明的“左倾关门主义”,在抗战阶段,又执行了王明的“右倾投降主义”,在现阶段虽然不能对王明直接下手,但通过整肃柯庆施,可为打倒王明预作准备。

康生了解柯庆施和刘少奇在历史上的矛盾。从~1941~年起,康生就主动向刘少奇靠拢,1943~年~3~月,刘少奇成为中共第二号人物,康生正急欲向刘少奇示好,正是基于这个目的,康生拒绝了杨尚昆、王鹤寿、凯丰等人的意见,坚持要斗争柯庆施\footnote{杨尚昆等认为,仅凭一条标语就判定柯庆施有问题,证据不足,参见刘家栋:《陈云在延安》,页~113。}。这样,在康生的指导下,由李富春主持的对柯庆施的斗争,就成为献给毛泽东、刘少奇的一份厚礼,既讨好了刘少奇,又讨好了毛泽东。

据王明记载,在批斗柯庆施后的第二天,刘少奇受毛泽东的委托,曾对柯庆施有如下谈话:

\begin{quoting}
我们反对你,是因你早在二十年代就认识王明,1930~年你在王明的领导下,参加了反对立三路线的斗争,而从~1939~年起,你是中央统战部部长王明的副手。可是整风运动造成声势已经多时,而你在反对王明方面连一句话也没讲过。\footnote{王明:《中共五十年》,页~148~—~49。}
\end{quoting}

笔者认为,王明的这段回忆基本符合事实。在延安整风之前,及整风开始后的一段时间,王明与柯庆施因为工作联系较多,私交不错,尤其当柯庆施受王明牵累,被当作王明替罪羊挨斗争的~1943~年,柯庆施与王明确实里惺惺惜惺惺。柯庆施曾探望过病中的王明,与王明有过私下交谈,对王明的境遇表示过同情与关心。王明对柯庆施在这段时期给予他的友谊始终未忘,1965~年他在莫斯科闻知柯庆施病逝的消息,还特作诗以示怀念。

柯庆施在“抢救”后获得解脱的详情,至今中国未透露任何资料。许多迹象表明,柯庆施获得解脱系出自毛泽东的援救。毛泽东完全清楚柯庆施与刘少奇之间的旧怨新恨,在柯庆施濒临绝境时,毛援之以手,从此柯庆施对毛矢志效忠。延安整风后,柯庆施在中共党内的地位逐渐上升,他与毛泽东的关系也更加紧密。1948~年柯出任中共占领华北的第一个大城市——石家庄市委书记,以后又经薄一波提名,进入了华北中央局,尽管刘少奇认为柯庆施任此职并不合适。\footnote{参见薄一波:《七十年的回忆与思考》,上卷,页~463。}五十年代后柯庆施官阶不断上升,且对刘少奇、周恩来多有怠慢,成为毛泽东制衡刘少奇的重要力量。

中央民运工作委员会在~1943~年~3~月~20~日后统由以刘少奇为书记的中央组织委员会领导,邓发任书记,民运委员会下辖中央妇委、中央职工委员会和中央青委。在“抢救”中,妇委、职委大搞极左,“纷纷突破”,青委正式工作人员只有四人,已经“抢救”了两人,但仍受到上级的多次批评。当时在中央青委工作的蒋南翔深感,“如不气势汹汹的威逼镇压,大喊大叫,那就不但要在如火如荼的抢救高潮中显得落后,而且会被认为是对反特斗争消极怠工,是对特务没有义愤”,因为民委负责人邓发就曾援引毛泽东《湖南农民运动的考察报告》,说反特、抢救“好的很”\footnote{中央书记处:《发动华中反特运动指示》(1943~年~11~月~15~日);载中国人民解放军国防大学党史党建政工教研室编:《中共党史教学参考资料》,第~17~册,页~385;另参见蒋南翔:《关于抢救运动的意见书》(1945~年~3~月),载《中共党史研究》;1988~年第~4~期。}。

在中直机关被整肃的另一个重要干部是当时担任中央组织部秘书长的武竟天。抢救运动展开后,武竟天遭到斗争关押,其理由是武竟天在北平上学期间,曾向一意大利籍的传教士学过外语,就凭这一条,武竟天被扣上“意大利特务”的帽子\footnote{《李逸民回忆录》,页~117;《中共党史资料》,第~37~辑,页~216。}。

在抢救运动中,中直机关的一般干部更是受到冲击。据当时在中央办公厅秘书处工作的曹瑛回忆,蔡畅领导下的中央妇委一名余姓女同志,年龄尚不足二十岁,就被认定参加了“红旗党”。为“抢救”她,特在杨家岭礼堂召开群众大会,“头天晚上开大会,一直搞到第二天天亮,非说她参加了‘红旗党’不可。大会不断高呼口号,也有谩骂的”,会议主持人威胁道,“不坦白,加倍治罪”,但是小余“死也不承认”。正在病中的吴玉章听说这件事,拄着棍子来到会场。年高德劭的吴玉章见到这种斗争场面,难过地流下了眼泪,他哭着劝说道,“小余呀,你就承认了吧”。最后,小余被迫承认自己参加了“红旗党,是特务,来延安是搞破坏的”\footnote{曹瑛:《在延安参加整风运动和七大》,载《中共党史资料》,第~58~辑(北京:中共党史出版社,1996~年),页~9~—~10。}。

在深挖“红旗党”的高潮中,1935~年的“一二九”运动也被怀疑为是国民党“红旗政策”的产物,直至~1944~年~4、5~月间,还能听到这种说法,一些单位仍“把它作为审查干部的尺度”\footnote{中央书记处:《发动华中反特运动指示》(1943~年~11~月~15~日);载中国人民解放军国防大学党史党建政工教研室编:《中共党史教学参考资料》,第~17~册,页~385;另参见蒋南翔:《关于抢救运动的意见书》(1945~年~3~月),载《中共党史研究》;1988~年第~4~期。}。

中央办公厅秘书处是专为毛泽东和中央书记处服务的机要部门,当时有六十余名工作人员,这些工作人员在调入中办时,都受过严格的审查,但是在抢救运动中,还是有十几人被打成“特务”。在电讯科工作的密家凡,原是湖北地下党派到延安参加七大的代表,因七大缓开,被借调来秘书处工作。密家凡被别人咬为“特务”后,秘书处连续几天开会对他进行“抢救”,与会者向他提出各种问题,诸如“你在白区工作,你被特务、内奸包围得水泄不通,你一定是叛变当了特务”,密家凡据理反驳,拒不承认。最后在一个半夜里,社会部派来一辆马车把他抓走,关进了西北公学\footnote{曹瑛:《在延安参加整风运动和七大》,载《中共党史资料》,第~58~辑(北京:中共党史出版社,1996~年),页~9~—~10。}。

抢救运动袭来,甚至毛泽东身边的工作人员也难于幸免。在毛泽东身边工作的公务员罗海章、苟兴录也被江青当作“坏人”,被送入西北公学去接受“抢救”。\footnote{参见修来荣:《陈龙传》,页~194、128;何盛明:《陈刚》,载《中共党史人物传》,第~34~卷,页~219。}毛并没有出面予以干预。

中社部是领导延安“抢救”运动的权势机关,然而中社部工作人员在审干、“抢救”运动中也蒙受了冲击。中社部治安科科长陈龙的妻子海宇原是中国女子大学的学生,社会部为促成陈龙的婚姻,于~1941~年~8~月,经严格审查,将海宇从女大调入中社部工作,经中杜部批准,1942~年~11~月~7~日,陈龙与海宇结婚。但婚后不久,海宇就因河南“红旗党”问题(海宇为河南地下党员),被集中到西北公学受审,陈龙有一年多时间不得与海宇见面,直到~1944~年后,在中社部重要干部陈刚(何叔衡的女婿)的帮助下,海宇才被解除了审查。陈刚也保护了中社部机要科女干部申余,她因主持机要科墙报,被指责和王实味《野百合花》相呼应,而被送入西北公学受审查,后在陈刚的关照下,才得到解脱\footnote{参见修来荣:《陈龙传》,页~194、128;何盛明:《陈刚》,载《中共党史人物传》,第~34~卷,页~219。}。

中共中央直属的《解放日报》在整风运动开展后,实际上已由陆定一负责。博古虽仍担任社长一职,但职责范围已大大缩小,处于挨整、被冷遇的境地,因此,对“抢救”十分小心。在“抢救”运动初期,《解放日报》并没有挖出什么“特务”,康生见此状极为不满,公开指责博古:你们清凉山(《解放日报》所在地)是特务成堆的地方,你们就是抓不出来?在康生的威胁、恐吓下,博古不得不派人前往西北公学“学习取经”,返回后在《解放日报》如法炮制。

《解放日报》被抢救出来的“特务”占全体人员的~95\%(另据温济泽叙述,《解放日报》社和新华社总共一百几十位工作人员中,被逼承认自已是“特务”的占~70\%~左右)。副总编辑余光生积极执行康生的指示,全力在报社“挖特务”,“抢救”前夕,接替丁玲工作的艾思奇此时也受到怀疑,被免去学委委员。副刊部的舒群、白朗、陈企霞、黎辛都被视为是嫌疑对象。副刊部秘书温济泽在对王实味的斗争中表现积极,此时也因他有一叔叔是国民党少将,被说成是国民党派来的特务。\footnote{《百年潮》,1997~年第~1~期,页~33;另参见李锐:《直言:李锐六十年忧与思》,页~44~—~45;温济泽:《第一个平反的“右派”:温济泽自述》(北京:中国青年出版社,1999~年),页~161、175~—~77。另参见李锐:《直言:李锐六十年的忧与思》,页~44。}李锐此时任《解放日报》国内部编辑,他的一个大学同学因“托派”嫌疑被抓,因不堪刑讯,承认自己是“特务”,并咬出李锐是他的“上级”。1943~年~4~月,在边区第一次大逮捕的浪潮中李锐被捕,作为重犯,从~1943~年~4~月至~1944~年~6~月,被关押在边区保安处\footnote{李锐:《清凉山的文宇生捱》,载《李锐往事琐忆》(南京:江苏人民出版社,1995~年),页~46。}。

在《解放日报》社,除了编辑、记者遭“抢救”外,抢救也在印刷厂进行,一位总务科长因不堪逼供,被迫刎颈自杀。博古对于毛泽东、康生的这套红色恐怖手法极为熟悉,早在~1931~年,博古就对远在江西的毛泽东“打~AB~团”的行为有所怀疑。1942~年~3~月,王震、贺龙曾就《解放日报》公开发表丁玲的《三八节有感》和王实味的《野百合花》,向博古发难,王震、贺龙或亲去清凉山博古的窑洞,或在党的高层会议上,严厉指责博古。时至~1943~—~1944~年,博古在党内的地位愈加脆弱,“破鼓万人捶”,博古在高干中已成为人人皆可唾责的对象,他已无法制止《解放日报》中的“抢救”极端行为。

在抢救运动中,中直机关的所有单位均受到冲击,中央医院~90\%~的医护勤杂人员受到怀疑,著名的马海德医生和他妻子苏菲也曾被“抢救”。对马海德提出的疑问是:“一个外国人抛弃了优裕的物质生活,从美国来到上海,又从上海到延安,它的背后是什么?”由于马海德是边区急需的医生,因而对他还较为客气,只是劝其“坦白交待”,而未将其关押\footnote{周森:《马海德》(北京:三联出版社,1982~年),页~25~—~26。}。

\section{军直机关的“抢救”}

“抢救运动”对中央军委直属机关也造成极大的冲击。设在延安王家坪的中央军委是毛泽东指挥八路军、新四军的最重要机构,毛泽东也通过军委的电讯系统搜集各根据地的动态以及国统区的各项情报。进入~1943~年后,原中央军委副主席王稼祥已靠边站。自~1941~年~2~月担任中央军委参谋长的叶剑英,其职权范围仅限于作战谋划、军事情报搜集等业务领域,对审干等政治工作的影响力很小,实际上,军委系统的审干、抢救领导权,基本掌握在康生机关手中。

在审干、“抢救”展开后不久,担任军委秘书长兼政治部秘书长的陶铸很快被“挖”出来。选择陶铸作为靶子,是因其在~1933~至~1937~年被国民党拘押于南京中央军人监狱,被怀疑有“特务”嫌疑,当时陶铸还兼任王稼祥的政治秘书,打击陶铸也有暗打王稼祥的用意。

陶铸被隔离审查后,情绪极为愤懑。当社会部干部李逸民前去探视他时,陶铸“暴跳如雷地在骂娘”。\footnote{李逸民:《参加延安“抢救运动”的片断回忆》,载《革命史资料》,第~3~辑(北京:文史资料出版社,1981~年),页~37。}陶铸落难迟于柯庆施,他曾在柯庆施被隔离后,前往软禁柯的窑洞探望过他。在延安审干、抢救中的这段遭遇,以后成了联络陶铸与柯庆施的感情纽带,柯庆施与陶铸这两位有过白区工作长期经历的干部与刘的关系一直比较冷淡,而刘少奇在审干中,位居决策地位,极有可能插手过对柯庆施、陶铸的审查。当毛泽东对柯、陶伸出援手后,这两人就成了死命效忠毛的大将。1949~年后,柯庆施、陶铸皆获毛泽东重用,六十年代初分任地位极显赫的中共华东局和中南局第一书记的职位。1953~年陶铸还一度涉入当时高层领导人内部对刘少奇的批评议论(即“高岗事件”),也是因毛泽东的保护才未被打入“高饶反党集团”,事后继续获毛泽东的重用。1965~年柯庆施病逝,陶铸极为难过,曾在家中悲伤落泪。1966~年春夏,毛泽东调陶铸入中央,实指望利用陶铸与刘少奇的宿怨,向刘少奇开炮。在中共八届十一中全会上,毛泽东将陶铸一下提拔为中共第四号人物,谁知陶铸竟辜负毛泽东的厚望,不忍对刘少奇过份打击,在被多次提醒后仍无改变,终遭毛泽东的抛弃。柯庆施则因早逝,未卷入文革,但江青在六十年代初于上海策划反刘少奇的阴谋,得到了柯庆施的全力支持。1943~年在延安被整的柯庆施和陶铸,成为毛泽东手中掌握的日后对付刘少奇的两颗棋子。

中央军委直属单位在“抢救”中受到严重冲击,使日常业务工作几乎一度陷于停顿。中央军委总参谋部下辖三个局:一局负责作战指挥,由伍修权任局长;二局负责情报收集与分析,由曹祥仁任局长;三局主管通讯,主要负责延安与各根据地的电讯联络,负责人为宁都暴动后加入红军的王铮。由于军委所属三个局从事的工作极具机密性,工作人员都已经过严格的审查,他们需要与家庭断绝通信联系,出门要与人同行。\footnote{《延安马列学院回忆录》,页~171。}尽管如此,在“抢救”中还是揪出了大量“特务”。

有关军委一局审干、“抢救”的详情至今仍缺乏资料,目前所知的仅是周秋野等受迫害的零星情况。在军委一局测绘科工作的周秋野(七十年代至八十年代曾任中国驻捷克斯洛伐克大使),在审干过程中,受到车轮战和逼供信的迫害。另据伍修权透露,一局的协理员张炽昌因为曾经在国统区做过兵运工作,也被关押起来。负责审查和看管他的人“竟蓄意折磨他,在给他吃的饭里加了盐,又不给水喝”。张炽昌被迫上吊自杀,幸亏被人发现,才得以生还\footnote{伍修权:《回忆与怀念》(北京:中共中央党校出版社,1991~年),页~195~—~96。}。

军委二局的审干、反奸与所有军委直属单位一样,受总学委和军直机关学分会领导,实际领导者仍是康生。当时年仅二十九岁、担任总政治部组织部长的胡耀邦作为军直系统整风领导小组成员,也曾参与领导二局的审干、反奸。

从~1942~年~12~月始,二局陆续发现“内奸特务分子”。在~1943~年~4~月~1~日大逮捕被抓获人员中,其中有二局的四个人。康生很快将这四人放回二局,要他们作坦白示范。1943~年~4~月~15~日前后,胡耀邦主持二局坦白大会,号召大家提高警觉,“嗅奸”、“监奸”、“证奸”,同时也提醒不得冤枉好人。\footnote{陈永发:《延安的阴影》,页~230、241~—~42。}会后,在很短时间内,二局共收到十余万字的检举材料。二局还创造出一整套劝说坦白分子的办法:“善劝、亲劝、你劝、我劝、软劝、笔劝、硬劝”等,甚至还发明了一种“雷公劈豆腐”的办法,即先集中攻下动摇分子,再对付顽固分子。\footnote{陈永发:《延安的阴影》,页~230、241~—~42。}到~5~月上旬,二局已有十人坦白,运动迅速走向热潮。在群众已充分发动的形势下,胡耀邦愈来愈趋向冷静,他在~5~月~6~日划出四条政策界限:一、严禁打入骂人;二、非经委员会批准,不得捆人;三、没有充分证据,不得逼供;四、严防自杀。对于二局的运动,社会部极为重视,\footnote{陈永发:《延安的阴影》,页~237、240~—~41。}康生派出李克农前来二局亲临指导,要求进一步开展坦白运动,挽救陷于特务泥沼的失足者。李克农并表示,对已坦白者,党保证他们有光明的前途。在这种形势下,一方面,胡耀邦布置开展新一轮坦白运动;另一方面,他也不失时机地强调要注意政策。胡耀邦提出,应控制检举次数,以书面检举代替口头检举;自首分子也应“实事求是”。“有冤枉就说出来,不要害怕,负责审干的领导,应为受冤人员申冤辩解”\footnote{陈永发:《延安的阴影》,页~241。}。这些情况都说明胡耀邦和那些利用审干、反奸、蓄意整人的干部有本质上的区别。

军委三局是军委几个局中工作人员最多的一个局,有工作人员近千名,大多为抗战后投奔延安的青年知识分子。在审干、反奸中,三局电讯学校二百人中已有一百七十人被扣押,遭斗争。三局各科室大多数工作人员随后都被打成“叛徒”和“特务”,由于三局“特务成堆”,一段时间,延安总部与各地的电讯联络已难以为继。由于三局所负责的工作极端重要,因此当~1944~年元旦,王铮率三局工作人员给毛泽东拜年时,毛向三局受审问者表示了歉意,首先解脱了他们。

中国人民抗日军政大学(简称抗大)的前身是瑞金时期的中国工农红军学校,1936~年~6~月易名为中国人民抗日红军大学(简称红大),1937~年春正式定名为中国人民抗日军政大学。全面抗战爆发后,毛泽东一再指示,抗大要向全国革命青年大开入学之门,把抗大招生的广告,从延安贴到西安,每根电线杆都贴上一张。\footnote{李志民:《革命熔炉》(北京:中共党史资料出版社,1986~年),页~241。}在中共的感召下,大批外来知识青年投奔延安,进入抗大学习,使抗大的规模不断扩大。1939~年~6~月,因边区粮食困难,中央政治局决定抗大总校转移到晋东南,1943~年春,延安又命令抗大总校返回陕甘宁边区,移驻绥德,并将延安军事学院、抗大两个分校等合并于抗大总校,任命徐向前为抗大校长,李井泉为政治委员,何长工、彭绍辉为副校长,此时全校共有学员六千余人。

抗大的整风、审干开始于~1943~年~8~月,运动初期,在以徐向前为书记的抗大总学委的领导下,一切进展平稳,徐向前领导成员讲红四方面军肃反扩大化的教训,提醒大家头脑保持清醒。\footnote{《当代中国人物传记》丛书编辑部编:《徐向前传》(北京:当代中国出版社,1992~年),页~399。}但是随着中央社会部派出以黄志勇为首的整风审干工作组到达抗大,形势迅速恶化。从~1943~年~10~月中旬到~12~月下旬,全校开展了深挖“反革命”、“特务”的“全面突破”竞赛。

徐向前在晚年出版的回忆录《历史的回顾》中详细地描述了抗大“抢救”的场景:

\begin{quoting}
此后两个月的时间里,抗大整风被弄得一塌糊涂。名堂多得很,什么“即席坦白”、“示范坦白”、“集体劝说”、“五分钟劝说”、“个别谈话”、“大会报告”、“抓水萝卜”(外红内白),应有尽有。更可笑的是所谓“照相”。开大会时,他们把人一批批地叫到台上站立,让大家给他们“照相”。如果面不改色,便证明没有问题;否则即是嫌疑分子,审查对象。他们大搞“逼供信”、“车轮战”……真是骇人听闻。\footnote{《徐向前传》,页~346;另参见李志民:《革命熔炉》,页~130~—~31。}
\end{quoting}

自从延安中社部派来的工作组进驻抗大总校后,徐向前事实上已被靠边站,在这种形势下,他已无事可做,只得于~10~月下旬返回延安。

徐向前经历过党内长期的残酷斗争,对极左的一套始终持有警惕的态度,返回延安后,他仍然经常过问抗大总校的运动情况,但他根本无法正面对抗这场由毛泽东、康生直接领导的运动。\footnote{《当代中国人物传记》丛书编辑部编:《徐向前传》(北京:当代中国出版社,1992~年),页~400。}和徐向前相比,抗大总校的其他领导人态度就不那么冷静了。韦君宜当时在绥德,亲耳听到杭大一位副校长介绍抗大的反特斗争原则。这位副校长说:别人说反对逼供信,我们就来个信供逼。我们先“信”,“供”给你听,你不承认,我们就“逼”!韦君宜说,这个副校长后来在文革中“闹得全家惨死”,“我不知道他曾否回想过~1943~年他自己说的这些话!”\footnote{韦君宜:《思痛录》,页~18。}

在极左风暴的袭击下,抗大总校在~1052~个排以上干部中,共挖出“坦白分子”、“嫌疑分子”、“特务分子”602~人,占总人数的~57.2\%。在干训队~496~人中,“嫌疑分子”竟达~373~人,其比例为~75.2\%。\footnote{《当代中国人物传记》丛书编辑部编:《徐向前传》(北京:当代中国出版社,1992~年),页~399~—~400。}陕甘宁晋绥联防司令部所属部队是驻守延安地区的卫戍部队,由贺龙担任司令员,该司令部所辖的中共唯一一个炮兵团,因八路军基本从事游击作战,炮兵团派不上用途,所以一直被置放在南泥湾搞开荒生产,与外界处于半隔绝状态,该团许多干部甚至对延安动态也知之甚少。炮团干部整体文化素质较高,有不少人甚至在苏联学习过。仅教导营就有近二百名知识分子。在“抢救”中,炮团~90\%~的人被打成特务分子,有个模范连的干部~100\%~被打成“特务分子”。确定“特务”的标准非常简单:出身知识分子,来自国统区,不是“日特”。就是“国特”;若原是中共地下党员,则是“红旗党”、“假党员”、“叛徒”。炮团参谋徐昭,因其父在国统区煤窑上管财务,他就顺理成章被打成了“特务”\footnote{郭化若:《我军第一所炮兵学校的组建》,载《中共党史资料》,第~41~辑,页~134~—~35。}。

\section{西北局和边区系统的“抢救”}

陕甘宁边区系统的整风、反奸、抢救,大致上分两摊进行,中共西北局和边区全范围的运动由高岗领导,边区政府机关部分由李维汉领导,中央书记处虽指定任弼时代表中央指导西北局,但实际上是由康生掌握领导运动的实权。

1943~年~7~月,西北局召开坦白、抢救大会,一批张克勤式的“特务分子”被安排上台自首。与中直机关的情况相类似,西北局机关被“抢救”的对象大多为从国统区来的中共地下党员和青年知识分子。西北局民运部共有十多个干部,多为原西北各省中共地下党党员,均为知识分子出身。1941~年,民运部机关党支部曾办有《西北风》壁报,对延安某些消极现象展开批评,由陈元方主编。该壁报在~1942~年春王实味事件发生后,迅即停刊。高岗指责《西北风》流露“小资产阶级情绪”,是“发泄对他的不满”,陈元方随即被调往靖边县委担任常委兼统战部长。1943~年~4~月,陈元方突然被通知返回西北局机关,“挎包未放下便被叫去参加批判会”,要陈元方交待《西北风》问题。批判者指责《西北风》与《轻骑队》一样,是“反党”的。随后陈元方遭到“无休止的批斗”,1943~年将陈元方作为“死顽固”的典型。进行“假枪毙”,最后送到边区保安处关押,而西北局民运部的所有干部都被打成“国特”、“叛徒”\footnote{王素园:《陕甘宁边区“抢救运动”始末》,载中共中央党史研究室编:《中共党史资料》,第~37~辑,页~215~—~18。}。

著名学者于光远此时在西北局第四局西北问题研究室作研究工作,在“抢救”运动中也被打成“特嫌”,被送到行政学院“特别班”接受审查。\footnote{于光远:《文革中的我》(上海:上海远东出版社,1995~年),页~48。}中共西北局党校的教员大多为青年知识分子,在运动中许多人被打成“特务”。张宣,原中共成都市委书记,来延安后被分配在西北局党校任教,因被诬为“国特”,被送至边区保卫处关押。与西北局机关的“抢救”相同步,边区政府机关的运动也走向高潮。指定李维汉领导陕甘宁边区政府系统的整风、审干,是毛泽东一项颇为微妙的安排。李维汉是中共元老,五四时期即与毛泽东熟识。但在~1927~年后,李维汉与毛泽东并无密切交往。1933~—~1934~年,李维汉在瑞金曾积极执行博古等打击“罗明路线”和“邓、毛、谢、古”的方针,与毛泽东的关系十分疏远。自三十年代中后期始,李维汉转变态度,主动接近毛泽东,尤其在~1942~年主持中央研究院工作期间,李维汉积极遵从毛泽东批判王实味的部署,与毛泽东的关系已经十分密切,但是毛对李维汉还在观察中。1942~年~9~月,李维汉调入边区政府后,康生下令禁止李维汉阅读中央电报,尽管边区政府主席林伯渠抵制了康生的这道禁令,仍指示将中央收发电文\footnote{董纯方等:《呕心呖血抓建设——关于林伯渠同志在陕甘宁边区的片断回忆》,载《怀念林伯渠同志》(长沙:湖南人民出版社,1986~年),页~125。}送给李维汉阅看,但是,李维汉不会不知道在康生这道禁令后所意味的是什么。对于康生的这个信号,李维汉没有表示出任何不满和消极,相反,李维汉以更积极的姿态来贯彻落实毛泽东和总学委的部署。

1943~年~5~月下旬,由李维汉主持,在边区大礼堂召开了坦白动员大会,当场有张克勤式的典型作坦白交代。在这次大会后,边区政府所属各单位都对审查对象采用了车轮战的方式进行逼供,“只有程度不同之分”。边区政府秘书处至少有三次车轮战,其中一次是专门斗争区棠亮,当时又名唐亮,罪名是参加了湖南“红旗党”。区棠亮曾任中央妇委书记蔡畅的秘书,后调任林伯渠的秘书,她被施以逼供信,“甚至要进行捆绑吊打”。\footnote{《林伯渠传》编写组:《林伯渠传》(北京:红旗出版社,1986~年),页~286。}另一次是“追托派”,被斗对象“被弄得精神失常”。第三组车轮战是“追叛徒”。

边区政府系统“抢救”的高潮是追逼当时已近六十岁的原中共四川省工委委员张曙时。这次会议由李维汉亲自主持,在边区政府大会议室进行,“约有二三十人参加”。李维汉在会上宣布,凡是有问题的,都应该坦白交待,但是会场上没有人发言。李维汉站起来说:“有些老同志回延安已经有些时候了,但问题没有交待”,会上仍是鸦雀无声,一片肃然。这时,张曙时问:这话是不是对着我讲的?李维汉大声说:就是对你讲的,今晚的火,就要烧到你的头上去。张曙时愤怒道,你有什么证据?把证据拿出来嘛,我没有问题,这种会我不参加。张曙时随即走出会场\footnote{陈永清:《纪念林伯渠同志诞生一百周年》,载《怀念林伯渠同志》,页~136。}。

张曙时原是国民党左派,名列南昌暴动后成立的革命委员会二十五名委员之一,1933~年参加了共产党,在上海从事情报工作。1940~年,张曙时与邹凤平等从四川撤回延安,先后任西北局统战部副部长和边区政府法制室主任,被打成了“红旗党”,本不为奇,所幸边区政府主席林伯渠宽厚正直;对张曙时多方安慰,才使张曙时得以熬过难关。对区棠亮遭受磨难,从不动怒的林伯渠甚至拍了桌子,才使区棠亮获得了保护。

林伯渠此时虽任边区政府主席,也名列边区系统的分学委负责人,但领导运动的大权却由李维汉掌握。林伯渠处事谨慎,待人诚恳,在边区享有崇高声望。在毛泽东的眼中,像林伯渠这样的忠厚长者只可列为“好好先生”一类,\footnote{1947~年受康生在晋绥推行极左土改的影响,陕甘宁边区土改也发生了乱批乱斗的现象,引起百姓恐慌。林伯渠为保护下属,主动承担责任,自称是“好好先生”,对犯错误同志帮助不够。参见刘景范:《回忆林伯渠同志在陕甘宁边区》,载《怀念林伯渠同志》,页~118。}显然不是领导此类整肃运动的合适人选,故有派李维汉来边区政府协助工作之举,而李维汉本人也属被考察的对象。

在~1943~年抢救高潮之际,边区系统许多原由林伯渠介绍、从国统区来延安的青年都被整成“特嫌”、“红旗党”(林伯渠曾任中共驻西安代表,中共出席重庆国民参政会代表,多次前往重庆),林伯渠虽然多加安慰,一时也无能为力,只能告诫这些被整青年,“你们可不要说违心的话,要实事求是”。\footnote{《林伯渠传》编写组:《林伯渠传》(北京:红旗出版社,1986~年),页~137。}1943~年~9~月后,边区政府系统一批被怀疑为有问题的干部先后被调到中央党校三部接受进一步的审查。

在“抢救”中,边区各机关“捷报频传”。边区政府研究室、法院、财政厅、教育厅都挖出了“特务”。散文家吴伯箫此时任教育厅中等教育科科长,被打成“国民党~CC~特务”。边区银行行长黄亚光更成为边区远近闻名的“特务头子”,1942~年下半年,康生在进行内部审干、肃特试点时就选中了黄亚光作为“国特”的标本,将其秘密逮捕和关押。到了“抢救”展开后,又将黄亚光拉出来,逼迫他指控所谓同伙,并全被送往保安处关押。

由著名戏剧家塞克任院长的延安青年艺术剧院,除干部受冲击外,还波及一般勤杂人员。有一个为剧院赶大车的人,被诬为“特务”,遭捆绑吊打,被“活活折磨致死”。\footnote{王素园:《陕甘宁边区“抢救运动”始末》,载中共中央党史研究室编:《中共党史资料》,第~37~辑,页~215~—~18。}

延安保育院为延安唯一收养干部子弟和中共烈士子弟的幼儿园,该院的工作人员也无法躲过“抢救”。李维汉承认,在保育院曾出现“车轮战”的过火现象。

在直属边区政府领导的延安县和边区各县也展开了紧张的反奸、“抢救”运动。

1943~年~4~月,康生将其妻、时任中宣部干部科科长的曹轶欧派往延安县担任县委委员和县委宣传部长,曹轶欧下放延安县的目的是创造一套群众性反奸运动的经验,以证明毛泽东、康生发动整风、审干、反奸运动的合理性。

曹轶欧甫抵延安县,迅速将所谓嫌疑分子集中到整风学习班,“大她以反右倾麻痹思想”、“提高对敌斗争警惕性”为口号,一口气把延安县宣传部部长许平、县政府第三科(教育科)科长谈锋、蟠龙区宣传科科长黄流、县委宣传部干事杨志功等打成“特务”、“汉奸”。曹轶欧继而发起“坦白运动”,再施之以“瓜蔓抄”战术,将一大批区、乡干部,小学教员,定为“特务”。川口区宣传科科长蓝琳彬在诱供、套供的巨大压力下,被迫承认自己是“三面间谍”,即“国民党特务”、“日本特务”和“意大利特务”,她何以荒唐至成为“意大利特务”,乃是蓝琳彬的丈夫武竟天(原中组部秘书长)此时已被定为“意大利间谍”。另一女干部,牡丹区宣传科科长苏平,因拒不承认自己是“特务”,被曹轶欧指控用“美人计”引诱农民干部做特务,下令将其逮捕人狱\footnote{陈永发:《延安的阴影》,页~282~—~83。}。

1943~年~7~月~7~日,延安县举行群众性反奸坦白大会,贾拓夫代表西北局出席,经曹轶欧训化的男、女干部纷纷上台坦白,承认组织暗杀队,企图杀害党的干部。曹轶欧则号召失足分子抓紧坦白,宣称党的政策是“坦白从宽,抗拒从严”。\footnote{陈永发:《延安的阴影》,页~279。}大会持续近一天,最后由胡乔木讲话,揭露国民党“特务政策”的罪恶\footnote{陈永发:《延安的阴影》,页~279。}。

9~月~4~日,延安县在蟠龙召开四千人参加的除奸动员大会,高岗发表了极具鼓动性的反特动员讲话,紧接着有二十三人上台坦白,这些坦白的人员在两天前受到高岗的亲自接见,高岗“赌咒发誓”,保证他们坦白后绝不会被杀头。\footnote{陈永发:《延安的阴影》,页~288。}最后大会安排一个已经坦白的“特务”的“白发老母,踩着小脚,蹒跚的步上主席台,在涕泗横流中感谢中共对她儿子的宽大为怀”\footnote{陈永发:《延安的阴影》。页~252。}。

曹轶欧在延安县的蹲点为开展群众性的反奸、抢救运动,提供了生动有力的证据:

一、各类特务运用种种办法已全面渗透到边区,不仅城里有特务,广大农村也有特务,敌情之严重,远远超过原有的估计。特务的破坏活动五花八门,从散扩谣言,挑拨本地干部与外来干部的关系,到收集情报,拉拢干部下水,以及组织暗杀队,阴谋杀害干部。因此必须发动群众性的肃奸反特运动。\footnote{陈永发:《延安的阴影》。页~252。}

二、特务分子多为文教系统的干部和小学教员,以外来知识分子为主,延安县所属各区的宣传科长几乎全部是“特务”,所以知识分子应是被清查的重点。

三、贯彻“坦白从宽,抗拒从严”的政策,可促使特务分化,而已坦白的特务又可作为示范,为挖出更多的“特务”服务。

四、利用召开坦白大会,让已坦白的特务上台现身说法,可以焕发边区人民对国民党“特务政策”的仇恨,大大提高边区人民对敌斗争的警惕性。过去边区自卫军(民兵)对锄奸没有感性认识,现在知道特务、汉奸就在身边,于是加强了警戒,仅延安县牡丹区,十几天内就在查路条中发现了三十二个嫌疑分子,全部交给了政府处理。\footnote{陈永发:《延安的阴影》,页~287~—~288。}

曹轶欧的经验为运动的深化加注了动力,边区各县以延安县的经验为榜样,纷纷开展群众性的反奸抢救运动。绥德地委在“抢救”中雷厉风行,将一大批知识分子打成特务。杨述(五十至六十年代任中共北京市委宣传部长)与其妻韦君宜当时是地委《抗战报》的编辑,很快就被当成“特嫌”揪了出来。杨述因是从四川来的地下党员,此时因四川党已被定为“伪党”,遂被关进整风班交待问题,每天凌晨,还要冒着十二月的寒风在无定河边劳动。韦君宜受其夫牵连,也被组织怀疑,她虽然带着一个一岁多的孩子,但仍被勒令搬出原住的窑洞,搬进一间四处漏风的破房。韦君宜万箭穿心,“穿件破褂子,脚上穿着捡来的别人不要的破鞋”,想着自己抱着纯真的理想投身革命,今天却被弄成这个样子,偷偷吟成一首小诗:

\begin{quoting}
小院徐行曳破衫,风回犹似旧罗纨。

十年豪气凭谁尽,补罅文章付笑谈。

自忏误吾唯识字,何似当初学纺棉。

隙院月明光似水,不知身在几何年。\footnote{韦君宜:《思痛录》,页~18。}
\end{quoting}

绥德师范是“抢救”取得重大“胜利”的模范单位。1943~年~9~月,绥师连续召开九天的控诉坦白大会,“自动坦白者二百八十余人,被揭发者一百九十余人”,“一个十四岁的小女孩——刘锦梅,走上台只比桌高一点”,坦白地参加了“复兴社”。“十六岁的男小孩——马逢臣”,“手里提着一大包石头”,坦白他是“石头队的负责人,这包石头是他在特务组织指使下,谋杀人用的武器”。据曾参加当年绥师坦白大会的作家韦君宜回忆:在坦白大会上,一个叫白国玺的学生交代说,“是特务组织指使他,叫他在厕所墙上胡乱画猥亵的画”。“又一学生说,他搞的‘特务破坏’是用洗脚盆给大家打饭打菜”。绥师“整风领导小组”还破获了一个“特务美人计”组织,“据说这些女学生竟接受了特务的口号:‘我们的岗位,是在敌人的床上’,而且按年级分组,一年级叫‘美人队’,二年级‘美人计’,三年级‘春色队’”\footnote{韦君宜:《思痛录》,页~8。}。最后,绥师竟挖出~230~个“特务”,占该校总人数的~73\%。

绥师的反特经验被上级指定登载在绥德地委的《抗敌报》。当时任该报编辑的韦君宜等将一个已经坦白的女学生刘国秀写的坦白文章加上《我的堕落史》的标题刊载。此文一经发表,“后来的来稿越来越踊跃,越写越奇,特务从中学生‘发展’到小学生,十二岁的,十一岁的,十岁的,一直到发现出有六岁的小特务”\footnote{韦君宜:《思痛录》,页~8。}!

陇东地区的“抢救”更是酷烈,地委负责人李景波公开提出,由于~1935~年国民党开始实行党化教育,所以从那时以后来边区的知识分子都有问题。\footnote{中央书记处:《发动华中反特运动指示》(1943~年~11~月~15~日);载中国人民解放军国防大学党史党建政工教研室编:《中共党史教学参考资料》,第~17~册,页~385;另参见蒋南翔:《关于抢救运动的意见书》(1945~年~3~月),载《中共党史研究》;1988~年第~4~期。}刘晓——离休前任辽宁省计委副主任,1942~年后被分配至陇东地委专署所在地的庆阳做财贸工作。“抢救”中被定为国民党“特务”,被押进专署保安处,受到十天十夜车轮战和假枪毙的威胁,而审讯人员则每两小时换一组人。刘晓最后被折磨得大吐血。

西北局和边区系统在反奸抢救中究竟有多少人受害,据李维汉称,当时边区政府机关只有工作人员三、四百人,受“抢救”的百余人,“其中的二、三十人嫌疑较大,送往保安处,五十余人有一般嫌疑,送往行政学院(临时审查机关)”。九十年代初,有史料称,延安所属各县在“抢救”中,被挖出的“特务”共~2,463~名,\footnote{王素园:《陕甘宁边区“抢救运动”始末》,载中共中央党史研究室编:《中共党史资料》,第~37~辑,页~215~—~18。}陇东地区外来知识分子几乎~99\%~被当作“失足者”给“抢救”过。\footnote{刘晓:《最美好的时光》,载《延安马列学院回忆录》,页~257。}在这一时期,仅延安一地自杀者就达五、六十人。至于西北局系统的“抢救”人数,官方至今未透露具体数目\footnote{王素园:《陕甘宁边区“抢救运动”始末》,载中共中央党史研究室编:《中共党史资料》,第~37~辑,页~215~—~18。}。

\section{中央党校的“抢救”}

中央党校原先只有一部和二部。党校一部设在小砭沟,在该部集中的大多为原准备参加中央七大的代表和中共师旅级及地委级高级干部。部主任原为黄火青,1942~年~2~月后,黄火青改任中共党校秘书长,由古大存接任一部主任,副主任为刘芝明。党校二部驻在王家坪,学员多为中共上级(中上层)干部和军队团级干部,部主任为张鼎丞,副主任为孙志远、安子文。党校三部驻在兰家坪,1943~年~5~月~4~日,中共中央决定将中央研究院并入党校成立第三部,学员基本上是抗战初期来延安的知识分子干部,部主任为郭述申,副主任为张如心和阎达开。党校四部由原军政学院合并而来,部主任为张启龙、张邦英,副主任为程世才、杨尚奎。五部、六部的前身是西北局党校,1944~年初被并入中央党校,学员多是边区县、区级干部和从国统区来的知识分子。五部主任为白栋材,副主任为强晓初、晁哲甫。六部主任为马国瑞,副主任为谷云亭。

中央党校一部集中了一大批二十年代或红军时期入党的中共高级干部,计有朱瑞(中共山东分局书记)、薄一波、孔原、罗瑞卿、邵式平、丁玲、阎红彦、陈奇涵、陈郁、陈赓、宋时轮、王树声、刘景范、李培芝(王若飞夫人)、陈锡联、马文瑞、韩先楚、舒同、陈再道、乌兰夫和蔡树藩等。

对于集中在党校一部的老干部,党校领导采取了区别对待的不同政策,老干部中凡出身井冈山,或参加过长征的,一般不属于“抢救对象”,但仍需在党校接受审查和提高“路线斗争觉悟”。

李伯钊是红军文艺宣传工作的开创者之一,她与其夫杨尚昆都是二十年代后期被派往苏联学习的留苏生,李伯钊~1931~年进入中央苏区后,长期领导红军中的宣传鼓动工作。在长征途中,李伯钊曾被分配至张国焘的红四方面军任宣传干事,由于不了解上层的争论与分歧,在毛、张草地分家后,写过《南下歌》和《谁的罪过》等歌曲。1942~年毛泽东为召开文艺座谈会,曾向李伯钊了解根据地和八路军的文艺情况,李就此向毛作了“认真的自我批评”,“解释和说明了一些情况”。在党校一部的审干中,李伯钊在所在支部进行了“认真的检查”,并向党组织递交了“几万字”的自传材料,对自己的思想和历史进行全面的反省\footnote{《李伯钊文集》(北京:中国华侨出版公司,1989~年),页~120~—~21。}。

朱瑞也是原留苏生,瑞金时代曾任红五军团政委,是中共六届五中全会上当选的中央候补委员,1944~年~2~月,从山东分局书记的任上调入党校一部学习。在审干中,朱瑞写了详细的自传和(整风学习思想小结),他反复检讨自己“为什么会被教条主义所提拔”,朱瑞认为,这其中“除了(自己)有莫斯科留学生的标号,有一定的工作能力,做了许多工作之外”,更在于“我的思想方法是教条的,颇合他们的口味”。在党校期间,朱瑞还给刘少奇写了一封信,称“这次反省十倍百倍甚于过去任何一次,痛痛地打动了我思想方法上主观主义这个悠久肥大的根株”\footnote{郑建英:《朱瑞传》(北京:中央文献出版社,1994~年),页~296、298。}。

李伯钊、朱瑞皆有留苏背景,在以反教条主义为中心的整风运动中,受到冲击自不待言,然而他们毕竟长期在红军中工作,所受到的审查相对说来,仍是属于“和风细雨”式的。相比之下,那些在国统区工作的老干部就没有这样幸运了,在党校的审干、抢救中,来自国统区的大批老干部被打成“特嫌”,邹凤平案是其中最引人注目的事例。

邹凤平原任四川省工委书记,是大革命时期的老党员,曾因从事地下斗争被捕,身体受到严重摧残,由于他的脊椎骨已断裂,身体一直不能直立。1938~年邹凤平曾在成都约见过来自昆明欲投奔延安的陈野萍,并分配陈野萍去宜宾作地下工作(陈野萍在六十年代及八十年代担任过中共中央组织部副部长)。1940~年,邹凤平因四川省委负责人罗世文被捕,川西地区大批疏散干部,与张曙时奉调回延安。邹凤平抵延安后入中央党校一部学习,在“抢救”前夕的审干中,就已被认定为“特务”受到批判。邹凤平新婚不久的妻子在压力下,也诬指邹凤平是“特务”,随后与他人同居,邹凤平陷入绝境,愤而自杀\footnote{邹凤平当时的妻子甘棠,原名阚思颖,1928~年参加中央特科工作,其兄阚俊民,后改名刘鼎,中共党内著名军工专家。1949~年后甘棠任重庆市妇委书记,四川省高级法院副院长,党组副书记。甘棠~1935~年长征至遵义时被疏散在地方,并与其他被疏散的红军组成游击队,1936~年一度被俘,延安整风期间受到严重冲击,晚年生活凄凉,1971~年因疾病不得医治而逝世。参见郭晨:《巾帼列传——红一方面军三十位长征女红军生平事迹》(北京:农村读物出版社,1986~年),页~157。}。

曾任中共四川省妇女部部长的曾淡如,在“抢救”中,因被诬为“特务”、“叛徒”而备受折磨,最终也绝望自杀。

中央党校一部的审干、抢救、“逼、供、信,恶性循环”,\footnote{张平化:《满怀热情地参加整风学习》见《延安中央党校的整风学习》,第~1~集(北京:中共中央党校出版社,1988~年),页~47。}一百多老干部被指控有政治历史方面的嫌疑,经彭真批准,将这批人分配至二部,编成两个支部,继续审查。丁玲则作为有问题暂时弄不清的干部,被“挂起来”,于~1944~年夏调往边区文协。中共老党员、南方局组织部负责人孔原(陈铁铮),中共西南工委负责人、鄂中鄂西区党委书记钱瑛被诬为“叛徒”、“特务”、“红旗党”,遭到大会、小会轮番批判。\footnote{孔原:《一位杰出的革命女性——忆战友钱瑛同志》,载《忆钱瑛》(北京:解放军出版社,1986~年),页~21。}

原河南省委负责人王志杰、郭晓棠等此时已被公开诬指为“特务”,他们也被集中在中央党校一部,康生亲自坐阵中央党校的批斗大会,逼迫王、郭承认“河南党是红旗党,是特务,是叛徒”。大会一开始,就把郭晓棠等揪到台上,郭拒不承认,继之,又将王志杰揪上台,限令王在五分钟内坦白交待。王也据理力争,声辩“河南党是执行党中央路线的”,大会主席当即制止王的发言,宣布开除王、郭的党籍,把两人捆起来,押往中央党校柳树湾禁闭院。(王志杰在抗战胜利后恢复党籍,调往太行根据地工作,郭晓棠的党籍在~1950~年才恢复,1966~年文革初期郭晓棠被河南省委抛出来,以“叛徒”、“特务”等罪名第一个在《河南日报》被点名批判。)前河南省委书记张维桢则被施之以车轮战、疲劳战,并被拖出去“假枪毙”,最终也被打成“特务”\footnote{张文杰:《河南党组织被康生诬陷为“红旗党”的历史真相》,载《河南党史通讯》第~1~期,引自廖盖隆主编:《中共党史文摘年刊》(1985)(北京:中共党史资料出版社,1987~年),页料~344~—~45。}。

在中央党校一部受审的河南省委干部中,叶剑英的前妻危拱之尤其引人注目。曾参加广州暴动、留学苏联、又参加过长征的危拱之是中共党内为数不多的女知识分子干部,早在瑞金时期就因所谓“托派嫌疑”遭到开除党籍的打击。抗战后,危拱之被派往河南工作,后被中共河南省委推举为参加中共七大的代表,于~1940~年~4~月来到延安。1943~年~3~月,上级组织以参加整风学习和帮助党审查河南干部为由将危拱之等调入中央党校,事实上是被集中在中央党校一部受审。不久危拱之就被扣上“特务”的帽子,受到隔离审查。危拱之因不堪迫害,悲愤难禁,手持剪刀朝自己喉咙猛扎一刀,血涌脖颈,但幸未死成\footnote{郭晨:《巾帼列传——红一方面军三十位长征女红军生平事迹》,页~148。}。

薄一波当年曾被编入党校一部担任第一支部干事,他在晚年回忆抢救运动时记述了他所亲眼目睹的“抢救”惨状。薄一波写道:

\begin{quoting}
有一件我难忘的往事,其情其景多年来不时的涌上心头,……那时我母亲也与我一起到了延安,我把她安置在深沟的一个窑洞居住。有一天,我去看她时,她说,“这里不好住,每天晚上鬼哭狼嚎,不知道怎么回事”。我于是向深沟里走去,一查看至少有六、七个窑洞关着约上百人,有许多人神经失常。问他们为什么?有的大笑,有的哭泣,……最后看管人才无可奈何地告我:他们都是“抢救”的知识分子,是来延安学习而遭到“抢救”的!\footnote{参见薄一波:《七十年的回忆与思考》,上卷,页~362。}
\end{quoting}

薄一波的回忆虽未具体指明这关押上百人的深沟属于哪个机关管理,但他明确记述了在中央党校也有关押干部的审洞。薄一波发现,“在中央党校西南角的窑洞里,也关押着‘抢救运动’中‘抢救’出来的一百五十名干部”,其中有武竞天、宋维铮等。

当时在一部接受审查的还有著名历史学家吕振羽。1942~年底,吕振羽夫妇随刘少奇从华中根据地来到延安。到延安后,吕振羽已不再与闻核心机要,而专作研究工作了。吕入中央党校一部后,很快就被卷入一宗“托派”案件中。在审干和“抢救”运动中,原与吕相识的王姓夫妇,被诱导要他们交待吕振羽的“托派”问题。在多次逼供下,王的妻子被迫供出吕振羽是“托派”。有关方面再以此追逼王本人,王某断然否认,并陈述其爱人从未见过吕振羽。然而在高压下,最后王也被迫指供吕为“托派”。

不久,王某即翻供。尽管王某已经翻供,但吕振羽仍受到审查,吕详细写出自己的自传,断然拒绝了对他“托派”的指控。吕振羽受此事牵累了近一年的时间,使其创作完成《简明中国通史》的计划最终未能实现\footnote{刘茂林、叶桂生:《吕振羽评传》(北京:社会科学文献出版社,1990~年),页~125。}。

党校一部主任古大存是一位广东籍的老资格共产党员,红军长征后,一直坚持在粤北山区开展游击战争,抗战爆发后来到延安,此时正受到毛泽东、刘少奇、彭真的信任,被委之以负责审查高级干部的重任。古大存在党校一部大刀阔斧、雷厉风行,将一大批老干部整成“特务”、“叛徒”和“特嫌”,引起许多老干部的不满。陶铸的妻子曾志此时也是一部的学员,古大存认为曾志的历史有疑点,但一时又无法查清,于是迟迟不肯为曾志作一个相信本人交待的结论,而是将其“挂起来”。曾志和陶铸对此都十分不满,陶铸曾当面批评古大存,为何没有证据,仅凭主观主义就将曾志的结论拖了两年。1954~年,古大存在中共七届四中全会期间发言,检讨自己在中央党校一部领导整风时曾伤害了一些干部。\footnote{杨立:《带刺的红玫瑰——古大存沉冤录》(广州:中共广东省委党史研究室,1997~年),页~51、31。}但是,延安审干运动还是在陶铸与古大存的关系上投下了阴影,五十至六十年代,陶铸主政广东,对担任广东省副省长的古大存多有压制和打击,最终将古定为“反党的地方主义分子”\footnote{杨立:《带刺的红玫瑰——古大存沉冤录》(广州:中共广东省委党史研究室,1997~年),页~51、31。}。

据安子文称,党校二部吸取了一部“抢救”的经验教训,“便没有搞抢救运动”。\footnote{陈野苹、韩劲草主编:《安子文传略》(太原:山西人民出版社,1985~年),页~46。}事实上,党校二部虽未搞大规模的“抢救”,但严格的审干继续进行,只是斗争的热度有所降温。

中央党校三部集中了当时延安几乎所有较有名气的知识分子,除了中央研究院知识分子干部外,住在延安文抗——中华全国文艺界抗敌协会延安分会的作家、文艺家(大多来自大后方和华北敌后根据地),在审干全面展开后,也被分别送往中央党校,编人第三部(1943~年春,延安文抗作为曾活跃于边区的一个文艺团体已无疾而终)。1943~至~1945~年,在党校三部受审的党员知识分子有范文澜、陈学昭(1945~年入党)、于黑丁、马加、吴伯萧、周而复、白朗、罗烽、方纪、冯兰瑞、曾克、刘白羽、欧阳山、草明、叶蠖生、陈波儿、金紫光、陈明、刘雪苇等。

三部的党员知识分子干部组成七个党支部,按照校部的部署,进行交待历史、检查思想的紧张斗争。

白朗——来自东北的著名流亡作家,来延安后,与其夫罗烽在延安“文抗”工作,后白朗被调入《解放日报》社副刊部。罗烽因《还是杂文时代》一文获罪于毛泽东和周扬,虽未被公开批判,但已被打入另册。白朗在抢救和审干中,受到极大的精神压力,先是在《解放日报》挨整,继而在党校三部遭斗争,“在长达一年半的时间,白朗糊里糊涂,痴呆麻木,整日不讲一句话”\footnote{王良:《罗烽、白朗蒙冤散记》,载《新文学史料》,1995~年第~2~期,页~176。}。

方纪是以写颂扬毛泽东赴重庆谈判的散文《挥手之间》而闻名的作家,他在“抢救”中受到冲击。吴伯箫则被定为“反共分子”,吴曾在黄埔军校学习,国统区传说吴伯箫被整死,在西安为其开了追悼会。1944~年~7~月~3~日,吴伯箫在《解放日报》发表《斥无耻的追悼会》一文,自述其在延安“愉快地生活和创作,从来没有挨过整”云云\footnote{韦君宜:《思痛录》,页~19;另见《解放日报》,1944~年~7~月~3~日。}。

1943~年~7~—~8~月,原被安置在西北局“创作之家”的几位著名非党作家塞克、艾青以及从华北敌后根据地前来的杨朔、周而复等被通知到中央党校三部报到,随后塞克妻子韦安,艾青妻子冯莎也被通知进入党校三部。1944~年~3~月,结束自愿流放,从延安县川口区乡下返回的萧军、王德芬夫妇也被送到三部,原先明确规定非党员不能入中央党校的原则,此时也打破了。塞克夫妇、萧军夫妇等被编在各个支部,除不参加党组织生活外,一样接受审查,参加转变思想的学习。

塞克才华横溢,是三十年代杰出的话剧演员、剧作家、诗人,曾以主演日本剧作家菊池宽《父归》一举成名,誉满沪上。抗战前夕,创作戏剧《流民三千万》,写下脍炙人口的《救国军歌》,奔赴延安后,又写下《二月里来》,本应是有功之臣。然而塞克性格刚烈,嫉恶如仇,从不阿谀奉承,加之在延安经常执手杖下山上街。被某些领导人视为是未曾改造的狂徒一类,终在延安不被重用。把塞克安置在党校三部接受审查,纯属题中应有之意。

高长虹的情况则是一个例外。高在~1926~年曾与鲁迅打过笔仗,又是二十年代“狂飙社”的主将,1941~年~11~月,经八路军驻第二战区办事处负责人王世英介绍,高长虹徒步走到延安,一度受到欢迎和尊重。但到了~1942~年后,高长虹消声匿迹了。在“抢救运动”中,高长虹因“直接向中央提意见,甚至向斯大林提意见”,被康生指为青年党,“曾要整他”,只是由于得到博古和张闻天的保护才幸免于难。高长虹未能进入党校三部,他可能是当时延安唯一的编外人员\footnote{1945~年~8~月,毛泽东曾约见高长虹,征求他对今后工作去向的意见,高长虹说他想去美国考察经济,毛闻之勃然大怒,立刻把他轰出去,谈话“闹了个不欢而散”。参见言行:《高长虹晚年的“萎缩”》,载《新文学史料》,1996~年第~4~期。}。

党校三部的审干、“抢救”及以后的甄别前后持续一年多,即使处于被审查状态,三部的知识分子仍不忘为配合当前政治任务而努力工作。1944~年,三部学员陈波儿与四部学员姚仲明合作,创作了一出抨击“王明右倾投降主义”的话剧《同志,你走错了路》,在延安广为演出,受到领导的重视和好评。1944~年,除三部之外的党校其它各部,均奉中央命令转入“两条路线”学习,上级规定,凡属于“有问题暂时未弄清的人”,不参加学习“党的路线”。丁玲虽在党校一部,但她不能和其他学员一同参加“路线学习”。\footnote{参见陈明:《丁玲在延安——她不是主张暴露黑暗派的代表人物》,载《新文学史料》,1993~年第~2~期。}党校三部——属于有严重问题的单位,则被整体打入另册。那些“历史复杂”、“思想复杂”、“狂妄自大”的知识分子只配永远“脱裤子、割尾巴”,交待历史,反省错误。在延安上层某些人眼中,三部的知识分子是一群“异类”,是不配学习“两条路线斗争历史”的。

1944~年夏,中外记者团在延安访问、参观期间,仰慕中央党校的名声,再三提出参观要求,都被婉拒,于是记者先生感叹:“夫子之墙数仞,不得其门而人”\footnote{金东平:《延安见闻录》(重庆:民族书店,1945~年),页~120。}。

由彭真直接领导的中央党校,在“抢救运动”中一马当先,造成大量冤假错案。但是,这一段历史一直被精心掩饰,一直到八十至九十年代还有人对之文过饰非。1986~年~7~月~26~日,曾任中央党校秘书长的黄火青、郭述申等发表《回顾延安中央党校的整风运动》一文,对当年在党校发生的“抢救”惨剧竟无一字描述。\footnote{黄火青:《一个平凡共产党员的经历》,页~260~—~76、163~—~64。}1995~年,黄火青出版《一个平凡共产党员的经历》的回忆录,虽然提到党校受到“反特扩大化”的“很大影响”,“伤了不少同志感情”,但马上强调党校的错误“及时得到纠正”,对党校“抢救”的具体情况一字不提。\footnote{黄火青:《一个平凡共产党员的经历》,页~260~—~76、163~—~64。}和黄火青形成鲜明对照的是薄一波,1996~年薄一波公正地指出,“中央党校是‘抢救运动’的重灾区之一”,成为批评当年中央党校“抢救”的中共唯一领导人。

\section{延安自然科学院的“抢救”}

延安自然科学院是延安唯一一所工业技术学校,成立于~1939~年~5~月,原名为自然科学研究院,初期为研究单位,1940~年~1~月改为工业技术专门学校性质,1940~年~9~月成立大学部,设生物、物理、化学、地矿四系,后改为机械、化工和农业三个系,另附设一个由十二、三岁中学生组成的补习班,共有六百余名师生。1940~年~12~月,中共中央任命延安著名教育家徐特立为延安自然科学院院长。整风转入审干后,徐特立被调回中宣部,负责干部教育教材的撰写工作,中共西北局另派原中组部干部陈伯村\footnote{陈伯村在~1954~年被宣布为“高饶反党联盟”成员,文革后得到解脱。}坐镇自然科学院,该院审干、反奸、“抢救”运动由西北局直接领导。

延安自然科学院是审干、抢救的重灾区之一,该院许多师生是从国统区前来延安的青年党员和知识分子。运动即起,马上就受到严重冲击。本科生和教师中的大多数人被打成“特务”,当年在自然科学院任教的武衡(1976~年后曾任中国科学院副院长)就是其中之一。

自然科学院“抢救”运动的进展引起徐特立的强烈不安。徐特立为人宽厚慈祥,在延安有“革命的好外婆”之称,虽然一些中共老干部都知道他曾在长沙第一师范做过毛泽东的老师,但是徐特立从不声张,有别人问时,徐特立也不承认。但若站在毛泽东“路线斗争”的角度上来观察徐特立,他的“路线斗争”敏感性似乎并不高。整风运动前,徐特立曾去马列学院作报告,在号召干部学哲学时,徐特立说:在我们党内,真正把马克思主义哲学学通了的只有两人,一个是润之(即毛泽东),一个是洛甫\footnote{江围:《难忘的岁月》,载《延安马列学院回忆录》,页~103。}。称赞毛泽东固然无可非议,但将张闻天与毛泽东并列,则犯了大忌。显然,由徐特立这样心慈的老人来领导审干肯定不合适,故有将徐特立调走之举。运动高潮阶段,保卫机关不断到院里抓人,此时,徐特立已被调回中宣部,虽未正式免去他的自然科学院院长一职,但已不许他过问自然科学院的运动,然而他仍几乎天天步行几十里去自然科学院。有一天徐特立在前往自然科学院的路上,正巧遇上保卫机关一帮人将一个怀孕的年轻女同志捆绑走了。徐特立将自己的上衣脱了下来,披在那个女同志身上,什么话也没说。事后,徐特立竟被指责为“同情反革命”\footnote{徐干:《历久长新的回忆——永远铭记父亲徐老的言传身教》,载《徐特立在延安》(西安:陕西教育出版社,1991~年),页~118。}。

身为自然科学院院长,在延安享有崇高威望,当时已高龄六十七岁的徐特立,在审干、抢救的洪水袭来之际,虽尽多方努力,仍无法保护该院的师生。运动依旧按照它的内在逻辑疯狂运转,入夏后,“抢救”深入到补习班(中学部),居然在这群少年中“抢救”出三十几个“特务”。到了~1943~年秋,自然科学院并入延安大学,相应的自然科学院中学部也并入延安大学中学部。至~1943~年底,延安大学中学部又挖出三十几个“特务”,前后共有七十余人落网。延安大学中学部的有学生二百余人,分为五个班,大多为中共烈士子弟和干部子弟,以及少量从部队调入的“小八路”,竟三分之一成了“特务”\footnote{蒋祖林:《胭脂河畔》,《新文学史料》~1993~年第~4~期,页~78~—~79。}。

在延安审干和“抢救”运动中,曾有一位被树为“特务”典型的受害者,此人即彭而宁(钱家骥),也是出自自然科学院。彭而宁被扣上“兼差特务”、“双料特务”的帽子(即“日特”兼“国特”),早在~1942~年就和边区银行行长黄亚光等人,被康生机关秘密逮捕,1943~年整风转入审干、抢救后,又被康生作为罪证确凿的“特务”抛出来,作为证明“特务如麻”的依据。\footnote{彭而宁在离休前任西北林学院院长、党委副书记,他在~1987~年撰文纪念徐特立,回忆延安自然科学院,仅泛泛谈论“延安精神”,和自然科学院的科研、教学活动,只字未提当年自己的遭遇,可能是不忍再触及心灵上弥久的伤痛。}彭而宁入狱,在很大程度上是康生及其机关的独立作业,而非自然科学院审干小组所能左右。

彭而宁出身世家,其父钱来苏为同盟会会员,曾任第二战区少将参议。1939~年,彭而宁毕业于北平中国大学国学系,1940~年辗转进入延安,被分配至自然科学院就读。整风之初,彭而宁作为非党人士,曾名列该院整顿三风委员会的常委,\footnote{《延安自然科学院史料》编辑委员会:《延安自然科学院史料》(北京:中共党史资料出版社、北京工业学院出版社,1986~年),页~119。}曾在壁报报头画了一幅向日葵,康生就此诬指此画意指“心向日本帝国主义”,将彭而宁秘密逮捕,最终将其定为“日特兼国特”。

彭而宁遭此劫的根本原因是康生对其来历极为怀疑。彭而宁来自日伪统治下的北平,家世背景复杂,本人也无中共北方局系统的介绍,站在保卫部门“怀疑一切”的立场,彭而宁肯定是个“问题人物”。康生其人本来就极左,对来历清楚的人,尚且能从鸡蛋里挑出骨头,对彭而宁这样“疑点”较多的人就更不在话下了。彭而宁~1940~年从华北来延安,对彭而言也是一个不利因素。抗战爆发后的~1937~至~1938~年,从华北奔赴延安的青年不在少数,对这个时期来延的人员,虽然也须经审查,但是多为组织部门经办,保卫部门插手不多。一般多能过关。但是~1939~年后,从华北来的青年愈来愈受到怀疑。其中不少人被控以“托派”罪名,遭到关押和处置,因此彭而宁落难,实属在劫难逃。

彭而宁被捕后,紧接着又牵连到其父钱来苏、妹钱家楣及其丈夫孙静远。1943~年~3~月,正当延安整风转入审干、反奸之际,还不知其子已被捕的钱来苏,因不满蒋介石,经八路军驻二战区办事处主任王世英的介绍,带着女儿、女婿,秘密投奔延安,殊不知此时的延安正弥漫着“提高警惕性”的浓厚气氛。钱来苏抵延后,受到客气的欢迎,被安置在延安交际处居住,其女儿、女婿则被送往延安大学接受审查。

钱来苏素来崇仰毛泽东、朱德,一直希望能见毛、朱一面,然而,钱氏的这个愿望并未能实现。此时已非~1937~—~1938~年,当年毛泽东为了政治上的需要,广结人缘,而现在毛对钱来苏这样一位已失去影响力的失意人物,再也产生不了兴趣,再加上其子已被定为“兼差特务”,见钱来苏一事,简直无从说起。

钱来苏千里兼程投奔延安,尽管毛泽东近在咫尺,却难见一里。不久,坏消息接踵而至,其女儿、女婿在延安大学双双被打成“特务”遭到隔离审查。一心想报效中共的钱来苏此时真正陷入了绝境:其子、其女和女婿,全都变成了蓄意危害革命的“特务”,自己虽未被捕,但成天被安置在交际处,似乎离被抓、坐牢也不远了。恰王世英又奉命来找钱来苏谈话,劝其“交待问题”,钱来苏怒斥道:当初介绍我到延安来的是你,现在把我说成特务的又是你!你到底是什么意思?从此次谈话后,钱来苏宣布绝食,以示抗议。

从表面上看,钱来苏被勒令交待问题是受其子彭而宁、其女钱家楣、女婿孙静远“特务案”的牵连,但更深一层的缘由乃是延安上层对钱来苏存有严重的怀疑。

据当年延安交际处处长金城透露,“负责整风审干运动的一位中央领导同志,怀疑钱来苏老先生同日本帝国主义有关系”,此人究竟是谁呢?肯定不是康生,如系康生所为,金城一定会直接点出其名。据笔者分析,此“中央领导同志”可能是彭真或任弼时。在整风审干期间,彭真是仅次于康生的活跃人物,彭真长期在华北地区工作,对当地的“敌情”、“政情”比较熟悉。以某种审干逻辑看,钱来苏身上的疑点确实较多:钱来苏在张学良时代曾长期在东北军界活动,与日本很难摆脱干系;“九一八”事变后钱避居北平,属失势人物一类;北平在“九一八”后鱼龙混杂,北洋失意政客、军界失意人物多与日本有关联;抗战爆发后,钱来苏又到了山西阎锡山第二战区长官部。钱来苏过去与中共毫无联系,现在投奔延安,目的何在,殊堪生疑。在延安负责审干的“领导同志”中,最了解华北地区的情况、最具有发言权的,除了彭真还有谁呢?

笔者提出任弼时只是一种大致的推测,因为任弼时作为毛泽东委派的中共中央指导西北局工作的负责人,对陕甘宁边区系统和西北局机关的审干、抢救运动,有很大的影响力,在这个时期,他经常过问具体案件。但笔者认为,对钱来苏一案负有直接责任的人,更大的可能是彭真。

\section{鲁艺(延安大学)的“抢救”}

延安鲁迅艺术文学院(简称鲁艺)创立于~1938~年~4~月,初设戏剧、音乐、美术三系,后扩大为文学、戏剧、美术、音乐四部,院长先后由毛泽东和中共元老吴玉章挂名,实际工作由副院长周扬负责。周扬既担任了边区政府教育厅厅长,又是鲁艺事实上的总管,这使他成为延安文艺、教育、宣传系统中叱咤风云的人物。

整风之初,中央文委系统组成学分会,由周扬负责,他在鲁艺依照中央总学委的部署,在~1942~年,领导全院教职学员开展了反王实味的斗争。到了整风转入审干阶段后,周扬又积极排队摸底,清查可疑分子,在~1943~年~4~月~1~日大逮捕前夜,已经揪出重大特务嫌疑人员二十九人,并将其全部移送边区保安处侦讯\footnote{陈永发:《延安的阴影》,页~194。}。

1943~年~3~月~16~日,中共西北局常委会议决定,将鲁艺、新文字干部学校、民族学院、自然科学院合并为延安大学,校址设于桥儿沟鲁艺内,合并后的延大校长,由吴玉章担任,周扬任主持工作的副校长。这次会议同时决定,“将政治上没有问题与不合条件的,调出分配工作;适合于继续学习的,留下学习;政治上有问题的,留下整风”\footnote{《中共中央西北局常委会议关于延大、自然科学院等精简问题的决议》(会议记录),载《延安自然科学院史料》,页~28~—~29。}。

延大人数达到一千六百多人,为了避免出现混乱情况,影响整风、审干的进行,西北局常委会议宣布,继续保留原各学校的单位形式,自然科学院的整风学习,由西北局直接领导;鲁艺等单位由周扬主持,受中宣部领导。

虽然~3~月~16~日西北局常委会议提到,“适合于继续学习的,留下学习”,但是在当时的形势下,这已完全不可能。在~1943~年~4~月~1~日大逮捕后,坦白运动已全面展开。5~月~18~日,周扬在合并后的延大主持召开第一次全校规模的坦白动员大会,部署在延大四个单位分别进行坦白大会的预备工作。几天后,四个单位都发现了特务嫌疑分子,其中由周扬直接领导的鲁艺成果最大,一举揪出各类问题人物十二人\footnote{陈永发:《延安的阴影》,页~196、201、204~—~205。}。5~月~21~日,鲁艺坦白大会如期召开,会计科科长蔡光华等八人登台坦白,大会气氛紧张,共开了十三个小时,直至深夜十一时左右才结束。周扬要求各单位以这些已坦白的人员为榜样,开展三天突击坦白活动,来一个坦白检举的大比赛\footnote{陈永发:《延安的阴影》,页~196、201、204~—~205。}。

在三天突击坦白活动中,鲁艺师生成立了许多规劝小组,按照组织的安排,纷纷找嫌疑分子谈心,“早上劝,下午劝,夜深人静还劝;今天劝,明天劝,后天仍劝,有的更深夜不眠劝”。在密集性的劝说攻势下,鲁艺有二十三人在三天突击坦白活动中交待问题,加上以前已坦白的二十五人,仅鲁艺就有五十二人承认自己是“特务”\footnote{陈永发:《延安的阴影》,页~196、201、204~—~205。}。

应该指出,在~1943~年~7~月“抢救”正式开场前,鲁艺的“坦白”活动总的说来是相对温和的,还没有使用捆绑吊打等暴力手段,但是在~7~月以后,情况发生重大变化。一方面,“如火如荼的群众反奸斗争以开大会,开小会,个别规劝,小组规劝等群众自己创造的各种形式进行着”;\footnote{《延安自然科学院史料》编辑委员会:《延安自然科学院史料》(北京:中共党史资料出版社、北京工业学院出版社,1986~年),页~179。}另一方面,暴力因素迅速渗入,鲁艺等单位不时传来有人因不堪斗争而自杀的消息。

当时在延安有几位在国内享有盛名的非党作家:塞克(陈凝秋)夫妇、艾青夫妇和高长虹等人,他们因来延安之前即闻名全国,因此在一段时间内受到优待。塞克等被安置在桥儿沟西山由中共西北局领导的“创作之家”,享有单间窑洞的待遇,获准可以进行创作。刚从华北根据地来延安的周而复、杨朔(1943~年还是非党作家)也被允许住进“创作之家”,“以文艺形式反映八路军和广大人民的战斗生活”。但是,“仿佛是世外桃源”的桥儿沟西山的“创作之家”,很快也受到“抢救运动”的冲击。

在“抢救”的暴风骤雨中,在塞克、周而复、高长虹等居住下方的鲁艺的“政治空气越来越紧张了”。该院有一位艺术家因不堪迫害,全家自焚。\footnote{韦君宜:《思痛录》;页~15。}鲁艺教员、木刻家彦涵亲眼看到鲁艺一位被审的青年,冲出关押室,纵身跳下悬崖,嘴里大声哭喊“冤枉”,凄厉的声音回荡在沟底。\footnote{孙志远:《感谢苦难:彦涵传》(北京:人民文学出版社,1997~年),页~242。}塞克等居高临下,西山下面的情况一目了然,只见“一批批干部被抓走,一声声逼供信的声音传来”,在山下不断传来的“地动山摇的口号声”中,原准备在“创作之家”创作剧本的塞克终究无法写完他的剧本。

塞克无法在阶级斗争的疾风暴雨中创作出剧本,只能说明他的思想水平低,真正的无产阶级革命文艺战士只会欢呼“让革命的暴风雨来得更猛烈些”——在“抢救”高潮中,革命文艺战线的“领导同志”周扬一刻也没忘了自己肩负的使命,他虽然为挖“特务”忙得不可开交,但还是想到要用文艺为当前的政治服务。在周扬的直接领导下,鲁艺竟然在短时间里排出一场《抢救失足者》的话剧!这场话剧由周立波、陈荒煤、袁文殊等集体编写,水华导演,“剧情是一个特务领导另一个人,他们潜人延安进行了破坏活动”,为了让戏剧效果逼真,还特别安排让正受审查的骆文扮演特务,美其名曰“亲自演,体会会更深”。这场话剧以后因为政治风向发生变化,又遵周扬命停止了排演,但骆文仍在夜晚被鲁艺戏剧系审干小组唤去接受“车轮战”式的审讯\footnote{骆文:《延安时代,他总是在思考探索》,载王蒙、袁鹰主编:《忆周扬》(呼和浩特:内蒙古人民出版社,1998~年),页~66~—~67。}。

1944~年~5~月,延安大学的规模又有新的扩大,西北行政学院也被并入了延大。西北行政学院成立于~1940~年~7~月,原是为边区培养行政干部的学校。但是到~1942~年~10~月,该校性质发生变化,被中共西北局改成集中审查嫌疑分子的反省机关。1942~年~11~月,王子宜被派到该校任副院长,不久,就在该校原领导机构和教员中发现了一批嫌疑分子。1943~年~4~月~17~日,保卫机关正式将原院长王仲言逮捕,至~4~月底该校共查出“特务”三十二人\footnote{陈永发:《延安的阴影》,页~223~—~224。}。

显然,西北行政学院不仅仅是一个集中关押边区各单位嫌疑分子的临时拘押营,它自身也在经受一次严格的清洗。

到了~1944~年~5~月,在延安大学~1877~名教职学员中,共清查出嫌疑人员~440~人\footnote{《延安大学概况》(1944~年~6~月),载《延安自然科学院史料》,页~179}。

\section{晋察冀、晋绥、太行根据地的“抢救”}

晋察冀根据地又称华北根据地,是抗战后中共最早开辟的战略根据地。1941~年春,担任边区党委书记的彭真奉调回延安,聂荣臻成为党政军一元化的最高领导。1943~年~8~月,聂荣臻接通知返延安参加中共七大,由晋察冀分局副书记兼军区政委的程子华负最高领导责任。

1942~年全面整风发动之际,正是晋察冀根据地最为艰苦的时期。在侵华日军“铁壁合围”和“大扫荡”的严酷战争环境下,整风只能间歇进行,基本限于学习文件的“正面教育”,直至~1944~年战争形势缓和后,华北根据地的全面整风才正式开始。

华北根据地的整风、审干是在程子华的主持下开展的。其主要方法是,上层领导干部在职学习整风文件,开展批评和自我批评,而将一些所谓“有问题”的干部集中到党校学习和接受审查。1944~年夏,时任《晋察冀日报》社长兼主编的邓拓被调入分局党校学习,在经过短期的文件学习后,整风迅速转入审干阶段,由于气氛紧张,邓拓与其妻丁一岚虽同在党校学习,但两人却“很少有时间接触”。对邓拓的审查主要集中在历史上曾经两次被捕的问题。抗战前邓拓曾在河南从事抗日救亡工作,此时河南地下党是所谓“红旗党”的说法正从延安传到晋察冀,邓拓自然成了被怀疑的重点对象。邓拓无辜被审,心情愁闷沉重,在给丁一岚写的一首《战地歌四拍》的长诗中,邓拓自抒“七度春秋销北地”,还写下“天才投笔误狂歌,伤心梦幻倍蹉跎”的诗句,流露出惆怅的感怀\footnote{王必胜:《邓拓评传》(北京:群众出版社,1986~年),页~107、95。}。

在整风、审干中,清查“托派”是重点的内容。当时任冀南区党委常委、四分区书记兼军分区政委的王任重被怀疑成“托派”,军区参谋长王蕴瑞也受到怀疑。王任重是知识分子干部,1938~年从延安来冀南,先后担任冀南区党委宣传部副部长、部长等职。在王任重受审查期间,冀鲁豫中央分局书记(也称平原分局)黄敬提议调王任重任平原分局宣传部长,但未被批准。王被调回延安后,其“问题”经审查得到甄别后才返回冀南\footnote{《宋任穷回忆录》(北京:解放军出版社,1994~年),页~498。}。

当时被怀疑为“托派”的人大多为知识分子干部,这些人在未进入根据地前一般都曾参加过国统区的抗日救亡运动,或从事过兵运工作。内战期间,国统区的共产党组织遭到严重破坏,一些失去组织联系的共产党员纷纷参加了各种左翼团体,有些人仍以共产党员的面目活动,并发展其它左翼青年入党。这批人以后多被认为是“托派”,成为清洗的对象。在国民党军队中从事兵运工作的中共党员,在抗战初期或拉出队伍进入共产党根据地,或因工作失利退回根据地,这批人也往往被认为是“托派”。清洗这些人,主要是党组织对彼等政治上的不信任,“托派”只是一个罪名,绝大多数受害者实际上与“托派”或“托派”思想毫无关联。

晋察冀的“肃托”、“锄奸”在~1938~年就已秘密进行,熊大正、李晓初是最初的牺牲品。熊大正抗战前在清华大学物理系任助教,清华大学理学院院长叶企荪在熊大正毕业后曾推荐其去德国留学,因抗战爆发而未成行。1938~年夏熊大正到冀中根据地,专门负责秘密从天津购买军火、医药、通讯器材。1939~年春,时任冀中军区供给部部长的熊大正突被晋察冀军区锄奸部秘密逮捕,被定为“汉奸”、“国民党特务”而遭处决。受熊案牵连,从平津来冀中的知识分子干部近百人被逮捕(后大多被释放)。熊大正错案直至~1986~年才由中共河北省委予以平反\footnote{《吕正操回忆录》(北京:解放军出版社,1988~年),页~180、101、157。}。

原东北军军官吕正操在抗战前夕秘密加入共产党,其入党介绍人为北方局打入东北军中的李晓初。“七七”事变后吕正操率东北军六九一团据守华北,开辟冀中抗日根据地,功勋卓著,受到毛泽东的赏识,在中共七大上被选为中央候补委员。但是,吕正操的入党介绍人李晓初却命运悲惨。1938~年~4~月,此时已离开吕正操部前往地方工作的李晓初,因“托派”案件被调往平汉路西晋察冀军区接受“审查”旋被处决。李晓初被审和遭处决,事前并没有通知冀中军区领导吕正操;李晓初一案也是直至~1985~年才由中共河北省委复查,确认系错案被平反昭雪\footnote{《吕正操回忆录》(北京:解放军出版社,1988~年),页~180、101、157。}。

中共对于军队一向极为重视,抗战初期八路军开赴华北,以老红军为骨干,联合改造由共产党地方组织领导的军队,创建中共根据地,在政治态度上只信任经过长征的红军干部,对其他干部则在工作中进行考察排队,一俟立稳脚跟,即对地方武装的领导进行审查,大多调任副职。吕正操之所以长期蒙受信任,被委之为冀中军区司令,一方面在于吕开疆辟土,贡献巨大;同时也因吕对晋察冀领导十分尊重,加之聂荣臻为人公正宽厚,故吕正操可以在党内斗争中较少受到冲击。然而,吕正操在政治上的顺利并不意味着吕部的干部也是如此。在冀中局面打开后,吕正操所率的旧部很快成了审查的重点对象。1938~年~8~月中旬,晋察冀军区将吕部调安平整训,军区政治部着手处理所谓“托派”案件,将吕正操旧部大部分干部送往延安审查,这批干部只有经审查确认为没问题才放回冀中工作。至于有多少干部被扣留在延安,有多少干部返回冀中,吕正操在其回忆录中都没有详述。\footnote{《吕正操回忆录》(北京:解放军出版社,1988~年),页~180、101、157。}

晋绥与陕甘宁隔黄河相望,是延安与边区的屏障,也是延安通向华北、山东、华中的唯一信道。晋绥是贺龙、关向应所率八路军一二〇师辖地,贺龙为党政最高负责人。1942~年贺龙被调回延安任陕甘宁晋绥联防军司令,晋绥分局书记关向应也因病在延安治疗,由林枫任分局代书记、晋绥分局整风总学委会主任。

1942~年是晋绥极端困难的时期,根据地在日军包围下急速缩小,只剩下几个完整和不完整的县,人口不足百万。但由于与陕甘宁相邻,受延安影响太深,在战争环境下,晋绥根据地仍仿效延安,大搞审干与“抢救”。1943~年~3~月~19~日,晋西北党委发出《关于整风学习中配合审查干部的指示》,全面推广延安的经验,要求干部写思想自传。5~月,武新宇在晋西北行署整风座谈会上作报告,号召各领导干部诚恳坦白反省自己。在这前后,晋绥各地的军政干部开始集中整风学习。

9~月,晋绥的整风、审干转入“抢救”。驻军干部在神府彩林村,边区一级党政民机关干部在杨家沟召开“反特务”斗争大会,各单位出席总人数为~1274~人,到~11~月,就查出一大批所谓“特嫌分子”。\footnote{《晋绥革命根据地大事记》(太原:山西人民出版社,1989~年)页~221、231。}12~月,晋绥军区下辖各分区又分别成立了“整风队”,对干部实行集中审干。据中共晋绥分局书记林枫称,参加运动的“有五千个干部,三千个战士,两万五千群众,共三万三千人”。\footnote{《林枫同志在分局高干会关于整风审干的结论》,1944~年~7~月(党内文件——这一文件请与康生同志的报告配合研究,并应以那一文件为依据而又是联合实际的研究),转引自郭华伦:《中共史论》,第~4~册,页~414。}在“抢救”高潮中,各部门“经常通报抓特务的比例,有的单位抓出特务竟占全部人员的百分之二十、三十”\footnote{张稼夫(时任中共晋绥分局副书记):《庚申忆逝》(之二),载《党史资料》,第~8~辑(北京:中共党史资料出版社,1983~年),页~251~—~52。}。

文革前任《光明日报》总编辑的穆欣当时在晋绥《抗战日报》工作,他因担任由范长江任社长的国际新闻社特派员及驻晋西北通讯站主任,曾向国新社写过稿件而被诬指为“特务”。周恩来对穆欣的情况比较了解,因为国新社受周恩来直接颁导,而范长江聘请穆欣事先也曾征得周恩来同意。1943~年~9~月,周恩来从重庆返回延安后,料想穆欣在国新社兼职事将会受到怀疑,于是主动托晋绥军区第八军分区司令员韩均给中共晋绥分局和《抗战日报》社社长廖井丹捎话,为穆欣作出证明。但是周恩来当时在中央的地位并不巩固,晋绥分局对周的证明不予理睬,穆欣仍遭到批斗,他在报社担任的领导职务也被免去。\footnote{穆欣:《秦城监狱里的~6813~号》,《中华儿女》~1998~年第~10~期。}穆欣在抢救运动中的这段遭遇给他留下了极其深刻的印象。穆欣称,他“亲眼目睹了‘逼供信’给党的事业所造成的严重损失,而且也亲身体会到‘逼供信’的极大危害”,以至于以后几十年对此教训仍时刻铭记在心\footnote{穆欣:《劫后长忆》(香港:新民出版社,1997~年),页~132。}。

与其它根据地的情况相类似,晋绥地区“抢救”的重点单位是晋绥分局党校,但是有关晋绥分局党校“抢救”的基本状况,外界至今仍不其清楚。原晋绥干部张鉴在“抢救”中蒙受严重打击,信念崩溃,在~1945~年叛逃投奔国民党后,曾著文对晋绥分局党校的“抢救”情况加以披露。笔者对张鉴的叙述与文革后大陆公开披露的相关材料互相印证,认为张鉴的叙述基本属实,可作为了解晋绥根据地“抢救”的佐证资料\footnote{国统区当时虽对“抢救”运动不甚清楚,但坊间偶尔亦有少量有关延安的叙述,对“抢救”略有涉及,以今天的眼光观之,其记载大体符合事实。例如:1944~年夏访问延安的中外记者团成员之一的金东平在他的《延安见闻录》中就描述过所谓“车轮战术”。金东平将“车轮战”称之为“疲劳检查”,其方法是对“他们认为有‘有问题’的人加以轮流的审问,日以继夜,夜以继日的永不停止……”。参见金东平:《延安见闻录》,页~128。}。

张鉴的文章名为《中共怎样整风?》,在这篇文章中,张鉴谈到晋绥分局党校的整风、他个人在分局党校整风、“抢救”运动中的遭遇,以及他所知道的晋绥几个知名人物在运动中所受冲击的情况。现就张鉴的叙述作以下归纳。

一、晋绥分局党校被“抢救”干部的人数。

整风运动开展后,被送到晋绥分局党校学习和受审查的主要为军队营以上、地方县以上的干部,这批人绝大部分受过多年的基层锻炼。张鉴说,分局党校的整风从民国三十二年十月开始,“由整思想问题转入整政治问题”(此说属实,1943~年~9~月后,晋绥整风进入到“抢救”阶段——笔者注)。“当时党校共二、三百人,其中七十多人被迫承认为特务分子,有日特、国特、阎特,三十多人未承认或半承认,总计党校被抢救的干部,占总人数的百分之六十至七十,这个数宇并不惊人,当时,晋西北有个师范学校,这个学校百分之九十九被整为特务,所以有人给师范学校送了一个‘特务学校’的绰号”!

二、晋绥分局党校被“抢救”的干部基本上是受冤枉的。

“经过三四个月的被迫集中,因为硬要给人戴特务分子的帽子,所以多少人被迫用菜刀、草刀自杀了,有的跳崖,投河自尽了!”(张鉴未提供具体人数——笔者注)“当时有晋西北《抗战日报》的一个记者,因为被硬指为特务,万分不得已跳了六、七丈高的崖,不幸虽身受重伤而未身死,抬回来以后被认为,‘特务分子牺牲自己性命,破坏中共政治影响’,仍强迫坦白!”“被抢救的干部,后来事实证明,都是大冤其枉,……他们的所谓‘坦白口供’都是被迫无奈生编硬捏的,因此,当时有人说‘坦白运动就是胡说运动’”。

三、张鉴本人在“抢救”中的遭遇。

张鉴说,自整风结束到现在,他“始终不敢提起,甚至不敢回忆”他个人在整风期间的遭遇,“因为提起来我就伤心掉泪!我自十四岁离开家参加共产党工作,我自信很忠实,很热情,很积极,当整风运动开始时,我担任共产党小组长,一开始我就首先被监视起来,接着党校负责人向我提出我有政治问题,要我坦白,用‘坦白是光荣’,‘坦白后有前途’等软方法说服我;用不坦白要‘杀头’、‘坐监’、‘判徒刑’等硬办法威胁我。我恐怕造成他们的抢救借口,硬压抑着自己的感情,勉强若无其事地活下去。这样被监视两个月以后,到十二月我的厄运来了,在一个群众大会上,他们要我马上承认我是特务……他们的口实是我的哥哥在二战区,从小就是国民党,我和我哥哥好。一定是特务。这种毫无根据的理由,我当然不能承认,结果他们廿多个人用车轮战术轮流与我谈话,一个谈完去休息另外换一个人谈,我一个人四天三夜没有休息一分钟,两天两夜没有吃饭!尤其惨痛的是:八号晚上把我作为他们的阶级敌人,有的用皮带,有的用木棒,有的用拳头对我又打又骂,并且在这十二月最寒冷的天气里,把我衣服脱去只剩下一个单短裤在院中冻了四十分钟,因为我拒绝出去,把我头发拉去一大片!这样直到九号,因为我有脑充血病,精神恍惚,身体实在支持不住了,只得在指供、引供下承认我是特务分子。我本希望可以了事了,谁知道造成了永远的祸害,说我承认特务不算,还要特务关系,我本来不是特务,不了解任何关系,尤其在脑子里已经几乎不能思虑下的情形下,越编越不像。结果,从我是特务,追到我哥哥是特务,我嫂嫂是特务,我母亲也成了半个特务……我不知道什么是特务,不了解任何关系,更不能无端再陷害人,迫不得已,最后只有将已作的口供一脚踢翻!……在抢救运动中,其它刑罚、手段还多得很!有一种刑罚是:对被抢救的人不给吃饭,饿急时给吃肉,放的食盐、辣子特别多,吃上以后渴得要命,但是,你不承认自己是特务及特务关系,永远不给你喝水!这种刑罚说来并不惊人,实际上却是一种最惨酷的肉刑!另外在抢救手段上,布置有炮手、打手,炮手多半是知识分子出身,抗战前的党员,打手都是工农分子。还有说客、突击队。方式上更有‘一打一拉,一拉一打,打中又拉,拉中又打’等等。……抗战期间我经过好多次战斗,与敌人也多接触过,酸甜苦辣也多尝过,但是我始终没有伤过心,这一次被抢救却使我伤心伤透了!……我对共党忠实、积极,对任何指示没有违背过,对任何工作没有贻误过,怛是,结果只换了一个代价。就是我被整为特务分子!”

四、张文昂、雷任民等在“抢救”中的遭遇。

张文昂,原是阎锡山下辖山西六专署专员,1939~年晋西北事变后到达根据地,曾任晋绥边区高等法院院长、山西新军总指挥部副总指挥,在~1943~年~10~月整风转入“抢救”后被抢救为“特务分子”、“自首分子”、“消极分子”。在“抢救”中,他的脸被吐过痰,“差不多一切难堪他都受过了”,至“抢救”结束时,仍被监视。

雷任民,建国后曾任外贸部副部长,被抢救为“国特”。其妻曾留学日本,被整为“日特”,接着又由此追及雷任民。

张隽轩、梁膺庸都是晋西北事变后进入根据地、并担任领导职务的干部,在“抢救”中也被整为“国特”、“日特”、“阎特”。

五、由“被抢救”而对中共知识分子政策提出批评。

抗战之初,他自述当时他与共产党接触,“觉得共产党很好,就毫不犹疑地参加了共产党”,直到整风以后,由于身心受严重打击,产生思想幻灭,他结合自己在党内八年的体验得出看法,认为在党内只有“不识几个字的工农青年”受重用,而“知识分子出身的青年,特别是中学、大学毕业的学生,在共党内没有前途,只有到处受轻视,受打击”。张鉴说,“我们也认为工农是全国人的大部分,应该解放,不过不能把知识分子踢出去。但是共产党内看到的只是工农分子的气势凌人,不可一世,我当时很后悔,我母亲不该守多少年寡送我念书!\footnote{在“抢救”运动中,蒋南翔发现抗战后来延安的知识分子大多被“抢救”或被怀疑,工农干部普遍看不起知识分子干部,因而产生看法,认为在党内“只有一些没有接触过其它别的思想学说的工农同志,才是保险和可靠的同志”。蒋南翔的看法事实上与张鉴的看法不谋而合。二十多年后,有更多的人具有和张鉴同样的想法。文革中,北京大学季羡林教授被批斗时,就后悔当初不该念书。见季羡林:《牛棚杂忆》(北京:中共党史出版社,1998~年),页~100。}”

六、因无辜受冤而对“真理是共产党掌握”信条的怀疑。

张鉴受党多年教育,早已接受“真理是共产党掌握”的信条,但在“抢救”中的遭遇使其对这一信条也发生了动摇。他说,“既然掌握了真理,为什么对抗战以来参加的人要监视他,怀疑他呢?为什么人民群众反对,干部情绪动摇呢?\footnote{参见《中共现形》(爱国出版社,1946~年),页~25~—~33,该书无出版地点。}”

在晋绥根据地,象张鉴这样“经不起考验”而投奔国民党的干部只是个别,绝大多数被整干部虽承受巨大压力和痛苦,仍维持着对党的信念,等待着有朝一日能够洗却自己所蒙受的各种“特务”罪名。据说晋绥根据地整风、审干也有搞得较好的单位。

1943~年冬至~1944~年秋冬,晋绥八分区在日寇据点四面包围封锁的交城县关头村举办了三期整风班,参加者达三百余人。据记载,八分区的整风、审干“以学习整风文件为主”,“主要采取和风细雨的方式”,让“那些还有政治历史问题尚未向党交待清楚的同志,自觉自愿地把自己的问题交待清楚”,被集中参加审干的干部,“行动自由,不加监视,但要求遵守作息时间和外出请假制度”。当时在八分区担任领导职务的罗贵波说,八分区在审干期间,“对交代得好的同志经征求本人同意”,召开了两次大会,“让他们发言,以此对其他同志进行启发和帮助,效果很好”,\footnote{罗贵波:《革命回忆录》(北京:中国档案出版社,1997~年),页~184、235。}因而八分区未发生逃跑、投敌或自杀的现象\footnote{罗贵波:《革命回忆录》(北京:中国档案出版社,1997~年),页~184、235。}。

晋绥的“抢救”在~1944~年落幕。8~月,中央书记处电示林枫等要求做好“抢救”的甄别工作。至此,晋绥分局对被错戴“特务分子”帽子的同志进行甄别平反。

晋绥根据地的“抢救”细节至今仍未完全披露。贺龙、林枫长期领导晋绥,贺为行伍出身,对毛的才能十分崇拜,在理论和文化素养方面远逊于朱德、刘伯承、聂荣臻、陈毅等,较易受极左思潮的影响。林枫在贺龙调回延安后,事实上是晋绥的最高领导人,林枫与刘少奇有很深的历史渊源(1936~年春,刘少奇赴天津任北方局书记,林枫为刘的秘书),深受刘少奇的信任。1943~年后,刘少奇进入延安审干反奸斗争领导核心,在党内的地位正如日中升,以林枫为首的晋绥分局可以将周恩来为穆欣作的证明搁置一边,照样批斗穆欣,却一定不会将刘少奇的意见束之高阁。加之晋绥紧邻陕甘宁,延安一阵风刮来,晋绥不仅照办,而且变本加厉。

太行根据地包括太行、太岳、冀南三部分,又称晋冀豫区,抗战期间八路军总部与中共北方局均设于此。1943~年~10~月,“抢救运动”波及太行根据地时,彭德怀、杨尚昆、刘伯承已被调回延安,根据地最高领导人是邓小平。

1943~年~11~月~5~日,毛泽东致电北方局并转大行、太岳、冀南各区党委,毛在电文中说:“据彭(德怀)、罗(瑞卿)诸同志说,太行太岳两区反特斗争中在许多地点捉了并杀了许多人”,毛要求“必须立即检查此种现象”,“立即停止杀人,目前一年内必须实行一个不杀的方针,不许任何机关杀死任何特务分子。将来何时要杀人,须得中央批准”。毛并指示,逮捕特务人数“不得超过当地特务总数百分之五(百人中至多只许捉五人),并且一经坦白,立即释放”。\footnote{《太行革命根据地史稿~1937~—~1949》(太原:山西人民出版社,1987~年),页~201。另参见李雪峰:《李雪峰回忆录(上)——太行十年》(北京:中共党史出版社。1998~年),页~204,~李雪峰时任太行区委书记;《中央关于反对反奸斗争左的错误给各地的指示》(1944~年~5~月~13~日),载中国人民解放军国防大学党史党建政工教研室编:《中共党史教学参考资料》,第~17~册,页~389;毛泽东~1943~年~11~月~5~日致北方局并转太行、太岳、冀南各区党委电,载《文献和研究》,1984~年第~8~期,页~7~—~8。}毛的这封电报对制止太行、太岳肃奸中的极端行为起了立竿见影的效果。

\section{华中根据地的“抢救”}

华中根据地包括苏北、苏中、苏南、淮南、淮北、鄂豫等地区,是新四军控制地区。1941~年“皖南事变”后,刘少奇正式就任华中局书记和新四军政委,成为华中根据地最高领导人。1942~年~3~月,刘少奇前往延安,推荐其老部下饶漱石任华中局代理书记和新四军代政委。刘少奇回到延安后仍不时就重大问题向华中发出指示。次年~10~月,饶漱石在新四军军部所在地盱胎县黄花塘组织召开高干会议。就历史上陈毅与毛泽东关系不和之事向陈毅发起围攻。会后,陈毅被调回延安,华中与新四军的整风、审干基本是在饶漱石领导下进行的。

与华北相比,华中根据地有其特殊性。抗战之初,国民党军队自华北大规模后撤,华北根据地面临国民党军的压力远比华中小。新四军所在的华中地区集聚着数十万国民党军队,根据地与国民党区域犬牙交错,承受着侵华日军、国民党军队,尤其是桂系军队的巨大压力。其次,由于华中根据地紧邻沪、宁、杭等大城市,许多知识分子先后投奔新四军,知识分子干部在新四军中的比例要大大高于华北的八路军。

1943~年华中根据地的整风转入审干后,有两类人员首当其冲,成为重点审查对象。

第一类人员是过去在白区从事地下工作的知识分子干部,和抗战后投奔新四军的知识青年;第二类人员系抗战初期曾在桂系军队或桂系控制的苏皖地区从事统战,或在国共合作政权担任过行政职务的共产党员\footnote{1938~—~1939~年,中共安徽工委曾派出共产党员,参加苏皖部分地区由国民党地方势力与共产党合作的地方政权。}。

在整风、审干运动中,华中各地及新四军各师的情况不尽相同,运动进行最为激烈的,是淮南与淮北地区。

1943~年,遵照延安的指示,各根据地都实行了一元化领导,谭震林成为淮南和新四军二师的最高负责人(谭任淮南区委书记,二师政委)。谭震林是毛泽东在井冈山的老部下,在“路线斗争”中一贯站在毛泽东一边。

在~1943~年黄花塘高干会议上,谭也以当事人的身份参加了对陈毅的批判。谭震林的工作作风一向是大刀阔斧,对落实延安部署的整风审干工作自是雷厉风行。

整风转入审干后,延安中央书记处给华中局发来密电,指示在军部和二师一部进行“一般的清查特务分子的运动(公开名义为审查干部),以便能在华中突破一点,推动全局”,紧接着二师有一大批知识分子干部被打成特务。在华中局召开会议期间,三师师长黄克诚向谭震林询问二师的“抢救”情况,谭震林告诉黄克诚,二师每个团的“特务”都是“数以百计”。其中,二师政治部一来自上海的女知识分子干部不仅自供为“特务”,还咬出其同学——二师师长曾希圣的爱人也是“特务”\footnote{中央书记处:《发动华中反特运动指示》(1943~年~11~月~15~日),载中国人民解放军国防大学党史党建政工教研室编:《中共党史教学参考资料》,第~17~册,页~385;另见黄克诚:《黄克诚回忆录》(北京:解放军出版社,1989~年),页~300~—~301、299。}。

新四军三师师长黄克诚性格深稳沉着,有较高的文化素养,又亲历过苏区“肃~AB~团”的斗争,因而对“抢救”持比较谨慎的态度。“抢救”开始后,军部接到延安电报,延安已有人供出扬帆是“特务”。军部要求黄克诚立即逮捕三师政治部保卫部部长扬帆,并将其押送军部。黄克诚没有将扬帆扣押,而是让其前往军部开会,扬帆在军部被逮捕,关押长达一年半,直到~1944~年末才被甄别释放。

1943~年~4、5~月间,黄克诚向华中局和军部建议,应接受苏区“肃~AB~团”的教训,华中不要搞“抢救”,以免发生伤害无辜的事件。黄认为,大敌当前,不应搞“抢救”运动,再则党内历次搞肃反,总是出现扩大化的偏差。但是黄的建议未被华中局和军部接受。在上级命令下,黄克诚不得不执行“抢救”的命令。为了稳妥起见,黄先抽调一批干部办训练班,同时在三师七旅小范围内试行“抢救”。黄发现被审查者一经审讯就乱咬一气,马上意识到“不对头了,看来老毛病一下子是改变不了的”,于是黄克诚下令立即停止搞“抢救”,把被抓的人统统释放。黄并迅速通知苏北各地委和三师各部队一律不开展“抢救”运动。\footnote{中央书记处:《发动华中反特运动指示》(1943~年~11~月~15~日),载中国人民解放军国防大学党史党建政工教研室编:《中共党史教学参考资料》,第~17~册,页~385;另见黄克诚:《黄克诚回忆录》(北京:解放军出版社,1989~年),页~300~—~301、299。}黄克诚与二师师长曾希圣是老战友,1928~—~1929~年两人还曾有过一段患难与共、共同找党的经历。黄在华中局会议期间见曾希圣因爱人被指称为特务而愁眉不展,特通过二师政委谭震林,亲自找那位咬出曾希圣爱人的二师政治部女干部谈话。该女干部一口咬定自己是“特务”,将情节说得活龙活现,在黄克诚的细心盘问下,女干部终于承认自己所说全系伪造,自述因为压力太大,非承认“特务”不能过关。这样,曾希圣爱人特嫌一事才得到了解脱。

与淮南区“抢救”的重点在军队内部有所不同,淮北区在“抢救”中造成的冤假错案主要集中在地方。淮北区为新四军四师辖地,1941~年春,四师师长彭雪枫遭国民党汤恩伯部突袭,蒙受重大损失。事变后,延安与华中局调军政治部主任邓子恢前来四师,不久任命邓子恢为四师政治委员,并任苏皖边区军政委员会书记——淮北区党委书记,成为淮北区一元化的最高领导。

1943~年夏,延安的“抢救”经验传到淮北,不久就发生了“淮中案件”和“泗阳案件”两起严重的肃特假案。

1943~年~8~月下旬,淮北中学一女生在她偷窃别人五十元边币行为被发现后,为推卸错误,捏造是受一“特嫌”女生指使所为。在校领导的压力下,偷钱女生编造了“进步青年建国团”的组织,结果该校领导对全校师生实行了刑讯逼供,将一个~220~人的中学中的四十二人打成“特务”。此案上报边区公安局后,情况不仅未被澄清,边区公安局采用逼供信的方式,将“特务”又增加到五十六人,使全校二分之一的教职员工都成了“特务”。此案最后在邓子恢、彭雪枫的主持下,于~1944~年~7~月~17~日予以平反,并对有关人员分别作出处理\footnote{淮北区委在《关于淮北中学第二次反特案件的错误及善后处理办法的决定》中指出,“学校的主要任务,是用新民主主义教育精神去教育青年,改造青年,使团结在党的周围”,“决不能把学校变成法庭,严刑逼供,捆绑,吊打的,行为是绝对不允许的”。见《邓子恢传》编委会:《邓子恢传》(北京:人民出版社,1996~年),页~295。}。

“泗阳案件”发生在~1943~年~10~月至~1944~年~2~月,在泗阳这个当时不到十万人口的县内,一连四次破获所谓“三青团县委”机关。短短五个月时间里,全县逼出一千四百多个“三青团员”,并逮捕了一百五十多人,范围涉及党政军民等各界和县区乡及部队连排班干部。1944~年~2~月,案件进一步扩大,泗阳县委筹划大规模逮捕县武装干部。此案全由刑讯逼供造成,指控罪犯全部听信口供,有些则凭主观臆测,毫无人证物证。在邓子恢主持下,此案第三、第四批被捕人员在~1944~年~3~月予以平反(第一、第二批案犯继续审查)。邓子恢并代表区党委向在这案件中蒙冤的干部群众道歉,淮北区党委还宣布改组泗阳县委,撤销县委书记、县长、公安局长的职务,给予纪律处分\footnote{邓子恢:《我的自传》,载《革命史资料》(北京:文史资料出版社。1982~年)(8),页~13~—~14;另见《邓子恢传》编委会:《邓子恢传》,页~297。}。

新四军五师远离军部中心地区,1943~年~11~月华中局和军部调郑位三前往五师所在的鄂豫边区担任华中局代表,起因是延安和华中局对五师师长李先念在政治上不信任,认为五师内部已有内奸打入。郑位三早年毕业于武汉甲种工业学校,是~1928~年红安、黄麻暴动的领导人之一,参与创建了鄂豫皖根据地,抗战后,曾担任过新四军二师政委。郑位三不仅革命资历雄厚,而且为人正直,处事十分谨慎。郑位三前往鄂豫边区后,没有公布中央两次任命他为鄂豫边区党委记兼五师政委的电令,“谨以华中局代表的身份出面工作”。郑位三甚至还私下提醒李先念:“要接受高敬亭被杀的教训!”郑经过调查,证明“内奸打入五师”一说完全不能成立。经过郑位三的工作,延安逐渐消除了对五师的“怀疑和误解”。郑位三、李先念在防止审干朝极左方面发展也起了重要作用。边区党委从无线电通讯中收到延安转发的许多“抢救”经验和指明某某是特务的电报,但是郑位三等吸取了内战时期滥施肃反的教训,在五师及鄂豫边区,都没有搞“抢救”,“仅在黄冈的审干曾一度发生偏差”\footnote{《李先念传》编写组编,朱玉主编:《李先念传(1909~—~1949)》,页~487~—~88、504、481;另见《新四军第五师抗日战争史稿》(武汉:湖北人民出版社,1989~年),页~213。}。

华中根据地和新四军内的审干、“抢救”虽在规模上小于延安,但是由于华中紧邻国民党区,党内清洗的激烈性和残酷性仍十分突出。多年来,华中和新四军内一直在进行日常性的锄奸、肃特、“肃托”工作,只是过去这类工作多由保卫部门秘密进行,“肃奸扩大化”一类事外人一般不得而知。还在“抢救”运动之前,1939~—~1942~年在淮北、淮南就先后以“托派”、“内奸”的罪名处决了戴季康、查化群、韦延安三位青年共产党员。

戴季康,1918~年出生在四川省巴县一农民家庭,1934~年来到上海,进入陶行知创办的大场山山海工学团,1935~年秋加入中共。抗战爆发后,经党组织批准,戴通过上层社会关系的介绍,进入皖东北泗县县政府做统战工作,公开职务是县政府政训处副处长,并担任了国民党地方部队安徽第六游击纵队第一支队政训工作。1939~年~11~月,戴以“托派”罪名在安徽泗县羊城镇(令属江苏省泗洪县)附近被杀。临刑前,呼喊“中国共产党万岁”的口号,年仅二十一岁\footnote{1981~年中组部复查,确认“戴季康托派案”为错案,予以平反昭雪。}。

查化群以“托嫌”、“内奸”罪名被处死是在~1942~年冬。查也是在苏皖边区从事统战工作的共产党员。1938~年底,查化群经鄂豫皖区党委批准,前往国共合作的安徽怀宁县石牌区任区长。1939~年国共关系恶化,查退回根据地,被分配到新四军二师四旅任副官主任,并担任过旅的作战参谋。处决查化群前,二师政治部主任张劲夫曾不同意,但不起作用,几十年后张劲夫披露道,此“是更高层决心处决的”\footnote{张劲夫:《怀念集》(北京:中共中央党校出版社,1994~年),页~221。}。张劲夫调四旅任政委后,对尚被怀疑为“托嫌”的旅政治部宣传科长余路尽力予以保护,使他未受牵连。查案也是在八十年代初才由安徽省委平反。

韦延安是广西学生军中的中共支部书记,受鄂豫皖区党委常委、民运部长张劲夫单线领导,在苏皖桂系部队做统战工作,1940~年调回新四军。韦延安在抗大四分校学习半年后,分配到新四军四师十一旅某团任连指导员,曾因作战英勇负伤,被提拔为营副教导员,1941~年也被无辜怀疑为“托派”,遭到处决。韦延安冤案在~1981~年~12~月获平反。

\section{唯一未开展“抢救”的山东根据地}

在“抢救”风暴席卷延安及各大根据地之时,山东根据地却按照原有的部署,依然在进行着整风和审干运动,没有在军队和地方的党政机关、学校大抓“特务”。

山东根据地未开展“抢救”的关键原因是主持山东根据地党、政、军一元化的最高负责人罗荣桓反对在山东搞“抢救”,而罗荣桓作出这个决定又与山东整风的特殊性有着密切的关系。

与其它根据地有所不同的是,山东根据地的整风所要解决的主要问题是山东分局领导机关内部的意见分歧问题。在整风运动期间,在各大根据地中,只有山东调整了领导机构,撤换了第一把手,罗荣桓就是在这次人事变动中,取代了中共山东分局书记朱瑞,成为山东根据地的最高领导。

罗荣桓是~1939~年~3~月率八路军一一五师一个主力团和师部机关进入山东的,在~1943~年以前,罗荣桓在党内的地位长期在朱瑞之下。罗荣桓与朱瑞是老熟人,江西时期曾经留学莫斯科克拉辛炮兵学院的朱瑞受到博古、周恩来等的重用。1932~年初,年仅二十七岁的朱瑞在上海中央工作两年后被派往中央苏区,不久就被委之以红五军团政委的重任。在~1934~年初召开的中共六届五中全会上,朱瑞成为中央候补委员,紧接着朱瑞又在“二苏大会”上,当选为中华苏维埃共和国临时中央政府执行委员。长征前夕,以李德、博古、周恩来组成的“三人团”紧急调配干部,被誉为“文武兼备的红军指挥员”的朱瑞被调至中央红军的主力——红一军团任政治部主任,他的副手就是罗荣桓。1937~年后,朱瑞被调离军队,在太原担任北方局军委书记,继而又任北方局驻太行区代表、北方局组织部长等职,主要从事对华北地方实力派的统战工作。1939~年~6~月,朱瑞奉命随徐向前率一支百余人的小部队自冀南入鲁,成为由徐向前任司令员的八路军第一纵队的政委,不久,中共山东分局书记郭洪涛奉调返回延安,朱瑞被任命为中共山东分局书记。

朱瑞入鲁之际,中共在山东已有两支武装,一支为罗荣桓率领的八路军一一五师,另一支为山东党负责人黎玉和从延安先期调鲁的张经武建立的八路军山东纵队,而太行八路军总部给徐向前、朱瑞的权限是以八路军第一纵队的名义,统一指挥一一五师、山东纵队和中共在苏北的部队。但是八路军总部的这道命令并没有得到贯彻和执行,而是在无形中取消了。1940~年~5~月,八路军第一纵队被撤销番号;同年~6~月,徐向前奉命返回延安预备参加拟议中要召开的中共七大,朱瑞则失去了军队的任何职务,从此不再参与指挥军队。

战争期间,情况瞬息万变,上级机关收回成命的事情,时有发生,惟取消八路军第一纵队番号事,却有蹊跷和微妙之含意。抗战初期,延安军委大部分领导成员都在华北敌后,驻扎在太行的八路军总部事实上起着指挥敌后抗战的中枢机关的作用,在党的组织系统,中共山东分局亦属中共北方局领导。在一段时期内,太行的八路军总部对华北、山东的影响相对于延安更大。经常的情形是,太行直接下令,然后报延安备案。例如,一一五师入鲁,就是奉八路军总部朱德、彭德怀之命,而朱、彭则是依照延安毛泽东的指示发出这道命令的。当然,延安始终保有对华北、山东的直接指挥权,毛泽东可以直接给军队和地方发出指示。

太行的八路军总部及中共北方局对华北、山东敌后根据地的指挥与领导,随着八路军总司令朱德和北方局书记杨尚昆相继返回延安后,开始发生变化,从此延安对敌后的指导作用日益突显出来,八路军第一纵队番号被无形撤销,就是在这个大背景下发生的。

在八路军第一纵队的番号被取消后,朱瑞的主要职责是领导中共山东分局。战争期间,军事第一,党的机关必须随军活动,最适当的安排就是实行一元化领导。晋察冀根据地自分局书记彭真于~1941~年春返回延安后,党、政、军全盘工作皆由聂荣臻负责。在太行,北方局书记继杨尚昆后,先后由彭德怀、邓小平挂帅。聂、彭、邓都是率兵之人,同时兼任党的领导职务,因而晋察冀、太行各项工作的展开,都较为顺利。至于大战略区两职分开者,往往易生纠葛。在山东,党的领导与军队领导,两职长期分开,各司一摊,执事者角度有异,难免产生意见分歧。

在山东,还有其特殊问题:一一五师与八路军山东纵队的统一指挥久未解决,受到战争环境的影响,山纵与一一五师分别处在不同的作战区域,事实上形成了领导山东抗战的两个中心。

延安已注意到山东领导机关的分散现象,在~1941~年后曾作过努力,试图加以改变,但是来自延安的指示又包含某种模糊性。1941~年~8~月,中央和军委规定,山东分局为统一山东党、政、军、民的领导机关,由朱瑞任书记;山东纵队归一一五师首长指挥;山东纵队和一一五师两军政委员会合组为山东军政委员会,由罗荣桓任书记。从字面上理解,朱瑞似乎应参与领导一一五师和山东纵队,但是朱瑞并不是由罗荣桓任书记的山东军政委员会成员,于是山东的多头领导现象在~1941~年后继续存在,一一五师与山东纵队仍没实现集中统一。

朱瑞在山东最闪亮的时期是~1939~至~1940~年,1941~年后,侵华日军施用“拉网合围”、“铁壁合围”等残酷的手段频繁进攻根据地,致使山东根据地的面积急剧缩小,全省根据地的人口从~1200~万锐减到~730~万。大批抗日军民惨遭日军杀害,在牺牲者的名单中,有朱瑞的妻子、山东分局妇女委员陈若克和她刚出生的儿子。

根据地蒙受重大损失,使领导机关内部的争论也趋于激烈。早在~1940~年~9、10~月间,山东分局领导和一一五师部之间就曾在军事战略方针问题、抗战问题和山东纵队与一一五师会拢等问题上有过一些争论,1941~年~4~月后,罗荣桓更对分局的某些决策不止一次提出过意见,但都未得到重视,进入~1942~年,形势愈加严峻,罗荣桓致电北方局和中央,建议分局召开扩大会,请中央派刘少奇来参加,以总结山东工作,明确今后目标。\footnote{《当代中国人物传记》丛书编辑部编:《罗荣桓传》(北京:当代中国出版社,1991~年),页~261。}毛泽东同意了罗荣桓的请求,电召正欲前往延安的刘少奇,在途经山东时,就地考察山东形势,并对领导机关内部的争论作出裁决\footnote{《刘少奇年谱》,上卷,页~392。}。

此时的刘少奇虽是政治局候补委员,但却担负着极重要的中共华中局书记、新四军政委等职务,正受到毛泽东的特别倚重。1942~年~4~月,刘少奇抵达山东分局和一一五师驻地,一住四个月,对山东问题作出结论:山东问题的症结所在,是中共山东分局主要领导同志在一系列重大问题上犯了错误。

刘少奇所指何许人也?朱瑞是也。刘少奇批评朱瑞对党的独立自主的方针执行不力,缺乏战略眼光,反顽(“国民党顽固派”)斗争瞻前顾后,失去许多“先机”;其二,严重忽视发动群众,未能广泛展开减租减息;其三,主观主义,形式主义,空谈主义,党八股;其四,在锄奸政策上犯有严重错误。刘少奇说,如果再不改,就要送一块匾,上面写四个字:“机会主义”\footnote{王力:《现场历史——文化大革命纪事》,页~128;另参阅萧华:《难忘的四个月——忆少奇同志在山东》,载中共山东省党史资料征集研究委员会编:《山东抗日根据地》(北京:中央党史资料出版社,1989~年),页~246~—~59。}。

刘少奇来鲁指导工作之际,正是山东根据地形势最困难的时期,若从毛泽东的立场论之,刘的批评基本符合实际,某些意见,罗荣桓过去也曾向朱瑞提过,作为山东分局负责人的朱瑞实难辞其咎。朱瑞被批评的最突出之处是他对中间势力“抗敌自卫军”的态度过份热心。该武装是在朱瑞支持下组成的“山东国民党抗敌同志协会”拉起的队伍。朱瑞大力帮助他们扩充军队,提供了一些经费和武器,使其在根据地成为与中共和八路军平起平坐的组织和军队。为此,罗荣桓和江华(山东纵队政治部主任,1949~年后长期担任浙江省委第一书记)都曾向朱瑞提过意见,但未被朱瑞接受。1940~年的朱瑞并没有真正吃透毛的统一战线的策略思想,在事关党的领导权的关键问题上犯了大忌。然而,若搬照条文,朱瑞似乎并无过错,因为在抗战前期,中共中央根据山东地区的具体情况,曾提出在山东与国民党力量共同创建抗日根据地的设想,朱瑞支持“抗敌自卫军”与“共同创建根据地”的指示有密切关系。至于未能全面推行减租减息,确是事实,站在毛泽东、刘少奇的立场,当是朱瑞的一大错误,可是如果完全顺应农民的要求,放任“群众自己解放自己”,似乎也会危及抗日民族统一战线,尤其在日军空前残酷进攻的形势下\footnote{熟悉山东现代史的读者不会忘记,几年以后,在康生直接领导下的山东渤海地区的土改,就是打着“自己解放自己”的旗号,而造成了极为恶劣的后果。}。

朱瑞虽已是久经考验,独当一面的高级领导干部,但他身上还保留某些书生气。朱瑞颇善于做鼓动性的大报告,口里经常是“从国际到山东”,\footnote{王力:《毛泽东谈整风审干和遵义会议——从对〈王若望自传〉中若干问题的说明谈起》,载王力:《现场历史——文化大革命纪事》,页~128。}他甚至为推动根据地婚姻制度的改革,作过几个小时的动员报告。以那时的眼光看,这就是夸夸其谈的“空谈主义”和党八股了。

山东根据地若干地区的“肃托”虽经制止,但终未断根,以后又与“锄奸”夹杂在一起,这其间自有深刻的原因,非朱瑞一人所能负责,但作为分局书记,朱瑞总有一份领导责任(在华北、华中均发生错误的“肃托”事件,却未见领导同志被批评)。

朱瑞,这颗一度闪烁的星辰黯淡了。1943~年~8~月,朱瑞奉调返回延安,而在此前的~3~月,中央军委就正式任命罗荣桓为山东军区司令员、政委和一一五师政委、代师长,统一指挥一一五师和山东纵队。朱瑞返回延安后,延安立即任命罗为山东分局书记,将领导山东党、政、军一元化的重任交付罗荣桓。

延安早有调整山东根据地领导班子的意图,1942~年刘少奇来鲁,延安就曾有过这一考虑,刘少奇返延安后毛即决定改组山东领导机构,并曾就此问题与在太行的彭德怀进行了“反复磋商”,\footnote{《当代中国人物传记》丛书编辑部编:《罗荣桓传》(北京:当代中国出版社,1991~年),页~285。}现在,朱瑞职务的变动,新的一元化体制的建立,正是水到渠成。

罗荣桓地位的上升与朱瑞之被贬谪是毛泽东整风全盘战略的一部分,毛不仅要解决中央层的路线与权力再分配问题,大区一级领导层的路线清算和机构改组也在他的视野之内。在刘少奇离开山东转赴延安后,1942~年~10~月~1~日,中共山东分局作出《四年工作总结》,检讨了过去在执行统战方针、减租减息政策等方面的“错误”。这个报告具体反映了刘少奇对山东工作的看法。刘少奇本意是想让罗荣桓来作这个报告的,刘认为,朱瑞作为责任者之一,已不适合再来作报告,但朱瑞“不自觉”(刘少奇语),主动接过刘少奇的话,要求作这个报告,刘少奇只好同意。\footnote{王力:《毛泽东谈整风审干和遵义会议——从对〈王若望自传〉中若干问题的说明谈起》,载王力:《现场历史——文化大革命纪事》,页~128。王力在整风期间是山东分局总学委秘书,1961~年初,刘少奇在湖南与王力谈了上述一段话。}从罗荣桓这方面讲,他与朱瑞确实存在意见分歧,但是朱瑞的下台却与罗个人无关。朱瑞被调离山东,乃是源于各种复杂因素的综合作用。朱瑞早年留苏的经历,瑞金时期蒙受重用,随着时空的转移,此时已成为消极性因素,朱瑞的上述经历,已够划入“教条主义者”一类。延伸下来,抗战前期,山东工作屡失“先机”,此又可与“王明右倾投降主义”挂上钩。更重要的是,朱瑞缺乏早年追随毛的经历,到达瑞金后,毛在党内已遭贬斥,在毛最困难的岁月里,朱瑞没能和毛建立起亲和性的个人联系。整风期间,朱瑞已年近四十,称得上是老同志了,然而依那时的党内习惯,朱瑞还称不上是党的元老级干部,那些大革命时期入党,曾去苏联短期学习随即返国,先后担任重要职务的同志,在整风运动中尽管也被指责犯了这类或那类错误(“教条主义”或“经验主义”,或“教条主义兼经验主义”),但是他们雄厚的革命历史和巨大的个人威望已与党的事业融为一体,使得毛在处理他们问题时不得不斟酌再三,一般在批评了他们以后,仍然派以重用,但朱瑞显然不属于这类元老级干部。朱瑞回到延安后,被安排在中央党校一部参加整风学习,在党校学习的其它一些原地方领导,一段时间后都程度不同地获得了工作机会,然而朱瑞在中央党校一学就是两年,直至中共七大后才被派任为延安炮兵学校的代理校长(朱瑞推辞了中央建议他担任的军委副总参谋长,主动要求干炮兵工作)。

朱瑞被调、罗荣桓接替山东全面工作之际,在全党范围正是整风转入审干、反特的阶段,如何开展山东的整风、审干运动马上就成了罗荣桓急待面对的紧迫问题。1942~年全党整风开始后,由朱瑞任书记的中共山东分局曾发出通知,部署全区开展整风文件学习和检查对照工作,但是一则当时战事频繁;二则刘少奇来鲁忙于调查领导班子中的意见分歧,山东的整风实际上尚未展开,所以山东根据地大规模的整风是在战事相对平稳的~1944~至~1945~年才进行的。

如前所述,山东的整风主要是解决上层的意见分歧及领导机构的调整,这个任务在罗荣桓就任山东分局书记后实际上就已完成,但整风并非仅仅是清算上层的路线,解决支持谁、反对谁的问题,它还包括在党的中下层普遍开展思想革命、培养新人、审干肃奸等方面的内容。正是在涉及这些关键问题上,罗荣桓显示了他谨慎、稳妥的领导风格。

整风转入审干后,康生的《抢救失足者》的小册子已寄到山东,但罗荣桓明确反对在山东搞“抢救”,他下令分局办公厅“不要向下分发”\footnote{李维民、潘天嘉:《罗荣桓在山东》(北京:人民出版社,1986~年),页~311。}。

康生的小册子虽然被停止下发,但开展“抢救”毕竟是来自延安的指示,更重要的是,延安已向各大根据地派出干部推广“抢救”经验,派到山东的特使,就是被安排担任中共山东分局常委兼秘书长、山东分局总学委副主任的舒同,他于~1944~年~9~月从延安来到山东,具体主持山东的审干工作。

舒同抵鲁并没有动摇或改变罗荣桓在山东的地位,罗作为执行“正确路线”的代表,刚刚被委任负责山东的全盘工作,山东的整风领导机关总学委也是由罗任主任。然而对于延安的指示,罗又不便直接加以反对,因此,罗荣桓同意挑出几个试点单位用延安的经验来指导运动。

延安的经验并无特别的奥妙,这就是在延安各机关、学校,尤其在中央党校、中央研究院普遍运用的以召开“民主检查大会”,暴露敌人,“引蛇出洞”,再累而歼之的策略。这个策略来源于~1943~年第二个“四三决定”,该决定提出,为了使内奸分子尽量暴露其反党面目,“继续整风的第一阶段,必须极大的提倡民主,公开号召参加整风的一切同志大胆说话,互相批评,提倡各学习单位出墙报,写文章,批评领导,批评工作,而一般地(特殊情况例外)绝不加以抑制”。舒同来鲁,即负有推广此经验的使命,根据延安的部署,“全国(各根据地)都要通过民主检查暴露特务”\footnote{王力:《毛泽东谈整风审干和遵义会议——从对〈王若望自传〉中若干问题的说明谈起》,载王力:《现场历史——文化大革命纪事》,载王力:《现场历史——文化大革命纪事》,页~124。}。

但是在公开场合,对召开“民主检查大会”却有一套冠冕堂皇的解释:这就是通过发扬民主,使下情上达,帮助和促进领导改进缺点。至于党的秘密策略,对外则秘而不宣,参加民主检查大会的绝大多数党员干部一点也不知道,召开这种大会的真正目的是“钓鱼”,即以开大会的方式使“内奸分子”产生错觉,以为有机可乘,跳将出来而暴露其“反党”面目\footnote{李维民、潘天嘉:《罗荣桓在山东》(北京:人民出版社,1986~年),页~311。}。

在舒同的影响下,中共山东分局于~1944~年~10~月~13~日发出《关于整风审干的基本总结与令后的指示》,要求全区工作一切以整风审干为主。11~月~1~日,山东分局又发出《关于目前整风审干的补充指示》,该文件不指名地批评了罗荣桓经常表示的某些观点,《补充指示》说,山东许多同志“对于特务世界性与群众性的认识不足,片面地强调山东的特殊性,如说山东外来干部少,国特不易打入,军队没有问题,我们的干部大都经过锄奸斗争与长期考验,等等……”,\footnote{《当代中国人物传记》丛书编辑部编:《罗荣桓传》(北京:当代中国出版社,1991~年),页~349。}针对上述情况,《补充指示》提出要“大搞民主”。然后于适当时机进入全面反省坦白运动\footnote{李维民、潘天嘉:《罗荣桓在山东》,页~312。}。

果其不然,在几个试点单位:省战时行政委员会(省的政权机构)、分局办公厅、《大众日报》(分局机关报)、军区特务团、军区卫生部根快出现了大呜大放的局面。上级领导鼓励大家“有话就请,有屁就放”(此是毛在整风期间创造的名言),大民主终于将“鱼”钓了出来,许多党员干部运用壁报、发言、漫画,将批评矛头直指上级领导。分局常委兼省战时行政委员会主席黎玉因“官僚主义”问题,分局宣传部副部长兼《大众日报》社长陈沂因“生活特殊化”问题首当其冲(陈沂有一匹专用乘马和他本人喜食辣椒,在当时被视为是“特殊化”),受到较多的批评,一时间各种自由化言论“就像洪水决堤似的泛滥起来”。\footnote{李维民、潘天嘉:《罗荣桓在山东》,页~313。}在这批人中,最有名的是当时在山东分局机关工作的王若望。在民主检查期间,王若望十分活跃,他表示拥护罗荣桓,但对黎玉十分不满,认为黎玉有“八大盲目性”,“路线方针都错了”\footnote{李维民、潘天嘉:《罗荣桓在山东》,页~313。该书未点王若望的名,也未提及黎玉的名字,但实际上指的就是王若望批评黎玉一事。参阅《当代中国人物传记》丛书编辑部编:《罗荣桓传》,页~351。该书提到黎玉被批评一事,但略去了王若望的名字。}。

本来开展民主检查的目的就是为了“引蛇出洞”,王若望自己跳出来,正好撞在枪口上,他被称之为“山东的王实味”,并被认为有一套自己的思想体系,煽动其他人来反党。在山东分局内部,舒同等人已决定将王若望定为“特务”,并把王若望的情况电告延安,延安也发来电报,同意将王若望定为“特务”\footnote{王力:《毛泽东谈整风审干和遵义会议——从对〈王若望自传〉中若干问题的说明谈起》,载王力:《现场历史——文化大革命纪事》,页~133。}。

在王若望即将陷于灭顶之灾的时刻,罗荣桓站出来讲话了。罗明确提出,王若望是思想作风问题,不是特务和敌人。罗荣桓的表态拯救了王若望,罗约王若望单独谈话,对王的言论没有全盘否定,但批评王唯我独尊,目空一切,否定山东全局是思想片面。

罗荣桓对于召开民主检查大会有自己的看法。他认为发扬民主。“放一把火,把领导同志烧一烧有好处”,但坚决反对用这种方法来“钓鱼”。\footnote{李维民、潘天嘉:《罗荣桓在山东》,页~312。}罗荣桓认为,山东形势特殊,处在对敌斗争的复杂环境里,敌人已为我们审查好了干部,如果某同志是“特务”,他早就自己跑了,根本用不着以发扬民主的方式来暴露敌人。至于“民主检查大会”,既然已经开了,就要明确目的,这就是检查“改进领导,而不是暴露特务,暴露敌人”\footnote{王力:《毛泽东谈整风审干和遵义会议——从对〈王若望自传〉中若干问题的说明谈起》,载王力:《现场历史——文化大革命纪事》,页~124。}。

在山东分局机关开始“民主检查”时,山东各战略区也在各自试点的单位进行了类似的活动。在胶东区,一个从延安派来的领导干部具体领导了区党校的审干,此人运用“钓鱼”的方式,将一批干部打成“特务”,并用“疲劳战”加以审讯定案。有的同志不堪逼供,供出“特务”的枪枝比一个军分区所拥有的枪枝还要多。罗荣桓在听取了胶东区区委书记林浩的汇报后,明确指示胶东区停止试点,将搞出来的材料全部烧掉。\footnote{李维民、潘天嘉:《罗荣桓在山东》,页~317、308~—~309。}罗荣桓对于军区卫生部的运动也提出了批评,他要求停止已开了六天的民主大会,不久日军又开始“扫荡”,卫生部内被认为有疑点的人在“反扫荡”中都英勇积极,没有一个叛变投敌,有力证实了罗荣桓判断的正确性。

罗荣桓拒绝用开“民主检查大会”的方式来“暴露敌人”,也抵制了来自延安的“抢救”指示,尽管舒同坚持运动不能收,整了风还要审干,还要找特务,但是罗荣桓毕竟是山东根据地的第一把手。罗荣桓只同意搞整风审干,到了~1944~年~10~月,山东全区约五千党员干部参加了反省坦白活动,在罗荣桓的领导下,山东全区在审干中基本未搞“车轮战”、“逼供信”,也没有杀一人\footnote{李维民、潘天嘉:《罗荣桓在山东》,页~317、308~—~309。}。

1944~年~12~月~31~日,罗荣桓以个人名义向全区发出《关于审干问题的意见》,同时报中共中央和毛泽东。罗提出,不应把整风与审干完全混淆起来,应避免“形成一种突击的倾向”,“轻易发动坦白和严重的逼供信”。罗认为整风的正确方法应是“以领导开场,以检查领导结束”。\footnote{李维民、潘天嘉:《罗荣桓在山东》,页~318。}1945~年~3~月~15~日,在罗荣桓主持下,山东分局在给延安的电报《关于民主检查的检讨》中一方面肯定整风与审干是不可分离的,在另一方面,又用较大的篇幅检查了“民主检查”的缺点。他说,“错误主要表现在,在发扬民主中,只是强调审干的目的,强调暴露的方针,因而发生极端化的偏向,甚至采取不正确的动员方法,去助长群众的偏向,结果是把发扬民主当成了暴露。”\footnote{李维民、潘天嘉:《罗荣桓在山东》,页~318。}

罗荣桓在左倾风暴席卷全党的形势下,敢于坚持自己的意见,用灵活的方法处理了山东区的整风审干中所暴露出的问题,不仅没有将“闹事”的干部打成了“特务”,更没有人云亦云,跟在延安后面在山东搞“抢救”运动,这在当时的情况下实属罕见,在各大根据地中仅此一家。罗荣桓为什么敢于抵制来自延安的“抢救”指示?首先,几年前,湖西“肃托”惨痛的教训给罗荣桓留下了极深的印象,促使他谨慎从事。罗荣桓曾亲自处理湖西“肃托”事件的善后工作,深知湖西“肃托”给山东根据地带来的巨大危害,而这次审干比当年“肃托”规模还要大。他认为如果按照延安的“抢救”和开“民主检查大会”暴露敌人的方法来搞审干,一定会搞垮山东根据地。\footnote{王力:《毛泽东谈整风审干和遵义会议——从对〈王若望自传〉中若干问题的说明谈起》,载王力:《现场历史——文化大革命纪事》,载王力:《现场历史——文化大革命纪事》,页~124。}第二,延安发出的第二个“四三决定”中的一段话,为罗荣桓抵制“抢救”提供了解释的理由。该决定提出,“敌后情况与延安大不相同,主观计划常为客观情况变化所中断,故应灵活化运用延安经验,着重于自己创造新经验。随着情况变化而修改自己的计划,采取适应环境的处理”。罗荣桓正是抓住这句话,反复强调山东情况特殊,不能机械照搬延安的经验。第三,罗荣桓勤于思考,性格方正,具有较高的文化修养(罗青年时期曾就读于青岛大学)。他也是中共历史上曾经有过的那种具有理想主义色彩的纯正共产党人的典范,罗对“革命”、“整风”有其自己的理解,因此在他权力范围内不能容忍那种以革命名义出现的阴谋诡计。最后,罗在山东战斗多年,在干部和群众中享有很高的威望,舒同根本无法与其抗衡。

对于远在山东发生的一切,毛泽东完全清楚,对于罗荣桓的“抗上”,毛不仅予以容忍,甚至还对山东的整风表示了赞赏,这又是什么原因呢?毛泽东十分了解罗荣桓的历史和为人。罗荣桓是唯一幸存的跟随毛参加秋收暴动、上井冈山、以后又长期追随毛、与毛共患难、在党内斗争中和毛共沉浮、现在又独当一面的党的高级干部。\footnote{王力:《毛泽东谈整风审干和遵义会议——从对〈王若望自传〉中若干问题的说明谈起》,载王力:《现场历史——文化大革命纪事》,页~93。据王力回忆,1963~年~12~月,毛对康生、王力说,凡是我倒霉的时候,罗荣桓都是和我一起倒霉的。}毛深知罗性格谨慎,不喜夸张,不好出风头,无个人野心,他对山东整风的意见和部署只是出于对山东大局的考虑,绝非有意“抗上”。1944~年~7~月~1~日,罗荣桓为纪念中共成立二十三周年发表《学习毛泽东同志的思想》一文,明确拥护毛的路线和主张,在毛与王明等的斗争中,旗帜鲜明地站在毛的一边。所有这一切都被毛一一看在眼里。在毛看来,罗荣桓是自己完全可以依赖的干部。毛也欣赏罗荣桓在山东为中共建立的特殊功勋,在罗接任山东工作后,中共军队在山东获得长足发展,地盘也不断扩大,到~1945~年抗战胜利前夕,八路军几乎完全控制山东的战略要点和交通线,从而使中共掌握了极为重要的战略主动权,为攻占东北,南下长江提供了充足的保证,为中共打败国民党立下了汗马功劳。数十年后,毛还念念不忘罗对中共革命的贡献,毛说换上罗荣桓一个人,山东全局的棋就下活了,罗在决定中国革命成败的地区为革命的胜利作出了巨大贡献。\footnote{王力:《毛泽东谈整风审干和遵义会议——从对〈王若望自传〉中若干问题的说明谈起》,载王力:《现场历史——文化大革命纪事》,页~94。}毛接受山东事实还有一个重要原因:1944~年末至~1945~年初正当罗荣桓在山东抵制“抢救”时,延安的“抢救”早已结束,山东由于未与延安同步开展“抢救”,在延安和各根据地已开始“甄别”工作之际,毛泽东实在不能也不便让山东“补课”。毛为了显示自己与“抢救”错误毫无关系,1944~年~5~月~13~日还以中共中央的名义发出《关于反对反奸斗争左的错误给各地的指示》,向全党通报山东整风的经验。1963~年,毛在小范围谈话中,又表扬罗荣桓正确开展整风,“以领导开始,以检查领导结束”。抵制了“抢救”运动。\footnote{王力:《毛泽东谈整风审干和遵义会议——从对〈王若望自传〉中若干问题的说明谈起》,载王力:《现场历史——文化大革命纪事》,页~94。}正是源于上述种种因素,山东根据地成为唯一未开展“抢救”的“特区”。中共虽有严格的组织纪律,但毛在运用这些纪律驾驭全党的时候却是大有区别,十分讲究的。因此对于罗荣桓在整风中“另搞一套”,毛不仅未予追究,相反,罗荣桓还长期深受毛的信任和重用。