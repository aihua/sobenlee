%# -*- coding:utf-8 -*-
%%%%%%%%%%%%%%%%%%%%%%%%%%%%%%%%%%%%%%%%%%%%%%%%%%%%%%%%%%%%%%%%%%%%%%%%%%%%%%%%%%%%%


\chapter{李瓶儿何家托梦\KG 提刑官引奏朝仪}


词曰:

\[
花事阑珊芳草歇,客里风光,又过些时节。小院黄昏人忆别,泪痕点点成红血。咫尺江山分楚越,目断神惊,只道芳魂绝。梦破五更心欲折,角声吹落梅花月。
\]

话说西门庆同何千户回来,走到大街,何千户就邀请西门庆到家一饭。西门庆再三固辞。何千户令手下把马环拉住,说道:“学生还有一事与长官商议。”于是并辔同到宅前下马。贲四同抬盒迳往崔中书家去了。原来何千户盛陈酒筵在家等候。进入厅上,但见兽炭焚烧,金炉香霭。正中独设一席,下边一席相陪,旁边东首又设一席。皆盘堆异果,花插金瓶。西门庆问道:“长官今日筵何客?”何千户道:“家公公今日下班,敢屈长官一饭。”西门庆道:“长官这等费心,就不是同僚之情。”何千户道:“家公公粗酌屈尊,长官休怪。”一面看茶吃了。西门庆请老公公拜见,何千户道:“家公公便出来。”

不一时,何太监从后边出来,穿着绿绒蟒衣,冠帽皂鞋,宝石绦环。西门庆展拜四拜:“请公公受礼。”何大监不肯,说道:“使不的。”西门庆道:“学生与天泉同寅晚辈,老公公齿德俱尊,又系中贵,自然该受礼。”讲了半日,何大监受了半礼,让西门庆上坐,他主席相陪,何千户旁坐。西门庆道:“老公公,这个断然使不得。同僚之间,岂可旁坐!老公公叔侄便罢了,学生使不的。”何太监大喜道:“大人甚是知礼,罢罢,我阁老位儿旁坐罢,教做官的陪大人就是了。”西门庆道:“这等,学生坐的也安。”于是各照位坐下。何太监道:“小的儿们,再烧了炭来。今日天气甚是寒冷。”须臾,左右火池火叉,拿上一包水磨细炭,向火盆内只一倒。厅前放下油纸暖帘来,日光掩映,十分明亮。何太监道:“大人请宽了盛服罢。”西门庆道:“学生里边没穿甚么衣服,使小价下处取来。”何太监道:“不消取去。”令左右接了衣服,“拿我穿的飞鱼绿绒氅衣来,与大人披上。”西门庆笑道:“老先生职事之服,学生何以穿得?”何太监道:“大人只顾穿,怕怎的!昨日万岁赐了我蟒衣,我也不穿他了,就送了大人遮衣服儿罢。”不一时,左右取上来,西门庆令玳安接去员领,披上氅衣,作揖谢了。又请何千户也宽去上盖陪坐。

又拿上一道茶来吃了,何太监道:“叫小厮们来。”原来家中教了十二名吹打的小厮,两个师范领着上来磕头。何太监就吩咐动起乐来,然后递酒上坐。何太监亲自把盏,西门庆慌道:“老公公请尊便。有长官代劳,只安放钟箸儿就是一般。”何太监道:“我与大人递一钟儿。我家做官的初入芦苇,不知深浅,望乞大人凡事扶持一二,就是情了。”西门庆道:“老公公说那里话!常言:同僚三世亲。学生亦托赖老公公余光,岂不同力相助!”何太监道:“好说,好说。共同王事,彼此扶持。”西门庆也没等他递酒,只接了杯儿,领到席上,随即回奉一杯,安在何千户并何太监席上,彼此告揖过,坐下。吹打毕,三个小厮连师范,在筵前银筝象板,三弦琵琶,唱了一套《正宫·端正好》“雪夜访赵普”、“水晶宫鲛绡帐”。唱毕下去。

酒过数巡,食割两道,看看天晚,秉上灯来。西门庆唤玳安拿赏赐与厨役并吹打各色人役,就起身,说道:“学生厚扰一日了,就此告回。”那公公那里肯放,说道:“我今日正下班,要与大人请教。有甚大酒席,只是清坐而已,教大人受饥。”西门庆道:“承老公公赐这等美馔,如何反言受饥!学生回去歇息歇息,明早还要与天泉参谒参谒兵科,好领札付挂号。”何太监道:“既是大人要与我家做官的同干事,何不令人把行李搬过来我家住两日?我这后园儿里有几间小房儿,甚是僻静,就早晚和做官的理会些公事儿也方便些,强如在别人家。”西门庆道:“在这里最好,只是使夏公见怪,相学生疏他一般。”何太监道:“没的说。如今时年,早晨不做官,晚夕不唱喏,衙门是恁偶戏衙门。虽故当初与他同僚,今日前官已去,后官接管承行,与他就无干。他若这等说,他就是个不知道理的人了。今日我定要和大人坐一夜,不放大人去。”唤左右:“下边房里快放桌儿,管待你西门老爹大官儿饭酒。我家差几个人,跟他即时把行李都搬了来。”又吩咐:“打扫后花园西院干净,预备铺陈,炕中笼下炭火。”堂上一呼,阶下百诺,答应下去了。西门庆道:“老公公盛情,只是学生得罪夏公了。”何太监道:“他既出了衙门,不在其位,不谋其政。他管他那銮驾库的事,管不的咱提刑所的事了。难怪于你。”不由分说,就打发玳安并马上人吃了酒饭,差了几名军牢,各拿绳扛,迳往崔中书家搬取行李去了。

何太监道:“又一件相烦大人:我家做官的到任所,还望大人替他看所宅舍儿,好搬取家小。今先教他同大人去,待寻下宅子,然后打发家小起身。也不多,连几房家人也只有二三十口。”西门庆道:“老公公吩咐,要看多少银子宅舍?”何太监道:“也得千金外房儿才够住。”西门庆道:“夏龙溪他京任不去了,他一所房子倒要打发,老公公何不要了与天泉住,一举两得其便。此宅门面七间,到底五层,仪门进去大厅,两边厢房,鹿角顶,后边住房、花亭,周围群房也有许多,街道又宽阔,正好天泉住。”何太监道:“他要许多价值儿?”西门庆道:“他对我说原是一千三百两,又后边添盖了一层平房,收拾了一处花亭。老公公若要,随公公与他多少罢了。”何太监道:“我托大人,随大人主张就是了。趁今日我在家,差个人和他说去,讨他那原文书我瞧瞧。难得寻下这房舍儿,我家做官的去到那里,就有个归着了。”

不一时,只见玳安同众人搬了行李来回话。西门庆问:“贲四、王经来了不曾?”玳安道:“王经同押了衣箱行李先来了。还有轿子,叫贲四在那里看守着哩。”西门庆因附耳低言:“如此这般上覆夏老爹,借过那里房子的原契来,何公公要瞧瞧。就同贲四一答儿来。”这玳安应的去了。不一时,贲四青衣小帽,同玳安拿文书回西门庆说:“夏老爹多多上覆:既是何公公要,怎好说价钱!原文书都拿的来了。又收拾添盖,使费了许多,随爹主张了罢。”西门庆把原契递与何太监亲看了一遍,见上面写着一千二百两,说道:“这房儿想必也住了几年,未免有些糟烂,也别要说收拾,大人面上还与他原价。”那贲四连忙跪下说:“何爷说的是。自古道:使的憨钱,治的庄田。千年房舍换百主,一番拆洗一番新。”何太监听了喜欢道:“你是那里人?倒会说话儿。常言成大事者不惜小费,其实说的是。他教甚么名字?”西门庆道:“他名唤贲四。”何太监道:“也罢,没个中人儿,你就做个中人儿,替我讨了文书来。今日是个好日期,就把银子兑与他罢。”西门庆道:“如今晚了,待的明日也罢了。”何太监道:“到五更我早进去,明日大朝。今日不如先交与他银子,就了事。”西门庆问道:“明日甚时驾出?”何太监道:“子时驾出到坛,三更鼓祭了,寅正一刻就回宫。摆了膳,就出来设朝,升大殿,朝贺天下,诸司都上表拜冬。次日,文武百官吃庆成宴。你每是外任官,大朝引奏过就没事了。”说毕,何太监吩咐何千户进后边,打点出二十四锭大元宝来,用食盒抬着,差了两个家人,同贲四、玳安押送到崔中书家交割。夏公见抬了银子来,满心欢喜,随即亲手写了文契,付与贲四等,拿来递上。何太监不胜欢喜,赏了贲四十两银子,玳安、王经每人三两。西门庆道:“小孩子家,不当赏他。”何太监道:“胡乱与他买嘴儿吃。”三人磕头谢了。何太监吩咐管待酒饭,又向西门庆唱了两个喏:“全仗大人余光。”西门庆道:“还是看老公公金面。”何太监道:“还望大人对他说说,早把房儿腾出来,就好打发家小起身。”西门庆道:“学生一定与他说,教他早腾。长官这一去,且在衙门公廨中权住几日。待他家小搬到京,收拾了,长官宝眷起身不迟。”何太监道:“收拾直待过年罢了,先打发家小去才好。十分在衙门中也不方便。”

说话之间,已有一更天气,西门庆说道:“老公公请安置罢!学生亦不胜酒力了。”何大监方作辞归后边歇息去了。何千户教家乐弹唱,还与西门庆吃了一回,方才起身,送至后园。三间书院,台榭湖山,盆景花木,房内绛烛高烧,篆内香焚麝饼,十分幽雅。何千户陪西门庆叙话,又看茶吃了,方道安置,归后边去了。

西门庆摘去冠带,解衣就寝。王经、玳安打发了,就往下边暖炕上歇去了。西门庆有酒的人,睡在枕畔,见满窗月色,翻来复去。良久只闻夜漏沉沉,花阴寂寂,寒风吹得那窗纸有声,况离家已久。正要呼王经进来陪他睡,忽听得窗外有妇人语声甚低,即披衣下床,靸着鞋袜,悄悄启户视之。只见李瓶儿雾鬓云鬟,淡妆丽雅,素白旧衫笼雪体,淡黄软袜衬弓鞋,轻移莲步,立于月下。西门庆一见,挽之入室,相抱而哭,说道:“冤家,你如何在这里?”李瓶儿道:“奴寻访至此。对你说,我已寻了房儿了,今特来见你一面,早晚便搬去了。”西门庆忙问道:“你房儿在于何处?”李瓶儿道:“咫尺不远。出此大街迤东,造釜巷中间便是。”言讫,西门庆共他相偎相抱,上床云雨,不胜美快之极。已而整衣扶髻,徘徊不舍。李瓶儿叮咛嘱咐西门庆道:“我的哥哥,切记休贪夜饮,早早回家。那厮不时伺害于你,千万勿忘!”言讫,挽西门庆相送。走出大街上,见月色如昼,果然往东转过牌坊,到一小巷,见一座双扇白板门,指道:“此奴之家也。”言毕,顿袖而入。西门庆急向前拉之,恍然惊觉,乃是南柯一梦。但见月影横窗,花枝倒影矣。西门庆向褥底摸了摸,见精流满席,余香在被,残唾犹甜。追悼莫及,悲不自胜。正是:

\[
玉宇微茫霜满襟,疏窗淡月梦魂惊。
凄凉睡到无聊处,恨杀寒鸡不肯鸣。
\]

西门庆梦醒睡不着,巴不得天亮。比及天亮,又睡着了。次日早,何千户家童仆起来伺候,打发西门庆梳洗毕,何千户又早出来陪侍,吃了姜茶,放桌儿请吃粥。西门庆问:“老公公怎的不见?”何千户道:“家公公从五更就进内去了。”须臾拿上粥来。吃了粥,又拿上一盏肉圆子馄饨鸡蛋头脑汤。一面吃着,就吩咐备马。何千户与西门庆冠冕,仆从跟随,早进内参见兵科。出来,何千户便分路来家,西门庆又到相国寺拜智云长老。长老又留摆斋。西门庆只吃了一个点心,余者收与手下人吃了,就起身从东街穿过来,要往崔中书家拜夏龙溪去。因从造釜巷所过,中间果见有双扇白板门,与梦中所见一般。悄悄使玳安问隔壁卖豆腐老姬:“此家姓甚名谁?”老姬答道:“此袁指挥家也。”西门庆于是不胜叹异。到了崔中书家,夏公才待出门拜人,见西门庆到,忙令左右把马牵过,迎至厅上,拜揖叙礼。西门庆令玳安拿上贺礼:青织金绫紵一端、色缎一端。夏公道:“学生还不曾拜贺长官,到承长官先施。昨日小房又烦费心,感谢不尽。”西门庆道:“昨日何太监说起看房,我因堂尊分上,就说此房来。何公讨了房契去看了,一口就还原价。果是内臣性儿,立马盖桥就成了。还是堂尊大福!”说毕,二人笑了。夏公道:“何天泉,我也还未回拜他。”因问:“他此去与长官同行罢了。”西门庆道:“他已会定同学生一路去,家小且待后。昨日他老公公多致意,烦堂尊早些把房儿腾出来,搬取家眷。他如今权在衙门里住几日罢了。”夏公道:“学生也不肯久稽,待这里寻了房儿,就使人搬取家小。也只待出月罢了。”说毕,西门庆起身,又留了个拜帖与崔中书,夏公送出上马,归至何千户家。何千户又早有午饭等候。西门庆悉把拜夏公之事说了一遍:“腾房已在出月。”何千户大喜,谢道:“足见长官盛情。”

吃毕饭,二人正在厅上着棋,忽左右来报:“府里翟爹差人送下程来了。抓寻到崔老爹那里,崔老爹使他这里来了。”于是拿帖看,上写着:“谨具金缎一端、云紵一端、鲜猪一口、北羊一腔、内酒一坛、点心二盒。眷生翟谦顿首拜。”西门庆见来人,说道:“又蒙你翟爹费心。”一面收了礼物,写回帖,赏来人二两银子,抬盒人五钱,说道:“客中不便,有亵管家。”那人磕头收了。王经在旁悄悄说:“小的姐姐说,教我府里去看看爱姐,有物事捎与他。”西门庆问:“甚物事?”王经道:“是家中做的两双鞋脚手。”西门庆道:“单单儿怎好拿去?”吩咐玳安:“我皮箱内有带的玫瑰花饼,取两罐儿。”就把口帖付与王经,穿上青衣,跟了来人往府里看爱姐不题。这西门庆写了帖儿,送了一腔羊、一坛酒谢了崔中书,把一口猪、一坛酒、两盒点心抬到后边孝顺老公公。何千户拜谢道:“长官,你我一家,如何这等计较!”

且说王经到府内,请出韩爱姐,外厅拜见了。打扮的如琼林玉树一般,比在家出落自是不同,长大了好些。问了回家中事务,管待了酒饭,见王经身上单薄,与了一件天青紵丝貂鼠氅衣儿,又与了五两银子,拿来回覆西门庆话。西门庆大喜。正与何千户下棋,忽闻绰道之声,门上人来报:“夏老爹来拜,拿进两个拜帖儿。”两个忙迎接到厅叙礼,何千户又谢昨日房子之事。夏公具了两分缎帕酒礼,奉贺二公。西门庆与何千户再三致谢,令左右收了。夏公又赏了贲四、玳安、王经十两银子,一面分宾主坐下。茶罢,共叙寒温。夏公道:“请老公公拜见。”何千户道:“家公公进内去了。”夏公又留下了一个双红拜帖儿,说道:“多顶上老公公,拜迟,恕罪!”言毕,起身去了。何千户随即也具一分贺礼,一匹金缎,差人送去,不在言表。

到晚夕,何千户又在花园暖阁中摆酒与西门庆共酌,家乐歌唱,到二更方寝。西门庆因昨日梦遗之事,晚夕令王经拿铺盖来书房地平上睡。半夜叫上床,搂在被窝内。两个口吐丁香,舌融甜唾。正是:

\[
不能得与莺莺会,且把红娘去解馋。
\]

一晚题过。到次日,起五更与何千户一行人跟随进朝。先到待漏院伺候,等的开了东华门进入。但见:

\[
星斗依稀禁漏残,禁中环佩响珊珊。
欲知今日天颜喜,遥睹蓬莱紫气皤。
\]

少顷,只听九重门启,鸣哕哕之鸾声;阊阖天开,睹巍巍之衮冕。当时天子祀毕南郊回来,文武百官聚集,等候设朝。须臾钟响,天子驾出大殿,受百官朝贺。须臾,香球拨转,帘卷扇开。正是:

\[
晴日明开青锁闼,天风吹下御炉香。
千条瑞霭浮金阙,一朵红云捧玉皇。
\]

这皇帝生得尧眉舜目,禹背汤肩,才俊过人,口工诗韵,善写墨君竹,能挥薛稷书,通三教之书,晓九流之典。朝欢暮乐,依稀似剑阁孟商王;爱色贪花,仿佛如金陵陈后主。当下驾坐宝位,静鞭响罢,文武百官秉简当胸,向丹墀五拜三叩头,进上表章。已而有殿头官口传圣旨道:“朕今即位二十祀矣。艮岳于兹告成,上天降瑞,今值覆端之庆,与卿共之。”言未毕,班首中闪过一员大臣来,朝靴踏地响,袍袖列风生。视之,乃左丞相崇政殿大学士兼吏部尚书太师鲁国公蔡京也。幞头象简,俯伏金阶,口称:“万岁,万岁,万万岁!臣等诚惶诚恐,稽首顿首,恭惟皇上御极二十祀以来,海宇清宁,天下丰稔,上天降鉴,祯祥叠见。三边永息兵戈,万国来朝天阙。银岳排空,玉京挺秀。宝箓膺颁于昊阙,绛宵深耸于乾宫。臣等何幸,欣逢盛世,交际明良,永效华封之祝,常沾日月之光。不胜瞻天仰圣,激切屏营之至!谨献颂以闻。”良久,圣旨下来:“贤卿献颂,益见忠诚,朕心嘉悦。诏改明年为重和元年,正月元旦受定命宝,肄赦覃赏有差。”蔡大师承旨下来。殿头官口传圣旨:“有事出班早奏,无事卷帘退朝。”言未毕,见一人出离班部,倒笏躬身,绯袍象简,玉带金鱼,跪在金阶,口称:“光禄大夫掌金吾卫事太尉太保兼太子太保臣朱勔,引天下提刑官员章隆等二十六员,例该考察,已更改补、缴换札付,合当引奏。未敢擅便,请旨定夺。”于是二十六员提刑官都跪在后面。不一时,圣旨传下来:“照例给领。”朱太尉承旨下来。天子袍袖一展,群臣皆散,驾即回宫。百官皆从端礼门两分而出。那十二象不待牵而先走,镇将长随纷纷而散。朝门外车马纵横,侍仗罗列。人喧呼,海沸波翻;马嘶喊,山崩地裂。众提刑官皆出朝上马,都来本衙门伺候。良久,只见知印拿了印牌来,传道:“老爷不进衙门了,已往蔡爷、李爷宅内拜冬去了。”以此众官都散了。

西门庆与何千户回到家中。又过了一夕,到次日,衙门中领了札付,又挂了号,又拜辞了翟管家,打点残装,收拾行李,与何千户一同起身。何太监晚夕置酒饯行,嘱咐何千户:“凡事请教西门大人,休要自专,差了礼数。”从十一月二十日东京起身,两家也有二十人跟随,竟往山东大道而来。已是数九严寒之际,点水滴冻之时,一路上见了些荒郊野路,枯木寒鸦。疏林淡日影斜晖,暮雪冻云迷晚渡。一山未尽一山来,后村已过前村望。比及刚过黄河,到水关八角镇,骤然撞遇天起一阵大风。但见:

\[
非干虎啸,岂是龙吟?卒律律寒飙扑面,急飕飕冷气侵人。初时节无踪无影,次后来卷雾收云。吹花摆柳白茫茫,走石扬砂昏惨惨。刮得那大树连声吼,惊得那孤雁落深濠。须臾,砂石打地,尘土遮天。砂石打地,犹如满天骤雨即时来;尘土遮天,好似百万貔貅卷土至。这风大不大?真个是吹折地狱门前树,乱起酆都顶上尘;常娥急把蟾官闭,列子空中叫救人。险些儿玉皇住不得昆仑顶,只刮得大地乾坤上下摇。
\]
西门庆与何千户坐着两顶毡帏暖轿,被风刮得寸步难行。又见天色渐晚,恐深林中撞出小人来,西门庆吩咐手下:“快寻那里安歇一夜,明日风住再行罢。”抓寻了半日,远远望见路旁一座古刹,数株疏柳,半堵横墙。但见:

\[
石砌碑横梦草遮,回廊古殿半欹斜。
夜深宿客无灯火,月落安禅更可嗟。
\]
西门庆与何千户忙入寺中投宿,上题着“黄龙寺”。见方丈内几个僧人在那里坐禅,又无灯火,房舍都毁坏,半用篱遮。长老出来问讯,旋吹火煮茶,伐草根喂马。煮出茶来,西门庆行囊中带得干鸡腊肉果饼之类,晚夕与何千户胡乱食得一顿。长老爨一锅豆粥吃了,过得一宿。次日风止天晴,与了和尚一两银子相谢,作辞起身往山东来。正是:

\[
王事驱驰岂惮劳,关山迢递赴京朝。
夜投古寺无烟火,解使行人心内焦。
\]
