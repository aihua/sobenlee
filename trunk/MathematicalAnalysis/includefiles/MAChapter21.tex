%# -*- coding:utf-8 -*-
%%%%%%%%%%%%%%%%%%%%%%%%%%%%%%%%%%%%%%%%%%%%%%%%%%%%%%%%%%%%%%%%%%%%%%%%%%%%%%%%%%%%%
%%  MAChapter21.tex'


\chapter{曲线积分}\label{ch:21}
\section{与曲线有关的一些概念}
\subsection{弧段的直径、弦长与对应的参数值}
\subsection{曲线的定向}
\subsection{可求长曲线}
\begin{exercise}

\end{exercise}
\section{第一型曲线积分}
\subsection{第一型曲线积分概念}
\subsection{第一型曲线积分化为定积分}
\subsection{第一型曲线积分在力学上的应用}
\begin{exercise}

\end{exercise}
\section{第二型曲线积分}
\subsection{第二型曲线积分概念}
\subsection{第二型曲线积分的存在与计算}
\subsection{用折线上的积分逼近曲线上的积分}
\subsection{第一、二型曲线积分的联系}
\subsection{第二型曲线积分的应用}
\begin{exercise}

\end{exercise}
\section{平面上的第二型曲线积分与~Green~公式}
\subsection{平面闭曲线的定向}
\subsection{Green~公式}
\subsection{Green~公式的若干应用与例子}
\subsection{平面上的分部积分公式与~Green~第一、第二公式}
\subsection{正则变换下闭曲线定向的变化}
\begin{exercise}

\end{exercise}
\begin{exercise*}

\end{exercise*}




\endinput
%%
%% End of file `MAChapter21.tex'.