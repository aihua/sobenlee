%# -*- coding:utf-8 -*-
%%%%%%%%%%%%%%%%%%%%%%%%%%%%%%%%%%%%%%%%%%%%%%%%%%%%%%%%%%%%%%%%%%%%%%%%%%%%%%%%%%%%%
%%  MAChapter2.tex'


\chapter{极\emspace 限}\label{ch:2}

高等数学与初等数学的差别,除了研究对象不同外,主要是研究方法上的不同。我们知道,初等数学的方法建立在有限观念上,而
高等数学的方法则是建立在无限观念之上。比如初等数学中要求一个数,通过有限步的代数运算,即可求出它的准确值。但在客观
上存在着这样一种数,若只进行有限步的代数运算,则无法求得其准确的值。例如圆的面积和周长,用有限步代数运算就不能求得
其准确值,必须通过无限步逼近,即所谓极限方法,才能求出它的准确值,这就是高等数学的方法。又比如初等数学要确定数的性
质(如是否是素数),理论上通过有限步运算,就能断定它是否具有这一性质;而高等数学要确定函数的性质,就要通过极限方法
才能确定它是否具有此性质。

所以理解极限概念、掌握极限方法,是能否学好高等数学的关键。只有掌握极限这把钥匙,才能打开通向微积分的大门,变门外汉
为驾驭微积分工具的主人。


\section{序列极限的定义}

\subsection{概念引入}

试求由抛物线~$y=x^2$、$x$~轴、直线~$x=1$~所围成的曲边三角形的面积。

\begin{wrapfigure}[9]{O}{0mm}
\somefigure
\caption{}\label{fig:sec2.1-1}
\end{wrapfigure}

我们会求直边形的面积,不会求曲边形的面积。但是我们可以分两步来求出曲边形面积:先是通过以直代曲,得到一系列越来越逼近曲边三
角形面积的近似值;然后考察这一系列近似值的变化趋势,从而确定出曲边三角形面积的准确值。为此,如\ref{fig:sec2.1-1}~所示,用分点
\[
  x=0,\enspace\frac1n,\enspace\frac2n,\dotsc\dotsc,\frac{n-1}n,1,
\]
把区间~$[0,1]$~分成~$n$~个小区间,相应的把曲边三角形分成~$n$~个细长的小曲边梯形。每一个小曲边梯形,用具有同底和左端点函数值
为高的矩形近似代替,因而得到曲边三角形面积的近似值
\begin{align*}
a\approx x_n&=\sum_{i=0}^{n-1}\mparenB{\frac in}^2\cdot\frac1n
 =\frac1{n^3}\sum_{i=0}^{n-1}i^2=\frac1{n^3}\frac{n(n-1)(2n-1)}6\\
&=\frac16\mparenB{1-\frac1n}\mparenB{2-\frac1n}。\footnotemark
\end{align*}
\footnotetext{利用\begin{align*}
\sum_{k=1}^mk^2
& =\sum_{k=1}^mk(k-1)+\sum_{k=1}^mk
  =\frac13\sum_{k=1}^{m+1}\mparenb{k(k-1)(k-2)-(k-1)(k-2)(k-3)}+\sum_{k=1}^mk\\
& =\frac13(m+1)m(m-1)+\frac12m(m+1)=\frac16m(m+1)(2m+1)。
\end{align*}}%
其中~$x_n$~表示图中阶梯形的面积,$n$~越大阶梯越多,近似程度也就越高,但不管~$n$~多大总是近似的。要想从曲边三角形面
积~$a$~的这一系列近似值~$x_n$~得到准确值~$a$,就必须考察~$n$~趋于无穷的过程。$n$~无限增大时,阶梯形面积~$x_n$~无限
逼近于所求的面积~$a$。令~$\ntoinf$,得到
\[
  x_n=\frac16\mparenB{1-\frac1n}\mparenB{2-\frac1n}\to\frac13。
\]
即通过考察~$n$~趋于无穷时,一系列近似值~$x_n$~的变化趋势,确定出曲边三角形的面积~$a=\sfrac13$。

在这个例子中,我们求出了值~$a=\sfrac13$,但我们目的不只是为了求~$a$~的值,也不只是关心如何以直代曲得到近似值~$x_n$~或~$x_n$~的
具体表示形式,而是要从考察~$n$~趋于无穷时,由~$x_n$~的变化趋势确定出值~$a$~这一步中,抽象出序列极限的定义。

\subsection{序列极限定义}

一个接一个排起来的一串数叫作\emph{序列}。例如
\begin{equation}\label{eq:ch1sec2.1-1}
1,2,3,4,\dotsc; \quad
1,\frac12,\frac13,\frac14,\dotsc;\quad
1,\frac12,\frac1{2^2},\frac1{2^3},\dotsc;\quad
1,-1,1,-1,\dotsc
\end{equation}
都是序列。我们只写出序列的前几项,后面用三点来表示未写出无穷多项。一般用
\[
  x_1,x_2,\dotsc,x_n,\dotsc
\]
来表示序列,简单记作~$\mbrace{x_n}$,这里~$x_n$~称为序列的\emph{通项}。给定序列的通项公式,就可以写出序列的每一项;反之,给
定一个序列,即给出一个映射~$\map f\mN\mR$,一般来说可以归纳出通项公式。比如~\ref{eq:ch1sec2.1-1}~中第一个序列的通项公式
为~$x_n$;第二个序列为~$x_n=\sfrac1n$;第三个序列为~$x_n=\sfrac1{2^{n-1}}$;第四个序列为~$x_n=(-1)^{n-1}$。但序列
\[
  1.4,1.41,1.414,\dotsc
\]
的第~$n$~项是~$\sqrt2$~取小数点后~$n$~位的近似值。虽然通项的意义是明确的,但不能用明显的公式表示出来。

在求曲边三角形面积的例子中,我们遇到了通项为
\[
  x_n=\frac16\mparenB{1-\frac1n}\mparenB{2-\frac1n}
\]
的序列~$\mbrace{x_n}$,并看出~$n$~趋于无穷时,通项~$x_n$~趋于~$a=\sfrac13$。在其它问题中,也要考察~$n$~趋于无穷时,序列
变化的趋势。为此我们引出序列极限的定性描述:

给定序列~$\mbrace{x_n}$,当~$n$~无限增大时,若~$x_n$~无限的接近~$a$,则称~$a$~为~$n$~趋于无穷时序列的
极限,记作~$x_n\to a\mcond{\ntoinf}$,或
\[
  \lim_\ntoinf x_n=a。
\]

这种描述显然是很不够的,因为“无限增大”,“无限接近”并未给出确切的含义,只当作一般性用语来理解。要想给出极限的定义,必须
分析一下“无限增大”,“无限接近”的确切含义是什么。

再回到曲边三角形的例子。所谓~$n$~无限增大,$x_n$~无限接近于~$a$,是指阶梯形面积~$x_n$~与曲边三角形面积~$a$~的误差要多小就
能多小,只要~$n$~充分大。具体来说,给定误差~$\sfrac1{100}$,只要去~$n=100$,即把~$[0,1]$~区间~$100$~等分,每一个小曲边梯形
与矩形的误差加起来,就是~$x_n$~与~$a$~的误差。要估计小曲边梯形与矩形误差总和,只需如\ref{fig:sec2.1-1}~所示,将表示误差的
图形平移到以~$1$~为高,以~$\sfrac1{100}$~为底的该图左边的矩形上,显然误差总和不超过矩形面积的~$\sfrac1{100}$,所以
\[
  \mabs{x_{100}-a}<\frac1{100}。
\]
若给定误差~$\num{1/1000}$,则只要取~$n=\num{1000}$,就有
\[
  \mabs{x_{1000}-a}<\frac1{\num{1000}}。
\]
由此可见,$x_n$~与~$a$~的误差要多小就能多小,只要~$x_n$~充分大。用希腊字母~$\e$~表示误差,我们先试着给出极限的定义如下:

若给定~$\e>0$,总能找到自然数~$n$,使得
\[
  \mabs{x_n-a}<\e,
\]
则称序列~$x_n$~的极限是~$a$,或~$a$~是序列的极限。

这个定义是否妥当呢?从字面上看,它只告诉我们序列有一项~$x_n$~与~$a$~的误差小于~$\e$,这项以后的各项与~$a$~的误差是
否小于~$\e$~呢?定义没有说,如果这项以后的项与~$a$~误差很大,则不能反映出~$x_n$~无限接近~$a$,所以定义应修改如下:

如果给定~$\e>0$,存在~$N$,使得当~$n>N$~时,就有
\[
  \mabs{x_n-a}<\e,
\]
则称~$x_n$~的极限时~$a$。

这样修改后,保证自某一项以后各项与~$a$~的误差都小于~$\e$,究竟是否妥当了呢?可能想法是对的,但表达还有问题。给定~$\e>0$,只是
说对一个误差~$\e$~能找到~$N$,对别的误差能否找到~$N$~呢?定义反映不出来。如果对于更小的误差~$\e_1$,不能保证序列自某一项以后
各项与~$a$~的误差小于~$\e_1$,也不能反映出~$x_n$~无效接近与~$a$。因此,我们就得到了严格的极限定义。

\begin{definition}\label{def:sec2.1-1}
如果对于任意~$\e>0$,都存在非负整数~$N$,使得当~$n>N$~时,就有
\[
  \mabs{x_n-1}<\e,
\]
则称序列~$\mbrace{x_n}$~的\emph{极限}为~$a$,记作
\[
  \lim_\ntoinf x_n=a,
\]
或者~$x_n\to a\mcond{\ntoinf}$。
\end{definition}

极限的几何意义为:作以~$a$~为中心,以~$\e$~为半径的邻域~$(a-\e,a+\e)$,当~$n>N$~时,有
\[
  a-\e<x_n<a+\e,
\]
即随着~$n$~的变化,观察序列~$x_n$~在数轴上的变化时,发现开头有限项可能落在邻域外面,或一会儿落在邻域里边,一会儿又落在
邻域外边,但当标号大于~$N$~后,序列的项总在邻域里变动,再也不会落在邻域外边。从静态来看,以~$a$~为中心,以任意~$\e$~为
半径作一邻域,若序列在邻域外边只有有限项,则序列的极限是~$a$。

\begin{example}
证明~$\lim_\ntoinf\dfrac1n=0$。
\end{example}
\begin{proof}
对于任意~$\e>0$,则
\[
  \mabsbb{\frac1n}=\frac1n<\e\impliedby n>\frac1\e。
\]
因此,取~$N=\mfloorbb{\dfrac1\e}$,则当~$n>N$~时,就有~$\mabsbb{\dfrac1n}<\e$,即
\[
  \lim_\ntoinf\frac1n=0。\qedhere
\]
\end{proof}

这个例子说明~$N$~是依赖于~$\e$,$\e$~越小,$N$~越大。具体找~$N$~时,可以用分析法,即要使结论成立,看~$n$~应多大。

\begin{example}
证明~$\lim_\ntoinf q^n=0\mcond{0<q<1}$。
\end{example}
\begin{proof}
对于任意~$\e>0$,不妨假定~$\e<1$,则
\[
  \mabs{q^n}=q^n<\e\impliedby n\ln q<\ln\e\impliedby n>\frac{\ln\e}{\ln q}。
\]
因此,取~$N=\mfloorbb{\dfrac{\ln\e}{\ln q}}$,则当~$n>N$~时,就有~$\mabs{q^n}<\e$,即
\[
  \lim_\ntoinf q^n=0。\qedhere
\]
\end{proof}

\begin{example}
证明~$\lim_\ntoinf\dfrac{a^n}{n!}\mcond{a>1}$。
\end{example}
\begin{proof}
因为
\[
  \frac{a^n}{n!}=\frac a1\cdot\frac a2\dotsm\frac a{\mfloor{a}}\cdot\frac a{\mfloor a+1}\dotsm \frac an
  <\frac{a^{\mfloor a}}{\mfloor a!}\cdot\frac an\eqcolon c\cdot \frac an,
\]
所以对于任意~$\e>0$,有
\[
  \mabsbb{\frac{a^n}{n!}}=\frac{a^n}{n!}<\e\impliedby \frac{ca}n<\e\impliedby n>\frac{ca}\e。
\]
取~$N=\mfloorbb{\dfrac{ca}\e}$,则当~$n>N$~时,就有~$\mabsbb{\dfrac{a^n}{n!}}<\e$,即
\[
  \lim_\ntoinf\dfrac{a^n}{n!}\mcond*{a>1}。\qedhere
\]
\end{proof}

在极限定义中,我们关心的不是~$N$~的具体值,而是~$N$~的存在性。所以做题时,不一定直接去解不等式,可以用适当放大法来
找~$N$。例如从不等式
\[
  \frac {a^n}{n!}<\e
\]
出发解~$n$~大于什么很困难,而放大后求出~$N$~就很容易。要注意,运用放大法时,不要把含有变数~$n$~的因子移到不等式的右端,只
允许将左端逐步放大。还应注意反推法书写时,是由后面不等式成立,推出前面不等式成立;而不是前面不等式成立,推后面不等式成立。

\begin{example}
证明~$\lim_\ntoinf\sqrt[n]a=1\mcond{a>1}$。
\end{example}
\begin{proof*}
对于任意~$\e>0$,有
\begin{align*}
\mabsb{\sqrt[n]a-1}=\sqrt[n]a-1<\e
&\impliedby\sqrt[n]a<1+\e\impliedby\frac1n\ln a<\ln(1+\e)\\
&\impliedby n>\frac{\ln a}{\ln(1+\e)}。
\end{align*}
因此取~$N=\mfloorbb{\dfrac{\ln a}{\ln(1+\e)}}$,则当~$n>N$~时,就有~$\mabs{\sqrt[n]a-1}<\e$,即
\[
  \lim_\ntoinf\sqrt[n]a=1\mcond*{a>1}。\qedhere
\]
\end{proof*}
\begin{proof*}
令~$\sqrt[n]a-1=h_n$,为了对~$h_n$~用适当放大法,我们先估计~$h_n$。因为
\[
  a=(1+h_n)^n=1+nh_n+\frac{n(n-1)}{2!}h_n^2+\dotsb+h_n^n>nh_n,
\]
所以
\[
  0<h_n<\frac an。
\]
因此,对任意~$\e>0$,
\[
  \mabs{\sqrt[n]a-1}=h_n<\e\impliedby \frac an<\e。
\]
取~$N=\mfloorbb{\dfrac an}$,则当~$n>N$~时,就有~$\mabs{\sqrt[n]a-1}<\e$。
\end{proof*}

\begin{exercise}
\item 用~$\e-N$~方法验证
\[
  \lim_\ntoinf\frac1{n^2-1}=0,
\]
并分别对~$\e=0.1,0.01$~确定相应的~$N$。
\item 用~$\e-N$~方法验证下列极限为零。
\begin{exlistcols}[3]
\item $\lim_\ntoinf\dfrac1{n^2+n}$;
\item $\lim_\ntoinf\dfrac1{n^4-n}$;
\item $\lim_\ntoinf\dfrac{\sqrt[3]{n^2}}{n-3}$;
\item $\lim_\ntoinf\mparenb{\sqrt{n+1}-\sqrt n}$;
\item $\lim_\ntoinf\dfrac{10^n}{n!}$;
\item $\lim_\ntoinf\dfrac{n!}{n^n}$;
\item $\lim_\ntoinf\dfrac n{a^n}\mcond{a>1}$;
\item $\lim_\ntoinf\mparenb{\ln(n+1)-\ln n}$。
\end{exlistcols}
\item 若~$\lim_\ntoinf x_n=a$,证明~$\lim_\ntoinf\mabs{x_n}=\mabs a$。
\item 设~$x_n>0\mcond{n\geq 1}$,且~$\lim_\ntoinf x_n=a$。证明~$\lim_\ntoinf\sqrt{x_n}=\sqrt a$。
\item 设~$\lim_\ntoinf x_n=a$,证明~$\lim_\ntoinf\sqrt[3]{x_n}=\sqrt[3]a$。
\item 设~$x_n\leq a\leq y_n\mcond{n\geq 1}$,且~$\lim_\ntoinf(y_n-x_n)=0$。证明~$\lim_\ntoinf x_n=\lim_\ntoinf y_n=a$。
\item 设~$n,p$~均为正整数。证明,
\begin{exlistcols}[2]
\item $n^p<\dfrac{(n+1)^{p+1}-n^p}{p+1}<(n+1)^p$;
\item $\sum_{k=1}^{n-1}k^p<\dfrac{n^{p+1}}{p+1}<\sum_{k=1}^nk^p$;
\item 求极限~$\lim_\ntoinf\dfrac1n\sum_{k=1}^n\mparenB{\dfrac kn}^p$。
\end{exlistcols}
\item 设~$\lim_\ntoinf x_n=a$,而~$\ell$~为确定的自然数。证明,$\lim_\ntoinf x_{n+\ell}=a$。试讨论相反的情况。
\item\begin{exlist}
  \item 证明~$\lim_\ntoinf x_n=a$~当且仅当~$\lim_\ntoinf x_{2n}=\lim_\ntoinf x_{2n+1}=a$;
  \item 已知~$\lim_\ntoinf x_{2n}$~和$\lim_\ntoinf x_{2n+1}$~都存在,试讨论~$\lim_\ntoinf x_n$~的存在性。
\end{exlist}
\item 在序列极限定义中,对于~$N$~请说明下列问题。
\begin{exlistcols}[2]
  \item $N$~是否唯一;
  \item $N$~是否是~$\e$~的函数;
  \item 前~$N$~项是否有~$\mabs{x_n-a}\geq\e$。
\end{exlistcols}
\item 判断并说明序列极限定义改成下面形式是否可以。
\begin{exlist}
  \item 对任意~$\e>0$,存在~$N>0$,使得当~$n\geq N$~时,有~$\mabs{x_n-a}<\e$;
  \item 对任意~$\e>0$,存在~$N>0$,使得当~$n> N$~时,有~$\mabs{x_n-a}\leq\e$;
  \item 对任意~$\e>0$,存在~$N>0$,使得当~$n> N$~时,有~$\mabs{x_n-a}<M\e$,这里~$M$~为某固定常数。
\end{exlist}
\item 若对任意~$N>0$,存在~$\e>0$,使得当~$n>N$~时,有~$\mabs{x_n-a}<\e$,则序列~$\mbrace{x_n}$~具有什么性质?
\item 若存在~$N>0$,对任意~$\e>0$,使得当~$n>N$~时,有~$\mabs{x_n-a}<\e$,则序列~$\mbrace{x_n}$~具有什么性质?
\item 判断并说明下述对~$\lim_\ntoinf\sqrt[n]n=1$~的证明的正确性。
\begin{exproof}
对任意~$\e>0$,有
\[
  \sqrt[n]n<1+\e\impliedby\frac1n\ln n<\ln(1+\e)\impliedby
  \frac1n<\frac{\ln(1+\e)}{\ln n}\leq\frac{\ln(1+\e)}{\ln 2}。
\]
取~$N=\mfloorbb{\dfrac{\ln 2}{\ln(1+\e)}}$,则当~$n>N$~时,有
\[
  1-\e<1<\sqrt[n]n<1+\e,
\]
由此即可得到~$\lim_\ntoinf\sqrt[n]n=1$。
\end{exproof}
\end{exercise}


\section{序列极限的性质与运算}

有了极限定义后,自然会问极限有什么性质和怎么求极限。为此,我们要讨论极限的唯一性、有界性,极限的四则运算和
极限不等式。

\begin{theorem}[唯一性]\label{thm:sec2.2-1}
若序列~$\mbrace{x_n}$~的极限存在,则极限值是唯一的。
\end{theorem}

这个定理保证了不管用什么方法求极限,得到的极限值应该是一样的。

\begin{proof}
我们用反证法。假设序列极限不唯一,至少有两个不相等的极限值,设为
\[
  \lim_\ntoinf x_n=a,\quad \lim_\ntoinf x_n=b,
\]
且~$a\neq b$。不妨设~$a<b$,取~$\e=\dfrac{b-a}2>0$,由极限的定义,存在~$N_1$,使得当~$n>N_1$~时,有
\[
  \mabs{x_n-a}<\e\implies x_n<a+\e=\frac{a+b}2。
\]
并且存在~$N_2$,使得当~$n>N_2$~时,有
\[
  \mabs{x_n-b}<\e\implies x_n>a-\e=\frac{a+b}2。
\]
所以当~$n>\max\mbrace{N_1,N_2}$~时,就有
\[
  x_n>\frac{a+b}2>x_n。
\]
显然,这是一个矛盾(见\ref{fig:sec2.1-2})。因此极限值确实是唯一的。
\end{proof}

\begin{wrapfigure}{O}{0mm}
\somefigure
\caption{}\label{fig:sec2.1-2}
\end{wrapfigure}

从函数观点来看,序列就是定义在自然数集合上的函数~$x_n=f(n)$,由函数有界定义,可以得到序列有界的定义。

若存在~$M>0$,使得对任意~$n$,都有~$\mabs{x_n}\leq M$,则称序列~$\mbrace{x_n}$~\emph{有界}。

\begin{theorem}[有界性]\label{thm:sec2.2-2}
若序列~$\mbrace{x_n}$~有极限,则~$\mbrace{x_n}$~有界。
\end{theorem}
\begin{proof}
设~$\lim_\ntoinf x_n=a$,取~$\e_0=1$,则存在~$N$,使得当~$n>N$~时,有~$\mabs{x_n-a}<1$,那么
\[
  \mabs{x_n}-\mabs a\leq\mabs{x_n-a}<1\implies\mabs{x_n}<\mabs a+1。
\]
令
\[
  M=\max\mbraceb{1+\mabs a,\mabs{x_1},\mabs{x_2},\dotsc,\mabs{x_N}},
\]
则对任意~$n$,均有
\[
  \mabs{x_n}\leq M,
\]
即序列~$\mbrace{x_n}$~确实是有界的。
\end{proof}

在证明时必须分清什么时候取定~$\e$,什么时候任意给定~$\e$。上面证明中就必须取定~$\e$,不能任意给定~$\e$。如果
改为任意给定~$\e>0$,则~$N$~随~$\e$~在变,找到的~$M$~也随~$\e$~在变,这时的界~$M$~的意义就不明确,有界性就难以保证。

\begin{theorem}[极限的四则运算]\label{thm:sec2.2-3}
设~$\lim_\ntoinf x_n=a$,$\lim_\ntoinf y_n=b$,则
\begin{enumlistcols}
  \item $\lim_\ntoinf(x_n\pm y_n)=a\pm b$;\label{thm:sec2.2-3-item-1}
  \item $\lim_\ntoinf x_n\cdot y_n=a\cdot b$;\label{thm:sec2.2-3-item-2}
  \item 若~$b\neq0$,且~$y_n\neq0$,则~$\lim_\ntoinf\dfrac{x_n}{y_n}=\dfrac ab$。\label{thm:sec2.2-3-item-3}
\end{enumlistcols}
\end{theorem}
\begin{proof}\begin{thmenumlist}
\item 任给~$\e>0$,由~$\lim_\ntoinf x_n=a$,存在~$N_1$,使得当~$n>N_1$~时,有
\[
  \mabs{x_n-a}<\frac\e2;
\]
又由~$\lim_\ntoinf y_n=b$,存在~$N_2$,使得当~$n>N_2$~时,有
\[
  \mabs{y_n-b}<\frac\e2。
\]
取~$N=\max\mbrace{N_1,N_2}$,则当~$n>N$~时,就有
\[
  \mabsb{(x_n\pm y_n)-(a\pm b)}\leq\mabs{x_n-a}+\mabs{y_n-b}<\frac\e2+\frac\e2=\e,
\]
即
\[
  \lim_\ntoinf(x_n\pm y_n)=a\pm b。
\]
\item 要使得~$n>N$~时,$\mabs{x_n\cdot y_n-ab}<\e$,常用的方法是插入一项,这项的两个因子,一个因子与第一项的
因子相同,另一个因子与第二项的因子相同。这样的话,
\begin{align*}
\mabs{x_n\cdot y_n-ab}
&=\mabs{x_n\cdot y_n-y_n\cdot a+y_n\cdot a-ab}\\
&\leq\mabs{y_n}\cdot\mabs{x_n-a}+\mabs a\cdot\mabs{y_n-b},
\end{align*}
所以要使结论成立,只要~$n>N$~时,利用~$\mbrace{y_n}$~有界证明
\[
  \mabs{y_n}\cdot\mabs{x_n-a}<\frac\e2,\quad \mabs a\cdot\mabs{y_n-b}<\frac\e2
\]
即可。这样我们就得到了下面的证明。

由\ref{thm:sec2.2-2},存在~$M_1>0$,使得对任意~$n$,均有~$\mabs{y_n}\leq M_1$,令
\[
  M=\max\mrangeb{M_1}{\mabs a}>0。
\]

对任意~$\e>0$,由~$\lim_\ntoinf x_n=a$,存在~$N_1>0$,使得当~$n>N_1$~时,有
\[
  \mabs{x_n-a}<\frac\e{2M};
\]
又由~$\lim_\ntoinf y_n=b$,存在~$N_2>0$~,使得当~$n>N_2$~时,有
\[
  \mabs{y_n-b}<\frac\e{2M}。
\]
取~$N=\max\mbrace{N_1,N_2}$,则当~$n>N$~时,就有
\begin{align*}
\mabs{x_n\cdot y_n-ab}
&\leq\mabs{y_n}\cdot\mabs{x_n-a}+\mabs a\cdot\mabs{y_n-b}\leq M\mabs{x_n-a}+M\mabs{y_n-b}\\
&<M\cdot\frac\e{2M}+M\cdot\frac\e{2M}=\e,
\end{align*}
即
\[
  \lim_\ntoinf x_n\cdot y_n=a\cdot b。
\]
\item 由~\ref{thm:sec2.2-3-item-2}~可知,只要证明~$\lim_\ntoinf\dfrac1{y_n}=\lim_\ntoinf\dfrac1b$~即可。要使~$n>N$~时
\[
  \mabsbb{\frac1{y_n}-\frac1b}=\mabsbb{\frac{y_n-b}{y_n\cdot b}}<\e,
\]
就要先估计~$\mbrace{y_n}$~的下界。为此,对于~$\dfrac{\mabs b}2>0$,由~$\lim_\ntoinf y_n=b$,存在~$N_1$,使得当~$n>N_1$~时,有
\[
  \mabs b-\mabs{y_n}\leq\mabs{y_n-b}<\frac{\mabs b}2\implies\frac{\mabs b}2\leq\mabs{y_n}。
\]

对任意~$\e>0$,仍由~$\lim_\ntoinf y_n=b$,存在~$N_2$,使得当~$n>N_2$~时,有
\[
  \mabs{y_n-b}<\frac{\mabs b^2}2\e。
\]
取~$N=\max\mbrace{N_1,N_2}$,则当~$n>N$~时,就有
\[
  \mabsbb{\frac 1{y_n}-\frac 1b}=\mabsbb{\frac{y_n-b}{y_n\cdot b}}\leq\frac{2\mabs{y_n-b}}{\mabs b^2}
  <\frac2{\mabs b^2}\cdot\frac{\mabs b^2}2\e=\e,
\]
即
\[
  \lim_\ntoinf\frac1{y_n}=\frac1b。
\]
再由~~\ref{thm:sec2.2-3-item-2},
\[
  \lim_\ntoinf\frac{x_n}{y_n}=\lim_\ntoinf x_n\cdot\frac1{y_n}=a\cdot\frac1b=\frac ab。\qedhere
\]
\end{thmenumlist}
\end{proof}

\ref{thm:sec2.2-3}~的结论可以推广至有限个序列的情形。例如如果每个序列极限都存在,则有
\begin{Align*}
  \lim_\ntoinf\mparenb{x_n+y_n+z_n}&=\lim_\ntoinf x_n+\lim_\ntoinf y_n+\lim_\ntoinf z_n,\\
  \lim_\ntoinf\mparenb{x_n\cdot y_n\cdot z_n}&=\lim_\ntoinf x_n\cdot\lim_\ntoinf y_n\cdot\lim_\ntoinf z_n。
\end{Align*}

\begin{example}
求极限
\[
  \lim_\ntoinf\frac{4n^2-6n+1}{3n^2+n+7}。
\]
\end{example}
\begin{solution}
\[
\lim_\ntoinf\frac{4n^2-6n+1}{3n^2+n+7}=\lim_\ntoinf\frac{4-\dfrac6n+\dfrac1{n^2}}{3+\dfrac1n+\dfrac7{n^2}}
=\frac{\dps\lim_\ntoinf\mparenB{4-\frac6n+\frac1{n^2}}}{\dps\lim_\ntoinf\mparenB{3+\dfrac1n+\dfrac7{n^2}}}
=\dfrac43。\qedhere
\]
\end{solution}

\begin{example}
求极限~$\lim_\ntoinf\mparenb{1+a+\dotsb+a^n}\mcond{0<a<1}$。
\end{example}
\begin{solution}
\begin{align*}
\lim_\ntoinf\mparenb{1+a+\dotsb+a^n}
&=\lim_\ntoinf\frac{1-a^{n+1}}{1-a}=\lim_\ntoinf\frac1{1-a}-\lim_\ntoinf\frac{a^{n+1}}{1-a}\\
&=\frac1{1-a}。\qedhere
\end{align*}
\end{solution}

\begin{quiz}
\begin{thmenumlist}
\item 判断并说明下面推导是否正确。
\begin{align*}
\lim_\ntoinf\mparenbb{1+\frac1n}^{\msp n}
&=\lim_\ntoinf\mparenB{1+\frac1n}\cdot\lim_\ntoinf\mparenB{1+\frac1n}\dotsm
  \lim_\ntoinf\mparenB{1+\frac1n}\\
&=1\times1\times\dotsm\times1=1;
\end{align*}
\item 在证明\ref{thm:sec2.2-3}~的结论~\ref{thm:sec2.2-3-item-3}~时,如果令~$\lim_\ntoinf\dfrac{x_n}{y_n}=c$,则
\[
  a=\lim_\ntoinf x_n=\lim_\ntoinf\frac{x_n}{y_n}\cdot y_n=c\cdot b\implies c=\frac ab,
\]
判断并说明上述推导的正确性。
\end{thmenumlist}
\end{quiz}

\begin{theorem}\label{thm:sec2.2-4}
给定序列~$\mbrace{x_n}$~和~$\mbrace{y_n}$,若~$x_n\leq y_n\mcond{n\geq1}$,且
\[
  \lim_\ntoinf x_n=a,\quad \lim_\ntoinf y_n=b,
\]
则~$a\leq b$。
\end{theorem}
\begin{proof}
用反证法。不妨假定~$a>b$,取
\[
  \e=\frac{a-b}2>0,
\]
由~$\lim_\ntoinf x_n=a$,存在~$N_1$,使得当~$n>N_1$~时,
\[
  \mabs{x_n-a}<\e\implies x_n>a-\e=\frac{a+b}2;
\]
又由~$\lim_\ntoinf y_n=b$,存在~$N_2$,使得当~$n>N_2$~时,有
\[
  \mabs{y_n-b}<\e\implies y_n<b+\e=\frac{a+b}2。
\]
因此,当~$n>\max\mbrace{N_1,N_2}$~时,得到
\[
  x_n>\frac{a+b}2>y_n。
\]
这与条件矛盾(见\ref{fig:sec2.1-3})。所以~$a\leq b$。
\end{proof}

\fixwrapfloatsep
\begin{wrapfigure}[10]{O}{0mm}
\somefigure
\caption{}\label{fig:sec2.1-3}
\end{wrapfigure}

下面我们来讨论极限的所谓夹逼原理。

\begin{theorem}\label{thm:sec2.2-5}
给定序列~$\mbrace{x_n},\mbrace{y_n}$~和~$\mbrace{z_n}$,满足
\[
x_n\leq z_n\leq y_n\mcond*{n\geq1},
\]
且
\[
  \lim_\ntoinf x_n=a=\lim_\ntoinf y_n,
\]
则~$\lim_\ntoinf z_n=a$。
\end{theorem}
\begin{proof}
对任意~$\e>0$,由~$\lim_\ntoinf x_n=a$,存在~$N_1$,使得当~$n>N_1$~时,有
\[
  \mabs{x_n-a}<\e\implies a-\e<x_n<a+\e;
\]
又由~$\lim_\ntoinf y_n=a$,存在~$N_2$,使得当~$n>N_2$~时,有
\[
  \mabs{y_n-a}<\e\implies a-\e<y_n<a+\e 。
\]
取~$N=\max\mbrace{N_1,N_2}$,则当~$n>N$~时,就有
\[
  a-\e<x_n\leq z_n\leq y_n<a+\e\implies\mabs{z_n-a}<\e,
\]
即~$\lim_\ntoinf z_n=a$。
\end{proof}

\begin{example}
设~$a,b>0$,证明
\[
  \lim_\ntoinf\sqrt[n]{a^n+b^n}=\max\mbrace{a,b}。
\]
\end{example}
\begin{proof}
因为
\[
  \mparenB{\mparenb{\max\mbrace{a,b}}^n}^{\frac1n}\leq\mparenb{a^n+b^n}^{\frac1n}\leq
  \mparenB{2\mparenb{\max\mbrace{a,b}}^n}^{\frac1n},
\]
即
\[
  \max\mbrace{a,b}\leq\sqrt[n]{a^n+b^n}\leq\max\mbrace{a,b}\cdot\sqrt[n]2。
\]
由~$\lim_\ntoinf\sqrt[n]2=1$~及\ref{thm:sec2.2-5}~即可得到
\[
  \lim_\ntoinf\sqrt[n]{a^n+b^n}=\max\mbrace{a,b}。\qedhere
\]
\end{proof}

\begin{quiz}\begin{thmenumlist}
\item 在\ref{thm:sec2.2-4}~中若将条件改为~$x_n<y_n$,结论能否改为~$a<b$?
\item 考虑能否利用\ref{thm:sec2.2-4}~直接得到\ref{thm:sec2.2-5},即
\[
  a=\lim_\ntoinf x_n\leq\lim_\ntoinf z_n\leq\lim_\ntoinf y_n=a。
\]
\end{thmenumlist}
\end{quiz}

作为序列极限的特殊情形,我们引入无穷小量的概念及其运算。

\begin{definition}
  极限为零的变量称为\emph{无穷小量}。
\end{definition}

若变量以序列形式表示,若序列极限为零,则称序列就叫作无穷小量。例如
\[
  \mbracebb{\frac1n},\quad \mbrace{q^n}\mcond{0<q<1},\quad
  \mbracebb{\frac{a^n}{n!}}\mcond{a>0}
\]
都是无穷小量。所以无穷小量不是很小的量,而是极限值为零的变量。

\begin{corollary}\label{cor:sec2.2-1}
\begin{thmenumlist}
  \item 无穷小量的绝对值是无穷小量;\label{cor:sec2.2-1-item-1}
  \item 无穷小量与有界变量的乘积是无穷小量。\label{cor:sec2.2-1-item-2}
\end{thmenumlist}
\end{corollary}
\begin{proof}\begin{thmenumlist}
\item 设~$\lim_\ntoinf x_n=0$,即对任意~$\e>0$,存在~$N$,使得当~$n>N$~时,有
\[
  \mabs{x_n-0}=\mabs{x_n}<\e\implies\mabsb{\mabs{x_n}-0}=\mabs{x_n}<\e,
\]
故~$\lim_\ntoinf\mabs{x_n}=0$。
\item 设~$\lim_\ntoinf x_n=0$,而~$\mabs{y_n}\leq M$,则
\[
  \mabs{x_ny_n}\leq M\mabs{x_n}\implies
  -M\mabs{x_n}\leq x_n\cdot y_n\leq M\mabs{x_n}。
\]
由~\ref{cor:sec2.2-1-item-1}~可知~$\lim_\ntoinf M\mabs{x_n}=\lim_\ntoinf(-M\mabs{x_n})=0$,利用\ref{thm:sec2.2-5}~可得
\[
  \lim_\ntoinf x_n\cdot y_n=0。\qedhere
\]
\end{thmenumlist}
\end{proof}

\begin{corollary}\label{cor:sec2.2-2}
变量有极限~$a$~当且仅当变量可以分解成~$a$~与无穷小量之和。
\end{corollary}
\begin{proof}
设~$\lim_\ntoinf x_n=0$,则
\[
  \lim_\ntoinf(x_n-a)=0。
\]
令~$\alpha_n=x_n-a$,由定义可知~$\mbrace{\alpha_n}$~是无穷小量,并有~$x_n=a+\alpha_n$。

反之,若~$x_n=a+\alpha_n$,且~$\lim_\ntoinf\alpha_n=0$,则
\[
  \lim_\ntoinf x_n=\lim_\ntoinf(a+\alpha_n)=a。\qedhere
\]
\end{proof}

\begin{example}
证明~$\lim_\ntoinf\sqrt[n]n=1$。
\end{example}
\begin{proof}
令~$\sqrt[n]n=1+h_n$,只要证明~$h_n$~是无穷小量。事实上,
\[
  n=(1+h_n)^n=1+nh_n+\frac{n(n-1)}2h_n^2+\dotsb+h_n^n\geq\frac{n(n-1)}2h_n^2,
\]
所以
\[
  0<h_n<\frac2{\sqrt{n-1}}\mcond*{n>1},
\]
由~$\lim_\ntoinf\dfrac2{\sqrt{n-1}}=0$~以及\ref{thm:sec2.2-5},可得~$\lim_\ntoinf h_n=0$,再
由\ref{cor:sec2.2-1}~就可以得到
\[
\lim_\ntoinf\sqrt[n]n=1。\qedhere
\]
\end{proof}

\begin{exercise}
\item 若~$\lim_\ntoinf(y_n-x_n)=0$,且~$\lim_\ntoinf x_n=a$。证明~$\lim_\ntoinf y_n=a$。判断并说明下述证法的正确性。
\[
  0=\lim_\ntoinf(y_n-x_n)=\lim_\ntoinf y_n-\lim_\ntoinf x_n=\lim_\ntoinf y_n-a\implies\lim_\ntoinf y_n=a。
\]
\item 求下列极限。
\begin{exlistcols}
  \item $\lim_\ntoinf\mparenB{\dfrac1{n^2}+\dfrac2{n^2}+\dotsb+\dfrac n{n^2}}$;
  \item $\lim_\ntoinf\dfrac{a^n}{1+a+\dotsb+a^{n-1}}\mcond{a>0}$;
  \item $\lim_\ntoinf\mparenbb{\dfrac1{1\cdot2}+\dfrac1{2\cdot3}+\dotsb+\dfrac1{n(n+1)}}$;
  \item $\lim_\ntoinf\mparenb{\sqrt[3]{n+1}-\sqrt[3]n}$;
  \item $\lim_\ntoinf\sqrt[n]a\mcond{0<a<1}$;
  \item $\lim_\ntoinf\dfrac{n^k}{a^n}\mcond{a>1,k>0}$。
\end{exlistcols}
\item 设~$\lim_\ntoinf x_n=a$,$\lim_\ntoinf y_n=b$。证明,
\begin{exlistcols}
  \item $\lim_\ntoinf\max\mbrace{x_n,y_n}=\max\mbrace{a,b}$;
  \item $\lim_\ntoinf\min\mbrace{x_n,y_n}=\min\mbrace{a,b}$。
\end{exlistcols}
\item 证明序列~$\mbrace{\cos n}$~的极限不存在。
\item 证明序列~$\mbrace{\tan n}$~与~$\mbrace{\cot n}$~的极限不存在。
\item 求下列极限。
\begin{exlistcols}
  \item $\lim_\ntoinf\dfrac{\sin n}{\sqrt[3]{1-n^2}}$;
  \item $\lim_\ntoinf\sqrt[n]{n\cdot\ln n}$;
  \item $\lim_\ntoinf\mparenbb{\dfrac{1\cdot3\cdot5\dotsm(2n-1)}{2\cdot4\cdot6\dotsm(2n)}}^{\msp\frac1n}$;
  \item $\lim_\ntoinf\dfrac{1\cdot3\cdot5\dotsm(2n-1)}{2\cdot4\cdot6\dotsm(2n)}$;
  \item $\lim_\ntoinf\dfrac1{\sqrt[n]{n!}}$。
\end{exlistcols}
\item 若~$\mabs{x_{n+1}}\leq k\mabs{x_n}\mcond{0<k<1}$~对任意~$n$~成立。证明~$\lim_\ntoinf x_n=0$。
\item 设~$f(x)=\dfrac{x+2}{x+1}$,而~$x_0=1$,$x_{n+1}=f(x_n)\mcond{n\geq0}$。证明~$\lim_\ntoinf x_n=\sqrt2$。
\item 设~$x_1=1$,而~$x_{n+1}=\sqrt{2+x_n}\mcond{n\geq1}$。求~$\lim_\ntoinf x_n$。
\item 设~$x_0=1$,而~$x_{n+1}=1+\dfrac1{x_n}\mcond{n\geq0}$。求~$\lim_\ntoinf x_n$。
\item\begin{exlist}
  \item 设~$b>1$,$\lim_\ntoinf x_n=0$。证明,对任意~$\e$,$0<\e<1$,存在~$N$,使得当~$n>N$~时,有
  \[
    \log_b(1-\e)<x_n<\log_b(1+\e);
  \]
  \item 设~$b>1$,$\lim_\ntoinf x_n=0$,证明~$\lim_\ntoinf b^{x_n}=1$;
  \item 设~$0<b<1$,$\lim_\ntoinf x_n=0$,证明~$\lim_\ntoinf b^{x_n}=1$;
  \item 设~$b>0$,$\lim_\ntoinf x_n=a$,证明~$\lim_\ntoinf b^{x_n}=b^a$。
\end{exlist}
\item\begin{exlist}
  \item 设~$\lim_\ntoinf y_n=1$。证明,对任意~$\e$,$0<\e<1$,存在~$N$,使得当~$n>N$~时,有~$\me^{-\e}<y_n<\me^\e$;
  \item 设~$\lim_\ntoinf y_n=1$,证明~$\lim_\ntoinf\ln y_n=0$;
  \item 设~$\lim_\ntoinf y_n=b\mcond{b>0}$,证明~$\lim_\ntoinf\ln y_n=b$;
  \item 设~$\lim_\ntoinf y_n=b\mcond{b>0}$,$\lim_\ntoinf x_n=a$,证明~$\lim_\ntoinf y_n^{x_n}=b^a$。
\end{exlist}
\item\begin{exlist}
  \item 设~$x_n>0$,且~$\lim_\ntoinf\sqrt[n]{x_n}=q<1$,证明~$\lim_\ntoinf x_n=0$;
  \item 设~$x_n>0$,且~$\lim_\ntoinf\sqrt[n]{x_n}=\ell>1$,证明~$\lim_\ntoinf\dfrac1{x_n}=0$。
\end{exlist}
\item 设~$\mbrace{x_n}$~为无穷小量。判断并说明,
\begin{exlistcols}
  \item $\mbrace{x_n^n}$~是否为无穷小量;
  \item $\mbrace{\!\sqrt[n]{x_n}}\mcond{x_n>0}$~是否为无穷小量。
\end{exlistcols}
\end{exercise}


\section{确界与单调有界序列}

在这一节中,我们将讨论极限的存在性。给定序列~$\mbrace{x_n}$,怎么判定它有没有极限呢?用极限定义判定,就必须先要看出极限值,这
对稍微复杂的序列是办不到的,所以只能通过序列本身来判断它有没有极限。判定一般序列是否有极限,我们以后再讨论。对单调序列有简单
的判别准则。这个准则要用到所谓确界的概念,为此,我们先讨论集合的确界。

设有实数集合~$E\subset\mR$,若存在实数~$M$,使得对任意~$x\in E$,均有
\[
  x\leq M,
\]
则称~$M$~是集合~$E$~的一个\emph{上界}。集合~$E$~有一个上界,显然就有无穷个上界,那么这些上界中有没有最小上界呢?

若集合~$E$~由无限个元素组成,那么它的最大元素就是集合的最小上界。若~$E$~是无限集,这时~$E$~可以没有最大集合,那么它的最小上界
是什么意思呢?它有没有最小上界呢?首先我们给出最小上界的定义。

\begin{definition}\label{def:sec2.3-1}
给定实数集合~$E$,若实数~$M$~满足
\begin{enumlist}
\item $M$~是集合~$E$~的上界;\label{def:sec2.3-1-item-1}
\item 若~$M'$~是集合~$E$~的一个上界,则必有~$M\leq M'$,\label{def:sec2.3-1-item-2}
\end{enumlist}
则称~$M$~是集合~$E$~的\emph{上确界}或\emph{最小上界},记作
\[
  M=\sup E\quad\text{或}\quad M=\sup_{x\in E}\mbrace x。
\]
\end{definition}

容易看出,集合~$E$~不可能有两个上确界,如果有两个上确界~$M_1,M_2$,由定义可得~$M_1\leq M_2$~及~$M_2\leq M_1$,即得~$M_1=M_2$。

既然上确界是集合的最小上界,那么比它更小的数就不可能是集合的上界,我们把它写成如下定理。

\begin{theorem}\label{thm:sec2.3-1}
$M$~是集合~$E$~的上确界当且仅当
\begin{enumlist}
  \item $M$~是集合~$E$~的上界;\label{thm:sec2.3-1-item-1}
  \item 对任意~$\e>0$,存在~$x'\in E$,使得~$x'>M-\e$。\label{thm:sec2.3-1-item-2}
\end{enumlist}
\end{theorem}
\begin{proof}
\emphitem{必要性}用反证法。假设~\ref{thm:sec2.3-1-item-2}~不成立,即存在~$\e_0>0$,对任意~$x\in E$,均有
\[
  x\leq M-\e_0,
\]
这说明~$M-\e_0$~是~$E$~的一个上界,由上确界的\ref{def:sec2.3-1}~有
\[
  M\leq M-\e_0\implies 0<\e_0\leq 0。
\]
这是一个矛盾,所以结论~\ref{thm:sec2.3-1-item-2}~成立。

\emphitem{充分性}用反证法。假设\ref{def:sec2.3-1}~的条件~\ref{def:sec2.3-1-item-2}~不成立,即存在~$M'$~是~$E$~的
上界,但~$M>M'$。令
\[
  \e=M-M'>0,
\]
有定理条件~\ref{thm:sec2.3-1-item-2},存在~$x'\in E$,使得
\[
  x'>M-\e=M',
\]
这与~$M'$~是~$E$~的上界矛盾,所以定义的条件~\ref{def:sec2.3-1-item-2}~成立。
\end{proof}

\begin{definition}\label{def:sec2.3-2}
给定实数集合~$E$,若实数~$m$~满足
\begin{enumlist}
\item $m$~是集合~$E$~的下界,即对任意~$x\in E$,都有~$x\geq m$;\label{def:sec2.3-2-item-1}
\item 若~$m'$~是集合~$E$~的一个下界,则必有~$M\leq M'$,\label{def:sec2.3-2-item-2}
\end{enumlist}
则称~$m$~是集合~$E$~的\emph{下确界}或\emph{最大上界},记作
\[
  M=\inf E\quad\text{或}\quad M=\inf_{x\in E}\mbrace x。
\]
\end{definition}

下面这个例子讨论三个集合~$\mathset{f(x)}{x\in X}$,$\mathset{g(x)}{x\in X}$~和~$\mathset{f(x)+g(x)}{x\in X}$~的上、下确界之间的关系。

\begin{example}
设~$f(x),g(x)$~是定义在~$X=[a,b]$~上的有界函数,证明
\[
  \sup_{x\in X}\mbraceb{f(x)+g(x)}\geq\sup_{x\in X}\mbraceb{f(x)}+\sup_{x\in X}\mbraceb{g(x)}。
\]
\end{example}
\begin{proof}
有上确界定义中的~\ref{def:sec2.3-1-item-2}~及下确界定义中的~\ref{def:sec2.3-2-item-1},对任意~$\e>0$,存在~$x'\in X$,使得
\[
  f(x')>\sup_{x\in X}\mbraceb{f(x)}-\e,\quad g(x')\geq\inf_{x\in X}\mbraceb{g(x)}。
\]
相加得到
\[
  f(x')+g(x')\geq\sup_{x\in X}\mbraceb{f(x)}+\inf_{x\in X}\mbraceb{g(x)}-\e 。
\]
再由上确界定义中的~\ref{def:sec2.3-1-item-1}
\[
  \sup_{x\in X}\mbraceb{f(x)+g(x)}\geq f(x')+g(x'),
\]
所以
\[
  \sup_{x\in X}\mbraceb{f(x)+g(x)}\geq\sup_{x\in X}\mbraceb{f(x)}+\inf_{x\in X}\mbraceb{g(x)}-\e 。
\]
又由于上、下确界是确定的数和~$\e$~的任意性,令~$\e\to0$,就得到结论。
\end{proof}

\begin{theorem}[确界原理]\label{thm:sec2.3-2}
一个非空的、有上(下)界的集合,必有上(下)确界。
\end{theorem}

从直观上看,这个所谓的确界原理是很显然的。但仔细一想,承认这件事相当于承认由上界组成的无穷集合必有最小值,这又不是
十分明显的。定理的证明牵涉到什么叫实数,等严格讨论过实数定义以后,我们再给出定理的证明。

有了确界概念之后,就可以解决单调序列极限的存在问题。

若序列~$\mbrace{x_n}$~满足
\[
  x_1\leq x_2\leq\dotsb\leq x_n\leq\dotsb,
\]
则称~$\mbrace{x_n}$~是\emph{单调上升}的;若满足
\[
  x_1\geq x_2\geq\dotsb\geq x_n\geq\dotsb,
\]
则称~$\mbrace{x_n}$~是\emph{单调下降}的。

\begin{theorem}[单调有界定理]\label{thm:sec2.3-3}
如果序列单调上升(下降)并且有上(下)界,那么序列极限存在。
\end{theorem}
\begin{proof}
已知序列单调上升,即
\[
  x_1\leq x_2\leq\dotsb\leq x_n\leq\dotsb,
\]
且有上界,即存在~$M$,使得
\[
  x_n\leq M\mcond*{n=1,2,\dotsc}。
\]

现在考虑集合~$\mathsetb{x_n}{n\in\mN}$,它是一个非空的、有上界集合,由\ref{thm:sec2.3-2}~可知集合~$\mathsetb{x_n}{n\in\mN}$~有
上确界,记为
\[
  a\coloneq\sup_{n\in\mN}\mbrace{x_n}。
\]

对任意~$\e>0$,由上确界性质,存在~$N$,使得~$a-\e<x_N$。故当~$x>N$~时,由序列单调上升可得
\[
  a-\e<x_N\leq x_n 。
\]
再由上确界定义,有
\[
  a-\e<x_n<a+\e\implies\mabs{x_n-a}<\e,
\]
也就是说
\[
  \lim_\ntoinf x_n=a=\sup_{n\in\mN}\mbrace{x_n}。
\]

同理,若序列~$\mbrace{x_n}$~单调下降、有下界,可得
\[
  \lim_\ntoinf x_n=\inf_{n\in\mN}\mbrace{x_n}。\qedhere
\]
\end{proof}

\begin{example}\label{ex:sec2.3-2}
求极限~$\lim_\ntoinf\dfrac{a^n}{n!}\mcond{a>1}$。
\end{example}
\begin{proof}
注意到
\[
  0<x_{n+1}=\frac{a^{n+1}}{(n+1)!}=\frac a{n+1}\cdot x_n\leq x_n\mcond*{x>\mfloor a},
\]
这说明~$n>\mfloor a$~以后,序列单调下降、有下界,因此序列极限存在,记极限值为~$\ell$。为了确定出~$\ell$,我们对等式
\[
  x_{n+1}=\frac a{n+1}\cdot x_n
\]
取极限,得到~$\ell=0\cdot\ell$,即~$\ell=0$。
\end{proof}

\begin{example}\label{ex:sec2.3-3}
证明极限~$\lim_\ntoinf\mparenbb{1+\dfrac1n}^{\msp n}$~存在。
\end{example}
\begin{proof}
先证明序列单调上升。事实上
\begin{spreadlines}{3pt}
\begin{align*}
x_n&=\mparenbb{1+\frac1n}^{\msp n}\\
&=1+1\frac{n(n-1)}{2!}\frac1{n^2}+\frac{n(n-1)(n-2)}{3!}\frac1{n^3}+\dotsb+
  \frac{n(n-1)\dotsm\mparenb{n-(n-1)}}{n!}\frac1{n^n}\\
&=1+1+\frac1{2!}\mparenB{1-\frac1n}+\frac1{3!}\mparenB{1-\frac1n}\mparenB{1-\frac2n}+\dotsb+
  \frac1{n!}\mparenB{1-\frac1n}\dotsm\mparenB{1-\frac{n-1}n}。
\end{align*}
并且
\begin{align*}
x_{n+1}&=\mparenbb{1+\frac1{n+1}}^{\msp n+1}\\
&=1\!\begin{lgathered}[t]
  {}+1+\frac1{2!}\mparenB{1-\frac1{n+1}}+\frac1{3!}\mparenB{1-\frac1{n+1}}\mparenB{1-\frac2{n+1}}+\dotsb\\
  {}+\frac1{n!}\mparenB{1-\frac1{n+1}}\dotsm\mparenB{1-\frac{n-1}{n+1}}
  +\frac1{(n+1)!}\mparenB{1-\frac1{n+1}}\dotsm\mparenB{1-\frac n{n+1}}。
  \end{lgathered}
\end{align*}
比较~$x_n$~与~$x_{n+1}$,$x_n$~有~$n+1$~项,$x_{n+1}$~有~$n+2$~项,其中前~$n+1$~项分别比~$x_n$~中相应的项要大或相等,最后
一项大于~$0$,所以序列~$\mbrace{x_n}$~单调上升。

再证明序列有上界。事实上
\begin{align*}
x_n
&\leq1+1+\frac1{2!}+\frac1{3!}+\dotsb+\frac1{n!}
 \leq1+1+\frac12+\frac1{2^2}+\dotsb+\frac1{2^{n-1}}\\
&=1+\frac{1-\sfrac1{2^n}}{1-\sfrac12}<3。
\end{align*}
\end{spreadlines}
由\ref{thm:sec2.3-3}~可知序列极限存在,其值记为~$\me$,它是一个无理数。即
\[
  \me\coloneq\lim_\ntoinf\mparenbb{1+\frac1n}^{\msp n}=\num{2.718281828459}\dotsc 。\qedhere
\]
\end{proof}

称以数~$\me$~为底的对数为\emph{自然对数},这时记作~$\log_\me x\eqcolon\ln x$。

\begin{exercise}
\item 确定下列集合的上、下确界。
\begin{exlistcols}
  \item $E=\mathsetbb{1+(-1)^n\dfrac{n+1}n}{n\in\mN}$;
  \item $E=\mathsetbb{\dfrac mn}{0<m<n,~m,n\in\mN}$;
  \item $E=\mathsetbb{\dfrac{n-1}{n+1}\cos\dfrac{2n}3\pi}{n\in\mN}$;
  \item $E=\mathsetb{\!\sqrt[n]{1+2^{n(-1)^n}}}{n\in\mN}$;
  \item $E=\mathsetB{x-\mfloor x}{x\in\mR}$。
\end{exlistcols}
\item 设~$f(x),g(x)$~在~$D$~上定义,且对任意~$x\in D$,有~$f(x)\leq g(x)$。证明
\begin{exlistcols}
  \item $\sup_{x\in D}\mbraceb{f(x)}\leq\sup_{x\in D}\mbraceb{g(x)}$;
  \item $\inf_{x\in D}\mbraceb{f(x)}\leq\inf_{x\in D}\mbraceb{g(x)}$。
\end{exlistcols}
\item 设~$f(x)$~在~$D$~上定义。证明
\begin{exlistcols}
  \item $\sup_{x\in D}\mbrace{-f(x)}=-\inf_{x\in D}\mbraceb{f(x)}$;
  \item $\inf_{x\in D}\mbrace{-f(x)}=-\sup_{x\in D}\mbraceb{f(x)}$。
\end{exlistcols}
\item 设~$f(x),g(x)$~在~$D$~上有界。证明
\begin{align*}
   \inf_{x\in D}\mbraceb{f(x)}+\inf_{x\in D}\mbraceb{g(x)}
  &\leq\inf_{x\in D}\mbrace{f(x)+g(x)}\leq\inf_{x\in D}\mbraceb{f(x)}+\sup_{x\in D}\mbraceb{g(x)}\\
  &\leq\sup_{x\in D}\mbrace{f(x)+g(x)}\leq\sup_{x\in D}\mbraceb{f(x)}+\sup_{x\in D}\mbraceb{g(x)}。
\end{align*}
\item 设~$f(x),g(x)$~在~$D$~上有界且大于~$0$。证明
\begin{align*}
   \inf_{x\in D}\mbraceb{f(x)}\cdot\inf_{x\in D}\mbraceb{g(x)}
  &\leq\inf_{x\in D}\mbrace{f(x)\cdot g(x)}\leq\inf_{x\in D}\mbraceb{f(x)}\cdot\sup_{x\in D}\mbraceb{g(x)}\\
  &\leq\sup_{x\in D}\mbrace{f(x)\cdot g(x)}\leq\sup_{x\in D}\mbraceb{f(x)}\cdot\sup_{x\in D}\mbraceb{g(x)}。
\end{align*}
\item 求下列序列的极限。
\begin{exlistcols}
  \item $\sqrt2$,$\sqrt{2\sqrt2}$,$\smbsqrt{2\sqrt{2\sqrt2}}$,\ldots ;
  \item $\sqrt2$,$\sqrt{2+\sqrt2}$,$\smbsqrt{2+\sqrt{2+\sqrt2}}$,\ldots 。
\end{exlistcols}
\item 利用单调有界定理,证明~$\lim_\ntoinf\sqrt[n]n=1$。
\item 设~$0<a_1<b_1$,对~$n=1,2,\dotsc$,令
\[
  a_{n+1}=\txts\smbsqrt{a_n\cdot b_n},\quad b_{n+1}=\dfrac{a_n+b_n}2。
\]
证明序列~$\mbrace{a_n},\mbrace{b_n}$~极限存在且相等。
\item 设~$0<a_1<b_1<c_1$,对~$n=1,2,\dotsc$,令
\[
  a_{n+1}=\frac1{\dfrac1{a_n}+\dfrac1{b_n}+\dfrac1{c_n}},\quad
  b_{n+1}=\txts\smbsqrt[3]{a_n\cdot b_n\cdot c_n},\quad
  c_{n+1}=\dfrac{a_n+b_n+c_n}3。
\]
证明序列~$\mbrace{a_n},\mbrace{b_n},\mbrace{c_n}$~极限存在且相等。
\item\begin{exlist}\FixExHead
  \item $\dfrac1{n+1}<\ln\mparenB{1+\dfrac1n}<\dfrac1n$;
  \item 序列~$x_n=1+\dfrac12+\dotsb+\dfrac1n-\ln n$~的极限存在;
  \item $\lim_\ntoinf\mparenB{1-\dfrac12+\dfrac13-\dfrac14+\dotsb+(-1)^{n-1}\dfrac1n}=\ln2$。
\end{exlist}
\item 设~$A>0$,$x_1>0$,$x_{n+1}=\dfrac12\mparenB{x_n+\dfrac A{x_n}}$。证明序列~$\mbrace{x_n}$~的极限存在,并求出此极限。
\item 设~$A>0$,$0<x_1<\dfrac1n$,$x_{n+1}=x_n(2-Ax_n)$。证明序列~$\mbrace{x_n}$~的极限存在,并求出此极限。
\item 设序列~$\mbrace{q_n}$~满足~$0<q_n<1$~且~$(1-q_n)q_{n+1}>\dfrac14$。证明~$\mbrace{q_n}$~单调上
升,且~$\lim_\ntoinf q_n=\dfrac12$。
\item 设~$\mbrace{a_n}$~单调下降收敛于~$0$,令
\[
  b_n=\frac{a_1+a_2+\dotsb+a_n}n。
\]
证明,
\begin{exlistcols}[3]
  \item $\mbrace{b_n}$~单调下降;
  \item $b_{2n}\leq\dfrac12(a_n+b_n)$;
  \item $\lim_\ntoinf b_n=0$。
\end{exlistcols}
\end{exercise}

\section{函数的极限}

先考察一个实例:求自由落体
\[
s=\dfrac12gt^2
\]
在~$t_0$~时刻的瞬时速度。

我们会求匀速运动的速度,现在自由落体不是作匀速运动,而是变速运动,下落的速度愈来愈快,怎么来求~$t_0$~时刻的瞬时速度呢?回顾
一下求曲边三角形面积的方法,那时我们遇到直与曲的矛盾,解决这个矛盾分两步,第一步在局部上以直代曲得到阶梯形面积,它是曲边三角
形面积的近似值;第二部通过取极限,由近似值得到准确值。现在解决速度匀速与不匀速的矛盾,仍用同样办法处理,并从第二步中抽象出函
数极限的定义。

考虑时间间隔~$[t_0,t]$,若在间隔较短情况下,速度变化不大,可以近似看作匀速运动,求出该时间间隔内的平均速度
\[
  \mbar v=\frac{s(t)-s(t_0)}{t-t_0}=\frac{\dfrac12gt^2-\dfrac12gt_0^2}{t-t_0}=\frac12g(t+t_0),
\]
这个速度是要求速度的近似值,若时间间隔取得越小,近似程度就越高,但不管间隔多小,总有一个近似值,而取间隔为~$0$,得到~$\sfrac00$,%
什么也得不到。

要想得到~$t_0$~时刻的瞬时速度,必须让~$t$~趋向于~$t_0$,这时平均速度就趋向于~$gt_0$,
\[
  \lim_{t\to t_0}\frac{s(t)-s(t_0)}{t-t_0}=gt_0。
\]
通过极限过程,我们求出了瞬时速度为~$gt_0$。

在给出函数定义前,为了叙述方便,先定义两个名词。

称集合
\[
  \mathsetb x{\mabs{x-x_0}<h}
\]
为~$x_0$~的一个\emph{邻域},记作~$U(x_0;h)$;称集合~$U(x_0;h)-\mbrace x$~为~$x_0$~的一个\emph{空心邻域},记作~$U_0(x_0;h)$。%
当不需要知道邻域半径时,我们用~$U(x_0)$~和~$U_0(x_0)$~分别表示~$x_0$~的邻域和~$x_0$~的空心邻域。

\begin{definition}\label{def:sec2.4-1}
设~$f(x)$~在~$U_0(x_0)$~上定义,如果对任意~$\e>0$,存在~$\delta>0$,使得当~$0<\mabs{x-x_0}<\delta$~时,有
\[
  \mabsb{f(x)-A}<\e,
\]
则称~$x$~趋向于~$x_0$~时,\emph{函数~$f(x)$~的极限}为~$A$,记作~$f(x)\to A\mcond{x\to x_0}$,或
\[
  \lim_{x\to x_0}f(x)=A,
\]
\end{definition}


%\fixwrapfloatsep
\begin{wrapfigure}{O}{0mm}
\somefigure
\caption{}\label{fig:sec2.4-1}
\end{wrapfigure}

函数定义的几何意义是,任给一~$A$~为中心的~$\e$~邻域,总可找出以~$x_0$~为中心的~$\delta$~空心邻域,当动点落在空心
邻域~$U_0(x_0;\delta)$~中时,动点函数值落在~$U(A;\e)$~内(见\ref{fig:sec2.4-1})。

\begin{example}
证明~$\lim_{x\to1}\dfrac{x^2-1}{x-1}=2$。
\end{example}
\begin{proof}
对任意~$\e>0$,有
\[
  \mabsbb{\frac{x^2-1}{x-1}-2}<\e\impliedby\mabs{x+1-2}=\mabs{x-1}<\e,
\]
取~$\delta=\e$,则当~$0<\mabs{x-1}<\delta$~时,就有
\[
  \mabsbb{\frac{x^2-1}{x-1}-2}<\e,
\]
即
\[
  \lim_{x\to1}\dfrac{x^2-1}{x-1}=2。\qedhere
\]
\end{proof}

\begin{example}
证明~$\lim_{x\to a}\sqrt x=\sqrt a\mcond{a>0}$。
\end{example}
\begin{proof}
对任意~$\e>0$,有
\[
  \mabsb{\sqrt x-\sqrt a}<\e\impliedby\mabsbb{\frac{x-a}{\sqrt x+\sqrt a}}\leq\frac1{\sqrt a}\mabs{x-a}<\e 。
\]
取~$\delta=\sqrt a\,\e$,则当~$0<\mabs{x-a}<\delta$~时,就有
\[
  \mabsb{\sqrt x-\sqrt a}<\e,
\]
即
\[
  \lim_{x\to a}\sqrt x=\sqrt a\mcond*{a>0}。\qedhere
\]
\end{proof}

\begin{example}
证明~$\lim_{x\to a}\dfrac1x=\dfrac1a\mcond{a\neq0}$。
\end{example}

由\ref{def:sec2.4-1}~的几何意义可以看出,对任意~$\e>0$~找到符合定义要求的~$\delta$~时,那么比~$\delta$~小的正数也符合定义的
要求。所以在找~$\delta$~前,可以先限制小于某正数。如这个例子中分母有~$x$,为了防止~$x$~接近原点,我们先对~$\delta$~作一限制。

\begin{proof}
考虑~$x$~满足
\[
  \mabs{x-a}<\frac{\mabs a}2\implies
  \mabs x>\mabs a-\frac{\mabs a}2=\frac{\mabs a}2。
\]

那么对任意~$\e>0$,有
\[
  \mabsbb{\frac1x-\frac1a}<\e\impliedby
  \mabsbb{\frac{x-a}{xa}}\leq\frac{2\mabs{x-a}}{\mabs a^2}<\e 。
\]
取
\[
  \delta=\min\mbracebb{\frac{\mabs a}2,~\frac{\mabs a^2}2\e},
\]
则当~$0<\mabs{x-a}<\delta$~时,就有
\[
  \mabsbb{\frac1x-\frac1a}<\e,
\]
即
\[
  \lim_{x\to a}\frac1x=\frac1a\mcond*{a\neq0}。\qedhere
\]
\end{proof}

函数~$\sfrac1x$~当~$x\to 0$~时无限增大,所以极限~$\lim_{x\to0}\sfrac1x$~不存在,函数~$\sin\sfrac1x$~当~$x\to0$~时,其值
不停地在~$-1$~与~$1$~之间振荡,所以极限~$\lim_{x\to0}\sin\sfrac1x$~也不存在。

关于函数极限,同样有唯一性、局部有界性、四则运算和极限的不等式,因为证法与序列相同,我们只挑选两个予以证明。

\begin{theorem}\label{thm:sec2.4-1}
若极限~$\lim_{x\to x_0}f(x)$~存在,则极限值唯一。
\end{theorem}


\begin{theorem}\label{thm:sec2.4-2}
若极限~$\lim_{x\to x_0}f(x)$~存在,则函数在~$x_0$~的某一空心邻域上有界。
\end{theorem}
\begin{proof}
取~$\e_0=1$,由条件,存在~$\delta>0$,当~$0<\mabs{x-x_0}<\delta$~时,有
\[
  \mabsb{f(x)-A}<1\implies\mabsb{f(x)}<\mabs A+1 。
\]
这说明函数在~$U_0(x_0;\delta)$~上有界,界为~$\mabs A+1$。
\end{proof}

\begin{remark}
函数在原定义域上不一定有界,所以我们只称函数局部有界。
\end{remark}

\begin{theorem}\label{thm:sec2.4-3}
设~$\lim_{x\to x_0}=A$,而~$\lim_{x\to x_0}g(x)=B$,则
\begin{enumlistcols}
  \item $\lim_{x\to x_0}\mparenb{f(x)\pm g(x)}=A\pm B$;\label{thm:sec2.4-3-item-1}
  \item $\lim_{x\to x_0}f(x)\cdot g(x)=A\cdot B$;\label{thm:sec2.4-3-item-2}
  \item $\lim_{x\to x_0}\dfrac{f(x)}{g(x)}=\dfrac AB\mcond{B\neq0}$。\label{thm:sec2.4-3-item-3}
\end{enumlistcols}
\end{theorem}
\begin{proof}
这里只证明~\ref{thm:sec2.4-3-item-3},而由~\ref{thm:sec2.4-3-item-2},只需证明
\[
  \lim_{x\to x_0}\frac1{g(x)}=\frac1B 。
\]

事实上,取~$\e_0=\dfrac{\mabs B}2>0$,由~$\lim_{x\to x_0}g(x)=B$,存在~$\delta_1$,当~$0<\mabs{x-x_0}<\delta_1$~时,有
\[
  \mabsB{\mabsb{g(x)}-\mabs B}\leq\mabsb{g(x)-B}<\frac{\mabs B}2\implies
  \mabsb{g(x)}\geq\mabs B-\frac{\mabs B}2=\frac{\mabs B}2。
\]

对任意~$\e>0$,同样由~$\lim_{x\to x_0}g(x)=B$,存在~$\delta_2$,当~$0<\mabs{x-x_0}<\delta_2$~时,有
\[
  \mabsb{g(x)-B}<\frac{\mabs B^2}2\e 。
\]
令~$\delta=\min\mbrace{\delta_1,\delta_2}>0$,所以当~$0<\mabs{x-x_0}<\delta$~时,就有
\[
  \mabsbb{\frac1{g(x)}-\frac1B}=\frac{\mabsb{g(x)-B}}{g(x)\cdot B}\leq
  \frac2{\mabs B^2}\mabsb{g(x)-B}<\frac2{\mabs B^2}\cdot\frac{\mabs B^2}2\e=\e 。
\]
即
\[
  \lim_{x\to x_0}\frac1{g(x)}=\frac1B 。\qedhere
\]
\end{proof}

\begin{theorem}\label{thm:sec2.4-4}
设在~$U_0(x_0)$~上有~$f(x)\leq g(x)$,且~$\lim_{x\to x_0}f(x)=A$,$\lim_{x\to x_0}g(x)=B$,则
\[
  A\leq B。
\]
\end{theorem}

\begin{theorem}\label{thm:sec2.4-5}
设在~$U_0(x_0)$~上有~$f(x)\leq g(x)\leq h(x)$,且
\[
  \lim_{x\to x_0}f(x)=A=\lim_{x\to x_0}g(x),
\]
则~$\lim_{x\to x_0}g(x)=A$。
\end{theorem}

类似序列情形,若~$\lim_{x\to x_0}f(x)=0$,则称~$x\to x_0$~时,函数~$f(x)$~是无穷小量。序列中关于无穷小量的结论,相应地
对~$f(x)$~也成立。

\begin{exercise}
\item 判断并说明函数极限定义与下列形式是否等价。
\begin{exlist}
  \item 对任意~$\dfrac1{2^n}$,都存在~$\delta>0$,当~$0<\mabs{x-a}<\delta$~时,有~$\mabsb{f(x)-A}<\dfrac1{2^n}$;
  \item 对任意~$\e>0$,都存在~$\dfrac1n>0$,当~$0<\mabs{x-a}<\dfrac1n$~时,有~$\mabsb{f(x)-A}<\e$;
  \item 对任意~$\e>0$,都存在~$\delta>0$,当~$0<\mabs{x-a}<\delta$~时,有~$\mabsb{f(x)-A}<\e^2$;
  \item 对任意~$\e>0$,都存在~$\delta>0$,当~$0<\mabs{x-a}<\e\cdot\delta$~时,有~$\mabsb{f(x)-A}<\e^2$;
  \item 对任意~$\e>0$,都存在~$\delta>0$,当~$0<\mabs{x-a}<\delta$~时,有~$\mabsb{f(x)-A}<\e\cdot\delta$。
\end{exlist}
\item 用~$\e-\delta$~方法验证下列各题。
\begin{exlistcols}[3]
  \item $\lim_{x\to a}\mabs x=\mabs a$;
  \item $\lim_{x\to a}x^2=a^2$;
  \item $\lim_{x\to a}\sqrt x=\sqrt a\mcond{a>0}$;
  \item $\lim_{x\to a}x^3=a^3$;
  \item $\lim_{x\to a}\dfrac1x=\dfrac1a\mcond{a\neq0}$。
\end{exlistcols}
\item 设~$\lim_{x\to a}f(x)=A$,用~$\e-\delta$~方法验证下列各题。
\begin{exlistcols}
  \item $\lim_{x\to a}\mabsb{f(x)}=\mabs A$;
  \item $\lim_{x\to a}f^2(x)=A^2$;
  \item $\lim_{x\to a}\smbsqrt{f(x)}=\sqrt A\mcond{A>0}$;
  \item $\lim_{x\to a}\smbsqrt[3]{f(x)}=\sqrt[3]A$;
  \item $\lim_{x\to a}\dfrac1{\smbsqrt{f(x)}}=\dfrac1{\sqrt A}\mcond{A>0}$。
\end{exlistcols}
\item 求下列极限。
\begin{exlistcols}
  \item $\lim_{x\to1}\dfrac{x^n-1}{x-1}$;
  \item $\lim_{x\to2}\dfrac{x^2-5x+6}{x^3-2x^2-4x+8}$;
  \item $\lim_{x\to0}\dfrac{\sqrt{1-x}-\sqrt[3]{1+x}}x$;
  \item $\lim_{x\to0}\dfrac{\sqrt[3]{1+x}-1}{\sqrt{1+x}-1}$。
\end{exlistcols}
\end{exercise}


\section{函数极限的推广}\label{sec:2.5}

\subsection{自变量趋于无穷的情形}

求曲线的渐近线时,遇到自变量趋于无穷时的极限。

\fixwrapfloatsep
\begin{wrapfigure}[8]{O}{0mm}
\somefigure
\caption{}\label{fig:sec2.5-1}
\end{wrapfigure}

设有一伸展到无穷的曲线~$y=f(x)$。当点~$(x,y)$~沿曲线趋于无穷时,若它到定直线~$y=kx+b$~的距离趋于零,则
称该直线为曲线的\emph{斜渐近线}(见\ref{fig:sec2.5-1})。

曲线上的点~$\mparenb{x,f(x)}$~到直线~$y=kx+b$~的距离为
\[
  \frac{\mabsb{f(x)-kx-b}}{1+k^3},
\]
所以直线是曲线的斜渐近线当且仅当
\[
  \lim_{\substack{x\to+\infty\\(x\to-\infty)}}\frac{\mabsb{f(x)-kx-b}}{1+k^3}=0
  \quad\text{或}\quad
  \lim_{\substack{x\to+\infty\\(x\to-\infty)}}\mparenb{f(x)-kx-b}=0 。
\]

怎么求出曲线~$y=f(x)$~的斜渐近线呢?即怎么确定出常数~$k$~与~$b$~的值。

由斜渐近线定义
\begin{equation}\label{eq:sec2.5-1}
\lim_{x\to+\infty}\mparenb{f(x)-kx-b}=0,
\end{equation}
即得
\begin{equation}\label{eq:sec2.5-2}
\lim_{x\to+\infty}\mparenb{f(x)-kx}=b。
\end{equation}
若知道~$k$,即可由上式确定出~$b$。那么怎么求出~$k$~呢?由~$\lim_{x\to+\infty}\dfrac1x=0$~和极限的乘法运算,可得
\[
  \lim_{x\to+\infty}\mparenB{\frac{f(x)}x-k}=0,
\]
即
\begin{equation}\label{eq:sec2.5-3}
  \lim_{x\to+\infty}\frac{f(x)}x=k 。
\end{equation}

具体做题时,先由~\ref{eq:sec2.5-3}~求出~$k$,再由~\ref{eq:sec2.5-2}~求出~$b$。因为~\ref{eq:sec2.5-2}~蕴涵~\ref{eq:sec2.5-1}。这
表明所求出的直线~$y=kx+b$~确实为曲线~$y=f(x)$~的斜渐近线。若~$k=0$,则称直线~$y=b$~为曲线~$y=f(x)$~的\emph{水平渐近线}。

在叙述自变量趋于无穷时的函数极限之前,我们先引入几个名词。

\begin{enumlist}
\item 称集合~$\mathsetb x{\mabs x>h}$~为~\emph{$\infty$~的邻域},记作~$U(\infty;h)$~或~$U(\infty)$;
\item 称集合~$\mathset x{h<x<+\infty}$~与~$\mathset x{-\infty<x<h}$~为~\emph{$\infty$~的单侧邻域},记作~$U^+(\infty;h)$~
与~$U^-(\infty;h)$~或记作~$U^+(\infty)$~与~$U^-(\infty)$。
\end{enumlist}

\begin{definition}
设~$f(x)$~在~$U^+(\infty)$~上定义,如果存在~$A$,使得对任意~$\e>0$,存在~$X$,当~$x>X$~(自然要求~$X\in U^+(\infty)$)时,有
\[
  \mabsb{f(x)-A}<\e,
\]
则称~$x\to+\infty$~时,\emph{函数~$f(x)$~的极限为~$A$},记作~$f(x)\to A\mcond{x\to+\infty}$,或
\[
  \lim_{x\to+\infty}f(x)=A 。
\]
\end{definition}

\begin{enumlist}
\item 设~$f(x)$~在~$U^-(\infty)$~上定义,式子~$\lim_{x\to-\infty}f(x)=A$~的定义为:对任意~$\e>0$,存在~$X$,使得
当~$x<X$~时,有
\[
  \mabsb{f(x)-A}<\e 。
\]
\item 设~$f(x)$~在~$U(\infty)$~上定义,式子~$\lim_{x\to\infty}f(x)=A$~的定义为:对任意~$\e>0$,存在~$X>0$,使得
当~$\mabs x<X$~时,有
\[
  \mabsb{f(x)-A}<\e 。
\]
\end{enumlist}

对于这类极限,同样有极限的唯一性、局部有界性、四则运算和极限不等式,就不再一一赘述。

\begin{example}
证明~$\lim_{x\to\infty}\sqrt{1+\dfrac1{x^2}}=1$。
\end{example}
\begin{proof}
对任意~$\e>0$,有
\[
  \mabsbb{\sqrt{1+\dfrac1{x^2}}-1}<\e\impliedby
  \mabsbb{\frac{\sfrac1{x^2}}{\sqrt{1+\sfrac1{x^2}}+1}}\leq
  \mabsbb{\frac1{x^2}}<\e\impliedby
  x^2>\frac1\e 。
\]
取~$X=\dfrac1{\sqrt\e}$,则当~$\mabs x>X$~时,就有
\[
  \mabsbb{\sqrt{1+\dfrac1{x^2}}-1}<\e 。\qedhere
\]
\end{proof}

\begin{example}
求极限~$\lim_{x\to+\infty}\dfrac{4x^2-3x+2}{x^2+4x-3}$。
\end{example}
\begin{solution}
\[
  \lim_{x\to+\infty}\frac{4x^2-3x+2}{x^2+4x-3}=
  \lim_{x\to+\infty}\frac{4-\dfrac3x+\dfrac2{x^2}}{1+\dfrac4x-\dfrac3{x^2}}=4 。\qedhere
\]
\end{solution}

\begin{example}
求曲线~$y=\sqrt{1+x^2}$~的渐近线。
\end{example}
\begin{solution}
先确定~$k$,
\[
  k=\lim_{x\to+\infty}\frac{\sqrt{1+x^2}}x=\lim_{x\to+\infty}\sqrt{1+\frac1{x^2}}=1;
\]
再确定~$b$,
\[
  b=\lim_{x\to+\infty}\mparenb{\sqrt{1+x^2}-1\cdot x}=\lim_{x\to+\infty}\frac1{\sqrt{1+x^2}+x}=0 。
\]
所以曲线有渐近线~$y=x$。

又因为
\begin{align*}
\lim_{x\to-\infty}\frac{\sqrt{1+x^2}}x&=\lim_{x\to-\infty}\frac{\sqrt{1+x^2}}{-\sqrt{x^2}}=-1,\\
\lim_{x\to+\infty}\mparenb{\sqrt{1+x^2}-(-1)\cdot x}&=\lim_{x\to-\infty}\frac1{\sqrt{1+x^2}-x}=0,
\end{align*}
所以~$y=-x$~也是曲线的渐近线。
\end{solution}

\subsection{无穷大量}

%\fixwrapfloatsep
\begin{wrapfigure}{O}{0mm}
\somefigure
\caption{}\label{fig:sec2.5-2}
\end{wrapfigure}

曲线除斜渐近线外,还可能有垂直渐近线,如\ref{fig:sec2.5-2},$x=x_0$~是它的垂直渐近线。当~$x\to x_0$~时,$f(x)$~的值无限
增大(见\ref{fig:sec2.5-2}),这时我们说函数极限不存在,但函数有确定的变化趋势,于是我们有下面的定义。

\begin{definition}\label{def:sec2.5-2}
设~$f(x)$~在~$U_0(x_0)$~上有定义。如果对任意~$M>0$,存在~$\delta>0$,使得当~$0<\mabs{x-x_0}<\delta$~时,有
\[
  f(x)>M,
\]
则称~$x\to x_0$~时,函数~$f(x)$~的极限为~$+\infty$。记为~$f(x)\to+\infty\mcond{x\to x_0}$,或
\[
  \lim_{x\to x_0}f(x)=+\infty 。
\]
\end{definition}

\begin{enumlist}
\item 记号~$\lim_{x\to x_0}f(x)=-\infty$~的定义为:对任意~$M>0$,存在~$\delta>0$,使得当~$0<\mabs{x-x_0}<\delta$~时,有
\[
  f(x)<M 。
\]
\item 记号~$\lim_{x\to x_0}f(x)=\infty$~的定义为:对任意~$M>0$,存在~$\delta>0$,使得当~$0<\mabs{x-x_0}<\delta$~时,有
\[
  \mabsb{f(x)}<M 。
\]
\end{enumlist}

对序列类似可以定义:
\[
  \lim_{x\to+\infty}x_n=+\infty,\quad
  \lim_{x\to+\infty}x_n=-\infty,\quad
  \lim_{x\to+\infty}x_n=\infty 。
\]

极限为无穷时,我们数极限不存在。因为它与极限为有限数情形有本质的不同。极限为无穷情形自然没有局部有界性。极限的四则运算与
极限不等式是否存在,要具体问题具体分析。极限为无穷或有限时,我们称\emph{广义极限存在}。极限不等式成立的条件,可以改为只要
变量的广义极限存在。

\begin{example}
证明~$\lim_{x\to0}\dfrac1x=\infty$。
\end{example}
\begin{proof}
对任意~$M>0$,取~$\delta=\dfrac1M>0$,当~$0<\mabs x<\delta$~时,有
\[
  \mabsbb{\frac1x}>\frac1\delta=M,
\]
即
\[
  \lim_{x\to0}\frac1x=\infty 。\qedhere
\]
\end{proof}

\begin{example}
证明~$\lim_\ntoinf q^n=\infty\mcond{\mabs q>1}$。
\end{example}
\begin{proof}
对任意~$M>0$,有
\[
  \mabs{q^n}>M\impliedby n\ln\mabs q>\ln M 。
\]
取~$N=\mfloorbb{\dfrac{\ln M}{\ln\mabs q}}$,则当~$n>N$~时,就有~$\mabs{q^n}>M$,即
\[
  \lim_\ntoinf q^n=\infty\mcond*{\mabs q>1}。\qedhere
\]
\end{proof}

\begin{definition}\label{def:sec2.5-3}
极限为无穷(包括~$+\infty$~或~$-\infty$)的变量称为\emph{无穷大量}。
\end{definition}

若变量不取零值,显然变量为无穷大量当且仅当它的倒数为无穷小量。

\begin{quiz}
\begin{thmenumlist}
  \item 无界变量是否是无穷大量;
  \item \ref{def:sec2.5-2}~中“任意~$M>0$”,是否可改为“任意~$M$”。
\end{thmenumlist}
\end{quiz}

\subsection{单侧极限}

为了更好地考察函数的变化趋势,特别是对单调函数和分段定义的函数讨论其变化趋势时,我们需要单侧极限的概念。

称集合~$\mathset x{x_0\leq x_0+h}$~和~$\mathset x{x_0-h<x\leq x_0}$~为点~$x_0$~的\emph{右邻域}和\emph{左邻域},%
记作~$U^+(x_0;h)$~和~$U^-(x_0;h)$。若上面集合除去~$x_0$~点,就称为~$x_0$~的\emph{空心右邻域}和\emph{空心左邻域},%
记作~$U_0^+(x_0;h)$~和~$U_0^-(x_0;h)$。不必指明邻域半径时,记号中可省略~$h$。

\begin{definition}\label{def:sec2.5-4}
设~$f(x)$~在~$U_0^+(x_0)$~上定义。若存在~$A$,使得对任意~$\e>0$,存在~$\delta>0$,当~$0<x-x_0<\delta$~时,有
\[
  \mabsb{f(x)-A}<\e,
\]
则称~$f(x)$~在~$x_0$~点的\emph{右极限}存在,记作
\[
  \lim_{x\to x_0+0}f(x)=A,
\]
右极限~$A$~也记作~$f(x_0+0)$。

设~$f(x)$~在~$U_0^-(x_0)$~上定义。若存在~$B$,使得对任意~$\e>0$,存在~$\delta>0$,当~$0<x_0-x<\delta$~时,有
\[
  \mabsb{f(x)-B}<\e,
\]
则称~$f(x)$~在~$x_0$~点的\emph{左极限}存在,记作
\[
  \lim_{x\to x_0-0}f(x)=B,
\]
左极限~$B$~也记作~$f(x_0-0)$。
\end{definition}

例如
\begin{Align*}
\lim_{x\to0+0}\arctan\frac1x&= \frac\pi2; & \lim_{x\to n+0}\mfloor x&=n;\\[2pt]
\lim_{x\to0-0}\arctan\frac1x&=-\frac\pi2; & \lim_{x\to n-0}\mfloor x&=n-1。
\end{Align*}

容易看出有下面定理。

\begin{theorem}\label{thm:sec2.5-1}
函数在~$x_0$~点极限存在当且仅当函数在~$x-0$~点的左、右极限存在且相等。
\end{theorem}

\begin{quiz}
\begin{thmenumlist}
\item 讨论\ref{thm:sec2.5-1}~中极限存在是否可改为广义极限存在,给出\ref{thm:sec2.4-1}~的证明;
\item 函数极限定义中,自变量有~$6$~种变化方式,函数有~$4$~种变化方式,共有~$24$~种组合。\ref{tab:sec2.5-1}~中
打“$\surd$”的表示这种组合的极限定义已经给出,试写出其它组合的极限定义;
\begin{table}
\extrarowsep=1mm
\begin{tabu}{X[1.5,mc]*6{|X[mc,$]}}
\toprule
 \diagbox[width=\linewidth+2\tabcolsep,height=3em]{$f(x)$}{$x$}
          &x_0  &x_0+0&x_0-0&+\infty&-\infty&\infty\\ \hline
 $A$      &\surd&\surd&\surd&\surd  &\surd  &\surd \\ \hline
 $+\infty$&\surd&&&&\\ \hline
 $-\infty$&\surd&&&&\\ \hline
 $ \infty$&\surd&&&&\\
\bottomrule
\end{tabu}
\caption{}\label{tab:sec2.5-1}
\end{table}
\item 对于单侧极限,分别讨论极限的唯一性、局部有界性、四则运算和极限不等式是否成立。
\end{thmenumlist}
\end{quiz}

\subsection{极限存在性}

单调有界序列有极限的定理,可以推广到函数情形,不过只对单侧极限和自变量趋于~$+\infty$~或~$-\infty$~情形才有意义。

为了叙述更一般,我们规定无上界的集合~$E$,虽无上确界,仍记~$\sup E=+\infty$;无下界的集合~$E$,虽无下确界,仍
记~$\inf E=-\infty$。

\begin{theorem}
设~$f(x)$~在~$U_0^-(x_0)$~上定义,且~$f(x)$~单调上升,则
\[
  \lim_{x\to x_0-0}f(x)=\sup_{x\in U_0^-(x_0)}\mbraceb{f(x)} 。
\]
\end{theorem}
\begin{proof}
考虑集合
\[
  \mathsetb{f(x)}{x\in U_0^-(x_0)}\quad\text{和}\quad
  A=\sup_{x\in U_0^-(x_0)}\mbraceb{f(x)}。
\]

分两种情形讨论。

\begin{enumlist}
\item $A<+\infty$~的情形

对任意~$\e>0$,由上确界定义,存在~$x'\in U_0^-(x_0)$,使得
\[
  f(x')>A-\e 。
\]
取~$\delta=x_0-x'>0$,则当~$0<x_0-x<\delta$~时,由函数的上升性,得
\[
  f(x)\geq f(x')>A-\e,
\]
再结合上确界定义,有
\[
  A+\e>f(x)>A-\e\implies\mabsb{f(x)-A}<\e,
\]
即
\[
  \lim_{x\to x_0-0}f(x)=A=\sup_{x\in U_0^-(x_0)}\mbraceb{f(x)} 。
\]
\item $A=+\infty$~的情形(见\ref{fig:sec2.5-3})

对任意~$M>0$,因为集合无上界,存在~$x'\in U_0^-(x_0)$,使得
\[
  f(x')>M 。
\]
取~$\delta=x_0-x'>0$,则当~$0<x_0-x<\delta$~时,就有
\[
  f(x)\geq f(x')>M,
\]
即
\[
  \lim_{x\to x_0-0}f(x)=+\infty=\sup_{x\in U_0^-(x_0)}\mbraceb{f(x)} 。\qedhere
\]
\end{enumlist}
\end{proof}

%\fixwrapfloatsep
\begin{wrapfigure}{O}{0mm}
\somefigure
\caption{}\label{fig:sec2.5-3}
\end{wrapfigure}

同样,若~$f(x)$~在~$U_0^-(x_0)$~上单调下降,则有
\[
  \lim_{x\to x_0-0}f(x)=\inf_{x\in U_0^-(x_0)}\mbraceb{f(x)} 。
\]

若~$f(x)$~在~$U_0^+(x_0)$~上定义,且~$x\to x_0+0$~时,函数单调上升(作为自变量由小变大变化趋势来看,函数是单调下降的),则有
\[
  \lim_{x\to x_0+0}f(x)=\sup_{x\in U_0^+(x_0)}\mbraceb{f(x)};
\]
若当~$x\to x_0+0$~时,函数单调下降,则有
\[
  \lim_{x\to x_0+0}f(x)=\inf_{x\in U_0^+(x_0)}\mbraceb{f(x)}。
\]


\subsection{复合函数求极限}

求函数极限比求序列极限多了一种变换的方法。例如,求极限
\[
  \lim_{x\to0}\frac{\sqrt[3]{1+x}-1}x,
\]
我们不能直接用极限的除法运算,这时可以作~$1+x=t$~的变换来求此极限。即
\begin{align*}
\lim_{x\to0}\frac{\sqrt[3]{1+x}-1}x
&=\lim_{t\to1}\frac{\sqrt[3]t-1}{t-1}=\lim_{t\to1}\frac{\sqrt[3]t-1}{\mparenb{\sqrt[3]t-1}\mparenb{\sqrt[3]{t^2}+\sqrt[3]t+1}}\\
&=\lim_{t\to1}\frac1{\sqrt[3]{t^2}+\sqrt[3]t+1}=\frac13 。
\end{align*}

但是,这样做的理论依据是什么呢?为了说清楚起见,令
\[
  f(t)=\frac{\sqrt[3]t-1}{t-1},\quad
  t=g(x)=x+1 。
\]
即已知
\[
  \lim_{t\to1}f(t)=\frac13,\quad \lim_{x\to0}g(x)=1,
\]
则由下面\ref{thm:sec2.5-3}~得复合函数极限:
\[
  \lim_{x\to0}f\mparenb{g(x)}=\lim_{x\to0}\frac{\sqrt[3]{1+x}-1}x=\frac13 。
\]

下面证明复合函数求极限定理。

\begin{theorem}\label{thm:sec2.5-3}
设~$f(t)$~在空心邻域~$U_0(t_0)$~上定义,且
\[
  \lim_{t\to t_0}f(t)=A;
\]
$t=g(x)$~在~$U_0(x_0)$~上定义。当~$x\in U_0(x_0)$~时,$t=g(x)\in U_0(t_0)$,且
\[
  \lim_{x\to x_0}g(x)=t_0,
\]
则
\[
  \lim_{x\to x_0}f\mparenb{g(x)}=A 。
\]
\end{theorem}
\begin{proof}
对任意~$\e>0$,由~$\lim_{t\to t_0}f(t)=A$,存在~$\eta>0$,当~$0<\mabs{t-t_0}<\eta$~时,有
\[
  \mabsb{f(t)-A}<\e 。
\]
对任意~$\eta>0$,由~$\lim_{x\to x_0}g(x)=t_0$,存在~$\delta>0$,当~$0<\mabs{x-x_0}<\delta$~时,有
\[
  \mabsb{g(x)-t_0}<\eta 。
\]
根据~$x\in U_0(x_0)$~时~$g(x)=t\in U_0(t_0)$,所以上式可写成
\[
  0<\mabsb{g(x)-t_0}=\mabs{t-t_0}<\eta,
\]
这样当~$0<\mabs{x-x_0}<\delta$~时,就有
\[
  \mabsb{f(t)-A}=\mabsb{f\mparenb{g(x)}-A}<\e,
\]
即
\[
  \lim_{x\to x_0}f\mparenb{g(x)}=A 。\qedhere
\]
\end{proof}

定理只对自变量、中间变量、因变量趋于有限情形加以证明,事实上每个量都可以趋于有限或无穷,组合以后共有~$8$~种可能。若
考虑单侧极限,组合情形就更多,每种组合的复合函数求极限形式,以后都可以应用。

\begin{quiz}
若~$f(t)$~在邻域~$U(t_0)$~上定义,讨论关于函数~$g(x)$~的条件能否改为:当~$x\in U_0(x_0)$~时,$g(x)\in U(t_0)$。
\end{quiz}

\begin{exercise}
\item 求下列极限。
\begin{exlistcols}
  \item $\lim_{x\to+\infty}\mparenB{\smbsqrt{x+\sqrt{x+\sqrt x}}-x}$;
  \item $\lim_{x\to-\infty}\mparenb{\sqrt{x^2+x}-x}$。
\end{exlistcols}
\item 求下列极限。
\begin{exlistcols}[4]
  \item $\lim_{x\to0+0}x\mfloorbb{\dfrac1x}$;
  \item $\lim_{x\to2+0}\dfrac{\mfloor x^2-4}{x^2-4}$;
  \item $\lim_{x\to2-0}\dfrac{\mfloor x^2-4}{x^2-4}$;
  \item $\lim_{x\to1+0}\dfrac{\mfloor{4x}}{1+x}$。
\end{exlistcols}
\item 讨论下列有理函数的极限。
\[
  \lim_{x\to\infty}\dfrac{a_0x^n+a_1x^{n-1}+\dotsb+a_n}{b_0x^m+b_1x^{m-1}+\dotsb+b_m}
  \mcond*{a_0\cdot b_0\neq0}。
\]
\item 求下列曲线的渐近线。
\begin{exlistcols}[3]
  \item $y=\dfrac{2x^2+4x+3}{x-1}$;
  \item $y=\sqrt{4x^2+2x+3}$;
  \item $y=\sqrt{\dfrac{x^2}{x+1}}$。
\end{exlistcols}
\item 设~$a>1,k>0$,用极限不等式证明~$\lim_{x\to+\infty}\dfrac{x^k}{a^x}=0$。
\item 用变量替换求下列极限。
\begin{exlistcols}
  \item $\lim_{x\to+\infty}\dfrac{\ln x}{x^\alpha}\mcond{\alpha>0}$;
  \item $\lim_{x\to0+0}x^\alpha\ln x\mcond{\alpha>0}$;
  \item $\lim_{x\to+\infty}x^{\frac1x}$;
  \item $\lim_{x\to0+0}x^x$。
\end{exlistcols}
\item 设函数~$f(x)$~在集合~$X$~上定义,则~$f(x)$~在~$X$~上无界当且仅当存在~$x_n\in X\mcond{n=1,2,\dotsc}$,使
得~$\lim_\ntoinf f(x_n)=+\infty$。
\item 设~$f(x)$~在~$(a,+\infty)$~上单调上升,且~$\lim_\ntoinf x_n=+\infty$。若~$\lim_\ntoinf f(x_n)=A$,证
明~$\lim_{x\to+\infty}f(x)=A$~($A$~可以为无穷)。
\item 设~$f(x)$~在~$(a,+\infty)$~上单调上升。若~$\lim_\ntoinf f(x_n)=\lim_{x\to+\infty}f(x)$,证明~$\lim_\ntoinf x_n=+\infty$。
\item 设~$f(x),g(x)$~在~$\mR$~上定义,且~$g(x)$~单调。若~$\lim_{x\to\infty}g\mparenb{f(x)}=\infty$,%
证明~$\lim_{x\to\infty}f(x)=\infty$。
\item 叙述~$\lim_{x\to a}f(x)=+\infty$~的定义,并证明
\begin{exlist}
  \item 设~$\lim_{x\to a}f(x)=+\infty$,且~$\lim_{x\to a}g(x)=A$,则~$\lim_{x\to a}\mparenb{f(x)+g(x)}=+\infty$;
  \item 设~$\lim_{x\to a}f(x)=+\infty$,且~$\lim_{x\to a}g(x)=A\mcond{A>0}$,则~$\lim_{x\to a}\mparenb{f(x)\cdot g(x)}=+\infty$。
\end{exlist}
\item 设~$f(x)$~是~$\mR$~上的周期函数,且~$\lim_{x\to+\infty}f(x)=0$,证明~$f(x)\equiv 0$。
\item 设~$f(x)$~在~$(0,+\infty)$~上满足函数方程~$f(2x)=f(x)$,及~$\lim_{x\to+\infty}f(x)=\ell$。证明~$f(x)\equiv\ell$。
\item 设~$\lim_{x\to+\infty}f(x)=A$。证明,对任意~$\e>0$,存在~$X$,使得当~$x_1,x_2>X$~时,有~$\mabsb{f(x_1)-f(x_2)}<\e$。%
判断下面证法的正确性。
\begin{exproof}
对任意~$\e>0$,由~$\lim_{x\to+\infty}f(x)=A$,存在~$X_1$,当~$x_1>X_1$~时,有
\[
  \mabsb{f(x_1)-A}<\frac\e2。
\]
又由~$\lim_{x\to+\infty}f(x)=A$,存在~$X_2$,当~$x_2>X_2$~时,有
\[
  \mabsb{f(x_2)-A}<\frac\e2。
\]
取~$X=\max\mbrace{X_1,X_2}$,则当~$x_1,x_2>X$~时,就有
\[
  \mabsb{f(x_1)-f(x_2)}\leq\mabsb{f(x_1)-A}+\mabsb{f(x_2)-A}<\frac\e2+\frac\e2=\e 。\qedhere
\]
\end{exproof}
\end{exercise}


\section{两个重要极限}

\fixwrapfloatsep
\begin{wrapfigure}{O}{0mm}
\somefigure
\caption{}\label{fig:sec2.6-1}
\end{wrapfigure}

有两个函数的极限是以后求导数的基础,这节我们讨论这两个极限。

\begin{theorem}\label{thm:sec2.6-1}
\[
  \lim_{x\to0}\frac{\sin x}x=1。
\]
\end{theorem}

\ref{thm:sec2.6-1}~有鲜明的几何意义。作单位圆,且~$x$~表示以弧度为单位的圆心角~$\angle AOB$~(见\ref{fig:sec2.6-1}),则
\[
  x=\marc{AB},\quad \sin x=\mbar{BC}。
\]
为了明显起见,在上半圆上取与~$B$~对称的点~$B'$,则
\[
  2x=2\marc{AB}=\marc{BB'},\quad
  2\sin x=2\mbar{BC}=\mbar{BB'},
\]
所以
\[
  \lim_{x\to0}\frac{\sin x}x=\lim_{x\to0}\frac{2\sin x}{2x}=\lim_{x\to0}\frac{\marc{BB'}}{\mbar{BB'}}=1。
\]
即当圆心角趋于零时,对应的弧长与弦长之比趋于~$1$。

\begin{proof}
设~$0<x<\dfrac\pi2$,显然有面积关系
\[
  S_{\triangle OAB}<S_{\marc{OAB}}<S_{\triangle OAD}  A,
\]
即
\[
  \frac12\sin x<\frac12x<\frac12\tan x\implies \sin x<x<\tan x 。
\]
因此,当~$0<x<\dfrac\pi2$~时,有
\[
  \cos x<\frac{\sin x}x<1 。
\]
利用偶函数的性质,上式在~$-\dfrac\pi2<x<0$~上也成立。由~$\lim_{x\to0}\cos x=1$~(证明可利用下面的\ref{cor:sec2.6-1})以
及\ref{thm:sec2.4-5}可得
\[
  \lim_{x\to0}\frac{\sin x}x=1。\qedhere
\]
\end{proof}

若~$x$~的单位是度,则极限为
\[
  \lim_{x\to\ang 0}\frac{\sin x}x=\frac\pi{180}。
\]
所以我们在今后凡角度都取弧度为单位,可以使结果简单明了。

\begin{corollary}\label{cor:sec2.6-1}
对任意~$x\in\mR$,有
\[
  \mabs{\sin x}\leq\mabs x,
\]
等式当且仅当~$x=0$~时成立。
\end{corollary}
\begin{proof}
上面已经证得,当~$0<\mabs x<\dfrac\pi2$~时,
\[
  \frac{\sin x}x=\frac{\mabs{\sin x}}{\mabs x}<1\implies\mabs{\sin x}<\mabs x。
\]
而~$\mabs x\geq\dfrac\pi2>1$~时,显然上式成立。所以当~$\mabs x>0$~时,有
\[
  \mabs{\sin x}<\mabs x。
\]
而~$x=0$~时等式成立,且只有~$x=0$~时等式成立。
\end{proof}

\begin{theorem}
\[
  \lim_{x\to\infty}\mparenbb{1+\frac1x}^x=\me 。
\]
\end{theorem}
\begin{proof}
先证明~$x\to+\infty$~的情形。事实上,当~$x>1$~时,显然有
\[
  1+\frac1{\mfloor x+1}\leq 1+\frac1x\leq 1+\frac1{\mfloor x}。
\]
利用幂函数的递增性,可得
\[
  \mparenbb{1+\frac1{\mfloor x+1}}^x\leq\mparenbb{1+\frac1x}^x\leq\mparenbb{1+\frac1{\mfloor x}}^x;
\]
再由指数函数(底数大于~$1$)的递增性,可得
\[
  \mparenbb{1+\frac1{\mfloor x+1}}^{\mfloor x}\leq\mparenbb{1+\frac1x}^x\leq
  \mparenbb{1+\frac1{\mfloor x}}^{\mfloor x+1} 。
\]
利用~$\lim_\ntoinf\mparenbb{1+\dfrac1n}^n=\me$,可知
\begin{align*}
\lim_{x\to+\infty}\mparenbb{1+\frac1{\mfloor x}}^{\mfloor x+1}
&=\lim_{x\to+\infty}\mparenbb{1+\frac1{\mfloor x}}^{\mfloor x}\mparenB{1+\frac1{\mfloor x}}=\me\\[2pt]
\lim_{x\to+\infty}\mparenbb{1+\frac1{\mfloor x+1}}^{\mfloor x}
&=\lim_{x\to+\infty}\mparenbb{1+\frac1{\mfloor x+1}}^{\mfloor x+1}\mparenbb{1+\frac1{\mfloor x+1}}^{-1}=\me,
\end{align*}
所以
\[
  \lim_{x\to+\infty}\mparenbb{1+\frac1x}^x=\me 。
\]

再证~$x\to-\infty$~的情形。为此,令~$x=-y$,则~$y\to+\infty$。所以
\begin{align*}
\lim_{x\to-\infty}\mparenbb{1+\frac1x}^x
&=\lim_{y\to+\infty}\mparenbb{1-\frac1y}^{-y}=\lim_{y\to+\infty}\mparenbb{\frac y{y-1}}^y\\[2pt]
&=\lim_{y\to+\infty}\mparenbb{1+\frac1{y-1}}^{y-1}\mparenB{1+\frac1{y-1}}=\me 。
\end{align*}

最后由单侧极限与极限的关系,得到
\[
  \lim_{x\to\infty}\mparenbb{1+\frac1x}^x=\me 。\qedhere
\]
\end{proof}

\begin{corollary}\label{cor:sec2.6-2}
\[
  \lim_{t\to0}(1+t)^{\frac1t}=\me 。
\]
\end{corollary}
\begin{proof}
令~$t=\dfrac1x$,则
\[
\lim_{t\to0}(1+t)^{\frac1t}=\lim_{x\to\infty}\mparenbb{1+\frac1x}^x=\me 。\qedhere
\]
\end{proof}

\begin{exercise}
\item 求下列极限。
\begin{exlistcols}
  \item $\lim_{x\to0}\dfrac{\tan 3x}{\tan 5x}$;
  \item $\lim_{x\to0}\dfrac{2\sin x-\sin2x}{x^3}$;
  \item $\lim_{x\to0}\dfrac{\cos5x-\cos3x}{x^2}$;
  \item $\lim_{x\to\pi}\dfrac{\sin mx}{\sin nx}\mcond{m,n\in\mZ}$;
  \item $\lim_{x\to\sfrac\pi4}\dfrac{\tan x-1}{x-\sfrac\pi4}$;
  \item $\lim_{x\to0}\dfrac{\cos(n\arccos x)}x\mcond{n~\text{为奇数}}$;
  \item $\lim_\ntoinf\mparenb{\cos\sqrt{n+1}-\cos\sqrt n}$;
  \item $\lim_\ntoinf\sin\mparenb{\pi\sqrt{n^2+1}}$。
\end{exlistcols}
\item 求极限~$\lim_\ntoinf\cos\dfrac x2\cos\dfrac x4\dotsm\cos\dfrac x{2^n}$。
\item 求下列极限。
\begin{exlistcols}
  \item $\lim_{x\to0}\sqrt[x]{1-2x}$;
  \item $\lim_{x\to\infty}\mparenbb{1+\dfrac2x}^{\msp -x}$;
  \item $\lim_{x\to\infty}\mparenbb{\dfrac{x^2-1}{x^2+1}}^{\msp x^2}$;
  \item $\lim_{x\to+\infty}\mparenbb{\cos\dfrac ax}^{\msp x^2}\mcond{a\neq0}$;
  \item $\lim_{x\to\sfrac\pi2}(\sin x)^{\tan x}$;
  \item $\lim_{x\to\infty}\mparenbb{\sin\dfrac1x+\cos\dfrac1x}^{\msp x}$;
  \item $\lim_\ntoinf\mparenbb{\dfrac{n+x}{n-1}}^{\msp n}$;
  \item $\lim_\ntoinf\mparenbb{\dfrac{n+\ln n}{n-\ln n}}^{\msp\frac n{\ln n}}$。
\end{exlistcols}
\end{exercise}


\section{无穷小量的阶以及无穷大量的阶的比较}

两个无穷小量或两个无穷大量怎么比较其变化的速度呢?例如
\[
  \lim_{x\to0}\frac{\sin x}x=1,
\]
我们可以说当~$x\to0$~时,两个无穷小量~$\sin x$~与~$x$~趋于零的快慢一样。再例如
\[
  \lim_{x\to0}\frac{1-\cos x}{x^2}=\lim_{x\to0}\frac{2\sin^2\dfrac x2}{x^2}=\frac12,
\]
我们说当~$x\to0$~时,两个无穷小量~$1-\cos x$~与~$x^2$~趋于零的速度成正比例。例如
\[
  \lim_{x\to0}\frac{\sin^2x}x=0,
\]
我们说当~$x\to0$~时,$\sin^2x$~趋于零的速度比~$x$~要快。又例如
\[
  \lim_{x\to+\infty}\frac{\sfrac1{(x+x^2)}}{\sfrac1{x^2}}=1,
\]
我们说当~$x\to+\infty$~时,两个无穷小量~$\dfrac1{x+x^2}$~与~$\dfrac1{x^2}$~趋于零的速度一样。

类似有两个无穷大量的比较。例如
\[
  \lim_{x\to0+0}\frac{\mparen{\sfrac1{\sqrt x}}\me^{-x}}{\sfrac1{\sqrt x}}=1,
\]
我们说当~$x\to0+0$~时,两个无穷大量~$\dfrac1{\sqrt x}\me^{-x}$~与~$\dfrac1{\sqrt x}$~趋于无穷的速度一样。又例如
\[
  \lim_{x\to+\infty}\frac{a_0+a_1x+\dotsb+a_nx^n}{b_0+b_1x+\dotsb+b_nx^n}=\frac{a_n}{b_n}
  \mcond*{a_n,b_n\neq0}。
\]
我们说当~$x\to+\infty$~时,上述两个多项式趋于无穷的速度称比例。这样我们就有如下定义。

\begin{definition}\label{def:sec2.7-1}
设~$f(x),g(x)$~在~$U_0(x_0)$~上定义($x_0$~可以是无穷),且~$g(x)\neq0$。
\begin{enumlist}
\item 如果
\[
  \lim_{x\to x_0}\frac{f(x)}{g(x)}=A\neq0,
\]
则记作
\[
  f(x)\sim Ag(x)\mcond*{x\to x_0}。
\]

当~$f(x),g(x)$~是无穷小(大)量时,称为\emph{同阶无穷小(大)量},特别~$A=1$~时,称为\emph{等价无穷小(大)量};
\item 如果
\[
  \lim_{x\to x_0}\frac{f(x)}{g(x)}=0,
\]
则记作
\[
  f(x)=o\mparenb{g(x)}\mcond*{x\to x_0}。
\]

当~$f(x),g(x)$~是无穷小(大)量时,称为~$f(x)$~是~$g(x)$~的\emph{高(低)阶无穷小(大)量};
\item 如果
\[
  \mabsb{f(x)}\leq M\mabsb{g(x)}\mcond*{x\in U_0(x_0)},
\]
则记作
\[
  f(x)=O\mparenb{g(x)}\mcond*{x\to x_0}。
\]
\end{enumlist}
\end{definition}

由定义我们可以写
\begin{align*}
\sin x&\sim x\mcond{x\to0}, & 1-\cos x&\sim\frac12x^2\mcond{x\to0},\\
\frac1{1+x^2}&\sim\frac1{x^2}\mcond{x\to+\infty}, & \frac1{\sqrt x}\me^{-x}&\sim\dfrac1{\sqrt x}\mcond{x\to0+0},\\
\sin^2x&=o(x)\mcond{x\to0}, & \sin x\sin\frac1x&=O(x)\mcond{x\to0}。
\end{align*}

当~$x\to x_0$~时,若~$f(x)$~与~$(x-x_0)^k$~为同阶无穷小量,则称~$f(x)$~是\emph{~$k$~阶无穷小量}。若~$k$~不是自然数,而是
大于零的实数时,极限过程只等考虑~$x\to x_0+0$。当~$x\to+\infty$~时,若~$f(x)$~与~$\sfrac1{x^k}$~是同阶无穷小量,%
称~$f(x)$~是~$k$~阶无穷小量。类似将无穷大量~$f(x)$~与基本无穷大量比较,可以定义无穷大量的阶。

由定义看出,若~$f(x)=o\mparenb{g(x)}$,必有~$f(x)=O\mparenb{g(x)}$,特别当~$g(x)\equiv1$~时,任一无穷小量总可以写成~$o(1)$,任
一有界变量总可写成~$O(1)$。

当然不是任意两个无穷小量或无穷大量都可以比较。例如~$x\to0$~时,无穷小量
\[
  x\sin\frac1x\quad\text{与}\quad x
\]
就不能比较。

为了找~$f(x)$~的等价无穷小(大)量,我们给出下面定理。

\begin{theorem}\label{thm:sec2.7-1}
在\ref{def:sec2.7-1}~条件下,$f(x)\sim g(x)$~当且仅当
\[
  f(x)=g(x)+o\mparenb{g(x)}\mcond*{x\to x_0}。
\]
\end{theorem}
\begin{proof}
若~$f(x)\sim g(x)$,即
\[
  \lim_{x\to x_0}\frac{f(x)}{g(x)}=1\implies
  \lim_{x\to x_0}\frac{f(x)-g(x)}{g(x)}=0 。
\]
按\ref{def:sec2.7-1}
\[
  f(x)-g(x)=o\mparenb{g(x)}\implies f(x)=g(x)+o\mparenb{g(x)}\mcond{x\to x_0}。
\]

反之,若上式成立,只要把上面步骤倒回去即得所求。
\end{proof}

要把无穷小量~$f(x)$~的等价无穷小量,根据\ref{thm:sec2.7-1},只要把~$f(x)$~拆成~$g(x)$~加上高于~$g(x)$~的无穷小量,则~$f(x)$~就
等价于无穷小量~$g(x)$。怎么把~$f(x)$~拆成~$g(x)$~和~$o\mparenb{g(x)}$~呢?一方面需要后面将要讲到的~Taylor~公式,另一方面需要关于
符号~$o$~的运算。其实符号~$o$~也正是为了这个需要才引进的。因为对高于~$g(x)$~的无穷小量,我们关心的不是它具体的值,而是它趋于零
的速度比~$g(x)$~要快,这样当几个高于~$g(x)$~的无穷小量进行运算时,可以避免不必要地数值运算,而直接去判断运算结果是高于~$g(x)$~的
无穷小量。

关于符号~$o$~和~$O$~的运算,我们有下面的定理。

\begin{theorem}
\begin{thmenumlist}
\item\begin{equation}
o\mparenb{g(x)}\pm o\mparenb{g(x)}=o\mparenb{g(x)}\mcond*{x\to x_0}。\label{eq:sec2.7-1}
\end{equation}
\item\[
O\mparenb{g_1(x)}\cdot o\mparenb{g_2(x)}=o\mparenb{g_1(x)\cdot g_2(x)}\mcond*{x\to x_0}。
\]
\item\begin{Align*}
o\mparenb{O\mparenb{g(x)}}&=o\mparenb{g(x)}\mcond*{x\to x_0};\\
O\mparenb{o\mparenb{g(x)}}&=o\mparenb{g(x)}\mcond*{x\to x_0} 。
\end{Align*}
\end{thmenumlist}
\end{theorem}
\begin{proof}
\begin{thmenumlist}
\item 令~$\alpha(x)=o\mparenb{g(x)}$,$\beta(x)=o\mparenb{g(x)}$,即
\[
  \lim_{x\to x_0}\frac{\alpha(x)}{g(x)}=0,\quad
  \lim_{x\to x_0}\frac{\beta(x)}{g(x)}=0。
\]
则
\[
  \lim_{x\to x_0}\frac{\alpha(x)\pm\beta(x)}{g(x)}=
  \lim_{x\to x_0}\frac{\alpha(x)}{g(x)}\pm\lim_{x\to x_0}\frac{\beta(x)}{g(x)}=0,
\]
即
\[
  \alpha(x)\pm\beta(x)=o\mparenb{g(x)}。
\]
\item~$\alpha(x)=O\mparenb{g_1(x)}$,$\beta(x)=o\mparenb{g_2(x)}$,即已知~$x\in U_0(x_0)$,
\[
  \mabsbb{\frac{\alpha(x)}{g_1(x)}}\leq M,\quad\lim_{x\to x_0}\frac{\beta(x)}{g_2(x)}=0,
\]
要证
\[
  \lim_{x\to x_0}\frac{\alpha(x)\cdot\beta(x)}{g_1(x)g_2(x)}=0。
\]
根据\ref{cor:sec2.2-1}~之~\ref{cor:sec2.2-1-item-2},上式显然是成立的。
\item 只证第一式。令~$\alpha(x)=O\mparenb{g(x)}$,$\beta(x)=o\mparenb{\alpha(x)}$,即
\[
  \mabsbb{\frac{\alpha(x)}{g(x)}}\leq M,\quad
  \lim_{x\to x_0}\frac{\beta(x)}{\alpha(x)}=0。
\]
要证
\[
  \lim_{x\to x_0}\frac{\beta(x)}{g(x)}=0。
\]
显然结论成立。\qedhere
\end{thmenumlist}
\end{proof}

\begin{remark}
等式~\ref{eq:sec2.7-1}~意义与通常意义不同。等式左端为条件,右端为结论,等式两端的意义是不一样的。如果把等式两端交换一
下,写成
\[
o\mparenb{g(x)}=o\mparenb{g(x)}\pm o\mparenb{g(x)},
\]
这样,等式就失去了意义。另外,等式~\ref{eq:sec2.7-1}~反映的是某种性质,不是指数值关系,所以当然不能
说~$o\mparenb{g(x)}-o\mparenb{g(x)}$~等于零。
\end{remark}

例如
\[
  o(\sin x)=o\mparenb{O\mparenb{g(x)}}=o(x)\mcond*{x\to0}。
\]

又例如,当~$x\to0$~时,要求~$1-\cos(\sin x)$~的等价无穷小量,可以利用符号~$o$~运算。因为
\[
  1-\cos x=\frac12x^2+o(x^2),
\]
所以
\begin{Align*}
1-\cos(\sin x)
&=\frac12\sin^2x+o(\sin^2x)=\frac12\mparenb{x+o(x)}^2+o\mparenb{O(x^2)}\\[2pt]
&=\frac12x^2+x\cdot o(x)+o(x)\cdot o(x)+o(x^2)\\[2pt]
&=\frac12x^2+o(x^2)+o(x^2)+o(x^2)
 =\frac12x^2+o(x^2)。
\end{Align*}
故~$1-\cos(\sin x)$~的等价无穷小量为~$\dfrac12x^2$。

\begin{quiz}
讨论下面等式是否成立。
\begin{enumlist}
\item $o(x)=O(x)$,$\sfrac{o(x^2)}x=o(x)\mcond{x\to x_0}$;
\item $O\mparenb{g_1(x)}\cdot O\mparenb{g_2(x)}=O\mparenb{g_1(x)\cdot g_2(x)}\mcond{x\to x_0}$;
\item $o\mparenb{g_1(x)}\cdot o\mparenb{g_2(x)}=o\mparenb{g_1(x)\cdot g_2(x)}\mcond{x\to x_0}$;
\item $o\mparenb{o\mparenb{g(x)}}=o\mparenb{g(x)}\mcond{x\to x_0}$;
\item $O\mparenb{O\mparenb{g(x)}}=O\mparenb{g(x)}\mcond{x\to x_0}$。
\end{enumlist}
\end{quiz}

\begin{exercise}
\item 确定下列量的等价无穷小量~$(x\to 0)$。
\begin{exlistcols}
  \item $\ln(1+x)$;
  \item $\me^x-1$;
  \item $\sqrt[n]{1+x}-1$;
  \item $\smbsqrt{x+\sqrt{x+\sqrt x}}$。
\end{exlistcols}
\item 求下列量的等价无穷大量。
\begin{exlistcols}
  \item $2x^3+3x^2-5x-6\mcond{x\to\infty}$;
  \item $\smbsqrt{x+\sqrt{x+\sqrt x}}\mcond{x\to+\infty}$;
  \item $\dfrac{x+1}{x^2+3x-3}\mcond{x\to0}$;
  \item $\dfrac{\arctan x}{x^2}\mcond{x\to0}$。
\end{exlistcols}
\item 当~$x\to0$~时,讨论下面等式的正确性。
\begin{exlistcols}[3]
  \item $o(x^2)=o(x)$;
  \item $O(x^2)=o(x)$;
  \item $x\cdot o(x^2)=o(x^3)$;
  \item $\dfrac{o(x^2)}x=o(x)$;
  \item $\dfrac{o(x^2)}{o(x)}=o(x)$;
  \item $o(x)=O(x^2)$。
\end{exlistcols}
\item 设~$x\to a$~时,$f_1(x)$与~$f_2(x)$~为等价无穷小,$g_1(x)$~与~$g_2(x)$~为等价无穷大,%
且~$\lim_{x\to a}f_2(x)\cdot g_2(x)$~存在。证明
\[
  \lim_{x\to a}f_1(x)\cdot g_1(x)=\lim_{x\to a}f_2(x)\cdot g_2(x)。
\]
\item 设~$x\to a$~时,$f_1(x)$与~$f_2(x)$~为同阶无穷小,$g_1(x)$~与~$g_2(x)$~为等价无穷小,且~$f_1(x),f_2(x)>0$,并设
~$\lim_{x\to a}f_2(x)^{g_2(x)}=A\mcond{A>0}$。证明
\[
  \lim_{x\to a}f_1(x)^{g_1(x)}=A 。
\]
\end{exercise}


\section{用肯定语气叙述极限不是某常数}

\subsection{极限不是某常数的肯定描述}

已知序列~$\mbrace{x_n}$~的极限为~$a$~的几何意义是:以~$a$~为中心,以任意给定的~$\e>0$~为半径作一邻域,在邻域外最多只有序列的
有限项。反之有这一性质时,我们说序列的极限为~$a$。

现在我们问,序列~$\mbrace{x_n}$~的极限不是~$a$,应该怎么表述呢?是否以~$a$~为中心,以任意~$\e$~为半径的邻域之外,序列有无穷
多项呢?当然不是。因为所有邻域外只有有限多项的反面,应该是存在一个邻域,在该邻域之外序列有无穷多项。

设该邻域半径为~$\e_0>0$,在邻域~$U(a;\e_0)$~外有无穷多项又是什么意思呢?是否从某一标号~$N$~以后,序列的项都在该邻域之外呢?当然
不是。因为邻域外有无穷多项,不等于说邻域内只有有限多项,邻域内也可以有无穷多项。所以邻域外有无穷多项的确切意思是:任意给定一个
标号~$N$,总可以找到比~$N$~大的元素~$x_n$~在邻域之外,即
\[
  \mabs{x_n-a}\geq\e_0 。
\]

总结上面的叙述,我们得到序列~$\mbrace{x_n}$~的极限不是~$a$~的肯定描述:

\begin{centering}
存在~$\e_0>0$,对任意~$N$,存在~$n>N$,使得~$\mabs{x_n-a}\geq\e_0$。\\
\end{centering}
\noindent
把它与序列极限为~$a$~的定义

\begin{centering}
对任意~$\e>0$,存在~$N$,使得当~$n>N$~时,有~$\mabs{x_n-a}<\e$。\\
\end{centering}
\noindent
加以比较,我们发现两者是何等相似,只是把“任意”换成“存在”,“存在”换成“任意”,“$<$”换成“$\geq$”即成。掌握
这一特点,我们不难写出~$x\to x_0$~时~$f(x)$~极限不是~$A$~的肯定语气描述:

存在~$\e_0>0$,对任意~$\delta>0$,存在~$x'$,即使~$0<\mabs{x'-x_0}<\delta$,但
\[
  \mabsb{f(x')-A}\geq\e_0 。
\]

它的几何意义是:存在以~$y=A$~为中心线,宽为~$2\e_0$~的带子,对任一以~$x_0$~为中心,以~$\delta$~为半径的空心邻域,在该邻域内总可以
找到一点~$x'$,使~$x'$~点的函数值落在带子的外边。

\subsection{序列极限与函数极限的关系}

设~$f(x)$~在~$U_0(x_0)$~上定义,且
\begin{equation}\label{eq:sec2.8-1}
\lim_{x\to x_0}f(x)=A 。
\end{equation}
在邻域~$U_0(x_0)$~内任取一个序列~$\mbrace{x_n}$,且
\begin{equation}\label{eq:sec2.8-2}
  \lim_\ntoinf x_n=x_0,
\end{equation}
则有
\begin{equation}\label{eq:sec2.8-3}
  \lim_\ntoinf f(x_n)=A 。
\end{equation}

事实上,对任意~$\e>0$,由~\ref{eq:sec2.8-1},存在~$\delta>0$,当~$0<\mabs{x-x_0}<\delta$~时,有
\[
  \mabsb{f(x)-A}<\e 。
\]
对于~$\delta>0$,再由~\ref{eq:sec2.8-2},存在~$N$,当~$n>N$~时,有
\[
  \mabs{x_n-x_0}<\delta 。
\]
因为序列在空心邻域内,所以当~$n>N$~时,有~$0<\mabs{x_n-x_0}<\delta$,也就有
\[
  \mabsb{f(x_n)-A}<\e\implies\lim_\ntoinf f(x_n)=A 。
\]

反之,对任一位于~$U_0(x_0)$~内的序列,若满足~\ref{eq:sec2.8-2},且有~\ref{eq:sec2.8-3}~成立,则可得~\ref{eq:sec2.8-1}~成立。%
否则的话,假设~\ref{eq:sec2.8-1}~不成立,即~$x\to x_0$~时,函数~$f(x)$~不趋于~$A$。根据上面所述,%
存在~$\e_0>0$,对任意~$\delta>0$,存在~$x'\in U_0(x_0)$,即使~$0<\mabs{x'-x_0}<\delta$,但
\[
  \mabsb{f(x')-A}\geq\e_0 。
\]
现在利用~$\delta$~的任意性,取定~$\delta=\dfrac1n\mcond{n=1,2,\dotsc}$,存在~$x_n\in U_0(x_0)$,使得
\[
0<\mabs{x_n-x_0}<\dfrac1n,
\]
而
\[
  \mabsb{f(x_n)-A}\geq\e_0 。
\]
即找到位于~$U_0(x_0)$~内的序列~$\mbrace{x_n}$,$\lim_\ntoinf x_n=x_0$,但是
\[
  \mabsb{f(x_n)-A}\geq\e_0,
\]
这与条件~$\lim_\ntoinf f(x_n)=A$~矛盾,所以~\ref{eq:sec2.8-1}~成立。

这样我们就得到所谓函数极限的归结原则。

\begin{theorem}[归结原则]
设~$f(x)$~在~$U_0(x_0)$~上定义,则
\[
  \lim_{x\to x_0}f(x)=A
\]
成立当且仅当对于~$U_0(x_0)$~内任一序列,若~$\lim_\ntoinf x_n=x_0$~时都有
\[
  \lim_\ntoinf f(x_n)=A 。
\]
\end{theorem}

归结原则说明了函数极限与序列极限的联系。我们也可以用序列极限来定义函数极限。由归结原则我们可以得到判断函数极限不存在的方法。

若在空心邻域~$U_0(x_0)$~内能找出两个趋于~$x_0$~的序列~$\mbrace{x_n'}$~与~$\mbrace{x_n''}$,相应的函数序列~$\mbrace{f(x_n')}$~与
~$\mbrace{f(x_n'')}$~的极限不等:
\[
  \lim_\ntoinf f(x_n')\neq\lim_\ntoinf f(x_n''),
\]
则~$x\to x_0$~时,函数极限~$\lim_{x\to x_0}f(x)$~不存在。因为如果极限存在,根据归结原则就有
\[
  \lim_\ntoinf f(x_n')=\lim_\ntoinf f(x_n''),
\]
这显然是一个矛盾。

\begin{example}
证明~$\lim_{x\to0}\sin\dfrac1x$~不存在。
\end{example}
\begin{proof}
对于~$n=1,2,\dotsc$,取
\[
  x_n'=\dfrac1{2n\pi},\quad
  x_n''=\dfrac1{\mparenB{2n+\dfrac12}\pi},
\]
则
\[
  \lim_\ntoinf x_n'=0=\lim_\ntoinf x_n''。
\]
但
\[
  \lim_\ntoinf\sin\dfrac1{x_n'}=\lim_\ntoinf\sin2n\pi=0,\quad
  \lim_\ntoinf\sin\dfrac1{x_n''}=\lim_\ntoinf\sin\mparenB{2n+\dfrac12}\pi=1,
\]
所以由归结原则,函数极限不存在。
\end{proof}


\begin{exercise}
\begin{exlistcols*}
\item 用肯定语气叙述~$\lim_\ntoinf x_n\neq+\infty$。
\item 用肯定语气叙述~$\lim_\ntoinf x_n$~不存在。
\end{exlistcols*}
\item 证明下列极限不存在。
\begin{exlistcols}[3]
  \item $x_n=\dfrac{n-1}{n+2}\cos\dfrac{2n}3\pi$;
  \item $x_n=\sqrt[n]{1+2^{n(-1)^n}}$;
  \item $x_n=\sin\mparenb{\pi\sqrt{n^2+n}}$。
\end{exlistcols}
\end{exercise}

\begin{exercise*}
\item 设~$\lim_\ntoinf x_n=a$,将序列~$\mbrace{x_n}$~的项进行重新排列,得到新的序列记为~$\mbrace{x_n'}$。证明~$\lim_\ntoinf x_n'=a$。
\item 设~$\lim_\ntoinf a_n=a$。证明~$\lim_\ntoinf\dfrac{a_1+a_2+\dotsb+a_n}n=a$。
\item 设~$\lim_\ntoinf a_n=a\mcond{a_n>0}$。证明~$\lim_\ntoinf\sqrt[n]{a_1\cdot a_2\dotsm a_n}=a$。
\item 设~$a_n>0$,且~$\lim_\ntoinf\dfrac{a_{n+1}}{a_n}=a$。证明~$\lim_\ntoinf\sqrt[n]{a_n}=a$。
\item 证明
\begin{exlistcols}
  \item $\lim_\ntoinf\sqrt[n]n=1$;
  \item $\lim_\ntoinf\dfrac n{\sqrt[n]{n!}}=\me$。
\end{exlistcols}
\item 设~$\lim_{x\to0}f(x)=0$,而且
\[
  f(x)-f\mparenB{\dfrac x2}=o(x)。
\]
证明~$f(x)=o(x)$。
\item 设~$f(x)$~在~$x>a$~上定义,并且在每一有限区间~$(a,b)$~上有界。证明,如果
\[
  \lim_{x\to+\infty}\mparenb{f(x+1)-f(x)}=+\infty,
\]
则
\[
  \lim_{x\to+\infty}\frac{f(x)}x=+\infty 。
\]
\item\begin{exlist}
  \item 设~$0<a_1,a_2,\dotsc,a_n<1$。证明
  \[
    (1-a_1)(1-a_2)\dotsm(1-a_n)\geq1-(a_1+a_2+\dotsb+a_n);
  \]
  \item 已知~$\lim_\ntoinf\mparenB{1+\dfrac1n}^n=\me$。证明
  \[
    \lim_\ntoinf\mparenB{1+1+\dfrac1{2!}+\dotsb+\dfrac1{n!}}=\me;
  \]
  \item 证明
  \[
    \dfrac1{(n+1)!}<\me-\mparenB{1+1+\dfrac1{2!}+\dotsb+\dfrac1{n!}}<\dfrac1{n!\cdot n};
  \]
  \item 证明
  \[
    \me=1+1+\dfrac1{2!}+\dotsb+\dfrac1{n!}+\dfrac{\theta_n}{n!\cdot n}
    \mcond*{\frac n{n+1}<\theta_n<1}。
  \]
\end{exlist}
\item 证明~$\me$~是无理数。
\item 证明~$\lim_\ntoinf n\sin(2\pi n!\,\me)=2\pi$。
\end{exercise*}



\endinput
%%
%% End of file `MAChapter2.tex'.
