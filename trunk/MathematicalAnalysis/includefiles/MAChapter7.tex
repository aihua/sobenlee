%# -*- coding:utf-8 -*-
%%%%%%%%%%%%%%%%%%%%%%%%%%%%%%%%%%%%%%%%%%%%%%%%%%%%%%%%%%%%%%%%%%%%%%%%%%%%%%%%%%%%%
%%  MAChapter7.tex'


\chapter{定积分}\label{ch:7}
\section{定积分的概念}
\subsection{实际问题中的例}
\subsection{定积分的概念及几何意义}
\begin{exercise}
\item 按定义判断下列函数在给定区间上是否可积?倘若可积,试求出积分值。
\begin{exlistcols}
  \item $f(x)=1$,$x\in\mintc {-1}1$;
  \item $f(x)=x$,$x\in\mintc {-1}1$;
  \item $f(x)=\begin{Bdcases}
    \frac1x, & \mabs x\leq1~\text{且}~x\neq0; \\ 0 , & x=0 。
  \end{Bdcases}$
\end{exlistcols}
\item 已知下列函数可积,用定义求下列积分,分法任意\mcond{b>a>0}。
\begin{exlistcols}
  \item $\int_a^bx\dif x$~(按算术平均值取点);
  \item $\int_a^b\dfrac1{x^2}\dif x$~(按几何平均值取点)。
\end{exlistcols}
\item 利用定积分的几何意义,求下列积分。
\begin{exlistcols}[3]
  \item $\int_a^bx\dif x$;
  \item $\int_a^b\txts\sqrt{(x-a)(b-x)}\dif x$;
  \item $\int_a^b\mabsbb{x-\dfrac{a+b}2}\dif x$。
\end{exlistcols}
\item 设~$f(x)$~在~$\mintc ab$~上可积。证明,
\begin{exlist}
  \item $\int_a^bf(x)\dif x=\lim_\ntoinf\dfrac{b-a}n\sum_{i=0}^{n-1}f\mparenB{a+\dfrac{i(b-a)}n}$;
  \item $\int_a^bf(x)\dif x=\lim_\ntoinf\dfrac{b-a}n\sum_{i=0}^{n-1}f\mparenB{a+\dfrac{(i+1)(b-a)}n}$;
  \item $\int_a^bf(x)\dif x=\lim_\ntoinf\dfrac{b-a}n\mparenbb{f(a)+2f\mparenB{a+\dfrac{b-a}n}+\dotsb+
        2f\mparenB{a+\dfrac{(n-1)(b-a)}n}+f(b)}$。
\end{exlist}
\item 对任意~$c\in\minto ab$,令
\[
  f(x)=\begin{Bdcases}
    1,& x=c;\\
    0,& x\in\mintc ab\difset\mbrace c 。
  \end{Bdcases}
\]
证明~$\int_a^bf(x)\dif x=0$。
\item 设~$f(t)$~在~$\mintc{a+c}{b+c}$~上可积。证明,$f(x+c)$~在~$\mintc ab$~上可积,且
\[
  \int_a^bf(x+c)\dif x=\int_{a+c}^{b+c}f(t)\dif t 。
\]
\item 设~$f(x)$~在~$\mintc ab$~上为可积的凸函数。证明,
\[
  (b-a)f\mparenB{\dfrac{a+b}2}\leq\int_a^bf(x)\dif x 。
\]
\end{exercise}

\section{Newton-Leibniz~公式}
\begin{exercise}
\item 若~$F(x)=\arctan\dfrac1x$,那么~$F'(x)=-\dfrac1{1+x^2}$,积分~$\int_{-1}^1-\dfrac{\dif x}{1+x^2}$~应取负值;但从
另一方面看,
\[
  \int_{-1}^1-\dfrac{\dif x}{1+x^2}=\arctan\dfrac1x\biggr\vert_{-1}^1=\frac\pi2 。
\]
请解释这个矛盾。
\item 计算下列定积分。
\begin{exlistcols}[4]
  \item $\int_0^\pi\cos^2x\dif x$;
  \item $\int_0^a\sqrt{a-x}\dif x$;
  \item $\int_0^ax\sqrt{a-x}\dif x$;
  \item $\int_{-4}^{-3}\dfrac{\dif x}{x\sqrt{x^2-4}}$。
\end{exlistcols}
\item 求下列极限。
\begin{exlistcols}
  \item $\lim_\ntoinf\sum_{k=1}^n\dfrac1n\sin\dfrac kn\pi$;
  \item $\lim_\ntoinf\mparenB{\dfrac1n+\dfrac1{n+1}+\dotsb+\dfrac1{2n}}$;
  \item $\lim_\ntoinf\dfrac1n\sqrt[n]{n(n+1)\dotsm(2n-1)}$。
\end{exlistcols}
\item%% 证明,
\begin{exlist}\FixExHead
  \item $\lim_\ntoinf\sum_{\substack{k=0\\ k\neq n}}^{2n}\dfrac{2n}{n^2+k^2}=2\arctan 2$;
  \item 若~$S_n(\theta)=\sum_{k=1}^n\dfrac1n\cos\dfrac kn\theta$,则~
  $
    \adjustlimits\lim_{\theta\to0}\lim_\ntoinf S_n(\theta)=
    \adjustlimits\lim_\ntoinf\lim_{\theta\to0}S_n(\theta)=1 。
  $
\end{exlist}
\item 求下列定积分。
\begin{exlistcols}
  \item $\int_0^\pi\cos x\dif x$;
  \item $\int_{-\frac12}^{\frac12}\dfrac{\dif x}{\sqrt{1-x^2}}$;
  \item $\int_{-1}^1\dfrac{\dif x}{1+x^2}$;
  \item $\int_{-1}^1\dfrac{\dif x}{x^2-2x\cos\alpha+1}\mcond{0<\alpha<\pi}$;
  \item $\int_{-\frac\pi2}^{\frac\pi2}\dfrac{\dif x}{\sin^2x+2\cos^2x}$;
  \item $\int_a^b\sin\mabs x\dif x$;
  \item $\int_0^{\frac\pi2}\dfrac{\dif x}{a^2\sin^2x+b^2\cos^2x}\mcond{a,b>0}$;
  \item $\int_0^{\frac\pi2}\dfrac{\cos x}{a^2\sin^2x+b^2\cos^2x}\dif x\mcond{a,b>0}$;
  \item $\int_{-1}^1\mabs{1-x}\dif x$;
  \item $\int_0^{2\pi}\dfrac{\dif x}{1+\e\cos x}\mcond{0<\e<1}$。
\end{exlistcols}
\end{exercise}

\section{可积函数}
\subsection{函数可积的充分必要条件}
\subsection{可积函数类}
\begin{exercise}
\item 判断下列函数的可积性。
\begin{exlistcols}
  \item $f(x)$~在~$\mintc01$~上有界,不连续点为~$x=\dfrac1n\mcond{n=1,2,\dotsc}$;
  \item $f(x)=\begin{cbbdcases}
    \frac1x-\mfloorbb{\frac1x}, & x\neq0;\\
    0, & x=0 ;
  \end{cbbdcases}$
  \item $f(x)=\begin{cBdcases}
    1\Big/\mfloorbb{\frac1x}, & x\neq0;\\
    0, & x=0;
  \end{cBdcases}$
  \item $f(x)=\begin{cbbdcases}
    0, & x=0,\frac1n;\\
    \sgn\mparenB{\sin\frac\pi x}, & x\in\mintoc 01~\text{且}~x\neq\frac1n 。
  \end{cbbdcases}$
\end{exlistcols}
\item 证明,当
\[
  \lambda=\max_{1\leq k\leq n}\Delta x_k\to 0
\]
时,Darboux~大和以上积分为极限,而~Darboux~小和以下积分为极限。
\item 设
\[
  f^+(x)=\max\mrange{f(x)}0,\qquad
  f^-(x)=\min\mrange{f(x)}0 。
\]
证明,$f(x)\in R\mintc ab$~当且仅当~$f^+(x)\in R\mintc ab$~与~$f^-(x)\in R\mintc ab$~同时成立。
\item 设~$\max\mrange{f(x)}{g(x)}\in R\mintc ab$,且~$\min\mrange{f(x)}{g(x)}\in R\mintc ab$。证
明~$f(x)\cdot g(x)\in R\mintc ab$。
\item 设~$f(x)\in R\mintc ab$。证明~$\me^{f(x)}\in R\mintc ab$。
\item 设~$f(x)\in R\mintc ab$,并且~$f(x)\geq\alpha>0$。证明,
\begin{exlistcols}
  \item $\dfrac1{f(x)}\in R\mintc ab$;
  \item $\ln f(x)\in R\mintc ab$。
\end{exlistcols}
\item 设~$f(x)\in R\mintc ab$。证明,对任意~$\e>0$,存在逐段为常数的函数~$\phi(x)$,使得
\[
  \int_a^b\mabsb{f(x)-\phi(x)}\dif x<\e 。
\]
\item 设~$f(x)$~在~$\mintc ab$~上有界,定义
\[
  \omega_f\mintc ab=\sup_{x\in\mintc ab}\mbraceb{f(x)}-\inf_{x\in\mintc ab}\mbraceb{f(x)} 。
\]
证明
\[
  \omega_f\mintc ab=\sup_{x',x''\in\mintc ab}\mbraceb{\mabsb{f(x')-f(x'')}} 。
\]
\item 设~$f(x)$~在~$x_0$~附近有定义且有界。定义
\[
  \omega_f(x_0)=\lim_\ntoinf\omega_f\mintoB{x_0-\frac1n}{x_0+\frac1n} 。
\]
证明,$f(x)$~在~$x_0$~连续当且仅当~$\omega_f(x_0)=0$。
\item 设~$f(x),g(x)\in R\mintc ab$。证明,
\[
  \lim_{\lambda\to0}\sum_{i=0}^{n-1}f(\xi_i)g(\eta_i)\Delta x_i=\int_a^bf(x)g(x)\dif x,
\]
其中~$\xi_i,\eta_i\in\mintc{x_i}{x_{i+1}}$,$\Delta x_i=x_{i+1}-x_i$,$i=0,1,\dotsc,n-1$,$x_0=a$,$x_n=b$,%
$\lambda=\max_{0\leq i\leq n-1}\Delta x_i$。
\end{exercise}

\section{定积分的性质}
\subsection{定积分的基本性质}
\subsection{积分第一中值定理}
\begin{exercise}
\item 设~$f(x)\in C\mintc ab$,且~$\int_a^bf^2(x)\dif x=0$。证明~$f(x)\equiv 0$,$x\in\mintc ab$。
\item 设
\[
  f(x)=\begin{Bdcases}
    \alpha_i, & x=c_i\in\mintc ab;\\
    0, & x\in\mintc ab~\text{但}~x\neq c_i,
  \end{Bdcases}
\]
其中~$i=1,2,\dotsc,m$。证明~$\int_a^bf(x)\dif x=0$。
\item 设~$f(x)\in R\mintc ab$,$g(x)$~与~$f(x)$~只在有限个点上取值不等。证明~$g(x)\in R\mintc ab$,并且
\[
  \int_a^bg(x)\dif x=\int_a^bf(x)\dif x 。
\]
\item%%% 证明,
\begin{exlist}\FixExHead
  \item 若~$t>0$,则~$\ln t\leq t-1$;
  \item 若~$f(x)\in R\mintc ab$,且~$f(x)\geq\alpha>0$,则
  \[
    \int_0^1\ln f(x)\dif x\leq\ln\int_0^1f(x)\dif x;
  \]
  \item 若~$f(x)\in R\mintc ab$,则
  \[
    \exp\mbracebb{\int_0^1f(x)\dif x}\leq\int_0^1\me^{f(x)}\dif x;
  \]
  \item 若~$a_i>0\mcond{i=1,\dotsc,n}$,则
  \[
    \sqrt[n]{a_1\dotsm a_n}\leq\dfrac{a_1+\dotsb+a_n}n 。
  \]
\end{exlist}
\item 设~$f(x),g(x)\in R\mintc ab$。证明
\[
  \mabsbb{\int_a^bf(x)\cdot g(x)\dif x}\leq\mparenbb{\int_a^bf^2(x)\dif x}^{\msp\frac12}
  \cdot\mparenbb{\int_a^bg^2(x)\dif x}^{\msp\frac12},
\]
而且等号成立当且仅当~$g(x)=\lambda f(x)$,这里~$\lambda$~为某常数。
\item 设~$f(x),g(x)\in R\mintc ab$。证明
\[
  \mabsbb{\int_a^b\mparenb{f(x)+g(x)}^2\dif x}\leq\mparenbb{\int_a^bf^2(x)\dif x}^{\msp\frac12}
  +\mparenbb{\int_a^bg^2(x)\dif x}^{\msp\frac12},
\]
而且等号成立当且仅当~$g(x)=\lambda f(x)$,这里~$\lambda\geq0$~为某常数。
\item 设~$f(x)\in R\mintc ab$,且~$f(x)\geq\alpha>0$。证明
\[
  \int_0^1\frac1{f(x)}\dif x\geq\mparenbb{\int_0^1f(x)\dif x}^{\msp-1}。
\]
\item 证明,
\begin{exlistcols}
  \item $\lim_{b\to1}\int_0^b\dfrac{\sin x}{\sqrt{1-x^2}}\dif x\leq1$;
  \item $\lim_{b\to1}\int_0^b\dfrac{\cos x}{\sqrt{1-x^2}}\dif x\geq1$。
\end{exlistcols}
\item 设
\[
  I_n=\int_0^1\frac{x^n}{1+x}\dif x 。
\]
试建立~$I_n$~的递推公式,并利用所得结果证明
\[
  \lim_\ntoinf\mparenB{1-\frac12+\frac13-\dotsb+(-1)^{n-1}\frac1n}=\ln 2 。
\]
\item\begin{exlistcols}
  \item 建立~$I_n=\int_0^{\frac\pi2}\dfrac{\sin^2nt}{\sin t}\dif t$~的递推公式;
  \item 证明~$\lim_\ntoinf\dfrac2{\ln n}\int_0^{\frac\pi2}\dfrac{\sin^2nt}{\sin t}\dif t=1$。
\end{exlistcols}
\item 证明下列极限。
\begin{exlistcols}
  \item $\lim_\ntoinf\int_0^b(1-x^2)^n\dif x=0\mcond{0<b<1}$;
  \item $\lim_\ntoinf\int_0^1(1-x^2)^n\dif x=0$;
  \item $\lim_\ntoinf\int_0^{\frac\pi2}\sin^nx\dif x=0$。
\end{exlistcols}
\item\begin{exlist}
  \item 任意给定~$\delta>0$,证明,
  \[
    \lim_\ntoinf\frac{\dps\int_\delta^1(1-t^2)^n\dif t}{\dps\int_0^1(1-t^2)^n\dif t}=0;
  \]
  \item 设~$f(x)\in C\mintc{-1}1$,令~$\lambda_n=2\int_0^1(1-t^2)^n\dif t$。证明,
  \[
    \lim_\ntoinf\frac1{\lambda_n}\int_{-1}^1(1-t^2)^nf(t)\dif t=f(0)。
  \]
\end{exlist}
\item 设~$a,b>0$,而~$f(x)\geq0$,且~$f(x)\in R\mintc ab$,又知~$\int_{-a}^bxf(x)\dif x=0$。证明,
\[
  \int_{-a}^bx^2f(x)\dif x\leq ab\int_{-a}^bf(x)\dif x 。
\]
\item 设~$f(x)\geq0$~与~$f''(x)\leq0$~对任意~$x\in\mintc ab$~成立。证明~$f(x)\leq\dfrac2{b-a}\int_a^bf(x)\dif x$。
\item 设~$f(x)\in C\mintc ab$。证明~$f(x)$~是凸函数当且仅当
\[
  f(x)\leq\frac1{2h}\int_{-h}^hf(x+t)\dif t
\]
对任意~$\mintc{x-h}{x+h}\subset\mintc ab$~成立。
\item 设~$f'(x)\in C\mintc ab$。证明
\[
  \max_{x\in\mintc ab}\mbraceb{\mabsb{f(x)}}\leq\mabsbb{\frac1{b-a}\int_a^bf(x)\dif x}+\int_a^b\mabsb{f'(x)}\dif x 。
\]
\item 设~$f(x)\in R\mintc ab$。证明,函数~$f(x)$~具有积分的连续性,即
\[
  \lim_{h\to0}\int_a^b\mabsb{f(x+h)-f(x)}\dif x=0 。
\]
这里假设当~$x$~在~$\mintc ab$~外时,$f(x)$~的取值为零。
\item 我们称函数~$S(x)$~是区间~$\mintc ab$~上的阶梯函数,是指区间~$\mintc ab$~可以分割为有限个子区间,而~$S(x)$~在每一个子区间上
取常数值。证明,
\begin{exlist}
  \item 阶梯函数~$S(x)$~一定可积;
  \item 若~$f(x)\in R\mintc ab$,则对于任意给定的~$\e>0$,存在两个阶梯函数~$S_1(x)$~与~$S_2(x)$,满足
  \[
    S_1(x)\leq f(x)\leq S_2(x)\mcond*{a\leq x\leq b},
  \]
  并且
  \[
    \int_a^b\mparenb{f(x)-S_1(x)}\dif x<\e,\qquad
    \int_a^b\mparenb{S_2(x)-f(x)}\dif x<\e 。
  \]
\end{exlist}
\item 设~$f(x)\in R\mintc ab$。证明,存在连续函数序列~$\mbrace{\phi_n(x)}$,使得
\[
  \lim_\ntoinf\int_a^b\phi_n(x)\dif x=\int_a^bf(x)\dif x 。
\]
\end{exercise}

\section{变限的定积分与原函数的存在性}
\begin{exercise}
\item\begin{exlist}
  \item 设
  \[
    f(x)=\begin{cBdcases}
      2x\sin\frac1x-\cos\frac1x, & x\neq0;\\
      0, & x=0 。
    \end{cBdcases}
  \]
  讨论~$f(x)$~在~$\mintc{-1}1$~上的可积性。
  \item 举出一反例,使得~$f(x)$~在~$\mintc ab$~上可积,且~$f(x)$~在~$x_0$~间断,但是
  \[
    \mparenbb{\int_0^xf(t)\dif t}'\biggr\vert_{x=x_0}=f(x_0) 。
  \]
\end{exlist}
\item 设~$f(x)\in C(\mR)$,并存在常数~$a$~满足~$5x^3+40=\int_a^xf(t)\dif t$。
\begin{exlistcols}
  \item 确定~$f(x)$;
  \item 确定常数~$a$。
\end{exlistcols}
\item 设~$f(x)\in C(\mR)$,求出下列函数的导数。
\begin{exlistcols}
  \item $F(x)=\int_0^{x^2}f(t)\dif t$;
  \item $F(x)=\int_{x+a}^{x+b}f(t)\dif t$。
\end{exlistcols}
\item 求下列极限。
\begin{exlistcols}
  \item $\lim_{h\to0}\dfrac1{h^2}\int_0^h\mparenB{\dfrac1\theta-\cot\theta}\dif\theta$;
  \item $\lim_{x\to\pinf}\dfrac{\me^{-x^2}}x\int_0^xt^2\me^{t^2}\dif t$。
\end{exlistcols}
\item 求下列函数在原点的导数~$F'(0)$。
\begin{exlistcols}
  \item $F(x)=\int_0^xt\sin\dfrac1t\dif t$;
  \item $F(x)=\int_0^x\cos\dfrac1t\dif t$。
\end{exlistcols}
\item 确定下列函数。
\begin{exlistcols}
  \item $f(x)=\int_0^x\sgn t\dif t$;
  \item $f(x)=\int_0^x\mabs t\dif t$;
  \item $f(x)=\int_0^1\mabs{x-t}\dif t$;
  \item $f(x)=\int_0^1t\mabs{x-t}\dif t$。
\end{exlistcols}
\item 设~$f(x)$~在正实轴上连续,且在正实轴上成立
\[
  \int_0^xf(t)\dif t=\frac12xf(x) 。
\]
证明~$f(x)=Cx$,这里~$C$~为某常数。
\item 设~$f(x)$~在~$x>0$~时连续,对任意~$a,b>0$,积分值~$\int_a^{ab}f(x)\dif x$~与~$a$~无关。证
明~$f(x)=\dfrac Cx$,这里~$C$~为某常数。
\item 设~$f'(x)\in R\mintc ab$,且~$f(a)=0$。证明,
\begin{exlistcols}
  \item $\mabsb{f(x)}\leq\int_a^x\mabsb{f'(t)}\dif t$;
  \item $\int_a^bf^2(x)\dif x\leq(b-a)^2\int_a^b\mparenb{f'(x)}^2\dif x$。
\end{exlistcols}
\item 设~$f'(x)\in R\mintc ab$。证明~$f(x)$~可以分解成两个单调函数之差。
\item 设~$P_n(x)$~为~$n$~次代数多项式。证明
\[
  \int_a^b\mabsb{P_n'(x)}\dif x\leq 2n\max_{x\in\mintc ab}\mbraceb{\mabsb{P_n(x)}}。
\]
\item 设~$f(x)\in C\mintc ab$,并且单调递增。证明,
\[
  \int_a^bxf(x)\dif x\geq\frac{a+b}2\int_a^bf(x)\dif x 。
\]
\item 设~$f(x)$~在任一有限区间上可积分,且~$\lim_{x\to\pinf}f(x)=\ell$。证明
\[
  \lim_{x\to\pinf}\frac1x\int_0^xf(t)\dif t=\ell 。
\]
\item 设~$f(x)$~是~$\mR$~上不恒为常数的单调函数,并且对任意~$x,y\in\mR$,有
\[
  f(x+y)=f(x)\cdot f(y) 。
\]
证明,
\begin{exlistcols}
  \item $f(0)=1$;
  \item $f(x)>0$;
  \item $f(x)$~在~$\mR$~上连续;
  \item $f(x)$~在~$\mR$~上可微;
  \item 确定~$f(x)$~所满足的微分方程;若~$f(1)=2$,解出~$f(x)$。
\end{exlistcols}
\end{exercise}

\section{定积分的换元法与分部积分法}
\subsection{定积分的换元法}
\subsection{定积分的分部积分法}
\subsection{积分第二中值定理}
\begin{exercise}
\item 计算下列定积分\mcond{a>0}。
\begin{exlistcols}[3]
  \item $\int_0^1\dfrac x{(1+x)^\alpha}\dif x$;
  \item $\int_0^1\ln(1+\sqrt x)\dif x$;
  \item $\int_0^a\sqrt{\dfrac{a-x}{\smash[b]{a+x}}}\dif x$;
  \item $\int_0^a\arctan\sqrt{\dfrac{a-x}{\smash[b]{a+x}}}\dif x$;
  \item $\int_0^{\frac a{\sqrt2}}\dfrac{\dif x}{(a^2-x^2)^{\sfrac32}}$;
  \item $\int_1^{\sqrt3}\dfrac{\sqrt{1+x^2}}x\dif x$;
  \item $\int_0^1x\sqrt{1-x}\dif x$;
  \item $\int_1^9x\sqrt[3]{1-x}\dif x$;
  \item $\int_0^{\frac12}\dfrac{x^3}{\sqrt{1-x^2}}\dif x$;
  \item $\int_0^a\sqrt{a^2-x^2}\dif x$;
  \item $\int_0^{\frac\pi2}\sin^2x\cos x\dif x$;
  \item $\int_0^ax^2\sqrt{a^2-x^2}\dif x$;
  \item $\int_{\frac14}^{\frac12}\dfrac{\arcsin\sqrt x}{\sqrt{x(1-x)}}\dif x$;
  \item $\int_a^{2a}\dfrac{\sqrt{x^2-a^2}}{x^4}\dif x$。
\end{exlistcols}
\item 证明,
\begin{exlistcols}
  \item $\int_0^{2\pi}\dfrac{1-r^2}{1-2r\cos\theta+r^2}\dif r=2\pi\mcond{0<r<1}$;
  \item $\int_0^1\dfrac{\dif x}{(1+x^n)\sqrt[n]{1+x^n}}=\dfrac1{\sqrt[n]2}$。
\end{exlistcols}
\item 证明,
\begin{exlistcols}
  \item $\int_0^\pi\cos nx\cos kx\dif x=\begin{Bdcases}
    \frac\pi2, & k=n;\\ 0, & k\neq n;
  \end{Bdcases}$
  \item $\int_0^{\frac\pi2}\cos nx\cos^nx\dif x=\dfrac\pi{2^{n+1}}$。
\end{exlistcols}
\item 设
\[
  I_n=\int_1^{1+\frac1n}\sqrt{1+x^n}\dif x 。
\]
\begin{exlistcols}
  \item 证明~$\lim_\ntoinf I_n=0$;
  \item 证明极限~$\lim_\ntoinf n\cdot I_n$~存在,并求出此极限值。
\end{exlistcols}
\item 设~$f(x)$~在~$\mR$~上连续。证明~$f(x)$~是周期为~$2\pi$~的函数当且仅当积分
\[
  \int_0^{2\pi}f(x+y)\dif y
\]
与~$y$~无关。
\item 证明,当~$n$~为奇数时,
\[
  F(x)=\int_0^x\sin^n\dif t
\]
是以~$2\pi$~为周期的周期函数;而当~$n$~为偶数时,$F(x)$~是线性函数与周期函数之和。
\item 设~$f(x)$~在~$\mintco 0\pinf$~上单调递增。证明,函数
\[
  F(x)=\frac1x\int_0^xf(t)\dif t
\]
在~$\mintco 0\pinf$~上单调递增。
\item 设~$f(x)$~在~$\mintco 0\pinf$~上的凸函数。证明,函数
\[
  F(x)=\frac1x\int_0^xf(t)\dif t
\]
在~$\mintco 0\pinf$~上的凸函数。
\item 给定积分~$I=\int_0^\pi xf(\sin x)\dif x$,而~$f(x)$~是~$\mintc01$~上的连续函数。
\begin{exlistcols}
  \item 证明~$I=\dfrac\pi2\int_0^\pi f(\sin x)\dif x$;
  \item 计算~$\int_0^\pi\dfrac{x\sin x}{1+\cos^2x}\dif x$。
\end{exlistcols}
\item 设~$f(x)\in C\mintc AB$,且~$A<a<b<B$。证明,
\[
  \lim_{h\to0}\int_a^b\dfrac{f(x+h)-f(x)}h\dif x=f(b)-f(a) 。
\]
\item 设~$f(x)$~是周期为~$T$~的连续函数。证明,
\[
  \lim_{x\to+\infty}\frac1x\int_0^xf(t)\dif t=\frac1T\int_0^Tf(x)\dif x 。
\]
\item 计算下列积分。
\begin{exlistcols}[3]
  \item $\int_0^1\ln\mparenb{x+\sqrt{1+x^2}}\dif x$;
  \item $\int_0^1x\arctan^2x\dif x$;
  \item $\int_0^\pi x^2\sin nx\dif x$;
  \item $\int_{-\pi}^\pi\me^x\cos nx\dif x$;
  \item $\int_0^1\dfrac{\dif x}{(2-x^2)^2}$;
  \item $\int_1^2\dfrac{\sqrt{1+x^2}}{x^2}\dif x$;
  \item $\int_0^1x^2(1-x^3)^5\dif x$;
  \item $\int_0^1(1-x)^2x^3\dif x$;
  \item $\int_1^2x^2\ln x\dif x$;
  \item $\int_0^1x^2\me^{\sqrt x}\dif x$。
\end{exlistcols}
\item 计算下列定积分。
\begin{exlistcols}
  \item $\int_0^{\frac\pi2}\me^{\alpha x}\sin\beta x\dif x$;
  \item $\int_0^{\frac\pi2}\me^{\alpha x}\cos\beta x\dif x$。
\end{exlistcols}
\item 计算下列定积分,其中~$n$~与~$m$~均为正整数。
\begin{exlistcols}
  \item $\int_0^1x^n\ln^nx\dif x$;
  \item $\int_0^1x^{m-1}(1-x)^{n-1}\dif x$;
  \item $\int_0^1\tan^{2n}x\dif x$;
  \item $\int_0^{2\pi}\sin^nx\cos^mx\dif x$,~$n+m$~为奇数。
\end{exlistcols}
\item 设
\[
  I_n=\int_0^{\frac\pi4}\dfrac{\cos nx}{\cos^nx}\dif x 。
\]
证明
\[
  I_n=2^{n-1}\mparenbb{\frac\pi2-\sum_{k=1}^{n-1}\frac{\sin\mparen{\sfrac{k\pi}4}}{k2^{\sfrac k2}}} 。
\]
\item 设~$a_n=\int_0^1x^n\sqrt{1-x^2}\dif x$。证明,
\begin{exlistcols}[3]
  \item $a_n=\dfrac{n-1}{n+2}a_{n-2}$;
  \item $a_n\leq a_{n-1}\leq a_{n-2}$;
  \item $\lim_\ntoinf\dfrac{a_n}{a_{n-1}}=1$。
\end{exlistcols}
\item 设~$f'(x)\in C\mintc 01$。证明
\[
  \int_0^1x^nf(x)\dif x=\frac{f(1)}n+o\mparenB{\frac1n}\mcond*{\ntoinf}。
\]
若只假设~$f(x)\in C\mintc 01$,上式是否成立。
\item\begin{exlist}
  \item 设~$f''(x)\in C\mintc 01$。证明,
  \[
    \int_0^1x^nf(x)\dif x=\frac{f(1)}n-\frac{f(1)+f'(1)}{n^2}+o\mparenB{\frac1{n^2}}\mcond*{\ntoinf};
  \]
  \item 证明,
  \[
    \int_0^1\frac{x^n}{1+x}\dif x=\frac1{2n}-\frac1{4n^2}+o\mparenB{\frac1n}\mcond*{\ntoinf}。
  \]
\end{exlist}
\item\begin{exlist}
  \item 设~$f(x)\in R\mintc 01$。证明,
  \[
    \int_0^1\sqrt[n]xf(x)\dif x=\int_0^1f(x)\dif x+o(1)\mcond*{\ntoinf};
  \]
  \item 证明,
  \[
    \int_0^1\dfrac{\dif x}{1+x^n}=1-\frac{\ln 2}n+o\mparenB{\frac1n}\mcond*{\ntoinf}。
  \]
\end{exlist}
\item 假设~$\pi$~为有理数,即~$\pi=\dfrac ab$,这里~$a,b$~为互素的正整数。令
\[
  f(x)=\frac{x^n(a-bx)^n}{n!}。
\]
\begin{exlistcols}
  \item 证明~$\int_0^\pi f(x)\sin x\dif x$~为一整数;\label{exer-7.6.21-1}
  \item 由~\ref{exer-7.6.21-1}~证明~$\pi$~不可能为一有理数。
\end{exlistcols}
\item 设~$f(x)\in C\mintc ab$,且在~$\minto ab$~上有~$m$~个相异零点,又对任意~$n=0,1,\dotsc,m$,有
\[
  \int_a^bx^nf(x)\dif x=0 。
\]
证明~$f(x)\equiv 0$,$x\in\mintc ab$。
\item 设~$f'(x)\in C\mintc ab$,且~$f(a)=0$。证明,
\[
  \mabsbb{\int_a^bf(x)\dif x}\leq\frac{(b-a)^2}2\max_{x\in\mintc ab}\mbraceb{\mabsb{f'(x)}}。
\]
\item 设~$f''(x)\in C\mintc ab$,且~$f(a)=f(b)=0$。证明,
\begin{exlist}
  \item $\int_a^bf(x)\dif x=\dfrac12\int_a^bf''(x)(x-a)(x-b)\dif x$;
  \item $\mabsbb{\int_a^bf(x)\dif x}\leq\dfrac{(b-a)^3}{12}\max_{x\in\mintc ab}\mbraceb{\mabsb{f''(x)}}$。
\end{exlist}
\item 设~$f'(x)\in C\mintc ab$,而~$f(a)=f(b)=0$,且~$\int_a^bf^2(x)\dif x=1$。证明,
\begin{exlistcols}
  \item $\int_a^bxf(x)f'(x)\dif x=-\dfrac12$;
  \item $\int_a^bf'^2(x)\dif x\int_a^bx^2f^2(x)\dif x>\dfrac14$。
\end{exlistcols}
\item 设~$f''(x)$~在~$\minto ab$~上连续,并且对任意~$x_0\in\minto ab$,取合适的~$r>0$,使得~$x_0\pm r\in\minto ab$。证明,%
存在~$\xi\in\minto{x_0-r}{x_0+r}$,使得
\[
  f''(\xi)=\frac3{r^3}\int_{x_0-r}^{x_0+r}\mparenb{f(x)-f(x_0)}\dif x 。
\]
\item 证明,对任意~$p>0$,有
\[
  \lim_\ntoinf\int_n^{n+p}\frac{\sin x}x\dif x=0 。
\]
\item 设~$f(x)$~在~$\mintc 0{2\pi}$~上单调有界。证明,
\begin{exlistcols}
  \item $\lim_{\lambda\to\infty}\int_0^{2\pi}f(x)\sin\lambda x\dif x=0$;
  \item $\lim_{\lambda\to\infty}\int_0^{2\pi}f(x)\cos\lambda x\dif x=0$。
\end{exlistcols}
\item 设~$f(x)=\int_x^{x+1}\sin t^2\dif t$。证明,当~$x>0$~时,有~$\mabsb{f(x)}<\dfrac1x$。
\item 设~$f(x)$~在~$\mintc 0{2\pi}$~上单调下降。证明,
\[
  b_n=\frac1\pi\int_0^{2\pi}f(x)\sin nx\dif x\geq0 。
\]
\item 设~$f(x)$~在~$\mintc {-\pi}\pi$~上单调下降。证明,
\[
  b_{2n}=\frac1\pi\int_{-\pi}^\pi f(x)\sin 2nx\dif x\geq0,\qquad
  b_{2n+1}=\frac1\pi\int_{-\pi}^\pi f(x)\sin(2n+1)x\dif x\leq0 。
\]
\item 设~$f(x)$~是~$\mintc 0{2\pi}$~上的凸函数,且~$f'(x)$~有界。证明,
\[
  a_n=\frac1\pi\int_0^{2\pi}f(x)\cos nx\dif x\geq 0 。
\]
\item 设~$f(x)$~是~$\mintc{-\pi}\pi$~上的凸函数,且~$f'(x)$~有界。证明,
\[
  a_{2n}=\frac1\pi\int_{-\pi}^\pi f(x)\cos 2nx\dif x\geq 0,\qquad
  a_{2n+1}=\frac1\pi\int_{-\pi}^\pi f(x)\cos(2n+1)x\dif x\leq 0 。
\]
\item 设~$\me^2<a<b$。证明~$\int_a^b\dfrac{\dif x}{\ln x}<\dfrac{2b}{\ln b}$。
\item 设
\[
  F(x)=\begin{cbbdcases}%\LEFTRIGHT
    \int_0^x\sin\frac1t\dif t, & x\neq0;\\
    0, & x=0 。
  \end{cbbdcases}
\]
证明,
\begin{exlistcols}
  \item $\lim_{x\to0}\dfrac{F(2x)-F(x)}x=0$;
  \item $F'(0)=0$。
\end{exlistcols}
\item 设~$p<3$。证明
\[
  \lim_\ntoinf\int_{\frac1n}^1\frac{\sin\dfrac nx}{x^p}\dif x=0 。
\]
\item 设~$f(x)$~在~$\mintc 01$~上严格单调下降。证明,
\begin{exlist}
  \item 存在~$\theta\in\minto01$,使得
  \[
    \int_0^1f(x)\dif x=\theta f(0)+(1-\theta)f(1);
  \]
  \item 对任意~$c>f(0)$,存在~$\theta\in\minto01$,使得
  \[
    \int_0^1f(x)\dif x=\theta c+(1-\theta)f(1) 。
  \]
\end{exlist}
\item 设~$F(x)$~是在~$\mintco 0\pinf$~上单调增加的正值函数,而~$y$~是微分方程
\[
  y''+F(x)y=0
\]
的解。证明~$y$~在~$\mintco 0\pinf$~上有界。
\end{exercise}

\section{定积分的近似计算}
\subsection{矩形法}
\subsection{梯形法}
\subsection{Simpson~公式}
\begin{exercise}
\item 利用~$n=12$~时的矩形公式,近似地计算
\[
  \int_0^{2\pi}x\sin x\dif x,
\]
并把结果同精确结果相比较。
\item 利用梯形公式计算下列积分,并估计误差。
\begin{exlistcols}
  \item $\int_0^1\dfrac{\dif x}{1+x}\quad(n=8)$;
  \item $\int_0^1\dfrac{\dif x}{\sqrt{1+x^3}}\quad{(n=10)}$。
\end{exlistcols}
\item 利用~Simpson~公式计算下列积分,并估计误差。
\begin{exlistcols}
  \item $\int_0^1\dfrac{\dif x}{1+x}\quad(n=8)$;
  \item $\int_0^1\dfrac{\dif x}{\sqrt{1+x^3}}\quad{(n=10)}$;
  \item $\int_1^9\sqrt x\dif x\quad(n=16)$;
  \item $\int_0^{\frac\pi4}\cos x^2\dif x\quad(n=10)$;
  \item $\int_0^\pi\sqrt{3+\cos x}\dif x\quad(n=6)$;
  \item $\int_0^{\frac\pi2}\dfrac{\sin x}x\dif x\quad(n=10)$。
\end{exlistcols}
\item 利用~Simpson~公式计算积分
\[
  \int_0^1\me^{-x^2}\dif x,
\]
精确到~$10^{-4}$。
\item 设~$f(x)\in C^{(4)}\mintc ab$,令
\[
  W(x)=\int_{c-x}^{c+x}f(t)\dif t-\frac x3\mparenb{f(c-x)+4f(c)+f(c+x)},
\]
其中~$c=\dfrac{a+b}2$,而~$x\in\mintc 0L$,$L=\dfrac{b-a}2$。
\begin{exlist}
  \item 求~$W(0)$,$W'(0)$,$W''(0)$~与~$W^{(3)}(x)$;
  \item 证明
  \[
    W(L)=\frac12\int_0^LW^{(3)}(x)(x-L)^2\dif x;
  \]
  \item 证明,存在~$\xi\in\mintc ab$,使得
  \[
    W(L)=-\dfrac{(b-a)^5}{\num{2880}}f^{(4)}(\xi)。
  \]
\end{exlist}
\end{exercise}

\begin{exercise*}
\item\begin{exlist}
  \item 设~$x\geq0$,而~$n$~为自然数。证明,
  \[
    x^n\geq n(x-1)+1;
  \]
  \item 对任意~$n\in\mN$,证明,
  \[
    \int_0^{1+\frac2{\sqrt n}}x^n\dif x>2;
  \]
  \item 设正数列~$\mbrace{a_n}$~满足
  \[
    \lim_\ntoinf\int_0^{a_n}x^n\dif x=2 。
  \]
  证明~$\lim_\ntoinf a_n=1$。
\end{exlist}
\item 设~$f(x)\in C\mintc01$,对于~$k=0,1,\dotsc,n-1$,有
\[
  \int_0^1f(x)x^k\dif x=0\quad\text{且}\quad\int_0^1f(x)x^n=1。
\]
证明,存在~$\xi\in\minto 01$,使得~$\mabsb{f(\xi)}\geq 2^n(n+1)$。
\item 设~$f(x),g(x)\in C\mintc ab$,并且~$f(x)$~单调下降,而~$0\leq g(x)\leq 1$,记~$\lambda=\int_a^bg(x)\dif x$。证明,
\[
  \int_{b-\lambda}^bf(t)\dif t\leq\int_a^bf(t)g(t)\dif t 。
\]
\item 设~$f'(x)\in C\mintc 01$,且满足~$0\leq f'(x)\leq 1$~而~$f(0)=0$。证明,
\[
  \int_0^1f^3(x)\dif x\leq\mparenbb{\int_0^1f(x)\dif x}^{\msp2}。
\]
\item 设~$f(x)$~在~$\mR$~上有界且可微,同时对任意~$x\in\mR$~有
\[
  \mabsb{f(x)+f'(x)}\leq 1 。
\]
证明~$\mabsb{f(x)}\leq1$。
\item 设~$f(x)$~在~$\mintco 0\pinf$~上可微,且~$0\leq f'(x)\leq f(x)$,而~$f(0)=0$。证明~$f(x)\equiv0$。
\item 设~$f(x)\in C^{(2)}\mintc 01$,并且任意取定~$x\in\minto 0{\sfrac13}$,$\eta\in\minto{\sfrac23}1$。证明,
\begin{exlist}
  \item $\mabsb{f'(x)}\leq3\mabsb{f(\xi)-f(\eta)}+\int_0^1\mabsb{f''(x)}\dif x$;
  \item $\mabsb{f'(x)}\leq9\int_0^1\mabsb{f(x)}\dif x+\int_0^1\mabsb{f''(x)}\dif x$。
\end{exlist}
\item 设~$f(x)\in C\mintco1\pinf$,且~$f(x)>0$,同时对任意~$x\geq1$,有
\[
  F(x)=\int_1^xf(t)\dif t\leq f^2(x) 。
\]
\begin{exlistcols}
  \item 证明~$F'(x)\geq\sqrt{F(x)}$;
  \item 证明~$f(x)\geq\dfrac12(x-1)$;
  \item 若~$\int_1^xf(t)\dif t\leq f^3(x)$,证明~$f(x)\geq\sqrt{\dfrac32(x-1)}\mcond{x\geq1}$。
\end{exlistcols}
\item 设~$f(x)\in C^{(2)}\mintc ab$,而~$f(a)=f(b)=0$。证明,
\begin{exlist}
  \item 对任意~$x\in\minto ab$,有
  \[
    \mabsbb{f(x)\cdot\frac{b-a}{(x-a)(x-b)}}\leq\int_a^b\mabsb{f''(x)}\dif x;
  \]
  \item
  \[
    \max_{x\in\mintc ab}\mbraceb{\mabsb{f(x)}}\cdot\frac4{b-a}\leq\int_a^b\mabsb{f''(x)}\dif x。
  \]
\end{exlist}
\item\begin{exlist}
  \item 设~$f(x)$~单调上升,且~$f(0)>0$。证明
  \[
    1\leq\int_0^1f(x)\dif x\cdot\int_0^1\frac1{f(x)}\dif x\leq\frac{\mparenb{f(0)+f(1)}^2}{4f(0)f(1)}。
  \]
  \item 设~$a_1,a_2,a_3\geq0$,且~$a_1+a_2+a_3=1$,而~$0<\lambda_1<\lambda_2<\lambda_3$。证明,
  \[
    1\leq\mparenbb{\sum_{i=1}^3a_i\lambda_i}\mparenbb{\sum_{i=1}^3\frac{a_i}{\lambda_i}}
     \leq\frac{(\lambda_1+\lambda_3)^2}{4\lambda_1\lambda_3}。
  \]
\end{exlist}
\item\begin{exlist}
  \item 设~$f(x)$~在~$\minto 0\pinf$~上连续且单调递减。证明,
  \[
    \int_1^{n+1}f(x)\dif x\leq\sum_{k=1}^nf(k)\leq f(1)+\int_1^nf(x)\dif x;
  \]
  \item 设~$S_n=\sum_{k=1}^n\dfrac1{\sqrt k}$,求极限~$\lim_\ntoinf\dfrac{S_{2n}-S_n}{\sqrt n}$;
  \item 设
  \[
    a_n=\frac1{1+\dfrac11}+\frac1{2+\dfrac12}+\dotsb+\frac1{n+\dfrac1n}-\ln\frac n{\sqrt 2} 。
  \]
  证明,序列~$\mbrace{a_n}$~收敛,并证明它的极限值位于~$0$~与~$\dfrac12$~之间。
\end{exlist}
\item 设~$a_1\geq a_2\geq\dotsb\geq a_n\geq 0$。
\begin{exlist}
  \item 若~$f(0)=0$,且~$f'(0)\geq0$,同时对任意~$x\in\mintco a\pinf$,有~$f''(x)\geq0$。证明,
  \[
    \sum_{k=1}^n(-1)^{k+1}f(a_k)\geq f\mparenbb{\sum_{k=1}^n(-1)^{k+1}a_k};
  \]
  \item 证明,当~$p>1$~时,
  \[
    \sum_{k=1}^n(-1)^{k+1}a_k^p\geq\mparenbb{\sum_{k=1}^n(-1)^{k+1}a_k}^{\msp p} 。
  \]
\end{exlist}
\item 设~$g(x)$~在~$\mintc ab$~上增加。证明,对任意~$c\in\minto ab$,有
\[
  f(x)=\int_c^xg(t)\dif t
\]
是凸函数。
\item 设~$f(x)$~在~$\mintc ab$~上时凸函数。证明,对任意~$c,x\in\minto ab$,有
\[
  f(x)-f(c)=\int_c^xf_-'(t)\dif t=\int_c^xf_+'(t)\dif t 。
\]
\item 证明~$\int_0^{\sqrt{2\pi}}\sin x^2\dif x>0$。
\item 请按下列提示的思路证明,若~$f(x)\in C\mintc ab$,且单调增加,则
\[
  \int_a^bf(x)\dif x\geq\frac{a+b}2\int_a^bf(x)\dif x 。
\]
\begin{description}[format=\bfseries,labelindent=\addleftskip,leftmargin=\addleftskip,labelsep=\ccwd]
\item[思路一]对积分~$\int_a^b\mparenB{x-\dfrac{a+b}2}f(x)\dif x$~分段使用第一中值定理;
\item[思路二]对积分~$\int_a^b\mparenB{x-\dfrac{a+b}2}f(x)\dif x$~使用第二中值定理;
\item[思路三]从~$\int_a^b\mparenB{x-\dfrac{a+b}2}\mparenbb{f(x)-f\mparenB{\dfrac{a+b}2}}\dif x\geq0$~出发;
\item[思路四]从
\[
  (t-x)\mparenb{f(t)-f(x)}\geq0\mcond*{t,x\in\mintc ab}
\]
出发,先固定住~$x$~对~$t$~积分,将所得结果再对~$x$~积分。
\end{description}
\item 设~$b>a>0$。证明
\[
  \ln\dfrac ba>\dfrac{2(b-a)}{a+b}。
\]
\end{exercise*}




\endinput
%%
%% End of file `MAChapter7.tex'.