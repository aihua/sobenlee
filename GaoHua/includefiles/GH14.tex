%# -*- coding:utf-8 -*-
%%%%%%%%%%%%%%%%%%%%%%%%%%%%%%%%%%%%%%%%%%%%%%%%%%%%%%%%%%%%%%%%%%%%%%%%%%%%%%%%%%%%%

\chapter{进两步,退一步:“抢救”的落潮}

\section{“审干九条”再颁布后“抢救”为什么愈演愈烈?}

1943~年春夏之际,“审干”逐步转入“反奸”、“抢救”,延安三万多党、政、军干部全被卷入进去,“特务”、“叛徒”、“内奸”,如滚雪球般越滚越多,人心浮荡,个个自危,一片肃杀气氛弥漫于各机关、学校。高压下的人们普遍感到惶恐,不知运动将往何处发展,个人的命运将有何变化。

7~月~1~日,毛泽东在给康生的批示中,提出了“防奸工作的两条路线”问题。毛说,正确路线是:“首长负责,自己动手,领导骨干与广大群众相结合,一般号召与个别指导相结合,调查研究,分清是非轻重,争取失足者,培养干部,教育群众”。错误路线是:“逼、供、信”。\footnote{毛泽东:《防奸工作的两条路线》(1943~年~7~月~1~日),载《文献和研究》,1984~年第~4~期。}此即是所谓“审干九条方针”的首次表述。毛的这段指示长期被认为是毛反对审干、肃奸极左倾向的有力依据,可是在这之后,延安的“抢救”反而一步步深入,走向高潮。

在“抢救”正处于高峰之时,康生动了开杀戒的念头,他提出要以边区政府的名义公审枪毙人犯。在这紧急关头,林伯渠立即将这一最新情况向毛泽东汇报,毛泽东否决了康生的提议,避免了一场眼看就要发生的内部残杀的惨剧。\footnote{《林伯渠传》编写组:《林伯渠传》,页~286~—~87。}1943~年~7~月~30~日,毛泽东指示停止“抢救失足者运动”。\footnote{刘家栋:《陈云在延安》,页~114。}

1943~年~8~月~15~日,中共总学委向全党及延安各机关、学校发出一道指示,要求有系统地进行一次关于国民党本质的教育,“决定自~8~月~16~日至~8~月~31~日这半个月中,各单位一律以主要力量来进行这个教育”,并宣布“抢救运动”告一段落。

同日,中共中央又一次作出《关于审查干部的决定》,重申毛泽东关于审查干部和肃清内奸的九条方针,这是在全党范围内第一次公布“审干九条方针”。该决定宣称,此次审干及“进一步审查一切人员”,“不称为肃反”,并将“逼供信”称之为“主观主义方针与方法”。

同日,毛泽东在康生提交的绥德反奸大会材料上批示,提出在“反特务斗争中”,必须坚持“一个不杀,大部不抓”的政策。

由此看来,延安的“抢救”应该停止了,因为毛泽东已经多次发话,并对“逼供信”提出了批评。但事实上延安的“抢救”丝毫没有降温,反而在~8~月~15~日审干决定颁布后,愈演愈烈,又掀起新的反奸、坦白高潮。此时,运动已易名为“自救”运动,但内容、实质与“抢救”别无二致。9~月~21、22~日,延安《解放日报》连续刊登《延安县开展防奸活动》和《绥师失足青年纷纷悔过,控诉国民党特务机关万恶罪行》的报道,将所谓参加了“复兴社”的十四岁小女孩刘锦楣和特务暗杀组织“石头队”的“负责人”、十六岁的小男孩马逢臣的坦白交待经验推向社会。10~月~6~日,延安市在边区参议会会场召开反汉奸特务大会,议期五天,\footnote{《谢觉哉日记》,上,页~543。}“抢救”的邪火越烧越旺。

难道毛泽东已失去对延安的控制力?难道毛泽东的话不再一言九鼎?难道康生胆大妄为、已不把毛泽东放在眼里——所有这些答案都是否定的,毛泽东一分钟也没失去权力,他只是采取了某些手法,在表面上批评一下“抢救”的“过火”行为,而实际上继续将“抢救”往更深入的方向推进。

1943~年初,毛泽东搬入了戒备森严的神秘机构——设在枣园的中央社会部机关,中社部并在通往枣园的要道小砭沟开办了一家小杂货铺,用以监视一切过往的“可疑人员”。毛泽东住进枣园后,与康生的联系更加方便,更加频繁,毛泽东每天起床的第一件事就是听取康生对审干、抢救的汇报。\footnote{金城:《延安交际处回忆录》(北京:中国青年出版社,1986~年),页~187。}在枣园,他虽然多次看到由交际处呈交的关于钱来苏动态的报告——交际处向毛泽东不断报告,目的就在等毛的一句话,好解脱钱来苏——报告详细反映钱来苏的焦虑、不安、惶恐以及钱来苏反复陈述白己不是汉奸、特务的内容,但是毛泽东就是不明确表示态度,致使钱来苏长期不得解脱。

人们可能会发出疑问,毛泽东不是已经批评“逼供信”了吗?他在和一些负责干部的个别谈话中,甚至提出要注重调查研究,不要搞肉刑,为什么他说的是一套,做的又是一套?

确实,毛泽东隐蔽的思想很难被一般人所发现,许多负责干部只看到毛批评过火行为的一些词语,而忽略了他这些话后面的更深涵义,毛泽东所强调的恰恰不是纠偏,而是将运动进一步搞深搞透。

毛泽东的这套谋略,充分反映在被许多干部视为是纠偏文件的~1943~年~8~月~15~日《关于审查干部的决定》中。

“八一五决定”提出“一个不杀,大部不捉”,是毛泽东鉴于内战时期“肃~AB~团”等极左肃反政策的教训而提出的一个重要方针。毛泽东当然知道当年他自己所作所为的真正动机是什么,只是现在毛泽东的身份、地位已不同于当年,延安的局面更非往昔江西时期的情况可比。眼下,主客观条件均不允许再重演“肃~AB~团”的一幕,因此毛明智地宣布“一个不杀,大部不捉”,此所谓“过一不过二”也。但是这个新方针的基本前提仍然是肯定有大批特务混入革命队伍,文件强调,“特务是一个世界性群众性的问题”,指出,认识这个基本前提才有可能采取正确方针。

所谓“一个不杀,大部不提”,并非认为运动方向错了,而是着眼于将运动向纵深方向推进。毛泽东解释道,一个不杀——将使特务敢于坦白;大部不捉——保卫机关只处理小部,各机关学校自己处理大部。毛泽东还具体规定了捕人的规模:普通嫌疑分子,占有问题人员的~80\%,留在各机关学校接受审查;10\%~的问题人员送入西北公学、行政学院反省机关;另有~10\%~的人员送入社会部、保安处的监狱系统。毛泽东规定,这三类人员要进行交流,即普通嫌疑问题严重者将升格进入二类、一类系统。反之,坦白、交待彻底的特务,也可降到二类、三类系统。

对于留在各单位受审人员的审查和监护,毛泽东也不厌其烦地进行具体指导:一切有问题人员都暂时禁止外出,在延安实行通行证制度,毛泽东并要求,“在一定时候实行戒严”\footnote{中央档案馆编:《中共中央文件选集》(1943~—~1944),第~14~册,页~89~—~96。}。

精明、仔细、对大、小事都要过问的毛泽东,\footnote{在师哲的回忆中,对毛泽东的精细有很生动的描述。其中有一段写道:当胡宗南军队攻占延安前夕,毛泽东亲自监督师哲销毁与莫斯科来往的文电密码及记录,毛最后还用小棍翻播灰烬,待确定已燃尽后,才放心离开。参见师哲:《在历史的巨人身边——师哲回忆录》,页~201~—~202。在延安时代与毛交往较多的萧三也认为毛做事很细。参见《谢觉哉日记》,下,页~681。}难道不知道在审干抢救中会发生左倾狂热?他当然知道,而且了解得十分透彻。“八一五决定”中称,在审查运动中,一定会有过左的行为发生,一定会犯逼供信错误(个人的逼供信与群众的逼供信),一定会有以非为是、以轻为重的情形发生。可是毛泽东明知故纵,偏不予制止,执意听之任之下去。“八一五决定”提出,纠左不能太早,亦不能太迟。既然运动已经发生偏差,为什么不立即制止?毛泽东自有一套逻辑:“对于过左偏向,纠正太早与纠正太迟都不好。太早则无的放矢,妨碍运动的开展,太迟则造成错误,损伤元气,故以精密注意,适时纠正为原则”。\footnote{中央档案馆编:《中共中央文件选集》(1943~—~1944),第~14~册,页~89~—~96。}恰恰由于毛泽东的这套逻辑,才使审干、抢救的极端行为恶性发展,因为谁都不知应在何时采取纠偏行为才谓“适时”,而“抢救”的野马,只有毛泽东才能勒住缰绳,他若不采取明确措施予以制止,谁都不敢,也无权力纠偏。

毛泽东执意扩大审干,精密筹划各种具体方法和措施,他提出审干的正确路线应是“首长亲自动手”,于是许多机关学校负责人就亲自审问“犯人”,亲自动手打入。毛泽东提出要依靠重众力量审干,于是各单位纷纷召开群众大会,造成恐怖的群众专政的声势。毛泽东提出要“调查研究”,拟定有问题与没问题两种人名单,对所谓“有问题的人”要结合平时言行,从其交待的历史资料中找出破绽,对他们进行“劝说”,“质问”,各单位如法炮制,车轮战、攻心战,纷纷上阵。毛泽东声称“愈是大特务,转变过来愈有用处”,表扬“延安几个月来已经争取一大批特务分子,很快地转变过来为我党服务,便利了我们的清查工作”,于是各单位纷纷利用坦白的“特务”进一步检举其他特务,“特务”一串一串地被揭露出来。毛泽东别出心裁,要求“着重注意,将反革命特务分子转变为革命的锄奸干部”\footnote{中央档案馆编:《中共中央文件选集》(1943~—~1944),第~14~册,页~89~—~96。}——按照共产党的逻辑,“反革命特务分子”与“革命的锄奸干部”本是风马牛不相及的两个概念,毛泽东究竟是指示“以毒攻毒”,或是暗示“只要为我所用,管他乌龟王八蛋”,语意含混,难得要领,结果是被诬为“大特务”的原中央政治研究室的成全等人,果真被留在中央社会部,转变为“革命的锄奸干部”。

1943~年~8~月以后,在毛泽东有关批评“逼供信”的只言片语的后面,隐藏着深深的玄机。他的面孔是多重的,有时,他会轻描淡写说几句“逼供信”不好,转眼间,他又会说“既然没问题,为什么怕审查呢”,“真金不怕火炼嘛”,\footnote{在“抢救”运动中,毛泽东和王世英说过类似的话,当时王世英已被人诬陷为“特务”,参见段建国、贾岷岫著,罗青长审核:《王世英传奇》,页~193。}毛泽东的“注意正确的审干方针”只是一句空话,他所关心的是如何彻底查整全党的干部。他的目的只有一个:以暴力震慑全党,造成党内的肃杀气氛,以彻底根绝一切个性化的独立思想,使全党完全臣服于唯一的、至高无上的权威之下——毛泽东的威权之下。应该说,毛泽东达到了他的目标,几十年后,当年经历过审查的干部还在说,对他们“教育”最大、使他们得到“锻炼”、真正触及了灵魂的是审干抢救运动,而不是前一阶段的整风学习。

\section{中央主要领导干部对“抢救”的反应}

由毛泽东、康生主导的延安“抢救”和审干、反奸,是以中央书记处、中央总学委的名义推行的,毛泽东已将中央政治局完全控制于掌中,当他需要时,才主持召开政治局扩大会议,此时的中央政治局实际上已名存实亡,在延安的中央政治局委员大多已处于被批判的地位,被分割在各高级学习组进行整风学习,在毛的高压下,中共大多数领导干部都难以表达不同意见。

作为整风头号目标的王明,自~1941~年~10~月住进医院以后,他的政治局委员、书记处书记和中央统战部部长等职务虽未免去,但已是形同虚设。1943~年春之后,毛泽东作出决定,由他亲自掌管重庆办事处,由任弼时负责驻西安办事处,在延安的中央统战部几乎已无事可做,王明真正陷入了四面楚歌的境地。1943~年~11~月,中央总学委、中央办公厅召开揭发王明错误大会,王明之妻孟庆树登台为其夫辩护,会场气氛一度对毛泽东十分不利。毛泽东大为光火,将这次批判大会斥之为“低级趣味”,下令终止这种允许被批判者登台辩解的斗争大会。\footnote{王明:《中共五十年》,页~148。}王明没有任何地方可以渲泻他的不满和“怨屈”,他怕苏联人去看他,从而招惹毛泽东、康生的忌恨,可是又控制不住想见苏联人,只能在前来探病的苏联医生面前“痛哭流涕”\footnote{弗拉基米洛夫:《延安日记》,页~108、164。据弗拉基米洛夫说,他是在得到毛泽东的许可后,才派奥洛夫医生于~1943~年~10~月~28~日前去为王明治病的,就是在这一天,王明在苏联医生面前哭了出来。}。

博古对整风、审干、抢救极度厌恶,但他也没有任何可以与之倾诉的对象。他以工作为由,找到苏联驻延安代表,痛骂毛泽东。博古深知康生情报机关的厉害,与苏联代表讲话时,不时出门观察门外动静,确定没有人偷听,才敢进屋与苏联代表倾谈\footnote{弗拉基米洛夫:《延安日记》,页~137。}。

在延安的几位德高望重的中共元老林伯渠、徐特立、吴玉章、谢觉哉均非整风目标。四老皆与毛泽东有历史旧谊,徐特立、谢觉哉还是毛泽东长沙时期的师友,林伯渠、吴玉章早在广州、武汉国共合作时期即与毛泽东共事,同为国民党中央执行委员。在瑞金时期林伯渠担任国民经济部部长和财政部部长,与时任中华苏维埃共和国中央执委会主席的毛泽东相处融洽。整风转入审干、抢救后,林伯渠等诸老日见愈来愈多的同志被打成“特务”、“叛徒”,国统区中共组织被诬为“红旗党”,均感到震惊,但是诸老性格各异,对党内斗争的体会各自深浅不同,其反应也是存有差别的。

林伯渠是诸老中唯一担负实际领导责任的。他曾为钱来苏一案多次呈文毛泽东。作为老政治家的林伯渠,深知此类运动能全面展开,非毛泽东批准而绝不可能。因此,他的态度极为谨慎。林伯渠曾劝慰对运动感到怀疑的同志说,运动高潮阶段,出现“逼供信”是不足为奇的,但运动后期一定会复查核实。他并表示,大批青年知识分子都经过中共驻重庆、西安办事处审查才介绍进延安的,他从未听说过整个大后方党组织已变质,对此问题,他心里有数。”\footnote{《林伯渠传》编写组:《林伯渠传》,页~138。}

徐特立长期从事教育工作,性格率真,他曾当面质问负责自然科学院审干抢救运动的陈伯村,凭什么证据,将一批批师生抓走。\footnote{《徐特立在延安》,页~45、118、139。}徐特立全然不管他实际上已被闲置的处境,愤然道,我是院长,我有责任,为什么不许我管!\footnote{《徐特立在延安》,页~45、118、139。}徐特立还亲自前往窑洞看望被关押的师生,当别人劝他应予以注意时,他再次表示:我是院长,我就是要保护人才\footnote{《徐特立在延安》,页~45、118、139。}。

谢觉哉此时担任边区参议会副议长兼党团书记,1943~年~7~月“抢救”乍起,他在初期也是从好的方面去理解。谢觉哉在~7~月~31~日的日记中写道,“我对某些失足青年怜多于恨,处在反动环境下……生死判诸俄倾,革命与反革命又其模糊,于是乎就失足了。”谢觉哉继续说:“这次反特务斗争,给我们教育不少……没有这次斗争要我们青年党员知道阶级斗争不易,就是老年党员也一样。”\footnote{《谢觉哉日记》,上,页~521、603、694。}但是很快,谢觉哉就发现“抢救”出了大问题,“反奸斗争被逼死的人,他说无法审查了,但其中未必有主要特务。”\footnote{《谢觉哉日记》,上,页~521、603、694。}

谢觉哉对“抢救”的疑问与其经历过的党内残酷斗争有关。1932~年在湘鄂西苏区,他曾亲眼目睹红军内部自相残杀的惨景,当时谢觉哉也被列入有待处置的肃反名单,只是幸而被国民党清剿部队俘虏,才侥幸躲过那场灾难\footnote{《谢觉哉日记》,上,页~521、603、694。}。于是尽其所能保护自己所在单位的干部。但因他在~1937~—~1938~年曾负责中共驻兰州办事处,眼下,中共甘肃工委已被康生打成“红旗党”,谢觉哉竟也遭到责难。康生白恃握有上方宝剑,对享有盛望的谢觉哉丝毫不看在眼里,公然指责“谢老是老右倾”。在抢救高潮中,康生直扑谢觉哉办公的窑洞,一进门就盛气凌人地指责谢觉哉,“据×××交待(笔者注:指张克勤),他的父亲是个老特务。看来兰州地下党全是特务,是个‘红旗党’,你这个兰办的党代表可真是麻木不仁啊”。面对康生的责难,谢觉哉明确表示不同意他的看法,竟被康生指责为“庇护特务组织”。谢觉哉一气之下。“干脆不去开会,不参加学习,呆在家里睡觉”。\footnote{《谢觉哉传》编写组:《谢觉哉传》(北京:人民出版社,1984~年),页~292~—~93。}面对抢救野火蔓延,谢觉哉只能自我安慰:“不可能没有吃冤枉的个人,只求没有吃冤枉的阶级。”\footnote{《谢觉哉日记》,上,页~708。}

在几老中间,稍微特殊的是吴玉章,此时吴玉章挂名延安大学校长,但该校实权由副校长周扬掌握,吴玉章只是做些“新文字”(汉字拉丁化)的研究和推行工作,并不具体过问延安大学的审干、肃奸、抢救工作。或许是因为在莫斯科曾与王明共过事,也可能是因为缺少在三十年代苏区生活的经历,吴玉章对来势凶猛的运动似乎感到有些紧张。吴玉章在“抢救”运动期间曾拄着拐杖、流着眼泪劝说被诬为“特务”的中直机关的青年向党坦白交代。他还通过写自传表态支持“抢救”,吴玉章写道:

\begin{quoting}
在整风中人人写思想自传,并且号召坦白运动。这里就发现了国民党派了不少特务到我们党内来,到我们边区来,到我们军队中,专门作破坏工作,这是整风初期所未料及的。我党以宽大政策,号召这些被国民党特务分子陷害了的青年改过自新,已经有不少的特务分子响应了党的号召,改过自新了,且愿为反对特务尽力。至于死心塌地、甘为反革命尽力的少数人,则已逮捕起来。这又是反共分子而料所不及的,真所谓“作伪心劳日拙”。\footnote{吴玉章:《我的思想自传》(1943~年),载《吴玉章文集》,下,页~1338。}
\end{quoting}

尽管林伯渠、徐特立、谢觉哉等诸老都对“抢救”表示了怀疑与不满,但是,他们并没有就此向毛泽东进言,林伯渠相信,一阵风以后,高潮过了,头脑发热的人会清醒下来。\footnote{《林伯渠传》编写组:《林伯渠传》,页~137。}他们要等毛泽东自己去纠偏,而不愿去冒犯忤的风险。

身为八路军总司令,但毫无实权的朱德,对毛泽东、康生的行径心知肚明,知道讲话没用,只能暗暗焦急,而无可奈何。

陈云此时仍是政治局委员和中组部部长,作为干部审查和管理的最高机关中组部的部长,陈云本应直接参与领导审干、抢救运动,但是毛泽东没有让他与闻审干和“抢救”运动,公开的理由是“毛主席关心陈云同志的身体,让他搬到枣园去休养”。然而真实的情况是毛泽东、康生对陈云领导下的中组部很不满意,康生曾指责中组部“坏人那么多,你们组织部都是怎么搞的呀?”认为中组部“在审查问题上右了,太宽了,使得特务钻到了我们党内”\footnote{刘家栋:《陈云在延安》,页~30、112、114。}。让陈云“休养”的另一个原因是陈云对开展“抢救”运动的态度很不积极。据陈云当年秘书回忆,当时陈云“根本就不同意搞这场‘抢救’运动”,他认为是夸大了敌情。他也不相信所谓“红旗党”的说法,认为这不符合事实。陈云更认为对许多青年知识分子和老干部实行“抢救”违背了中央关于如何整风的规定。正是因为陈云的这种态度,毛泽东就不要他过问审干、反奸一类事,而是让陈云“到他身边去治疗休养”。从~1943~年~3~月陈云住进枣园到~1944~年~3~月,他离开枣园,调往西北财经办事处,恰是审干、反奸、抢救从开场到落幕的一年,在这一年间,陈云实际是被“靠边站”了,他没有参与有关决策,“许多事情都不知道”。\footnote{刘家栋:《陈云在延安》,页~114。}中组部部长一职则由彭真代理,当陈云调往西北财经办事处后,彭真就被正式任命为中组部部长。

林彪于~1943~年~7~月与周恩来等一行从重庆返回延安,受到毛泽东的特别关照,毛嘱林彪休息,林彪只是挂名担任中央党校副校长,并不具体过问党校的具体工作(此时延安除整风、审干外,没有任何紧急工作)。林彪在延安对康生一直保持距离,对审干、抢救持沉默态度,完全置身于运动之外。

身为中央军委参谋长的叶剑英,在审干、抢救中,曾向中央负责人反映军委直属机关抢救中出现的严重问题,他明确表示延安不可能有这么多特务,不能这样搞运动。但是叶剑英本人也受到康生的怀疑。康生以叶剑英长期在国统区工作,社会关系广泛,不时在毛泽东面前进谗言,并对叶剑英在延安的亲属进行“抢救”迫害。叶剑英前妻危拱之,被打成“河南红旗党”的特务,从~1943~年秋至~1945~年春被长期关押,“身心遭受严重损伤”,精神一度失常。\footnote{任质斌:《纪念党的好女儿——危拱之同志》,载《怀念危拱之》(郑州:河南人民出版社,1986~年),页~20。}叶剑英虽未隔离审查,但两次被剥夺参加讨论内战时期中共路线的政治局扩大会议。据苏联驻延安观察员透露,叶剑英对康生深恶痛绝。中共几位重要将领刘伯承、聂荣臻、陈毅此时均被召回延安,正要对各自在内战时期及抗战初期的“错误”进行反省,他们虽然均对审干、抢救中的极端行为不满,但是,他们的地位和身份都使他们难以开口。

除了少数几个积极参与康生“抢救”的重要干部,大多数领导人都对抢救、审干的过火行为表示不满。陈云、王若飞等人均曾在私下对“抢救”有所疑问,但是他们都不曾在毛泽东面前表示。在当时的肃杀气氛下,这些怀疑和不满都处在分散状态,没有人敢于在重要会议上将问题正式提出,更有甚者,一些重要干部,包括中央委员们都已中断来往,大家只有在公众场合才能见面,互相交谈都极为谨慎小心\footnote{弗拉基米洛夫:《延安日记》,页~186~—~87。}。

对“抢救”正式向毛泽东、康生表示怀疑的中共高层领导干部仅有周恩来、任弼时、张闻天、高岗。

1943~年~7~月~16~日,周恩来返回延安,准备参加中央核心层的路线检讨。周恩来一返回,就发现由他直接领导的国统区中共地下党已被诬为国民党特务组织“红旗党”,给周恩来造成巨大压力。周恩来本人甚至也受到康生的怀疑,认为周等“在白区天天与国民党接触,靠不住”。\footnote{《聂荣臻回忆录》(中)(北京:解放军出版社,1984~年),页~562。}周恩来一方面为许多被康生机关及各单位关押的原部下写证明材料,另一方面,周在与李维汉等谈话时,明确表示不存在所谓“红旗党”,国统区中共地下党的情况是清楚的。周恩来此时在党内的地位十分软弱,且是带罪之身。但他仍直接向毛泽东进言,表示了自己对运动的看法。

除了周恩来,敢于向毛泽东表达怀疑的还有任弼时和张闻天。任弼时此时处于权力核心,但他为人较为正直,对毛的一些做法颇不以为然。毛虽感觉到任弼时有些碍于碍脚,但为了利用他作为老干部的象征,分化打击王明、周恩来等,故对任弼时仍予以容忍。在~1943~年秋冬之际,任弼时两次向毛泽东提出“抢救”的严重弊端,要求予以纠正。

和处于权力核心的任弼时相比,张闻天早已成为失势人物,且正处在被批判斗争的地位。然而,张闻天却直接向康生表示他对“抢救”成果的怀疑,他明确告诉康生,社会部所编辑的《防奸经验》全是假的。\footnote{刘英:《在历史的激流中——刘英回忆录》,页~127~—~28。}和那些明哲保身的其他高干相比,张闻天全不计较个人得失,显示出他身上仍保有一些书生本色。

高岗在整风、“抢救”中原是一个“积极分子”,但随着“抢救”不断深入,他也感到似乎出了问题。据师哲透露,高岗曾向毛泽东反映,抢救的“作法过激”。\footnote{师哲:《峰与谷——师哲回忆录》,页~157。}

在次一级的负责干部中,也有人通过不同的方式,向毛泽东和中共中央表示对“抢救”的怀疑和反对,在这些人中,最具胆识的是原八路军驻二战区办事处主任王世英和社会部治安科科长陈龙。

1942~年整风运动开始后,王世英被调回延安,先在王家坪中央军委学分会工作,后调入中央党校参加整风学习。整风转入审干后,王世英参加了党校的审干小组。但是很快就对运动产生了怀疑,一些过去受王世英领导、在白区从事秘密工作的同志,被人揭发成为“特务”;王世英在经手调查党校“特务”的案件时,也发现指控与事实不符。对此,王世英在支部会上公开表示了对运动的怀疑,并写出了《关于请求中央纠正抢救失足者运动过左问题的报告》,上书毛泽东、刘少奇、康生。王世英在这份报告里,明确提出运动发展已经过“左”,要求中央予以纠正,并以自己的党籍和脑袋作担保,为已被打成“特务”或“特嫌”的钱来苏、白天(即以后成为名作家的魏巍)等六人申诉。在这六人中,由王世英亲自介绍前来延安的钱来苏受到长期怀疑,一直未能解脱;另两人也在车轮战下供认自己是“特务”。王世英上书的举动引起康生的强烈反弹,康生连夜给王世英覆信,指责王是“主观主义”、“好人观点”。在中央学委会上,康生当面责骂王是“大自由主义者,想逞英雄”,威胁王世英“有几个脑袋”?\footnote{段建国、贾岷岫著,罗青长审核:《王世英传奇》,页~191~—~92、192、193。}不久,果然出现了针对王世英的行动,在中央党校千人干部大会上,有人公开指认王世英和孔原是“大特务”\footnote{段建国、贾岷岫著,罗青长审核:《王世英传奇》,页~191~—~92、192、193。}。王世英虽然受到“抢救”的波及,但是毛泽东与他曾有过多次个别接触,对王世英在山西开展的统战和情报工作较为满意,因此,王世英并没有因上书反对“抢救”而遭致较大的不幸。以后他在《自传》中提及此事时说:“问题虽然提出来了(指有人诬指王为“特务”一事),始终没有向我开火,说明中央是很关心我和爱护我的,而且说明也是很了解我的。”\footnote{段建国、贾岷岫著,罗青长审核:《王世英传奇》,页~191~—~92、192、193。}

在王世英为反对“抢救”上书之际,领导“抢救”的社会部内也有工作人员对运动表示了怀疑。治安科长陈龙当面向顶头上司康生陈述他对运动情况的不理解。依照规定,杜会部治安科每周要写一份简报,分送毛泽东、刘少奇、朱德、任弼时、康生等五至七人,陈龙和社会部工作人员甘露通过这份手写的材料,曲折向毛等表示对运动的异议。经陈龙等汇总的材料上有详细的统计数目,具体反映各单位运动进展情况:奸细、特务的比例,自杀身亡人数,被关押人数等。材料的最后结论是:延安各单位~50\%~以上的干部已被抢救。\footnote{陈龙等的上报材料估计反映的是运动初期——~1943~年~7~至~8~月的情况,因为在此之后,几乎所有外来知识分子干部都程度不同地被抢救。}陈龙等整理的资料依正常程序上报后,中央总学委原计划开持续七天的延安党、政、军、学校参加的全市规模的“抢救”大会,结果开到第三天就没再继续下去。\footnote{修来荣:《陈龙传》,页~148~—~49485~特洛夫的绝密电报,这份电报涉及到一系列重要的问题,客观上促成了毛对“抢救”的刹车。}当然,不召开全市抢救大会,并不表明运动就降温了,陈龙等毕竟不能真正影响毛泽东的决策,以后,各机关、学校在内部继续开抢救大会,挖出来的“特务”、“内奸”比以前更多。

在延安的中共高层领导干部对“抢救”极端行为的不满议论,通过种种渠道传到毛泽东那里,然而毛泽东何尝不知道这些人的态度,他所关心的并非是他们的不满——毛泽东所要的是另一种效果,这就是,即使中共高层领导干部腹有怨言,但绝大多数人已不敢在他面前陈述。毛泽东借助审干、反奸、抢救达到了他多年来一直孜孜追求的目标,从精神上完全控制住昔日这批敢于与他斗争的同僚。

\section{1943~年~12~月~22~日季米特洛夫来电与“抢救”的中止}

周恩来、任弼时等为扭转抢救、审干中的极端行为,直言相劝毛泽东究竟有无作用?毛泽东是否立即采纳周恩来、任弼时等的意见,下令纠偏,停止运动?与人们一般的推测相反,毛泽东并没有立即部署纠偏,对于刚愎自用的毛泽东,只有当他自己意识到必须转弯时,他才会采取行动。所谓“适时纠正”的恰当时机,只有他才能决定,勿需别人多嘴。

毛泽东一点也不认为抢救、反奸有什么过错,他不是多次批示“一个不杀,大部不捉”吗?他不是提出反对“逼供信”吗?如此,继续运动又有何害?无非是过左一些,无非是受一点委屈,可是又没要你们的命,多坐几天班房又有什么关系呢?如果不对广大干部真正有所触动,“两条心”、“半条心”,能转变为“一条心”吗?

当然,对于任弼时、周恩来等的意见,毛泽东还是会加以周全考虑的,因为毛心里明白,延安不可能有那么多特务,毛总要想出一个办法,来收抬眼下这个局面。恰在这时,毛泽东收到一份来自莫斯科季米季米特洛夫电报全文如下:
\begin{quoting}
1943~年~12~月~22~日\\毛泽东(亲启)

一、关于令郎。我已安排他在军政学院学习,他毕业后当能在马克思列宁主义和现代军事方面获得扎实的学识。这个小伙子很能干,我相信您会把他培养成一个可靠的好帮手。他向您致以热烈的敬意。

二、关于政治问题。不言而喻,在共产国际解散之后,它过去的任何领导人都不得干预各国共产党的内部事务。但是从私人友情考虑,我又不能不告诉您我对中国共产党党内状况的担忧。您知道,从~1935~年起,我就不得不经常密切过问中国的事务。我认为,从反抗外国侵略者的斗争中退缩的方针,以及明显偏离民族统一战线的政策,在政治上都是错误的,在中国人民进行民族战争期问,采取这样的方针,有把党孤立于人民群众之外的危险,有导致内战加剧的危险。这只能有利外国侵略者及其在国民党内的代理人。我认为,发动反对周恩来和王明的运动,指控他们执行了共产国际推荐的民族统一战线,说他们把党引向分裂,这在政治上是错误的。不应该把周恩来和王明这样的人排除在党之外,而应该把他们保留在党内,千方百计利用他们为党工作。另外一件使我担心的事是,一部分党的干部对苏联抱有不健康的情绪。我对康生所起的作用也心存疑虑。清除党内敌对分子和把党团结起来的党内正确措施,被康生及其机构扭曲得面目全非,这样做只能散布互相猜疑的情绪,引起普通党员群众的无比愤怒,帮助敌人瓦解党。早在今年~8~月,我们就从重庆获得完全可靠的消息说,国民党决定派遣奸细混入延安挑动您同王明和党内其他活动家争吵,挑起敌对情绪以反对所有在莫斯科居留和学习过的人。关于国民党的这一诡计,我已及时预先通知了您。国民党秘而不宣的打算是,从内部瓦解共产党,从而轻易把它摧毁。我毫不怀疑,康生的所作所为正在为这些奸细助长声势。请原谅我这种同志式的坦率。我对您怀有深深的敬意,坚信您作为全党公认的领袖,定能洞察事物的真相。仅仅由于这一点,我才如此坦率地同您谈问题。请按我给您发送这封信的方式给我一封回信。紧紧与您握手。\named{季米特洛夫}\footnote{原载《共产国际与中国革命(文件资料集)。页~295~—~96(莫斯科:1986)》,引自《国外中国近代史研究》,第~13~辑,郑厚安译(北京:中国社会科学出版社,1989~年),页~2~—~3。}
\end{quoting}

季米特洛夫来电是一个严重事件,1943~年~5~月共产国际解散以后,自毛泽东已彻底放开了手脚,事实上,当毛决定向国际派摊牌之时,他就没有把莫斯科太多放在眼中。但是问题还有另外一面:共产国际虽解散了,苏共和苏联并没解散,现在莫斯科已完全知悉延安党内高层斗争的最新动态,斯大林通过季米特洛夫,以间接的方式对毛泽东发出警告,并且特别关注王明、周恩来的政治命运,似乎也影射到毛泽东的个人品质问题。季米特洛夫的来电特别提到康生,直指康生行为可疑,此说亦对毛泽东构成沉重打击。

接到季米特洛夫来电后,毛泽东立即精密部署,\footnote{接到季米特洛夫~1943~年~12~月~22~日来电后,毛一时情绪激动,他在~1944~年~1~月~2~日通过苏联驻延安观察员给季米特洛夫发出一份覆电。毛声明中共没有削弱对日本的斗争,与国民党合作的方针也没有改变。针对季氏对周恩来、王明的关心,毛答复道:“我们与周恩来的关系是好的,我们毫无把他开除出党的打算。周已经取得了相当大的进步。”至于王明,毛掩饰不住心中的愤恨,在电文中说“王明一直从事各种反党活动”,“在我看来,王明是不可靠的”。毛举出两个例子予以说明:一、王明过去被国民党逮捕过,在狱中承认了自己的党员身份,后来才被释放出来(在弗拉基米洛夫的《延安日记》中也提到作者本人强烈感受到毛对王明的痛恨,在~1943~年~11~月~29~日的日记中,弗拉基米洛夫写道,针对王明的新指控是“国民党同谋,反革命”,证据之一是王明曾被国民党逮捕,又给放了出来。参见《延安日记》,页~190、185~—~86);二、王明与米夫的关系可疑。毛对康生则表现出完全信赖的态度,他告诉季米特洛夫“康生是一个值得信赖的人”。一天以后,毛又后悔日前发出的电报可能会造成远方的误解,于是找到弗拉基米洛夫,询问昨天的电报是否发出,他告诉苏联观察员,前电可能不妥。紧接着,毛开展对苏联人的热情公关,据弗拉基米洛夫记载,1~月~4~日,毛泽东夫妇单独邀弗氏同观京剧,毛向弗氏大谈他如何尊敬苏联,尊敬斯大林,尊敬那些过去在苏联学习过的中国同志,以及如何感激季米特洛夫,参见《延安日记》,页~199~—~200。1~月~6~日,毛、刘、周邀请弗氏等苏联人畅叙友情。1~月~7~日,毛单独访问弗氏,再一次谈他如何深深地尊重斯大林和季米特洛夫,参见《延安日记》,页~203。在谈话中,毛完全改变了原先对王明的强烈敌对态度,其态度之友善使弗氏大吃一惊,毛请弗拉基米洛夫再给季米特洛夫发一电报,并告诉弗氏,团结的方针同样适用于王明。参见《弗拉基米洛夫转毛泽东给季米特洛夫的电报》(1944~年~1~月~3~日)《弗拉基米洛夫转毛泽克给季米特洛夫电及情况说明》,,引自杨奎松:《毛泽东发动延安整风的台前幕后》,载《近代史研究》,1998~第~4~期,页~51~—~54。另参见《延安日记》,页~190、185~—~86、199~—~200、202~—~205。}除了频频向苏联驻延安代表详剖心迹,强调整风的重要和他的光明正大,又派任弼时、周恩来与苏联代表谈话,用任、周等的嘴,澄清毛整人的“流言”。毛泽东同时加紧对王明的“诱”、“压”,迫使王明承认错误,让莫斯科无言以对。

毛泽东出台的措施可谓周密完善:莫斯科要求停止党内斗争,毛偏在此时召开上层会议,逼使所有同僚检讨、反省,用周恩来、王明等人的检讨堵住莫斯科的嘴,给莫斯科造成既成事实;莫斯科指责康生的反奸肃特是执行敌人的分化破坏阴谋,纯属胡说八道,延安的整肃全在毛的一手指挥下进行;莫斯科讨厌康生,正说明康生对毛的忠诚不贰,毛全然不顾莫斯科的警告,照样倚重康生。

然而,在季米特洛夫来电后继续抢救、反奸的极端行为,似乎已显得不妥。莫斯科已明确提出反对意见,此时的苏德战场形势已明显有利于苏联,而中共的未来将有赖于斯大林的支持,对莫斯科的意见毕竟不能完全置之不理;党内怨言继续蔓延终将损害毛泽东的个人威信,况且,审干、反奸、抢救所要达到的震慑人心的目的已基本实现,现在应是调整政策的“适时”时候了。

正是在上述背景下,1943~年~12~月~22~日中央书记处召开工作会议,讨论听取康生作的反特务斗争的汇报,任弼时在发言中提出,那种认为百分之八十的新知识分子是特务分子的看法应于否定,新知识分子中的百分之八十至九十是好的,现在应该进行甄别。毛泽东接受了任弼时的意见,同意进行甄别工作。\footnote{《胡乔木回忆毛泽东》,页~278~—~80。}在这次会议之后,延安的“抢救”开始逐渐落潮,但是,毛泽东精密掌握落潮的速度,不使运动骤然停下,避免广大干部对运动的“合理性”产生怀疑。1944~年初,延安各单位纷纷接待绥德县“坦白运动先进典型报告团”,该团由绥德师范师生组成,他们住在社会部所属的交际处租用的旅店,每天分头到各机关、学校做“现身说法”式的报告。其中一个十二、三岁的女学生,描述自己怎样受国民党特务机关派遣,专门施用“美人计”引诱革命干部……尽管毛泽东已开始着手准备“纠偏”,但是却放任“抢救”、坦白的闹剧继续演下去。

对钱来苏一案的处理,也反映出毛泽东欲维护“抢救”的复杂心态。自“抢救”运动开始,一直被软禁在交际处的钱来苏心情极为抑郁,多次表示后悔当初投奔延安。林伯渠等人欲救无力,只能等毛泽东的最后发话,1944~年~2~月~8~日,毛泽东在交际处处长金城呈交的有关钱来苏情况的报告上批示:
\begin{quoting}
金城同志:钱拯(即钱来苏,引者注)应优待他,他可能不是汉奸,他的子婿是否特务,也还是疑问,如不是,应平反的。\footnote{金城:《延安交际处回忆录》(北京:中国青年出版社,1986~年),页~186。}
\end{quoting}

在这个批示中,毛泽东虽然提出应予钱来苏优待等,但没有用明确的语言肯定钱来苏及子婿不是汉奸、特务,毛泽东的模棱两可,为保留“抢救”成果预埋了伏笔。

\section{甄别:在毛泽东“道歉”的背后}

1944~年春夏之际,审干、抢救运动进入到甄别阶段,各机关、学校原有的审干小组一变为“甄别委员会”,仍由原先领导审干、抢救的班子负责对干部的甄别工作。

所谓“甄别”,有异于“平反”。若干结论有不实之处,予以改正,谓之“甄别”。“平反”则是推翻原有错误结论,给蒙冤的对象恢复名誉。延安的审干、抢救的纠偏工作,名曰“甄别”,不称“平反”,其含义即在此。

甄别绝非一风吹,而是将受审坦白的人员划分成六类。据~1994~年出版的《胡乔木回忆毛泽东》一书透露,1943~至~1944~一年内,延安清出的“特务”共一万五千人,\footnote{《胡乔木回忆毛泽东》,页~280。}现在就是要对这一万五千人作出具体的划分:

“第一类是职业特务。他们是受一定的特务机关或特务人员的主使,对我们进行过或进行着特务工作(长期埋伏,也是一种潜伏工作),确有真凭实据的”。“但这类职业特务是极少数,仅占全体坦白分子百分之十左右,其中又有自觉被迫首要胁从之别”。

“第二类是变节分子。其中有的破坏过党的组织,捉过人,杀过人的;有的自首过写过反共文件,但未做过其它坏事的;有的被敌人短促突击,接受了敌人的任务,但回来既未实行也未报告的;有的是内战时做过坏事,抗战后中立或改过的等等”。“这类人在坦白分子中也是少数”。

“第三类是党派问题。他们加入过国民党、三青团或其它党派。在加入我党后并未向党报告,但还不是特务分子,这类人亦占颇大数目”。

“第四类是被特务利用和蒙蔽的分子。有的是在敌人红旗政策下不自觉地被特务利用和蒙蔽的分子,有的因半条心或幼稚无知,作了特务的工具”。

“第五类是党内错误。如假造历史,虚报党龄,与坏人来往,泄露秘密、包庇亲友、政治错误及贪污腐化等等,在坦白运动中被怀疑而误认为特务”。

“第六类是在审干时完全弄错或被特务诬害的”,“这类人虽然是少数,但确实是有的,甚至在逮捕的人中也还有的”\footnote{参见《中央关于坦白分子的六种分析给各地的指示》(1944~年~1~月~24~日),载《中共党史教学参考资料》,第~17~册,页~387;另参见《刘少奇年谱》,上卷,页~435;王素园:《陕甘宁边区“抢救”始末》。载《中共党史资料》,第~37~辑,页~225。}。

从以上对六类被审人员的划分依据看,被审查人员或多或少都有问题,完全搞错的只占一小部分,且放在最后一类,以示审干、抢救的成绩巨大,缺点是次要的。在上述划分标准中,中共中央仍坚持原先对国民党所谓“红旗政策”的判断。显而易见周恩来对“红旗党”的辩诬根本没被毛泽东等接受,中共在国统区尤其在西南地区的地下组织,在政治上仍不被充分信任。

对于这六类人员如何处理,中共中央也做了规定:

对坦白了的特务分子和变节分子,如果证据确凿,采取一个不杀,团结抗日的政策;如果没有真凭实据,不要加以深追,以免造成对立僵局,有碍争取;或中敌人诬陷同志的奸计。

对有真凭实据的暗藏的破坏分子,应继续执行宽大政策。即以宽大为主,镇压为辅;感化改造为主,惩罚为辅,给改过自新者以将功赎罪的出路。

对一时审查不清的重大嫌疑分子,不要急于求得解决,以免造成逼供信。可以有意识地放松一个时期,或暂时按他讲的作一个结论,然后继续进行调查研究和秘密侦查。

对有党派问题的,被欺骗蒙蔽的,或仅属于党内错误这三种人,在分清是非后,均应平反,取消特务帽子,按其情况,作出适当结论。对完全弄错或被特务诬陷的,一经查清,立即平反\footnote{参见王素园:《陕甘宁边区“抢救”始未》,载《中共党史资料》,第~37~辑,页~225~—~26、226。}。

上述这个规定,有许多似是而非、自相矛盾之处,既然没有真凭实据,为何不立即解脱,何以谈上“争取”,还要争取什么?所谓“被特务诬陷”,大量的检举揭发全是在逼供信下发生的,这些干部都是被迫指咬旁人的,又何以能称之为“特务”?更有甚者,对“一时审查不清的重大嫌疑分子”,还布置“继续进行调查研究和秘密侦查”,只是在表面上“有意识地放松一个时期,或暂时按他讲的做一结论”。

有关甄别政策中所隐含的深意,在毛泽东对延安受审干部的“道歉”中也充分地展现出来。

从~1944~年春夏至~1945~年春,由甄别所引发的延安广大干部对审干、抢救的强烈不满处于半公开的状态,在这种情况下,毛泽东先后在行政学院、中央党校、边区政府等场合,向延安干部“脱帽”、“道歉”。毛泽东绝口不提“抢救”为一错误运动,只是说一两句:运动搞过火了,使一些同志受了委屈云云。1944~年元旦,军委三局局长王铮带领一批原受审查、刚被解脱还没做结论的干部给毛泽东拜年(军委三局承担延安与各根据地的电讯往来业务,工作极其繁重,故最先解脱),在毛住所前黑压压站了一片,目的在于向毛泽东讨一个说法。毛泽东似真似假地说,本意为同志们洗澡,灰猛氧放多了一些,伤了同志们娇嫩的皮肤。毛泽东虽然向大家敬一个礼表示“道歉”,但言下之意又似乎在暗责受审干部斤斤计较,对接受党的审查耿耿于怀。

毛泽东的这种暧昧的态度,当然影响到各单位的甄别复查工作。自甄别展开后,经过三个月的复查,延安仅甄别了八百人,占坦白人数的四分之一。中直机关、边区政府、中央社会部、边区保安处、中央党校、延安大学、陕甘宁晋绥联防司令部等七个单位,经过八个月的甄别,在~487~人中被确定为“特务”的有六十四人,“叛徒”四十一人,合占总甄别人数的~22\%。其中康生直接掌管的中央社会部甄别二十七人,定为“职业特务”(当时的术语,“特务”分“职业与非职业”两类)的有六人,“叛徒”二人,两者占甄别人数的~30\%。由周兴任处长的边区保安处,甄别了九十六人,其中定为“特务”的有三十六人,“叛徒”二人,占~40\%。\footnote{参见王素园:《陕甘宁边区“抢救”始未》,载《中共党史资料》,第~37~辑,页~225~—~26、226。}在甄别复查中,将“特务”、“叛徒”的比例定得如此之高,无非是企图证明,开展审干、“坦白”、“抢救”是完全正确的。

中央党校作为“抢救”的重点单位直到~1944~年~9~月才全部转入甄别阶段。一部第六支部书记朱瑞与薄一波、倪志亮等组成一个小组。协助对集中较多问题人物的特别支部进行甄别工作。朱瑞等同情危拱之的遭遇,认为危拱之在“抢救”中虽然有自杀行为,并提出退党要求,但这都是精神错乱所致,“结论是没问题”。然而朱瑞等对危拱之的结论却受到党校一部负责人的批评,认为朱瑞等“代危抗辩,没有原则立场”,朱瑞等为此作了大量的工作,最后才解决了危拱之的结论问题。\footnote{郑建英:《朱瑞传》,页~294。}从这件事可以看出,“抢救”受害者的甄别工作往往会出现波折,一旦被“抢救”,要想完全洗清罪名,并非容易。

甄别、复查进展缓慢,激起延安广大干部的不满,为了平息、舒缓干部中的不满情绪,更重要的是,进入~1945~年后,国内、国际形势急剧变化,客观上要求尽速解决大量积压的审干、抢救遗留问题。在这种背景下,甄别、复查工作的进度有所加快,到了~1945~年春,延安各单位的甄别工作已基本结束,对~2,475~人作出了组织结论。\footnote{参见王素园:《陕甘宁边区“抢救”始末》,载《中共党史资料》,第~37~辑,页~228。1943~至~1945~年,延安有三万党员及非党员干部,受抢救冲击的大多为抗战后投奔延安的青年知识分子干部,也有相当数量的老干部,主要为国统区中共地下组织的领导人,以及从苏联返回的原留苏干部,据胡乔木提供的数字,被抢救的干部达一万五千人。}这个数字也许只是当时被关押进几个重要反省机关的“重犯”被甄别的数目,因为根据胡乔木透露的数目,延安“特务”的总数为一万五千人。

即使受审人员被作了结论,也绝非意味着万事大吉。凡受审人员均按四种情况分别给予不同等级的结论:

问题已澄清,完全可以做结论的;

对有关被查重点疑问问题做部分结论的;

在结论中留有待查尾巴,即仍有疑点,有待再查的;

无法查证,不予结论的。

在总计~2475~受审人员的结论中,有党派政治问题的约占~30\%,其中叛徒、特务、自首三类人员各占~10\%;党内错误问题,约占~40\%,完全弄错的约占~26\%,保留疑问不予结论的约~4\%。\footnote{参见王素园:《陕甘宁边区“抢救”始未》,载《中共党史资料》,第~37~辑,页~228、221。}对于这一部分人的处理方法,谁都不敢作主。直到毛泽东最后发话,大意是现在东北快解放了,需要大批干部,让他们到前线自己去做结论吧,是共产党人,一定留在共产党内,是国民党人让他跑到国民党去,怕什么呢。毛泽东虽然是这般说了,然而延安不仅没有释放任何一个像王实味这样已“定性”的人,那些未做结论的干部,也没有按照党的组织程序分配至各单位,而是仍然受到社会部的监控。这些干部虽然没有跑到国民党去,但他们在政治上还是继续受到怀疑与歧视。他们档案中的“疑点”和“尾巴”,在~1949~年后仍然被长期揪住不放,给当事人带来无穷的灾难,使他们的大好年华全被葬送在连绵不断的审干、肃反等运动中。

彭而宁——钱来苏之子,背着沉重的“特嫌”黑锅,在~1949~年后每一次审干运动中受审,直到~1980~年康生被揭露和清算后,才获彻底平反。一位匿名女干部,当年曾被诬为“日特”兼“国特”的“双料特务”,八十年代沉痛地回忆说,一生前后被审查长达十四年,前七年是我一踏进革命的门就受审查,还是美好的青年时期,当时只有十九岁,后七年正当壮年,是可以很好工作的时期,都丧失在康生的反动血统论和主观主义的逼供信下了\footnote{参见王素园:《陕甘宁边区“抢救”始未》,载《中共党史资料》,第~37~辑,页~228、221。}。

张克勤——当年康生精心培养出的一个坦白典型,康生在抢救高潮时还以张克勤为例,自夸自己已将反革命特务分子转变为革命服务。到了~1945~年甄别时期,康生还不放过他,坚决拒绝为张克勤做结论,康生要将张克勤作为证明抢救正确性的“成果”继续保持下去。1945~年~11~月,张克勤随社会部部分干部向东北转移,经历了严峻的考验,由于得到社会部第三室主任陈龙的关心和照顾,曾一度被安排在北安市公安局担任股长,但其“问题”一直无法解决。1949~年~11~月,又是在陈龙的关照下,张克勤随陈龙从哈尔滨去北京,经中组部介绍去西北局澄清其历史问题。直到~1950~年~4~月,“经中央有关部门批准,组织上才作了历史上没有问题的结论”,张克勤的党籍得到了恢复,此时恰是康生在政治上失意,自我赋闲的阶段。张克勤以后虽曾官至中共兰州大学党委书记,但在各种运动中均被波及。1986~年,时任甘肃省政协常委的张克勤无限感伤地回忆道:“今年是我参加革命五十周年,五十年中一半时间是在挨整”。“1943~年‘抢救’开始就戴上‘特务’帽子。抗战胜利后,戴着‘帽子’调到东北”,“1959~年又打成‘右倾机会主义分子’,‘文革’中又被康生点名,关了五年半监狱”。

延安还有一位叫蔡子伟的干部,曾任边区中学校长,在《谢觉哉日记》中,还有他在~1938~年~9~月活动的记载,以后此人即从延安公众生活中消失。蔡子伟被长期关押,详情外界不知,此人在八十年代曾任全国政协委员。

延安最后一批人的甄别,是在~1945~年~8~月日本投降后进行的。这批人全是边区保安处关押的重犯,总数约五六百人。由于当时中共中央要派大批干部前往东北,催促社会部和保安处抓紧甄别工作,1945~年~11~月~9~日,中社部负责干部陈刚和陈龙率二百多干部步行前往东北。陈刚,四川人,即富田事变中被扣的中央提款委员刘作抚。他在从江西返回上海后长期领导中央交通局,1932~年与何叔衡之女何实山结婚,1935~年刘作抚和孔原秘密前往苏联,何实山稍迟也抵苏。1937~年底,刘作抚夫妇回到延安,1938~年春参与组建“敌区工作委员会”,主办了八期秘密工作干部训练班。延安时代刘作抚早已易名为陈刚,在中社部主管人事,1945~年参加了中共七大,1948~年~12~月被任命为中社部副部长,从~1956~年起,陈刚任中共四川省委书记,1963~年升任西南局书记处书记。1945~年被陈刚带往东北的干部,其中有一半人是被“抢救”而未作甄别的,这批人多在建国初才得到甄别。

最后,对那些留在延安参加甄别的犯人,保安处决定,由他们本人甄别自己,给自己写出结论,再交保安处三科审阅后,本人签字定论,到~1946~年上半年,被关押的大部分人基本甄别完毕。

和那些已作结论或虽然带着“帽子”仍然被派往东北的那批前“犯人”相比,被继续关押在保安处的一百多个人的命运就太不幸了,这批人中有王实味等,他们将被作为抢救审干的牺牲品送上祭坛。1947~年春,国民党军队进攻延安,保卫部门押着这批“犯人”向山西临县转移,经康生批准,于黄河边全部被处决\footnote{参见仲侃:《康生评传》,页~95。}。

这批被杀的人,除了王实味,都没有留下名字(当然,原保安处会有这批人的详细档案)。与此同时,在山西晋绥根据地贺龙辖区,也相继处决一批受审人员,被誉为“爱国五青年”的蔺克义,就是被冤杀者之一。

蔺克义被捕前为晋绥《抗战日报》社出版发行部主任,1936~年他在西安师范读书时即参加了中共地下党,“在兰州、西安等地从事地下秘密工作和抗日救亡工作中,一直表现很好。在与国民党反动当局进行斗争中,立场坚定,勇敢顽强”。他在~1939~年~9~月到延安,先后在中央青委、中央出版发行部等单位工作,1940~年冬被派至晋西北。延安“抢救”展开后,有人被逼供咬出蔺克义是“特务”,检举材料由中央社会部转到晋绥《抗战日报》,蔺克义以“特嫌”被单独看管,最后被转至晋绥公安总局关押审查。“1947~年胡宗南进攻延安,一位负责人指示,要求在历史悬案(指整风中关押起来的)中,罪行比较严重的处死一批”,蔺克义便成了这个“负责人”的刀下鬼,时年仅二十九岁,其冤案直到十一届三中全会后才获平反\footnote{参见王素园:《陕甘宁边区“抢救”始未》,载《中共党史资料》,第~37~辑,页~223。}。

另据师哲披露,在~1947~年山西被处决的人中间,还有四名外国人。1944~年初,有四名外国人从晋察冀边区经晋西北押送到延安,其中三名是俄罗斯人,一名是南斯拉夫人,他们本是假道中共根据地设法去南洋或澳洲谋生的。这四名洋人被康生交边区保安处长期关押。直到~1947~年初,随其他“犯人”向山西永坪转移。康生在转往山西参加土改、途经永坪时,下令将这四名外国人秘密处决,事后把尸体全部塞进一口枯井,以后被国民党胡宗南部发现,造成很大轰动,彭德怀、周恩来、陆定一均表示了强烈不满(毛泽东呢,转战陕北时,周恩来、陆定一一直和毛泽东在一起,他不可能不知道此事),于是保安处处长周兴代康生受过,“只好自己承担责任,受批评,挨斗争”\footnote{师哲:《峰与谷——师哲回忆录》,页~217、216。}。

在被杀、或被释放解脱的人之外,还有另一类人,他们活着被抓进社会部或保安处,却再也没见他们出来。在这些神秘失踪的人中间,有一个叫王遵极的姑娘,1939~年奔赴延安时,年仅十九岁。据师哲称,王遵极“长相漂亮,举止文雅”,因其是大汉奸王克敏的侄女,一来延安就被关押,经反复审查还是没发现问题,经办此事的师哲建议“在一定条件”下释放她,却遭到康生及其妻曹轶欧的坚决反对。师哲称,“其中原委,始终令人不解”,王遵极以后下落不明。\footnote{师哲:《峰与谷——师哲回忆录》,页~217、216。}(另据仲侃《康生评传》称,王遵极从~1939~年至~1946~年在延安被长期关押,暗示她在~1946~年获释,但未交代此人以后的行踪。参见该书,页~78。)

从~1942~年揭幕的审干、坦白、反奸、抢救运动,到~1947~年王实味、蔺克义等被秘密处决,终于完全落幕。1945~年~3~月,蒋南翔给刘少奇写了一份(关于抢救运动的意见书),对于这场灾难进行了较为公允的批评(蒋南翔批评抢救运动“得不偿失”)。然而刘少奇本人也与这场运动有千丝万缕的联系,他的主要部属彭真更是运动的主要领导人之一,因此刘少奇不敢、也不愿对这场由毛泽东亲自主持,康生幕前指挥的运动说些什么。不仅如此,蒋南翔的《意见书》还被认为是“错误”的,蒋本人也受到了党内批评。从此,“抢救”的历史被彻底掩埋,凡经历过这场风暴的人们都知道应对此三缄其口,人们从书本、报刊、报告中只知道“伟大的整风运动”,一直到毛泽东、康生离世后的八十年代初,有关“抢救”的内幕才陆续被披露出来,此时已距当年近四十年。