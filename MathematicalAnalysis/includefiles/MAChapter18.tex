%# -*- coding:utf-8 -*-
%%%%%%%%%%%%%%%%%%%%%%%%%%%%%%%%%%%%%%%%%%%%%%%%%%%%%%%%%%%%%%%%%%%%%%%%%%%%%%%%%%%%%
%%  MAChapter18.tex'


\chapter{多元微分学的应用\\——几何应用与极值问题}\label{ch:18}

\section{曲线的表示法和它的切线}
\subsection{空间曲线的参数方程与切线}
\subsection{平面曲线的表示法和它的切线}
\begin{exercise}
\item $\mR{3}$中曲线由~$y=y(x)$,$z=z(x)$~表示,求曲线上相应于~$x=x_0$~点处的切线和法平面方程。
\item 在指定点写出下列曲线的切线和法平面方程。
\begin{exlist}
  \item $x=a\sin^2t$,$y=b\sin t\cos t$,$z=c\cos^2t$,在点~$t=\dfrac\pi4$;
  \item $x=t-\cos t$,$y=3=\sin^2t$,$z=1+\cos3t$,在点~$t=\dfrac\pi2$。
\end{exlist}
\item 在曲线~$y=x^2$,$z=x^3$~上求出一点,使此点的切线平行于平面~$x+2y+z=4$。
\item 证明圆柱螺旋线
\[
  x=a\cos t,\quad y=a\sin t,\quad z=bt
\]
的切线与~$z$~轴的夹角为常数。
\item 证明曲线
\[
  x=a\me^t\cos t,\quad y=a\me^t\sin t,\quad z=a\me^t
\]
与锥面~$x^2+y^2=z^2$~的各母线相交的角度相同。
\item 设~$F(x,y)\in C^{(2)}\mparenb{U(M_0)}$,$M_0=\minto{x_0}{y_0}$,且~$F_x'(M_0)$~与~$F_y'(M_0)$~不同时为零,而
~$F(x_0,y_0)=0$。求由~$F(x,y)=0$,$y(x_0)=y_0$~(或~$x_0=x(y_0)$)确定的曲线在点~$\minto{x_0}{y_0}$~邻域的曲率。
\item 在柱面~$x^2+y^2=R^2$~上求一曲线,使它通过~$(R,0,0)$~且每点的切向量与~$x$~轴、$z$~轴的夹角相等。
\item 在球面~$x^2+y^2+z^2=R^2$~上求一条曲线,使其上每一点的法线与平面~$2x+3y+6z-1=0$~的夹角为~$\ang{30}$。
\item 设有一圆锥,底半径为~$R$,高位~$h\mcond{h>R}$。现在锥面上有一条曲线,每一点的切线与~$z$~轴的夹角为~$\ang{45}$,若曲线上任意
一点在~$Oxy$~平面的投影的极坐标为~$\minto{r(\theta)}\theta$。
\begin{exlistcols}
  \item 证明~$\diff r\theta=-\dfrac{Rr}{\sqrt{h^2-R^2}}$;
  \item 求曲线的参数方程。
\end{exlistcols}
\end{exercise}

\section{空间曲面的表示法和它的切平面}
\begin{exercise}
\item 求出下列曲面在指定点~$P_0$~的切平面和法线方程。
\begin{exlistcols}
  \item $x^2+y^2+z^2=169$,~$P_0\mparen{3,4,12}$;
  \item $z=\arctan\dfrac xy$,~$P_0\mparenB{1,1,\dfrac\pi4}$;
  \item $3x^2+2y^2=2z+1$,~$P_0(1,1,2)$;
  \item $z=y+\ln\dfrac xz$,$P_0(1,1,1)$。
\end{exlistcols}
\item 求下列曲线在指定点~$M_0$~的切平面方程。
\begin{exlist}
  \item $x=a\cos\phi\cos\theta$,$y=b\cos\phi\sin\theta$,$z=c\sin\phi$,于~$M_0(\theta_0,\phi_0)$~处;
  \item $x=u\cos v$,$y=u\sin v$,$z=av$,于~$M_0\minto{u_0}{v_0}$~处。
\end{exlist}
\item 证明,函数~$F(x,y,z)$~在~$P_0(x_0,y_0,z_0)$~点的梯度向量是函数~$F(x,y,z)$~在点~$P_0$~的等量面的法向
量。(假设~$F\in C^{(1)}$)
\item 设~$P_0\mparen{x_0,y_0,z_0}$~是曲面~$F(x,y,z)=C$~的非起一点,而~$F\in C\mparenb{U(P_0)}$~为~$n$~次齐次函数。证明,此曲面在
~$P_0$~处的切平面方程为
\[
  xF_x'(P_0)+yF_y'(P_0)+zF_z'(P_0)=n 。
\]
\item 求曲面~$x^2+2y^2+3z^2=21$~的平行于平面~$x+4y+6z=0$~的各切平面。
\item 证明曲面~$z=x\me^{\frac xy}$~的每一切平面都通过原点。
\item 确定正数~$\lambda$,使曲面~$xyz=\lambda$~与椭球面~$\dfrac{x^2}{a^2}+\dfrac{y^2}{b^2}+\dfrac{z^2}{c^2}=1$~在某点相切。
\item 求圆柱面~$x^2+y^2=a^2$~与曲面~$bz=xy$~的交角。
\item 求椭球面~$\dfrac{x^2}{a^2}+\dfrac{y^2}{b^2}+\dfrac{z^2}{c^2}=1$~上点的法向量与~$x$~轴、$z$~轴成等角的点的轨迹。
\item 求曲面~$x^2+y^2+z^2=x$~的切平面,使其垂直于平面~$x-y-z=2$~和~$x-y-\dfrac12z=2$。
\item 证明,曲面~$\sqrt x+\sqrt y+\sqrt z=\sqrt a$~的切平面在各坐标轴上的截距之和为常量。
\item 证明,曲面~$F\mintob{x-az}{y-bz}=0$~的切平面与某定直线平行,其中~$a,b$~为常数。
\item 证明,曲面~$F\mintoB{\dfrac{x-a}{z-c}}{\dfrac{y-b}{z-c}}=0$~的切平面通过一个定点,其中~$a,b,c$~为常数。
\item 证明,曲面~$ax+by+cz=\Phi\mparenb{x^2+y^2+z^2}$~在~$P_0\mparen{x_0,y_0,z_0}$~点的法向量与向量
~$\mparen{x_0,y_0,z_0}$~及~$\mparen{a,b,c}$~共面。
\item 证明,下述三张曲面
\[
  \frac{xy}z=u,\quad\txts\smbsqrt{x^2+y^2}+\smbsqrt{y^2+z^2}=v,\quad \smbsqrt{x^2+z^2}-\smbsqrt{y^2+z^2}=w
\]
垂直相交于一点,其中~$u,v,w$~为定值。
\item 求下列曲线在给定点处的切线方程。
\begin{exlistcols}
  \item $x^2+y^2+z^2=6$,~$x+y+z=0$,于~$(1,-2,1)$~处;
  \item $x^2+z^2=10$,~$y^2+z^2=10$,于~$(1,1,3)$~处;
  \item $z=x^2+y^2$,~$2x^2+2y^2-z^2=0$,于~$(1,1,2)$~处。
\end{exlistcols}
\item 求两张曲面的交线
\[
  F(x,y,z)=0,\quad G(x,y,z)=0
\]
在~$Oxy$~平面上的投影曲线的切线方程。
\end{exercise}

\section{简单极值问题}
\subsection{极值的必要条件}
\subsection{极值的充分条件}
\subsection{求最大值或最小值的方法}
\begin{exercise}
\item 设~$\mvec x=\mparen{\mvec x{1,:,m}}$,而~$\vecfunc f{x}$~在~$\mvec a=\mparen{\mvec a{1,:,m}}$~取极小(大)值,并存在
~$\dfrac{\pdif^2\vecfunc f{a}}{\pdif x_i^2}\mcond{i=1,\dotsc,m}$。证明
~$\dfrac{\pdif^2\vecfunc f{a}}{\pdif x_i^2}\geq0\mcond{\leq 0}$。
\item 设~$f(x,y$~在~$Oxy$~平面的闭圆域上连续,其最小(大)值在边界上某点~$P_0\minto{x_0}{y_0}$~达到,边界圆在~$P_0$~的内法向(即
指向圆心的法线方向)为~$\mvec n$,单位向量~$\mvec\ell$~与~$\mvec n$~成锐角,且~$\pdiff{f(P_0)}{{\mvec\ell}}$~存在。证明
~$\pdiff{f(P_0)}{{\mvec\ell}}\geq0\mcond{\leq0}$。
\item 对下列函数按定义判断~$\minto00$~是否是其稳定点,是否是其极值点,并考虑这些例子的意义。
\begin{exlistcols}
  \item $f\minto xy=x^2-4xy+5y^2-1$;
  \item $f\minto xy=\smbsqrt{x^2+y^2}$;
  \item $f\minto xy=(x+y)^2-y^2$。
\end{exlistcols}
\item 求下列函数的极大值点与极小值点。
\begin{exlistcols}
  \item $f\minto xy=x^2\mparen{y-1}^2$;
  \item $f\minto xy=xy\mparenb{x^2+y^2-1}$;
  \item $f\minto xy=\mparenb{\smbsqrt{x^2+y^2}-1}^2$;
  \item $f\minto xy=3x^2y-x^4-2y^2$;
  \item $f\minto xy=\sin x+\sin y+\sin\mparen{x+y}$;
  \item $f\minto xy=\tan x+\tan y-\tan\mparen{x+y}$。
\end{exlistcols}
\item 求下列函数的极大值点和极小值点。
\begin{exlistcols}
  \item $f(x,y,z)=x^2+y^2+z^2-4xy+6x+2z$;
  \item $f(x,y,z)=\mparen{x+y+z}\me^{-x^2-y^2-z^2}$。
\end{exlistcols}
\item 用隐函数微分法求下列方程确定的隐函数~$z=z(x,y)$~的极大值和极小值。
\begin{exlistcols}
  \item $x^2+y^2+z^2-2x-2y-4z-10=0$;
  \item $(x+y)^2+(y+z)^2+(z+x)^2=3$;
  \item $x^2+y^2+z^2-xz-yz+2x+2y+2z-2=0$;
  \item $z^2+xyz-x^2-xy^2-9=0$。
\end{exlistcols}
\item 设~$f(x,y)=3x^2y-x^4-2y^2$。证明~$(0,0)$~不是~$f(x,y)$~的极值点,但沿过~$\minto 00$~点的每条直线,$\minto 00$~
都是它的极大值点。
\item\begin{exlist}\FixExHead
\item $f(x,y)=xy+\dfrac1x+\dfrac1y$~在~$0<x,y<\pinf$~上有最小值,但无最大值;
\item $f(x,y)=Ax^2+2Bxy+Cy^2+2Dx+2Ey+F$~在~$\mR{2}$~上有最小值,但无最大值,其中~$A>0$,而~$B^2<AC$;
\item $f(x,y,z)=\mparenb{ax+by+cz}\me^{-(x^2+y^2+z^2)}$~在~$\mR{3}$~上有最大值和最小值,其中~$a^2+b^2+c^2>0$。
\end{exlist}
\item 求下列函数在指定区域~$\Omega$~上的最大值和最小值(若它们存在的话)。
\begin{exlist}
  \item $f(x,y)=x-x^2-y^2$,其中~$\Omega=\mathsetb{(x,y)}{x^2+y^2\leq1}$;
  \item $f(x,y,z)=\mparenb{ax+by+cz}\me^{-(x^2+y^2+z^2)}$,其中~$a^2+b^2+c^2>0$,而~$\Omega=\mR{3}$。
\end{exlist}
\item 作容积为~$V$~的开口长方形容器,问尺寸怎样时,用料最省。
\item 要制作一个中间是圆柱,两端为相等的正圆锥的空浮标,它的体积是一定的,要使所用材料最省,应当怎样选择这个圆柱和圆锥的尺寸。
\item 有一块铁片,宽~$b=24$~厘米,要把它的两边折起做成一个槽,使得容积最大,求每边的倾角~$\alpha$~和折起的宽度~$x$。
\item 设有四个正数~$x,y,z,t$,它们的乘积保持常数,即~$xyzt=c^4$。证明,当~$x=y=z=t=c$~时,它们的和~$u=x+y+z+t$~取到最小值。
\item 在半径为~$R$~的已知圆的一切内接三角形中,求出其面积最大者。
\item 在椭球面~$\dfrac{x^2}{a^2}+\dfrac{y^2}{b^2}+\dfrac{z^2}{c^2}=1$~的内接长方体中,求体积为最大的那个长方体。
\item 设~$\closure\Omega\subset\mR{2}$~是有界闭区域,而~$u(x,y)$~定义在~$\closure\Omega$~上,在
~$\closure\Omega\difset\bound\Omega$~上满足
\[
  \pdiff u{x2}+\pdiff u{y2}+cu=0,
\]
其中~$c<0$~为常数。证明,
\begin{exlist}
  \item $u$~在~$\closure\Omega$~上的正最大值(负最小值)不能在~$\Omega$~内部取到;
  \item 又若~$u$~在~$\closure\Omega$~上连续,且~$u\mrest{\bound\Omega}=0$,则~$u\equiv=0$。
\end{exlist}
\end{exercise}

\section{条件极值问题}
\subsection{条件极值的必要条件与~Lagrange~乘子法}
\subsection{几个例子}
\begin{exercise}
\item 设~$f(x,y),g(x,y)\in C^{(1)}$,而~$\minto{x_0}{y_0}$~是~$z=f(x,y)$~在条件~$g(x,y)=0$~下的极值点,且~$g$~
在~$\minto{x_0}{y_0}$~梯度~$\grad g\minto{x_0}{y_0}\neq0$。证明,
\begin{exlist}
  \item $z=f(x,y)$~过~$\minto{x_0}{y_0}$~的等高线~$f(x,y)=f\minto{x_0}{y_0}$~与平面曲线~$\Gamma\colon g(x,y)=0$~在
  ~$\minto{x_0}{y_0}$~相切,这里又设~$\grad f\minto{x_0}{y_0}\neq0$;
  \item 三维空间中,曲面~$z=f(x,y)$~与柱面~$g(x,y)=0$~的交线在点~$\mparenb{x_0,y_0,f\minto{x_0}{y_0}}$~的切线与~$Oxy$~平面平行。
\end{exlist}
\item 设~$f(x,y,z)\in C^{(1)}$,且曲线~$\Gamma\colon x=x(t),y=y(t),z=z(t)$~是可微的。同时
~$P_0\mparen{x_0,y_0,z_0}=\mparenb{x(t_0),y(t_0),z(t_0)}$~是~$f(x,y,z)$~在曲线~$\Gamma$~上的极值点,且
\[
  \grad f(P_0)\neq0,\quad\mparenb{x'(t_0),y'(t_0),z'(t_0)}\neq0 。
\]
证明,等量面~$f(x,y,z)=f\mparenb{x_0,y_0,z_0}$~与曲线~$\Gamma$~在~$P_0$~点相切。
\item 设~$\vecfunc f{x}\in C(\mR{m})$。证明~$\vecfunc f{x}$~在条件~$\minp{\mvec a}{\mvec x}=1$,$x_i\geq0\mcond{i=1,2,\dotsc,m}$~
下存在最大值与最小值,其中,$\mvec a=\mparen{\mvec a{1,:,m}}$,且~$a_i>0$。
\item 设~$f(x,y,z),g(x,y,z)\in C^{(1)}$,令
\[
  F(x,y,z)=f(x,y,z)=\lambda_0g(x,y,z)。
\]
\begin{exlist}
  \item 设~$P_0\mparenb{x_0,y_0,z_0}$~是~$F$~的极小(极大)值点,且~$g\mparenb{x_0,y_0,z_0}=0$,问~$P_0$~是否是~$f(x,y,z)$~
  在条件~$g(x,y,z)=0$~下的极小(极大)值点;
  \item 设~$P_0$~是~$F$~的稳定点,且~$g\mparenb{x_0,y_0,z_0}=0$。若~Hessian~矩阵~$H_F(P_0)$~是正定(负定)的。证明~$P_0$~
  是~$f(x,y,z)$~在条件~$g(x,y,z)=0$~下的极小(极大)值点;若~$H_F(P_0)$~是不定的,能否断定~$P_0$~不是~$f(x,y,z)$~在条件
  ~$g(x,y,z)=0$~下的极值点。
\end{exlist}
\item 求下列条件极大值和条件极小值。
\begin{exlist}
  \item $x^2+y^2=1$,求~$f(x,y)=ax^2+2hxy+by^2$~的极值;
  \item $ax+by+cz=k$,求~$f(x,y,z)=x^\ell y^mz^n$~的极值,其中~$\ell,m,n\in\mN$,而~$a,b,c,k$~为常数;
  \item $\mparenb{x^2+y^2+z^2}^2=a^2x^2+b^2y^2+c^2z^2$,$\ell x+my+nz=0$,求~$f(x,y,z)=x^2+y^2+z^2$~的极值;
  \item $\ell x+my+nz=0$,$\dfrac{x^2}{a^2}+\dfrac{y^2}{b^2}+\dfrac{z^2}{c^2}=1$,求
  ~$f(x,y,z)=\dfrac{x^2}{a^4}+\dfrac{y^2}{b^4}+\dfrac{z^2}{c^4}$~的极值。
\end{exlist}
\item 设~$x^2+y^2+z^2\leq1$,求~$x^3+y^3+z^3-2xyz$~的最大值和最小值。
\item 求函数~$z=\dfrac12\mparenb{x^n+y^n}$~在条件~$x+y=\ell$~之下的极值,其中~$\ell>0$,而~$n\geq1$。并证明,当
~$a,b\geq0$,$n\geq1$~时,有
\[
  \mparenbb{\dfrac{a+b}2}^{\msp n}\leq\dfrac{a^n+b^n}2 。
\]
\item 求~$f\mparenb{\mvec x{1,2,:,n}}=x_1x_2\dotsm x_n$~在条件~$\dfrac1{x_1}+\dfrac1{x_2}+\dotsb+\dfrac1{x_n}=\dfrac1a$~之下的
极值,其中~$x_i>0$,$a>0$。并证明,当诸~$a_i>0$~时,有
\[
  n\mparenbb{\dfrac1{a_1}+\dfrac1{a_2}+\dotsb+\dfrac1{a_n}}^{\msp-1}\leq\mparenb{a_1a_2\dotsm a_n}^{\frac1n}。
\]
\item 设~$x_k>0\mcond{k=1,2,\dotsc,n}$。证明,
\begin{exlist}
  \item 若~$p>1$,则~$\sum_{k=1}^nx_k^p\geq n\mparenbb{\dfrac{x_1+x_2+\dotsb+x_n}n}^{\msp p}$;
  \item 若~$0<p<1$,则~$\sum_{k=1}^nx_k^p\leq n\mparenbb{\dfrac{x_1+x_2+\dotsb+x_n}n}^{\msp p}$。
\end{exlist}
\item 求圆的外切三角形中面积最小者。
\item 证明,椭圆的内接三角形中,面积最大的三角形的某一个顶点处的椭圆法线必与三角形的该顶点的对边垂直;并求椭圆中面积最大的内接
三角形。
\item 长为~$a$~的铁丝切成两段,一段围成一个正方形,另一段围成一个圆。确定这两段的长度,使得由它们围成的正方形和圆的面积最大。
\item 其抛物线~$y=x^2$~和直线~$x-y=1$~间的最短距离。
\item 为了减少渠道的漏水量,经常要在渠道表面铺砌一层水泥。根据流量的大小,其截面积就确定了。假定要修的渠道是直的,其横断面是一个
等腰梯形。试确定梯形的腰和底,使得水泥用量最少。
\item 凸四边形各边长分别为~$a,b,c,d$,求最大面积者。
\item 求一点~$O$,使其与一个凸四边形的四顶点距离之和最小。
\item 证明椭球面~$ax^2+by^2+cz^2+2dxy+2exz+2fyz=1$~的最大轴长~$\ell$~为如下方程之最大实根:
\[
  \begin{vmatrix}
    a-\dfrac1{\ell^2} & d & e \\
    d & b-\dfrac1{\ell^2} & f \\
    e & f & c-\dfrac1{\ell^2}
  \end{vmatrix}=0 。
\]
\item 设~$f(x,y)\in C^{(1)}(\Omega)$,在~$\Omega$~内有以~$\minto\xi\eta$~为圆心的闭圆域~$S_R$,且
~$\minto{x_0}{y_0}\in\bound{S_R}$~是~$f(x,y)$~在~$\bound{S_R}$~上的最小值点。记~$r=\smbsqrt{(x-\xi)^2+(y-\eta)^2}$,而
~$\mvec n$~是~$\bound{S_R}$~上点~$\minto{x_0}{y_0}$~的单位外法向。证明,
\begin{exlist}
  \item 若~$\pdif{f(x_0,y_0)}r=0$,则对任意单位向量~$\mvec\ell$,都有~$\pdiff{f(x_0,y_0)}{{\mvec\ell}}=0$;
  \item 若~$\pdif{f(x_0,y_0)}r<0$,则对任意单位向量~$\mvec\ell$,当~$\mangle{\mvec\ell}{\mvec n}<\dfrac\pi2$~时,有
  ~$\pdiff{f(x_0,y_0)}{{\mvec\ell}}<0$。
\end{exlist}
\end{exercise}

\section{最小二乘法}
\begin{exercise}
\item 已知~$x,y$~满足线性关系~$y=ax+b$。根据测得~$n$~对数据~$\minto{x_i}{y_i}\mcond{i=1,2,\dotsc,n}$,利用最小二乘法近似
地求出~$a$~和~$b$。
\item 在平面上已知~$n$~个点~$\minto{x_i}{y_i}\mcond{i=1,2,\dotsc,n}$。设直线
\[
  x\cos\alpha+y\sin\alpha-p=0
\]
与已知点的偏差的平方和最小。证明,
\[
  p=\mbar x\cos\alpha+\mbar y\sin\alpha,\quad
  \tan\alpha=\frac{2\mparenb{\mbar x\cdot\mbar y-\mbar{xy}}}{\mparenb{\mbar{x^2}-\mbar x^2}\mparenb{\mbar{y^2}-\mbar y^2}},
\]
其中
\[
  \mbar x=\frac1n\sum_{i=1}^nx_i,\enspace\mbar y=\frac1n\sum_{i=1}^ny_i,\enspace
  \mbar{x^2}=\frac1n\sum_{i=1}^nx_i^2,\enspace\mbar{y^2}=\frac1n\sum_{i=1}^ny_i^2,\enspace
  \mbar{xy}=\frac1n\sum_{i=1}^nx_iy_i,
\]
而且~$\mbar{x^2}-\mbar x^2\neq0$,$\mbar{y^2}-\mbar y^2\neq0$。
\item 已知~$u=Ax+By+Cz$,现观测得一组数据~$\mparen{x_i,y_i,z_i}\mcond{i=1,2,\dotsc,n}$,利用最小二乘法求系数~$A,B,C$~所满足的
三元一次方程组。
\item 设有一经验函数~$f(x)=K\me^{\alpha x}$,其中~$K,\alpha$~为未知数。又设已得数据如下表所示。
\[
  \begin{tabu} to .6\linewidth{X[2c,$$]|*5{X[3c,$$]}}
    x & 0   & 1   & 2   & 3   & 4 \\ \hline
    f & y_0 & y_1 & y_2 & y_3 & y_4
  \end{tabu}
\]
\begin{exlist}
  \item 试由最小二乘法导出~$K,\alpha$~满足的非线性方程组;
  \item 试由前两组数据定出~$K=K_0$,$\alpha=\alpha_0$;
  \item 对给定的~$x$,视~$f$~为~$K,\alpha$~的函数,求出它在~$K=K_0$,$\alpha=\alpha_0$~处的一阶近似公式;
  \item 以求得的一阶近似公式代替~$f$,由所得数据,利用最小二乘法确定出~$K,\alpha$~满足的线性方程组。
\end{exlist}
\end{exercise}

\begin{exercise*}
\item 经过空间曲线~$\Gamma$~上任意三点之平面当此三点无限趋于~$\Gamma$~上的~$P_0$~点时的极限平面称为~$\Gamma$~在~$P_0$~点
的\emph{密接面}。设有空间曲线~$\Gamma:$
\[
  x=f(t),\quad y=g(t),\quad z=h(t),
\]
其中~$f,g,h\in C^{(2)}$。证明~$t=t_0$~处~$\Gamma$~的密接面方程为
\[
  \begin{vmatrix}
    x-f(t_0) & f'(t_0) & f''(t_0)\\
    y-g(t_0) & g'(t_0) & g''(t_0)\\
    z-h(t_0) & h'(t_0) & h''(t_0)
  \end{vmatrix}=0 。
\]
\item 设
\[
  A=\begin{vmatrix}
    x_1 & x_2 & x_3 \\
    y_1 & y_2 & y_3 \\
    z_1 & z_2 & z_3
  \end{vmatrix}。
\]
用条件极值的方法证明,
\[
  \det A\leq\mparenbb{\sum_{i=1}^3x_i}^{\msp\frac12}\mparenbb{\sum_{i=1}^3y_i}^{\msp\frac12}
  \mparenbb{\sum_{i=1}^3z_i}^{\msp\frac12}
\]
\item 用条件极值的方法证明不等式,
\[
  \sum_{i=1}^n\mabsb{a_i}\mabsb{b_i}\leq\mparenbb{\sum_{i=1}^n\mabsb{a_i}^q}^{\msp\frac1q}
  \mparenbb{\sum_{i=1}^n\mabsb{b_i}^p}^{\msp\frac1p},
\]
其中~$p,q>0$,且~$\dfrac1p+\dfrac1q=1$。
\item 设~$\funcvec*f{x}\in C^{(1)}\minto{\mR{m}}{\mR{m}}$,并且存在常数~$\alpha>0$,使得对任意~$\mvec x,\mvec y\in\mR{m}$,有
\[
  \mabsb{\funcvec*f{x}-\funcvec*f{y}}\geq\alpha\mabsb{\mvec x-\mvec y}。
\]
\begin{exlist}\FixExHead
  \item 对任意~$\mvec x\in\mR{m}$,有~$\det\Dif\funcvec*f{x}\neq0$;
  \item 函数
  \[
    \vecfunc\psi{x}=\mabsb{\funcvec*f{x}-\mvec y}^2=\minpb{\funcvec*f{x}-\mvec y}{\funcvec*f{x}-\mvec y}
  \]
  在~$\mR{m}$~上存在最大值,但无最小值;
  \item $f\mparenb{\mR{m}}=\mR{m}$。
\end{exlist}
\item 记
\[
  Lu=\pdiff ut-a^2\pdiff u{x2}+b(x,t)\pdiff ux,
\]
其中~$a$~为常数,并且
\[
\begin{aligned}
  \Omega&=\mathsetb{(x,t)}{x\in\mintc0\ell,~t\in\mintc0T},\\
  \Gamma&=\mathsetb{(x,t)}{x\in\mrange01,~t\in\mintco 0T\text{;或~}x\in\mintc0\ell,~t=0}。
\end{aligned}
\]
\begin{exlist}
  \item 设~$u(x,t)$~满足当~$x\in\minto0\ell$~且~$t\in\mintoc0T$~时,有~$Lu<0$,并且~$u$~在~$\Omega$~上的最大值在某点
  ~$\minto{x_0}{t_0}$~取到,试证明~$\minto{x_0}{t_0}\in\Gamma$;
  \item 设~$u(x,t),w(x,t)\in C(\Omega)$,并满足当~$x\in\minto0\ell$~且~$t\in\mintoc0T$~时,有~$Lu=0$,$w>0$~与~$Lw<0$。证明,对
  任意~$\e>0$,有
  \[
    \max_{(x,t)\in\Omega}\mbraceb{u(x,t)}\leq\max_{(x,t)\in\Gamma}\mbraceb{u+\e w};
  \]
  \item 设~$u(x,t)\in C(\Omega)$,并满足当~$x\in\minto0\ell$~且~$t\in\mintoc0T$~时,有~$Lu=0$。证明,
  \[
    \max_{(x,t)\in\Omega}\mbraceb{u(x,t)}=\max_{(x,t)\in\Gamma}\mbraceb{u(x,t)}。
  \]
\end{exlist}
\item 设~$\sum_{i,j=1}^na_{ij}\xi_i\xi_j$~是正定二次型,且~$\vecfunc u{x}\in\mintob{\closure\Omega}{\mR}$,其中~$\Omega$~是
~$\mR{n}$~中的有界开区域。若~$u\in C^{(2)}(\Omega)$,且~$u$~在~$\closure\Omega$~上的最小值于~$\mvec x_0\in\Omega$~取到,证明,
\[
  \sum_{i,j=1}^na_{ij}\pdiff{\vecfunc u{x_0}}{{x_i}{x_j}}\geq0 。
\]
\end{exercise*}




\endinput
%%
%% End of file `MAChapter18.tex'.