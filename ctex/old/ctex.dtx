% \iffalse meta-comment
%
% Copyright 2003-2006
% CTEX.ORG and any individual authors listed elsewhere in this file.
%
% This file is part of the ctex package project.
% ----------------------------------------------
%
% It may be distributed under the conditions of the LaTeX Project Public
% License, either version 1.2 of this license or (at your option) any
% later version. The latest version of this license is in
%    http://www.latex-project.org/lppl.txt
% and version 1.2 or later is part of all distributions of LaTeX
% version 1999/12/01 or later.
%
%<*!(cfg|fd)>
% \fi
%
%% \CharacterTable
%%  {Upper-case    \A\B\C\D\E\F\G\H\I\J\K\L\M\N\O\P\Q\R\S\T\U\V\W\X\Y\Z
%%   Lower-case    \a\b\c\d\e\f\g\h\i\j\k\l\m\n\o\p\q\r\s\t\u\v\w\x\y\z
%%   Digits        \0\1\2\3\4\5\6\7\8\9
%%   Exclamation   \!     Double quote  \"     Hash (number) \#
%%   Dollar        \$     Percent       \%     Ampersand     \&
%%   Acute accent  \'     Left paren    \(     Right paren   \)
%%   Asterisk      \*     Plus          \+     Comma         \,
%%   Minus         \-     Point         \.     Solidus       \/
%%   Colon         \:     Semicolon     \;     Less than     \<
%%   Equals        \=     Greater than  \>     Question mark \?
%%   Commercial at \@     Left bracket  \[     Backslash     \\
%%   Right bracket \]     Circumflex    \^     Underscore    \_
%%   Grave accent  \`     Left brace    \{     Vertical bar  \|
%%   Right brace   \}     Tilde         \~}
%%
%
% \CheckSum{3181}
%
% \iffalse meta-comment
%</!(cfg|fd)>
%
%<*driver>
\ProvidesFile{ctex.dtx}
%</driver>
%
%<sty|cls>\NeedsTeXFormat{LaTeX2e}[1995/12/01]
%<ctex>\ProvidesPackage{ctex}
%<ctexcap>\ProvidesPackage{ctexcap}
%<article>\ProvidesClass{ctexart}
%<report>\ProvidesClass{ctexrep}
%<book>\ProvidesClass{ctexbook}
%<cjk>\ProvidesFile{ctexcjk.clo}
%<cct>\ProvidesFile{ctexcct.clo}
%<def>\ProvidesFile{ctex.def}
%<cap>\ProvidesFile{ctexcap.cfg}
%<conf>\ProvidesFile{ctex.cfg}
%<rm&fd>\ProvidesFile{c19rm.fd}
%<sf&fd>\ProvidesFile{c19sf.fd}
%<tt&fd>\ProvidesFile{c19tt.fd}
  [2007/05/06 v0.8a ctex
%<sty>   macros package]
%<cls>   document class]
%<clo>   document class option]
%<cfg>   configuration file]
%<fd>   font definition file]
%
%<*driver>
   bundle source file]
%</driver>
%
%<*driver>
\documentclass[11pt,a4paper,driverfallback=dvipdfmx]{ltxdoc}
\usepackage[c5size,fzfonts,hyperref,UTF8]{ctex-new}
\usepackage[numbered]{hypdoc}
\hypersetup{pdfstartview=FitH}
\usepackage{texnames}
 \topmargin 0.5 true cm
 \oddsidemargin 2.5 true cm
 \evensidemargin 2.5 true cm
 \textheight 21 true cm
 \textwidth 14 true cm
\EnableCrossrefs
 %\DisableCrossrefs % Say \DisableCrossrefs if index is ready
\CodelineIndex
\RecordChanges      % Gather update information
 %\OnlyDescription  % comment out for implementation details
 %\OldMakeindex     % use if your MakeIndex is pre-v2.9
\begin{document}
  \DocInput{ctex.dtx}
\end{document}
%</driver>
%
%<*def>

%</def>
%
%<*cap>

%% Chinese captions
%%
%% character set: GBK
%% encoding: EUC

%</cap>
%
%<*conf>

%</conf>
%
%<*fd>

%% Chinese characters (extension of GB 2312)
%%
%% character set: GBK
%% font encoding: CJK (extended)

%</fd>
%
% \fi
%
%
% \changes{v0.0}{2003/04/26}{Initial version}
% \changes{v0.1}{2003/08/15}{First beta release}
% \changes{v0.2}{2004/01/16}{Add support for CCT}
% \changes{v0.2c}{2004/02/13}{Add CJKpunct as standard configuration}
% \changes{v0.2d}{2004/04/23}{Change option c5size to base on 10pt basic class}
% \changes{v0.5}{2004/08/23}{Move Chinese definitions from ctex.cfg to ctex.def}
% \changes{v0.6}{2005/09/24}{Adapt to cct-0.6180-0}
% \changes{v0.7}{2005/11/25}{Read configuration files before preamble}
% \changes{v0.8}{2006/06/09}{Split ctex.sty to ctex.sty and ctexcap.sty}
%
%
% \DoNotIndex{\begin,\end,\begingroup,\endgroup}
% \DoNotIndex{\ifx,\ifdim,\ifnum,\ifcase,\else,\or,\fi}
% \DoNotIndex{\let,\def,\xdef,\newcommand,\renewcommand}
% \DoNotIndex{\expandafter,\csname,\endcsname,\relax,\protect}
% \DoNotIndex{\Huge,\huge,\LARGE,\Large,\large,\normalsize}
% \DoNotIndex{\small,\footnotesize,\scriptsize,\tiny}
% \DoNotIndex{\normalfont,\bfseries,\slshape,\interlinepenalty}
% \DoNotIndex{\hfil,\par,\vskip,\vspace,\quad}
% \DoNotIndex{\centering,\raggedright}
% \DoNotIndex{\c@secnumdepth,\@startsection,\@setfontsize}
% \DoNotIndex{\ ,\@plus,\@minus,\p@,\z@,\@m,\@M,\@ne,\m@ne}
% \DoNotIndex{\@@par}
%
%
% \GetFileInfo{ctex.dtx}
%
%
% \MakeShortVerb{\|}
% \setcounter{StandardModuleDepth}{1}
%
%
% \newcommand{\ctex}{\texttt{ctex}}
% \newcommand{\ctexorg}{\texttt{ctex.org}}
%
%
% \setlength{\parskip}{0.75ex plus .2ex minus .5ex}
% \renewcommand{\baselinestretch}{1.2}
%
%
% \makeatletter
% \def\parg#1{\mbox{$\langle${\it #1\/}$\rangle$}}
% \def\@smarg#1{{\tt\string{}\parg{#1}{\tt\string}}}
% \def\@marg#1{{\tt\string{}{\rm #1}{\tt\string}}}
% \def\marg{\@ifstar\@smarg\@marg}
% \def\@soarg#1{{\tt[}\parg{#1}{\tt]}}
% \def\@oarg#1{{\tt[}{\rm #1}{\tt]}}
% \def\oarg{\@ifstar\@soarg\@oarg}
% \makeatother
%
%
% \title{\bf \ctex{}~宏包说明\thanks
%   {本文件版本号为~\fileversion{},最后修改日期~\filedate{}。}}
% \author{\it 吴凌云\thanks{aloft@ctex.org}}
% \date{\small 打印日期:~\today}
% \maketitle
%
%
% \begin{abstract}
% \ctex{}~宏包提供了一个统一的中文~\LaTeX{}~文档框架,底层支持~CCT~和
% ~CJK~两种中文~\LaTeX{}~系统。\ctex{}~宏包提供了编写中文~\LaTeX{}~文档
% 常用的一些宏定义和命令。
%
% \ctex{}~宏包需要~CCT~系统或者~CJK~宏包的支持。
% 主要文件包括~\texttt{ctexart.cls}、
% ~\texttt{ctexrep.cls}、~\texttt{ctexbook.cls}~和
% ~\texttt{ctex.sty}、~\texttt{ctexcap.sty}。
%
% \ctex{}~宏包由~\ctexorg{}~制作并负责维护。
% \end{abstract}
%
%
% \tableofcontents
%
%
% \section{简介}
%
% 这个宏包的部分原始代码来自于由王磊编写~\texttt{cjkbook.cls}~文档类,
% 还有一小部分原始代码来自于吴凌云编写的~\texttt{GB.cap}~文件。
% 原来的这些工作都是零零碎碎编写的,没有认真、系统的设计,
% 也没有用户文档,非常不利于维护和改进。所以我们用~\texttt{doc}~
% 和~\texttt{docstrip}~工具重新编写了整个文档,并增加了许多新的功能。
%
% 最初~Knuth~设计开发~\TeX{}~的时候没有考虑到支持多国语言,
% 特别是多字节的中日韩语言。这使得~\TeX{}~以至后来的
% ~\LaTeX{}~对中文的支持一直不是很好。即使在~CJK~解决了中文字符
% 处理的问题以后,中文用户使用~\LaTeX{}~仍然要面对许多困难。
% 最常见的就是中文化的标题。由于中文习惯和西方语言的不同,
% 使得很难直接使用原有的标题结构来表示中文标题。因此需要对
% 标准~\LaTeX{}~宏包做较大的修改。此外,还有诸如中文字号的对应
% 关系等等。~\ctex{}~宏包正是尝试着解决这些问题。中间很多地方
% 用到了在~\ctexorg{}~论坛上的讨论结果,在此对参与讨论的
% 朋友们表示感谢。
%
% \ctex{}~宏包由五个主要文件构成:
% ~\texttt{ctexart.cls}、~\texttt{ctexrep.cls}、~\texttt{ctexbook.cls}~和
% ~\texttt{ctex.sty}、~\texttt{ctexcap.sty}。~\texttt{ctex.sty}~主要是提供整合的
% 中文环境,可以配合大多数文档类使用。而~\texttt{ctexcap.sty}~则是对~\LaTeX{}~
% 的三个标准文档类的格式进行修改以符合中文习惯,该宏包只能配合这三个标准文档类使用。
% ~\texttt{ctexart.cls}、~\texttt{ctexrep.cls}、~\texttt{ctexbook.cls}~则是
% ~\texttt{ctex.sty}、\texttt{ctexcap.sty}~分别和三个标准文档类结合产生的新文档类,
% 除了包含~\texttt{ctex.sty}、\texttt{ctexcap.sty}~的所有功能,还加入了一些修改文档类
% 缺省设置的内容(如使用五号字体为缺省字体)。
%
% \vskip 10pt
% {\kaishu
% 这份说明文档可以通过用~\LaTeX{}~编译~\texttt{ctex.dtx}~文件来得到。
% 编译说明文档需要~CJK~宏包和~\ctex{}~宏包。
% 为了生成正确的索引和版本记录,需要使用如下命令
% \begin{verbatim}
% makeindex -s gind.ist -o ctex.ind ctex.idx
% makeindex -s gglo.ist -o ctex.gls ctex.glo
% \end{verbatim}
% }
%
%
% \section{使用帮助}
%
% \ctex{}~宏包的使用十分简单。如果是使用~\ctex{}~的文档类,只需用
% ~\texttt{ctexart}、~\texttt{ctexrep}~或者~\texttt{ctexbook}~替换原来的
% 文档类就可以了。你也可以继续使用原来的文档类,而用~\texttt{ctex.sty}~和
% ~\texttt{ctexcap.sty}~宏包来配合使用,两者的效果是一样的
% (除了不能修改一些文档设置如缺省字体大小)。
%
% \subsection{使用~CJK}
%
% 这是~\ctex{}~宏包的缺省设置。\ctex{}~宏包会自动调用~CJK~宏包,你无需再自己调用。
% 此外,\ctex{}~宏包会在~|\begin{document}|~和~|\end{document}|~
% 之间自动加入一个~CJK~环境,你无需再添加~CJK~环境。~CJK~宏包的命令都可以
% 在~|\begin{document}|~和~|\end{document}|~之间正常使用。
%
% 例子1:使用文档类宏包
% \begin{verbatim}
% \documentclass{ctexart}
% \begin{document}
% 中文宏包测试
% \end{document}
% \end{verbatim}
%
% 例子2:使用普通宏包
% \begin{verbatim}
% \documentclass{article}
% \usepackage{ctex}
% \begin{document}
% 中文宏包测试
% \end{document}
% \end{verbatim}
%
% \subsection{使用~CCT}
% \ctex{}~宏包也可以配合新版的~CCT~使用,只需在使用~\ctex{}~宏包时加上~CCT~选项即可。
% 缺省~CCT~会使用~CJK~字库,因为这种字库方式比传统~CCT~字库更方便,兼容性也更好。
% 如果要使用传统~CCT~字库,则还要加上~CCTfont~选项。
%
% 例子3:使用~CJK~方式字库
% \begin{verbatim}
% \documentclass[CCT]{ctexart}
% \begin{document}
% 中文宏包测试
% \end{document}
% \end{verbatim}
%
% 例子2:使用~CCT~方式字库
% \begin{verbatim}
% \documentclass[CCT,CCTfont]{ctexart}
% \begin{document}
% 中文宏包测试
% \end{document}
% \end{verbatim}
%
%
% \subsection{选项}
%
% 宏包的选项用于改变一些缺省风格的设置。缺省的设置已经针对中文
% 的习惯进行了尽量的修改,所以一般用户无需使用这些选项。
% 如果你觉得某些设置不合适,可以向作者反映。我们会考虑在后面的
% 版本中予以改进。我们也欢迎关于增加或者删减选项的建议。
%
%
% 下面的选项可能会是最经常使用的。但是它们只能用于文档类
% (\texttt{ctexart}、~\texttt{ctexrep}~和~\texttt{ctexbook})。
% \begin{description}
% \item[cs4size] 使用小四字号为缺省字体大小。
% \item[c5size] 使用五号字为缺省字体大小。{\heiti 这个是
% ~\ctex{}~文档类的缺省格式。}
% \end{description}
%
%
% 下面这些则可以在文档类宏包和~\texttt{ctex.sty}~上使用。
% \begin{description}
% \item[CCT] 使用~CCT~代替~CJK~做为底层的中文支持系统。
%
% \item[CCTfont] 使用传统的~CCT~字库方式,该选项会自动激活~CCT~选项。
%
% \item[punct] 对中文标点的位置(宽度)进行调整。
%
% \item[nopunct] 不对中文标点的位置进行调整(每个标点占有相同的宽度)。
% \item[space] 使用~CJK~的保留空格模式,保留中文字符间的空格(类似英文的
% 习惯)。你需要自己处理中文字符间的空格以及换行产生的空格(在行尾加上
% ~\%~符号可以避免),否则排版结果可能不符合中文习惯。这种模式可以通过
% ~|\CTEXnospace|~转换到~nospace~模式。
%
% \item[nospace] 使用~CJK~的忽略空格模式,也就是~CJK*~环境的模式。
% ~CJK~会自动忽略中文字符间的空格,比较符合中文习惯。在这种模式下,
% 可以使用~\textasciitilde~来分隔中英文字符,产生的间距稍小于普通空格,
% 排版效果比较美观。这种模式可以通过~|\CTEXspace|~命令转换到~space~模式。
% {\heiti 这个是~\ctex{}~宏包的缺省模式。}
%
% \item[cap] 使用中文的标题样式。{\heiti 这个是文档类宏包的缺省模式。}
%
% \item[nocap] 保留使用英文的标题样式。
%
% \item[indent] 使用中文的段首缩进模式,即缩进两个汉字宽度,同时每个段落
% 都缩进。{\heiti 这个是~\ctex{}~宏包的缺省模式。}
%
% \item[noindent] 使用原来的段首缩进模式,章节标题后的第一段不缩进。
%
% \item[psfont] 使用~PostScript~字库来代替~CM~字库。这个选项只影响英文
% 字库的使用,对中文没有作用。
%
% \item[fancyhdr] 保持和~\texttt{fancyhdr}~宏包的兼容性。该选项将使得
% ~\texttt{fancyhdr}~宏包被自动调用。
%
% \item[amstex] 保持和~\AMSLaTeX{}~宏包的兼容性。
%
% \item[fntef] 为~\texttt{CJKfntef}~宏包和~\texttt{CCTfntef}~宏包提供统一接口。
% 该选项将使得~\texttt{CJKfntef}~宏包或者~\texttt{CCTfntef}~宏包被自动调用。
% \end{description}
%
%
% 下面这些则可以在文档类宏包和~\texttt{ctexcap.sty}~上使用。
% \begin{description}
% \item[cap] 使用中文的标题样式,缺省格式由~\texttt{ctexcap.cfg}~配置文件
% 内的定义给出。{\heiti 这个是文档类宏包的缺省模式。}
%
% \item[nocap] 保留使用英文的标题样式。
%
% \item[sub3section] 将~|\paragraph|~命令产生的标题改为~section~类格式。
% 此时~|\subparagraph|~命令产生的标题会具有原来~|\paragraph|~的格式。
%
% \item[sub4section] 将~|\paragraph|~和~|\subparagraph|~命令产生的标题
% 都改为~section~类格式。
% \end{description}
%
%
% \vskip 10pt
% {\kaishu
% 总结:\ctex{}~宏包的缺省选项是~nospace cap indent,文档类
% 的缺省选项是~nospace cap indent c5size。
% }
%
%
% \subsection{基本命令}
%
% \ctex{}~宏包给用户提供一个通用的文档框架,使得用户可以自由地在不同的
% 底层中文系统间切换。为此,我们为~CJK~定制了一些模拟~CCT~的命令,
% 也对部分~CCT~命令进行了修改,使得两者保持一致。
% 此外,我们还定义了用于设置文档参数的高级设置命令。
%
% \subsubsection{字体设置}
%
% 中文字体很多,但是常用的就那么几个。我们为~CJK~常用的六种中文
% 字体定义了简单易用的命令。它们是:
%
% \DescribeMacro{\songti}
% 宋体:~|\songti|,~CJK~等价命令~|\CJKfamily{song}|
%
% \DescribeMacro{\heiti}
% 黑体:~|\heiti|,~CJK~等价命令~|\CJKfamily{hei}|
%
% \DescribeMacro{\fangsong}
% 仿宋:~|\fangsong|,~CJK~等价命令~|\CJKfamily{fs}|
%
% \DescribeMacro{\kaishu}
% 楷书:~|\kaishu|,~CJK~等价命令~|\CJKfamily{kai}|
%
% \DescribeMacro{\lishu}
% 隶书:~|\lishu|,~CJK~等价命令~|\CJKfamily{li}|
%
% \DescribeMacro{\youyuan}
% 幼圆:~|\youyuan|,~CJK~等价命令~|\CJKfamily{you}|
%
% \vskip 10pt
% {\kaishu
% \TeX{}~系统中必须已经定义好这六种中文字体,并且使用和~\CTeX{}~套装中
% 一致的字体名称。(参见上面~CJK~等价命令的参数)
%
% 上面的字体命令和~CCT~中的一致,但传统的~CCT~字库中没有隶书和
% 仿宋两种字体,需要用户自行安装定义。如果使用~CCT~时选择~CJK~字库方式,
% 则可以使用这两种中文字体。
% }
%
% \subsubsection{字号、字距、字宽和缩进}
%
% \DescribeMacro{\zihao}
% 中文字号的设置命令是~|\zihao|\marg*{字号},例如~|\zihao{3}|。
% 可以使用的参数有~16~个,小号字体在前面加负号表示,从大到小依次为
% \begin{center}
% \begin{tabular}{cccccccc}
% \hline
% 初号 & 小初 & 一号 & 小一 & 二号 & 小二 & 三号 & 小三 \\
% 0 & -0 & 1 & -1 & 2 & -2 & 3 & -3 \\
% \hline
% 四号 & 小四 & 五号 & 小五 & 六号 & 小六 & 七号 & 八号 \\
% 4 & -4 & 5 & -5 & 6 & -6 & 7 & 8 \\
% \hline
% \end{tabular}
% \end{center}
% \noindent 英文字体大小会始终保持和中文字体一致。
%
% \DescribeMacro{\ziju}
% 汉字字距的调整使用命令~|\ziju|\marg*{字宽的倍数}。参数可以是任意的数字,
% 例如~|\ziju{5}|~设置汉字字距为当前汉字字宽的~5~倍,~|\ziju{0.5}|~设置汉字
% 字距为当前汉字字宽的一半。这里的汉字字宽指的是实际汉字的宽度,
% 不包含当前字距。该命令不影响英文字距。
%
% \DescribeMacro{\ccwd}
% 当前汉字的字宽保存在宏~|\ccwd|~中。字宽是相邻两个汉字中心的距离,
% 也就是说字距会被计算在内。
%
% \DescribeMacro{\CTEXindent}
% 正常的缩进两个汉字字宽的距离,同时在汉字大小和字距改变的
% 情况都可以自动修改缩进距离。
%
% \DescribeMacro{\CTEXnoindent}
% 取消缩进。
%
% \DescribeMacro{\CTEXsetfont}
% |\CTEXsetfont|~命令用于更新当前的中文字体信息,包括当前字距和缩进
% 距离。一般来说,用户无需使用这个命令。
%
%
% \subsubsection{中文数字转换}
%
% \DescribeMacro{\CTEXnumber}
% 使用~CJK~提供的~|\CJKnumber|~命令可以将阿拉伯数字转换为中文数字。
% 由于~\LaTeX{}~臭名昭著的脆弱命令的原因,当~|\CJKnumber|~被用在
% 章节标题等地方的时候,要么出现错误无法使用,要么无法达到预期目的,
% 例如在产生~PDF~书签的时候。于是我们定义了一个~|\CTEXnumber|~命令,
% 可以将产生的中文数字保存下来。该命令的格式为
% \begin{quote}
% |\CTEXnumber|\marg*{result}\marg*{number}
% \end{quote}
% 其中~\parg{result}~必须是一个~\TeX{}~宏的名字,不需要预先定义。
% 例如
% \begin{quote}
% |\CTEXnumber{\test}{100002005}|
% \end{quote}
% 则~|\test|~中的内容就是“一亿零二千零五”(不包括引号)。
%
% \DescribeMacro{\CTEXdigits}
% |\CTEXdigits|~命令和~|\CTEXnumber|~命令类似,用于代替~CJK~提供的
% ~|\CJKdigits|~命令。它和~|\CTEXnumber|~命令的不同之处在于转换后
% 结果是中文数字串,而不是按照中文习惯的数字。该命令的格式为
% \begin{quote}
% |\CTEXdigits|\marg*{result}\marg*{number}
% \end{quote}
% 其中~\parg{result}~必须是一个~\TeX{}~宏的名字,不需要预先定义。
% 例如
% \begin{quote}
% |\CTEXnumber{\test}{100002005}|
% \end{quote}
% 则~|\test|~中的内容就是“一○○○○二○○五”(不包括引号)。
%
% \DescribeMacro{\chinese}
% 对于经常需要转换的计数器,我们特别定义了一个~|\chinese|~命令。
% 该命令可以象罗马数字转换命令~|\roman|、~|\Roman|~一样使用。
% 具体格式是
% \begin{quote}
% |\chinese|\marg*{counter}
% \end{quote}
% 其中~\parg{counter}~是一个~\LaTeX{}~计数器(counter),即由
% ~|\newcounter|~命令产生的,例如~|section|,~|figure|~等。
%
%
% \subsection{高级设置}
%
% \DescribeMacro{\CTEXoptions}
% \ctex{}~宏包中一般的设置通过~|\CTEXoptions|~命令完成。
% 这个命令的基本格式是
% \begin{quote}
% |\CTEXoptions|\oarg{\parg{key1}={\parg{val1}},
%                   \parg{key2}={\parg{val2}}, ...}
% \end{quote}
% 其中~\parg{key1}, \parg{key2}~是设置选项,
% ~\parg{val1}, \parg{val2}~则是对应选项的设置内容。
% 多个选项可以在一个语句中完成设置。
%
% \DescribeMacro{\CTEXsetup}
% 部分设置如章节标题则通过~|\CTEXsetup|~命令完成。这个命令比
% ~|\CTEXoptions|~多一个参数,用于指定设置对象。
% 基本格式是
% \begin{quote}
% |\CTEXsetup|\oarg{\parg{key1}={\parg{val1}},
%                   \parg{key2}={\parg{val2}}, ...}\marg*{type}
% \end{quote}
% 其中~\parg{type}~是设置的对象类型,如~|part|, |chapter|, |section|,
% |subsection|, |subsubsection|, |paragraph|, |subparagraph|~等。
% ~\parg{key1}, \parg{key2}~是设置选项,如~|name|, |number|, |format|,
% |nameformat|, |numberformat|, |aftername|, |titleformat|~等。
% ~\parg{val1}, \parg{val2}~则是对应选项的设置内容。
% 同一个目标类型的多个选项可以在一个语句中完成设置。
%
% {\bf 如果以上命令的参数中包含中文字符,则命令必须放在
% ~|\begin{document}|~之后才能正常工作。}
% \footnote{从~v0.7~版本开始支持在导言区使用中文。}
%
%
% \subsubsection{章节标题设置}
%
% 普通章节标题的格式全部通过~|\CTEXsetup|~命令完成。
% 章节类型在~|\CTEXsetup|~命令的第二个参数中指定。
% {\bf 如果使用了宏包选项~cap~(缺省情况即是如此),则所有
% 对章节标题的修改必须在~|\begin{document}|~以后进行。原因是
% 缺省的中文标题设置文件~\texttt{ctexcap.cfg}~文件是在
% ~|\begin{document}|~之后才会自动装入,因而之前的修改都
% 会被覆盖而无效。}这一限制对后面的附录标题以及其他标题设置
% 一样有效。\footnote{从~v0.7~版本开始,\texttt{ctexcap.cfg}~文件
% 在宏包文件结束时就已经被装入,因此可以在导言区使用设置命令。}
%
% \begin{description}
%
% \item[name=\{\parg{prename},\parg{postname}\}]
% 该选项用于设置章节的名字,包括章节编号前后的词语,两个之间用逗号分开。
% 例如
% \begin{quote}
% |\CTEXsetup[name={第,节}]{section}|
% \end{quote}
% 会使得~section~的标题使用形如“第1节”的名字。注意{\bf 不要}使用中文
% 的逗号。
%
% 该选项的缺省设置是
% \begin{center}
% \begin{tabular}{lll}
% \hline\hline
%    & 使用宏包选项~cap~ & 使用宏包选项~nocap~ \\
% \hline
% part & \{第,部分\} & \{Part\cs{space},\} \\
% chapter & \{第,章\} & \{Chapter\cs{space},\} \\
% section & 同右 & \{,\} \\
% subsection & 同右 & \{,\} \\
% subsubsection & 同右 & \{,\} \\
% paragraph & 同右 & \{,\} \\
% subparagraph & 同右 & \{,\} \\
% \hline\hline
% \end{tabular}
% \end{center}
%
% \item[number=\{\parg{number}\}]
% 该选项用于设置章节编号的数字样式。例如
% \begin{quote}
% |\CTEXsetup[number={\roman{section}}]{section}|
% \end{quote}
% 会使得~section~的标题使用小写罗马数字作为编号。常用的数字样式命令有
% \begin{description}
% \item \cs{chinese}\marg*{counter}: 一, 二, 三, ...
% \item \cs{arabic}\marg*{counter}: 1, 2, 3, ...
% \item \cs{roman}\marg*{counter}: i, ii, iii, ...
% \item \cs{Roman}\marg*{counter}: I, II, III, ...
% \item \cs{alph}\marg*{counter}: a, b, c, ...
% \item \cs{Alph}\marg*{counter}: A, B, C, ...
% \end{description}
%
% 该选项的缺省设置是
% \begin{center}
% \begin{tabular}{lll}
% \hline\hline
%    & 使用宏包选项~cap~ & 使用宏包选项~nocap~ \\
% \hline
% part & \{\cs{chinese}\marg{part}\} & \{\cs{Roman}\marg{part}\} \\
% chapter & \{\cs{chinese}\marg{chapter}\} & \{\cs{arabic}\marg{chapter}\} \\
% section & 同右 & \{\cs{thesection}\} \\
% subsection & 同右 & \{\cs{thesubsection}\} \\
% subsubsection & 同右 & \{\cs{thesubsubsection}\} \\
% paragraph & 同右 & \{\cs{theparagraph}\} \\
% subparagraph & 同右 & \{\cs{thesubparagraph}\} \\
% \hline\hline
% \end{tabular}
% \end{center}
%
% \item[format=\{\parg{format}\}]
% 用于控制章节标题的全局格式,作用域为章节名字和随后的标题内容。
% 常用于控制章节标题的对齐方式。
%
% 该选项的缺省设置是
% \begin{center} \small
% \begin{tabular}{lll}
% \hline\hline
%    & 使用宏包选项~cap~ & 使用宏包选项~nocap~ \\
% \hline
% part (article) & \{\cs{centering}\} & \{\cs{raggedright}\} \\
% part & \{\cs{centering}\} & \{\cs{centering}\} \\
% chapter & \{\cs{centering}\} & \{\cs{raggedright}\} \\
% section & \{\cs{Large}\cs{bfseries}\cs{centering}\} & \{\cs{Large}\cs{bfseries}\} \\
% subsection & \{\cs{large}\cs{bfseries}\cs{centering}\} & \{\cs{large}\cs{bfseries}\} \\
% subsubsection & \{\cs{normalsize}\cs{bfseries}\cs{centering}\} & \{\cs{normalsize}\cs{bfseries}\} \\
% paragraph & \{\cs{normalsize}\cs{bfseries}\cs{centering}\} & \{\cs{normalsize}\cs{bfseries}\} \\
% subparagraph & \{\cs{normalsize}\cs{bfseries}\cs{centering}\} & \{\cs{normalsize}\cs{bfseries}\} \\
% \hline\hline
% \end{tabular}
% \end{center}
%
% \item[nameformat=\{\parg{nameformat}\}]
% 用于控制章节名字的格式,作用域为章节名字,包括编号。
%
% 该选项的缺省设置是
% \begin{center}
% \begin{tabular}{lll}
% \hline\hline
%    & 使用宏包选项~cap~ & 使用宏包选项~nocap~ \\
% \hline
% part (article) & 同右 & \{\cs{Large}\cs{bfseries}\} \\
% part & 同右 & \{\cs{huge}\cs{bfseries}\} \\
% chapter & 同右 & \{\cs{huge}\cs{bfseries}\} \\
% section & 同右 & \{\} \\
% subsection & 同右 & \{\} \\
% subsubsection & 同右 & \{\} \\
% paragraph & 同右 & \{\} \\
% subparagraph & 同右 & \{\} \\
% \hline\hline
% \end{tabular}
% \end{center}
%
% \item[numberformat=\{\parg{numberformat}\}]
% 用于控制章节编号的格式。一般为空,当你需要编号的格式和前后的章节名字
% 不一样时使用。
%
% \item[aftername=\{\parg{aftername}\}]
% 用于控制章节标题中章节名字和随后的标题内容之间的格式变换。
% 常用于控制标题内容是否另起一行。
%
% 该选项的缺省设置是
% \begin{center}
% \begin{tabular}{lll}
% \hline\hline
%    & 使用宏包选项~cap~ & 使用宏包选项~nocap~ \\
% \hline
% part (article) & \{\cs{quad}\} & \{\cs{par}\cs{nobreak}\} \\
% part & 同右 & \{\cs{par}\cs{vskip} 20pt\} \\
% chapter & \{\cs{quad}\} & \{\cs{par}\cs{vskip} 20pt\} \\
% section & 同右 & \{\} \\
% subsection & 同右 & \{\} \\
% subsubsection & 同右 & \{\} \\
% paragraph & 同右 & \{\} \\
% subparagraph & 同右 & \{\} \\
% \hline\hline
% \end{tabular}
% \end{center}
%
% \item[titleformat=\{\parg{titleformat}\}]
% 用于控制标题内容的格式,作用域为章节标题内容。
%
% 该选项的缺省设置是
% \begin{center}
% \begin{tabular}{lll}
% \hline\hline
%    & 使用宏包选项~cap~ & 使用宏包选项~nocap~ \\
% \hline
% part (article) & \{\cs{Large}\cs{bfseries}\} & \{\cs{huge}\cs{bfseries}\} \\
% part & \{\cs{huge}\cs{bfseries}\} & \{\cs{Huge}\cs{bfseries}\} \\
% chapter & \{\cs{huge}\cs{bfseries}\} & \{\cs{Huge}\cs{bfseries}\} \\
% section & 同右 & \{\} \\
% subsection & 同右 & \{\} \\
% subsubsection & 同右 & \{\} \\
% paragraph & 同右 & \{\} \\
% subparagraph & 同右 & \{\} \\
% \hline\hline
% \end{tabular}
% \end{center}
%
% \item[beforeskip=\{\parg{beforeskip}\}]
% 用于控制章节标题前的空距。
%
% 该选项的缺省设置是
% \begin{center}
% \begin{tabular}{lll}
% \hline\hline
%    & 使用宏包选项~cap~ & 使用宏包选项~nocap~ \\
% \hline
% part (article) & 同右 & \{4ex\} \\
% part & 无效 & 无效 \\
% chapter & 同右 & \{50pt\} \\
% section & 同右 & \{-3.5ex plus -1ex minus -.2ex\} \\
% subsection & 同右 & \{-3.25ex plus -1ex minus -.2ex\} \\
% subsubsection & 同右 & \{-3.25ex plus -1ex minus -.2ex\} \\
% paragraph & 同右 & \{3.25ex plus 1ex minus .2ex\} \\
% subparagraph & 同右 & \{3.25ex plus 1ex minus .2ex\} \\
% \hline\hline
% \end{tabular}
% \end{center}
%
% 在~section~及以下的标题中,使用负的距离表示标题后的段落不缩进
% (如标准的英文~LaTeX~文档),否则缩进。标题上方真正的空距是该参数的绝对值。
%
% \item[afterskip=\{\parg{afterskip}\}]
% 用于控制章节标题后的空距。
%
% 该选项的缺省设置是
% \begin{center}
% \begin{tabular}{lll}
% \hline\hline
%    & 使用宏包选项~cap~ & 使用宏包选项~nocap~ \\
% \hline
% part (article) & 同右 & \{3ex\} \\
% part & 无效 & 无效 \\
% chapter & 同右 & \{40pt\} \\
% section & 同右 & \{2.3ex plus .2ex\} \\
% subsection & 同右 & \{1.5ex plus .2ex\} \\
% subsubsection & 同右 & \{1.5ex plus .2ex\} \\
% paragraph & 同右 & \{-1em\} \\
% subparagraph & 同右 & \{-1em\} \\
% \hline\hline
% \end{tabular}
% \end{center}
%
% 在~section~及以下的标题中,正的距离表示向下留出的空距(如标准的~section~标题),
% 使用负的距离则表示向右留出的空距的负值(如标准的~paragraph~标题)。
%
% \item[indent=\{\parg{indent}\}]
% 用于控制章节标题本身的缩进。
%
% 该选项的缺省设置是
% \begin{center}
% \begin{tabular}{lll}
% \hline\hline
%    & 使用宏包选项~cap~ & 使用宏包选项~nocap~ \\
% \hline
% part (article) & 同右 & \{0pt\} \\
% part & 无效 & 无效 \\
% chapter & 同右 & \{0pt\} \\
% section & 同右 & \{0pt\} \\
% subsection & 同右 & \{0pt\} \\
% subsubsection & 同右 & \{0pt\} \\
% paragraph & 同右 & \{0pt\} \\
% subparagraph & 同右 & \{\cs{parindent}\} \\
% \hline\hline
% \end{tabular}
% \end{center}
%
% \end{description}
%
%
% \subsubsection{部分修改标题格式}
%
% 如果只想修改标题格式中的某些参数而不是完全重新设置,可以使用带~+~号的
% 设置选项。例如
% \begin{quote}
% |\CTEXsetup[format+={\zihao{1}}]{section}|
% \end{quote}
% 则~section~的标题使用一号字体,而其他格式设置保持不变。
%
% 标题格式相关的选项都支持这一功能,包括~|format|, |nameformat|, |numberformat|,
% |aftername|~和~|titleformat|,而且对所有文档类型都有效。
%
% \subsubsection{附录标题设置}
%
% 附录(appendix)的标题也使用~|\CTEXsetup|~命令进行设置,
% 第二个参数设为~|appendix|。但是只能使用~|name|~和~|number|~
% 两个设置选项。在使用了~|\appendix|~命令之后,附录
% 的名字和编号会被自动使用。{\bf 附录的名字和前面的章节不同,
% 它只有一个部分,放在编号之前。}在~article~类文档中,
% 附录是用~section~实现的,而在~report~和~book~类文档中附录
% 使用的是~chapter~的设置。因此在设置附录的编号的时候要注意
% 使用正确的计数器。如果你要设置其他格式的附录标题,
% 可以根据使用的文档类直接用~section~或者~chapter~的设置命令来控制,
% 但是要记住把设置命令放在~|\appendix|~(如果有的话)的后面,
% 否则会被~|\appendix|~命令的设置覆盖。
%
% 附录的缺省设置是
% \begin{center}
% \begin{tabular}{lll}
% \hline\hline
%    & 使用宏包选项~cap~ & 使用宏包选项~nocap~ \\
% \hline
% name (article) & 同右 & \{\} \\
% name & \{附录\textasciitilde\} & \{Appendix\cs{space}\} \\
% number (article) & 同右 & \{\cs{Alph}\marg{section}\} \\
% number & 同右 & \{\cs{Alph}\marg{chapter}\} \\
% \hline\hline
% \end{tabular}
% \end{center}
%
%
% \subsubsection{其他标题设置}
%
% 除章节标题外其他标题的设置通过~|\CTEXoptions|~设置。包括
%
% \begin{description}
% \item[contentsname] 目录名
% \item[listfigurename] 表格目录
% \item[listtablename] 插图目录
% \item[figurename] 图
% \item[tablename] 表
% \item[abstractname] 摘要
% \item[indexname] 索引
% \item[bibname] 参考文献
% \end{description}
%
% 例如
% \begin{quote}
% |\CTEXoptions[indexname={总索引}]|
% \end{quote}
% 把索引的名字改为“总索引”。
%
%
% \subsubsection{其他设置}
%
% \paragraph{设置~\cs{today}~的日期格式}
% 使用~|\CTEXoptions|~可以设置~|\today|~命令产生的日期格式。
% 支持的格式包括
%
% \begin{enumerate}
%
% \item 阿拉伯数字加中文年月日
% \begin{quote}
% |\CTEXoptions[today=small]|
% \end{quote}
% \CTEXoptions[today=small]
% |\today|~生成的日期例子为“\today”。
%
% \item 中文数字加中文年月日
% \begin{quote}
% |\CTEXoptions[today=big]|
% \end{quote}
% \CTEXoptions[today=big]
% |\today|~生成的日期例子为“\today”。
%
% \item \LaTeX{}~标准格式
% \begin{quote}
% |\CTEXoptions[today=old]|
% \end{quote}
% \CTEXoptions[today=old]
% |\today|~生成的日期例子为“\today”。
%
% \end{enumerate}
%
%
% \paragraph{设置图表标题的分隔符}
% 使用~|\CTEXoptions|~可以设置~|\caption|~命令产生的图表标题的分隔符。
% 这个分隔符缺省是使用冒号~:~。可以通过命令
% \begin{quote}
% |\CTEXoptions[captiondelimiter={|\parg{string}|}]|
% \end{quote}
% 设置为任意的单个字符或者字符串~\parg{string}。
%
%
% \subsection{配置文件}
%
% 主要的配置文件有~\texttt{ctex.def}~和~\texttt{ctexcap.cfg}~以及几个字体
% 定义文件~\texttt{*.fd}。字体定义文件的内容请参考~\ref{sec:fontdef}~的
% 内容。
%
% \texttt{ctex.def}~是一些中文字符串参数的定义,会被所有的宏包使用。
% 如果你想改用其他的中文字符,例如繁体字,可以修改这个文件。
%
% \texttt{ctexcap.cfg}~是缺省中文标题格式的定义,当你使用~\texttt{cap}~
% 选项时就会使用这里的定义。你可以把它改为你经常使用的格式,这样就不用
% 每次都在正文中修改了。~\texttt{ctexcap.cfg}~中的设置都可以通过宏包提供
% 的设置命令在正文中进行修改。
%
% 最后,宏包还将读入~\texttt{ctex.cfg}~文件,该文件中的设置将覆盖其他配置
% 文件中的设置。用户可以在该文件中加入自己的定义。
%
% 在修改这些配置文件的时候,你可以修改系统目录中的文件,也可以拷贝一份放到
% 当前目录下,然后修改。~TeX~会优先使用当前目录下的同名文件。这样你可以针对
% 不同的应用设置不同的缺省配置文件。
%
%
%
%
% \StopEventually{
% } ^^A end StopEventually
%
%
%
%
% \section{源代码说明}
%
%
% \subsection{选项}
%
%
% \begin{macro}{\ifCTEX@cct}
% 用于判断是否使用~CCT~中文系统,缺省是不使用。
%    \begin{macrocode}
%<*ctex|cls>
\newif\ifCTEX@cct \CTEX@cctfalse
\DeclareOption{CCT}{\CTEX@ccttrue}
%</ctex|cls>
%    \end{macrocode}
% \end{macro}
%
% \begin{macro}{\ifCTEX@cctfont}
% 用于判断~CCT~是使用传统~CCT~字库还是~CJK~字库,缺省是使用~CJK~字库。
%    \begin{macrocode}
%<*ctex|cls>
\newif\ifCTEX@cctfont \CTEX@cctfontfalse
\DeclareOption{CCTfont}{\CTEX@ccttrue\CTEX@cctfonttrue}
%</ctex|cls>
%    \end{macrocode}
% \end{macro}
%
% \begin{macro}{\ifCTEX@punct}
% \changes{v0.2c}{2004/02/13}{增加判断是否调整中文标点宽度的选项}
% 用于判断是否对中文标点宽度进行调整,缺省是调整。
%    \begin{macrocode}
%<*ctex|cls>
\newif\ifCTEX@punct \CTEX@puncttrue
\DeclareOption{punct}{\CTEX@puncttrue}
\DeclareOption{nopunct}{\CTEX@punctfalse}
%</ctex|cls>
%    \end{macrocode}
% \end{macro}
%
% \begin{macro}{\ifCTEX@space}
% 用于判断是否忽略汉字间的空格,缺省是忽略。
%    \begin{macrocode}
%<*ctex|cls>
\newif\ifCTEX@space \CTEX@spacefalse
\DeclareOption{space}{\CTEX@spacetrue}
\DeclareOption{nospace}{\CTEX@spacefalse}
%</ctex|cls>
%    \end{macrocode}
% space~参数使得缺省的中文环境不会吃掉中文字符后面的空格。
% 这种情况下很容易造成汉字之间产生多余的空格,需要小心使用。
% \end{macro}
%
% \begin{macro}{\ifCTEX@caption}
% 用于判断是否使用中文标题,缺省是使用。
%    \begin{macrocode}
%<*ctex|ctexcap|cls>
\newif\ifCTEX@caption \CTEX@captiontrue
\DeclareOption{nocap}{\CTEX@captionfalse}
\DeclareOption{cap}{\CTEX@captiontrue}
%</ctex|ctexcap|cls>
%    \end{macrocode}
% \end{macro}
%
% \begin{macro}{\ifCTEX@indent}
% 用于判断是否使用中文的缩进格式,缺省是使用。
%    \begin{macrocode}
%<*ctex|cls>
\newif\ifCTEX@indent \CTEX@indenttrue
\DeclareOption{noindent}{\CTEX@indentfalse}
\DeclareOption{indent}{\CTEX@indenttrue}
%</ctex|cls>
%    \end{macrocode}
% \end{macro}
%
% \begin{macro}{\ifCTEX@psnfss}
% 用于判断是否使用~PostScript~字体替代~CM~字体,缺省是不使用。
% psnfss~选项使得~\LaTeX{}~使用~PostScript~字体替代缺省的~CM~字体。
%    \begin{macrocode}
%<*ctex|cls>
\newif\ifCTEX@psfont \CTEX@psfontfalse
\DeclareOption{psfont}{\CTEX@psfonttrue}
%</ctex|cls>
%    \end{macrocode}
% \end{macro}
%
% \changes{v0.3b}{2004/05/11}{增加~fancyhdr~选项}
% \begin{macro}{\ifCTEX@fancyhdr}
% 用于判断是否使用~\texttt{fancyhdr}~宏包,缺省是不使用。
% fancyhdr~选项使得~\texttt{ctex}~宏包保持和~\texttt{fancyhdr}~宏包兼容。
%    \begin{macrocode}
%<*ctex|cls>
\newif\ifCTEX@fancyhdr \CTEX@fancyhdrfalse
\DeclareOption{fancyhdr}{\CTEX@fancyhdrtrue}
%</ctex|cls>
%    \end{macrocode}
% \end{macro}
%
% \changes{v0.7}{2005/11/25}{增加~fntef~选项}
% \begin{macro}{\ifCTEX@fntef}
% 用于判断是否使用~\texttt{CJKfntef}~或者~\texttt{CCTfntef}~宏包,缺省是不使用。
% fntef~选项为使用~\texttt{CJKfntef}~和~\texttt{CCTfntef}~宏包提供了统一接口。
%    \begin{macrocode}
%<*ctex|cls>
\newif\ifCTEX@fntef \CTEX@fnteffalse
\DeclareOption{fntef}{\CTEX@fnteftrue}
%</ctex|cls>
%    \end{macrocode}
% \end{macro}
%
% \changes{v0.1a}{2003/08/15}{修正~\texttt{ctex.sty}~中无法使用
%                 ~sub3section~和~sub4section~选项的问题}
% \changes{v0.3}{2004/04/24}{修正~sub3section~和~sub4section~选项无效的问题}
% 支持~|\subsubsection|~以下的小节标题编号,通过修改~|\paragraph|~和
% ~|\subparagraph|~实现。
%    \begin{macrocode}
%<*ctexcap|cls>
\newcounter{CTEX@sectiondepth}
\setcounter{CTEX@sectiondepth}{2}
\DeclareOption{sub3section}{\setcounter{CTEX@sectiondepth}{3}}
\DeclareOption{sub4section}{\setcounter{CTEX@sectiondepth}{4}}
%</ctexcap|cls>
%    \end{macrocode}
%
%
% 用于文档类的一些选项:
%
% \begin{macro}{\ifCTEX@sfoursize}
% 用于判断是否使用中文小四号字,缺省是不使用。
%    \begin{macrocode}
%<*cls>
\newif\ifCTEX@sfoursize \CTEX@sfoursizefalse
\DeclareOption{cs4size}{\CTEX@sfoursizetrue\CTEX@fivesizefalse}
%</cls>
%    \end{macrocode}
% \end{macro}
%
% \begin{macro}{\ifCTEX@fivesize}
% 用于判断是否使用中文五号字,缺省是使用。
%    \begin{macrocode}
%<*cls>
\newif\ifCTEX@fivesize \CTEX@fivesizetrue
\DeclareOption{c5size}{\CTEX@fivesizetrue\CTEX@sfoursizefalse}
%</cls>
%    \end{macrocode}
% \end{macro}
%
% \changes{v0.4}{2004/05/13}{如果指定了标准的~\LaTeX{}~字体大小,则不使用中文字号}
% 如果指定了标准的~\LaTeX{}~字体大小(10pt/11pt/12pt),则不使用中文字号作为缺省大小。
%    \begin{macrocode}
%<*cls>
\DeclareOption{10pt}{%
  \CTEX@sfoursizefalse\CTEX@fivesizefalse%
  \PassOptionsToClass{\CurrentOption}%
%<article>    {article}}
%<report>    {report}}
%<book>    {book}}
\DeclareOption{11pt}{%
  \CTEX@sfoursizefalse\CTEX@fivesizefalse%
  \PassOptionsToClass{\CurrentOption}%
%<article>    {article}}
%<report>    {report}}
%<book>    {book}}
\DeclareOption{12pt}{%
  \CTEX@sfoursizefalse\CTEX@fivesizefalse%
  \PassOptionsToClass{\CurrentOption}%
%<article>    {article}}
%<report>    {report}}
%<book>    {book}}
%</cls>
%    \end{macrocode}
%
%
% 把没有定义的选项传递给缺省的文档类
%    \begin{macrocode}
%<cls>\DeclareOption*{\PassOptionsToClass{\CurrentOption}%
%<article>  {article}}
%<report>  {report}}
%<book>  {book}}
%    \end{macrocode}
%
% 处理选项
%    \begin{macrocode}
%<sty|cls>\ProcessOptions
%    \end{macrocode}
%
%
% 如果使用中文的缺省字号大小,则需要传递合适的参数给标准的~\LaTeX{}~文档类。
%    \begin{macrocode}
%<*cls>
\ifCTEX@sfoursize
  \PassOptionsToClass{12pt}%
%<article>    {article}
%<report>    {report}
%<book>    {book}
\fi
\ifCTEX@fivesize
  \PassOptionsToClass{10pt}%
%<article>    {article}
%<report>    {report}
%<book>    {book}
\fi
%</cls>
%    \end{macrocode}
%
%
% 装入缺省的文档类
%    \begin{macrocode}
%<article>\LoadClass{article}
%<report>\LoadClass{report}
%<book>\LoadClass{book}
%    \end{macrocode}
%
%
%
%
% \subsection{宏包}
%
%
% \changes{v0.3a}{2004/04/30}{修改命令~\cs{CCTpuncttrue}~的拼写错误}
% 我们需要使用的一些宏包,其中~cctbase~和~CJK~用来处理中文:
%    \begin{macrocode}
%<*cct>
\ifCTEX@cctfont
  \RequirePackage{cctbase}[2003/04/05]
\else
  \RequirePackage[CJK]{cctbase}[2003/04/05]
\fi
\ifCTEX@punct
  \CCTpuncttrue
\else
  \CCTpunctfalse
\fi
%</cct>
%<*cjk>
\RequirePackage{CJK}[2003/03/28]
\RequirePackage{CJKnumb}[2003/03/28]
\ifCTEX@punct
  \RequirePackage{CJKpunct}[2004/02/13]
\fi
%</cjk>
%
% 以及一些其他工具宏包:
%<*ctex|cls>
\RequirePackage{keyval}[1999/03/16]
\ifCTEX@indent
  \RequirePackage{indentfirst}
\fi
\ifCTEX@psfont
  \RequirePackage[T1]{fontenc}
  \RequirePackage{textcomp}
  \RequirePackage{mathptmx}
  \RequirePackage[scaled]{helvet}
  \RequirePackage{courier}
\else
  \RequirePackage{type1cm}
\fi
%</ctex|cls>
%    \end{macrocode}
%
% fancyhdr~宏包,控制页眉页脚的设置
%    \begin{macrocode}
%<*ctex|cls>
\ifCTEX@fancyhdr
  \RequirePackage{fancyhdr}
\fi
%</ctex|cls>
%    \end{macrocode}
%
% CCTfntef~宏包和~CJKfntef~宏包
% \changes{v0.7b}{2005/12/09}{调整宏包导入位置,解决~fntef~类宏包早于相应中文宏包导入的问题}
% \changes{v0.7d}{2005/12/28}{在~fntef~类宏包后使用~\cs{normalem}~恢复~\cs{em}~宏的缺省定义}
%    \begin{macrocode}
%<*cct>
\ifCTEX@fntef
  \RequirePackage{CCTfntef}
  \normalem
\fi
%</cct>
%<*cjk>
\ifCTEX@fntef
  \RequirePackage{CJKfntef}
  \normalem
\fi
%</cjk>
%    \end{macrocode}
%
% ctexcap~宏包需要先装入~ctex~宏包。
%    \begin{macrocode}
%<*ctexcap>
\ifCTEX@caption
  \PassOptionsToPackage{cap}{ctex}
\fi
\RequirePackage{ctex}
%</ctexcap>
%    \end{macrocode}
%
% 文档类还需要的宏包:
%    \begin{macrocode}
%<*cls>
%</cls>
%    \end{macrocode}
%
%
%
%
% \subsection{辅助定义}
%
%
% 我们定义一些将在后面使用的宏。
%
%    \begin{macrocode}
%<*ctex|cls>
\DeclareRobustCommand\CTeX{C\kern-.05em\TeX{}}
\newcommand*\CTEX@key{\define@key{CTEX}}
\newcommand*\CTEXoptions[1][]{\setkeys{CTEX}{#1}}
\newcommand*\CTEX@subkey[1]{\define@key{CTEX#1}}
\newcommand*\CTEXsetup[2][]{\setkeys{CTEX#2}{#1}}
%</ctex|cls>
%    \end{macrocode}
%
%
%    \begin{macrocode}
%<*ctex|cls>
%<article>\def\CTEX@cls@article{}
%<report>\def\CTEX@cls@report{}
%<book>\def\CTEX@cls@book{}
%<*ctex>
\@ifclassloaded{article}{\def\CTEX@cls@article{}}{}
\@ifclassloaded{report}{\def\CTEX@cls@report{}}{}
\@ifclassloaded{book}{\def\CTEX@cls@book{}}{}
%</ctex>
\def\ifCTEX@cls#1{%
  \expandafter\ifx\csname CTEX@cls@#1\endcsname\relax
    \expandafter\@secondoftwo
  \else
    \expandafter\@firstoftwo
  \fi}
%</ctex|cls>
%    \end{macrocode}
%
%
% \changes{v0.7c}{2005/12/20}{增加对~\cs{if@mainmatter}~的判断,以兼容~amsbook~宏包}
%    \begin{macrocode}
%<*ctexcap>
\@ifundefined{if@mainmatter}{\let\if@mainmatter\iftrue}{}
%</ctexcap>
%    \end{macrocode}
%
%
%    \begin{macrocode}
%<*ctex|cls>
\def\CTEX@replacecommand#1#2#3{%
  \expandafter\expandafter\expandafter\let\expandafter
    \csname #1#3\expandafter\endcsname
    \csname #2#3\endcsname
  \expandafter\expandafter\expandafter\def\expandafter
    \csname #2#3\expandafter\endcsname
    {\csname #1#3\endcsname}}
%</ctex|cls>
%    \end{macrocode}
%
%
%
%
% \subsection{通用中文设置}
%
%
% 和~CCT~以及~CJK~相关的内容分放在各自的~\texttt{.clo}~文件中
%    \begin{macrocode}
%<*ctex|cls>
\ifCTEX@cct
  \input{ctexcct.clo}
\else
  \input{ctexcjk.clo}
\fi
%</ctex|cls>
%    \end{macrocode}
%
%
% \changes{v0.7}{2005/11/25}{将~ctex.def~文件和~ctexcap.cfg~文件的读取时间前移,使得可以在导言中使用设置命令}
% 所有包含中文字符的定义都需要从~\texttt{ctex.def}~读入。
%    \begin{macrocode}
%<*ctex|cls>
\AtEndOfPackage{%
  \makeatletter
  \InputIfFileExists{ctex.def}{}{%
    \PackageError{ctex}{%
      Can't find ctex.def}{%
      There will be unexpected errors if you continue.}}
  \makeatother}
%</ctex|cls>
%    \end{macrocode}
%
%
% 如果使用中文标题,则还要读入~\texttt{ctexcap.cfg}~的内容。
%    \begin{macrocode}
%<*ctexcap|cls>
\ifCTEX@caption
  \AtEndOfPackage{%
    \makeatletter
    \InputIfFileExists{ctexcap.cfg}{}{%
      \PackageError{ctex}{%
        Can't find ctexcap.cfg}{%
        The english captions are used if you continue.}}
    \makeatother}
\fi
%</ctexcap|cls>
%    \end{macrocode}
%
%
% \changes{v0.7a}{2005/11/28}{将~ctex.cfg~文件的读取时间前移,使得导言中的设置命令优先}
% 最后再从~\texttt{ctex.cfg}~读入用户的自定义设置。
%    \begin{macrocode}
%<*ctex|cls>
\AtEndOfPackage{%
  \makeatletter
  \InputIfFileExists{ctex.cfg}{}{%
    \PackageWarning{ctex}{%
      Can't find ctex.cfg}}
  \makeatother}
%</ctex|cls>
%    \end{macrocode}
%
%
% \begin{macro}{\CTEXindent}
% \begin{macro}{\CTEXnoindent}
% 段首缩进~2~个汉字的距离,需要考虑到字距。
%    \begin{macrocode}
%<*ctex|cls>
\newcommand*\CTEXindent{\CTEXsetfont\parindent2\ccwd}
\newcommand*\CTEXnoindent{\parindent\z@}
\ifCTEX@indent
  \AtBeginDocument{\CTEXindent}
\fi
%</ctex|cls>
%    \end{macrocode}
% \end{macro}
% \end{macro}
%
%
% \begin{macro}{\CTEX@spaceChar}
% \changes{v0.2a}{2004/02/11}{加快处理速度,改善和~CJKpunct~的兼容性}
%    \begin{macrocode}
%<ctex|cls>\def\CTEX@spaceChar{\hskip \f@size \p@}
%    \end{macrocode}
% \end{macro}
%
%
% \begin{macro}{\baselinestretch}
% \changes{v0.2a}{2004/02/11}{增加对行距的设置}
% \changes{v0.4b}{2004/07/13}{把行距参数从~1.2~改为~1.3}
% 通常中文文档需要较宽的行距。
%    \begin{macrocode}
%<ctex|cls>\def\baselinestretch{1.3}
%    \end{macrocode}
% \end{macro}
%
%
% \subsubsection{中文字号}
%
% 首先给出中文字号和~\TeX{}~字体大小的对应关系。
% 定义中使用~pt~为单位。中文字号大小参考~Word~的定义。
%
% 1 in = 72 bp = 72.27 pt
%
% 行距 = 字体大小 $\times$ 1.2
%
% \changes{v0.2b}{2004/02/13}{修改缺省的行距}
% \changes{v0.4}{2004/05/13}{中文字号定义改为直接使用~pt~为单位}
%    \begin{macrocode}
%<*ctex|cls>
\def\CTEX@fs@eight{5.02}   \def\CTEX@fs@eightskip{6.02}   %八号字    5bp
\def\CTEX@fs@seven{5.52}   \def\CTEX@fs@sevenskip{6.62}   %七号字  5.5bp
\def\CTEX@fs@ssix{6.52}    \def\CTEX@fs@ssixskip{7.83}    %小六号  6.5bp
\def\CTEX@fs@six{7.53}     \def\CTEX@fs@sixskip{9.03}     %六号字  7.5bp
\def\CTEX@fs@sfive{9.03}   \def\CTEX@fs@sfiveskip{10.84}  %小五号    9bp
\def\CTEX@fs@five{10.54}   \def\CTEX@fs@fiveskip{12.65}   %五号字 10.5bp
\def\CTEX@fs@sfour{12.05}  \def\CTEX@fs@sfourskip{14.45}  %小四号   12bp
\def\CTEX@fs@four{14.05}   \def\CTEX@fs@fourskip{16.86}   %四号字   14bp
\def\CTEX@fs@sthree{15.06} \def\CTEX@fs@sthreeskip{18.07} %小三号   15bp
\def\CTEX@fs@three{16.06}  \def\CTEX@fs@threeskip{19.27}  %三号字   16bp
\def\CTEX@fs@stwo{18.07}   \def\CTEX@fs@stwoskip{21.68}   %小二号   18bp
\def\CTEX@fs@two{22.08}    \def\CTEX@fs@twoskip{26.50}    %二号字   22bp
\def\CTEX@fs@sone{24.09}   \def\CTEX@fs@soneskip{28.91}   %小一号   24bp
\def\CTEX@fs@one{26.10}    \def\CTEX@fs@oneskip{31.32}    %一号字   26bp
\def\CTEX@fs@szero{36.14}  \def\CTEX@fs@szeroskip{43.36}  %小初号   36bp
\def\CTEX@fs@zero{42.16}   \def\CTEX@fs@zeroskip{50.59}   %初号字   42bp
%</ctex|cls>
%    \end{macrocode}
%
% 定义相应的数学字体的大小(用于上下脚标)。
%    \begin{macrocode}
%<*ctex|cls>
\DeclareMathSizes{\CTEX@fs@eight}{\CTEX@fs@eight}{5}{5}
\DeclareMathSizes{\CTEX@fs@seven}{\CTEX@fs@seven}{5}{5}
\DeclareMathSizes{\CTEX@fs@ssix}{\CTEX@fs@ssix}{5}{5}
\DeclareMathSizes{\CTEX@fs@six}{\CTEX@fs@six}{5}{5}
\DeclareMathSizes{\CTEX@fs@sfive}{\CTEX@fs@sfive}{6}{5}
\DeclareMathSizes{\CTEX@fs@five}{\CTEX@fs@five}{7}{5}
\DeclareMathSizes{\CTEX@fs@sfour}{\CTEX@fs@sfour}{8}{6}
\DeclareMathSizes{\CTEX@fs@four}
                 {\CTEX@fs@four}{\CTEX@fs@five}{\CTEX@fs@six}
\DeclareMathSizes{\CTEX@fs@sthree}
                 {\CTEX@fs@sthree}{\CTEX@fs@sfour}{\CTEX@fs@sfive}
\DeclareMathSizes{\CTEX@fs@three}
                 {\CTEX@fs@three}{\CTEX@fs@four}{\CTEX@fs@five}
\DeclareMathSizes{\CTEX@fs@stwo}
                 {\CTEX@fs@stwo}{\CTEX@fs@sthree}{\CTEX@fs@sfour}
\DeclareMathSizes{\CTEX@fs@two}
                 {\CTEX@fs@two}{\CTEX@fs@three}{\CTEX@fs@four}
\DeclareMathSizes{\CTEX@fs@sone}
                 {\CTEX@fs@sone}{\CTEX@fs@stwo}{\CTEX@fs@sthree}
\DeclareMathSizes{\CTEX@fs@one}
                 {\CTEX@fs@one}{\CTEX@fs@two}{\CTEX@fs@three}
\DeclareMathSizes{\CTEX@fs@szero}
                 {\CTEX@fs@szero}{\CTEX@fs@sone}{\CTEX@fs@stwo}
\DeclareMathSizes{\CTEX@fs@zero}
                 {\CTEX@fs@zero}{\CTEX@fs@one}{\CTEX@fs@two}
%</ctex|cls>
%    \end{macrocode}
%
%
% \begin{macro}{\zihao}
% \changes{v0.1b}{2003/08/17}{删除多余的~\cs{newcount}~命令}
% \changes{v0.4}{2004/05/13}{删除~\cs{CTEX@fontsize}~命令,
%                            改为直接使用~\cs{fontsize}~命令}
% 这个命令用于改变中文字号。当前中文字号保存在~|\CTEX@zihao|~中。
% 注意,如果没用过~|\zihao|~命令,则~|\CTEX@zihao|~没有定义。
% \begin{macro}{\CTEX@zihao}
%    \begin{macrocode}
%<*ctex|cls>
\def\CTEX@zihao{}
%    \end{macrocode}
% \end{macro}
% 如果是负数,表示是小号字体
%    \begin{macrocode}
\DeclareRobustCommand*\zihao[1]{\def\CTEX@zihao{#1}%
  \ifnum #11<0%
    \@tempcnta=-#1
    \ifcase\@tempcnta%
        \fontsize\CTEX@fs@szero\CTEX@fs@szeroskip%
    \or \fontsize\CTEX@fs@sone\CTEX@fs@soneskip%
    \or \fontsize\CTEX@fs@stwo\CTEX@fs@stwoskip%
    \or \fontsize\CTEX@fs@sthree\CTEX@fs@sthreeskip%
    \or \fontsize\CTEX@fs@sfour\CTEX@fs@sfourskip%
    \or \fontsize\CTEX@fs@sfive\CTEX@fs@sfiveskip%
    \or \fontsize\CTEX@fs@ssix\CTEX@fs@ssixskip%
%    \end{macrocode}
% 如果不在预定义的字号范围~(-0 -- -6)~,则报告一个错误
%    \begin{macrocode}
    \else \PackageError{ctex}{%
            Undefined Chinese font size in command \protect\zihao}{%
            The old font size is used if you continue.}%
    \fi%
%    \end{macrocode}
% 否则是正常字号
%    \begin{macrocode}
  \else%
    \@tempcnta=#1
    \ifcase\@tempcnta%
        \fontsize\CTEX@fs@zero\CTEX@fs@zeroskip%
    \or \fontsize\CTEX@fs@one\CTEX@fs@oneskip%
    \or \fontsize\CTEX@fs@two\CTEX@fs@twoskip%
    \or \fontsize\CTEX@fs@three\CTEX@fs@threeskip%
    \or \fontsize\CTEX@fs@four\CTEX@fs@fourskip%
    \or \fontsize\CTEX@fs@five\CTEX@fs@fiveskip%
    \or \fontsize\CTEX@fs@six\CTEX@fs@sixskip%
    \or \fontsize\CTEX@fs@seven\CTEX@fs@sevenskip%
    \or \fontsize\CTEX@fs@eight\CTEX@fs@eightskip%
%    \end{macrocode}
% 如果不在预定义的字号范围~(0 -- 8)~,则报告一个错误
%    \begin{macrocode}
    \else \PackageError{ctex}{%
            Undefined Chinese font size in command \protect\zihao}{%
            The old font size is used if you continue.}%
    \fi%
  \fi%
  \selectfont\ignorespaces}
%</ctex|cls>
%    \end{macrocode}
% \end{macro}
%
%
% \subsubsection{缺省字号大小}
%
% \changes{v0.2b}{2004/02/13}{修改缺省的字号大小}
% \changes{v0.2d}{2004/04/23}{补上字号定义中行间距参数中缺少的~\cs{CTEX@bp}}
% 缺省字体只对文档类有效,下面使用中文小四号字时的设置。
%
%    \begin{macrocode}
%<*cls>
\ifCTEX@sfoursize
  \renewcommand\normalsize{% 12bp
    \@setfontsize\normalsize{\CTEX@fs@sfour}{\CTEX@fs@sfourskip}%
    \abovedisplayskip 12\p@ \@plus3\p@ \@minus7\p@
    \abovedisplayshortskip \z@ \@plus3\p@
    \belowdisplayshortskip 6.5\p@ \@plus3.5\p@ \@minus3\p@
    \belowdisplayskip \abovedisplayskip
    \let\@listi\@listI}
  \normalsize
  \renewcommand\small{% 10.5bp
    \@setfontsize\small{\CTEX@fs@five}{\CTEX@fs@fiveskip}%
    \abovedisplayskip 11\p@ \@plus3\p@ \@minus6\p@
    \abovedisplayshortskip \z@ \@plus3\p@
    \belowdisplayshortskip 6.5\p@ \@plus3.5\p@ \@minus3\p@
    \def\@listi{\leftmargin\leftmargini
                \topsep 9\p@ \@plus3\p@ \@minus5\p@
                \parsep 4.5\p@ \@plus2\p@ \@minus\p@
                \itemsep \parsep}%
    \belowdisplayskip \abovedisplayskip}
  \renewcommand\footnotesize{% 9bp
    \@setfontsize\footnotesize{\CTEX@fs@sfive}{\CTEX@fs@sfiveskip}%
    \abovedisplayskip 10\p@ \@plus2\p@ \@minus5\p@
    \abovedisplayshortskip \z@ \@plus3\p@
    \belowdisplayshortskip 6\p@ \@plus3\p@ \@minus3\p@
    \def\@listi{\leftmargin\leftmargini
                \topsep 6\p@ \@plus2\p@ \@minus2\p@
                \parsep 3\p@ \@plus2\p@ \@minus\p@
                \itemsep \parsep}%
    \belowdisplayskip \abovedisplayskip}
  \renewcommand\scriptsize{% 7.5bp
    \@setfontsize\scriptsize{\CTEX@fs@six}{\CTEX@fs@sixskip}}
  \renewcommand\tiny{% 6.5bp
    \@setfontsize\tiny{\CTEX@fs@ssix}{\CTEX@fs@ssixskip}}
  \renewcommand\large{% 15bp
    \@setfontsize\large{\CTEX@fs@sthree}{\CTEX@fs@sthreeskip}}
  \renewcommand\Large{% 18bp
    \@setfontsize\Large{\CTEX@fs@stwo}{\CTEX@fs@stwoskip}}
  \renewcommand\LARGE{% 22bp
    \@setfontsize\LARGE{\CTEX@fs@two}{\CTEX@fs@twoskip}}
  \renewcommand\huge{% 24bp
    \@setfontsize\huge{\CTEX@fs@sone}{\CTEX@fs@soneskip}}
  \renewcommand\Huge{% 26bp
    \@setfontsize\Huge{\CTEX@fs@one}{\CTEX@fs@oneskip}}
\fi
%</cls>
%    \end{macrocode}
%
% \changes{v0.2d}{2004/04/23}{修改缺省的字号大小}
% 缺省字体使用中文五号字时的设置。
%
%    \begin{macrocode}
%<*cls>
\ifCTEX@fivesize
  \renewcommand\normalsize{% 10.5bp
    \@setfontsize\normalsize{\CTEX@fs@five}{\CTEX@fs@fiveskip}%
    \abovedisplayskip 10\p@ \@plus2\p@ \@minus5\p@
    \abovedisplayshortskip \z@ \@plus3\p@
    \belowdisplayshortskip 6\p@ \@plus3\p@ \@minus3\p@
    \belowdisplayskip \abovedisplayskip
    \let\@listi\@listI}
  \normalsize
  \renewcommand\small{% 9bp
    \@setfontsize\small{\CTEX@fs@sfive}{\CTEX@fs@sfiveskip}%
    \abovedisplayskip 8.5\p@ \@plus3\p@ \@minus4\p@
    \abovedisplayshortskip \z@ \@plus2\p@
    \belowdisplayshortskip 4\p@ \@plus2\p@ \@minus2\p@
    \def\@listi{\leftmargin\leftmargini
                \topsep 4\p@ \@plus2\p@ \@minus2\p@
                \parsep 2\p@ \@plus\p@ \@minus\p@
                \itemsep \parsep}%
    \belowdisplayskip \abovedisplayskip}
  \renewcommand\footnotesize{% 7.5bp
    \@setfontsize\footnotesize{\CTEX@fs@six}{\CTEX@fs@sixskip}%
    \abovedisplayskip 6\p@ \@plus2\p@ \@minus4\p@
    \abovedisplayshortskip \z@ \@plus\p@
    \belowdisplayshortskip 3\p@ \@plus\p@ \@minus2\p@
    \def\@listi{\leftmargin\leftmargini
                \topsep 3\p@ \@plus\p@ \@minus\p@
                \parsep 2\p@ \@plus\p@ \@minus\p@
                \itemsep \parsep}%
    \belowdisplayskip \abovedisplayskip}
  \renewcommand\scriptsize{% 6.5bp
    \@setfontsize\scriptsize{\CTEX@fs@ssix}{\CTEX@fs@ssixskip}}
  \renewcommand\tiny{% 5.5bp
    \@setfontsize\tiny{\CTEX@fs@seven}{\CTEX@fs@sevenskip}}
  \renewcommand\large{% 12bp
    \@setfontsize\large{\CTEX@fs@sfour}{\CTEX@fs@sfourskip}}
  \renewcommand\Large{% 15bp
    \@setfontsize\Large{\CTEX@fs@sthree}{\CTEX@fs@sthreeskip}}
  \renewcommand\LARGE{% 18bp
    \@setfontsize\LARGE{\CTEX@fs@stwo}{\CTEX@fs@stwoskip}}
  \renewcommand\huge{% 22bp
    \@setfontsize\huge{\CTEX@fs@two}{\CTEX@fs@twoskip}}
  \renewcommand\Huge{% 26bp
    \@setfontsize\Huge{\CTEX@fs@one}{\CTEX@fs@oneskip}}
\fi
%</cls>
%    \end{macrocode}
%
%
%
%
% \subsection{~CCT~相关配置}
%
%
%    \begin{macrocode}
%<*cct>
%    \end{macrocode}
%
%
% \subsubsection{CCT~环境设置}
%
%    \begin{macrocode}
\ifCTEX@space
  \AtBeginDocument{\CCTspace}
\else
  \AtBeginDocument{\CCTnospace}
\fi
%    \end{macrocode}
%
% \begin{macro}{\CTEXspace}
% \begin{macro}{\CTEXnospace}
%    \begin{macrocode}
\def\CTEXspace{\CCTspace}
\def\CTEXnospace{\CCTnospace}
%    \end{macrocode}
% \end{macro}
% \end{macro}
%
%
% \subsubsection{常用的中文字体}
%
% \begin{macro}{\songti}
% \begin{macro}{\heiti}
% \begin{macro}{\fangsong}
% \begin{macro}{\kaishu}
% \begin{macro}{\lishu}
% \begin{macro}{\youyuan}
% 定义常用的中文字体命令:宋体、黑体、楷书、仿宋、隶书、幼圆。
%    \begin{macrocode}
\ifCTEX@cctfont\else
  \CCTdefziti A song song    % 宋体
  \CCTdefziti B hei  song    % 黑体
  \CCTdefziti C kai  song    % 楷体
  \CCTdefziti D fs   song    % 仿宋
  \CCTdefziti E bs   hei     % 标宋
  \CCTdefziti F li   hei     % 隶书
  \CCTdefziti G you  kai     % 幼圆
  \DeclareRobustCommand*{\songti}{\ziti{A}\relax}   % 宋体
  \DeclareRobustCommand*{\heiti}{\ziti{B}\relax}    % 黑体
  \DeclareRobustCommand*{\kaishu}{\ziti{C}\relax}   % 楷书
  \DeclareRobustCommand*{\fangsong}{\ziti{D}\relax} % 仿宋
  \DeclareRobustCommand*{\biaosong}{\ziti{E}\relax} % 标宋
  \DeclareRobustCommand*{\lishu}{\ziti{F}\relax}    % 隶书
  \DeclareRobustCommand*{\youyuan}{\ziti{G}\relax}  % 幼圆
\fi
%    \end{macrocode}
% \end{macro}
% \end{macro}
% \end{macro}
% \end{macro}
% \end{macro}
% \end{macro}
%
%
% \subsubsection{中文字号}
%
% \begin{macro}{\set@fontsize}
% \changes{v0.6}{2005/09/24}{cct~从~0.6180~开始将宏~\cs{oset@fontsize}~改为~\cs{CCT@set@fontsize}}
% \changes{v0.6a}{2005/09/30}{增加对~\cs{CCT@set@fontsize}~的判断}
% CCT~对~|\set@fontsize|~命令进行了重定义,我们需要修改这个定义,
% 让中英文保持一致的大小,并及时更新字体信息。
%    \begin{macrocode}
\ifx\CCT@set@fontsize\undefined
  \let\CCT@set@fontsize\oset@fontsize
\fi
\let\CTEX@save@set@fontsize\set@fontsize
\def\set@fontsize#1#2#3{%
  \CCT@set@fontsize{#1}{#2}{#3}%
  \zihaoAny{#2}%
  \CTEXsetfont}
%    \end{macrocode}
% \end{macro}
%
%
% \subsubsection{其他字体命令}
%
% \begin{macro}{\ziju}
% \changes{v0.4a}{2004/05/15}{修改~CCT~的字距命令使得缩进保持一致}
% 定义调整汉字字距的命令。
%    \begin{macrocode}
\let\CTEX@save@ziju\ziju
\renewcommand*\ziju[1]{% 字距
  \CTEX@save@ziju{#1}%
  \CTEXsetfont}
%    \end{macrocode}
% \end{macro}
%
%
% \begin{macro}{\CTEXsetfont}
% |\CTEXsetfont|~获得当前的汉字信息。
%    \begin{macrocode}
\newcommand*\CTEXsetfont{%
  \ifdim\parindent=0pt\else\parindent2\ccwd\fi}
%    \end{macrocode}
% \end{macro}
%
%
% \subsubsection{CCTfntef~宏包接口}
%
%
% 为~\texttt{CCTfntef.sty}~宏包的命令提供统一接口。
%    \begin{macrocode}
\ifCTEX@fntef
  \def\CTEXunderdot{\CCTunderdot}
  \def\CTEXunderline{\CCTunderline}
  \def\CTEXunderdblline{\CCTunderdblline}
  \def\CTEXunderwave{\CCTunderwave}
  \def\CTEXsout{\CCTsout}
  \def\CTEXxout{\CCTxout}
  \def\CTEXfilltwosides{\CCTfilltwosides}
  \def\endCTEXfilltwosides{\endCCTfilltwosides}
  \CTEX@replacecommand{CTEX}{CCT}{underdotbasesep}
  \CTEX@replacecommand{CTEX}{CCT}{underdotsep}
  \CTEX@replacecommand{CTEX}{CCT}{underlinebasesep}
  \CTEX@replacecommand{CTEX}{CCT}{underlinesep}
  \CTEX@replacecommand{CTEX}{CCT}{underdbllinebasesep}
  \CTEX@replacecommand{CTEX}{CCT}{underdbllinesep}
  \CTEX@replacecommand{CTEX}{CCT}{underwavebasesep}
  \CTEX@replacecommand{CTEX}{CCT}{underwavesepa}
  \CTEX@replacecommand{CTEX}{CCT}{underwavesep}
  \CTEX@replacecommand{CTEX}{CCT}{southeight}
  \CTEX@replacecommand{CTEX}{CCT}{xoutheight}
  \CTEX@replacecommand{CTEX}{CCT}{underdotcolor}
  \CTEX@replacecommand{CTEX}{CCT}{underwavecolor}
  \CTEX@replacecommand{CTEX}{CCT}{underlinecolor}
  \CTEX@replacecommand{CTEX}{CCT}{underdbllinecolor}
  \CTEX@replacecommand{CTEX}{CCT}{soutcolor}
  \CTEX@replacecommand{CTEX}{CCT}{xoutcolor}
\fi
%    \end{macrocode}
%
%
%    \begin{macrocode}
%</cct>
%    \end{macrocode}
%
%
%
%
% \subsection{~CJK~相关配置}
%
%
%    \begin{macrocode}
%<*cjk>
%    \end{macrocode}
%
%
% \subsubsection{CJK~环境设置}
%
%
% \changes{v0.7}{2005/11/25}{支持在导言中使用中文}
%    \begin{macrocode}
\AtEndOfPackage{\CJK@makeActive}
%    \end{macrocode}
%
%
% \changes{v0.5b}{2004/09/29}{改变设置~CJK~环境结束语句的
%                 ~\cs{AtEndDocument}~执行的位置,以减少宏包冲突}
% \changes{v0.5c}{2004/09/29}{避免重复执行设置~CJK~环境结束语句}
% \changes{v0.7f}{2006/04/12}{采用修改~\cs{AtBeginDocument}~和
%                 ~\cs{AtEndDocument}~命令的方式来设置~CJK~环境,
%                 以减少宏包冲突}
% 缺省情况下,我们把整个文档都包含到~CJK~环境中。如果~|\ifCTEX@space|~
% 设置为~true,我们使用~CJK~环境,否则使用~CJK*~环境。
%    \begin{macrocode}
\ifCTEX@space
  \def\CTEX@beginCJK{\begin{CJK}{GBK}{rm}}
  \def\CTEX@endCJK{\clearpage\end{CJK}}
\else
  \def\CTEX@beginCJK{\begin{CJK*}{GBK}{rm}\CJKtilde}
  \def\CTEX@endCJK{\clearpage\end{CJK*}}
\fi
\let\CTEX@begindocumenthook\@begindocumenthook
\let\CTEX@enddocumenthook\@enddocumenthook
\def\AtBeginDocument{\g@addto@macro\CTEX@begindocumenthook}
\def\AtEndDocument{\g@addto@macro\CTEX@enddocumenthook}
\def\@begindocumenthook{\CTEX@begindocumenthook\CTEX@beginCJK}
\def\@enddocumenthook{\CTEX@endCJK\CTEX@enddocumenthook}
%    \end{macrocode}
%
%
% \begin{macro}{\CTEXspace}
% \begin{macro}{\CTEXnospace}
%    \begin{macrocode}
\def\CTEXspace{\CJKspace}
\def\CTEXnospace{\CJKnospace}
%    \end{macrocode}
% \end{macro}
% \end{macro}
%
%
% \subsubsection{常用的中文字体}
%
%
% \begin{macro}{\songti}
% \begin{macro}{\heiti}
% \begin{macro}{\fangsong}
% \begin{macro}{\kaishu}
% \begin{macro}{\lishu}
% \begin{macro}{\youyuan}
% 定义常用的中文字体命令:宋体、黑体、楷书、仿宋、隶书、幼圆。
%    \begin{macrocode}
\newcommand*{\songti}{\CJKfamily{song}} % 宋体
\newcommand*{\heiti}{\CJKfamily{hei}}   % 黑体
\newcommand*{\kaishu}{\CJKfamily{kai}}  % 楷书
\newcommand*{\fangsong}{\CJKfamily{fs}} % 仿宋
\newcommand*{\lishu}{\CJKfamily{li}}    % 隶书
\newcommand*{\youyuan}{\CJKfamily{you}} % 幼圆
%    \end{macrocode}
% \end{macro}
% \end{macro}
% \end{macro}
% \end{macro}
% \end{macro}
% \end{macro}
%
%
% \subsubsection{中文字号}
%
%
% \begin{macro}{\CTEX@save@set@fontsize}
% \begin{macro}{\set@fontsize}
% 对~|\set@fontsize|~命令需要进行重定义,以便及时更新字体信息。
%    \begin{macrocode}
\let\CTEX@save@set@fontsize\set@fontsize
\def\set@fontsize#1#2#3{%
  \CTEX@save@set@fontsize{#1}{#2}{#3}%
  \CTEXsetfont}
%    \end{macrocode}
% \end{macro}
% \end{macro}
%
%
% \subsubsection{其他字体命令}
%
%
% \begin{macro}{\ziju}
% \changes{v0.2}{2004/02/11}{参数的单位由绝对距离改为相对于当前汉字大小的倍数}
% 定义调整汉字字距的命令。
%    \begin{macrocode}
\newcommand*\ziju[1]{% 字距
  \settowidth\@tempdima{\CTEX@spaceChar}%
  \renewcommand{\CJKglue}{\hskip #1\@tempdima}%
  \CTEXsetfont}
%    \end{macrocode}
% \end{macro}
%
%
% \begin{macro}{\ccwd}
% |\ccwd|~是当前的汉字字宽。
%    \begin{macrocode}
\newdimen\ccwd % 字宽
%    \end{macrocode}
% \end{macro}
%
%
% \begin{macro}{\CTEXsetfont}
% \changes{v0.2}{2004/02/11}{\cs{CTEXfontinfo}~命令改为~\cs{CTEXsetfont}}
% |\CTEXsetfont|~获得当前的汉字信息。
%    \begin{macrocode}
\newcommand*\CTEXsetfont{%
  \settowidth\ccwd{\CTEX@spaceChar\CJKglue}%
  \ifdim\parindent=0pt\else\parindent2\ccwd\fi}
%    \end{macrocode}
% \end{macro}
%
%
% \subsubsection{CJKfntef~宏包接口}
%
%
% 为~\texttt{CJKfntef.sty}~宏包的命令提供统一接口。
%    \begin{macrocode}
\ifCTEX@fntef
  \def\CTEXunderdot{\CJKunderdot}
  \def\CTEXunderline{\CJKunderline}
  \def\CTEXunderdblline{\CJKunderdblline}
  \def\CTEXunderwave{\CJKunderwave}
  \def\CTEXsout{\CJKsout}
  \def\CTEXxout{\CJKxout}
  \def\CTEXfilltwosides{\CJKfilltwosides}
  \def\endCTEXfilltwosides{\endCJKfilltwosides}
  \CTEX@replacecommand{CTEX}{CJK}{underdotbasesep}
  \CTEX@replacecommand{CTEX}{CJK}{underdotsep}
  \CTEX@replacecommand{CTEX}{CJK}{underlinebasesep}
  \CTEX@replacecommand{CTEX}{CJK}{underlinesep}
  \CTEX@replacecommand{CTEX}{CJK}{underdbllinebasesep}
  \CTEX@replacecommand{CTEX}{CJK}{underdbllinesep}
  \CTEX@replacecommand{CTEX}{CJK}{underwavebasesep}
  \CTEX@replacecommand{CTEX}{CJK}{underwavesep}
  \CTEX@replacecommand{CTEX}{CJK}{southeight}
  \CTEX@replacecommand{CTEX}{CJK}{underdotcolor}
  \CTEX@replacecommand{CTEX}{CJK}{underwavecolor}
  \CTEX@replacecommand{CTEX}{CJK}{underlinecolor}
  \CTEX@replacecommand{CTEX}{CJK}{underdbllinecolor}
  \CTEX@replacecommand{CTEX}{CJK}{soutcolor}
  \CTEX@replacecommand{CTEX}{CJK}{xoutcolor}
\fi
%    \end{macrocode}
%
%
%    \begin{macrocode}
%</cjk>
%    \end{macrocode}
%
%
%
%
% \subsection{中文数字处理}
%
% 由于脆弱命令的原因,CJK~提供的~|\CJKnumber|~在章节编号中直接使用
% 会引发不少问题。在生成目录和书签等辅助文件时,我们需要未经~CJK~
% 处理过的中文字符串,而不能是~|\CJKchar{...}|~的形式。此外,这些
% 中文数字必须在被使用之前就已经生成好。
% 于是我们修改了~\texttt{CJKnumb}~宏包中的一些定义来处理章节编号。
%
%
% 首先是一些基本数字,需要在~\texttt{ctex.def}~文件中重新定义。
%    \begin{macrocode}
%<*def>
\def\CTEXnullspace{0pt}
\def\CTEX@null{\kern\CTEXnullspace○\kern\CTEXnullspace}
\def\CTEX@zero{零}
\def\CTEX@one{一}
\def\CTEX@two{二}
\def\CTEX@three{三}
\def\CTEX@four{四}
\def\CTEX@five{五}
\def\CTEX@six{六}
\def\CTEX@seven{七}
\def\CTEX@eight{八}
\def\CTEX@nine{九}
\def\CTEX@ten{十}
\def\CTEX@hundred{百}
\def\CTEX@thousand{千}
\def\CTEX@tenthousand{万}
\def\CTEX@hundredmillion{亿}
\def\CTEX@minus{-}

%</def>
%    \end{macrocode}
%
%
%    \begin{macrocode}
%<*ctex|cls>
%    \end{macrocode}
%
%
% 一些条件定义和计数器
%    \begin{macrocode}
\newif\ifCTEX@zero@
\newif\ifCTEX@previous@
\newif\ifCTEX@null@
\newcount\CTEX@q
\newcount\CTEX@r
%    \end{macrocode}
%
%
% \begin{macro}{\CTEX@appendstring}
% 中文数字处理命令~|\CTEX@appendstring|~用于拼接字符串。
%    \begin{macrocode}
\def\CTEX@appendstring#1#2{%
  \expandafter\def\expandafter#1\expandafter{#1#2}}
%    \end{macrocode}
% \end{macro}
%
%
% \begin{macro}{\CTEX@appendnumber}
% 中文数字处理命令~|\CTEX@appendnumber|~用于拼接数字。
%    \begin{macrocode}
\def\CTEX@appendnumber#1#2{%
  \ifcase #2\relax
    \ifCTEX@null@
      \CTEX@appendstring{#1}{\CTEX@null}%
    \else
      \CTEX@appendstring{#1}{\CTEX@zero}%
    \fi
  \or \CTEX@appendstring{#1}{\CTEX@one}%
  \or \CTEX@appendstring{#1}{\CTEX@two}%
  \or \CTEX@appendstring{#1}{\CTEX@three}%
  \or \CTEX@appendstring{#1}{\CTEX@four}%
  \or \CTEX@appendstring{#1}{\CTEX@five}%
  \or \CTEX@appendstring{#1}{\CTEX@six}%
  \or \CTEX@appendstring{#1}{\CTEX@seven}%
  \or \CTEX@appendstring{#1}{\CTEX@eight}%
  \or \CTEX@appendstring{#1}{\CTEX@nine}%
  \fi}
%    \end{macrocode}
% \end{macro}
%
%
% \begin{macro}{\CTEX@splitnumber}
% 中文数字处理命令~|\CTEX@splitnumber|~将大的数字分为几段四位以内的数字。
%    \begin{macrocode}
\def\CTEX@splitnumber#1{%
  \CTEX@q #1\relax
  \CTEX@r #1\relax
%
  \divide\CTEX@q \@M
  \begingroup
    \multiply\CTEX@q \@M
    \advance\CTEX@r -\CTEX@q
    \ifnum\CTEX@r = \z@
      \xdef\CTEX@low{}%
    \else
      \xdef\CTEX@low{\number\CTEX@r}%
    \fi
  \endgroup
%
  \ifnum\CTEX@q > \z@
    \CTEX@r \CTEX@q
%
    \divide\CTEX@q \@M
    \begingroup
      \multiply\CTEX@q \@M
      \advance\CTEX@r -\CTEX@q
      \ifnum\CTEX@r = \z@
        \xdef\CTEX@high{}%
      \else
        \xdef\CTEX@high{\number\CTEX@r}%
      \fi
    \endgroup
%
    \ifnum\CTEX@q > \z@
      \xdef\CTEX@yi{\number\CTEX@q}%
    \else
      \xdef\CTEX@yi{}%
    \fi
  \else
    \xdef\CTEX@high{}%
    \xdef\CTEX@yi{}%
  \fi
}
%    \end{macrocode}
% \end{macro}
%
%
% \begin{macro}{\CTEX@processnumber}
% 中文数字处理命令~|\CTEX@processnumber|~处理四位以内的数字,并将得到的
% 中文数字存放在第一个参数中。
%    \begin{macrocode}
\def\CTEX@processnumber#1#2{%
  \CTEX@zero@false
%
  \CTEX@q #2\relax
  \CTEX@r #2\relax
%
  \divide\CTEX@q \@m
  \ifnum\CTEX@q = \z@
    \ifCTEX@previous@
      \CTEX@zero@true
    \fi
  \else
    \ifCTEX@zero@
      \CTEX@appendstring{#1}{\CTEX@zero}%
    \fi
    \CTEX@appendnumber{#1}{\CTEX@q}%
    \CTEX@appendstring{#1}{\CTEX@thousand}%
    \CTEX@previous@true
    \CTEX@zero@false
  \fi
%
  \multiply\CTEX@q \@m
  \advance\CTEX@r -\CTEX@q
  \CTEX@q \CTEX@r
%
  \divide\CTEX@q 100\relax
  \ifnum\CTEX@q = \z@
    \ifCTEX@previous@
      \CTEX@zero@true
    \fi
  \else
    \ifCTEX@zero@
      \CTEX@appendstring{#1}{\CTEX@zero}%
    \fi
    \CTEX@appendnumber{#1}{\CTEX@q}%
    \CTEX@appendstring{#1}{\CTEX@hundred}%
    \CTEX@previous@true
    \CTEX@zero@false
  \fi
%
  \multiply\CTEX@q 100
  \advance\CTEX@r -\CTEX@q
  \CTEX@q \CTEX@r
%
  \divide \CTEX@q 10\relax
  \ifnum\CTEX@q = \z@
    \ifCTEX@previous@
      \CTEX@zero@true
    \fi
  \else
    \ifCTEX@zero@
      \CTEX@appendstring{#1}{\CTEX@zero}%
    \fi
    \ifnum\CTEX@q = \@ne
      \ifCTEX@previous@
        \CTEX@appendstring{#1}{\CTEX@one}%
      \fi
    \else
      \CTEX@appendnumber{#1}{\CTEX@q}%
    \fi
    \CTEX@appendstring{#1}{\CTEX@ten}%
    \CTEX@previous@true
    \CTEX@zero@false
  \fi
%
  \multiply\CTEX@q 10
  \advance\CTEX@r -\CTEX@q
%
  \ifnum\CTEX@r = \z@
  \else
    \ifCTEX@zero@
      \CTEX@appendstring{#1}{\CTEX@zero}%
    \fi
    \CTEX@appendnumber{#1}{\CTEX@r}%
    \CTEX@previous@true
  \fi}
%    \end{macrocode}
% \end{macro}
%
%
% \begin{macro}{\CTEXnumber}
% \changes{v0.7e}{2006/03/22}{使用~\cs{DeclareRobustCommand}~命令来定义~\cs{CTEXnumber}}
% 中文数字处理命令~|\CTEXnumber|~将第二个参数中的数字转换为中文并
% 保存在第一个参数中。
%    \begin{macrocode}
\DeclareRobustCommand\CTEXnumber[2]{%
  \def#1{}%
  \CTEX@null@false
%
  \CTEX@q #2\relax
%
  \ifnum\CTEX@q < \z@
    \multiply\CTEX@q \m@ne
    \CTEX@appendstring{#1}{\CTEX@minus}%
  \fi
%
  \CTEX@previous@false
  \CTEX@zero@false
%
  \ifnum\CTEX@q = \z@
    \CTEX@appendstring{#1}{\CTEX@zero}%
  \else
    \CTEX@splitnumber{\CTEX@q}%
%
    \ifx\CTEX@yi \@empty
    \else
      \CTEX@processnumber{#1}{\CTEX@yi}%
      \CTEX@appendstring{#1}{\CTEX@hundredmillion}%
    \fi
%
    \ifx\CTEX@high \@empty
    \else
      \CTEX@processnumber{#1}{\CTEX@high}%
      \CTEX@appendstring{#1}{\CTEX@tenthousand}%
    \fi
%
    \ifx\CTEX@low \@empty
    \else
      \ifx\CTEX@yi \@empty
      \else
        \ifx\CTEX@high \@empty
          \CTEX@appendstring{#1}{\CTEX@zero}% this catches 100002345
        \fi
      \fi
      \CTEX@processnumber{#1}{\CTEX@low}%
    \fi
  \fi}
%    \end{macrocode}
% \end{macro}
%
%
% \begin{macro}{\CTEX@getdigit}
% \changes{v0.7e}{2006/03/22}{除去多余的空格}
% 中文数字处理命令~|\CTEX@getdigit|~用于提取最高位的数字。
%    \begin{macrocode}
\def\CTEX@getdigit#1#2\@nil{%
  \edef\CTEX@tempa{#1}%
  \edef\CTEX@tempb{#2}}
%    \end{macrocode}
% \end{macro}
%
%
% \begin{macro}{\CTEXdigits}
% \changes{v0.4a}{2004/05/15}{增加~\cs{CTEXdigits}~命令}
% \changes{v0.7e}{2006/03/22}{除去多余的空格}
% 中文数字处理命令~|\CTEXdigits|~将第二个参数中的数字串为中文数字串
% 并保存在第一个参数中。
%    \begin{macrocode}
\DeclareRobustCommand\CTEXdigits[2]{%
  \def#1{}%
  \CTEX@null@true
  \edef\CTEX@tempa{}%
  \edef\CTEX@tempb{#2}%
  \ifx\CTEX@tempb \@empty
  \else
    \loop
      \expandafter\CTEX@getdigit\CTEX@tempb\@nil
      \CTEX@appendnumber{#1}{\CTEX@tempa}%
      \ifx\CTEX@tempb \@empty
      \else
    \repeat
  \fi}
%    \end{macrocode}
% \end{macro}
%
%
% \begin{macro}{\CTEXcounter}
% \changes{v0.7e}{2006/03/22}{使用~\cs{DeclareRobustCommand}~命令来定义~\cs{CTEXcounter}}
% ~|\CTEXcounter|~用于生成对应于计数器~|\FOO|~的中文数字~|\cc@FOO|。
%    \begin{macrocode}
\DeclareRobustCommand\CTEXcounter[1]{%
  \@ifundefined{c@#1}{}{%
    \CTEXnumber{\reserved@a}{\@arabic\csname c@#1\endcsname}%
    \expandafter\expandafter\expandafter\def%
    \expandafter\expandafter\csname cc@#1\endcsname%
    \expandafter{\reserved@a}}}
%    \end{macrocode}
% \end{macro}
%
%
% \begin{macro}{\setcounter}
% \begin{macro}{\addtocounter}
% \begin{macro}{\stepcounter}
% \changes{v0.1d}{2003/09/27}{将对~\cs{setcounter}~和~\cs{addtocounter}~
%                 的修改放到导言的最后以和其他宏包兼容}
% \changes{v0.4c}{2004/07/26}{增加判断以避免嵌套定义~\cs{setcounter}~和~
%                 \cs{addtocounter}}
% \changes{v0.8}{2006/06/09}{增加对~\cs{stepcounter}~的重定义,以和~\texttt{calc}~宏包兼容}
% 重新定义~|\setcounter|~和~|\addtocounter|~以及时更新~|\cc@FOO|。
%    \begin{macrocode}
\AtBeginDocument{%
  \makeatletter%
  \@ifundefined{CTEX@save@setcounter}{%
    \let\CTEX@save@setcounter\setcounter%
    \def\setcounter#1#2{%
        \CTEX@save@setcounter{#1}{#2}%
        \CTEXcounter{#1}}}{}
  \@ifundefined{CTEX@save@addtocounter}{%
    \let\CTEX@save@addtocounter\addtocounter%
    \def\addtocounter#1#2{%
        \CTEX@save@addtocounter{#1}{#2}%
        \CTEXcounter{#1}}}{}
  \@ifundefined{CTEX@save@stepcounter}{%
    \let\CTEX@save@stepcounter\stepcounter%
    \def\stepcounter#1{%
        \CTEX@save@stepcounter{#1}%
        \CTEXcounter{#1}}}{}
  \makeatother}
%    \end{macrocode}
% \end{macro}
% \end{macro}
% \end{macro}
%
%
% \begin{macro}{\chinese}
% ~|\chinese|~用于获得计数器~|\FOO|~对应的中文数字~|\cc@FOO|。
%    \begin{macrocode}
\def\chinese#1{%
  \@ifundefined{cc@#1}{\CTEX@null}{\csname cc@#1\endcsname}}
%    \end{macrocode}
% \end{macro}
%
%
%    \begin{macrocode}
%</ctex|cls>
%    \end{macrocode}
%
%
% \subsection{中文化的标题结构}
%
% 在中文标题的处理上,需要修改~\LaTeX{}~标准文档的定义来实现中文标题。
% 这个功能最早是通过~\texttt{GB.cap}~来实现的,但是~CJK~提供的文件只能
% 配合~koma-script~文档类使用而不支持标准文档类。原因是标准文档类需要
% 作出较大的改动。我们最早尝试过把对标准文档类的修改也放到~GB.cap~文件
% 中,在简单应用中可以达到目的。但是我们还是意识到由于受到~GB.cap~文件
% 装入时间的限制,这种方案不可避免的带有兼容性问题。解决的方法就是通过
% 自定义的文档类来实现这个功能。此外,中文的标题格式也和英文有所不同,
% 这些都只有修改标准文档类的定义才能实现,于是就有了下面这些代码。
%
%
% \subsubsection{章节标题基本结构}
%
% 以下定义章节标题的基本结构单元。
%
%
%    \begin{macrocode}
%<*ctexcap|cls>
%    \end{macrocode}
%
%
% \begin{macro}{\CTEX@defsecname}
% 对于每种章节类型~|FOO|,我们都要定义一个章节名字~|\CTEXtheFOO|,
% 由~|\CTEX@preFOO|、~|\CTEX@theFOO|~和~|\CTEX@postFOO|~组成。
% 所以我们用下面的这个命令来完成这些重复工作。
%    \begin{macrocode}
\def\CTEX@defsecformat#1{%
%    \end{macrocode}
% 首先是无格式信息的章节标题。
%    \begin{macrocode}
  \expandafter\def\csname CTEX@pre#1\endcsname{}%
  \expandafter\def\csname CTEX@post#1\endcsname{}%
  \expandafter\def\csname CTEX@the#1\endcsname{%
    \csname the#1\endcsname}%
  \expandafter\def\csname CTEXthe#1\endcsname{%
    \csname CTEX@pre#1\endcsname%
    \csname CTEX@the#1\endcsname%
    \csname CTEX@post#1\endcsname}%
%    \end{macrocode}
% 然后是包含格式信息的章节标题,先定义一些用于控制格式的宏:
% \begin{description}
% \item~|\CTEX@FOO@format|~作用于整个标题,一般用于控制对齐方式;
% \item~|\CTEX@FOO@nameformat|~作用于整个章节名字(|\CTEXtheFOO|);
% \item~|\CTEX@FOO@numberformat|~作用于章节编号(|\CTEX@theFOO|);
% \item~|\CTEX@FOO@aftername|~作用于章节名字后的部分,一般用于控制章节
% 名字和章节标题内容之间是否换行等;
% \item~|\CTEX@FOO@titleformat|~用于控制标题内容的格式。
% \end{description}
% ~|\CTEX@FOOname|~则是在~|\CTEXtheFOO|~中加入了上面的这些格式控制得到的。
%    \begin{macrocode}
  \expandafter\def\csname CTEX@#1@format\endcsname{}%
  \expandafter\def\csname CTEX@#1@nameformat\endcsname{}%
  \expandafter\def\csname CTEX@#1@numberformat\endcsname{}%
  \expandafter\def\csname CTEX@#1@aftername\endcsname{}%
  \expandafter\def\csname CTEX@#1@titleformat\endcsname{}
  \expandafter\def\csname CTEX@#1name\endcsname{%
    \csname CTEX@#1@nameformat\endcsname%
    \csname CTEX@pre#1\endcsname%
    \begingroup%
    \csname CTEX@#1@numberformat\endcsname%
    \csname CTEX@the#1\endcsname%
    \endgroup%
    \csname CTEX@post#1\endcsname%
    \csname CTEX@#1@aftername\endcsname}%
%    \end{macrocode}
% 然后是标题上下的空距和缩进:
%    \begin{macrocode}
  \expandafter\def\csname CTEX@#1@beforeskip\endcsname{}%
  \expandafter\def\csname CTEX@#1@afterskip\endcsname{}%
  \expandafter\def\csname CTEX@#1@indent\endcsname{}%
}
%    \end{macrocode}
% \end{macro}
%
%
% 标准的七种章节类型都需要定义。
%    \begin{macrocode}
\CTEX@defsecformat{part}
\CTEX@defsecformat{chapter}
\CTEX@defsecformat{section}
\CTEX@defsecformat{subsection}
\CTEX@defsecformat{subsubsection}
\CTEX@defsecformat{paragraph}
\CTEX@defsecformat{subparagraph}
%    \end{macrocode}
%
% 缺省的一些设置,在不使用中文标题的时候使用。
%    \begin{macrocode}
\def\CTEX@prepart{Part\space}
\def\CTEX@prechapter{Chapter\space}
%    \end{macrocode}
%
% part~的缺省格式
%    \begin{macrocode}
%<ctex>\ifCTEX@cls{article}{
%<*ctex|article>
\def\CTEX@part@format{\raggedright}
\def\CTEX@part@nameformat{\Large\bfseries}
\def\CTEX@part@aftername{\par\nobreak}
\def\CTEX@part@titleformat{\huge\bfseries}
\def\CTEX@part@beforeskip{4ex}
\def\CTEX@part@afterskip{3ex}
\def\CTEX@part@indent{\z@}
%</ctex|article>
%<ctex>}{
%<*ctex|report|book>
\def\CTEX@part@format{\centering}
\def\CTEX@part@nameformat{\huge\bfseries}
\def\CTEX@part@aftername{\par\vskip 20\p@}
\def\CTEX@part@titleformat{\Huge\bfseries}
%</ctex|report|book>
%<ctex>}
%    \end{macrocode}
%
% chapter~的缺省格式
%    \begin{macrocode}
\def\CTEX@chapter@format{\raggedright}
\def\CTEX@chapter@nameformat{\huge\bfseries}
\def\CTEX@chapter@aftername{\par\nobreak\vskip 20\p@}
\def\CTEX@chapter@titleformat{\Huge\bfseries}
\def\CTEX@chapter@beforeskip{50\p@}
\def\CTEX@chapter@afterskip{40\p@}
\def\CTEX@chapter@indent{\z@}
%    \end{macrocode}
%
% section~的缺省格式
%    \begin{macrocode}
\def\CTEX@section@format{\Large\bfseries}
\def\CTEX@section@aftername{\quad}
\def\CTEX@section@beforeskip{-3.5ex \@plus -1ex \@minus -.2ex}
\def\CTEX@section@afterskip{2.3ex \@plus .2ex}
\def\CTEX@section@indent{\z@}
%    \end{macrocode}
%
% subsection~的缺省格式
%    \begin{macrocode}
\def\CTEX@subsection@format{\large\bfseries}
\def\CTEX@subsection@aftername{\quad}
\def\CTEX@subsection@beforeskip{-3.25ex \@plus -1ex \@minus -.2ex}
\def\CTEX@subsection@afterskip{1.5ex \@plus .2ex}
\def\CTEX@subsection@indent{\z@}
%    \end{macrocode}
%
% subsubsection~的缺省格式
%    \begin{macrocode}
\def\CTEX@subsubsection@format{\normalsize\bfseries}
\def\CTEX@subsubsection@aftername{\quad}
\def\CTEX@subsubsection@beforeskip{-3.25ex \@plus -1ex \@minus -.2ex}
\def\CTEX@subsubsection@afterskip{1.5ex \@plus .2ex}
\def\CTEX@subsubsection@indent{\z@}
%    \end{macrocode}
%
% 支持~|\subsubsection|~以下的小节标题编号
% paragraph~的缺省格式
%    \begin{macrocode}
\def\CTEX@paragraph@format{\normalsize\bfseries}
\def\CTEX@paragraph@aftername{\quad}
\ifnum\c@CTEX@sectiondepth>2
  \def\CTEX@paragraph@beforeskip{-3.25ex \@plus -1ex \@minus -.2ex}
  \def\CTEX@paragraph@afterskip{1ex \@plus .2ex}
\else
  \def\CTEX@paragraph@beforeskip{3.25ex \@plus1ex \@minus .2ex}
  \def\CTEX@paragraph@afterskip{-1em}
\fi
\def\CTEX@paragraph@indent{\z@}
%    \end{macrocode}
%
% subparagraph~的缺省格式
%    \begin{macrocode}
\def\CTEX@subparagraph@format{\normalsize\bfseries}
\def\CTEX@subparagraph@aftername{\quad}
\ifnum\c@CTEX@sectiondepth>3
  \def\CTEX@subparagraph@beforeskip{-3.25ex \@plus -1ex \@minus -.2ex}
  \def\CTEX@subparagraph@afterskip{1ex \@plus .2ex}
\else
  \def\CTEX@subparagraph@beforeskip{3.25ex \@plus1ex \@minus .2ex}
  \def\CTEX@subparagraph@afterskip{-1em}
\fi
\ifnum\c@CTEX@sectiondepth>2
  \def\CTEX@subparagraph@indent{\z@}
\else
  \def\CTEX@subparagraph@indent{\parindent}
\fi
%    \end{macrocode}
%
%
%
%
% \begin{macro}{\CTEX@appendixname}
% 附录的标题
%    \begin{macrocode}
%<ctexcap>\ifCTEX@cls{article}{
%<*ctexcap|article>
\def\CTEX@appendixname{}
%</ctexcap|article>
%<ctexcap>}{
%<*ctexcap|report|book>
\def\CTEX@appendixname{Appendix\space}
%</ctexcap|report|book>
%<ctexcap>}
%    \end{macrocode}
% \end{macro}
%
% \begin{macro}{\CTEX@appendixnumber}
% 附录的编号格式,缺省是大写英文字母。
%    \begin{macrocode}
%<ctexcap>\ifCTEX@cls{article}{
%<*ctexcap|article>
\def\CTEX@appendixnumber{\@Alph\c@section}
%</ctexcap|article>
%<ctexcap>}{
%<*ctexcap|report|book>
\def\CTEX@appendixnumber{\@Alph\c@chapter}
%</ctexcap|report|book>
%<ctexcap>}
%    \end{macrocode}
% \end{macro}
%
%
% \begin{macro}{\CTEX@save@appendix}
% \begin{macro}{\appendix}
% 重定义~|\appendix|~命令以改变附录标题
%    \begin{macrocode}
\let\CTEX@save@appendix\appendix
%<ctexcap>\ifCTEX@cls{article}{
%<*ctexcap|article>
\renewcommand*\appendix{\CTEX@save@appendix%
  \gdef\CTEX@presection{\CTEX@appendixname}%
  \gdef\CTEX@thesection{\CTEX@appendixnumber}%
  \gdef\CTEX@postsection{}}%
%</ctexcap|article>
%<ctexcap>}{
%<*ctexcap|report|book>
\renewcommand*\appendix{\CTEX@save@appendix%
  \gdef\CTEX@prechapter{\CTEX@appendixname}%
  \gdef\CTEX@thechapter{\CTEX@appendixnumber}%
  \gdef\CTEX@postchapter{}}%
%</ctexcap|report|book>
%<ctexcap>}
%    \end{macrocode}
% \end{macro}
% \end{macro}
%
%
%    \begin{macrocode}
%</ctexcap|cls>
%    \end{macrocode}
%
%
%
%
% \subsubsection{part~的标题}
%
%
% 以下一段修改标准~\LaTeX{}~文档类以实现中文标题需要的结构。
%
% part~的标题修改,首先是~article~类:
%    \begin{macrocode}
%<ctexcap>\ifCTEX@cls{article}{
%<*ctexcap|article>
\renewcommand\part{%
   \if@noskipsec \leavevmode \fi
   \par
%  \addvspace{4ex}%
   \addvspace{\CTEX@part@beforeskip}%
   \@afterindentfalse
   \secdef\@part\@spart}
\def\@part[#1]#2{%
  \ifnum \c@secnumdepth >\m@ne
    \refstepcounter{part}%
%   \addcontentsline{toc}{part}{\thepart\hspace{1em}#1}%
    \addcontentsline{toc}{part}{\CTEXthepart\hspace{1em}#1}%
  \else
    \addcontentsline{toc}{part}{#1}%
  \fi
  {\interlinepenalty \@M
%  \normalfont \parindent \z@ \raggedright
   \normalfont \parindent \CTEX@part@indent \CTEX@part@format
   \ifnum \c@secnumdepth >\m@ne
%    \Large\bfseries\partname\nobreakspace\thepart\par\nobreak
     \CTEX@partname
   \fi
%  \huge\bfseries #2%
   \CTEX@part@titleformat #2%
   \markboth{}{}\par}%
  \nobreak
% \vskip 3ex
  \vskip \CTEX@part@afterskip
  \@afterheading}
\def\@spart#1{%
    {\interlinepenalty \@M
%    \normalfont \parindent \z@ \raggedright
     \normalfont \parindent \CTEX@part@indent \CTEX@part@format
%    \huge \bfseries #1\par}%
     \CTEX@part@titleformat #1\par}%
     \nobreak
%    \vskip 3ex
     \vskip \CTEX@part@afterskip
     \@afterheading}
%</ctexcap|article>
%    \end{macrocode}
%
% 然后是~report~和~book~类:
%    \begin{macrocode}
%<ctexcap>}{
%<*ctexcap|report|book>
\def\@part[#1]#2{%
  \ifnum \c@secnumdepth >-2\relax
    \refstepcounter{part}%
%   \addcontentsline{toc}{part}{\thepart\hspace{1em}#1}%
    \addcontentsline{toc}{part}{\CTEXthepart\hspace{1em}#1}%
  \else
    \addcontentsline{toc}{part}{#1}%
  \fi
  \markboth{}{}%
  {\interlinepenalty \@M
%  \normalfont \centering
   \normalfont \CTEX@part@format
   \ifnum \c@secnumdepth >-2\relax
%    \huge\bfseries\partname\nobreakspace\thepart\par\vskip 20\p@
     \CTEX@partname
   \fi
%  \Huge\bfseries #2\par}%
   \CTEX@part@titleformat #2\par}%
  \@endpart}
\def\@spart#1{%
    {\interlinepenalty \@M
%    \normalfont \centering
     \normalfont \CTEX@part@format
%    \Huge \bfseries #1\par}%
     \CTEX@part@titleformat #1\par}%
    \@endpart}
%</ctexcap|report|book>
%<ctexcap>}
%    \end{macrocode}
%
%
%
%
% \subsubsection{chapter~的标题}
%
%
% \changes{v0.1c}{2003/08/19}{去掉生成的~\texttt{.out}~文件里
%                 章的标题前的多余空格}
% ~chapter~的标题修改,首先是~report~类的
%    \begin{macrocode}
%<ctexcap>\ifCTEX@cls{report}{
%<*ctexcap|report>
\def\@chapter[#1]#2{%
  \ifnum \c@secnumdepth >\m@ne
    \refstepcounter{chapter}%
%   \typeout{\@chapapp\space\thechapter.}%
    \typeout{\CTEXthechapter}%
    \addcontentsline{toc}{chapter}
%     {\protect\numberline{\thechapter}#1}%
      {\protect\numberline{\CTEXthechapter\hspace{0.3em}}#1}%
  \else
    \addcontentsline{toc}{chapter}{#1}%
  \fi
  \chaptermark{#1}%
  \addtocontents{lof}{\protect\addvspace{10\p@}}%
  \addtocontents{lot}{\protect\addvspace{10\p@}}%
  \if@twocolumn
    \@topnewpage[\@makechapterhead{#2}]%
  \else
    \@makechapterhead{#2}%
  \@afterheading
  \fi}
\def\@makechapterhead#1{%
% \vspace*{50\p@}%
  \vspace*{\CTEX@chapter@beforeskip}%
% {\normalfont \parindent \z@ \raggedright
  {\normalfont \parindent \CTEX@chapter@indent \CTEX@chapter@format
   \ifnum \c@secnumdepth >\m@ne
%    \huge\bfseries\@chapapp\space\thechapter\par\nobreak\vskip 20\p@
     \CTEX@chaptername
   \fi
   \interlinepenalty\@M
%  \Huge \bfseries #1\par\nobreak
   \CTEX@chapter@titleformat #1\par\nobreak
%  \vskip 40\p@
   \vskip \CTEX@chapter@afterskip
  }}
%</ctexcap|report>
%<ctexcap>}{}
%    \end{macrocode}
%
% 然后是~book~类的
%    \begin{macrocode}
%<ctexcap>\ifCTEX@cls{book}{
%<*ctexcap|book>
\def\@chapter[#1]#2{%
  \ifnum \c@secnumdepth >\m@ne
    \if@mainmatter
      \refstepcounter{chapter}%
%     \typeout{\@chapapp\space\thechapter.}%
      \typeout{\CTEXthechapter}%
      \addcontentsline{toc}{chapter}
%       {\protect\numberline{\thechapter}#1}%
        {\protect\numberline{\CTEXthechapter\hspace{0.3em}}#1}%
    \else
      \addcontentsline{toc}{chapter}{#1}%
    \fi
  \else
    \addcontentsline{toc}{chapter}{#1}%
  \fi
  \chaptermark{#1}%
  \addtocontents{lof}{\protect\addvspace{10\p@}}%
  \addtocontents{lot}{\protect\addvspace{10\p@}}%
  \if@twocolumn
    \@topnewpage[\@makechapterhead{#2}]%
  \else
    \@makechapterhead{#2}%
  \@afterheading
  \fi}
\def\@makechapterhead#1{%
% \vspace*{50\p@}%
  \vspace*{\CTEX@chapter@beforeskip}%
% {\normalfont \parindent \z@ \raggedright
  {\normalfont \parindent \CTEX@chapter@indent \CTEX@chapter@format
   \ifnum \c@secnumdepth >\m@ne
     \if@mainmatter
%      \huge\bfseries\@chapapp\space\thechapter\par\nobreak\vskip 20\p@
       \CTEX@chaptername
     \fi
   \fi
   \interlinepenalty\@M
%  \Huge \bfseries #1\par\nobreak
   \CTEX@chapter@titleformat #1\par\nobreak
%  \vskip 40\p@
   \vskip \CTEX@chapter@afterskip
  }}
%</ctexcap|book>
%<ctexcap>}{}
%    \end{macrocode}
%
% 有一些是~report~类和~book~类共有的
%    \begin{macrocode}
%<*ctexcap|report|book>
\def\@makeschapterhead#1{%
% \vspace*{50\p@}%
  \vspace*{\CTEX@chapter@beforeskip}%
% {\normalfont \parindent \z@ \raggedright
  {\normalfont \parindent \CTEX@chapter@indent \CTEX@chapter@format
   \interlinepenalty\@M
%  \Huge \bfseries  #1\par\nobreak
   \CTEX@chapter@titleformat #1\par\nobreak
%  \vskip 40\p@
   \vskip \CTEX@chapter@afterskip
  }}
%</ctexcap|report|book>
%    \end{macrocode}
%
%
%
%
% \subsubsection{section~的标题}
%
%
%    \begin{macrocode}
%<*ctexcap|cls>
%    \end{macrocode}
%
%
% 下面修改节的标题的显示方式
% \changes{v0.6b}{2005/11/07}{将节以下编号和标题之间的空距定义转移到相应的~aftername~变量中}
%    \begin{macrocode}
\def\@seccntformat#1{%
  \@ifundefined{CTEX@#1name}%
    {\csname the#1\endcsname\quad}%
    {\csname CTEX@#1name\endcsname}}
\def\@sect#1#2#3#4#5#6[#7]#8{%
  \ifnum #2>\c@secnumdepth
    \let\@svsec\@empty
  \else
    \refstepcounter{#1}%
    \protected@edef\@svsec{\@seccntformat{#1}\relax}%
  \fi
  \@tempskipa #5\relax
  \ifdim \@tempskipa>\z@
    \begingroup
      #6{%
        \@hangfrom{\hskip #3\relax\@svsec}%
%       \interlinepenalty \@M #8\@@par}%
        \interlinepenalty \@M
        \csname CTEX@#1@titleformat\endcsname #8\@@par}%
    \endgroup
    \csname #1mark\endcsname{#7}%
    \addcontentsline{toc}{#1}{%
      \ifnum #2>\c@secnumdepth \else
%       \protect\numberline{\csname the#1\endcsname}%
        \protect\numberline{\@ifundefined{CTEXthe#1}%
                              {\csname the#1\endcsname}%
                              {\csname CTEXthe#1\endcsname}}%
      \fi
      #7}%
  \else
    \def\@svsechd{%
    #6{\hskip #3\relax
%     \@svsec #8}%
      \@svsec \csname CTEX@#1@titleformat\endcsname #8}%
    \csname #1mark\endcsname{#7}%
    \addcontentsline{toc}{#1}{%
      \ifnum #2>\c@secnumdepth \else
%       \protect\numberline{\csname the#1\endcsname}%
        \protect\numberline{\@ifundefined{CTEXthe#1}%
                              {\csname the#1\endcsname}%
                              {\csname CTEXthe#1\endcsname}}%
      \fi
      #7}}%
  \fi
  \@xsect{#5}}
%    \end{macrocode}
%
% 通常中文章节标题编号会比较长,因此目录中的缩进距离也要调整。
% 我们通过修改~|\numberline|~命令来实现。
%    \begin{macrocode}
\def\numberline#1{%
  \settowidth\@tempdimb{#1\hspace{0.5em}}%
  \ifdim\@tempdima<\@tempdimb%
    \@tempdima=\@tempdimb%
  \fi%
  \hb@xt@\@tempdima{#1\hfil}}
%    \end{macrocode}
%
%    \begin{macrocode}
\renewcommand\section{\@startsection{section}{1}%
                                   {\CTEX@section@indent}%
                                   {\CTEX@section@beforeskip}%
                                   {\CTEX@section@afterskip}%
                                   {\normalfont\CTEX@section@format}}
\renewcommand\subsection{\@startsection{subsection}{2}%
                                   {\CTEX@subsection@indent}%
                                   {\CTEX@subsection@beforeskip}%
                                   {\CTEX@subsection@afterskip}%
                                   {\normalfont\CTEX@subsection@format}}
\renewcommand\subsubsection{\@startsection{subsubsection}{3}%
                                   {\CTEX@subsubsection@indent}%
                                   {\CTEX@subsubsection@beforeskip}%
                                   {\CTEX@subsubsection@afterskip}%
                                   {\normalfont\CTEX@subsubsection@format}}
\renewcommand\paragraph{\@startsection{paragraph}{4}%
                                   {\CTEX@paragraph@indent}%
                                   {\CTEX@paragraph@beforeskip}%
                                   {\CTEX@paragraph@afterskip}%
                                   {\normalfont\CTEX@paragraph@format}}
\renewcommand\subparagraph{\@startsection{subparagraph}{5}%
                                   {\CTEX@subparagraph@indent}%
                                   {\CTEX@subparagraph@beforeskip}%
                                   {\CTEX@subparagraph@afterskip}%
                                   {\normalfont\CTEX@subparagraph@format}}
%    \end{macrocode}
%
%
%    \begin{macrocode}
%</ctexcap|cls>
%    \end{macrocode}
%
%
%
%
% \subsubsection{页眉信息的修改}
%
%
% \changes{v0.3}{2004/04/27}{对页眉设置进行微调}
% 修改页眉中的标题,首先是~article~类中的定义
%    \begin{macrocode}
%<*ctexcap>
\ifCTEX@cls{article}{
%</ctexcap>
%<*ctexcap|article>
\if@twoside
  \def\ps@headings{%
      \let\@oddfoot\@empty\let\@evenfoot\@empty
      \def\@evenhead{\thepage\hfil\slshape\leftmark}%
      \def\@oddhead{{\slshape\rightmark}\hfil\thepage}%
      \let\@mkboth\markboth
    \def\sectionmark##1{%
      \markboth {\MakeUppercase{%
        \ifnum \c@secnumdepth >\z@
%         \thesection\quad
          \CTEXthesection\quad %
        \fi
        ##1}}{}}%
    \def\subsectionmark##1{%
      \markright {%
        \ifnum \c@secnumdepth >\@ne
%         \thesubsection\quad
          \CTEXthesubsection\quad %
        \fi
        ##1}}}
\else
  \def\ps@headings{%
    \let\@oddfoot\@empty
    \def\@oddhead{{\slshape\rightmark}\hfil\thepage}%
    \let\@mkboth\markboth
    \def\sectionmark##1{%
      \markright {\MakeUppercase{%
        \ifnum \c@secnumdepth >\m@ne
%         \thesection\quad
          \CTEXthesection\quad %
        \fi
        ##1}}}}
\fi
%</ctexcap|article>
%    \end{macrocode}
%
% 然后是~report~类中的页眉定义
%    \begin{macrocode}
%<ctexcap>}{\ifCTEX@cls{report}{
%<*ctexcap|report>
\if@twoside
  \def\ps@headings{%
      \let\@oddfoot\@empty\let\@evenfoot\@empty
      \def\@evenhead{\thepage\hfil\slshape\leftmark}%
      \def\@oddhead{{\slshape\rightmark}\hfil\thepage}%
      \let\@mkboth\markboth
    \def\chaptermark##1{%
      \markboth {\MakeUppercase{%
        \ifnum \c@secnumdepth >\m@ne
%         \@chapapp\ \thechapter. \ %
          \CTEXthechapter \quad %
        \fi
        ##1}}{}}%
    \def\sectionmark##1{%
      \markright {\MakeUppercase{%
        \ifnum \c@secnumdepth >\z@
%         \thesection. \ %
          \CTEXthesection \quad %
        \fi
        ##1}}}}
\else
  \def\ps@headings{%
    \let\@oddfoot\@empty
    \def\@oddhead{{\slshape\rightmark}\hfil\thepage}%
    \let\@mkboth\markboth
    \def\chaptermark##1{%
      \markright {\MakeUppercase{%
        \ifnum \c@secnumdepth >\m@ne
%         \@chapapp\ \thechapter. \ %
          \CTEXthechapter \quad %
        \fi
        ##1}}}}
\fi
%</ctexcap|report>
%    \end{macrocode}
%
% 然后是~book~类中的页眉定义
%    \begin{macrocode}
%<ctexcap>}{
%<*ctexcap|book>
\if@twoside
  \def\ps@headings{%
      \let\@oddfoot\@empty\let\@evenfoot\@empty
      \def\@evenhead{\thepage\hfil\slshape\leftmark}%
      \def\@oddhead{{\slshape\rightmark}\hfil\thepage}%
      \let\@mkboth\markboth
    \def\chaptermark##1{%
      \markboth {\MakeUppercase{%
        \ifnum \c@secnumdepth >\m@ne
          \if@mainmatter
%           \@chapapp\ \thechapter. \ %
            \CTEXthechapter \quad %
          \fi
        \fi
        ##1}}{}}%
    \def\sectionmark##1{%
      \markright {\MakeUppercase{%
        \ifnum \c@secnumdepth >\z@
%         \thesection. \ %
          \CTEXthesection \quad %
        \fi
        ##1}}}}
\else
  \def\ps@headings{%
    \let\@oddfoot\@empty
    \def\@oddhead{{\slshape\rightmark}\hfil\thepage}%
    \let\@mkboth\markboth
    \def\chaptermark##1{%
      \markright {\MakeUppercase{%
        \ifnum \c@secnumdepth >\m@ne
          \if@mainmatter
%           \@chapapp\ \thechapter. \ %
            \CTEXthechapter \quad %
          \fi
        \fi
        ##1}}}}
\fi
%</ctexcap|book>
%<*ctexcap>
}}
%</ctexcap>
%    \end{macrocode}
%
% 让前面的页眉定义生效:
%    \begin{macrocode}
%<ctexcap|cls>\pagestyle{headings}
%    \end{macrocode}
%
%
% \begin{macro}{\ps@fancy}
% \changes{v0.3}{2004/04/27}{解决与~\texttt{fancyhdr}~的冲突}
% \changes{v0.4d}{2004/08/14}{增加对~mainmatter~的判断}
% 如果使用了~\texttt{fancyhdr}~宏包,需要修改其定义的宏,以正确显示中文页眉。
%    \begin{macrocode}
%<*ctexcap|cls>
\@ifundefined{ps@fancy}{}{%
  \def\ps@fancy{%
  \@ifundefined{@chapapp}{\let\@chapapp\chaptername}{}%for amsbook
  \@ifundefined{MakeUppercase}{\def\MakeUppercase{\uppercase}}{}%
  \@ifundefined{chapter}{\def\sectionmark##1{\markboth
  {\MakeUppercase{\ifnum \c@secnumdepth>\z@
%  \thesection\hskip 1em\relax \fi ##1}}{}}%
   \CTEXthesection\quad \fi ##1}}{}}%
  \def\subsectionmark##1{\markright {\ifnum \c@secnumdepth >\@ne
%  \thesubsection\hskip 1em\relax \fi ##1}}}%
   \CTEXthesubsection\quad \fi ##1}}}%
  {\def\chaptermark##1{\markboth {\MakeUppercase{\ifnum \c@secnumdepth>\m@ne
%  \@chapapp\ \thechapter. \ \fi ##1}}{}}%
   \ifCTEX@cls{book}{\if@mainmatter\CTEXthechapter \quad\fi}{\CTEXthechapter \quad}
   \fi ##1}}{}}%
  \def\sectionmark##1{\markright{\MakeUppercase{\ifnum \c@secnumdepth >\z@
%  \thesection. \ \fi ##1}}}}%
   \CTEXthesection \quad \fi ##1}}}}%
  \ps@@fancy
  \gdef\ps@fancy{\@fancyplainfalse\ps@@fancy}%
  \ifdim\headwidth<0sp
      \global\advance\headwidth123456789sp\global\advance\headwidth\textwidth
  \fi}
}
%</ctexcap|cls>
%    \end{macrocode}
% \end{macro}
%
%
% \subsubsection{标签引用的中文化}
%
% \begin{macro}{\refstepcounter}
% \changes{v0.1e}{2003/11/05}{修正~\cs{ref}~命令后多出空格的问题}
% \changes{v0.4d}{2004/08/14}{修改~\cs{ref}~命令,不再包含除编号外的内容}
% \changes{v0.7}{2005/11/25}{删除在~.aux~文件中加入的~CJK~环境}
% 为了支持直接将中文数字写入~\texttt{.aux}~文件而做的修改。
%    \begin{macrocode}
%<*ctex|cls>
\ifCTEX@caption
  \let\CTEX@save@refstepcounter\refstepcounter
  \def\refstepcounter#1{\stepcounter{#1}%
    \protected@edef\@currentlabel
      {\csname p@#1\endcsname%
       \@ifundefined{CTEX@the#1}%
         {\csname the#1\endcsname}%
         {\csname CTEX@the#1\endcsname}%
       }}%
\fi
%</ctex|cls>
%    \end{macrocode}
% \end{macro}
%
%
%
%
% \subsubsection{其他中文标题定义}
%
%
%    \begin{macrocode}
%<*ctexcap|cls>
%    \end{macrocode}
%
%
% 除了章节标题,还有一些其他标题如目录、插图等,也都需要中文化。
% 标题中使用的中文都保存在以~|\CTEX@|~开头的宏中,这里先用英文初始化,
% 真正的中文定义放在~\texttt{ctexcap.cfg}~文件中。
%
% \begin{macro}{\CTEX@contentsname}
% \begin{macro}{\CTEX@listfigurename}
% \begin{macro}{\CTEX@listtablename}
% 目录、插图目录、表格目录的标题
%    \begin{macrocode}
\def\CTEX@contentsname{Contents}
\def\CTEX@listfigurename{List of Figures}
\def\CTEX@listtablename{List of Tables}
%    \end{macrocode}
% \end{macro}
% \end{macro}
% \end{macro}
%
% \begin{macro}{\CTEX@figurename}
% \begin{macro}{\CTEX@tablename}
% 插图和表格的标题
%    \begin{macrocode}
\def\CTEX@figurename{Figure}
\def\CTEX@tablename{Table}
%    \end{macrocode}
% \end{macro}
% \end{macro}
%
% \begin{macro}{\CTEX@abstractname}
% \begin{macro}{\CTEX@indexname}
% \begin{macro}{\CTEX@bibname}
% 摘要、索引和参考文献的标题
%    \begin{macrocode}
\def\CTEX@abstractname{Abstract}
\def\CTEX@indexname{Index}
\def\CTEX@bibname{Bibliography}
%    \end{macrocode}
% \end{macro}
% \end{macro}
% \end{macro}
%
%
% 然后修改原始的标题定义:
%
% \begin{macro}{\contentsname}
% \begin{macro}{\listfigurename}
% \begin{macro}{\listtablename}
% 目录、插图目录、表格目录的标题
%    \begin{macrocode}
\renewcommand*\contentsname{\CTEX@contentsname}
\renewcommand*\listfigurename{\CTEX@listfigurename}
\renewcommand*\listtablename{\CTEX@listtablename}
%    \end{macrocode}
% \end{macro}
% \end{macro}
% \end{macro}
%
% \begin{macro}{\figurename}
% \begin{macro}{\tablename}
% 插图和表格的标题
%    \begin{macrocode}
\renewcommand*\figurename{\CTEX@figurename}
\renewcommand*\tablename{\CTEX@tablename}
%    \end{macrocode}
% \end{macro}
% \end{macro}
%
% \begin{macro}{\abstractname}
% ~\texttt{book}~类中没有定义摘要,所以要先判断是否在使用~\texttt{book}~类。
%    \begin{macrocode}
\@ifundefined{abstractname}{}{
  \renewcommand*\abstractname{\CTEX@abstractname}}
%    \end{macrocode}
% \end{macro}
%
% \begin{macro}{\partname}
% \begin{macro}{\chaptername}
% \begin{macro}{\appendixname}
% 章节的标题,主要是为了和其他宏包的兼容性。
%    \begin{macrocode}
\renewcommand*\partname{\CTEX@thepart}
\@ifundefined{chaptername}{}{
  \renewcommand*\chaptername{\CTEX@thechapter}}
\renewcommand*\appendixname{\CTEX@appendixname}
%    \end{macrocode}
% \end{macro}
% \end{macro}
% \end{macro}
%
% \begin{macro}{\indexname}
% 索引的标题
%    \begin{macrocode}
\renewcommand*\indexname{\CTEX@indexname}
%    \end{macrocode}
% \end{macro}
%
% \begin{macro}{\refname}
% \changes{v0.1f}{2003/12/24}{修正~\texttt{article}~类中参考文献标题
%                 没有使用中文的问题}
% \begin{macro}{\bibname}
% 参考文献的标题。~\texttt{article}~和~\texttt{book}~使用了不同的名字,
% 所以要先判断是在使用哪个文档类
%    \begin{macrocode}
%<ctexcap>\ifCTEX@cls{article}{
%<ctexcap|article>  \renewcommand*\refname{\CTEX@bibname}
%<ctexcap>  }{
%<ctexcap|report|book>  \renewcommand*\bibname{\CTEX@bibname}
%<ctexcap>  }
%    \end{macrocode}
% \end{macro}
% \end{macro}
%
%
%    \begin{macrocode}
%</ctexcap|cls>
%    \end{macrocode}
%
%
%
%
% \subsubsection{用户设置命令}
%
%
%    \begin{macrocode}
%<*ctexcap|cls>
%    \end{macrocode}
%
%
% 这里的命令提供给用户在正文中修改设置中文标题。首先要定义一些内部命令,
% 用户可以通过~|\CTEXoptions|~的统一接口来访问这些内部命令。
%
% \begin{macro}{\CTEX@setsecname}
% 下面的命令用于重新定义章节标题
%    \begin{macrocode}
\def\CTEX@setsecname#1[#2,#3]{%
  \expandafter\def\csname CTEX@pre#1\endcsname{#2}%
  \expandafter\def\csname CTEX@post#1\endcsname{\ignorespaces#3}}
%    \end{macrocode}
% \end{macro}
%
% \begin{macro}{\CTEX@setsecnumber}
% 下面的命令用于重新定义章节格式
%    \begin{macrocode}
\def\CTEX@setsecnumber#1[#2]{%
  \expandafter\def\csname CTEX@the#1\endcsname{#2}}
%    \end{macrocode}
% \end{macro}
%
% \begin{macro}{\CTEX@setsecformat}
% 下面的命令用于重新定义章节格式
%    \begin{macrocode}
\def\CTEX@setsecformat#1#2[#3]{%
  \expandafter\def\csname CTEX@#1@#2\endcsname{#3}}
%    \end{macrocode}
% \end{macro}
%
% \begin{macro}{\CTEX@addsecformat}
% 下面的命令用于重新定义章节格式
%    \begin{macrocode}
\def\CTEX@addsecformat#1#2[#3]{%
  \expandafter\let\expandafter\reserved@a\csname CTEX@#1@#2\endcsname%
  \expandafter\expandafter\expandafter\def\expandafter\expandafter%
    \csname CTEX@#1@#2\endcsname\expandafter{\reserved@a#3}}
%    \end{macrocode}
% \end{macro}
%
% \changes{v0.2}{2004/02/01}{增加部分修改标题格式设置的选项}
% \changes{v0.2}{2004/02/01}{增加修改标题前后空距设置的选项}
% 定义对应的~key,使得用户可以通过~|\CTEXsetup|~命令来访问。
%    \begin{macrocode}
\def\CTEX@defseckey#1{%
  \CTEX@subkey{#1}{name}{\CTEX@setsecname{#1}[##1]}%
  \CTEX@subkey{#1}{number}{\CTEX@setsecnumber{#1}[##1]}%
  \CTEX@subkey{#1}{format}{\CTEX@setsecformat{#1}{format}[##1]}%
  \CTEX@subkey{#1}{format+}{\CTEX@addsecformat{#1}{format}[##1]}%
  \CTEX@subkey{#1}{nameformat}{\CTEX@setsecformat{#1}{nameformat}[##1]}%
  \CTEX@subkey{#1}{nameformat+}{\CTEX@addsecformat{#1}{nameformat}[##1]}%
  \CTEX@subkey{#1}{numberformat}{\CTEX@setsecformat{#1}{numberformat}[##1]}%
  \CTEX@subkey{#1}{numberformat+}{\CTEX@addsecformat{#1}{numberformat}[##1]}%
  \CTEX@subkey{#1}{aftername}{\CTEX@setsecformat{#1}{aftername}[##1]}%
  \CTEX@subkey{#1}{aftername+}{\CTEX@addsecformat{#1}{aftername}[##1]}%
  \CTEX@subkey{#1}{titleformat}{\CTEX@setsecformat{#1}{titleformat}[##1]}%
  \CTEX@subkey{#1}{titleformat+}{\CTEX@addsecformat{#1}{titleformat}[##1]}%
  \CTEX@subkey{#1}{beforeskip}{\CTEX@setsecformat{#1}{beforeskip}[##1]}%
  \CTEX@subkey{#1}{afterskip}{\CTEX@setsecformat{#1}{afterskip}[##1]}%
  \CTEX@subkey{#1}{indent}{\CTEX@setsecformat{#1}{indent}[##1]}%
}
%    \end{macrocode}
%
%    \begin{macrocode}
\CTEX@defseckey{part}
\CTEX@defseckey{chapter}
\CTEX@defseckey{section}
\CTEX@defseckey{subsection}
\CTEX@defseckey{subsubsection}
\CTEX@defseckey{paragraph}
\CTEX@defseckey{subparagraph}
%    \end{macrocode}
%
% 附录标题和编号的设置。
%    \begin{macrocode}
\CTEX@subkey{appendix}{name}{\def\CTEX@appendixname{#1}}
\CTEX@subkey{appendix}{number}{\def\CTEX@appendixnumber{#1}}
%    \end{macrocode}
%
% 其它标题的设置。
%    \begin{macrocode}
\CTEX@key{contentsname}{\def\CTEX@contentsname{#1}}
\CTEX@key{listfigurename}{\def\CTEX@listfigurename{#1}}
\CTEX@key{listtablename}{\def\CTEX@listtablename{#1}}
\CTEX@key{figurename}{\def\CTEX@figurename{#1}}
\CTEX@key{tablename}{\def\CTEX@tablename{#1}}
\CTEX@key{abstractname}{\def\CTEX@abstractname{#1}}
\CTEX@key{indexname}{\def\CTEX@indexname{#1}}
\CTEX@key{bibname}{\def\CTEX@bibname{#1}}
%    \end{macrocode}
%
%
%    \begin{macrocode}
%</ctexcap|cls>
%    \end{macrocode}
%
%
% \subsubsection{标题配置文件}
%
% \texttt{ctexcap.cfg}~文件中使用中文重新定义标题名称。
% 如果使用了中文标题的选项,则在~|\begin{document}|~
% 之后被自动载入。
%
%    \begin{macrocode}
%<*cap>
%    \end{macrocode}
%
% 首先是章节标题名称。
%    \begin{macrocode}
%%%%%%%%%%%%%%%%%%%%%%%%%%%%%%%%
%% caption name
%%%%%%%%%%%%%%%%%%%%%%%%%%%%%%%%

%    \end{macrocode}
%
% 目录、插图目录、表格目录的标题
%    \begin{macrocode}
\def\CTEX@contentsname{目录}
\def\CTEX@listfigurename{插图}
\def\CTEX@listtablename{表格}

%    \end{macrocode}
%
% 插图和表格的标题
%    \begin{macrocode}
\def\CTEX@figurename{图}
\def\CTEX@tablename{表}

%    \end{macrocode}
%
% 摘要、附录、索引和参考文献的标题
%    \begin{macrocode}
\def\CTEX@abstractname{摘要}
\def\CTEX@indexname{索引}
\def\CTEX@bibname{参考文献}

%    \end{macrocode}
%
% \changes{v0.3}{2004/04/24}{对中文标题的章节编号格式进行调整,去掉~\cs{S}}
% 章节标题,都以~pre~和~post~成对出现
%    \begin{macrocode}
\def\CTEX@prepart{第}
\def\CTEX@postpart{部分}
\def\CTEX@prechapter{第}
\def\CTEX@postchapter{章}
\def\CTEX@presection{}
\def\CTEX@postsection{}
\def\CTEX@presubsection{}
\def\CTEX@postsubsection{}
\def\CTEX@presubsubsection{}
\def\CTEX@postsubsubsection{}
\def\CTEX@preparagraph{}
\def\CTEX@postparagraph{}
\def\CTEX@presubparagraph{}
\def\CTEX@postsubparagraph{}

%    \end{macrocode}
%
% 附录的标题
%    \begin{macrocode}
\ifCTEX@cls{article}{
  \def\CTEX@appendixname{}
}{
  \def\CTEX@appendixname{附录~}
}

%    \end{macrocode}
%
% 然后是章节编号的格式。
%    \begin{macrocode}
%%%%%%%%%%%%%%%%%%%%%%%%%%%%%%%%
%% caption number
%%%%%%%%%%%%%%%%%%%%%%%%%%%%%%%%

%    \end{macrocode}
%
% \changes{v0.3}{2004/04/27}{修改为使用~\cs{chinese}~命令以避免产生错误}
% 篇和章的编号格式,缺省是中文数字。
%    \begin{macrocode}
\def\CTEX@thepart{\chinese{part}}
\def\CTEX@thechapter{\chinese{chapter}}

%    \end{macrocode}
%
% 节的编号格式,采用标准文档类的格式(阿拉伯数字)。
%    \begin{macrocode}
\def\CTEX@thesection{\thesection}
\def\CTEX@thesubsection{\thesubsection}
\def\CTEX@thesubsubsection{\thesubsubsection}
\def\CTEX@theparagraph{\theparagraph}
\def\CTEX@thesubparagraph{\thesubparagraph}

%    \end{macrocode}
%
% 附录的编号格式,缺省是大写英文字母。
%    \begin{macrocode}
\ifCTEX@cls{article}{
  \def\CTEX@appendixnumber{\@Alph\c@section}
}{
  \def\CTEX@appendixnumber{\@Alph\c@chapter}
}

%    \end{macrocode}
%
% 下面是章节标题的格式。
%    \begin{macrocode}
%%%%%%%%%%%%%%%%%%%%%%%%%%%%%%%%
%% caption format
%%%%%%%%%%%%%%%%%%%%%%%%%%%%%%%%

%    \end{macrocode}
%
% part~的缺省格式
%    \begin{macrocode}
\ifCTEX@cls{article}{
  \def\CTEX@part@format{\centering}
  \def\CTEX@part@nameformat{\Large\bfseries}
  \def\CTEX@part@aftername{\quad}
  \def\CTEX@part@titleformat{\Large\bfseries}
  \def\CTEX@part@beforeskip{4ex}
  \def\CTEX@part@afterskip{3ex}
  \def\CTEX@part@indent{\z@}
}{
  \def\CTEX@part@format{\centering}
  \def\CTEX@part@nameformat{\huge\bfseries}
  \def\CTEX@part@aftername{\par\vskip 20\p@}
  \def\CTEX@part@titleformat{\huge\bfseries}
}

%    \end{macrocode}
%
% chapter~的缺省格式
%    \begin{macrocode}
\def\CTEX@chapter@format{\centering}
\def\CTEX@chapter@nameformat{\huge\bfseries}
\def\CTEX@chapter@aftername{\quad}
\def\CTEX@chapter@titleformat{\huge\bfseries}
\def\CTEX@chapter@beforeskip{50\p@}
\def\CTEX@chapter@afterskip{40\p@}
\def\CTEX@chapter@indent{\z@}

%    \end{macrocode}
%
% section~的缺省格式
%    \begin{macrocode}
\def\CTEX@section@format{\Large\bfseries\centering}
\def\CTEX@section@aftername{\quad}
\def\CTEX@section@beforeskip{-3.5ex \@plus -1ex \@minus -.2ex}
\def\CTEX@section@afterskip{2.3ex \@plus .2ex}
\def\CTEX@section@indent{\z@}

%    \end{macrocode}
%
% subsection~的缺省格式
%    \begin{macrocode}
\def\CTEX@subsection@format{\large\bfseries\flushleft}
\def\CTEX@subsection@aftername{\quad}
\def\CTEX@subsection@beforeskip{-3.25ex \@plus -1ex \@minus -.2ex}
\def\CTEX@subsection@afterskip{1.5ex \@plus .2ex}
\def\CTEX@subsection@indent{\z@}

%    \end{macrocode}
%
% subsubsection~的缺省格式
%    \begin{macrocode}
\def\CTEX@subsubsection@format{\normalsize\bfseries\flushleft}
\def\CTEX@subsubsection@aftername{\quad}
\def\CTEX@subsubsection@beforeskip{-3.25ex \@plus -1ex \@minus -.2ex}
\def\CTEX@subsubsection@afterskip{1.5ex \@plus .2ex}
\def\CTEX@subsubsection@indent{\z@}

%    \end{macrocode}
%
% paragraph~的缺省格式
%    \begin{macrocode}
\def\CTEX@paragraph@format{\normalsize\bfseries\flushleft}
\def\CTEX@paragraph@aftername{\quad}
\ifnum\c@CTEX@sectiondepth>2
  \def\CTEX@paragraph@beforeskip{-3.25ex \@plus -1ex \@minus -.2ex}
  \def\CTEX@paragraph@afterskip{1ex \@plus .2ex}
\else
  \def\CTEX@paragraph@beforeskip{3.25ex \@plus1ex \@minus .2ex}
  \def\CTEX@paragraph@afterskip{-1em}
\fi
\def\CTEX@paragraph@indent{\z@}

%    \end{macrocode}
%
% subparagraph~的缺省格式
%    \begin{macrocode}
\def\CTEX@subparagraph@format{\normalsize\bfseries\flushleft}
\def\CTEX@subparagraph@aftername{\quad}
\ifnum\c@CTEX@sectiondepth>3
  \def\CTEX@subparagraph@beforeskip{-3.25ex \@plus -1ex \@minus -.2ex}
  \def\CTEX@subparagraph@afterskip{1ex \@plus .2ex}
\else
  \def\CTEX@subparagraph@beforeskip{3.25ex \@plus1ex \@minus .2ex}
  \def\CTEX@subparagraph@afterskip{-1em}
\fi
\ifnum\c@CTEX@sectiondepth>2
  \def\CTEX@subparagraph@indent{\z@}
\else
  \def\CTEX@subparagraph@indent{\parindent}
\fi

%    \end{macrocode}
%
%    \begin{macrocode}
%%%%%%%%%%%%%%%%%%%%%%%%%%%%%%%%
%% other configurations
%%%%%%%%%%%%%%%%%%%%%%%%%%%%%%%%

%    \end{macrocode}
%
%    \begin{macrocode}
%</cap>
%    \end{macrocode}
%
%
%
%
% \subsection{文档类}
%
%
%    \begin{macrocode}
%<*cls>
%    \end{macrocode}
%
% 针对中文习惯,对文档类的缺省设置进行修改。
%
%
%
%    \begin{macrocode}
%</cls>
%    \end{macrocode}
%
%
%
%
% \subsection{其它功能}
%
% \subsubsection{中文日期格式}
%
% \begin{macro}{\CTEX@todayold}
% \begin{macro}{\CTEX@todaysmall}
% \begin{macro}{\CTEX@todaybig}
% 几种常用的日期格式,都用~|\today|~进行初始化
%    \begin{macrocode}
%<*ctex|cls>
\let\CTEX@todayold\today
\let\CTEX@todaysmall\today
\let\CTEX@todaybig\today
%</ctex|cls>
%    \end{macrocode}
% |\CTEX@todaysmall|~和~|\CTEX@todaybig|~使用中文,需要在
% ~\texttt{ctex.def}~中重新定义
%    \begin{macrocode}
%<*def>
\def\CTEX@todaysmall{~\the\year~年~\the\month~月~\the\day~日}
\def\CTEX@todaybig{%
  \CJKdigits{\the\year}年\CJKnumber{\the\month}月\CJKnumber{\the\day}日}

%</def>
%    \end{macrocode}
% \end{macro}
% \end{macro}
% \end{macro}
%
% 缺省是使用阿拉伯数字的中文日期格式
%    \begin{macrocode}
%<*ctex|cls>
\ifCTEX@caption
  \renewcommand*\today{\CTEX@todaysmall}
\fi
%    \end{macrocode}
% 定义修改~|\today|~格式的用户命令
%    \begin{macrocode}
\CTEX@key{today}{\CTEX@settoday{#1}}
\newcommand*\CTEX@settoday[1]{%
  \@ifundefined{CTEX@today#1}
    {\PackageError{ctex}{%
       unknown today format}{%
       Available today format are "old", "small", and "big".}}
    {\renewcommand*\today{\csname CTEX@today#1\endcsname}}}
%</ctex|cls>
%    \end{macrocode}
%
%
% \subsubsection{图表标题的分隔符}
%
% \changes{v0.3}{2004/04/23}{增加对图表标题分隔符的设置}
% \changes{v0.5a}{2004/09/06}{修改图表标题分隔符设置中的错误}
% 重新定义图表标题的格式
%    \begin{macrocode}
%<*ctexcap|cls>
\def\CTEX@caption@delimiter{: }
\long\def\@makecaption#1#2{%
  \vskip\abovecaptionskip
  \sbox\@tempboxa{#1\CTEX@caption@delimiter #2}%
  \ifdim \wd\@tempboxa >\hsize
    #1\CTEX@caption@delimiter #2\par
  \else
    \global \@minipagefalse
    \hb@xt@\hsize{\hfil\box\@tempboxa\hfil}%
  \fi
  \vskip\belowcaptionskip}
%    \end{macrocode}
% 定义修改~|\today|~格式的用户命令
%    \begin{macrocode}
\CTEX@key{captiondelimiter}{\CTEX@setcaptiondelimiter{#1}}
\newcommand*\CTEX@setcaptiondelimiter[1]{%
  \def\CTEX@caption@delimiter{#1}}
%</ctexcap|cls>
%    \end{macrocode}
% 配置文件中给出缺省的分隔符定义
%    \begin{macrocode}
%<*cap>
\def\CTEX@caption@delimiter{: }

%</cap>
%    \end{macrocode}
%
%
%
%
% \subsection{用标准字体命令修改中文字体}
%
%
% \subsubsection{字体定义文件} \label{sec:fontdef}
%
%
% 本节的内容用于生成中文字体定义文件,这些定义文件将被~\ctex{}~宏包
% 做为缺省字体设置装入,用于在改变英文字体时相应的改变中文字体。
% ~\texttt{c19rm.fd}~文件定义的字体将在使用~|\rm|~系列字体命令时使用。
% 类似的,~\texttt{c19sf.fd}~文件定义的字体将在使用~|\sf|~系列字体命
% 令时使用;~\texttt{c19tt.fd}~文件定义的字体将在使用~|\tt|~系列字体
% 命令时使用。
%
%
% 首先,使用~\LaTeX{}~的~\texttt{NFSS}~命令定义新的字体名称,
% 都使用~GBK~编码。
%    \begin{macrocode}
%<*fd>
%<rm>\DeclareFontFamily{C19}{rm}{\hyphenchar \font\m@ne}
%<sf>\DeclareFontFamily{C19}{sf}{\hyphenchar \font\m@ne}
%<tt>\DeclareFontFamily{C19}{tt}{\hyphenchar \font\m@ne}
%    \end{macrocode}
%
%
% \changes{v0.8a}{2007/05/06}{增加~bold~字体的定义}
% 然后定义在各种情况下对应的真正汉字字体。
% 中文正常字体加黑都采用黑体代替,意大利体采用楷书代替,
% 意大利体加黑采用隶书代替。
% ~\texttt{rm}~字体中的普通字体采用宋体:
%    \begin{macrocode}
%<*rm>
\DeclareFontShape{C19}{rm}{m}{n}{<-> CJK * gbksong}{}
\DeclareFontShape{C19}{rm}{b}{n}{<-> CJK * gbkhei}{}
\DeclareFontShape{C19}{rm}{bx}{n}{<-> CJK * gbkhei}{}
\DeclareFontShape{C19}{rm}{m}{sl}{<-> CJK * gbksongsl}{}
\DeclareFontShape{C19}{rm}{b}{sl}{<-> CJK * gbkheisl}{}
\DeclareFontShape{C19}{rm}{bx}{sl}{<-> CJK * gbkheisl}{}
\DeclareFontShape{C19}{rm}{m}{it}{<-> CJK * gbkkai}{}
\DeclareFontShape{C19}{rm}{b}{it}{<-> CJKb * gbkkai}{\CJKbold}
\DeclareFontShape{C19}{rm}{bx}{it}{<-> CJKb * gbkkai}{\CJKbold}
%</rm>
%    \end{macrocode}
% ~\texttt{sf}~字体中的普通字体采用幼圆:
%    \begin{macrocode}
%<*sf>
\DeclareFontShape{C19}{sf}{m}{n}{<-> CJK * gbkyou}{}
\DeclareFontShape{C19}{sf}{b}{n}{<-> CJKb * gbkyou}{\CJKbold}
\DeclareFontShape{C19}{sf}{bx}{n}{<-> CJKb * gbkyou}{\CJKbold}
\DeclareFontShape{C19}{sf}{m}{sl}{<-> CJK * gbkyousl}{}
\DeclareFontShape{C19}{sf}{b}{sl}{<-> CJKb * gbkyousl}{\CJKbold}
\DeclareFontShape{C19}{sf}{bx}{sl}{<-> CJKb * gbkyousl}{\CJKbold}
\DeclareFontShape{C19}{sf}{m}{it}{<-> CJK * gbkyou}{}
\DeclareFontShape{C19}{sf}{b}{it}{<-> CJKb * gbkyou}{\CJKbold}
\DeclareFontShape{C19}{sf}{bx}{it}{<-> CJKb * gbkyou}{\CJKbold}
%</sf>
%    \end{macrocode}
% ~\texttt{tt}~字体中的普通字体采用仿宋:
%    \begin{macrocode}
%<*tt>
\DeclareFontShape{C19}{tt}{m}{n}{<-> CJK * gbkfs}{}
\DeclareFontShape{C19}{tt}{b}{n}{<-> CJKb * gbkfs}{\CJKbold}
\DeclareFontShape{C19}{tt}{bx}{n}{<-> CJKb * gbkfs}{\CJKbold}
\DeclareFontShape{C19}{tt}{m}{sl}{<-> CJK * gbkfssl}{}
\DeclareFontShape{C19}{tt}{b}{sl}{<-> CJKb * gbkfssl}{\CJKbold}
\DeclareFontShape{C19}{tt}{bx}{sl}{<-> CJKb * gbkfssl}{\CJKbold}
\DeclareFontShape{C19}{tt}{m}{it}{<-> CJK * gbkfs}{}
\DeclareFontShape{C19}{tt}{b}{it}{<-> CJKb * gbkfs}{\CJKbold}
\DeclareFontShape{C19}{tt}{bx}{it}{<-> CJKb * gbkfs}{\CJKbold}
%</tt>
%</fd>
%    \end{macrocode}
% 这些字体对应关系以后有可能根据用户意见做出调整。
%
%
% \subsubsection{字体命令修改}
%
%
% \begin{macro}{\rmfamily}
% \begin{macro}{\ttfamily}
% \begin{macro}{\sffamily}
% \begin{macro}{\CTEX@save@rmfamily}
% \begin{macro}{\CTEX@save@ttfamily}
% \begin{macro}{\CTEX@save@sffamily}
% 重新定义标准的字体命令,使得中文字体随着英文字体改变。
%    \begin{macrocode}
%<*cjk>
\let\CTEX@save@rmfamily\rmfamily
\renewcommand*\rmfamily{\CTEX@save@rmfamily\CJKfamily{rm}}
\let\CTEX@save@sffamily\sffamily
\renewcommand*\sffamily{\CTEX@save@sffamily\CJKfamily{sf}}
\let\CTEX@save@ttfamily\ttfamily
\renewcommand*\ttfamily{\CTEX@save@ttfamily\CJKfamily{tt}}
\let\CTEX@save@normalfont\normalfont
\renewcommand*\normalfont{\CTEX@save@normalfont\CJKfamily{rm}}
%</cjk>
%    \end{macrocode}
% \end{macro}
% \end{macro}
% \end{macro}
% \end{macro}
% \end{macro}
% \end{macro}
%
%
% \Finale
%
% \setcounter{IndexColumns}{2}
% \IndexPrologue{\section*{索引} {\it 意大利体的数字表示描述对应索引项的页码;
%                带下划线的数字表示定义对应索引项的代码行号;
%                罗马字体的数字表示使用对应索引项的代码行号。}}
%
% \GlossaryPrologue{\section*{版本更新}}
%
% \PrintIndex \PrintChanges
\endinput
