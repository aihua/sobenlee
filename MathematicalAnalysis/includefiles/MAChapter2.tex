%# -*- coding:utf-8 -*-
%%%%%%%%%%%%%%%%%%%%%%%%%%%%%%%%%%%%%%%%%%%%%%%%%%%%%%%%%%%%%%%%%%%%%%%%%%%%%%%%%%%%%
%%  MAChapter2.tex'


\chapter{极\emspace 限}\label{ch:2}

高等数学与初等数学的差别,除了研究对象不同外,主要是研究方法上的不同。我们知道,初等数学的方法建立在有限观念上,而
高等数学的方法则是建立在无限观念之上。比如初等数学中要求一个数,通过有限步的代数运算,即可求出它的准确值。但在客观
上存在着这样一种数,若只进行有限步的代数运算,则无法求得其准确的值。例如圆的面积和周长,用有限步代数运算就不能求得
其准确值,必须通过无限步逼近,即所谓极限方法,才能求出它的准确值,这就是高等数学的方法。又比如初等数学要确定数的性
质(如是否是素数),理论上通过有限步运算,就能断定它是否具有这一性质;而高等数学要确定函数的性质,就要通过极限方法
才能确定它是否具有此性质。

所以理解极限概念、掌握极限方法,是能否学好高等数学的关键。只有掌握极限这把钥匙,才能打开通向微积分的大门,变门外汉
为驾驭微积分工具的主人。


\section{序列极限的定义}

\subsection{概念引入}

试求由抛物线~$y=x^2$、$x$~轴、直线~$x=1$~所围成的曲边三角形的面积。

我们会求直边形的面积,不会求曲边形的面积。但是我们可以分两步来求出曲边形面积:先是通过以直代曲,得到一系列越来越逼近
曲边三角形面积的近似值;然后考察这一系列近似值的变化趋势,从而确定出曲边三角形面积的准确值。






\subsection{序列极限定义}
\begin{exercise}
\item
\end{exercise}
\section{序列极限的性质与运算}
\begin{exercise}
\item
\end{exercise}
\section{确界与单调有界序列}
\begin{exercise}
\item
\end{exercise}
\section{函数的极限}
\begin{exercise}
\item
\end{exercise}
\section{函数极限的推广}
\subsection{自变量趋于无穷的情形}
\subsection{无穷大量}
\subsection{单侧极限}
\subsection{极限存在性}
\subsection{复合函数求极限}
\begin{exercise}
\item
\end{exercise}
\section{两个重要极限}
\begin{exercise}
\item
\end{exercise}
\section{无穷小量的阶以及无穷大量的阶的比较}
\begin{exercise}
\item
\end{exercise}
\section{用肯定语气叙述极限不是某常数}
\subsection{极限不是某常数的肯定描述}
\subsection{序列极限与函数极限的关系}
\begin{exercise}
\item
\end{exercise}
\begin{exercise*}
\item
\end{exercise*}



\endinput
%%
%% End of file `MAChapter2.tex'.