%# -*- coding:utf-8 -*-
%%%%%%%%%%%%%%%%%%%%%%%%%%%%%%%%%%%%%%%%%%%%%%%%%%%%%%%%%%%%%%%%%%%%%%%%%%%%%%%%%%%%%


\chapter{玳安儿窃玉成婚\KG 吴典恩负心被辱}


诗曰:

\[
寺废僧居少,桥滩客过稀。家贫奴负主,官懦吏相欺。
水浅鱼难住,林稀鸟不栖。人情皆若此,徒堪悲复凄。
\]

话说孙雪娥在洒家店为娼,不题。却说吴月娘,自从大姐死了,告了陈敬济一状,大家人来昭也死了,他妻子一丈青带着小铁棍儿,也嫁人去了。来兴儿看守门户,房中绣春,与了王姑子做徒弟,出家去了。那来兴儿自从他媳妇惠秀死了,一向没有妻室。奶子如意儿,要便引着孝哥儿在他屋里顽耍,吃东西。来兴儿又打酒和奶子吃,两个嘲勾来去,就刮剌上了,非止一日。但来前边,归入后边就脸红。月娘察知其事,骂了一顿。家丑不可外扬,与了他一套衣裳,四根簪子,拣了个好日子,就与来兴儿完房,做了媳妇了。白日上灶看哥儿,后边扶持,到夜间往前边他屋里睡去。

一日,八月十五日,月娘生日。有吴大妗、二妗子,并三个姑子,都来与月娘做生日,在后边堂屋里吃酒。晚夕,都在孟玉楼住的厢房内听宣卷。到二更时分,中秋儿便在后边灶上看茶,由着月娘叫,都不应。月娘亲自走到上房里,只见玳安儿正按着小玉在炕上干得好。看见月娘推门进来,慌的凑手脚不迭。月娘便一声儿也没言语,只说得一声:“臭肉儿,不在后边看茶去,且在这里做甚么哩。”那小玉道:“我叫中秋儿灶上顿茶哩。”低着头,往后边去了。玳安便走出仪门,往前边来。

过了两日,大妗子、二妗子,三个女僧都家去了。这月娘把来兴儿房腾出收拾了,与玳安住。却教来兴儿搬到来昭屋里,看守大门去了。替玳安做了两床铺盖,一身装新衣服,盔了一顶新网新帽,做了双新靴袜;又替小玉编了一顶\textuni{4BFC}髻,与了他几件金银首饰,四根金头银脚簪,环坠戒指之类,两套段绢衣服,择日就配与玳安儿做了媳妇。白日里还进来在房中答应,只晚夕临关仪门时便出去和玳安歇去。这丫头拣好东好西,甚么不拿出来和玳安吃?这月娘当看见只推不看见。常言道:“溺爱者不明,贪得者无厌”,“羊酒不均,驷马奔镇”,“处家不正,奴婢抱怨”。

却说平安儿见月娘把小玉配与玳安,衣服穿戴胜似别人。他比玳安倒大两岁,今年二十二岁,倒不与他妻室。一日在假当铺,看见傅伙计当了人家一副金头面,一柄镀金钩子,当了三十两银子。那家只把银子使了一个月,加了利钱就来赎讨。傅伙计同玳安寻取来,放在铺子大橱柜里。不提防这平安儿见财起心,就连匣儿偷了,走去南瓦子里武长脚家——有两个私窠子,一个叫薛存儿,一个叫伴儿,在那里歇了两夜。忘八见他使钱儿猛大,匣子蹙着金头面,撅着银挺子打酒买东西。报与土番,就把他截在屋里,打了两个耳刮子就拿了。

也是合当有事,不想吴典恩新升巡简,骑着马,头里打着一对板子,正从街上过来,看见,问:“拴的甚么人?”土番跪下禀说:“如此这般,拐带出来瓦子里宿娼,拿金银头面行使。小的可疑,拿了。”吴典恩分付:“与我带来审问。”一面拿到巡简厅儿内。吴典恩坐下,两边弓皂排列。土番拴平安儿到根前,认的是吴典恩当初是他家伙计:“已定见了我就放的。”开口就说:“小的是西门庆家平安儿。”吴典恩说:“你既是他家人,拿这金东西在这坊子里做甚么?”平安道:“小的大娘借与亲戚家头面戴,使小的敢去,来晚了,城门闭了,小的投在坊子,权借宿一夜,不料被土番拿了。”吴典恩骂道:“你这奴才,胡说!你家这般头面多,金银广,教你这奴才把头面拿出来老婆家歇宿行使?想必是你偷盗出来的。趁早说来,免我动刑!”平安道:“委的亲戚家借去头面,家中大娘使我讨去来,并不敢说谎。”吴典恩大怒,骂道:“此奴才真贼,不打如何肯认?”喝令左右:“与我拿夹棍夹这奴才!”一面套上夹棍,夹的小厮犹如杀猪叫,叫道:“爷休夹小的,等小的实说了罢。”吴典恩道:“你只实说,我就不夹你。”平安儿道:“小的偷的假当铺当的人家一副金头面,一柄镀金银子。”吴典恩问道:“你因甚么偷出来?”平安道:“小的今年二十二岁,大娘许了替小的娶媳妇儿,不替小的娶。家中使的玳安儿小厮才二十岁,倒把房里丫头配与他,完了房。小的因此不愤,才偷出假当铺这头面走了。”吴典恩道:“想必是这玳安儿小厮与吴氏有奸,才先把丫头与他配了。你只实说,没你的事,我便饶了你。”平安儿道:“小的不知道。”吴典恩道:“你不实说,与我拶起来。”左右套上拶子,慌的平安儿没口子说道:“爷休拶小的,等小的说就是了。”吴典恩道:“可又来,你只说了,须没你的事。”一面放了拶子。那平安说:“委的俺大娘与玳安儿有奸。先要了小玉丫头,俺大娘看见了,就没言语,倒与了他许多衣服首饰东西,配与他完房。”这吴典恩一面令吏典上来,抄了他口词,取了供状,把平安监在巡简司,等着出牌,提吴氏、玳安、小玉来,审问这件事。

那日,却说解当铺橱柜里不见了头面,把傅伙计唬慌了。问玳安,玳安说:“我在生药铺子里吃饭,我不知道。”傅伙计道:“我把头面匣子放在橱里,如何不见了?”一地里寻平安儿寻不着,急的傅伙计插香赌誓。那家子讨头面,傅伙计只推还没寻出来哩。那人走了几遍,见没有头面,只顾在门前嚷闹,说:“我当了一个月,本利不少你的,你如何不与我?头面、钩子值七八十两银子。”傅伙计见平安儿一夜不来家,就知是他偷出去了。四下使人找寻不着,那讨头面主儿又在门首嚷乱。对月娘说,赔他五十两银子,那人还不肯,说:“我头面值六十两,钩子连宝石珠子镶嵌共值十两,该赔七十两银子。”傅伙计又添了他十两,还不肯,定要与傅伙计合口。正闹时,有人来报说:“你家平安儿偷了头面,在南瓦子养老婆,被吴巡简拿在监里,还不教人快认赃去!”这吴月娘听见吴典恩做巡简,“是咱家旧伙计。”一面请吴大舅来商议,连忙写了领状,第二日教傅伙计领赃去。有了原物在,省得两家领。

傅伙计拿状子到巡简司,实承望吴典恩看旧时分上,领得头面出来,不想反被吴典恩老狗奴才尽力骂了顿。叫皂隶拉倒要打,褪去衣裳,把屁脱脱了半日,饶放起来,说道:“你家小厮在这里供出吴氏与玳安许多奸情来,我这里申过府县,还要行牌提取吴氏来对证。你这老狗骨头,还敢来领赃!”倒吃他千奴才、万老狗,骂将出来,唬的往家中走不迭。来家不敢隐讳,如此这般,对月娘说了。月娘不听便罢了,听了,正是“分开八块顶梁骨,倾下半桶冰雪来”,慌的手脚麻木。又见那讨头面人,在门前大嚷大闹,说道:“你家不见了我头面,又不与我原物,又不赔我银子,只反哄着我两头来回走。今日哄我去领赃,明日等领头面,端的领的在那里?这等不合理。”那傅伙计赔下情,将好言央及安抚他:“略从容两日,就有头面来了。若无原物,加倍赔你。”那人说:“等我回声当家的去。”说毕去了。

这吴月娘忧上加忧,眉头不展。使小厮请吴大舅来商议,教他寻人情对吴典恩说,掩下这桩事罢。吴大舅说:“只怕他不受人情,要些贿赂打点他。”月娘道:“他当初这官,还是咱家照顾他的,还借咱家一百两银子,文书俺爹也没收他的,今日反恩将仇报起来。”吴大舅说:“姐姐,说不的那话了。从来忘恩背义,才一个儿也怎的?”吴月娘道:“累及哥哥,上紧寻个路儿,宁可送他几十两银子罢。领出头面来还了人家,省得合口费舌。”打发吴大舅吃了饭去了。

月娘送哥哥到大门首,也是合当事情凑巧,只见薛嫂儿提着花箱儿,领着一个小丫头过来。月娘叫住,便问:“老薛,你往那里去?怎的一向不来走走?”薛嫂道:“你老人家到且说的好,这两日好不忙哩。偏有许多头绪儿,咱家小奶奶那里,使牢子大官儿,叫了好几遍,还不得空儿去哩。”月娘道:“你看妈妈了撒风,他又做起俺小奶奶来了。”薛嫂道:、如今不做小奶奶,倒做了大奶奶了。”月娘道:“他怎的倒大奶奶?”薛嫂道:“你老人家还不知道,他好小造化儿!自从生了哥儿,大奶奶死了,守备老爷就把他扶了正房,做了封赠娘子。正经二奶奶孙氏不如他。手下买了两个奶子,四个丫头扶侍。又是两个房里得宠学唱的姐儿,都是老爷收用过的。要打时就打,老爷敢做主儿?自恁还恐怕气了他。那日不知因甚么,把雪娥娘子打了一顿,把头发都撏了,半夜叫我去领出来,卖了八两银子。今日我还睡哩,又使牢子叫了我两遍,教我快往宅里去,问我要两副大翠重云子钿儿,又要一副九凤钿儿。先与了我五两银子。银子不知使的那里去了,还没送与他生活去哩。这一见了我,还不知怎生骂我哩。”月娘道:“你到后边,等我瞧瞧怎样翠钿儿。”一面让薛嫂到后边坐下。薛嫂打开花箱,取出与吴月娘看。只见做的好样儿,金翠掩映,背面贴金。那个钿儿,每个凤口内衔着一挂宝珠牌儿,十分奇巧。薛嫂道:“只这副钿儿,做着本钱三两五钱银子;那副重云子的,只一两五钱银子,还没寻他的钱。”

正说着,只见玳安走来,对月娘说:“讨头面的又在前边嚷哩,说等不的领赃,领到几时?若明日没头面,要和傅二叔打了,到个去处理会哩。傅二叔心里不好,往家去了。那人嚷了回去了。”薛嫂问:“是甚么勾当?”月娘便长吁了一口气,如此这般,告诉薛嫂说:“平安儿奴才,偷去印子铺人家当的一副金头面,一副镀金钩子,走在城外坊子里养老婆,被吴巡简拿住,监在监里。人家来讨头面没有,在门前嚷闹。吴巡简又勒掯刁难,不容俺家领赃,又要打将伙计来要钱,白寻不出个头脑来。死了汉子,败落一齐来,就这等被人欺负,好苦也!”说着那眼中泪纷纷落将下来。

薛嫂道:“好奶奶,放着路儿不会寻。咱家小奶奶,你这里写个贴儿,等我对他说声,教老爷差人分付巡简司,莫说一副头面,就十副头面也讨去了。”月娘道:“周守备,他是武职官,怎管的着那巡简司?”薛嫂道:“奶奶,你还不知道,如今周爷,朝廷新与他的敕书,好不管的事情宽广。地方河道,军马钱粮,都在他手里打卯递手本。又河东水西,捉拿强盗贼情,正在他手里。”月娘听了,便道:“既然管着,老薛就累你,多上覆庞大姐说声。一客不烦二主,教他在周爷面前美言一句儿,问巡简司讨出头面来。我破五两银子谢你。”薛嫂道:“好奶奶,钱恁中使。我见你老人家刚才凄惶,我到下意不去。你教人写了帖儿,等我到府里和小奶奶说。成了,随你老人家;不成,我还来回你老人家话。”这吴月娘一面叫小玉摆茶与薛嫂吃。薛嫂儿道:“不吃罢,你只教大官儿写了贴儿来,你不知我一身的事哩。”月娘道:“你也出来这半日了,吃了点心儿去。”小玉即便放卓儿,摆上茶食来。月娘陪他吃茶。薛嫂儿递与丫头两个点心吃。月娘问丫头几岁了,薛嫂道:“今年十二岁了。”不一时,玳安前边写了说贴儿。薛嫂儿吃了茶,放在袖内,作辞月娘,提着花箱出门,径到守备府中。

春梅还在暖床上睡着没起来哩。只见大丫鬟月桂进来说:“老薛来了。”春梅便叫小丫头翠花,把里面窗寮开了。日色照的纱窗十分明亮。薛嫂进来说道:“奶奶,这咱还未起来?”放下花箱,便磕下头去。春梅道:“不当家化化的,磕甚么头?”说道:“我心里不自在,今日起来的迟些。”问道:“你做的翠云子和九凤钿儿拿了来不曾?”薛嫂道:“奶奶,这两副钿儿,好不费手!昨日晚夕我才打翠花铺里讨将来,今日要送来,不想奶奶又使了牢子去。”一面取出来,与春梅过目。春梅还嫌翠云子做的不十分现撇,还放在纸匣儿内,交与月桂收了。看茶与薛嫂儿吃。薛嫂便叫小丫鬟进来,“与奶奶磕头。”春梅问:“是那里的?”薛嫂儿道:“二奶奶和我说了好几遍,说荷花只做的饭,教我替他寻个小孩儿,学做些针指。我替他领了这个孩子来了。到是乡里人家女孩儿,今年才十二岁,正是养材儿。”春梅道:“你亦发替他寻个城里孩子,还伶便些。这乡里孩子,晓的甚么?”因问:“这丫头要多少银子?”薛嫂儿道:“要不多,只四两银子,他老子要投军使。”春梅叫海棠:“你领到二娘房里去,明日兑银子与他罢。”又叫月桂:“大壶内有金华酒,筛来与薛嫂儿烫寒。再有甚点心,拿一盒子与他吃。省得他又说,大清早辰拿寡酒灌他。”

薛嫂道:“桂姐,且不要筛上来,等我和奶奶说了话着,刚才也吃了些甚么来了。”春梅道:“你对我说,在谁家?吃甚来?”薛嫂道:“刚才大娘那头,留我吃了些甚么来了。如此这般,望着我好不哭哩。说平安儿小厮,偷了印子铺内人家当的金头面,还有一把镀金钩子,在外面养老婆,吃番子拿在巡简司拶打。这里人家又要头面嚷乱。那吴巡简旧日是咱那里伙计,有爹在日,照顾他的官。今日一旦反面无恩,夹打小厮,攀扯人,又不容这里领赃。要钱,才把傅伙计打骂将来。唬的伙计不好了,躲的往家去了。央我来,多多上覆你老人家。可怜见,举眼儿无亲的。教你替他对老爷说声,领出头面来,交付与人家去了,大娘亲来拜谢你老人家。”春梅问道:“有个贴儿没有?不打紧,你爷出巡去了,怕不的今晚来家,等我对你爷说。”薛嫂儿道:“他有说贴儿在此。”向袖中取出。春梅看了,顺手就放在窗户台上。

不一时,托盘内拿上四样嗄饭菜蔬,月桂拿大银钟,满满斟了一钟,流沿儿递与薛嫂。薛嫂道:“我的奶奶,我怎捱的这大行货子?”春梅笑道:“比你家老头子那大货差些儿。那个你倒捱了,这个你倒捱不的,好歹与我捱了。要不吃,月桂,你与我捏着鼻子灌他。”薛嫂道:“你且拿了点心,与我打个底儿着。”春梅道:“老妈子,单管说谎。你才说吃了来,这回又说没打底儿。”薛嫂道:“吃了他两个茶食,这咱还有哩?”月桂道:“薛妈妈,你且吃了这大钟酒,我拿点心与你吃。俺奶奶怪我没用,要打我哩。”这薛嫂没奈何,只得灌了一钟,觉心头小鹿儿劈劈跳起来。那春梅努个嘴儿,又叫海棠斟满一钟教他吃。薛嫂推过一边说:“我的那娘,我却一点儿也吃不的了。”海棠道:“你老人家捱一月桂姐一下子,不捱我一下子,奶奶要打我。”那薛嫂儿慌的直撅儿跪在地下。春梅道:“也罢,你拿过那饼与他吃了,教他好吃酒。”月桂道:“薛妈妈,谁似我恁疼你,留下恁好玫瑰馅饼儿与你吃。”就拿过一大盘子顶皮酥玫瑰饼儿来。那薛嫂儿只吃了一个,别的春梅都教他袖在袖子里:“到家稍与你家老王八吃。”薛嫂儿吃了酒,盖着脸儿,把一盘子火薰肉,腌腊鹅,都用草纸包裹,塞在袖内。海棠使气白赖,又灌了半钟酒。见他呕吐上来,才收过家伙,不要他吃了。春梅分付:“明日来讨话说,兑丫头银子与你。”临出门,春梅又分付:“妈妈,你休推聋装哑,那翠云子做的不好,明日另带两副好的我瞧。”薛嫂道:“我知道。奶奶叫个大姐送我送,看狗咬了我腿。”春梅笑道:“俺家狗都有眼,只咬到骨秃根前就住了。”一面使兰花送出角门来。

话休饶舌。周守备至日落时分,出巡来家,进入后厅,左右丫鬟接了冠服。进房见了春梅、小衙内,心中欢喜。坐下,月桂、海棠拿茶吃了,将出巡之事告诉一遍。不一时,放桌儿摆饭。饭罢,掌上烛,安排杯酌饮酒。因问:“前边没甚事?”春梅一面取过薛嫂拿的贴儿来,与守备看,说吴月娘那边,如此这般,“小厮平安儿偷了头面,被吴巡简拿住监禁,不容领赃。只拷打小厮,攀扯诬赖吴氏奸情,索要银两,呈详府县”等事。守备看了说:“此事正是我衙门里事,如何呈详府县?吴巡简那厮这等可恶!我明日出牌,连他都提来发落。”又说:“我闻得吴巡简是他门下伙计,只因往东京与蔡太题进礼,带挈他做了这个官,如何倒要诬害他家!”春梅道:“正是这等说。你替他明日处处罢。”一宿晚景题过。

次日,旋教吴月娘家补了一纸状,当厅出了大花栏批文,用一个封套装了。上批:“山东守御府为失盗事,仰巡简司官连人赃解缴。右差虞侯张胜、李安。准此。”当下二人领出公文来,先到吴月娘家。月娘管待了酒饭,每人与了一两银子鞋脚钱。傅伙计家中睡倒了,吴二舅跟随到巡简司。吴巡简见平安监了两日,不见西门庆家中人来打点,正教吏典做文书,申呈府县。只见守御府中两个公人到了,拿出批文来与他。见封套上朱红笔标着:“仰巡简司官连人解缴”,拆开,见里面吴氏状子,唬慌了。反赔下情,与李安、张胜每人二两银子。随即做文书解人上去。到于守备府前,伺候半日。待的守备升厅,两边军牢排下,然后带进入去。这吴巡简把文书呈递上去,守备看了一遍,说:“此是我衙门里事,如何不申解前来?只顾延捱监滞,显有情弊。”那吴巡简禀道:“小官才待做文书申呈老爷案下,不料老爷钧批到了。”守备喝道:“你这狗官可恶!多大官职?这等欺玩法度,抗违上司!我钦奉朝廷敕命,保障地方,巡捕盗贼,提督军务,兼管河道,职掌开载已明。你如何拿了这件,不行申解,妄用刑杖拷打犯人,诬攀无辜?显有情弊!”那吴巡简听了,摘去冠帽,在阶前只顾磕头。守备道:“本当参治你这狗官,且饶你这遭,下次再若有犯,定行参究。”一面把平安提到厅上,说道:“你这奴才,偷盗了财物,还肆言谤主。人家都是你恁般,也不敢使奴才了。”喝左右:“与我打三十大棍,放了。将赃物封贮,教本家人来领去。”一面唤进吴二舅来,递了领状。守备这里还差张胜拿贴儿同送到西门庆家,见了分上。吴月娘打发张胜酒饭,又与了一两银子。走来府里,回了守备、春梅话。

那吴巡简干拿了平安儿一场,倒折了好几两银子。月娘还了那人家头面、钩子儿。是他原物,一声儿没言语去了。傅伙计到家,伤寒病睡倒了,只七日光景,调治不好,呜呼哀哉死了。月娘见这等合气,把印子铺只是收本钱赎讨,再不解当出银子去了。止是教吴二舅同玳安,在门首生药铺子日逐转得来,家中盘缠。此事表过不题。

一日,吴月娘叫将薛嫂儿来,与了三两银子。薛嫂道:“不要罢,传的府里奶奶怪我。”月娘道:“天不使空人,多有累你,我见他不题出来就是了。”于是买下四盘下饭,宰了一口鲜猪,一坛南酒,一匹纻丝尺头,薛嫂押着来守备府中,致谢春梅。玳安穿着青绢褶儿,拿着礼贴儿,薛嫂领着径到后堂。春梅出来,戴着金梁冠儿,上穿绣袄,下着锦裙,左右丫鬟养娘侍奉。玳安扒到地下磕头。春梅分付:“放桌儿,摆茶食与玳安吃。”说道:“没甚事,你奶奶免了罢。如何又费心送这许多礼来,你周爷已定不肯受。”玳安道:“家奶奶说,前日平安儿这场事,多有累周爷、周奶奶费心,没甚么,些少微礼儿,与爷、奶奶赏人罢了。”春梅道:“如何好受的?”薛嫂道:“你老人家若不受,惹那头又怪我。”春梅一面又请进守备来计较了,止受了猪酒下饭,把尺头带回将来了。与了玳安一方手帕,三钱银子,抬盒人二钱。春梅因问:“你几时笼起头去,包了网巾?几时和小玉完房来?”玳安道:“是八月内来。”春梅道:“到家多顶上你奶奶,多谢了重礼。待要请你奶奶来坐坐,你周爷早晚又出巡去。我到过年正月里,哥儿生日,我往家里来走走。”玳安道:“你老人家若去,小的到家对俺奶奶说,到那日来接奶奶。”说毕,打发玳安出门。薛嫂便向玳安说:“大官儿,你先去罢,奶奶还要与我说话哩。”那玳安儿押盒担回家,见了月娘说:“如此这般,春梅姐让到后边,管待茶食吃。问了回哥儿好,家中长短。与了我一方手帕,三钱银子,抬盒人二钱银子。多顶上奶奶,多谢重礼,都不受来,被薛嫂儿和我再三说了,才受了下饭猪酒,抬回尺头。要不是请奶奶过去坐坐,一两日周爷出巡去。他只到过年正月孝哥生日,要来家里走走。”又告说:“他住着五间正房,穿着锦裙绣袄,戴着金梁冠儿,出落的越发胖大了。手下好少丫头、奶子侍奉!月娘问:“他其实说明年往咱家来?”玳安儿道:“委实对我说来。”月娘道:“到那日,咱这边使人接他去。”因问:“薛嫂怎的还不来?”玳安道:“我出门,他还坐着说话,教我先来了。”自此两家交往不绝。正是:世情看冷暖,人面逐高低。有诗为证:

\[
得失荣枯命里该,皆因年月日时栽。
胸中有志应须至,蠹里无财莫论才。
\]
