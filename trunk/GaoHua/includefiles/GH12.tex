%# -*- coding:utf-8 -*-
%%%%%%%%%%%%%%%%%%%%%%%%%%%%%%%%%%%%%%%%%%%%%%%%%%%%%%%%%%%%%%%%%%%%%%%%%%%%%%%%%%%%%

\chapter{革命向最高阶段发展:审干、肃奸与抢救运动}

\section{康生机关与~1937~年后延安的“肃托”}

在中共历史上,将暴力大规模引入党内的政治生活,使用强制手段调查党员的思想与历史情况,在~1949~年之前,以延安整风中的抢救运动最为典型,但是,抢救运动并非是从天上掉下来的。抢救运动的思想逻辑和运作方式有一个历史演化的过程,实际上,早在江西时期就已显出端倪,而~1937~年后在延安和其它中共根据地秘密进行的“肃托”则在一定程度上是抢救的先导和试验。

然而在中共党史编纂学中,却有一个建立在对历史事实改写基础上的系列神话:王明不仅是江西时期肃反“扩大化”的罪魁,而且是~1937~年后“肃托”的总策划者。对于王明与“肃托”的关系,中共党史编纂学却语焉不清,除了揭露出王明在莫斯科厉行“肃托”的一些事例外,对延安及其它根据地的“肃托”则一直讳莫如深,即使对具体领导延安“肃托”的康生也只是点到为止,似乎特别吝惜笔墨。毛泽东,这位延安的最高统治者,在中共党史编纂学中,更成为与“肃托”毫无关系的局外人。

毛泽东、王明与江西时期肃反的关系,本书前一部分已有专论,这里从略。“肃托”虽非毛泽东首创,但需要指出的是,与毛在江西肃反中所扮演的角色相类似,毛对于从莫斯科泊来的“肃托”主张,也是依据自己的需要迅速加以利用,使其完全服从于自己的政治目标,毛的所作所为不仅远甚于王明,更在中共党内造成不良的影响,为将暴力大规模引入党内政治生活开辟了一条危险的信道。

毛泽东在“肃托”问题上所要承担的责任具体表现在两个方面:第一,毛对王明、康生从莫斯科带来的“肃托”主张没有作过任何抵制,而是移花接木,接过“肃托”的口号,为其政治目标服务;第二,毛放手支持康生在延安“肃托”。

“肃托”的真正始作俑者是斯大林。王明作为这一主张在中国的主要引进人之一,\footnote{中共历史上第一次清除托洛茨基分子的事件发生在~1929~年,中共中央政治局在该年的~10~月~5~日,通过周恩来起草的《中央关于反对党内机会主义与托洛茨基分子反对派的决议》,警告陈独秀等必须主动服从中央决议,停止一切托派主张的宣传,11~月~5~日,中共中央宣布开除陈独秀等四入党籍。}对中共党内的“肃托”负有不可推卸的责任。从~1927~年始,斯大林就开始以“肃清反对派(托洛茨基派)”为由在苏共党内,继而在全苏范围内,大肆清洗、镇压他所认为的公开和潜在的政治对手。这种清洗、镇压在~1937~年后甚至演变为超恐怖的大屠杀。王明、康生在莫斯科期间,正是“肃托”逐渐趋于激烈的时期,1937~年~10~月下旬,王明、康生在苏联“肃托”恐怖达到最高潮的时刻返抵新疆迪化,同年~12~月下旬,盛世才以“托派”和“阴谋暴动”的罪名,将由苏联派往新疆工作的俞秀松、周达文逮捕。俞、周过去在莫斯科中山大学期间曾反对过王明,而王明返抵迪化后也曾散布俞秀松、周达文有托派嫌疑的言论,因此,俞、周被捕,王明、康生是摆脱不了干系的。1938~年~6~月~25~日,俞秀松被苏军押往苏联,不久就被苏联格伯乌秘密杀害。1938~年春,原红四方面军重要干部、参加西路军后撤退至新疆的李特和黄超在迪化被秘密处决。然而李特、黄超被杀的内情极为复杂,因为王明、康生与彼毫无个人恩怨,相反,李特、黄超却因参与张国焘与毛泽东的对立,与毛的关系极为紧张。李特、黄超之死,究竟是王明、康生为讨好毛而对邓发施加了影响,抑或是邓发得到延安密电而加害于两人?1937~年~11~月底王明、康生已回到延安,在迪化任八路军办事处处长的邓发在两人被杀事件中扮演了什么角色?知道其详情的,早些年也许只剩当年在迪化担任中共代表的陈云一人,而陈云数十年一直对此事守口如瓶。

毛泽东对王明从莫斯科贩来的“肃托”主张的态度颇为微妙。原来,在陈独秀重回中共问题上,毛已有所松动,但此议遭王明反对后,毛就不再坚持。毛原本就对陈独秀重新回党一事兴趣不大,既然奉斯大林之命的王明反对陈独秀回党,毛就更无必要为区区陈独秀事而与莫斯科把关系搞僵,王明还指责张闻天在莫斯科学习期间思想曾受托派影响,由此毛就更不反对“肃托”主张了。王明打击张闻天,正做了毛泽东自己想做又一时不便做的事,他所希冀的国际派的分裂眼看就要成为现实,何乐而不为?

于是,从~1938~年初开始,毛迅速接过“肃托”口号,不论演说和撰文,都忘不了数说一遍“日帝、托派汉奸的罪恶”。

毛泽东对“肃托”的热心远未停留在一般号召,与~1938~年后仅在口头上说说“肃托”的王明不同,毛立即将“肃托”与日常的肃反锄奸工作相衔接、毛在涉及军政肃反机要方面一向精细异常,然而就在他的眼皮底下,康生动用社会部,在来延安人员中进行了秘密调查和秘密逮捕,制造了一系列恐怖事件。

1937~年后的延安“肃托”,其处置对象主要集中在下列三类人员身上:

一、被怀疑与中国托派组织有联系的来延人员。

1938~年~3~月,边区保安处秘密逮捕陕北公学学员张醒及随张醒同来延安的两位青年妇女。保安处根据“情报”,认定张醒是托派山西分委书记(张醒的公开身份是阎锡山晋军的一个军长),两位女子,其中一人是项英前妻,1934~年红军长征后被国民党逮捕,并与一国民党“特务”结婚;另一位女子是胡宗南下属的宪兵队长的妻子。1938~年~6~月前后,在经历数月的审讯后,张醒援引张慕陶的话,交代康生在~1930~年被国民党逮捕后曾参加托派。当审讯者陈湖生按规定向边区保安处负责人周兴汇报后,张醒即迅速被秘密处决。张醒的一副翻毛皮手套和高级皮大衣随即出现在周兴住的塞洞外的院子里。与张醒同时被捕的两位妇女从此失踪,下落不明。不久,阎锡山还专门来电,向保安处要张醒,覆电是:延安无此人。很快,负责审理张醒一案的保安处侦察部部长陈湖生被投人监狱,险遭杀害。后在滕代远等人的保护下,陈湖生虽免一死,但仍长期坐牢。1944~年~2~月,时任边区参议会党团书记、并参加边区司法领导的谢觉哉调阅陈湖生案卷,认为陈湖生“所犯罪尚无反革命嫌疑,因某种顾虑致久羁抑殊有不妥”。\footnote{《谢觉哉日记》,上,页~575。}谢觉哉日记所言“某种顾虑”即陈湖生触犯康生事也,直至~1944~年~6~月,陈湖生在坐牢七年后才被释放。\footnote{陈复生(陈湖生)《三次被开除党籍的人——一个老红军的自述》:(北京:文化艺术出版社,1992~年),页~21~—~28。按照共产党内一个不成文的看法,只要被敌人逮捕,不论是否变节,都说明此人有疑点,需要进行严格的审查。是故,康生从未承认自己曾被捕过,直至八十年代,其妻曹轶欧仍坚持此说。但据叛变国民党的~1931~年临时中央政治局成员卢福坦的说法,康生在~1930~年,曾在沪被捕,并出卖过同志。建国后卢福坦被长期关押,1969~年被康生下令秘密处死。}1937~年后,在延安类似张醒这样的托派嫌疑分子被秘密处决的事件,并非个别。据司马璐回忆,1938~年初,曾经在边区医院住院的张抱平(1935~年温济泽曾和张抱平同囚于苏州江苏省陆军军人监狱)、镇江左翼青年李明、边区剧团演员林萍均因被怀疑为托派而遭秘密逮捕,从此永久失踪。\footnote{司马璐:《斗争十八年》(全本),页~52、56、60~—~62。}另据陶晶孙\footnote{陶晶孙在三十年代曾参加“左联”,1937~年后,奉潘汉年之命加入南京汪精卫政权组织的“中国文化协会”。}之子陶坊资披露,陶晶孙的两个妹妹~1937~年奔赴延安,其中之一的陶凯孙,“被康生打成反革命,在~1939~年被秘密杀害”\footnote{陶坊资:《回忆父亲》;陶瀛孙、陶乃煌:《陶晶孙小传》,载《新文学史料》,1992~年第~4~期,页~167、162。}。

康生认为中共广西党已被“托派”破坏,将在延安的一批广西籍干部秘密关押,其中有广西工委书记陈岸(八十年代任广西自治区人大副主任)和刘毅生(八十年代任广西自治区党委秘书长),陈岸被关押了两年半。\footnote{《广西部分老同志座谈党史资料征集工作》,载中共中央党史征集委员会、中共中央党史研究室编:《党史资料通讯》(1981~年合订本)(北京:中共中央党史出版社,1982~年),页~133。}

二、曾经在苏联学习工作过的人员。

顾顺平,上海人,1935~年底随张浩经外蒙辗转来到陕北,后作为重犯,长期被戴上脚镣秘密关押,与陈湖生同监。在关押期间,顾顺平曾在某夜偷锯脚镣企图逃跑,因陈湖生告发而事败。\footnote{陈复生(陈湖生)《三次被开除党籍的人——一个老红军的自述》:(北京:文化艺术出版社,1992~年),页~35、28。}顾顺平以后的下落不明,最大的可能是在逃跑失败后被秘密处决。

三、虽与托派毫无牵连,但被强加于“托派反革命”之名的原西路军干部。

据陈湖生揭露,1938~年后,某些返延安被集中审查的原西路军干部惨遭康生及其保卫机关秘密杀害。\footnote{陈复生(陈湖生)《三次被开除党籍的人——一个老红军的自述》:(北京:文化艺术出版社,1992~年)}这些历经千辛万苦返回延安的原红四方面军干部因与张国焘有较多的工作关系,遭到了与李特、黄超同样悲惨的命运。陈湖生之说是否属实?如果是假话,为何没有人出面澄清?如果属实,在延安被秘密杀害的西路军干部究竟有多少?至今仍不详。这悲惨的一页,徐向前、李先念等原西路军领导人都不忍重新揭开,直至去世都未泄露一字\footnote{李先念在“文革”后仍不愿谈有关西路军的历史,直至九十年代初,他才向其传记写作组人员谈了西路军的若干情况。徐向前逝世后,《炎黄春秋》杂志公布了他在~1982~年与廖盖隆等人的谈话,澄清了一些有关西路军历史的真相。徐向前坦承过去他为了与中央口径保持一致,写的一些文章并不完全符合历史事实,但是已公布的徐向前与廖盖隆的谈话仍未涉及延安处决西路军干部事。参见廖盖隆:《徐向前元帅生前的肺腑之言》,载《炎黄春秋》,1993~年第~1~期。}。

西路军干部被秘密处决,究竟是康生私自所为,抑或是奉旨行事,迄今仍无第一手档案资料证明,但综合各方面情况分析,1938~年的康生在没有得到明确的指令前,似乎还无胆量向红军干部动刀。1937~年后,返延的原西路军高级领导人徐向前、陈昌浩、张琴秋、李先念一直处于受压状态,毛决不放心让徐向前等执掌大战略区领导权。1939~年,毛委徐向前以八路军第一纵队司令员的头衔,让其带领一支小队伍前往冀南和山东,仅仅一年后,毛又以参加七大为由,将徐向前召回延安。陈昌浩则于~1939~年前往苏联,张琴秋被安排在延安中国女子大学作教务处长。李先念运气稍好一些。毛让李赤手空拳返回鄂豫地区,使李获得了重建功业的机会。在此种形势下,徐向前等绝无能力保护自己的部下。

“肃托”的罪恶尤其体现在非人道的审讯方式方面,在早期红军粗糙的战时审讯方法的基础上,康生又系统输人苏联格伯乌惩罚经验,使“逼供信”成了审讯业务中根深蒂固的传统。

中共粗糙的战时审讯方法形成于国共长期残酷的战争环境。从中共革命的角度来看,在面临生存巨大危险的非常时期,运用红色恐怖手段,用快速和极简单的程序处置敌对分子,本是天经地义。但是,这种依赖于暴力惩戒和单纯口供的审讯方法在执行中极易造成大量的冤假错案,被审讯者一般惧于肉刑而乱咬口供,而主持肃反机关的某些品质不良的领导者则有意利用虚假的口供向上邀功,这样,在革命队伍内部就不可避免发生持续性的自相残杀的惨剧。1932~年周恩来抵达瑞金后,虽然基本纠正了保卫机关受毛泽东影响而处于肃反狂热的极端行为,使肃反机关的矛头重新对外。\footnote{据李一氓回忆,1932~至~1934~年,江西瑞金的国家政治保卫局的“工作情况比较平稳”,“肃~AB~团”、“肃社会民主党”等事件均已过去。参见李一氓:《模糊的荧屏——李一泯回忆录》,页~159。}然而,由于周恩来等忌惮于毛泽东可能产生的过份反应,对毛多有退让,既未正面批评毛在肃反问题上的错误,也未建立起较为系统和完善的审讯制度。结果,这种粗糙的战时审讯方法在康生接管社会部后,不仅重又萌发,而且和苏联格伯乌“先进经验”迅速融合,甚至被视作中国革命和国际革命经验结合的产物被逐渐固定化,从而成为中共审讯工作的基本方法。

康生机关审讯案犯的常规方法就是动用肉刑逼供、诱供、套供。肉刑又包含有拷打、坐老虎凳、用电话线通电等种种刑罚。然而对于康生,一般的肉刑逼供也许已不能满足其虐杀心理的需要,他还需要更尖端的虐杀才能填补其灵魂的空虚。也正是在康生的主持下,延安“肃托”演出了令人发指的一幕。据原社会部成员、任弼时政治秘书师哲揭发:1940~—~1941~年间,师哲与陈郁参观延安柳树甸和平医院,曾被一女护士长带到一间大厅,发现一个大槽内用福尔马林浸泡着一具年约三十岁的男尸。护士长介绍说,“这是医学解剖用的,原来有三具。他们都是反革命分子,是由康生批准处理的。他们的姓名、来历,我们一概不知道”。师哲、陈郁听了颇觉诧异,便询问他们被送来时是否是活人,女护士长回答,“当然,以医疗的名义送来,然后处理的”\footnote{师秋朗(师哲之女)《:《我所知道的康生》一文被删部分补正》,载《炎黄春秋》,1992~年第~6~期。师哲所写的(我所知道的康生)刊载于《炎黄春秋》也~92~年第~5~期,但该文曾被《炎黄春秋》大量删节,且所删某些内容又恰是康生最阴暗的部分。师秋朗为此向《炎黄春秋》表示意见,《炎黄春秋》在同年第~6~期补发了师哲被删文章的部分内容。另参见师哲:《我所了解的的康生》,载《峰与谷——师哲回忆录》,页~215.}。

和这种处置方法相比,对于监禁的犯人采取特殊的着装与蓄发制度,就是小巫见大巫了。1937~年~10~月~27~日,陕甘宁边区高等法院曾为此发出通知:“……为防止犯人逃跑及易于辨认起见,特令犯人穿着半红半蓝的衫裤,发蓄在头之左右方,头顶及前后部之发,一律剃光”。\footnote{《陕甘宁边区法制史话》(诉讼狱政篇)(北京:法律出版社,1987~年),页~247~—~48。}1939~年这种制度曾经在一部分刑事犯中停止执行,但在政治犯中间仍然继续推行,因触犯康生而被秘密关押的原边区保安处侦察部长陈湖生,就穿着这种“以胸背的中线为界,左边红,右边黑”的对襟衣服,头上留着“一条不宽不窄的‘马路’”,戴着手铐脚镣,在延安过了“五年零四个月的带镣生活”(卸镣后又押了一年零七个月)。

在~1939~年秘密“肃托”的同时,康生还一手制造了三起著名的假案:

一、“钱惟人案”。钱惟人系当时的边区公路局局长,曾负责与边区交界处的国民党地方当局商谈修路事宜,却被康生无端诬蔑为“内奸”,并利用钱的妻子对钱进行侦察。钱惟人在延安被关押长达七年。

二、“王尊极案”。十九岁的王尊极陷入冤案系由她的家庭背景所致,由于王是大汉奸王克敏的侄女,就被康生凭空定为“日特”和“国特”。王尊极于~1939~年被捕人狱,受到三天三夜的车轮战,“搞得她两腿粗肿”,最后被迫招供。

三、“李凝案”。李凝原是东北地下党员,1938~年前来延安,只因“走路像日本女人”,“有一件日本式的女衬衣”,就被捕入狱。李凝最后下落不明\footnote{仲侃:《康生评传》,页~77~—~78;另参见《峰与谷——师哲回忆录》,页~216。}。

以上三个案件当时在延安都曾被康生作为自己的反特“政绩”在一定范围内予以公开,这和“肃托”的隐蔽化大不一样。

延安的“肃托”在极秘密的状态下基本由康生领导的中社部负责执行,主管干部审查和管理的中组部并不能广泛与闻有关“肃托”的详细过程。有资料显示,当时任中组部部长的陈云在“肃托”问题上持有比较谨慎的态度,曾经保护过被指称有“托派嫌疑”的同志。1938~年秋,曾参加过平津同学会南下宣传团的丁秀(1949~年后曾任中共鞍山市委副书记)夫妇因“托派嫌疑”被调回延安。在陈云的关照下,仅仅两周就查清了问题,陈云当面向丁秀夫妇宣布,不存在托派问题。如果不是陈云亲自过问此事,快速澄清问题,而是将案例移交中社部,丁秀夫妇以后的命运将不堪设想\footnote{刘家栋:《陈云在延安》,页~111~—~12。}。

尽管陈云在“肃托”问题上态度比较谨慎,但是在中共党内,除了毛泽东,没有任何人可以真正刹住“肃托”快车,其他人充其量只能做些纠偏一类的工作。

延安的“肃托”对其它根据地产生了恶劣的示范作用。1939~年~8~至~10~月,山东湖西地区(亦称苏鲁豫边区),在湖西区军政委员会主席王凤鸣、湖边地委组织部长王须仁和苏鲁豫边区党委书记白子明的主持下,掀起疯狂的“肃托”运动,滥杀党政军重要干部约三百人,被处决的方式有集体被刺刀捅死、马刀劈死、集体枪杀、活活打死、集体活埋、任凭狗咬、并用木榔头将脑壳砸烂。\footnote{《湖西“肃托事件”(中共济宁市委党史资料征集研究委员会调查报告)》,载中共中央党史研究室编:《中共党史资料》,第~32~辑(北京:中共党史资料出版社,1989~年),页~212~—~24;另参见郭影秋:《往事漫忆》(北京:中国人民大学出版社~1986~年),页~101~—~105。}在湖西“肃托”中,被关押待处决的党员、干部多达数百人,所使用的刑罚包括毒打、坐老虎凳、灌辣椒水、过电(用手摇电话机)、点香烧脸、用子弹刮肋骨、压杠子等七十二种酷刑。\footnote{《湖西“肃托事件”(中共济宁市委党史资料征集研究委员会调查报告)》,载中共中央党史研究室编:《中共党史资料》,第~32~辑(北京:中共党史资料出版社,1989~年),页~212~—~24;另参见郭影秋:《往事漫忆》(北京:中国人民大学出版社~1986~年),页~101~—~105。}在“红色恐怖”的风暴下,“肃托”中心地带的中共党政组织纷纷解体。只是由于八路军一一五师政委罗荣桓在~11~月赶到湖西,才制止住这场疯狂的内部屠杀,并从屠刀下救了郭影秋(1953~—~1957~年任云南省省长)等人的生命。

湖西事件的消息传到延安后,陈云于~1940~年初在中组部主持召开了湖西事件汇报会,康生、李富春、王鹤寿及原山东分局书记郭洪涛参加了这次会议。陈云就湖西事件作了基调发言,认为湖西肃托的错误在于当地党、军队领导人在主观上夸大了托派的力量,同时简单地将一切工作表现不好的人员都看成了托派,凭口供任意拘捕,使暗害分子王须仁有机可乘,以逞其借刀杀人的阴谋诡计。会后,根据陈云的意见,起草了《中央关于湖西地区锄奸工作错误的决定(草稿)》,经中央批准发至各根据地。\footnote{刘家栋:《陈云在延安》,页~115~—~17。}这份文件在制止“肃托”极端化方面有积极作用,但是仍坚持认为“肃托”是正确的,湖西确有“七个真托派”,错误仅在于扩大化。由于有这个基本判断,1940~至~1942~年,鲁南、滨海继续发生残杀干部的“肃托”事件,虽然罗荣桓在他力所能及的范围内都进行了干预,挽救了一百多人的生命,但山东的“肃托”,直到~1942~年~4~月刘少奇在前往延安、路经山东检查工作后,才完全停止下来。建国后,湖西“肃托”中深受迫害冤枉的人员在~1955~年的审干肃反运动中又遇到麻烦,被杀的“真托派”的家属遭受到数十年的歧视\footnote{郭影秋:《往事漫忆》,页~109~—~15。}。湖西的“肃托”案件的彻底平反是在~1983~年底,\footnote{1983~年~12~月,中共中央转发山东省委(关于湖西“肃托事件”遗留问题处理意见的报告),指出湖西“肃托”是一起重大的冤假错案,不是扩大化的错误,应予彻底平反。}距当年惨案的发生整整过去了四十三年。

在远离延安的山东,“肃托”已燃成野火一片,可是在“肃托”的策源地延安,“肃托”却一直处于地下秘密状态,各学校、机关不时发生人员神秘失踪的事件,不久就传来失踪人员是“托派”的流言。风声鹤唳,人人自危,在紧张、神秘的氛围中,延安的干部和党员都患上了“托派恐惧症”,“托派”一词成了瘟疫的代名词,个个谈“托”色变,惟恐避之不及。人们愈加护言慎行,生怕交友不慎,一下子上了保卫部门的“勾魂簿”。

在日益紧张的气氛中,康生机关的形象愈来愈高大,人们怀着恐惧、紧张的心情注视着枣园(中央社会部所在地),因为延安的人逐渐意识到,社会部是延安最有权势的机关,每个人的政治命运都操在它的手中。

\section{1940~年的审干与干部档案制度的建立}

进入~1940~年后,“肃托”继续在延安各机关、学校秘密进行,与此同时,新一轮“审干”已在公开状态下全面推开,与以往的情况类似,这一次的审干也是在社会部的密切配合下进行的。

对干部进行经常性的政治审查是中共的一项既定政策,但是~1940~年的审干却有其特定的背景,这次审干的主要目的在于对~1938~至~1939~年党员大发展阶段吸收入党的新党员实行政治过滤。

如前所述,“审干”发端于国共激烈对抗的十年内战时期,基于对国民党和国内其它党派的高度警惕,中共一向极为重视清理内部,即使为此错整或错杀了“自己人”也在所不惜。1937~年抗战爆发,中共获得合法地位后,大批青年投奔延安,一些原先失去组织关系的党员也纷纷归队,中共组织部门在社会部的配合下已对进入延安的每一个人实行了严格的审查,换言之,只有经过组织审查的人,才能被分配至各机关、学校、部队。

抗战初期,中共为了要在战时国内政治舞台上发挥更大的作用。以及为了未来与国民党一争天下,对于发展中共党员的数量有着强烈的紧迫感。十年内战结束时,中共的党员数量只剩下不足三万人,其中绝大部分在军队。显然党员数目之少已与中共在国内政治舞台所扮演的第二大党的角色十分不符。当时担任中组部部长的陈云就力主要与国民党争夺知识分子,他认为“谁抢到了知识分子,谁就抢到了天下”,“否则将来悔之晚矣”。\footnote{刘家栋:《陈云在延安》,页~30、94。}为此,1938~年~3~月~5~日,中共中央发出《关于大量发展党员的决定》,在以后的一年半的时间里,前往延安与各根据地的大批青年知识分子被吸收入党,以至于个别部门和地区,为了完成上级布置的发展党员的任务,反复动员、说服青年知识分子参加共产党,造成所谓在发展党员工作中“拉夫主义”的错误。待发展党员的任务基本完成后,中共中央又重新回到过去那种严格入党条件的道路。1939~年~8~月~25~日,中共中央发出《关于巩固党的决定》,要求用自上而下的方法,审查党员成份和各级干部,但又明确提出,这种审查“不应成为普遍的清党运动”。在一年半之内,中共在发展党员的政策上,一下子从敞开大门到暂停发展,其实并无矛盾,中共为了壮大党,需要“广招天下士”(陈云语),中共也是从现实需要出发,认定只有经过严格的审查,才能真正巩固党。在毛泽东和其他中共领袖看来,中共应该网罗天下英才为我所用,至于是否有嫌疑人员乘机混人共产党则大可不必惊慌,中共自有妙计对付此事,这个法宝就是审干。

就在党员大发展的~1939~年,针对新党员和从事白区斗争干部的审干就已同步进行。这个时期,中央社会部通过其密布在各个机关、学校的“网员”,对包括知识分子在内的各类人员进行秘密调查,这种调查虽然有成效,但是局限性也很明显。

首先,由于调查是在秘密状态下进行的,不易得到各级党组织的配合,调查的深度和广度都受到限制。

其次,中社部工作人员较少,无法应付延安几万名干部的秘密调查,大量的材料上报社会部后被长期积压。

正是基于以上情况,1940~年,中共中央决定正式开展审干工作。这次审干的目的有两个方面:第一,为了发现干部的长处与优点,以便加以“适当的培养、使用、提拔和调动干部”;第二,“为了发现混在党内的异己分子,以便清洗他们出党而巩固党的行列”。\footnote{参见~1940~年~8~月~14~日,中组部对审查干部经验的总结,引自中共中央党史研究室编:《中共党史大事年表说明》(北京:中央党校出版社,1983~年),页~100~—~101。}在这之前不久,中共中央也改变了中社部原垂直侦察系统,改由各级社会部受同级党委和中社部双重领导。在这次审干中,中组部和各级党委的组织科(干部科)处在前台,中社部则在幕后予以配台和协调。

1940~年审干所获取的最重要的成果是初步建立起中共干部的档案资料管理制度。

十年内战时期,中共对党员的个人档案资料管理尚处于非正规化的水平,由于战争环境和地下斗争的险恶,干部的个人档案资料很难予以有效的保存。1935~年~10~月,中央红军历经艰险抵达陕北,带至陕北的党和军队的核心机密文电总数仅五十余斤。\footnote{费云东主编:《中共保密工作简史(1921~—~1949)》(北京:金城出版社,1994~年),页~100。}为了安全的因素,干部的个人档案资料在长征前夕已尽行销毁。

党员干部个人档案资料管理制度的建立及逐渐正规化,开始于抗战初期。七七事变后,大批青年知识分子和失去组织关系的党员纷纷来到延安,中共也获得了陕甘宁边区这块较为稳定的地盘,于是,在中组部干部科的策划下,逐渐建立起党员干部的个人档案管理制度。

1938~年初,司马璐在经过一段时间的审查后,又经中组部部长陈云面谈,由中组部分配去位于枣园的“敌区工作委员会”报到。中组部工作人员交给司马璐一封信,要他面交枣园机关主任秘书曾希圣,司马璐在途中偷拆了信,发现其中一份是介绍信,另一份是中组部对司马璐作的“鉴定表”,上面写着几段评话:

\begin{quoting}
有组织能力和斗争经验,但具有浓厚的个人英雄主义,个性强,骄傲,缺少劳动观念,组织生活锻炼不够\footnote{司马璐:《斗争十八年》(全本),页~68。}。
\end{quoting}这个时期,中共的组织和干部工作尚处于恢复和重建时期,许多手续还不健全。以后,组织部门给干部的“鉴定”一般不由干部面交,即或交由干部转交,也需在信封放口处加上火漆,以防干部偷拆。

延安初期的干部档案包括以下几个方面的内容:

一、干部填写的履历表,

二、党的组织部门对干部履历的鉴定或结论,

三、干部所在部门党组织对干部政治思想及各方面表现的鉴定,

四、有关干部履历的其它证明材料。

干部个人档案由各机关、学校的干部科(组织科)管理,领导干部的个人档案由上一级组织部门管理。

审干开始后,运作方法仍是动员干部自己报告个人历史,审干人员则从干部填写的材料中查找疑点,在党员按照组织要求,多次填写的各种履历表格中很快就可发现各种问题。

1940~年~6~月,延安马列学院在~29~个抗战后入党的新党员填写的表格中,发现前后一致的仅~103~人,占总数的~33\%,“其他~67\%~都是经过了党的无数次教育解释工作才改正了、补充了自己的历史的。\footnote{马洪:《马列学院审查干部工作中的一些经验》,载《延安马列学院回忆录》,页~47。}”

为什么会出现前后表格不一致的现象?基本情况有两类:

一、许多新党员担心自己的阶级成份是地、富,害怕受到歧视。因此将家庭出身改成贫农、中农或“没落的小资产阶级”。

二、一些在白区工作的老党员一度失去组织关系,深恐引起组织的怀疑,来延安时没有详细交待,现在仍“将错就错,填下去”。

上述情况在延安各单位都普遍存在,自然引起中组部的高度重视。1940~年~3~月,陈云发表了《党员对党要忠诚》的文章。中组部要求各级党组织加强对党员的审干教育,对党员进行反复的说服、教育、启发,解除党员“惧怕”、“怀疑”审干的心理,让党员明白审干“是有利于党,有利于他自己的”,只要党员改正了过去填表不真实的错误,他过去的行为,不仅不会受到党的怀疑,相反,党会认为他在政治上取得了进步。与此同时,中组部还要求从事审干的同志要加强对干部表格和各类资料的分析、调查。由于干部填表前后不一致成为一种普遍现象,中组部形成了几个基本的看法:

党员最初填写的表不一定是完整和真切的,有的干部是不会一次向党老实交待自己历史的,为此必须多方搜集干部的个人资料。

如何搜集呢?

首先要求干部提供个人历史的证明人,这又包括几个环节:

一、在干部本身的材料中找证明人;

二、从干部提供的证明处找证明人;

三、从谈话中找新的证明人;

四、从相同时间、相同地区、相同事件中找证明人;

五、从此人的材料中发现彼人的证明人。

即使有了证明人,也不能完全相信,因为存在几种可能性:

一、提供证明人与被调查对象暗中串联,互相包庇,互相吹捧对方。例如,经常会出现证明人提供这样的材料,证明某人是“非常坚定的无产阶级战士”,但实际情况是他们同样在监狱中出卖过同志。

二、有的证明人因惧怕牵累,不敢为被调查对象出具证明。

经过反复细致的说服动员后,干部的各种审查资料都陆续到齐,下一步就是具体地分析和判别这些材料。这个过程又包括以下几个步骤:

一、首先研究党组织对这个干部的介绍资料,这种介绍材料一般具有两个核心部分:

组织上对这个干部的基本评价,

提出对该干部需要重点考察的问题。

二、检查这个干部本人写的材料:

先看他最近写的材料,再从中找出疑点,即组织介绍材料中所提出的重点考察部分,把两者进行互相印证。

继而从他本人写的各种材料中找出可疑与不清的问题。

三、检查他人提供的旁证材料,这也需要几个步骤:

首先确定旁证人的可靠程度,

从旁证人材料中找出组织介绍材料中提出的重点问题,从旁证材料中再找出受审干部自己所写材料中暴露的可疑部分。

经过对以上材料的反复对比研究,就可以查明该干部所存在的问题的性质。

下一步的工作就是将问题提到干部科科务会议,进行逐个研究,对这个干部作出组织结论或鉴定。

在作出组织结论之前,还有一个程序,这就是干部科的同志与受审干部进行个别谈话。

这种谈话的目的在于进一步核对材料,或发现该干部过去在填写各种表格时未予反映的内容,因此在谈话中,审查人员不应打断对方的谈话,尽量从被审查对象的谈话中发现问题。当这一切都完成后,就到了做正式结论的阶段。

结论一般包括两个部分:

一、历史审查的结论。一般由干部科会议集体作出,可以向当事者公开。在这部分的结论中,应对干部的历史中已搞清楚的问题,作出肯定的结论。但是,如果问题仍不清楚,旁证材料不够齐全,肯定的结论也可不做。

至于干部某段历史尚无证人证明,则对干部有证人证明的历史,和组织上已清楚了解的历史作出一般的结论。但需写明,某段历史尚不能肯定。

对另一些问题严重的干部,一时无法找到证人证明,则将其所有问题提出,全部保留,并由组织上对此人做出一般结论,以供日后继续考察。

二、在单位表现的鉴定。这类鉴定由党员所在的党小组、党支部在会议上讨论并作出鉴定,被鉴定者本人可以列席会议并表示自己的意见。在党支部鉴定的基础上,该干部所在单位的党组织结合对干部历史的考察,做出干部在单位表现的鉴定。党组织对干部表现的鉴定通常不与本人见面。它应该包含被鉴定者的政治思想状况、党性修养以及干部个性、特长、工作经验与能力,以及对干部今后工作发展方向的建议等方面的内容。

到了这一步,对干部的审查就可以告一段落,而审干过程中形成的具体文字资料就成了干部的个人档案。从此这份档案就尾随干部,像一个无形的影子,干部调到哪儿,这份档案就跟着他到哪儿。以后每逢审干或政治运动,这份档案都会增加内容,党组织都会在这份档案中写上对这个干部的考察意见,它将决定这个干部在政治上的前途,或被提拔重用,“不或得重用”,或被“控制使用”。于是干部档案就成为决定干部命运的一件利器,它同时也成了一只“不死鸟”。它既属于这个干部,又是完全独立于干部个人的异己物(干部通常不知道领导在自己的档案中写些什么),两者相依相随,一直到这个干部离开人世,这份干部个人档案也还未寿终正寝。它被置放在某个文件柜中,在对这个干部的妻子儿女、亲戚朋友的政治审查中还将继续发挥作用。

在延安各机关、学校审干工作正紧锣密鼓全面展开的同时,社会部的秘密侦察工作也在有条不紊地同步进行。

如前所述,1939~年后,社会部加强了对延安各机关、学校人员的秘密考察业务,被列人考察对象的人员包括以下几类:

从国民党监狱释放来延安的人员;

来延安时介绍信不清的人员;

年龄与相貌不符的人员;

喜欢打探小道消息的人员;

在政治上、经济上有空隙可以被敌人利用的人员……

社会部如何得知这些干部的背景?没有组织部门的协助和提供介绍,显然是不可能的。尽管社会部在各机关、学校派有单线联系的秘密情报员——“网员”,但依当时的规定,各单位的工作人员互相不得打听彼此的背景(有些同志或有可能被派往国统区工作),因此,社会部获取干部资料的渠道主要是各级组织部门。

1940~年~9~月~20~日,中央社会部发布《除奸工作指示》,要求延安各机关、学校划出审干中的嫌疑对象,将其材料上报社会部。\footnote{参见修来荣:《陈龙传》(北京:群众出版社,1995~年),页~92。}根据这份指示,一大批嫌疑分子的材料被集中到社会部,另有一些人,嫌疑程度尚不足上报社会部,其材料则由各单位组织部门自行掌握。

从~1940~至~1941~年上半年,社会部会同各机关、学校的组织部门和保卫委员会,对集中在社会部的嫌疑分子材料进行鉴别,并展开对这些嫌疑分子的秘密侦察。

然而,确定“嫌疑对象”并没有太多的事实依据,对大多数嫌疑分子的怀疑,主要依据的是他们本人所填写的各种表格,而这些表格中所反映的问题,也大多是“剥削阶级”家庭出身和社会关系复杂一类。以及曾集体参加过国民党、三青团、复兴社。

当时,这批已被内定为“嫌疑分子”的人员大多是在陕北公学、中央党校学习的学员,他们之中,除了少数人是在国统区自行报考进入延安的(陕北公学曾在国统区刊登过招生广告),绝大多数人都是经各地中共党组织或八路军、新四军办事处推荐介绍来延安的,在“嫌疑分子”中也有一些党龄较长的老干部。可是他们依据事实填写的各种表格竟成为将他们定为“嫌疑分子”的唯一依据,他们将由此被长期秘密审查,一“挂”就是几年,非党员的不得入党,也不被分配到急需干部的前线。

林纳,延安中国女子大学政治处副处长,1940~年秋在审干高潮中被王明主持的校务委员会免去职务,将其调往由张琴秋担任处长的教务处,做什么工作,担任什么职务一概不予宣布。林纳被免职的真正原因是受到其夫的株连,因而受到党的怀疑,被认为是“嫌疑分子”。林纳与其夫都是留苏干部,抗战爆发后,夫妇俩奉命返国,但在临行前,其夫被苏联格伯乌逮捕,林纳一人返回了延安,被分配在女大任政治处副处长。女大的审干由政治处处长孟庆树(王明之妻)和政治处干部科科长叶群负责,但立案审查林纳,是中央社会部的决定。为了让林纳“坦白交待”问题,叶群经常找林纳谈话,对其施加种种心理压力,有时“拍桌子,瞪眼睛”,有时又显出“很怜悯林纳的样子”。每一次谈话后,林纳总要“大哭一场”(其实在这时,叶群自己也因历史上的问题受到审查)。以后,中组部也参与对林纳的调查,经过反复研究,报经“中央领导同志同意”,才作出了“林纳无问题”的结论\footnote{参见谢燕:《张琴秋的一生》(北京:中国纺织出版社,1995~年),页~184~—~86。建国后,林纳在齐齐哈尔特殊钢厂工作,文革期间,康生、叶群公开点林纳的名,最后林纳惨遭迫害而死。}。

在社会部调查的“嫌疑分子”中,著名作家萧军也榜上有名。萧军来延安后,长期未被分配工作,他被安置在兰家坪招待所,成为一个闲散人员。萧军性格粗犷,初来延安时对纪律严明、等级井然的新秩序颇难适应。由于没有工作在身,萧军经常到桥儿沟鲁艺找朋友聊天,言谈中对延安的生活常有牢骚之语。某次,中央文委负责人艾思奇奉命与肃军谈话,由于话不投机,萧军竟“从怀中掏出了匕首”。萧军的言行立即引起上级的警惕,“一些领导干部”要求社会部尽快拿出一个明确的结论,以确定萧军究竟“是友是敌”,而另一些人则要求社会部从快对萧军作出处理\footnote{参见修来荣:《陈龙传》(北京:群众出版社,1995~年),页~113~—~14、95、117~—~18、97、97~—~98。}。

包括肃军在内的大批嫌疑分子的材料集中在社会部等待鉴别,这项工作量大繁重,而主持调查的社会部治安科人手又少,治安科工作人员陈龙(建国后任公安部副部长)系东北抗联出身,曾在苏联学习,有较高的知识水平和文化素养,对排查“嫌疑分子”的简单化方法持有保留意见。陈龙征得治安科科长汪金祥(建国后任公安部副部长)的同意,两人一起向康生提出改变“反革命嫌疑分子”确定方法的意见。此时正值中央书记处发出《关于调查研究的决定》的前夕,康生接受了陈龙等的建议,将此作为他重视开展调查研究的一项政绩。

在此背景下,1941~年~4~月~10~日,社会部发出《中央社会部关于清理反革命嫌疑分子的指示》,文件提出各地已经“堆积着相当数量的反革命嫌疑案例,没有切实审查”,是因为“各地侦察工作薄弱所致”,另外的原因则“是由于有的除奸同志幼稚,主观夸大,推测附会,捕风捉影,自造了一些所谓的嫌疑分子……”。文件要求在重新审定原有的嫌疑分子时,必须做到“详细研究”和“慎重考察”,“要把真正的反革命嫌疑分子与主观附会、传说自造的反革命嫌疑分子严格分开;把党内错误、思想意识不好或组织关系与历史不清等问题与反革命问题分别清楚”\footnote{参见修来荣:《陈龙传》(北京:群众出版社,1995~年),页~113~—~14、95、117~—~18、97、97~—~98。}。

1941~年春夏,中社部已全面开展清理嫌疑分子的工作,尽快对萧军作出结论,是清理中的一项重要任务。陈龙此时已升任社会部治安科科长,他布置治安科青年干部慕丰韵装扮成从其它根据地来延安的干部,住进兰家坪招待所邻近箫军的塞洞里,就近观察萧军。经过一段时间的观察,慕丰韵发现萧军喜爱京剧,正好慕会拉京胡,就以操京胡伴箫军清唱与箫交上了朋友。萧军毫无城府,“不出几天就对慕丰韵无话不谈”,慕丰韵将所了解到的萧军的思想动态向陈龙和社会部领导汇报后,最终才解除了对萧军政治上的怀疑。\footnote{参见修来荣:《陈龙传》(北京:群众出版社,1995~年),页~113~—~14、95、117~—~18、97、97~—~98。}1941~年~7~月,毛泽东会见了萧军,与他进行了颇为友好的交谈,萧军当然不知道,在此之前社会部已对他进行了这么细致的侦察活动和甄别工作。

萧军是延安的知名人士,得到中央的特别关照,由中社部直接经手对萧军的甄别工作也进行得比较顺利,但对于那些已有工作单位的其他“嫌疑分子”,这项工作的开展就并非一帆风顺。

延安各机关、学校对于中社部提出重新审查嫌疑分子的决定,反应并不一致,有的予以配合,有的则以各种借口加以推诿,甚至认为,保留嫌疑分子没什么不好,“清不清没什么必要”。\footnote{参见修来荣:《陈龙传》(北京:群众出版社,1995~年),页~113~—~14、95、117~—~18、97、97~—~98。}在陈龙、汪金祥的努力下,决定以中社部的名义再发一个文件。1941~年~8~月~2~日,社会部发出《中央社会部关于清理嫌疑分子的指示》第二号,文件分析了各单位清理工作开展缓慢的主要原因是“一、把组织中个别未查清或未解决的个别问题与真正的反革命嫌疑分子混淆起来;二、把通常的、复杂的社会关系与真正的反革命嫌疑分子混淆起来;三、把各种不良现象或倾向与真正的反革命嫌疑分子混淆起来;四、把一般不满言论和牢骚与有意制造破坏混淆起来;五、甚至还有把出于正义感的某些批评与恶意的政治污蔑混淆起来”。“总之,……是把现象当本质,把可能当作事实,把推测附会当作具体事实,不分内外,不分性质,自造了一批所谓的嫌疑分子……”\footnote{参见修来荣:《陈龙传》(北京:群众出版社,1995~年),页~113~—~14、95、117~—~18、97、97~—~98。}。

从中央社会部~1941~年~4~月和~8~月两份文件的提出,可以形成以下几点看法:

一、中社部确有一批政策水平和文化素质皆高的干部,他们因广泛接触各方面情况,视野较为开阔,有的干部对过左的审干方法持有异议。例如,陈龙曾力主排除对萧军的怀疑。在党内形势比较正常的气候下,这些干部会从自己的业务工作的角度出发,向上级机关提出不同意见。

二、中社部负责人康生在一般情况下无法兴风作浪。在较为正常的大气候下,康生也会接受下属的建议,提出慎重处理不同性质矛盾一类的意见,尽管他抱有私心,一心想突出自己,并把下属的成绩记在自己的功劳簿上。

三、1941~年~4~月、8~月的两份中社部文件也存在不足。例如在第一份文件中说,“要把真正的反革命嫌疑分子与主观附会、传说自造的反革命嫌疑分子严格分开”,既然是主观附会、传说自造,就不能再视为是“反革命嫌疑分子”。文件中某些用语措词方面的模糊,在实际贯彻中不可避免将向过左的方面倾斜,从而影响纠偏的进行。

与以后的历次审干运动相比,1940~年的审干是属于比较温和、比较稳妥的,其最重要的一点是在审干方法上没有渗入强制的因素。社会部虽然全面渗入审干,但社会部并不直接主持审干,社会部在这一时期甚至还起着某种中和的作用。正是在社会部的主导下,1941~年春夏开始进行对嫌疑分子的甄别工作,解脱了一批干部。1940~年的审干基本依据“审查干部的材料,主要的依据本人的报告”的原则。没有动用“逼供信”、“车轮战”等手段;在组织部门与被审查干部的关系上,也没有事先假定被审查对象是“特务”的框框;在与被审查对象谈话时,审干人员的态度也较和气,一般并不采用“法官问案式”的谈话方式。这个时候还强调,对新同志的谈话要注意“客气些”,让他们自由地随便地去谈,“务使被召来谈的人不感枯燥而乐于畅谈”。

1940~年审干未酿成严重事件的更重要原因是这一时期党内政治生活还较为正常,主持审干的中组部部长陈云以及负责延安文宣工作的中央书记张闻天等人在审查干部问题上持有比较慎重和实事求是的态度。在对待知识分子问题上,陈云、张闻天持有相当开明的观点,陈云提出中共不仅要“广招天下士”,还要“诚纳四海人”,主张信任和提拔青年知识分子。陈云并参与起草或代中央起草了几份关于吸收知识分子入党的决定。张闻天也强调中共应尊重知识分子的工作和生活特点。陈云认为,审干是必要的,但务必慎重。他们的看法与毛泽东的意见并不一致,而在~1940~年毛毕竟还不能在延安完全决定一切。在他们的影响下,以知识分子为主要对象的~1940~年审干没有采用搞政治运动的方式,也没有事先划定框框,规定一定要排出多少百分比的“叛徒”、“特务”。尽管~1940~年的审干已经包含某些过左的因素,但是在对干部历史问题的估计上,多少还是考虑到“干部是生长在中国错综复杂社会”这层因素,因此在对干部作出政治结论和鉴定时,一般还比较客观。以丁玲为例,丁玲~1933~至~1936~年被国民党软禁在南京的一段历史,在她赴延安后,成为套在她头上的一道紧箍咒,“自首分子”的帽子若隐若现,长期在她的头上浮动。1938~年上半年,康生担任中央党校校长,公开在党校大会上宣布,丁玲“不是我们的同志”,党校不接受丁玲前来学习,\footnote{《延安马列学院回忆录》,页~286。}致使丁玲长期蒙受严重的政治压力。1940~年审干中对丁玲这段历史正式作出结论,明确宣布,丁玲应被视为忠诚的共产党员。

再以王实味为例,王实味在赴延安前曾与托派有联系,在~1940~年审干中,王实味主动向中组部谈出这个问题,事后王实味仍在马列学院工作,他的中共党员的党籍也继续保留。

以后随着党内政治生态环境的恶化,1940~年审干对丁玲、王实味的结论分别在~1957~年和~1942~年全被推翻,不再做数了。

1940~年延安的审干在~1941~年上半年基本结束,然而时隔一年半,从~1943~年初起,一场比~1940~年审干规模不知要大多少倍的新一轮审干又平地掀起,由于这一次审干的规模和范围都远远超过历史上的任何审干,使其有了“审干运动”的名称。

既然~1940~年审干已经结束,1940~年后也没有大量新人进入延安,为何还要兴师动众进行又一轮审干呢?其根本原因是进入~1942~年后,党内的大气候已发生深刻的变化,兼之中共在长期的对国民党的斗争中已经形成某种习惯性的思维,这就是国民党特务无孔不入,任何审干都不可能彻底,总会有漏网之鱼潜伏下来,即使搞了审干,也不足以完全解决问题,唯一的方法就是不断进行审干,对自己的内部进行经常的、无情的洗刷,而共产党就是在与外部和内部敌人的不断斗争中壮大起来的。上述习惯思维早已成为党的性格中的一部分,如果党内政治生活比较正常,它会受到一定的限制,被控制在一定的范围,但是,一旦党内环境发生巨变,极左的敌情估计马上就会占据上风,将原先比较稳妥的审干政策冲得一干二净。1942~年整风之初,延安知识分子批评时政一时蔚为风潮,引致毛泽东的极度警惕,其结果是重新祭起审干肃反的宝器。社会部在~1941~年~4~月、8~月制定的文件,被康生自己废弃一边,重演一遍文件中所列举的各种极左的错误,且比~1940~年更加变本加利。曾经在~1941~年春夏被解除嫌疑的人又被翻了烧饼,问题更是连升几级,从“特嫌”上升为“特务分子”,其中多数人在~1943~年~4~月被秘密逮捕。究其原因,系党内恶化的大气候所致,大气候之形成,其主导者为毛泽东,尽管康生在其中也起到推波助澜的作用。

\section{“整风必然转入审干,审干必然转入反奸(肃反)”}

整风运动与审干运动、抢救运动的关系是研究延安整风历史不可回避的一个重大问题,“整风必然转入审干,审干必然转入反奸(肃反)”\footnote{师哲:《在历史的巨人身边——师哲回忆录》,页~249。}是康生的名言,此话究竟是康生对毛泽东整风部署的蓄意篡改,抑或是他对毛泽东整风意图的正确理解和阐释?换言之,整风运动发展到审干和抢救(反奸、肃反)阶段,是康生一个人的“错误”指导所致,抑或是毛泽东、康生共同规划、共同领导的结果?

1980~年代以来,大陆史学界(包括党史学界)对抢救运动与整风运动之关系有过短时间的探讨,占支配性的意见认为,整风审干是毛泽东正确、英明的决策,抢救运动则是整风运动后期出现的一个插曲,是由康生为破坏整风、蓄意背离毛泽东的部署而擅自发动,且一经出现,很快就被毛泽东所制止,是故,抢救与整风审干无关,抢救运动不能纳入整风的过程,延安整风运动与抢救运动是性质完全不同的两码事\footnote{参见《延安整风与审干运动是性质不同的两个运动》,载《中共党史文摘年刊(1986)(北京:中共党史资料》出版社,1988~年),页~337~—~38。}。

对上述看法作出最具权威性表达的是原中共中央书记处书记、中央宣传部部长邓力群(1942~年中央政治研究室成员)。1991~年~12~月~10~日,邓力群在接受《党的文献》编辑采访时说,抢救运动只“搞了十来天”,以后很快进行“甄别”,“没有一个同志受到冤屈”,“全都做了符合实际的结论”,“实现了同志间没有芥蒂的真诚团结”。\footnote{邓力群:《回忆延安整风》,载《党的文献》,1992~年第~2~期。}显而易见,按照邓力群的上述思路,仅仅搞了十来天的抢救运动非但不能归人延安整风运动之中,甚至连提一下的必要也没有,即使要涉及这个问题,也应该“用历史的发展的眼光”,多从其积极效果方面着眼,因为“没有抢救运动,恐怕就没有九条方针”(指~1943~年~8~月~15~日;毛泽东提出的审查干部的九条方针)\footnote{邓力群:《回忆延安整风》,载《党的文献》,1992~年第~2~期。}。

笔者认为,将抢救运动强行从整风运动中分离开来的观点严重违背了历史事实。邓力群的看法值得商榷,抢救运动并非仅进行“十来天”,所打击的对象更不是“全部都做了符合实际的结论”。至于整风、审干、抢救对党内团结的影响,则是见仁见智,这里暂不作讨论,可是用“坏事变好事”的眼光来评价抢救运动则是很不恰当的。因为,我们绝不能因为有了世界反法西斯统一战线就肯定法西斯运动,同样,我们也绝不能因为中国在八十年代进行改革开放,就肯定“文化大革命”。

整风运动与审干、抢救运动的关系本来并不特别复杂。某些人之所以有意回避、曲解这段历史事实,纯粹是为了政治上的需要。简言之,他们是为了维护毛泽东和其他领导人的形象,而有意将毛泽东等与康生截开,让康生一人扮演魔鬼的角色,由他承担所有的历史责任。

“整风必然转入审干,审干必然转入肃反”,是康生对由毛泽东亲自发动的整风——审干——抢救三运动之有机联系性的准确、客观的体会与描述,毛泽东开动的整风机器就是依照其内在逻辑,沿着整风——审干——抢救的轨迹依次快速递进,而在整个过程中,毛泽东始终处于决策的主导地位。

毛泽东作为延安整风的总策划人,他发动整风的目的和为推行其意图施展的基本策略本身就蕴含整风运动逐步升级的内部动因。

毛泽东发动整风运动的根本目的——彻底肃清国际派在中共的影响,打击和争取以周恩来为代表的“经验主义”者的力量,用自己的思想改造中央,进而确立毛个人在中共党内的绝对统治地位,原本就孕育着可能导致中共分裂的巨大风险,为了避免整风可能带来的这种危险,使即将发生的党内结构的重大改组完全置于自己的控制之下,毛泽东始终小心翼翼,稳扎稳打,绝不轻易冒进。谨慎地施用说教(文的一手)和镇制(武的一手)两种手段,成为毛的基本策略。

文武两手的交替使用并非始于~1942~年,早在~1941~年~9~月,毛泽东决定和王明正式挂牌之际,毛就将此种策略用于中共党内的高级政治生活。一方面,毛在中央政治局扩大会议上主动挑起争论,以“反主观主义”为名,诱使王明集团四分五裂;另一方面,毛又频频向王明显示自己一手控制的中央警卫团的力量,\footnote{参见王明:《中共五十年》,页~160~—~11。}给王明施加压力。1942~年后,康生更加强了对王明、博古的监控,将国际派与中共其他重要干部和驻延安的苏联代表的联系基本切断\footnote{参见弗拉基米洛夫:《延安日记》,页~124。}。

对于~1942~年~2~月揭幕的大规模的全党整风,毛泽东在一个短时期内(从~2~月至~3~月)主要施用“文”的一手(动员学习整风文件,反省思想),但随着~3~月末开始反击王实味,“武”的一手在整风中所占的比重急剧增加。在毛泽东的亲自领导下,经由康生、彭真、李富春等的协助,文武两手互相渗透,互相补充和促进,已经完全渗入整风的过程,并且一直持续到~1945~年中共七大召开。

文武两手在整风运动中所占的比重是灵活而富于弹性的。变化的时机、节奏不仅依据于毛泽东的意志和运动自身发展的规律,而且还受到外界环境的制约。1942~年~11~月,毛泽东在西北局高干会议上提出分清“一条心”和“两条心”,使“文”的一手退隐于“武”的一手之后,从~1942~年~12~月至~1943~年底是整风镇制的一面大显身手的时期。

毛泽东、康生、刘少奇(1942~年底抵延安)、彭真根据整风运动进行中所发生的新的变化,因势利导。先是铺开坦白、审干运动,继审干运动之后,又在延安和各根据地领导了为时近一年的抢救运动(部分单位和地区的抢救及其扫尾工作一直持续到~1945~年)。但是在遭到来自莫斯科的压力和党的核心层内多数成员的消极反对后,毛泽东又审时度势,决定终止抢救,引导整风运动转入“文”的方面——学习中共两条路线斗争历史。文武两手的交替使用,终于使毛泽东的既定目标完全实现。1945~年中共七大正式确立了毛泽东的领袖地位,并决议以毛泽东思想作为中共意识形态的指导思想。至此,针对党内的文武两手遂被搁置,中共的全部力量集中于推翻国民党政权的斗争上。

在毛泽东运用文武两手重建中共的过程中,毛与康生互相支持,互相配合和互相依赖,康生以自己的忠诚和“创造性”的工作全力辅助毛实现自己的政治目标;毛则予以康生特殊的信任,提升和扩大康生在中共党内的影响。毛、康的亲密合作不仅源于彼此充分看重对方,还在于双方在许多重大问题的看法上完全一致,毛对康生的赏识和信任程度远超于当时其它与毛关系接近的中共领导人。因此,在康生遵循毛的意志具体领导整风、审干、抢救的所有重大战役的紧要关头,毛泽东都积极支持康生,为康生开展工作创造有利的环境:

一、1942~年春,毛泽东舍弃较孚众望的任弼时,委派康生出任自己的副手——中央总学委副主任,全面主持整风的日常工作。在毛的支持下,康生领导的中央社会部全力负责延安各重要单位的审干业务,使康生机关的力量急剧膨胀,其特派人员渗透于延安中央各机关、学校和边区一切重要单位。

二、毛泽东在对待王实味问题的看法上也与康生完全合拍,毛对康生处理王实味的措施给予全面肯定。1942~年夏秋开始的审干、反奸试点工作得到毛的充分支持和高度重视。

三、1943~年~4~月~3~日,中宣部颁布第二个“四三决定”,有充分证据说明此决定是由毛泽东参与制定的。该决定强调整风——审干——肃反的必然联系性,全面反映了毛泽东的肃反观——对于奸细、特务,宁可信其有,不可信其无;运动初期必须打击自由主义的右倾思想,大胆怀疑以造成普遍震动;运动后期,则适当纠偏——康生在~1943~年~4~月初的行为完全符合毛的肃反观。4~月后,康生放手大干,毛泽东听之任之,不作任何干预,使抢救野火四处蔓延。

四、1943~年~7~月后,抢救运动形成高潮,毛有意维护康生,虽然在~8~月~15~日颁布审干九条方针,但对落实执行却一反常态,不予强调。结果九条方针颁布后,抢救不仅未停止,反而在更大范围内发展。

五、抢救是在~1943~年~12~月下旬毛泽东接到季米特洛夫干预电报后才真正刹车的,尽管中共其他高级领导人对康生都表示了不满,毛仍竭力保护康生,结果康生有恃无恐,即使当毛向被伤害党员道歉后,康生也拒不作任何自我批评。当然,从康生的角度看,他没有理由承认错误,因为毛从未说整风、审干是错误,所以,即使康生拒不为毛承担责任,毛也无可奈何。好在毛、康两人心中都有数,只是未捅破那一层纸而已。

六、1945~年后,整风中的抢救一幕成为毛泽东最大的禁区之一,毛严禁任何出版物涉及抢救的历史,即使在康生政治上失意的五十年代前期,亦不准语涉康生在抢救运动中的错误\footnote{1955~年~8~月,中共中央在《关于彻底肃清暗藏的反革命分子的指示》中,宣称“延安审干运动,中央订出了九条方针,是完全正确的。审干运动把许多反革命分子和坏分子清查了出来,纯洁了革命队伍,在组织上保证了抗日战争和第三次国内革命战争的胜利,这个成绩是很大的,应该加以充分的估计的”。对于抢救运动,则仅指出“是有偏向的,其结果是犯了逼供信的错误”,但强调此错误在甄别时得到了纠正。参见贺晋:《对延安抢救运动的初步探讨》,引自中国现代史学会编:《中国现代史论文摘编》(郑州:河南人民出版社,1984~年)页~358。}。1967~年~2~月,毛闻知陈毅在怀仁堂中央碰头会上批评四十年代抢救运动时,顿时勃然大怒,不仅将陈毅打人冷宫,还一举废黜了中央政治局。

上述事例只是说明康生所描绘的“整风必然转入审干,审干必然转入反奸(肃反)”并非康生的个人发明,而是毛泽东、康生共同的思路和整风运动发展的客观过程。所谓客观过程,不仅是指整风、审干、抢救的依次递进性和不可逆性,而且也是当年运动开展情况的真实写照。当时担任中共晋绥分局书记的林枫曾对此有过具体描述,他在其所作的关于整风审干的结论中指出:晋绥整风三个时期各有其特点:第一个时期主要是整学风,性质是党内斗争,主要是反对主观主义……第二个时期是反特斗争,这个时期是从党内斗争转到党外斗争。第三时期又开始整风(指“路线学习”——笔者注),成为党内党外两种斗争的汇合。\footnote{《林枫同志在分局高干会关于整风审干的结论》(1944~年~7~月~7~日),转引自郭华伦《中共史论》,第~4~册,页~414。}需要指出的是,在毛泽东施用文武两手彻底改造中共的工程中,除了康生扮演了特别重要的角色外,其他领导人也或多或少起了他们的独特作用。将整风、审干和抢救割裂开来,将全部历史责任推在康生一人身上,都是背离历史真实的虚构。

\section{毛泽东的“肃反”情结}

从“肃~AB~团”、“肃托”到“抢救”,从江西时期到延安时期,和毛泽东直接有关的中共几次内部整肃斗争:“肃~AB~团”、“肃托洛茨基派”(“肃托”)和“抢救运动”,都是以“肃清国民党渗透奸细”、“肃清反革命”和“肃清汉奸托匪”等名目进行的,然而每到运动后期,党的上层都发现,兴师动众开展斗争所取得的实际结果与原有的估计大相径庭:所发现和已被镇压的“敌人”绝大多数都是自己的同志,于是再来进行一番甄别和抚恤工作(但为了维持领导者的“英明”形象,照例保留一批“问题人物”不予解脱)。可是隔不多久,新一轮肃反斗争又在酝酿中,……在毛泽东主政的年代,这已几乎成为一种规律性的现象。

导致残酷的肃反斗争恶性循环的根本原因,一是毛泽东对党内敌情的过份估计,极左的肃反观已形成固定的思考模式;二是毛出于其个人的目的而对“肃反”的误导。

毛泽东的极左的肃反观是中共对国民党屠杀中共政策的激烈反应,以及他个人对国民党特务活动超常估计的产物。1927~年后,为生存而奋斗的中共,长期处在被封锁和剿杀的极端残酷的环境下,作为一种自卫反应,中共和毛习惯将国民党的反共行为给予严重的估计,在对诸如国民党向共产党区域派遣破坏特务,国民党利用“自首政策”胁迫中共人员充当特务等问题上,毛看得尤其严重。在激烈的国共斗争中,毛已形成一种思维定式:即对于国民党在共产党区域的活动,宁可信其有,不可信其无。若从“警惕性”方面而言,在中共高级领导人中间,任何人都未超过毛泽东。且不论王明等人从未执掌过军队和肃反机关,即使作为中共情报肃反机关创始人的周恩来,在对待“敌情”的估计上,也从未像毛泽东那样持如此极端的态度。由于对“敌情”的极端警惕和采取了一系列严密的审查和防范措施,在江西和陕甘宁边区,中共确实挫败了多起国民党针对中共的破坏活动。但是,从总的情况分析,国民党对中共组织所造成的破坏,基本限于国民党统治区域。在中共区域,由于中共组织的高度严密化及对社会的全面和彻底的控制,国民党的渗透几乎不可能,国民党特务活动对中共的危害远小于中共肃反所造成的自相残害的严重程度。

另一方面,利用“肃反”为其政治目标服务,又一直是毛泽东功利主义政治谋略的一个重要组成部分。在谋取个人对中共武装的控制、进而夺取中共最高领导权的长期斗争中,具有极强自信的毛泽东对来自党内的任何异见都予以强烈的排斥,尤其对向他个人权力挑战的举措更是怀有高度的警觉。为了打击党内异己,巩固自己的权力地位,毛在自己能够控制的范围内,善于巧妙利用来自莫斯科的口号和条文,“拉大旗作虎皮”,或自创罪名,将反对派和潜在的反对分子,指为“反革命”。在镇压“AB~团”的过程中,毛发明了“扯起红旗造反”的概念,用来打击党内那些敢于向其权威发起挑战的人。抗战时期,他又放任康生制造“红旗党”冤案,把一大批共产党员打成执行国民党“红旗政策”的“特务”。1966~年文革爆发,毛泽东更是创造出“打着红旗反红旗”的概念,把大批被诬为“叛徒”、“特务”的老党员、老干部投人监狱,将斗争的矛头指向全党和全国人民。尽管毛泽东肃反手法多变,与一味屠杀党内同志的斯大林有明显区别,但两人在利用肃反消灭政敌方面,却有着惊人的相似。

毛泽东运用肃反手段打击党内不同意见,在不同的历史时期,有不同的表现,中共当时所处的环境和毛泽东个人地位之强弱成了决定毛肃反态度变化的基本因素,表现在肃反手段上也有显著的差别。在毛尚未掌全党领导权之前,其肃反手段更直接且更具残酷性;在毛实际已控制中共后,为维护自己作为全党领袖的贤明和公正的形象,他对运用肃反手段处理党内不同意见则稍存谨慎之心,一般多喜施一纵一收之术,以威慑为主,以镇压为辅,且擅长幕后操纵。但在其个人地位完全巩固后,毛的暴戾之态复又重现,对使用肃反威慑手段解决党内问题的兴趣愈来愈浓。

一、苏维埃运动早期(1930~—~1931)。

这个时期江西中共根据地处于极其艰苦的环境下,毛泽东个人在全党的地位不仅还未确立,甚至在根据地内部,毛的领导地位仍然受到党内不同意见的反对。偏于山沟一隅,远离上海中央给毛泽东提供了行动上的充分自主性,为了实现其个人对江西红军的全面控制,毛将“野性一面”(“虎气”)充分发挥,“山大王”的气质不加丝毫约束。结果,由毛泽东直接参与,在赣南造成数千名红军将士和地方共产党员无辜被杀的人间惨剧。\footnote{江西苏区的“肃~AB~团”运动前后历经两个阶段:第一阶段:1930~年“二七”会议后至~1931~年~1~月;第二阶段:1931~年~4~月至~1932~年初。在第一阶段“打~AB~团”的~1930~年~10~月至次年~1~月,毛泽东及其领导的红一方面军总前委在其中发挥了主导作用。据初步统计,在这一阶段,仅红一方面军被杀官兵就达~4,500~人,而至~1930~年~10~月,赣西南特委已消灭“AB~团”份子~1,000~余人,这一数目尚不包括在这之后根据地内党政机构被杀党员的人数。主持江西苏区“AB~团”第二阶段的是以任弼时为首的中央代表团和毛泽东领导的红一方面军总前委,被杀对象主要是参加富田事变的赣西南红军的干部,以及赣西南地方政权的的干部,具体的死亡人数,说法不一。若加上闽西“肃社党”中被杀人数,在中央苏区的肃反惨祸中被杀害的红军官兵和共产党员、普通群众超过一万人。资料来源:一、毛泽东:《总前委答辩的一封信》(1930~年~12~月~20~日)载中国人民解放军政治学院:《中共党史教学参考资料》第~14~册,页~634~—~37;二、《萧克谈中央苏区初期的肃反运动》,载中国革命博物馆编:《党史研究资料》,1982~年第~5~期;三、《江西苏区中共省委工作总结报告》(1932~年~5~月),载江西省档案馆,中共江西省委党校党史教研室编:《中央革命根据地史料选编》,上册,页~477~—~78、480;四、《赣西南会议记录——关于组织问题》,载《中央革命根据地史料选编》,上册,页~631;五、廖盖隆~1981~年~9~月~23~日说:“红一方面军当时在苏区不过三、四万人;前后两次肃反,搞了六千多人,其中一半是杀掉了,就是说,十个红军中有一个被杀掉了,而且差不多都是干部。”廖盖隆在~1980~年~12~月~10~日也引用毛泽东的话:“毛主席说:我们杀了四千五百人,但我们保存了四万红军。”引自中共中央党史资料征集委员会、中央中央党史研究室编:《党史资料通讯》(1981~年合订本)(北京:中共中央党校出版社,1982~年),页~89、144;六、郭华伦:《中共史论》,第~2~册,页~262;七、《闽西“肃清社会民主党”历史冤案已平反昭雪》,载中共中央党史研究室编:《党史通讯》,1986~年第~5~期。}

二、延安前期(1937~—~1941)。

抗战爆发,国共实现第二次合作,使中共所处的环境大大改善,在中共内部,毛泽东也取得了优势地位,并正积极谋取对中共党领导权的全面控制,但正是在这个时刻,莫斯科对中共的影响力也得到恢复。在新的形势下,迫于各种条件的限制,毛开始收敛个性中的“野性的一面”,而有意显示作为全党领袖的气度,对中共核心层的不同意见,一般多用迂回、曲折的方式予以分化、消解。在党内斗争中,主要诉诸政治策略的运用,而较少显示暴力震慑。但是,国共长期兵戎相见造成的警觉意识以及对党内外社会民主主义思想的防范并未有丝毫减退。尽管毛不再、也不能将肃反手段直接用之于党内上层的政治纷争,也没有再重演“肃~AB~团”那样大规模的镇压事件,但对于党内中下层的假想敌却继续沿用肃反手段,这主要表现在~1937~年后,放任康生及其保卫机关在延安及各根据地推行“肃托”的镇压政策。历时数年的“肃托”,其残酷性和血腥性并不亚于~1930~—~1931~年的“肃~AB~团”,只是规模较小,且极端隐蔽。

三、整风时期(1942~—~1945)

随着毛泽东在中共党内领导地位的加强和巩固,毛故态重萌,再一次祭起肃反的宝器。面对党内知识分子大规模的不满,和社会民主主义思想的蔓延,毛迅速决定,将“肃托”、“反特”、“肃奸”正式纳入整风轨道,以便使全党在自己的新权威下彻底就范。毛泽东有意放虎出笼,支持和放纵康生将原处于秘密状态下针对少数重点对象的肃反手段公开施之于党内,造成大量的冤假错案。当然,此时的毛泽东已身为中共领袖,他十分清楚,在延安的干部和党员中,根本不可能有大批“托派”、“国特”和“日特”,因此毛执意在党内开展肃反的最终目的,主要还是在全体党员的心目中植下对自己的崇拜和敬畏,所以毛声明,在审干肃反中执行“一个不杀,大部不抓”的政策,于是,从形式上看,1942~—~1945~年的整风审干运动远没有“肃~AB~团”事件那么残酷。

在长期战争环境下形成的中共肃反政策,由于“宁可信其有,不可信其无”的思维定式,加之毛泽东出于其个人目的,滥用权力,对肃反有意误导,久而久之,致使中共领导人和广大高、中级干部培养成一种用阶级斗争的眼光看待一切的习惯:阶级敌人既可以是“国民党特务”、“暗害分子”,更可以是党内任何有异于当道意见的分子。正是由于有了如此深厚的思想基础,毛泽东每发动一次肃反运动,中共组织和广大党员都闻风而动,积极予以响应,不管有无确凿证据,先挖出一批反革命,运动后期再说。至于这种运动的方式对中共党组织和广大党员的伤害,则是从不予以考虑的。作为极左的肃反政策的始作俑者,毛泽东在运用这一套手法对付他的假想敌人时确已达到得心应手、出神人化的地步。虽然从“肃~AB~团”、“肃托”到“抢救”,毛的操作形式有所不同,但极左的肃反观却是一脉相承,其基本精神并没有任何重大改变。只是毛泽东的角色变幻莫测,忽跃至台前,忽退隐幕后,致使人不易看清毛的真实意图和毛在历次肃反运动中所起的决定性作用。

\section{毛泽东、中央总学委和中央社会部的关系}

自整风运动展开后,中共中央的组织形态也随之出现若干新变化,尽管在与各战略根据地和与重庆南方局的联络来往方面,毛泽东继续以中共中央、中央政治局或中央书记处的名义发布指示,但在延安,中央政治局和书记处的大部分权限已被中央总学委所取代。中共中央大多数部委的职权范围也大大缩小,仅限于维持一般业务工作的水平。在毛泽东的精心策划下,中央总学委实际上已成为凌驾于政治局和书记处之上的中共最高决策和权力机关,尽管毛在~1942~年已完全控制了政治局和书记处,但他还是感到这两个机构有些碍手碍脚,毛要创设一个完全听命于他个人,由他一人支配的组织。然而中央总学委并非是一个固定的实体,它只是毛泽东为掩饰其在党内实行一人统治而设置的临时机构,在整风运动期间,它时而出现,时而又消失的无影无踪——中央总学委的存在及其退隐全凭毛的个人意志决定。1942~年是中央总学委大显神威的时期,在它的核心层仅有毛泽东和康生两人,而康生则完全听命服从于毛。中央总学委的关键部分是由毛、康直接领导的各系统分学习委员会,各系统分学委作为贯彻毛泽东意图的得力工具,在整风初期即发挥了重要作用,但就在各分学委唱主角的同时,长期被宠罩着一层神秘面纱的中央社会部,也从过去所处的幕后一步跃入到前台。

中央社会部“出山”的直接和表面的原因是,各分学委的审干工作超负荷运转已难以为继,急待社会部支持。整风转入审干阶段后,中直系统、军直系统等分学委的工作量急剧增加,各基层单位汇报上来的有关“重点人群”的反省笔记、“小广播表”和个人历史自传以及其它交代和审查材料堆积如山,大量的文字材料需要整理、汇编和核实;各系统的分学委还承担了与被怀疑对象个别谈话、对某些人实行侦讯、布置监控等繁重任务。对于这类具有“特工”性质的工作,经历过江西时期肃反斗争的各分学委的领导,虽然并不陌生,操作起来,一般也得心应手;但是,今天的形势某些特殊的调查已非各分学委独立所能完成:首先,早已不同于江西时期,被审查对象面广量大,互相交叉,几乎覆盖延安各机关、学校,没有社会部的协调,调查几乎不可能进行;其次,审查要求也有所提高,某些审查项目已带有特殊的专业色彩,例如通过电台异地调查、邮检等,使许多分学委的领导同志深感有社会部配合的必要,甚至如何提高审讯工作的效率,也急需专门机关的指导;第三,运动的发展和深化令人目不暇接;一些参加各分学委、原先负责审查别人的干部不久自己就成了被怀疑对象,并被关押和“控制”起来。上述情况使得各分学委普遍感到人手紧张,纷纷向中央总学委告急。正是在这样的背景之下,中共情报肃反的专门机构——中央社会部,被毛泽东、康生引入到领导审干的中心位置。于是,从表面上看,在延安主持整风审干的机关仍然是各分学委,但内里却是中央社会部在掌握、控制和操办一切。

中央社会部在毛泽东的部署下,迅速介入正在展开的审干运动,还有更深层的原因——这就是由康生领导的社会部绝对服从和效忠于毛,是毛完全可以支配的力量。中央社会部的前身是~1931~年~11~月在江西瑞金成立的国家政治保卫局,该局由原上海中共中央特科部分成员和中央苏区原红军干部组成,邓发长期担任局长一职,其副手是与周恩来有较深历史渊源的原上海特科重要干部李克农、潘汉年和李一氓。\footnote{1932~年秋李一氓任国家政治保卫局执行部长,李克农则被调往前方任一方面军保卫局局长。参见李一氓:《模糊的荧屏——李一氓回忆录》,页~147。}在~1932~至~1935~年遵义会议之前,周恩来对国家政治保卫局具有支配性的影响。在周恩来、博古、张闻天有意识的安排下,一些与毛泽东关系密切、曾参与“肃~AB~团”的干部,如曾山、陈正人、古柏等,被安置在地方党政部门或军队工作,因此从~1932~年起,毛泽东对中共肃反机关已不再具有影响力。

在这里需要指出的是,在江西时期,周恩来还直接主管中共的机要情报部门。邓颖超具体承办中共最核心机密——与莫斯科的秘密电讯联络工作。由于处在频繁的战争环境,中共的机要情报、肃反、反间谍工作常常是互相交叉重迭——在长征之前,还有一个大概的分工,即中共中央(苏区中央局)秘书处,中央军委秘书处负责党和军队系统的机要联络,国家政治保卫局主管肃反和情报收集工作。但是在长征前夕,为了军事行动的需要,中共所有的机要情报系统就已全部集中于中央军委机要科,中央秘书处事实上已停止活动,只保留个别工作人员。邓颖超作为中央秘书长,和刘英(以后成为张闻天夫人)一直随博古、李德、周恩来率领的首脑部门转移,主要承办会议记录一类的文书工作,而未和其他妇女干部被安置在“干部休养连”。

毛泽东对周恩来一手掌管机要、情报、肃反部门极为不满,1935~年~1~月遵义会议后,就开始采取行动,逐步蚕食周恩来的领地。1935~年~6~月后,毛泽东亲自派自己的秘书王首道接管中央军委机要科,不久又派王首道接替原由邓发负责的政治保卫局领导一职。尽管~1935~年末召开的瓦窑堡政治局扩大会议宣布恢复中央秘书处和中央各部委,并任命原保卫局干部张文彬为秘书处长,但不久张文彬即被调做统战工作,实权仍控制在副秘书处长王首道手中。

中共中央抵达陕北后,情报肃反部门职责分工的关系开始逐渐理顺。在原国家政治保卫局基础上,先建立方面军保卫局,继而改称西北政府保卫局,由周兴担任局长,受王首道节制。中央军委机要科则一分为三,分别组成中央秘书处机要科、中央军委机要科和方面军保卫局(西北政府保卫局)机要科。中央秘书处机要科负责党务机要电讯;中央军委机要科负责红军军事机要电讯;保卫局机要科负责情报系统电讯和秘密电台的管理。从形式上看,中共已形成了三足鼎立的机要情报系统,然而实际上,所有机要情报大权已完全集中于毛泽东一人之手。因为,这三个系统全归王首道领导,而王首道则对毛绝对服从,深获毛的信任。为了彻底控制中共的机要情报机构,毛还将自己的老部下曾三调入中央秘书处机要科,配合王首道开展工作。与此同时,邓颖超因病远赴北平治疗,完全离开了机要情报系统。

由王首道统掌中共机要、情报、肃反系统的局面在~1937~年~11~月康生返回延安后发生变化。毛泽东经过细心考察,对康生的忠诚、情报专业经验和工作能力作出了肯定的判断,遂在~1938~年~9~月以后。任命康生担任新成立的中央社会部兼情报部部长(中社部正式成立于~1939~年~2~月,但在这之前,实际上已经运作)。但是毛泽东并没有将中共情报、机要、肃反所有大权完全交给康生,康生仅负责肃反和反间谍业务,同时兼管一部分情报业务(1939~年情报部成立半年后,毛即宣布撤消情报部),党务机要、国际通讯等情报业务仍由毛的老部下王首道和王观澜掌管。为了获得更大的权力,进而成为中共情报保卫部门的最高负责人,并在党内占据更重要的地位,1938~年后,康生在全力支持毛泽东对付王明等国际派的同时,竭尽全力加强中央社会部的内部建设,将社会部办成了一个机构齐全、集中大批专业人员、情报网密布全国的中共第一大部。

在康生的主持下,社会部依照苏联格伯乌的结构加强了组织建制,使社会部成了门类齐全的情报反间谍机构。社会部下辖有五个局:一局主管组织、人事;二局主管情报;三局主管反间谍;四局主管情报分析;五局主管特工训练。社会部还有两个直属部门:保卫部和执行部。为了培养派往国统区的特工人员和根据地内的情报、肃反干部,社会部还办有西北公学。

在社会部系统,一批经验丰富的老情报专家:李强(真名叫曾培洪,原中央特科成员,1938~年初从苏联回到延安后被任命为军工局和军委三局副局长,是老资格的电讯专家,其工作与中社部有交叉)、许建国、曾希圣、邹大鹏、冯铉、李士英、罗青长、黄赤波、杨奇清等分别担任了各局室领导职务,少数具有外语、电讯业务知识的青年知识分子干部,例如符浩等也被吸收到各局工作,社会部可以称得上是延安专业人才最集中的单位。

在康生的领导下,社会部在延安广布情报侦察网,触角伸及四面八方。在延安重要的党、政部门和教育系统,都有社会部的特派员和秘密网点,在特派员之下,另有受特派员领导的“网员”,专门从事情报搜集和侦察工作。受社会部直接领导和管理的秘密情报员,一般都具有公开活动的合法身份,他们或是各单位的负责人,或是学校在读的学生。社会部还建立起对延安社会情况的全面监控,在邮局、旅店、饭庄、交通车队、供销合作社等单位都派有专门人员从事情报收集工作。延安著名的“西北旅社”即为社会部所办,社会部干部汪金祥、曲日新都曾担任过“西北旅社”的经理。社会部对前来参观访问的外来人员的监控更是极端重视,主持接待来访人员的专门机构——延安交际处,名义上属边区政府的编制,但边区政府对交际处并无支配力,因为交际处实际是社会部的下属部门。

在毛泽东的首肯下,社会部也将它的工作网络扩伸到中央办公厅和中央军委办公厅(在延安时期,这两个机构是融为一体的),与斯大林身边的工作人员皆属于格伯乌的情况相类似,毛身边的工作人员同时也是社会部成员,不仅叶子龙参加社会部工作,甚至毛的妻子江青在组织关系上也隶属社会部。\footnote{参见王力:《现场历史——文化大革命纪事》(香港:牛津大学出版社,1993~年),页~26。}由此可见,在这一时期,毛对康生领导的社会部在政治上是完全信任的。

毛泽东对在康生领导下的社会部所发生的变化深表满意,康生的凌厉的工作作风和对毛的绝对效忠,给毛留下深刻的印象。为了进一步削弱周恩来在中共情报和肃反系统中的影响,在情报和肃反部门树立起自己的绝对权威,同时也为了彻底弄清机要情报系统干部的历史状况和现实的思想表现,毛决定对康生进一步放权。

1942~年~4~月~4~日,中共中央书记处发出“中央关于成立机要局的通知”,宣布将原先三足鼎立的中央秘书处机要科、中央军委机要科和中央社会部机要科合并为统一的中央机要局,由康生兼任局长。\footnote{中央书记处:《中央关于成立中央机要局的通知》,引自费云东、余贵华:《中共秘书工作简史》(1921~—~1949)(沈阳:辽宁人民出版社,1992~年),页~206、209。此书作者系中央档案馆副研究馆员,该书引用了许多珍贵的档案资料,其中有不少是第一次披露。该书在出版前曾征求过中共机要工作元老王首道和童小鹏的意见。}康生终方全实现了梦寐已求的对中共机要情报系统的控制权,成了名副其实的中共情报和肃反系统的首脑,同时成为中共核心层中仅次于毛泽东的最有影响力的关键人物。

1942~年康生的职务表:

中共中央政治局委员兼中央书记处书记;

中央总学习委员会副主任;

中央党和非党干部审查委员会主任;

中央社会部部长;

中央情报部部长;(1941~年~10~月成立)

中央机要局局长。(1942~年~4~月~4~日成立)

康生在获得中央机要局局长一职后,为了报答毛泽东的信任,在原中央秘书处机要科和军委机要科厉行清洗,将一批资深的机要干部清除出机要系统,使原先三个机要科的二百个工作人员,在机构合并后只剩下九十九人。\footnote{中央书记处:《中央关于成立中央机要局的通知》,引自费云东、余贵华:《中共秘书工作简史》(1921~—~1949)(沈阳:辽宁人民出版社,1992~年),页~206、209。此书作者系中央档案馆副研究馆员,该书引用了许多珍贵的档案资料,其中有不少是第一次披露。该书在出版前曾征求过中共机要工作元老王首道和童小鹏的意见。}在人数减少一半,工作量急剧增加的~1943~年,中央机要局的干部(1942~年~4~月~18~日又易名为中央机要科)超负荷运转,然而这一年却被誉为是“大转变的一年”,并获得毛的高度称赞。

康生身兼数职,直接掌管中共的机要、情报、肃反、反间谍事务,但中央社会部却仍是所有秘密机关的真正核心部门。由于社会部的干部技能全面,经验丰富,很快就取代了合并单位原先的干部,成为各主要科室的骨干。康生并布置社会部干部从事整风审干的新任务。于是,康生机关的势力在~1942~年后迅速膨胀,其工作性质也发生了重大的变化。在毛泽东的支持和关心下,正当中共中央各部委纷纷萎缩时,中央社会部却成了唯一获得大发展的部门,成为毛泽东领导整风审干运动所依靠的核心组织。

中央杜会都插手整风审干运动一般通过下列方式:

一、社会部隐身于总学委和各系统分学委之中,以总学委和各系统分学委的名义开展活动。康生身兼中社部部长和总学委副主任两职,为社会部的渗透活动提供了极大的便利。在更多的情况下,总学委与社会部的关系就如同两块牌子,一个机关那样,几乎很难看出两者的区别。只是在公告有关整风全面部署时,才动用中央总学委的名义。

二、社会部与中央组织部、中央党校分学委等重要部门密切合作,派员直接指导有关单位的审干。康生与中组部部长陈云均是~1931~年~5~月顾顺章事件后改组成立的中央特科负责人,1935~至~1937~年,两人又在莫斯科长期共事,1937~年~11~月同机返回延安。整风运动期间,陈云与康生同属毛泽东所倚重的少数几个中共领导人之列。中央组织部作为党的干部管理和审查的专门机构与中央社会部有着密切的业务联系:中组部负责为社会部选送干部;社会部在侦讯、调查重点怀疑对象时,往往也需要中组部提供有关背景材料;两部门关系一向十分紧密。彭真虽与康生无历史渊源,但从整风之初,两人关系就十分密切。中社部直接派人深入到中央党校各部配合审干,而中央党校“挖出”的重点“反革命”,也大多移送社会部关押。中社部还具体指导关押边区系统嫌疑干部的西北行政学院的审干业务。

中央社会部虽然广泛深人地卷入党内斗争,但是并没有越出毛泽东的控制范围。因为早在~1940~年,社会部派驻各单位的特派员制度就被取消,而代之以新成立的保卫委员会,由各单位的党组织和社会部共同负责考察嫌疑分子的工作。此举改变了社会部原先模仿苏联格伯乌搞的垂直型的情报和侦察制度。整风运动展开后,毛在扩大康生权限的同时,为了防止康生权力的过份发展,从而威胁自己的地位,还采取了一些特殊的防范措施。

首先,毛泽东不允许康生插手他本人与斯大林和共产国际来往的电讯联系系统,而是指定任弼时具体负责此事。王观澜、吴德峰(吴崇宝)、帅孟奇领导的“中央农委”(农村工作部)受毛和任弼时的直接领导。中共其他任何领导人不得过问毛和莫斯科的来往秘电。康生和社会部只是负责绝密电讯系统的外部保卫工作,及对该系统人员的政治审查。尽管毛泽东的俄文翻译师哲属于社会部系统,但师哲作为任弼时的秘书和毛泽东与斯大林来往文电的译员,与康生仅维持一般的工作关系,而绝不向康生透露有关毛和斯大林来往密电的内容。康生则因师哲所处的特殊地位,不断向其示好。

第二,毛泽东在社会部内安排非康系人物作康生的副手,以牵制康生。周恩来由于历史因素,对中共的情报工作,长期承担重要的领导责任。1938~年后,周对情报工作的领导权虽被毛部分转移到康生的手中,但是,因周恩来在毛与王明的斗争中迅速转变立场,加之周的丰富的情报工作经验和从事统战工作的便利条件,毛泽东仍让周恩来分管领导国民党统治区的政治、军事情报及对英、美的国际情报的工作。同时,在社会部内,继续保留大批与周恩来关系密切的情报干部。周恩来的两个老部下:李克农和潘汉年,经毛的同意,也一直担任中央社会部副部长的职务。1941~年~3~月,李克农自重庆返回延安,毛一方面用康生圈住李克农,压抑李在政治上的发展,不让李克农参与自己与王明等争斗的上层核心机密;但在另一方面,又将李克农作为一个经验丰富的情报专家使用,使李克农成为康生在社会部的第一副手。毛的这种安排,虽说并非有意针对康生,但站在康生的角度,毛让一个周恩来的老部下做自己的副手,也包含对康生某种制约的含意。在康生的社会部系统,毛泽东还另外安插了一些与自己有着较深历史渊源的老部下担负重要工作,例如毛指定曾三长期负责中共的核心机要,即使在中央机要局系统一归由康生领导后,曾三仍然是康生不敢轻视的人物。毛让彼此背景不同,来自各个山头的情报干部在社会部内协同工作,使康生永远难以达到“清一色”。

第三,毛泽东在让康生兼管中央军委情报工作的同时,限制康生在军队的情报和保卫部门发展自己的势力。1942~年前后,毛泽东对军方尤其对彭德怀的不满十分强烈,因此毛有意让康生进入毛一向护卫极严的禁区——军队情报系统,以挫伤彭德怀等军中将领的“傲气”。然而,毛又十分掌握分寸,决不允许中央社会部垂直在八路军中建立组织,而是规定由各大战略根据地的党委和军队领导各自的社会部。中央社会部与各根据地和八路军、新四军军中保卫部门的关系,只限于业务指导,这样不仅可以避免军队和党的特工部门的冲突,同时也避免了康生系统尾大不掉的危险。

第四,毛泽东有意给予边区保安处一定的独立工作权力,让社会部与边区保安处形成某种相互制约的关系。边区保安处是延安地区公开的治安管理机构,在形式上隶属边区政府领导,但实际上,边区保安处的真正的上级是中共中央西北局和中央社会部。边区保安处的前身是~1935~年成立的方面军保卫局,1936~年易名为西北政府保卫局,由毛泽东的老部下,当年“肃~AB~团”的活跃分子周兴任局长,1937~年又改名为边区保安处,成为拱卫中共中央的主要机构之一,仍由周兴负责。中央社会部成立之前,边区保安处代行社会部的职能,组成社会部和情报部的工作人员,基本上也是从边区保安处调出,所以保安处与社会部原本是一母所生的关系。但是,由于隶属关系的交叉性和工作分工的差别,保安处开始受到高岗的影响,并逐渐形成了自己的工作范围和干部队伍,已不单纯是社会部的下属机构。边区保安处事实上成为仅次于社会部的第二个保卫系统,而康生对边区保安处并不具有绝对的支配力。

综上所述,康生和中央社会部纯粹是毛泽东棋盘上的一颗棋子,当毛需要借助康生打垮政敌、实现自己的战略意图时,毛就“放虎出山”,授予康生及其机关极大的权力,将社会部的职能扩大为既对外,又对内,使其成为仅服从自己,而凌驾于党的其它部门的太上机关。然而就在社会部最风光的~1942~至~1943~年,毛也留有一手,他要使康生明白,毛永远是他的主人。

\section{在“试验田”里制造出的“张克勤案”}

1942~年夏,根据毛泽东的指示,延安各重要单位的整风领导机构已开始秘密将运动的重心向审干方面倾斜,此时此刻,毛泽东最大的需要不是别的,而是一批从事地下破坏活动的“特务”:为给整风转入审干的必要性提供生动、直观的充足证据,毛需要特务;从“宁可信其有,不可信其无”的极左肃反观出发,毛也相信,延安有大批特务。

1942~年~6~月~19~日,毛泽东在小范围内就审干问题发表重要讲话,他借王实味事件大加发挥,明确指示,要在干部中发现托派、国特和日特三种坏人。毛并暗示审干反特的重点对象是知识分子。毛对审干的策略和方法也作了明确的部署:“要区别好人和犯错误的同志,各机关都要冷静观察,此项工作应有计划的布置”。

然而在毛泽东~6~月~19~日讲话后,延安审干的“战绩”并不明显,也许是毛的“要区别好人和犯错误的同志”的指示发挥了作用,除了破获了中央党校“吴奚如特务案”和“王实味、成全、潘芳、王里、宗铮五人反党集团案”等几个案件外,一时乏善可陈。延安大多数机关和学校都还没有挖出特务的捷报,这种状况引起了毛泽东的强烈不满。10~月~19~日,毛在西北局高干会议开幕词中怒斥对敌特破坏麻木不仁的自由主义。11~月,毛更有意扩大审查范围,指示不仅要查清革命与反革命的“两条心”,还要查清无产阶级与非无产阶级的“半条心”。康生对于毛泽东这种焦灼和愤怒的情绪心领神会,他在经过“冷静观察”和“有计划的布置”后,迅速向毛泽东奉献上一个特务标本——张克勤案。

“张克勤反革命特务案”是康生精心培育的一株“奇葩”,1942~年~9~月后,康生在挖出“王实味五人反党集团”后,即在其直辖领地社会部情报学校——西北公学,开辟了一块审干试验田。

设在延安枣园后沟的西北公学其前身为社会部主办的保卫干部训练班,经常有学生三百多人,1943~年在校学生有五百余人,由社会部副部长兼情报部副部长李克农任校长,社会部第一局局长李逸民任副校长。该校的重要领导成员还有汪东兴(中社部第二室主任)、毛诚(女,延安时代曾任中社部秘书长)和吴德(1942~年被调入西北公学,负有监视李逸民的秘密使命)。\footnote{《李逸民回忆录》(长沙:湖南人民出版社,1986~年),页~112~—~15。}在康生的主持下,1942~年夏秋之际,西北公学日常的情报训练业务活动已经停止,成为秘密审干的试点单位。

张克勤原名樊大畏,原在陕北公学学习,后作为情报侦察干部的候选人员被调入西北公学接受训练。从张克勤各方面的情况看,他被康生看中选为特务标本,实属“理所当然”,因为张克勤太符合一个“特务”所应具备的基本特征,他简直就是康生臆想中的特务世界的一个标准的特务:

在这里,我们暂且站在康生的角度,为张克勤整理出一份政治和历史履历表:

\begin{quoting}
姓名:张克勤

性别:男

年龄:十九岁

政治身份:中共党员

家庭成份:自由职业者

本人成份:学生

何时参加革命工作?何时入党?1936~年~10~月在西安参加民族解放先锋队,抗战爆发后由西安八路军办事处派到兰州做地下工作,并在兰州入党。1939~年~6~月因被国民党怀疑,经中共甘肃工委和中共代表林伯渠介绍调回延安。

主要社会关系及政治面貌:父樊执一,1939~年被国民党逮捕后叛变。妻朱方兰,1939~年被国民党逮捕后叛党。

党组织对张克勤近期表现的考察:

一、1942~年整风期间,延安保健药社接到重庆李炬寄给张克勤的一份国民党刊物——《中央周刊》,据张克勤称,李炬可能是和他妻子鬼混过的特务。

二、西安三青团的报纸曾经刊登过一批共产党自首人员的名单,为首的便是张克勤。

三、延安鲁艺转来一份检举张克勤是特务的材料。
\end{quoting}

这份人事材料对于张克勤实在是太不利了,纵使张克勤满身是嘴,他也澄清不了履历表中所反映的大量疑点,因为按照康生的逻辑:

一、张克勤年仅十九岁,又是一个小资产阶级知识分子,反革命特务最容易从青年知识分子中产生。

二、张克勤来自国统区,国统区的中共组织早已被国民党特务渗透,国民党特务机关对延安也一直采取派遣特务打进来的策略。

三、张克勤家庭背景和社会关系复杂,父、妻皆是叛党特务,张克勤必然受到反动家庭的思想影响,甚至极有可能已经叛党。

四、重庆方面与张克勤迄今仍有通讯联系,寄信人就是特务。

五、西安三青团报纸已刊出张克勤自首的消息,虽然同名同姓的人不少,但难保不是延安的张克勤。

六、有外单位揭发材料,且不论揭发是否属实,揭发本身就说明问题。为什么别人没被揭发,而只揭发张克勤,且揭发人又是与张一同前来延安的。

有了上述六个方面的推理,张克勤已被假定有罪,下一步就是取得当事人的口供,来证实假定了。

1942~年~11~月间,遵照毛泽东审干要“有计划的布置”的秘密指示,在康生、李克农的直接领导下,西北公学审干领导小组几位领导成员:李逸民、吴德、汪东兴、王涛江、毛诚开始在汪东兴办公的窑洞里提审张克勤。

向张克勤提出的第一个问题是“你是怎样来延安的”?张把自己来延安的详细经过叙述一遍。

向张克勤提出的第二个问题是“你来延安干什么”?张陈述自己是由兰州党组织依正常组织手续介绍前来延安学习革命理论的。

向张克勤提出的第三个问题就正式切入主题了,审讯者单刀直人对张克勤说,“已有人揭发你在延安是搞特务的”。张克勤被这突如其来的提问震住了,但他迅速冷静下来,坚决否认指控,并为自己的清白辩解。

审讯者开始向张克勤迂回进攻,他们将主攻方向转移到张克勤的家庭关系——盘问张的父亲(医师)与其病人的关系问题。围绕这个问题,审讯者步步深入,坚持要张克勤承认其父与国民党官员有着政治上的特殊联系。

此时,或许是审讯者并未真正掌握张克勤其父叛变的确凿证据,一时拿不出过硬的材料,更重要的是,张克勤并不知道其父在他赴延安后已叛变的消息,因此双方陷入了僵持局面。入夜,一枝蜡烛早已点完,李逸民和大多数审讯者主张暂停审讯,集中研究下一步的审讯战术。但是汪东兴却援引他在江西中央苏区搞肃反斗争的经验,坚持应连续突击审讯。于是李逸民、吴德、汪东兴等分成两个小组,对张克勤施行“车轮战”,自己轮班休息。可是,直至“第三天天快亮时,蜡烛用完了,但张还未交代”。这时李逸民建议休息一下,给李克农打了一个电话,但未料却遭到李克农的批评。李克农指示,关键时刻已到,应该继续审讯,并派人送来一箱蜡烛。这样,审讯一直坚持到第三天凌晨五时,果然张克勤支持不住,表示愿意坦白了。

在三天三夜轮番“轰炸”下,张克勤的精神终于彻底崩溃。一旦缴械投降,马上就进入到与审讯者密切合作的新阶段,换言之,审讯者要什么,张克勤就提供什么,其主动、积极与几天前的顽抗相比,简直判若两人。

张克勤身上表现出的积极变化,使中央社会部的领导欣喜异常。在康生、李克农的指示下,李逸民、汪东兴等迅即将张克勤“包装”完毕,第二天就召开西北公学全校师生大会,邀请延安各机关、学校、团体代表参加,让张克勤在大会上现身说法。张克勤似乎也具备某种“表演”才能,在会上,他“痛哭流涕地讲着自己如何参加了甘肃假共产党,又如何受派遣来延安搞特务活动”。当然,张克勤只是一具供人摆布、操纵的玩偶,这场活剧的真正导演是中社部的大人物,因为张克勤活剧已具备了上级领导所需要的一切要素:在他的现身说法中,不仅有自己参加特务组织的内容,还有揭发同伙的内容——张克勤一口气交代了十几个“特务”,当然包括那个最先揭发他是特务、与他一同从兰州来延安的朋友。最后,张克勤交代的最精彩的一笔是,他还谈了自己思想转变的过程,张克勤衷心感激党组织对自己的抢救,表示将脱胎换骨,重新做人\footnote{《李逸民回忆录》(长沙:湖南人民出版社,1986~年),页~112~—~15。}。

这样,一个既具特殊性,又有普遍性,兼能体现党之感召力和对自新特务给出路政策的“特务”样板就活灵活现地出现在延安的政治舞台上了。“张克勤特务案”向延安干部和全党敲响了警钟:国民党特务已渗入中共各要害机关,“张克勤”、“李克勤”、“刘克勤”就生活在我们中间。“张克勤案”也给全党一个提示和启发:出身于知识分子,来自于国统区的干部在政治上是最不可靠的,而国统区的中共组织十之八九已被国民党渗入,成了执行国民党“红旗政策”的红皮白心的“红旗党”。

对于康生及其后台,张克勤案的另一重要价值还在于它为在延安和各根据地全面推开肃奸、反特运动提供了具体的工作方法和经验。“张克勤案”的被破获充分说明对于被怀疑对象,事先假定有罪,再运用各种手段取得口供,以证实假定,是克敌制胜、行之有效的方法。在获得口供的过程中,使用诱供、套供、逼供,再辅之以心理感化,任何顽固的堡垒都可以攻克。从破获“张克勤案”中还可以总结出对敌斗争的一条成功经验,这就是办案人员首先必须破除右倾思想,只要肃反干部立场坚定,旗帜鲜明,敢于发扬不怕疲劳、连续作战的精神(车轮战,疲劳战),再狡猾的敌人最后也会缴械投降。

如此观之,“张克勤案”对康生及其后台的好处实在太大,一方面,它为打击周恩来领导的国统区地下党制造了舆论;另一方面,又为毛泽东的“反右倾麻痹”、“开展反特斗争”的论断提供了生动、直观的证据。攻下张克勤,不仅是反特斗争的一个重大战果,而且通过此案还创造出一整套具有普遍指导意义的工作方法和斗争经验,为在更大范围内推广反特斗争提供了可供仿效的样板,同时又可在对敌斗争的第一线培养、锻炼党的肃反保卫干部。果然时隔不久,康生宣布河南党是国民党特务领导的“红旗党”,大后方的四川、云南党也被国民党“红旗政策”所破坏,而延安各机关、学校、团体已在热火朝天地批斗着各自的“张克勤”、“李克勤”、“刘克勤”……!

\section{“抢救”的全面发动与刘少奇进入“反奸”领导核心}

对于康生一手炮制的假案——“张克勤特务案”,毛泽东的反应如何?

据师哲回忆,毛和其他中共领导人都“传阅”过张克勤的口供。可是,毛的态度究竟怎样,师哲没有明说,他只是说,康生对毛泽东“是多少有些影响的”\footnote{师哲:《峰与谷——师哲回忆录》,页~202~—~203。}。

如前所述,站在毛泽东的立场,他是没有任何理由拒绝康生所取得的这项最新成果的,正是由于得到毛的鼓励(师哲声称,对于康生的工作,“党中央和毛主席等领导同志不便轻易开口”),\footnote{师哲:《峰与谷——师哲回忆录》,页~202~—~203。}进入~1943~年后,延安的审干规模迅速扩大。

康生充分认识到张克勤案的价值,现在扩大审干已有了名正言顺的理由,康生对张克勤案作了如下的解释:

\begin{quoting}
(这个案子)使我们对国民党的特务政策有了一个新的认识,对大后方的党组织不能不重新估计,对延安的特务分子数目得到了一个解答,使我们的右倾思想有了一个触动\footnote{王素园:《陕甘宁边区“抢救运动”始末》。载中共中央党史研究室编:《中共党史资料》,第~37~辑(北京:中央党史出版社,1991~年),页~210。}。
\end{quoting}

康生对张克勤案的分析将毛泽东对审干肃奸的指示进一步具体化了。因此,当~1943~年~1~月下旬,中共中央指示康生“搜集审查干部的经验”时,一套符合毛意图的审干经验很快就由康生制造了出来。

1943~年~3~月~16~日,毛泽东在政治局会议上明确提出,整风不仅要整小资产阶级思想,同时也要“整反革命”。毛说,国民党对我党实行特务政策,过去我们招军、招生、招党,招了很多人,难于识别。\footnote{《胡乔木回忆毛泽东》,页~276;《毛泽东年谱》中卷对毛这段话未予反映。}3~月~20~日,康生在中央政治局会议上汇报审干工作,他的发言中心意旨有二:第一、康生提出,抗战以来,国民党对中共普遍实行奸细政策,最近从审查干部中才发现这一政策的阴谋。第二、康生提议,1943~年党的工作,要把审干作为重要的一项,并把延安的审干经验,写成文件通知全国\footnote{参见王秀鑫:《延安“抢救运动”述评》,载《党的文献》,1990~年第~3~期。}。

康生在这次政治局会议上汇报审干工作具有极其重要的意义,此举充分说明毛泽东对审干的高度重视和毛对康生的有力支持。因为这次政治局会议的性质与以往任何一次会议都不同,这是毛泽东正式登上中共中央政治局主席、中央书记处主席职位的特殊日子。在~3~月~20~日政治局会议上,毛终于获得了梦寐以求的在中央核心层执掌“最后决定权”的绝对权力。\footnote{据胡乔木称,1943~年~3~月~20~日政治局会议通过的《关于中央机构调整及精简的决定》中有关书记处会议由主席召集,主席有“最后决定全权”,乃是指书记处处理日常工作的决定之权。政治局决定大政方针,并无哪一个人有最后决定之权的规定。参见《胡乔木回忆毛泽东》,页~273。但事实上,毛泽东根本不管这些区别,而是利用他的双主席的地位,当仁不让地执掌起“最后决定权”。}尽管目前尚未披露详尽资料,使我们无从得知毛泽东在~3~月~20~日政治局会议上对康生汇报的反应,但是,康生在这天会上受到毛的肯定、鼓励和嘉许则是毫无疑问的。

毛泽东的态度可以从刘少奇对康生报告的反应中略见端倪。刘少奇是从苏北新四军根据地经长途跋涉,于~1942~年底抵达延安的。1943~年~1~月~1~日,刘少奇在延安新落成的中央大礼堂新年团拜会上正式亮相,随即作为中共重要领导人在西北局高干会议上作介绍华中工作经验的报告。通过这些安排,刘少奇在中共核心层中的地位迅速突出。

刘少奇甫抵延安,正值审干日趋激烈的时刻,性格谨慎的刘只是静观事态,而未敢深深卷入。但是到了~1943~年~3~月~20~日,刘少奇放胆了。在这一天举行的中央政治局会议上,刘少奇与毛泽东、任弼时三人组成中央书记处,刘并成为中央军委唯一的副主席(周恩来、朱德、彭德怀、王稼祥的军委副主席职务在~1943~—~1944~年不再被提及),刘被毛泽东正式擢升为中共中央第二号人物。就在这一天,刘少奇向华中局陈毅、饶漱石发出《关于警惕国民党特务政策问题》的电报。刘少奇要求新四军和华中根据地也仿效延安,迅速展开审干工作。他指出,“最近延安在整风及全面清查干部思想历史的过程中,发现大批国民党特务与日本特务,……今天国民党向我们斗争的主要方式是特务斗争”\footnote{王秀鑫在《延安“抢救运动”述评》一文中提到刘少奇给陈毅、饶漱石的这份电报于~1943~年~3~月~20~日发出,但《刘少奇年谱》中却提及此电系~6~月~29~日发出。参见王秀鑫:《延安“抢救运动”述评》,载《党的文献》,1990~年第~3~期,页~72;《刘少奇年谱》,上卷,页~425。查刘树发主编的《陈毅年谱》上卷在~1943~年~6~月~29、30~日均无收到此电文的记载,而在~1943~年~3~月~21~日记载中则提到“接中央数电”,其中有对释放韩德勤的意见,至于中央电文的其它内容,则未予反映。经笔者查证,《陈毅年谱》中有关~1943~年~3~月~21~日的记载来源于《赖传珠日记》(赖传珠当时,任新四军参谋长,负责新四军军部和华中局的机要电讯系统),其措词、用语完全一样。在《赖传珠日记》的~1943~年~6~月~29、30~日栏中,均无收到延安来电的记载。参见刘树发主编:《陈毅年谱》,上卷(北京:人民出版社,1995~年),页~405;另参见沈阳军区《赖传珠日记》整理编辑领导小组:《赖传珠日记》(北京:人民出版社,1989~年),页~441。}。

从刘少奇的电报中不难看出:1943~年~3~月~20~日中央政治局会议正式批准了康生的汇报,审干经验已被中央政治局认可,作为中央一项重要政策被推广于全党。

笔者的上述判断,还可以从~1984~年披露的一份~1943~年~3~月的中共中央文件——《中央关于继续开展整风运动的指示》得到证实\footnote{《中央关于继续开展整风运动的指示》(1943~年~3~月),载《文献和研究》,1984~年第~9~期。《文献和研究》杂志在公布这份文件时说明:此文件“没有写明时间,没有发文机关,也不知是否已作为正式文件发出,现有时间是《中共中央文件汇集》的编者判定的”,但说明此文件中“内有毛泽东同志的修改字样”。}。

这份经毛泽东修改的中央指示显然不是准备公开发布的,文件中的措词和涉及的党内斗争策略表明,这份文件是给各级领导机关负责干部的一份党内指示。文件明确提出,日寇和国民党已“派遣了大批内奸分子混人我党”,因而必须“有步骤地清除内奸分子”、“向内奸作斗争”。文件批评党内存在着浓厚的自由主义倾向,强调“现在党内斗争的主要偏向是自由主义,不是过火斗争,故应强调反对前者,不应强调后者”,“因为假若过早地提出防止过火斗争,则势必不能展开纠正错误思想与肃清内奸的斗争”,“也有被内奸分子利用此种号召借以掩藏其内奸面目的危险。”

那么是否马上就应在党内大张旗鼓地开展反自由主义的斗争呢?答案是否定的。虽然自~1942~年末以来延安就结合动员填“小广播表”开展过反自由主义的斗争,但眼下的情况与几个月前相比又有新的发展,这次的反自由主义主要针对的是党内负责干部的“右倾温情”思想,目的是挖出更多的“内奸”。因此这份文件提出“反对自由主义偏向,在党内提出普遍号召的时机,亦不宜过早,因为假若过早地提出此种号召,则错误思想的暴露与内奸面目的暴露都将受影响”。于是,这份中央指示一面大谈党内应克服自由主义倾向问题,在另一方面又要求暂不在全党号召反自由主义,“先让自由主义偏向尽量暴露”\footnote{《中央关于继续开展整风运动的指示》(1943~年~3~月),载《文献和研究》,1984~年第~9~期。《文献和研究》杂志在公布这份文件时说明:此文件“没有写明时间,没有发文机关,也不知是否已作为正式文件发出,现有时间是《中共中央文件汇集》的编者判定的”,但说明此文件中“内有毛泽东同志的修改字样”。}。现在,开展反内奸斗争的目标已经完全确定,“引蛇出洞”的计划也已部署完成,下一步就是按程序具体开展反奸斗争了。

3~月~20~日政治局会议以后,延安的审干形势急剧激化,审干已完全被纳人“反特务”斗争的轨道。而胡公冕此时来延安,恰为运动的升级提供了最合适的借口。

胡公冕,原为中共党员,后投靠国民党,但其思想一向左倾,同情中共,抗战后被胡宗南延聘为高级参谋。1943~年~4~月初,按照事先议定的日程,胡宗南派胡公冕率领一个代表团来延安与中共方面谈判。这原本是一项正常的人员来往,但是他的到访却意外地促成了~4~月~1~日的延安大逮捕。

康生下令捕人的理由似乎名正言顺,这即是蒋介石企图发兵进攻延安,而胡公冕恰在此时来延,为了防止边区内部的“特务”与胡公冕“来往”和“联络”,必须先下手为强。

4~月~1~日夜,在延安共逮捕了多少人?八十年代后期,有关出版物一致声称,约逮捕二百人或二百多人。但是根据师哲提供的数字,4~月~1~日夜逮捕的延安及边区干部超过了四百人,因为就在延安实施大逮捕的同时,边区的其它城镇也同步抓人,绥德专区逮捕了一百人,关中也抓了人(具体数目不详)。

这样,1943~年~4~月~1~日夜,在延安及边区其它城镇共逮捕四百六十人,\footnote{师哲:《峰与谷——师哲回忆录》,页~6。毛泽东在~1942~年~4~月~28~日政治局会议上也说到:今年以来拘捕的特务共有四百人。参见《胡乔木回忆毛泽东》,页~276。}分列如下:

延安(边区系统):二百六十人

延安(中共中央直属机关):一百人

绥德专区:一百人

关中地区:人数不详

这四百六十多人绝大多数是在没有充足证据的情况下被秘密逮捕的,康生说的言简意赅:

“有材料还要审问?”“先抓起来再说,正因为不清楚才关起来审问,审问就是为了弄清问题。”

由于认定嫌疑人员不需任何确凿证据,而仅凭保安机关的主观判断,因而确定被捕人员名单几乎就不费任何周折,完全任由康生个人意志决定了。师哲作为参与逮捕行动的保卫人员,详细地回忆了~1943~年~3~月末,康生圈定被捕人员名单的情形:


\begin{quoting}
康生手里拿着名单,一边同我们谈话,一边在名单上作记号,打圈点,嘴里念叨:这个是“复兴”,这个是“C.C”、“汉奸”、“叛徒”、“日特”……\footnote{师哲:《峰与谷——师哲回忆录》,页~196。}。
\end{quoting}

1943~年~4~月~3~日,就在~4~月~1~日延安大逮捕后的第三天,中共中央发布了旨在进一步扩大审干、反奸的《关于继续开展整风运动的决定》(又称第二个《四三决定》),这是毛泽东正式就任中央政治局、中央书记处主席双职,获有对中共决策“最后决定权”之后,所发布的一个带有全局性影响的重要文件。该决定正式提出整风的目的还在于“肃清党内暗藏的反革命分子”,号召各级党组织“必须极大的提倡民主”,使坏人得到暴露,同时要求大胆怀疑,放手大干,防止运动冷冷清清。文件也稍带几句,在审干反奸中应“注意稳妥”。此种言辞,颇似毛泽东风格,乍看面面俱到,似乎既反右,又防左,实则重点在前,“稳妥”一词,纯系点缀。4~月~5~日,毛泽东主持中央书记处会议,会议决定,号召特务奸细分子自首。\footnote{《毛泽东年谱》,中卷,页~433、434。}4~月~22~日,毛覆信凯丰,同意编印肃奸教育资料。毛特别强调指出,“目前还是让自由主义暴露的时候”\footnote{《毛泽东年谱》,中卷,页~433、434。}。4~月~28~日,毛泽东又在政治局会议上重点谈肃奸问题\footnote{《胡乔木回忆毛泽东》,页~276。《毛泽东年谱》中卷对这一天毛在政治局会议上有关“肃奸”的言论未予反映。}。

第二个《四三决定》的颁布,迅速将延安的反奸斗争引向高潮,实际上成了动员“抢救”的号角,就在这个时刻,刘少奇也在极秘密的状态下进入到领导延安反奸斗争的核心层。

关于刘少奇在延安审干、反奸斗争中扮演何种角色的问题,长期以来一直扑朔迷离,即使在文革期间对刘少奇的“大批判”中,对此问题也没有任何涉及。1980~年刘少奇被平反,大量回忆文章和研究刘少奇的论著纷纷问世,但是,几乎无一篇文字论及刘少奇与延安审干、反奸斗争的关系。1998~年~10~月经官方批准,《刘少奇传》出版。该书在谈到延安审干“出现了扩大化的偏差”时,对刘少奇在其间的作用也只有一句话的描述:“当然,他也要负一定的领导责任”,\footnote{中共中央文献研究室编:《刘少奇传》,上,页~495。}但并未提供任何具体细节。以致于到今天,人们只知道康生在延安整风运动中起了“破坏”作用,对包括刘少奇在内的其他领导人在延安整风运动中的活动均不甚了了。

但是,历史的真实却并非如此,在整风运动期间,刘少奇不仅与康生有密切的工作联系,他还是领导延安审干、反奸斗争的主要负责人。据弗拉基米洛夫的《延安日记》透露,刘在~3~月~28~日与弗氏谈话时,对康生有所不满,并对王明流露出某种同情,但在~4~月~8~日后刘的态度突然转变,开始与康生拉关系了\footnote{参见弗拉基米洛夫:《延安日记》,页~122~—~25。}。

1943~年~3~月~20~日中央书记处改组以后,领导延安的反奸斗争就成为中共核心层的主要任务之一,刘少奇作为书记处第二号人物,参加了书记处部署反奸工作的所有会议。4~月~5~日,书记处举行会议,这次会议决定,号召延安及边区的“失足分子”向党“自首”\footnote{《刘少奇年谱》,上卷,页~419、420、421。}。十天以后,书记处又召开会议,通过了在延安进行清查特务、开展防奸教育的决定。4~月~24~日书记处再次召开会议,决定在~5、6、7~三个月,专门进行防奸教育,并宣布重新恢复总学委,日常事务由康生负责。4~月~28~日,中央政治局开会,讨论党内反奸斗争问题,并决定成立中央反内奸斗争委员会,以刘少奇、康生、彭真、高岗为委员,刘少奇任主任\footnote{《刘少奇年谱》,上卷,页~419、420、421。}。至此,刘少奇成了延安反奸斗争的最高领导人(毛泽东隐身其后),原先具体领导反奸工作的康生也成了刘少奇的部属。

在紧接着展开的大规模反奸、“抢救”运动中,康生冲锋在前,在各种场合频频亮相,刘少奇一般却不公开露面。那么刘少奇又在做什么呢?身为延安反奸斗争主要领导人的刘少奇,其实做的是审干和反奸的政策和策略的制定、谋划工作,故而在各种动员、坦白大会上不见他的踪影。

1943~年~5~月~16~日,刘少奇参加书记处会议,这次会议决定审查内奸。5~月~21~日,政治局召开会议,讨论前一日季米特洛夫发来的有关解散共产国际的电报(莫斯科预备在~5~月~22~日正式公布),会议并讨论了防奸工作,规定了防奸工作的基本原则是:“首长负责;自己动手;调查研究;分清是非轻重;争取失足者。”\footnote{《刘少奇年谱》,上卷,页~419、420、421。}

刘少奇在审干、反奸斗争中将其所擅长的理论与政策分析水平大大地发挥了出来。1943~年~7~月,刘少奇在延安作《关于审干中几个问题的意见的报告》,\footnote{在《刘少奇年谱》中无这次报告具体日期、作报告地点和听报告对象的记载,对报告内容只作了少量反映。}在这个报告中,刘少奇针对干部对审干和反奸斗争的疑虑,对审干和“清查内奸”的意义作了权威性的阐释。刘少奇指出:审干和反奸斗争是整风运动的继续发展,是整风精神在实际工作中的具体运用。刘少奇还明确规定了审干的任务是:“清查内奸,争取失足者,训练干部”\footnote{《刘少奇年谱》,上卷,页~429。另见《延安整风运动纪事》,页~419。}。

刘少奇作为反奸斗争的指导者,不仅在政策制定方面起重要作用,而且在指挥华中根据地的审干反奸和调配审干干部方面也具有很大的权力。刘少奇回到延安后不时就开展反奸斗争的策略给华中发出电报。1943~年~6~月~29~日,刘少奇电示陈毅、饶漱石,告诉他们“大后方的党几乎全部被国民党破坏”,要求彼等针锋相对,“用说服利诱与威胁等办法”,争取被俘将释的“顽方人员”“为我服务,为我作情报”。刘指导道:“采用强迫威逼自首的办法则须选择对象,选择那些在他不肯自首为我作事时,我能长期拘押或秘密枪决的人威逼之,不要对一切人都用威逼办法,可多采用说服利诱办法,或用公布他私人秘密等办法威逼之”。刘少奇认为,“特务斗争是一种非常高级的细密的科学,需要详细研究和学习”,“必须使全党一切忠实党员都学会特务斗争”,否则“我们一定失败无疑”\footnote{刘少奇:《对反特政策与方法指示》(1943~年~6~月~29~日),载中国人民解放军国防大学党史党建政工教研室编:《中共党史教学参考资料》(北京:中国人民解放军国防大学印行,1989~年),第~17~册,页~379。此电文与刘少奇~3~月~20~日电恐非同一电报。也许因该电极机密,《赖传珠日记》中才无记载。}。

1943~年~4~月~5~日,就在那次决定开展动员“失足分子”向党“自首”的书记处会议上,还有另一个与刘少奇有关的重要决定,这就是会议责成由刘少奇负责向晋西北、晋东南、晋察冀派出整风学习组,以指导、帮助上述区域的整风、审干运动。同年~8~月~9~日,书记处会议再次作出决定,由刘少奇选派干部前去华北、华中根据地帮助整风,此时全党范围内的整风已先后进入审干、反奸斗争阶段,指导整风就是指导审干、反奸和“抢救”。在这次会议后,由刘少奇派往各根据地的审干钦差大臣陆续到达各地区。具体传播延安的审干、反奸、抢救经验,对当地运动向极左方向发展起到推波助澜的作用。

刘少奇在审干、反奸斗争中位居领导地位,但他深居简出,外界只知他出任了中央总学委副主任,却不知他实际上还是秘密机构——中央反内奸斗争委员会的主任。

权力极大的中央总学委在~1942~年夏之后,事实上已经取代了政治局和书记处,成为毛一手操纵的太上机关。但是至~1943~年~3~月~20~日书记处改组,在形式上,党的机构已经得到恢复,于是中央总学委暂停工作,可是不久,中央总学委在~4~月~24~日又被恢复。与此差不多同步,4~月~28~日,又正式成立了以刘少奇为首的中央反内奸斗争委员会。既有中央总学委,又有反内奸斗争委员会,这两个机构的功能与分工又是如何呢?事实上,进入~1943~年春之后,中央总学委与反内奸斗争委员会所干的工作是一致的,这就是领导审干与反奸。只是中央反内奸斗争委员会对外严格保密,由总学委在前面出头露面。加之总学委在各机关、学校、军队系统皆有学分会,总学委负责人康生又身兼中社部部长,社会部已与总学委水乳交融,于是外界只知有总学委,作为延安审干、反奸最高决策机构的反内奸斗争委员会反而不为外界所知。因此,才有刘少奇来延安后担任总学委副主任之说的出现。

刘少奇真正担任总学委副主任是在~1943~年~10~月~5~日。在这天召开的书记处会议上,决定毛泽东为总学委主任,刘少奇、康生为副主任。此时总学委的功能又有变化,它实际上是毛泽东为正在召开的以清算王明为目标的政治局扩大会议而成立的核心小圈子,其主要任务是为配合毛在党内上层开展路线斗争,为毛提供理论解释\footnote{在~1943~年~10~月至~1944~年~4~月的政治局扩大会议上,毛泽东、刘少奇、康生果然配合默契,猛攻王明、博古、周恩来。}。至于领导审干、反奸的工作则基本由中央反内奸斗争委员会承担,由于这两个机构的主要负责人都有刘少奇和康生,因此上述两个机构的工作又存在交叉性的关系。

刘少奇在中共党内素有稳健、长于埋头苦干的声誉。在一个相当长的历史时期内,刘少奇因在白区工作中主张隐蔽、退却的意见还蒙受“右倾”的指责,何以他在~1943~年返回延安后不久,就在审干、反奸问题上显出明显的左的姿态,并在~1943~年春夏之交,当延安审干、反奸、抢救狂潮兴起后,听任极左恶浪翻江倒海而基本保持沉默?

笔者认为,刘少奇当时左的姿态与~1943~年他在党内地位的急剧上升有密切的关系。第一、左的审干、反奸政策的始作俑者是毛泽东,刘返抵延安后,很快就了解到这一点。然而毛、刘在反王明问题的立场与利益关系完全一致,毛且大力提拔刘少奇,使刘从政治局候补委员一跃成为中共第二号人物,在这种情况下,刘不管是有意还是违心,都只能全力配合毛。第二、刘少奇与其老部下彭真的关系一向密切,彭真被调回延安后受到毛泽东的重用,在审干、反奸斗争中是一个风云人物,刘少奇来延安后,彭真在中央党校创造出的审干、反奸经验正作为一套成功的经验向延安各机关、学校推广,刘不能站在彭真的对立面。第三、刘少奇虽有谨慎、稳健之名,但刘同时又是一个喜欢表现的人,尤其喜好就某些理论、政策问题发表意见,刘来延安后,被委之以领导审干、反奸的重任,使刘在这一方面的才干可以就此发挥,他也没有理由放弃这一有利于扩大其在全党影响的机会\footnote{据中共中央文献研究室编的《刘少奇传》记载,刘少奇在~1954~年中共七届四中全会期间,曾向会议作自我批评,其中对“1943~年审干期间发生的一些问题”,作了检讨和说明,但该书并没有提供刘检讨的具体内容。参见中共中央文献研究室编:《刘少奇传》,下,页~755。}。

1943~年开始的审干、反奸斗争对刘少奇有极重要的意义,刘少奇从此正式涉足中共的干部和组织系统,在~3~月~20~日书记处会议上。决定成立以刘少奇为书记的中央组织委员会,统一领导中组部(包括中央党务委员会)、统战部、民运工作委员会、海外工作委员会以及中央政治研究室(即中央研究局)。至此,刘少奇取代了陈云(1944~年后转做财经领导工作),成为中共组织系统的掌门人。在审干、反奸斗争中,刘少奇在彭真的大力协助下,通过谈话等方式,广泛熟悉中共各“山头”的情况,刘少奇的影响和在党内的支配力也从北方局系统、新四军系统向全党延伸。在这个时期,刘少奇还直接过问中央党校一部的审干工作,党校一部主任古大存就曾多次向其汇报工作。1944~年,彭真正式出任中组部部长,显示刘少奇在中央组织和干部系统的影响力得到巩固和加强。

虽然刘少奇在审干、反奸斗争中位居决策层,但他与康生毕竟有明显区别。康对毛泽东只是一个家臣,而那时的刘则是毛最重要的盟友,刘少奇不屑于做康生所好的那种“魔鬼”工作,刘少奇的兴趣只是在制定审干、反奸的政策和策略方面,他没有在台前大声疾呼抓“特务”,也没有像康生那样,亲自去审讯“特务”、“叛徒”。有记载说,在运动高潮之际,刘少奇曾对运动的过火方面发出过疑问,\footnote{华世俊、胡育民:《延安整风始末》,页~69。}然而刘的“疑问”只是后话,在~1943~年春夏之交,他还没有这种疑问,相反刘少奇正兴趣盎然地研究、部署审干、反奸的发动工作。现在刘已进入审干、反奸的领导核心,他和毛泽东一样,以党的最高领袖的身份,隐身于康生的背后,在幕后密切关注、并指导在延安,继而在全党开始的新一轮党内大整肃运动。

