%# -*- coding:utf-8 -*-
%%%%%%%%%%%%%%%%%%%%%%%%%%%%%%%%%%%%%%%%%%%%%%%%%%%%%%%%%%%%%%%%%%%%%%%%%%%%%%%%%%%%%


\chapter{西门庆捐金助朋友\KG 常峙节得钞傲妻儿}


诗曰:

\[
清河豪士天下奇,意气相投山可移。
济人不惜千金诺,狂饮宁辞百夜期。
雕盘绮食会众客,吴歌赵舞香风吹。
堂中亦有三千士,他日酬恩知是谁?
\]

话说西门庆留下两个歌童,随即打发苗家人回书礼物,又赏了些银钱。苗实领书,磕头谢了出门。后来不多些时,春燕死了,止春鸿一人,正是:

\[
千金散尽教歌舞,留与他人乐少年。
\]

却说常峙节自那日求了西门庆的事情,还不得到手,房主又日夜催逼。恰遇西门庆从东京回家,今日也接风,明日也接风,一连过了十来日,只不得个会面。常言道:见面情难尽。一个不见,却告诉谁?每日央了应伯爵,只走到大官人门首问声,说不在,就空回了。回家又被浑家埋怨道:“你也是男子汉大丈夫,房子没间住,吃这般懊恼气。你平日只认的西门大官人,今日求些周济,也做了瓶落水。”说的常峙节有口无言,呆瞪瞪不敢做声。到了明日,早起身寻了应伯爵,来到一个酒店内,便请伯爵吃三杯。伯爵道:“这却不当生受。”常峙节拉了坐下,量酒打上酒来,摆下一盘熏肉、一盘鲜鱼。酒过两巡,常峙节道:“小弟向求哥和西门大官人说的事情,这几日通不能会面,房子又催逼的紧,昨晚被房下聒絮了一夜,耐不的。五更抽身,专求哥趁着大官人还没出门时,慢慢的候他。不知哥意下如何?”应伯爵道:“受人之托,必当终人之事。我今日好歹要大官人助你些就是了。”两个又吃过几杯,应伯爵便推早酒不吃了。常峙节又劝一杯,算还酒钱,一同出门,径奔西门庆家里来。

那时,正是新秋时候,金风荐爽。西门庆连醉了几日,觉精神减了几分。正遇周内相请酒,便推事故不去,自在花园藏春坞,和吴月娘、孟玉楼、潘金莲、李瓶儿五个寻花问柳顽耍,好不快活。常峙节和应伯爵来到厅上,问知大官人在屋里,满心欢喜。坐着等了好半日,却不见出来。只见门外书童和画童两个抬着一只箱子,都是绫绢衣服,气吁吁走进门来,乱嚷道:“等了这半日,还只得一半。”就厅上歇下。应伯爵便问:“你爹在那里?”书童道:“爹在园里顽耍哩。”伯爵道:“劳你说声。”两个依旧抬着进去了。不一时,书童出来道:“爹请应二爹、常二叔少待,便来也。”两人又等了一回,西门庆才走出来。二人作了揖,便请坐的。伯爵道:“连日哥吃酒忙,不得些空,今日却怎的在家里?”西门庆道:“自从那日别后,整日被人家请去饮酒,醉的了不的,通没些精神。今日又有人请酒,我只推有事不去。”伯爵道:“方才那一箱衣服,是那里抬来的?”西门庆道:“目下交了秋,大家都要添些秋衣。方才一箱,是你大嫂子的。还做不完,才勾一半哩。”常峙节伸着舌道:“六房嫂子,就六箱了,好不费事!小户人家,一匹布也难得。哥果是财主哩。”西门庆和应伯爵都笑起来。伯爵道:“这两日,杭州货船怎的还不见到?不知买卖货物何如。这几日,不知李三、黄四的银子,曾在府里头开了些送来与哥么?”西门庆道:“货船不知在那里担搁着,书也没捎封寄来,好生放不下。李三、黄四的,又说在出月才关。”应伯爵挨到身边坐下,乘闲便说:“常二哥那一日在哥席上求的事情,一向哥又没的空,不曾说的。常二哥被房主催逼慌了,每日被嫂子埋怨,二哥只麻作一团,没个理会。如今又是秋凉了,身上皮袄儿又当在典铺里。哥若有好心,常言道:救人须救急时无,省的他嫂子日夜在屋里絮絮叨叨。况且寻的房子住着,也是哥的体面。因此,常二哥央小弟特地来求哥,早些周济他罢。”西门庆道:“我曾许下他来,因为东京去,费的银子多了,本待等韩伙计到家,和他理会。如今又恁的要紧?”伯爵道:“不是常二哥要紧,当不的他嫂子聒絮,只得求哥早些便好。”西门庆踌躇了半晌道:“既这等,也不难。且问你,要多少房子才够住?”伯爵道:“他两口儿,也得一间门面、一间客坐、一间床房、一间厨灶——四间房子,是少不得的。论着价银,也得三四个多银子。哥只早晚凑些,教他成就了这桩事罢。”西门庆道:“今日先把几两碎银与他拿去,买件衣服,办些家活,盘搅过来,待寻下房子,我自兑银与你成交,可好么?”两个一齐谢道:“难得哥好心。”西门庆便叫书童:“去对你大娘说,皮匣内一包碎银取了出来。”书童应诺。不一时,取了一包银子出来,递与西门庆。西门庆对常峙节道:“这一包碎银子,是那日东京太师府赏封剩下的十二两,你拿去好杂用。”打开与常峙节看,都是三五钱一块的零碎纹银。常峙节接过放在衣袖里,就作揖谢了。西门庆道:“我这几日不是要迟你的,你又没曾寻的。只等你寻下,待我有银,一起兑去便了。”常峙节又称谢不迭。三个依旧坐下,伯爵便道:“多少古人轻财好施,到后来子孙高大门闾,把祖宗基业一发增的多了。悭吝的,积下许多金宝,后来子孙不好,连祖宗坟土也不保。可知天道好还哩!”西门庆道:“兀那东西,是好动不喜静的,怎肯埋没在一处!也是天生应人用的,一个人堆积,就有一个人缺少了。因此积下财宝,极有罪的。”

正说着,只见书童托出饭来。三人吃毕,常峙节作谢起身,袖着银子欢喜走到家来。刚刚进门,只见浑家闹吵吵嚷将出来,骂道:“梧桐叶落——满身光棍的行货子!出去一日,把老婆饿在家里,尚兀自千欢万喜到家来,可不害羞哩!房子没的住,受别人许多酸呕气,只教老婆耳朵里受用。”那常二只是不开口,任老婆骂的完了,轻轻把袖里银子摸将出来,放在桌儿上,打开瞧着道:“孔方兄,孔方兄!我瞧你光闪闪、响当当无价之宝,满身通麻了,恨没口水咽你下去。你早些来时,不受这淫妇几场气了。”那妇人明明看见包里十二三两银子一堆,喜的抢近前来,就想要在老公手里夺去。常二道:“你生世要骂汉子,见了银子,就来亲近哩。我明日把银子买些衣服穿,自去别处过活,再不和你鬼混了。”那妇人陪着笑脸道:“我的哥!端的此是那里来的这些银子?”常二也不做声。妇人又问道:“我的哥,难道你便怨了我?我也只是要你成家。今番有了银子,和你商量停当,买房子安身却不好?倒恁地乔张致!我做老婆的,不曾有失花儿,凭你怨我,也是枉了。”常二也不开口。那妇人只顾饶舌,又见常二不揪不采,自家也有几分惭愧,禁不得掉下泪来。常二看了,叹口气道:“妇人家,不耕不织,把老公恁地发作!”那妇人一发掉下泪来。两个人都闭着口,又没个人劝解,闷闷的坐着。常二寻思道:“妇人家也是难做。受了辛苦,埋怨人,也怪他不的。我今日有了银子不采他,人就道我薄情。便大官人知道,也须断我不是。”就对那妇人笑道:“我自耍你,谁怪你来!只你时常聒噪,我只得忍着出门去了,却谁怨你来?我明白和你说:这银子,原是早上耐你不的,特地请了应二哥在酒店里吃了三杯,一同往大官人宅里等候。恰好大官人正在家,没曾去吃酒,亏了应二哥许多婉转,才得这些银子到手。还许我寻下房子,兑银与我成交哩!这十二两,是先教我盘搅过日子的。”那妇人道:“原来正是大官人与你的,如今不要花费开了,寻件衣服过冬,省的耐冷。”常二道:“我正要和你商量,十二两纹银,买几件衣服,办几件家活在家里。等有了新房子,搬进去也好看些。只是感不尽大官人恁好情,后日搬了房子,也索请他坐坐是。”妇人道:“且到那时再作理会。”正是:

\[
惟有感恩并积恨,万年千载不生尘。
\]

常二与妇人说了一回,妇人道:“你吃饭来没有?”常二道:“也是大官人屋里吃来的。你没曾吃饭,就拿银子买了米来。”妇人道:“仔细拴着银子,我等你就来。”常二取栲栳望街上买了米,栲栳上又放着一大块羊肉,拿进门来。妇人迎门接住道:“这块羊肉,又买他做甚?”常二笑道:“刚才说了许多辛苦,不争这一些羊肉,就牛也该宰几个请你。”妇人笑指着常二骂道:“狠心的贼!今日便怀恨在心,看你怎的奈何了我!”常二道:“只怕有一日,叫我一万声:‘亲哥,饶我小淫妇罢!’我也只不饶你哩。试试手段看!”那妇人听说,笑的往井边打水去了。当下妇人做了饭,切了一碗羊肉,摆在桌儿上,便叫:“哥,吃饭。”常二道:“我才吃的饭,不要吃了。你饿的慌,自吃些罢。”那妇人便一个自吃了。收了家活,打发常二去买衣服。常二袖着银子,一直奔到大街上来。看了几家,都不中意。只买了一件青杭绢女袄、一条绿绸裙子、一件月白云绸衫儿、一件红绫袄子、一件白绸裙儿,共五件。自家也对身买了一件鹅黄绫袄子、一件丁香色绸直身,又买几件布草衣服。共用去六两五钱银子。打做一包,背到家中,叫妇人打开看看。妇人看了,便问:“多少银子买的?”常二道:“六两五钱银子。”妇人道:“虽没便宜,却值这些银子。”一面收拾箱笼放好,明日去买家活。当日妇人欢天喜地过了一日,埋怨的话都掉在东洋大海里去了,不在话下。

再表应伯爵和西门庆两个,自打发常峙节出门,依旧在厅上坐的。西门庆因说起:“我虽是个武职,恁的一个门面,京城内外也交结许多官员,近日又拜在太师门下,那些通问的书柬,流水也似往来,我又不得细工夫料理。我一心要寻个先生在屋里,教他替写写,省些力气也好,只没个有才学的人。你看有时,便对我说。”伯爵道:“哥,你若要别样却有,要这个倒难。第一要才学,第二就要人品了。又要好相处,没些说是说非,翻唇弄舌,这就好了。若是平平才学,又做惯捣鬼的,怎用的他!小弟只有一个朋友,他现是本州秀才,应举过几次,只不得中。他胸中才学,果然班马之上,就是人品,也孔孟之流。他和小弟,通家兄弟,极有情分。曾记他十年前,应举两道策,那一科试官极口赞好。不想又有一个赛过他的,便不中了。后来连走了几科,禁不的发白鬓斑。如今虽是飘零书剑,家里也还有一百亩田、三四带房子住着。”西门庆道:“他家几口儿也够用了,却怎的肯来人家坐馆?”应伯爵道:“当先有的田房,都被那些大户人家买去了,如今只剩得双手皮哩。”西门庆道:“原来是卖过的田,算什么数!”伯爵道:“这果是算不的数了。只他一个浑家,年纪只好二十左右,生的十分美貌,又有两个孩子,才三四岁。”西门庆道:“他家有了美貌浑家,那肯出来?”伯爵道:“喜的是两年前,浑家专要偷汉,跟了个人,走上东京去了,两个孩子又出痘死了,如今只存他一口,定然肯出来。”西门庆笑道:“恁他说的他好,都是鬼混。你且说他姓甚么?”伯爵道:“姓水,他才学果然无比,哥若用他时,管情书柬诗词,一件件增上哥的光辉。人看了时,都道西门大官人恁地才学哩!”西门庆道:“你都是吊慌,我却不信。你记的他些书柬儿,念来我听,看好时,我就请他来家,拨间房子住下。只一口儿,也好看承的。”伯爵道:“曾记得他捎书来,要我替他寻个主儿。这一封书,略记的几句,念与哥听:

\[
\cipaim{黄莺儿}书寄应哥前,别来思,不待言。满门儿托赖都康健。舍字在边,傍立着官,有时一定求方便。羡如椽,往来言疏,落笔起云烟。”
\]

西门庆听毕,便大笑将起来,道:“他既要你替他寻个好主子,却怎的不捎书来,到写一只曲儿来?又做的不好。可知道他才学荒疏,人品散荡哩。”伯爵道:“这到不要作准他。只为他与我是三世之交,自小同上学堂。先生曾道:‘应家学生子和水学生子一般的聪明伶俐,后来一定长进。”落后做文字,一样同做,再没些妒忌,极好兄弟。故此不拘形迹,便随意写个曲儿。况且那只曲儿,也倒做的有趣。”西门庆道:“别的罢了,只第五句是甚么说话?”白爵道:“哥不知道,这正是拆白道字,尤人所难。‘舍’字在边,旁立着‘官’字,不是个‘馆’字?——若有馆时,千万要举荐。因此说:‘有时定要求方便。’哥,你看他词里,有一个字儿是闲话么?只这几句,稳稳把心窝里事都写在纸上,可不好哩!”西门庆被伯爵说的他恁地好处,到没的说了。只得对伯爵道:“到不知他人品如何?”伯爵道:”他人品比才学又高。前年,他在一个李侍郎府里坐馆,那李家有几十个丫头,一个个都是美貌俊俏的。又有几个伏侍的小厮,也一个个都标致龙阳的。那水秀才连住了四五年,再不起一些邪念。后来不想被几个坏事的丫头小厮,见他似圣人一般,反去日夜括他。那水秀才又极好慈悲的人,便口软勾搭上了。因此,被主人逐出门来,哄动街坊,人人都说他无行。其实,水秀才原是坐怀不乱的。若哥请他来家,凭你许多丫头、小厮,同眠同宿,你看水秀才乱么?再不乱的。”西门庆笑骂道:“你这狗才,单管说慌吊皮鬼混人。前月敝同僚夏龙溪请的先生倪桂岩,曾说他有个姓温的秀才。且待他来时再处。”正是:

\[
将军不好武,稚子总能文。
\]
