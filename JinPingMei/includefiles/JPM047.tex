%# -*- coding:utf-8 -*-
%%%%%%%%%%%%%%%%%%%%%%%%%%%%%%%%%%%%%%%%%%%%%%%%%%%%%%%%%%%%%%%%%%%%%%%%%%%%%%%%%%%%%


\chapter{苗青贪财害主\KG 西门枉法受赃}


诗曰:

\[
怀璧身堪罪,偿金迹未明。龙蛇一失路,虎豹屡相惊。
暂遣虞罗急,终知汉法平。须凭鲁连箭,为汝谢聊成。
\]

话说江南扬州广陵城内,有一苗员外,名唤苗天秀。家有万贯资财,颇好诗礼。年四十岁,身边无子,止有一女尚未出嫁。其妻李氏,身染痼疾在床,家事尽托与宠妾刁氏,名唤刁七儿。原是娼妓出身,天秀用银三百两娶来家,纳为侧室,宠嬖无比。忽一日,有一老僧在门首化缘,自称是东京报恩寺僧,因为堂中缺少一尊镀金铜罗汉,故云游在此,访善纪录。天秀问之,不吝,即施银五十两与那僧人。僧人道:“不消许多,一半足矣。”天秀道:“吾师休嫌少,除完佛像,余剩可作斋供。”那僧人问讯致谢,临行向天秀说道:“员外左眼眶下有一道死气,主不出此年当有大灾。你有如此善缘与我,贫僧焉敢不预先说知。今后随有甚事,切勿出境。戒之戒之。”言毕,作辞而去。

那消半月,天秀偶游后园,见其家人苗青正与刁氏亭侧私语,不意天秀卒至看见,不由分说,将苗青痛打一顿,誓欲逐之。苗青恐惧,转央亲邻再三劝留得免,终是切恨在心。不期有天秀表兄黄美,原是扬州人氏,乃举人出身,在东京开封府做通判,亦是博学广识之人。一日,寄一封书来与天秀,要请天秀上东京,一则游玩,二者为谋其前程。苗天秀得书大喜,因向其妻妾说道:“东京乃辇毂之地,景物繁华,吾心久欲游览,无由得便。今不期表兄书来相招,实慰平生之意。”其妻李氏便说:“前日僧人相你面上有灾厄,嘱咐不可出门。此去京都甚远,况你家私沉重,抛下幼女病妻在家,未审此去前程如何,不如勿往为善。”天秀不听,反加怒叱,说道:“大丈夫生于天地之间,桑弧蓬矢,不能邀游天下,观国之光,徒老死牖下,无益矣。况吾胸中有物,囊有余资,何愁功名不到手?此去表兄必有美事于我,切勿多言!”于是吩咐家人苗青,收拾行李衣装,多打点两箱金银,载一船货物,带了个安童并苗青,上东京。嘱咐妻妾守家,择日起行。

正值秋末冬初之时,从扬州码头上船,行了数日,到徐州洪。但见一派水光,十分阴恶。但见:

\[
万里长洪水似倾,东流海岛若雷鸣,
滔滔雪浪令人怕,客旅逢之谁不惊?
\]
前过地名陕湾,苗员外看见天晚,命舟人泊住船只。也是天数将尽,合当有事,不料搭的船只却是贼船。两个艄子皆是不善之徒:一个名唤陈三,一个乃是翁八。常言道:不着家人,弄不得家鬼。这苗青深恨家主,日前被责之仇一向要报无由,口中不言,心内暗道:“不如我如此这般,与两个艄子做一路,将家主害了性命,推在水内,尽分其财物。我回去再把病妇谋死,这分家私连刁氏,都是我情受的。”正是:

\[
花枝叶下犹藏刺,人心怎保不怀毒。
\]
这苗青于是与两个艄子密密商量,说道:“我家主皮箱中还有一千两金银,二千两缎匹,衣服之类极广。汝二人若能谋之,愿将此物均分。”陈三、翁八笑道:“汝若不言,我等亦有此意久矣。”

是夜天气阴黑,苗天秀与安童在中舱里睡,苗青在橹后。将近三鼓时分,那苗青故意连叫有贼。苗天秀梦中惊醒,便探头出舱外观看,被陈三手持利刀,一下刺中脖下,推在洪波荡里。那安童正要走时,吃翁八一闷棍打落水中。三人一面在船舱内打开箱笼,取出一应财帛金银,并其缎货衣服,点数均分。二艄便说:“我若留此货物,必然有犯。你是他手下家人,载此货物到于市店上发卖,没人相疑。”因此二艄尽把皮箱中一千两金银,并苗员外衣服之类分讫,依前撑船回去了。这苗青另搭了船只,载至临清码头上,钞关上过了,装到清河县城外官店内卸下,见了扬州故旧商家,只说:“家主在后船,便来也。”这个苗青在店发卖货物,不题。

常言:人便如此如此,天理未然未然。可怜苗员外平昔良善,一旦遭其仆人之害,不得好死,虽是不纳忠言之劝,其亦大数难逃。不想安童被一棍打昏,虽落水中,幸得不死,浮没芦港。忽有一只渔船撑将下来,船上坐着个老翁,头顶箬笠,身披短蓑,听得啼哭之声。移船看时,却是一个十七八岁小厮,慌忙救了。问其始末情由,却是扬州苗员外家安童,在洪上被劫之事。这渔翁带下船,取衣服与他换了,给以饮食,因问他:“你要回去,却是同我在此过活?”安童哭道:“主人遭难,不见下落,如何回得家去?愿随公公在此。”渔翁道:“也罢,你且随我在此,等我慢慢替你访此贼人是谁,再作理会。”安童拜谢公公,遂在此翁家过活。

一日,也是合当有事。年除岁末,渔翁忽带安童正出河口卖鱼,正撞见陈三、翁八在船上饮酒,穿着他主人衣服,上岸来买鱼。安童认得,即密与渔翁说道:“主人之冤当雪矣。”渔翁道:“何不具状官司处告理?”安童将情具告到巡河周守备府内。守备见没赃证,不接状子。又告到提刑院。夏提刑见是强盗劫杀人命等事,把状批行了。从正月十四日差缉捕公人,押安童下来拿人。前至新河口,只把陈三、翁八获住到案,责问了口词。二艄见安童在旁执证,也没得动刑,一一招了。供称:“下手之时,还有他家人苗青,同谋杀其家主,分赃而去。”这里把三人监下,又差人访拿苗青,一起定罪。因节间放假,提刑官吏一连两日没来衙门中问事,早有衙门透信的人,悄悄把这件事儿报与苗青。苗青慌了,把店门锁了,暗暗躲在经纪乐三家。

这乐三就住在狮子街韩道国家隔壁,他浑家乐三嫂,与王六儿所交极厚,常过王六儿这边来做伴儿。王六儿无事,也常往他家行走,彼此打的热闹。这乐三见苗青面带忧容,问其所以,说道:“不打紧,间壁韩家就是提刑西门老爹的外室,又是他家伙计,和俺家交往的甚好,几事百依百随,若要保得你无事,破多少东西,教俺家过去和他家说说。”这苗青听了,连忙下跪,说道:“但得我身上没事,恩有重报,不敢有忘。”于是写了说帖,封下五十两银子,两套妆花缎子衣服,乐三教他老婆拿过去,如此这般对王六儿说。王六儿喜欢的要不的,把衣服银子并说帖都收下,单等西门庆,不见来。

到十七日日西时分,只见玳安夹着毡包,骑着头口,从街心里来。王六儿在门首,叫下来问道:“你往那里去来?”玳安道:“我跟爹走了个远差,往东平府送礼去来。”王六儿道:“你爹如今来了不曾?”玳安道:“爹和贲四两个先往家去了。”王六儿便叫进去,和他如此这般说话,拿帖儿与他瞧,玳安道:“韩大婶,管他这事!休要把事轻看了,如今衙门里监着那两个船家,供着只要他哩。拿过几两银子来,也不够打发脚下人哩。我不管别的帐,韩大婶和他说,只与我二十两银子罢。等我请将俺爹来,随你老人家与俺爹说就是了。”王六儿笑道:“怪油嘴儿,要饭吃休要恶了火头。事成了,你的事甚么打紧?宁可我们不要,也少不得你的。”玳安道:“韩大婶,不是这等说。常言:君子不羞当面。先断过,后商量。”王六儿当下备几样菜,留玳安吃酒。玳安道:“吃的红头红脸,怕家去爹问,却怎的回爹?”王六儿道:“怕怎的?你就说在我这里来。”玳安只吃了一瓯子,就走了。王六儿道:“好歹累你,说是我这里等着哩。”

玳安一直来家,交进毡包。等的西门庆睡了一觉出来,在厢房中坐的。这玳安慢慢走到跟前,说:“小的回来,韩大婶叫住小的,要请爹快些过去,有句要紧话和爹说。”西门庆说:“甚么话?我知道了。”说毕,正值刘学官来借银子。打发刘学官去了,西门庆骑马,带着眼纱、小帽,便叫玳安、琴童两个跟随,来到王六儿家。下马进去,到明间坐下,王六儿出来拜见了。那日,韩道国铺子里上宿,没来家。老婆买了许多东西,叫老冯厨下整治。见西门庆来了,慌忙递茶。西门庆吩咐琴童:“把马送到对门房子里去,把大门关上。”妇人且不敢就题此事,先只说:“爹家中连日摆酒辛苦。我闻得说哥儿定了亲事,你老人家喜呀!”西门庆道:“只因舍亲吴大妗那里说起,和乔家做了这门亲事。他家也只这一个女孩儿,论起来也还不般配,胡乱亲上做亲罢了。”王六儿道:“就是和他做亲也好,只是爹如今居着恁大官,会在一处,不好意思的。”西门庆道:“说甚么哩!”说了一回,老婆道:“只怕爹寒冷,往房里坐去罢。”一面让至房中,一面安着一张椅儿,笼着火盆,西门庆坐下。妇人慢慢先把苗青揭帖拿与西门庆看,说:“他央了间壁经纪乐三娘子过来对我说:这苗青是他店里客人,如此这般,被两个船家拽扯,只望除豁了他这名字,免提他。他备了些礼儿在此谢我。好歹望老爹怎的将就他罢。”西门庆看了帖子,因问:“他拿了多少礼物谢你?”王六儿向箱中取出五十两银子来与西门庆瞧,说道:“明日事成,还许两套衣裳。”西门庆看了,笑道:“这些东西儿,平白你要他做甚么?你不知道,这苗青乃扬州苗员外家人,因为在船上与两个船家杀害家主,撺在河里,图财谋命。如今见打捞不着尸首,他原跟来的一个小厮安童与两个船家,当官三口执证着要他。这一拿去,稳定是个凌迟罪名。那两个都是真犯斩罪。两个船家见供他有二千两银货在身上。拿这些银子来做甚么?还不快送与他去!”这王六儿一面到厨下,使了丫头锦儿把乐三娘子儿叫了来,将原礼交付与他,如此这般对他说了去。

那苗青不听便罢,听他说了,犹如一桶水顶门上直灌到脚底下。正是:

\[
惊开六叶连肝肺,唬坏三魂七魄心。
\]
即请乐三一处商议道:“宁可把二千货银都使了,只要救得性命家去。”乐三道:“如今老爹上边既发此言,一些半些恒属打不动。两位官府,须得凑一千货物与他。其余节级、原解、缉捕,再得一半,才得够用。”苗青道:“况我货物未卖,那讨银子来?”因使过乐三嫂来,和王六儿说:“老爹就要货物,发一千两银子货与老爹。如不要,伏望老爹再宽限两三日,等我倒下价钱,将货物卖了,亲往老爹宅里进礼去。”王六儿拿礼帖复到房里与西门庆瞧。西门庆道:“既是恁般,我吩咐原解且宽限他几日,教他即便进礼来。”当下乐三子得此口词,回报苗青,苗青满心欢喜。西门庆见间壁有人,也不敢久坐,吃了几钟酒,与老婆坐了回,见马来接,就起身家去了。

次日,到衙门早发放,也不题问这件事。这苗青就托经纪乐三,连夜替他会了人,撺掇货物出去。那消三日,都发尽了,共卖了一千七百两银子。把原与王六儿的不动,又另加上五十两银子、四套上色衣服。到十九日,苗青打点一千两银子,装在四个酒坛内,又宰一口猪。约掌灯以后,抬送到西门庆门首。手下人都是知道的,玳安、平安、书童、琴童四个家人,与了十两银子才罢。玳安在王六儿这边,梯已又要十两银子。须臾,西门庆出来,卷棚内坐的,也不掌灯,月色朦胧才上来,抬至当面。苗青穿青衣,望西门庆只顾磕头,说道:“小人蒙老爹超拔之恩,粉身碎骨难报。”西门庆道:“你这件事情,我也还没好审问哩。那两个船家甚是攀你,你若出官,也有老大一个罪名。既是人说,我饶了你一死。此礼我若不受你的,你也不放心。我还把一半送你掌刑夏老爹,同做分上。你不可久住,即便星夜回去。”因问:“你在扬州那里?”苗青磕头道:“小的在扬州城内住。”西门庆吩咐后边拿了茶来,那苗青在松树下立着吃了,磕头告辞回去。又叫回来问:“下边原解的,你都与他说了不曾?”苗青道:“小的外边已说停当了。”西门庆吩咐:“既是说了,你即回家。”那苗青出门,走到乐三家收拾行李,还剩一百五十两银子。苗青拿出五十两来,并余下几匹缎子,都谢了乐三夫妇。五更替他雇长行牲口,起身往扬州去了。正是:

\[
忙忙如丧家之狗,急急似漏网之鱼。
\]

不说苗青逃出性命去了。单表次日,西门庆、夏提刑从衙门中散了出来,并马而行。走到大街口上,夏提刑要作辞分路,西门庆在马上举着马鞭儿说道:“长官不弃,到舍下一叙。”把夏提刑邀到家来。进到厅上叙礼,请入卷棚里,宽了衣服,左右拿茶吃了。书童、玳安就安放桌席。夏提刑道:“不当闲来打搅长官。”西门庆道:“岂有此理。”须臾,两个小厮用方盒摆下各样鸡、蹄、鹅、鸭、鲜鱼下饭。先吃了饭,收了家伙去,就是吃酒的各样菜蔬出来。小金钟儿,银台盘儿,慢慢斟劝。饮酒中间,西门庆方题起苗青的事来,道:“这厮昨日央及了个士夫,再三来对学生说,又馈送了些礼在此。学生不敢自专,今日请长官来,与长官计议。”于是,把礼帖递与夏提刑。夏提刑看了,便道:“恁凭长官尊意裁处。”西门庆道:“依着学生,明日只把那个贼人、真赃送过去罢,也不消要这苗青。那个原告小厮安童,便收领在外,待有了苗天秀尸首,归结未迟。礼还送到长官处。”夏提刑道:“长官,这就不是了。长官见得极是,此是长官费心一番,何得见让于我?决然使不得。”彼此推辞了半日,西门庆不得已,还把礼物两家平分了,装了五百两在食盒内。夏提刑下席来,作揖谢道:“既是长官见爱,我学生再辞,显的迂阔了。盛情感激不尽,实为多愧。”又领了几杯酒,方才告辞起身。西门庆随即差玳安拿食盒,还当酒抬送到夏提刑家。夏提刑亲在门上收了,拿回帖,又赏了玳安二两银子,两名排军四钱,俱不在话下。

常言道:火到猪头烂,钱到公事办。西门庆、夏提刑已是会定了。次日到衙门里升厅,那提控、节级并缉捕、观察,都被乐三上下打点停当。摆设下刑具,监中提出陈三、翁八审问情由,只是供称:“跟伊家人苗青同谋。”西门庆大怒,喝令左右:“与我用起刑来!你两个贼人,专一积年在江河中,假以舟楫装载为名,实是劫帮凿漏,邀截客旅,图财致命。见有这个小厮供称,是你等持刀戮死苗天秀波中,又将棍打伤他落水,见有他主人衣服存证,你如何抵赖别人!”因把安童提上来,问道:“是谁刺死你主人?是谁推你在水中?”安童道:“某日三更时分,先是苗青叫有贼,小的主人出舱观看,被陈三一刀戮死,推下水去。小的便被翁八一棍打落水中,才得逃出性命。苗青并不知下落。”西门庆道:“据这小厮所言,就是实话,汝等如何展转得过?”于是每人两夹棍,三十榔头,打的胫骨皆碎,杀猪也似喊叫。一千两赃货已追出大半,余者花费无存。这里提刑做了文书,并赃货申详东平府。府尹胡师文又与西门庆相交,照原行文书叠成案卷,将陈三、翁八问成强盗杀人斩罪。

安童保领在外听候。有日走到东京,投到开封府黄通判衙内,具诉:“苗青夺了主人家事,使钱提刑衙门,除了他名字出来。主人冤仇,何时得报?”通判听了,连夜修书,并他诉状封在一处,与他盘费,就着他往巡按山东察院里投下。这一来,管教苗青之祸从头上起,西门庆往时做过事,今朝没兴一齐来。有诗为证:

\[
善恶从来报有因,吉凶祸福并肩行。
平生不作亏心事,夜半敲门不吃惊。
\]
