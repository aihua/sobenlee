%# -*- coding:utf-8 -*-
%%%%%%%%%%%%%%%%%%%%%%%%%%%%%%%%%%%%%%%%%%%%%%%%%%%%%%%%%%%%%%%%%%%%%%%%%%%%%%%%%%%%%


\chapter{弄私情戏赠一枝桃\KG 走捷径探归七件事}


词曰:

\[
碧桃花下,紫箫吹罢。蓦然一点心惊,却把那人牵挂,向东风泪洒。东风泪洒,不觉暗沾罗帕,恨如天大。那冤家既是无情去,回头看怎么!
\]

话说安童领着书信,辞了黄通判,径往山东大道而来。打听巡按御史在东昌府住扎,姓曾,双名孝序,乃都御史曾布之子,新中乙未科进士,极是个清廉正气的官。这安童自思:“我若说下书的,门上人决不肯放。不如等放告牌出来,我跪门进去,连状带书呈上。老爹见了,必然有个决断。”于是早把状子写下,揣在怀里,在察院门首等候多时。只听里面打的云板响,开了大门,曾御史坐厅。头面牌出来,大书告亲王、皇亲、驸马、势豪之家;第二面牌出来,告都、布、按并军卫有司官吏;第三面牌出来,才是百姓户婚田土词讼之事。这安童就随状牌进去,待把一应事情发放净了,方走到丹墀上跪下。两边左右问是做甚么的,这安童方才把书双手举得高高的呈上。只听公座上曾御史叫:“接上来!”慌的左右吏典下来把书接上去,安放于书案上。曾公拆开观看,端的上面写着甚言词?书曰:

\[
寓都下年教生黄端肃书奉大柱史少亭曾年兄先生大人门下:违越光仪,倏忽一载。知己难逢,胜游易散。此心耿耿,常在左右。去秋忽报瑶章,开轴启函,捧诵之间而神游恍惚,俨然长安对面时也。未几,年兄省亲南旋,复闻德音,知年兄按巡齐鲁,不胜欣慰。叩贺,叩贺。惟年兄忠孝大节,风霜贞操,砥砺其心,耿耿在廊庙,历历在士论。今兹出巡,正当摘发官邪,以正风纪之日。区区爱念,尤所不能忘者矣。窃谓年兄平日抱可为之器,当有为之年,值圣明有道之世,老翁在家康健之时,当乘此大展才猷,以振扬法纪,勿使舞文之吏以挠其法,而奸顽之徒以逞其欺。胡乃如东平一府,而有挠大法如苗青者,抱大冤如苗天秀者乎?生不意圣明之世而有此魍魉。年兄巡历此方,正当分理冤滞,振刷为之一清可也。去伴安童,持状告诉,幸垂察,不宣。\named{仲春望后一日具}
\]

这曾御史览书已毕,便问:“有状没有?”左右慌忙下来问道:“老爷问你有状没有。”这安童向怀中取状递上。曾公看了,取笔批:“仰东平府府官,从公查明,验相尸首,连卷详报。”喝令安童东平府伺候。这安童连忙磕头起来,从便门放出。

这里曾公将批词连状装在封套内,钤了关防,差人赍送东平府来。府尹胡师文见了上司批下来,慌得手脚无措,即调委阳谷县县丞狄斯彬——本贯河南舞阳人氏,为人刚方不要钱,问事糊突,人都号他做狄混。先是这狄县丞往清河县城西河边过,忽见马头前起一阵旋风,团团不散,只随着狄公马走。狄县丞道:“怪哉!”便勒住马,令左右公人:“你随此旋风,务要跟寻个下落。”那公人真个跟定旋风而来,七八将近新河口而止,走来回覆了狄公话。狄公即拘集里老,用锹掘开岸上数尺,见一死尸,宛然颈上有一刀痕。命仵作检视明白,问其前面是那里。公人禀道:“离此不远就是慈惠寺。”县丞即拘寺中僧行问之,皆言:“去冬十月中,本寺因放水灯儿,见一死尸从上流而来,漂入港里。长老慈悲,故收而埋之。不知为何而死。”县丞道:“分明是汝众僧谋杀此人,埋于此处。想必身上有财帛,故不肯实说。”于是不由分说,先把长老一箍两拶,一夹一百敲,余者众僧都是二十板,俱令收入狱中。报与曾公,再行查看。各僧皆称冤不服。曾公寻思道:“既是此僧谋死,尸必弃于河中,岂反埋于岸上?又说干碍人众,此有可疑。”因令将众僧收监。将近两月,不想安童来告此状。即令委官押安童前至尸所,令其认视。安童见尸大哭道:“正是我的主人,被贼人所伤,刀痕尚在。”于是检验明白,回报曾公,即把众僧放回。一面查刷卷宗,复提出陈三、翁八审问,俱执称苗青主谋之情。曾公大怒,差人行牌,星夜往扬州提苗青去了。一面写本参劾提刑院两员问官受赃卖法。正是:

\[
污吏赃官滥国刑,曾公判刷雪冤情。
虽然号令风霆肃,梦里输赢总未真。
\]

话分两头,却表王六儿自从得了苗青干事的那一百两银子、四套衣服,与他汉子韩道国就白日不闲,一夜没的睡,计较着要打头面,治簪环,唤裁缝来裁衣服,从新抽银丝\textuni{4BFC}髻。用十六两银子,又买了个丫头——名唤春香——使唤,早晚教韩道国收用不题。

一日,西门庆到韩道国家,王六儿接着。里面吃茶毕,西门庆往后边净手去,看见隔壁月台,问道:“是谁家的?”王六儿道:“是隔壁乐三家月台。”西门庆吩咐王六儿:“如何教他遮住了这边风水?你对他说,若不与我即便拆了,我教地方吩咐他。”这王六儿与韩道国说:“邻舍家,怎好与他说的。”韩道国道:“咱不如瞒着老爹,买几根木植来,咱这边也搭起个月台来。上面晒酱,下边不拘做马坊,做个东净,也是好处。”老婆道:“呸!贼没算计的。比时搭月台,不如买些砖瓦来,盖上两间厦子却不好?”韩道国道:“盖两间厦子,不如盖一层两间小房罢。”于是使了三十两银子,又盖两间平房起来。西门庆差玳安儿抬了许多酒、肉、烧饼来,与他家犒赏匠人。那条街上谁人不知。

夏提刑得了几百两银子在家,把儿子夏承恩——年十八岁——干入武学肄业,做了生员。每日邀结师友,习学弓马。西门庆约会刘薛二内相、周守备、荆都监、张团练、合卫官员,出人情与他挂轴文庆贺,俱不必细说。

西门庆因坟上新盖了山子卷棚房屋,自从生了官哥,并做了千户,还没往坟上祭祖。叫阴阳徐先生看了,从新立了一座坟门,砌的明堂神路,门首栽桃柳,周围种松柏,两边叠成坡峰。清明日上坟,要更换锦衣牌匾,宰猪羊,定桌面。三月初六日清明,预先发柬,请了许多人,搬运了东西、酒米、下饭、菜蔬,叫的乐工、杂耍、扮戏的。小优儿是李铭、吴惠、王柱、郑奉;唱的是李桂姐、吴银儿、韩金钏,董娇儿。官客请了张团练、乔大户、吴大舅、吴二舅、花大舅、沈姨夫、应伯爵、谢希大、傅伙计、韩道国、云理守、贲第传并女婿陈敬济等,约二十余人。堂客请了张团练娘子、张亲家母、乔大户娘子、朱台官娘子、尚举人娘子、吴大妗子、二妗子、杨姑娘、潘姥姥、花大妗子、吴大姨、孟大姨、吴舜臣媳妇郑三姐、崔本妻段大姐,并家中吴月娘、李娇儿,孟玉楼、潘金莲、李瓶儿、孙雪娥、西门大姐、春梅、迎春、玉箫、兰香、奶子如意儿抱着官哥儿,里外也有二十四五顶轿子。先是月娘对西门庆说:“孩子且不消教他往坟上去罢。一来还不曾过一周,二者刘婆子说这孩子\textuni{29544}门还未长满,胆儿小。这一到坟上路远,只怕唬着他。依着我不教他去,留下奶子和老冯在家和他做伴儿,只教他娘母子一个去罢。”西门庆不听,便道:“此来为何?他娘儿两个不到坟前与祖宗磕个头儿去!你信那婆子老淫妇胡说,可可就是孩子\textuni{29544}门未长满,教奶子用被儿裹着,在轿子里按的孩儿牢牢的,怕怎的?”那月娘便道:“你不听人说,随你。”从清早晨,堂客都从家里取齐,起身上了轿子,无辞。

出南门,到五里外祖坟上,远远望见青松郁郁,翠柏森森,新盖的坟门,两边坡峰上去,周围石墙,当中甬道,明堂、神台、香炉、烛台都是白玉石凿的。坟门上新安的牌匾,大书“锦衣武略将军西门氏先茔”。坟内正面土山环抱,林树交枝。西门庆穿大红冠带,摆设猪羊祭品桌席祭奠。官客祭毕,堂客才祭。响器锣鼓,一齐打起来。那官哥儿唬的在奶子怀里磕伏着,只倒咽气,不敢动一动儿。月娘便叫:“李大姐,你还不教奶子抱了孩子往后边去哩,你看唬的那腔儿!我说且不教孩儿来罢,恁强的货,只管教抱了他来。你看唬的那孩儿这模样!”李瓶儿连忙下来,吩咐玳安:“且叫把锣鼓住了。”连忙撺掇掩着孩儿耳朵,快抱了后边去了。

须臾,祭毕,徐先生念了祭文,烧了纸。西门庆邀请官客在前客位。月娘邀请堂客在后边卷棚内,由花园进去,两边松墙竹径,周围花草,一望无际。正是:

\[
桃红柳绿莺梭织,都是东君造化成。
\]

当下,扮戏的在卷棚内扮与堂客们瞧,四个小优儿在前厅官客席前弹唱。四个唱的,轮番递酒。春梅、玉箫、兰香、迎春四个,都在堂客上边执壶斟酒,就立在大姐桌头,同吃汤饭点心。

吃了一回,潘金莲与玉楼、大姐、李桂姐、吴银儿同往花园里打了回秋千。原来卷棚后边,西门庆收拾了一明两暗三间房儿。里边铺陈床帐,摆放桌椅、梳笼、抿镜、妆台之类,预备堂客来上坟,在此梳妆歇息,糊的犹如雪洞般干净,悬挂的书画,琴棋潇洒。奶子如意儿看守官哥儿,正在那洒金床炕上铺着小褥子儿睡,迎春也在旁和他顽耍。只见潘金莲独自从花园蓦地走来,手中拈着一枝桃花儿,看见迎春便道:“你原来这一日没在上边伺候。”迎春道:“有春梅、兰香、玉箫在上边哩,俺娘叫我下边来看哥儿,就拿了两碟下饭点心与如意儿吃。”奶子见金莲来,就抱起官哥儿来。金莲便戏他说道:“小油嘴儿,头里见打起锣鼓来,唬的不则声,原来这等小胆儿。”于是一面解开藕丝罗袄儿,接过孩儿抱在怀里,与他两个嘴对嘴亲嘴儿。忽有陈敬济掀帘子走入来,看见金莲逗孩子顽耍,便也逗那孩子。金莲道:“小道士儿,你也与姐夫亲个嘴儿。”可霎作怪,那官哥儿便嘻嘻望着他笑。敬济不由分说,把孩子就搂过来,一连亲了几个嘴。金莲骂道:“怪短命,谁家亲孩子,把人的鬓都抓乱了!”敬济笑戏道:“你还说,早时我没错亲了哩。”金莲听了,恐怕奶子瞧科,便戏发讪,将手中拿的扇子倒过柄子来,向他身上打了一下,打的敬济鲫鱼般跳。骂道:“怪短命,谁和你那等调嘴调舌的!”敬济道:“不是,你老人家摸量惜些情儿。人身上穿着恁单衣裳,就打恁一下!”金莲道:“我平自惜甚情儿?今后惹着我,只是一味打。”如意儿见他顽的讪,连忙把官哥儿接过来抱着,金莲与敬济两个还戏谑做一处。金莲将那一枝桃花儿做了一个圈儿,悄悄套在敬济帽子上。走出去,正值孟玉楼和大姐、桂姐三个从那边来。大姐看见,便问:“是谁干的营生?”敬济取下来去了,一声儿也没言语。堂客前戏文扮了四大折。但见:

\[
窗外日光弹指过,席前花影座间移。
\]

看看天色晚来,西门庆吩咐贲四,先把抬轿子的每人一碗酒、四个烧饼、一盘子熟肉,分散停当,然后,才把堂客轿子起身。官家起马在后,来兴儿与厨役慢慢的抬食盒煞后。玳安、来安、画童、棋童儿跟月娘众人轿子,琴童并四名排军跟西门庆马。奶子如意儿独自坐一顶小轿,怀中抱着哥儿,用被裹得紧紧的进城。月娘还不放心,又使回画童儿来,叫他跟定着奶子轿子,恐怕进城人乱。

且说月娘轿子进了城,就与乔家那边众堂客轿子分路,来家先下轿进去,半日西门庆、陈敬济才到家下马。只见平安儿迎门就禀说:“今日掌刑夏老爹,亲自下马到厅,问了一遍去了。落后又差人问了两遍。不知有甚勾当。”西门庆听了,心中犹豫。到于厅上,只见书童儿在旁接衣服。西门庆因问:“今日你夏老爹来,留下甚么话来?”书童道:“他也没说出来,只问爹往那去了:‘使人请去,我有句要紧话儿说。’小的便道:‘今日都往坟上烧纸去了,至晚才来。’夏老爹说:‘我到午上还来。’落后又差人来问了两遭,小的说:‘还未来哩!’”西门庆心下转道:“却是甚么?”

正疑惑之间,只见平安来报:“夏老爹来了。”那时已有黄昏时分,只见夏提刑便衣坡巾,两个伴当跟随。下马到于厅上叙礼,说道:“长官今日往宝庄去来?”西门庆道:“今日先茔祭扫,不知长官下降,失迎,恕罪,恕罪!”夏提刑道:“有一事敢来报与长官知道。”因说:“咱们往那边客位内坐去罢。”西门庆令书童开卷棚门,请往那里说话,左右都令下去。夏提刑道:“今朝县中李大人到学生那里,如此这般,说大巡新近有参本上东京,长官与学生俱在参例。学生令人抄了个底本在此,与长官看。”西门庆听了,大惊失色,急接过邸报来灯下观看,端的上面写着甚言词?

\[
巡按山东监察御史曾孝序一本,参劾贪肆不职武官,乞赐罢黜,以正法纪事:臣闻巡搜四方,省察风俗,乃天子巡狩之事也;弹压官邪,振扬法纪,乃御史纠政之职也。昔《春秋》载天王巡狩,而万邦怀保,民风协矣,王道彰矣,四民顺矣,圣治明矣。臣自去年奉命巡按山东齐鲁之邦,一年将满,历访方面有司文武官员贤否,颇得其实。兹当差满之期,敢不循例甄别,为我皇上陈之!除参劾有司方面官员,另具疏上请。参照山东提刑所掌刑金吾卫正千户夏延龄,\textuni{26D91}茸之材,贪鄙之行,久于物议,有玷班行。昔者典牧皇畿,大肆科扰,被属官阴发其私。今省理山东刑狱,复著狼贪,为同僚之箝制。纵子承恩冒籍武举,倩人代考,而士风扫地矣。信家人夏寿监索班钱,被军腾詈而政事不可知乎!接物则奴颜婢膝,时人有丫头之称;问事则依违两可,群下有木偶之诮。理刑副千户西门庆,本系市井棍徒,夤缘升职,滥冒武功,菽麦不知,一丁不识。纵妻妾嬉游街巷而帷薄为之不清;携乐妇而酣饮市楼,官箴为之有玷。至于包养韩氏之妇,恣其欢淫,而行检不修;受苗青夜赂之金,曲为掩饰,而赃迹显著。此二臣者,皆贪鄙不职,久乖清议,一刻不可居任者也。伏望圣明垂听,敕下该部,再加详查。如果臣言不谬,将延龄等亟赐罢斥,则官常有赖而俾圣德永光矣。
\]

西门庆看了一遍,唬的面面相觑,默默不言。夏提刑道:“长官,似此如何计较?”西门庆道:“常言:兵来将挡,水来土掩。事到其间,道在人为。少不的你我打点礼物,早差人上东京央及老爷那里去。”于是,夏提刑急急作辞,到家拿了二百两银子、两把银壶。西门庆这里是金镶玉宝石闹妆一条、三百两银子。夏家差了家人夏寿,西门庆这里是来保,将礼物打包端正,西门庆写了一封书与翟管家,两个早雇了头口,星夜往东京干事去了,不题。

且表官哥儿自从坟上来家,夜间只是惊哭,不肯吃奶。但吃下奶去就吐了。慌的李瓶儿走来告诉月娘,月娘道:“我那等说,还未到一周的孩子,且休带他出城门去。浊漒货他生死不依,只说:‘今日坟上祭祖为甚么来?不教他娘儿两个走走!’只象那里搀了分儿一般,睁着眼和我两个叫。如今却怎么好?”李瓶儿正没法儿摆布。况西门庆又因巡按参了,和夏提刑在前边说话,往东京打点干事,心上不遂,家中孩子又不好。月娘使小厮叫婆子来看,又请小儿科太医,开门阖户,乱了一夜。刘婆子看了说:“哥儿着了些惊气入肚,又路上撞见五道将军。不打紧,买些纸儿退送退送就好了。”又留了两服朱砂丸药儿,用薄荷灯心汤送下去,那孩儿方才宁贴睡了一觉,不惊哭吐奶了。只是身上热还未退,李瓶儿连忙拿出一两银子,教刘婆子备纸去。后又带了他老公,还和一个师婆来,在卷棚内与哥儿烧纸跳神。那西门庆早五更打发来保、夏寿起身,就乱着和夏提刑往东平府胡知府那里打听提苗青消息去了。吴月娘听见刘婆说孩子路上着了惊气,甚是抱怨如意儿,说他:“不用心看孩儿,想必路上轿子里唬了他了。不然,怎的就不好起来?”如意儿道:“我在轿子里,将被儿包得紧紧的,又没(石店)着他。娘叫画童儿来跟着轿子,他还好好的,我按着他睡。只进城七八到家门首,我只觉他打了个冷战,到家就不吃奶,哭起来了。”

按下这里家中烧纸,与孩子下神。且说来保、夏寿一路攒行,只六日就赶到东京城内。到太师府内见了翟管家,将两家礼物交割明白。翟谦看了西门庆书信,说道:“曾御史参本还未到哩,你且住两日。如今老爷新近条陈了七件事,旨意还未曾下来。待行下这个本去,曾御史本到,等我对老爷说,交老爷阁中只批与他‘该部知道’。我这里差人再拿帖儿吩咐兵部余尚书,把他的本只不覆上来。交你老爹只顾放心,管情一些事儿没有。”于是把二人管待了酒饭,还归到客店安歇,等听消息。

一日蔡太师条陈本,圣旨准下来了。来保央府中门吏暗暗抄了个邸报,带回家与西门庆瞧,不在话下。一日等的翟管家写了回书,与了五两盘缠,与夏寿取路回山东清河县。来到家中,西门庆正在家耽心不下,那夏提刑一日一遍来问信。听见来保二人到了,叫至后边问他端的。来保对西门庆悉把上项事情诉说一遍,道:“翟爹看了爹的书,便说:‘此事不打紧,教你爹放心。见今巡按也满了,另点新巡按下来了。况他的参本还未到,等他本上时,等我对老爷说了,随他本上参的怎么重,只批该部知道,老爷这里再拿帖儿吩咐兵部余尚书,只把他的本立了案不覆上去,随他有拨天关本事也无妨。’”西门庆听了,方才心中放下。因问:“他的本怎还不到?”来保道:“俺们一去时,昼夜马上行去,只五日就赶到京中,可知在他头里。俺每回来,见路上一簇响铃驿马,背着黄色袱,插着两根雉尾、两面牙旗,怕不就是巡按衙门进送实封才到了。”西门庆道:“得他的本上的迟,事情就停当了。我只怕去迟了。”来保道:“爹放心,管情没事。小的不但干了这件事,又打听得两桩好事来,报爹知道。”西门庆问道:“端的何事?”来保道:“太师老爷新近条陈了七件事,旨意已是准行。如今老爷亲家户部侍郎韩爷题准事例:在陕西等三边开引种盐,各府州郡县设立义仓,官粜粮米。令民间上上之户赴仓上米,讨仓钞,派给盐引支盐。旧仓钞七分,新仓钞三分。咱旧时和乔亲家爹,高阳关上纳的那三万粮仓钞,派三万盐引,户部坐派。如今蔡状元又点了两淮巡盐,不日离京,倒有好些利息。”西门庆听言问道:“真个有此事?”来保道:“爹不信,小的抄了个邸报在此。”向书箧中取出来与西门庆观看。因见上面许多字样,前边叫了陈敬济来念与他听。陈敬济念到中间,只要结住了,还有几个眼生字不认的。旋叫了书童儿来念。那书童倒还是门子出身,荡荡如流水不差,直念到底。端的上面奏着那七件事?

\[
崇政殿大学士吏部尚书鲁国公蔡京一本,为陈愚见,竭愚衷,收人才,臻实效,足财用,便民情,以隆圣治事:
第一曰罢科举,取士悉由学校升贡。窃谓教化凌夷,风俗颓败,皆由取士不得真才,而教化无以仰赖。《书》曰:“天生斯民,作之君,作之师。”汉举孝廉,唐兴学校,我国家始制考贡之法,各执偏陋,以致此辈无真才,而民之司牧何以赖焉?今皇上寤寐求才,宵旰图治。治在于养贤,养贤莫如学校。今后取士,悉遵古由学校升贡。其州县发解礼闱,一切罢之。每岁考试上舍则差知贡举,亦如礼闱之式。仍立八行取士之科。八行者,谓孝、友、睦、姻、任、恤、忠、和也。士有此者,即免试,率相补太学上舍。
二曰罢讲议财利司。窃惟国初定制,都堂置讲议财利司。盖谓人君节浮费,惜民财也。今陛下即位以来,不宝远物,不劳逸民,躬行节俭以自奉。盖天下亦无不可返之俗,亦无不可节之财。惟当事者以俗化为心,以禁令为信,不忽其初,不弛其后,治隆俗美,丰亨豫大,又何讲议之为哉?悉罢。
三曰更盐钞法。窃惟盐钞,乃国家之课以供边备者也。今合无遵复祖宗之制盐法者。诏云中、陕西、山西三边,上纳粮草,关领旧盐钞,易东南淮浙新盐钞。每钞折派三分,旧钞搭派七分。今商人照所派产盐之地下场支盐。亦如茶法,赴官秤验,纳息请批引,限日行盐之处贩卖。如遇过限,并行拘收;别买新引增贩者,俱属私盐。如此则国课日增,而边储不乏矣。
四曰制钱法。窃谓钱货,乃国家之血脉,贵乎流通而不可淹滞。如有厄阻淹滞不行者,则小民何以变通,而国课何以仰赖矣?自晋末鹅眼钱之后,至国初琐屑不堪,甚至杂以铅铁夹锡。边人贩于虏,因而铸兵器,为害不小,合无一切通行禁之也。以陛下新铸大钱崇宁、大观通宝,一以当十,庶小民通行,物价不致于踊贵矣。
五曰行结粜俵籴之法。窃惟官籴之法,乃赈恤之义也。近年水旱相仍,民间就食,上始下赈恤之诏。近有户部侍郎韩侣题覆钦依:将境内所属州县各立社会,行结粜俵籴之法。保之于党,党之于里,里之于乡,倡之结也。每乡编为三户,按上上、中中、下下。上户者纳粮,中户者减半,下户者退派粮数关支,谓之俵粜。如此则敛散便民之法得以施行,而皇上可广不费之仁矣。惟责守令核切举行,其关系盖匪细矣。
六曰诏天下州郡纳免夫钱。窃惟我国初寇乱未定,悉令天下军徭丁壮集于京师,以供运馈,以壮国势。今承平日久,民各安业,合颁诏行天下州郡,每岁上纳免夫钱,每名折钱三十贯,解赴京师,以资边饷之用。庶两得其便,而民力少苏矣。
七曰置提举御前人船所。窃惟陛下自即位以来,无声色犬马之奉。所尚花石,皆山林间物,乃人之所弃者。但有司奉行之过因而致扰,有伤圣治。陛下节其浮滥,仍请作御前提举人船所。凡有用悉出内帑,差官取之,庶无扰于州郡。伏乞圣裁。
奉旨曰:“卿言深切时艰,朕心嘉悦,足见忠猷,都依拟行。”该部知道。
\]

西门庆听了,又看了翟管家书信,已知礼物交得明白。蔡状元见朝,又点了两淮巡盐,不日往此经过,心中不胜欢喜。一面打发夏寿回家:“报与你老爹知道。”一面赏了来保五两银子、两瓶酒、一方肉,回房歇息,不在话下。正是:树大招风风损树,人为名高名丧身。有诗为证:

\[
得失荣枯命里该,皆因年月日时栽。
胸中有志终须至,囊内无财莫论才。
\]
