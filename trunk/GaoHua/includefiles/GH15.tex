%# -*- coding:utf-8 -*-
%%%%%%%%%%%%%%%%%%%%%%%%%%%%%%%%%%%%%%%%%%%%%%%%%%%%%%%%%%%%%%%%%%%%%%%%%%%%%%%%%%%%%

\chapter{“毛主席万岁”——延安整风的完成}

\section{“毛泽东主义”的提出与修正}

1938~年~10~月中共六届六中全会后,毛泽东成为中共第一号人物,在中共领导核心中的地位已经牢牢树立,但是毛泽东的“理论家”名号却是在数年后才确定的。从~1940~年始,在延安的《解放》周刊和《中国文化》等刊物上,逐渐出现称颂毛泽东对马列主义理论贡献的文章,陈伯达、艾思奇、和培元、张如心等纷纷撰文,赞颂毛泽东“深刻地灵活地根据辩证唯物主义理论与方法阐明中国革命的规律性”,使马列理论与中国具体的革命实践相结合,与中国的历史实际相结合\footnote{和培元:《论哲学的特性与新哲学的中国化》。《中国文化》第~3~卷,第~2、3~期合刊,1941~年~8~月~20~日出版;和培元:《论中国的特殊性》,《中国文化》创刊号。}。陈伯达、艾思奇、和培元都是毛泽东当时所亲近的“笔杆子”,陈伯达、和培元更是毛泽东的秘书,由这批“秀才”率先宣传毛泽东在理论方面的贡献,中共高层领导不会不知其中的含义。

然而,对于中共领导层一班人而言,将毛泽东树为“理论家”却是一件新鲜事。长期以来,他们都知道毛擅长军事指挥,也都逐渐信服毛在军事指挥方面的才干。自长征结束以来,毛泽东在政治方面的领导成为事实,也被众人逐渐习惯以至承认,但大家在心理上,还是将“理论家”的名号与张闻天、王明联系在一起。尤其张闻天这位前党内“总负责”现在也还在主管中央宣传工作和马列学院,所以当周恩来于~1940~年自莫斯科返回延安后,也就情不自禁地将共产国际对张闻天的评价向毛泽东和其他领导人和盘捧出,谁知周恩来的传达竟遭到毛泽东的当头棒喝,毛怒斥道,什么理论家,背了几大麻袋教条回来!

既然张闻天都够不上“理论家”,那么与张闻天差不多的王明背回来的更是“教条”,周恩来称不上是理论家,至于康生、任弼时、陈云等更是与“理论家”搭不上边,在毛的眼中,领导同志中除了他自己以外,具有理论眼光的只有刘少奇一人。

1941~年后,刘少奇似乎进入到一个“理论喷涌期”。他在盐城的华中局党校开始频频作各种大报告,他不仅谈“中国革命的战略与策略问题”,也谈党内轻视理论的经验主义传统,刘少奇甚至学起毛泽东,站在哲学的高度谈论起“人的阶级性”,“人为什么会犯错误”以及“人性善恶”等抽象命题。从当时毛泽东的立场和角度看,刘少奇的大部分观点虽然都可以接受,但也不是完全没有问题。例如,1941~年~6~月~3~日,刘少奇在盐城参议会发表的演讲“我们在敌后干些什么”,1942~年~10~月~10~日在返回延安途中于北方局党校作的“中国革命的战略与策略问题”报告都回避了毛的“新民主主义论”,而大谈中共应坚持“三民主义”。刘在“人的阶级性”的报告中还发明了“封建阶级党性”的概念。\footnote{《刘少奇年谱》,上卷,页~357~—~58。}这些都是毛不能同意的观点,因此,即使刘少奇颇具理论水平,但中共党内真正的理论家只能非毛泽东莫属。

进入~1942~年后,毛泽东在党内的权势已如日中天。2~月~8~日,延安举行“泽东日”,徐特立、萧三作关于毛泽东生平报告,听众的千余人。现在将毛泽东在理论上的地位加以鼓吹,已刻不容缓。此时,一个原张闻天属下的留苏派马列教员张如心站出来为之起劲摇旗呐喊,显得特别引人注目。

张如心原名张恕安,于二十年代后期在莫斯科中山大学学习,原属国民党左派,后转变至中共阵营,三十年代初返国进入江西中央苏区,后随长征到达延安,长期在张闻天领导下的马列学院从事马列主义理论的教学工作。此人政治嗅觉颇为灵敏,早在~1941~年~2~月,即在其撰写的文章中首先提出“毛泽东同志的思想”的概念,可惜当时未引起广泛注意(他的文章名为《布尔什维克的教育家》)。当时谈“布尔什维克”的文章比比皆是,人们很难注意到张如心在这篇“八股腔”文章里宝贵的“诗眼”,但是有一个人却注意到了。1941~年~12~月底,张如心被调至毛泽东身边,任毛的读书秘书。1942~年~2~月~8~日,张如心为“泽东日”作《怎样学习毛泽东》的报告。2~月~18~日,张如心又在《解放日报》上第一次对“毛泽东主义”作了阐释。显然未经许可,张如心不可能提出这个概念,而在《解放日报》上提出如此重要的概念,陆定一、博古必定事先将其文章送审,向毛泽东、任弼时请示汇报或打过招呼。

“毛泽东主义”的概念一经登报,马上流播开来。1942~年~7~月~1~日,邓拓在中共中央晋察冀分局机关报《晋察冀日报》上发表社论《纪念七·一,全党学习掌握毛泽东主义》。但毛泽东经过全盘思考,感觉“毛泽东主义”一词有所不妥,遂于~1943~年~4~月~22~日,覆信给凯丰,声称自己思想还未成熟,现在还不是鼓吹的时候,“要鼓吹只宜以某些片断去鼓吹(例如整风文件中的几件)”。\footnote{《毛泽东年谱》,中卷,页~434~—~35。}在毛看来,称“毛泽东主义”也许有些刺眼,因为斯大林也只是提“列宁——斯大林主义”,还未敢提“斯大林主义”,毛泽东似乎担心这个说辞会引起莫斯科的不快。再说,称“主义”虽然好听、好看,却无什么创新,这可能也是搁置“毛泽东主义”的一个原因。

就在这个时刻,王稼祥似乎觉察到在“毛泽东主义”问题上出现的微妙的僵局。1943~年~7~月~5~日,他率先提出“毛泽东思想”的概念,并且作了阐释。整风深入后,王稼祥作为原党的领导人之一,处于被整地位,心情压抑,他对由他负责的国际问题研究室的工作不闻不问,对其下属的请求汇报一言不发。此时,王稼祥竟挥笔写文,谈起“毛泽东思想”,向毛泽东献上一份厚礼,再明显不过是企求毛泽东网开一面。但是,王稼祥作为“错误路线”代表人物,又不深刻检讨,仅凭提出“毛泽东思想”,就想溜之大吉,是万难成功的,有资格对毛泽东在理论上的贡献加以总结的人,只能是党的正确路线的代表人物。

这个重任自然而然地落在了党的第二号人物、“白区正确路线”的代表刘少奇的肩上,无论是张如心、邓拓还是王稼祥,资历和地位皆不够格,只有刘少奇才堪此重任。

1943~年~7~月~6~日,刘少奇在《解放日报》上发表《清算党内的孟什维主义思想》,继而在中共七大上作《关于修改党的章程的报告》(即著名的《论党》),刘少奇在报告中,提到“毛泽东”的名字达~105~次,全面地阐释了毛泽东在理论上对马列主义所做的贡献,正式提出中共的思想理论基础为“马列主义与中国革命实际相结合的产物——毛泽东思想”。从此,刘少奇成为“毛泽东思想”概念的首创者——版权归刘少奇。以至二十多年后,刘少奇被毛泽东抛弃,遭受残酷批判时。还抱屈叫冤,声称“毛泽东思想”是他首先提出,并号召全党奉为指针的。

\section{刘少奇等对毛泽东的颂扬}

在中共党内,在一个相当长的历史时期内,无论是在大革命时期,抑或是江西苏维埃时期,都没有颂扬党的领袖的传统,对党的领袖进行大规模的颂扬、赞美,始于四十年代初。首先开创这个先例的是中共中央领导层内毛泽东的同僚,他们率先对自己原来的同事毛泽东进行热烈的赞美,迅速地将毛泽东捧成凌驾于中央集体之上的“尊神”。

在中共中央领导层中,第一个站出来吹捧毛泽东的人是王明。1940~年~5~月,王明发表《学习毛泽东》一文,对毛的“革命意志”,“非凡的革命胆略”,进行全面的赞颂。然而,毛泽东看透了王明讨好的用意,一点也没放松清算王明的准备工作。王明的如意算盘全部落空,不仅讨好的目的未达到,还徒增毛泽东对他的轻视和鄙夷。

从~1942~年开始,中共重要领导人对毛泽东的赞美已形成高潮,几乎所有党的领导人、各大区党和军队的领导人,都加入到歌颂毛泽东的大合唱中。刘少奇写道:

\begin{quoting}
……我们的党在这二十二年中,在三次连续不断的全国性的革命战争中,是经过了各方面的严格考验的……而特别值得提出的,就是在二十二年长期复杂的革命斗争中,终于使我们的党,使我们的无产阶级与我国革命的人民找到了自己的领袖毛泽东同志。我们的毛泽东同志是二十二年来,在艰苦复杂的革命斗争中经过考验的完全精通马列主义的战略战术的,对中国工人阶级与中国人民解放团结抱无限忠心的坚强伟大的革命家\footnote{刘少奇:《肃清党内盂什维主义思想》,载《毛泽东选集》,第~1~卷;代序,《论毛泽东思想》(苏中出版社,1945~年),页~1。}。
\end{quoting}

朱德说:

\begin{quoting}
中国共产党是马列主义的普遍真理与中国革命的具体实践相结合的党,它吸收了世界各国工人运动的综合归纳起来的宝贵经验,它继承了中国几千年历史积累下的优良遗产,它在大革命、土地革命、抗日战争三大阶段中锻炼了自己,丰富了自己,在这剧烈无比的锻炼中,它把马列主义中国化了,把历史遗产进化为适合于现实社会的需要了,这种光辉的成就,体现在我们党有了伟大的领袖毛泽东同志,及以毛泽东同志为首的党中央\footnote{朱德:《“七一”二十二周年感言》,载《毛泽东选集》,第~1~卷,代序,《论毛泽东思想》,页~1。}。
\end{quoting}

彭德怀写道:

\begin{quoting}
毛泽东同志在抗战几年中,对于理论上的伟大贡献,有持久战的理论和新民主主义的理论,《新民主主义论》就是辉煌的杰作……毛泽东同志《新民主主义论》,既不混同于旧三民主义,又不混同于假空喊社会主义,而是马克思列宁主义正确的提出于中国目前的具体环境及历史阶段,成为中国革命现阶段的指针\footnote{彭德怀:《民主政治与三三制政权》,载《毛泽东选集》,第~1~卷,代序,《论毛泽东思想》,页~14~—~15。}。
\end{quoting}

陈毅充满激情地写下他读毛泽东《新民主主义论》的体会:

\begin{quoting}
这是一百年来中国学术思想上及社会实践问题上其中新旧争论的最正确的解决,这是一本马列主义的新创获的著作,也是一本马列主义的古典著作。这是中国共产党以及中华民族理论战线上的光荣代表。
\end{quoting}

他还说:

\begin{quoting}

因此二十一年来的中国共产党,是久经考验过的布尔什维克的党,他的党员和干部,他的领导中枢,党的中央和他的领袖毛泽东同志,久经考验,能征惯战,基本上已经走上完全的布尔什维克化的道路,对一切斗争环境均能适应,一切斗争方向均能掌握。
\end{quoting}

陈毅对刘少奇也作出高度评价,他说:

\begin{quoting}
刘少奇同志许多关于党的论文恰可作全党在这方面极优秀的代表\footnote{陈毅:《伟大的二十一年》,载《毛泽东选集》,第~1~卷,代序,《论毛泽东思想》,页~8、12、11。}。
\end{quoting}

罗荣桓在《学习毛泽东同志的思想》的报告中称颂道:

\begin{quoting}
毛泽东同志成为中国共产党的领袖,人民革命的领袖,不是自己封的,他是代表党的正确的方向,胜利的方向,而与党的整个视野相结合,成为不可分离的关系而得到成就的……毛泽东同志的思想,是马列主义的思想。在民族化方面,有了他的发展。……毛泽东同志的思想是“来自群众中,再到群众中去”,因此,他是掌握了马列主义的基本精神,实事求是的精神,这是教条主义所不了解的。\footnote{参见《毛泽东选集》,第~1~卷,代序,《论毛泽东思想》,页~22;另见《文献和研究》,1986~年第~5~期,页~321。}
\end{quoting}

除了刘少奇等党和军队的领导人外,在延安的毛泽东的密友们也积极参加了歌颂毛泽东的大合唱。

康生在~1943~年~7~月~15~日作的《抢救失足者》报告中,号召以毛泽东的革命精神去肃清一切反革命。他鼓动道:

\begin{quoting}
一切忠实的共产党员们,要学习毛泽东同志的思想、理论与实际,以坚决的革命精神去进行无产阶级与非无产阶级的思想斗争;用毛泽东同志的正确路线,去反对党内一切公开的、暗藏的投降主义,以坚决的革命精神,去进行革命与反革命的斗争\footnote{《毛泽东选集》,第~1~卷,代序,《论毛泽东思想》,页~19。}!
\end{quoting}

陆定一说:

\begin{quoting}
假如我们对于日寇的战略不是主张毛泽东同志所说的持久战,而主张速决战;假如我们对于大资产阶级反共派,或者只有斗争无联合,或者只有联合无斗争,或许抗战的情景就已经不堪设想。不但如此,在毛泽东同志的四部著作中,在中央的许多决定指示中,可以看见我们党的中央确是掌握了辩证唯物论的思想方法,已经有本领善于把马列主义应用到实际中去了。\footnote{陆定一:《为什么整风是党的思想革命》,载《毛泽东选集》,第~1~卷,代序,《论毛泽东思想》,页~13。}
\end{quoting}

原国际派代表人物在歌颂毛泽东方面更是争先恐后,试图以此向毛表示他们的忠心。

王稼祥率先提出“毛泽东思想”这个说辞,他说:

\begin{quoting}
中国民族解放整个过程中——过去现在与未来——的正确道路,就是毛泽东同志的思想,就是毛泽东同志在其著作中与实践中所指出的道路。毛泽东思想就是中国的马克思列宁主义,中国的布尔什维克主义,中国的共产主义\footnote{王稼祥:《中共与中华民族解放的道路》,载《毛泽东选集》,第~1~卷。代序,《论毛泽东思想》,页~5。}。
\end{quoting}

博古写道:

\begin{quoting}
我们有保卫的力量,我们有八十万党员,我们有五十余万在党领导下的军队,我们有巩固的根据地,我们有二十二年斗争的经验,我们有全国人民的拥护,我们有无数身经百战的干部,最后异常重要的是我们有党的领袖中国革命的舵手——毛泽东同志,他的方向就是我们全党的方向,也是全国人民的方向,他总是在最艰难困苦之中领导党和人民走向胜利与光明,我们有取得胜利的一切条件。军事威胁不足以征服共产党,挑拨离间不足以分化共产党,相反我们将更亲密地团结在毛泽东同志为首的中央周围,在毛泽东旗帜下战斗并且取得胜利。\footnote{博古:《在毛泽东旗帜下,为保卫中国共产党而战!,载《毛泽东选集》》,第~1~卷,代序,《论毛泽东思想》,页~20。}
\end{quoting}

邓发在毛泽东的“帐簿”里是一个经验主义者,长期以来邓发与毛没有亲近关系,1943~年邓发也写文章向毛表示敬意:

\begin{quoting}
青年们!我们究竟走“中国之命运”的道路呢?还是走毛泽东新民主主义的道路呢?我想一切有国家观念、有民族天良的热血青年,为了他的理想,为了他的人格,为了保持他那纯洁的良心,他们绝对不愿意走那法西斯黑暗统治的“中国之命运”的道路的,我相信中国青年是会选择引导中国走向独立、自由、民主的毛泽东新民主主义的道路的\footnote{邓发:《谁爱护青年?谁戕害青年?》,载《毛泽东选集》,第~1~卷,代序,《论毛泽东思想》,页~17。}。
\end{quoting}

在延安的理论家们对歌颂毛泽东更是责无旁贷。艾思奇写道:

\begin{quoting}
中国共产党人把马克思列宁主义的普遍真理与中国革命的具体实践相结合,这结合的过程,是根据了中国社会的具体情况,和中国工农群众广大人民的斗争经验的……这一切事实和思想,都和中国共产党的领袖——毛泽东同志的名字分不开,到了今天,铁的事实已经证明,只有毛泽东同志根据中国的实际情况发展了和具体化了的辩证法唯物论,才是能够把“中国之命运”引到光明前途去的科学的哲学,才是人民的革命哲学。\footnote{艾思奇:《中国之命运》的愚民哲学》,《载《毛泽东选集》,第~1~卷,代序;《请毛泽东思想》,页~21。}
\end{quoting}

在延安的一些党的元老,如吴玉章、徐特立、谢觉哉也纷纷写诗、撰文,表达他们对毛泽东的尊崇。徐特立写道:

\begin{quoting}
朱毛在国际在苏区外最大多数的人都以他们两人一定是英雄,是怪物,是天上的,但是苏区的群众却认为他们是老实人。而且联系到说中央政府,中央局负责的,都是老实人。我曾听得江西群众唱的农歌有一句,“好人朱德毛泽东”。又有一次,我参加瑞金的群众大会,有人在会场上说:“朱总司令毛泽东是老实人,中央政府都是老实人。群众的认识,是十分正确的。……现在我来作一个结论,毛主席的工作作风是列宁的作风。列宁的作风是俄国的革命精神和美国的实际精神相结合的作风\footnote{徐特立:《毛主席的实际精神》,载《毛泽东选集》,第~1~卷,代序,《论毛泽东思想》,页~5。}。
\end{quoting}

吴玉章欢呼中共有毛泽东作领袖:

\begin{quoting}
我党得此领袖也同联共有斯大林同志一样,有了高明的舵师,革命一定会胜利的。\footnote{吴玉章:《我的思想自传》,载《吴玉章文集》,下,页~1337~—~38。}
\end{quoting}

在这一系列对毛泽东的赞美中,以周恩来~1943~年~7~月在延安中央办公厅为欢迎他从重庆返回的招待会上的发言最为引人注目。周恩来当着毛泽东的面,对毛大唱赞歌:我们党在这三年中做了比过去二十年还要伟大,“这是全党团结在毛泽东同志领导之下得到的”!周恩来慷慨激昂地说道:

\begin{quoting}
没有比这三年来事变的发展再明白的了,过去一切反对过、怀疑过毛泽东同志领导或其意见的人,现在彻头彻尾地证明其为错误了。我们党二十二年的历史,证明只有毛泽东同志的意见是贯穿者整个历史时期,发展成为一条马列主义中国化,也就是中国共产主义的路线。毛泽东同志的方向,就是中国共产党的方向。毛泽东同志的路线,就是中国的布尔什维克的路线。\footnote{周恩来:《在延安欢迎会的演讲》,载《毛泽东选集》,第~1~卷,代序。《论毛泽东思想》,页~17~—~18。}
\end{quoting}

周恩来的颂扬有着比其他人更重要的意义,作为党的几个历史时期的主要领导人,周恩来对毛表示心悦诚服,对其他老干部将有着重要的示范作用,如今周恩来都向毛泽东表示了忠诚,党内还有谁不能低下他们高贵的头呢?

延安整风展开后,毛泽东对党内昔日同僚的精神优势已完全建立,以往那种平起平坐、随意交谈的局面已经一去不复返。一般情况下,高级领导人已不能随时见毛泽东,除非毛召见,他们需要电话请示或写报告,依程序呈交,毛泽东则为了显示其至尊地位,开始有意识拉开与昔日同僚的距离。毛泽东在一片颂扬声中,悄悄采取与原同事拉开距离的措施很快见效,当美国记者白修德(Theodore H.White)于~1944~年~10~月访问延安时,他所看到的是:毛泽东发表演说,一班高级领导人聚精会神手执笔记本奋笔疾书,其状似一群恭敬的小学生在聆听老师的教诲,而周恩来则坐在毛面前的“第一排,手持小笔记本,稍微有点晃动,引人注目地在记录那篇伟大的讲话,以便主席和所有其他的人都看到他对伟大导师的尊重”\footnote{白修德著,马清槐、方生译:《探索历史》,页~163。}。

\section{摧毁“两个宗派”:对王明、博古、周恩来、彭德怀等人的清算}

中共高层干部对毛泽东的赞颂,从~1942~年后,日益形成风气,不管是在延安的领导干部,还是各大战略区的军政领导人,每逢重大纪念日,照例会撰文鼓吹毛泽东一番。但从毛的角度看,这一切并不完全说明中共领导层已对自己心悦诚服。写文章、作报告赞美、称颂毛固然很好,但很难说每一个人都能心口一致,只有结合实际——也就是从每个领导干部的既往历史,特别是对毛泽东本人的态度来进行检查反省,进而彻底否定原来的“自我”,才能表明是真正服从毛泽东。而要做到这一点,必须名正言顺,即通过检讨党的历史——“学习路线”,明确何谓毛泽东的“正确路线”,何谓“错误路线”,进而联系领导干部的个人实际,对号人座,如此方能真正解除领导干部的思想武装,使他们失去最后的阵地,从而使党的领导干部不仅在行动上,而且在灵魂和精神方面,毫无保留地听命于毛泽东。

毛泽东借讨论中共历史问题而树立自己权威始于~1941~年~9~月政治局扩大会议,随后因王明表示异议,会议于~10~月暂停。紧接着,毛正式在全党鼓动反教条主义,中间穿插了批判王实味,召开整肃文艺界的文艺座谈会,接着又马不停蹄地领导部署审干反奸,继而在~1943~年~3~月,改组中央书记处,正式荣任中共中央政治局主席、中央书记处主席的职务,毛泽东在政治上的主要对手——“教条主义宗派”早已溃不成军。但是,上层斗争还有待进一步深入,因为除王明、博古、张闻天以外的一大批领导干部还未被触及。正是基于这些原因,在毛泽东的命令下,从~1943~年~9~月到~1944~年~4~月,连续召开中央政治局扩大会议,以贯彻、实现毛泽东整肃中共上层的意图。

毛泽东指令召开的这次会议,名曰中央政治局整风会议,参加者却并非都是政治局委员,还有在延安的重要军政领导干部及来延安的一些大区领导人,计有政治局委员、候补委员:毛泽东、刘少奇、任弼时、朱德、周恩来、陈云、康生、彭德怀、张闻天、博古、邓发;中直机关、军直机关、西北局及各大区领导人:彭真、李富春、杨尚昆、林伯渠、吴玉章、高岗、王若飞、李维汉、叶剑英、刘伯承、聂荣臻、贺龙、林彪、罗瑞卿、陆定一、陈伯达、萧向荣和胡乔木。毛泽东开宗明义,宣称要打倒“两个宗派”,一类为“教条主义宗派”,另一类为“经验主义宗派”。毛首先拿王明、博古开刀,再炮火横扫周恩来。切人点是~1938~年中共长江局所犯的“新陈独秀主义”、“阶级投降主义”的“路线错误”,再引向遵义会议前中共中央所犯的“左倾机会主义路线错误”。

1943~年~11~月,延安杨家岭中央大礼堂热闹非凡,为了配合正在举行的中央政治局整风会议,中央总学委在康生、李富春的指挥下,正在这里连续举行包括中央机关所有工作人员和来延安参加七大的代表参加的批判王明、博古的“反右大会”。1943~年~11~月~1~日,大会勒令曾在中共驻共产国际代表团工作过的李国华在会上揭发王明在共产国际所犯的错误,\footnote{《谢觉哉日记》,上,页~550。}李国华在“抢救”运动中已被打成“特务”,让李在会上揭发王明是给其一个“将功赎罪的机会”。11~月~2~日,王明妻子孟庆树在大会发言,\footnote{《谢觉哉日记》,上,页~550。}坚认《八一宣言》由王明起草,她说,今天有人在会上肯定,《八一宣言》是康生写的,我要问一问康生,他敢不敢承认这是他写的?孟庆树继续说:我想问问大家,共产党员应不应该知羞耻?在孟庆树的追问下,康生一言不发,当场并有高自立(在共产国际工作期间化名周和森)站起来作证,但他的发言被会场上的口号打断。\footnote{王明:《中共五十年》,页~146~—~47。}孟庆树情绪激动,泪流满面,直扑坐在台下的毛泽东,要毛主持公道。毛表情严肃,“一动不动”,坐在毛身边的张闻天夫人刘英立即判断,毛泽东对批判王明已下定决心。\footnote{刘英:《在历史的激流中——刘英回忆录》,页~128。}这一天的大会因孟庆树的发言,造成与会者思想的极大混乱,完全离开了会议揭发、批判王明的主题,遭致毛泽东的震怒,他当众斥责大会主席李富春,指责大会充满低级趣味,毫无教育意义,下令停止召开这类大会。\footnote{王明:《中共五十年》,页~148。}从此,王明、孟庆澍再也没有在大会申辩的机会了。

杨家岭“反右大会”上出现的曲折,对毛泽东执意批判王明、博古丝毫不发生影响,对毛而言,需作改变的仅是不开大会,不给王明等有在公众面前声辩的机会,小范围的揭批会则照开不误。政治局整风会议期间,在毛泽东的强势进攻下,除了毛泽东、刘少奇,以及几个毛、刘的亲信康生、李富春、高岗、彭真等,几乎所有领导干部,只能按照毛泽东开的方子,对号人座,分别给自己贴上“教条主义”或“经验主义”的标签,进行自我谴责。

“教条主义”和“宗派主义”的识别标签并不难辨。毛泽东在~1943~年~10~月~6~日政治局整风会议上直言不讳道,现在的中央是以王明、博古时代为基础的,“大宗派来实行篡党”,除了他和刘少奇,其他的人都拥护王明、博古的路线。毛并警告道:“不要像《西游记》中的鲤鱼精,打一下,吐一字\footnote{《胡乔木回忆毛泽东》,页~289、290。}”——意在告诫周恩来等休想逃脱。这样,凡是与王明、博古渊源较深,从莫斯科返回后担任重要职务的领导干部,如张闻天、王稼祥、凯丰、杨尚昆等皆属“教条主义宗派”,而曾经与王明、博古有过合作关系的周恩来、任弼时、彭德怀、邓发、李维汉、叶剑英等则属“经验主义宗派”。某些领导干部则身兼“教条主义”、“经验主义”双重特征,如刘伯承等。毛泽东张起的通天大网,将几乎中共所有领导干部都一网收尽。

王明、博古此时在中共高层已是人人皆可唾骂的对象。在紧张的斗争气氛中,王稼祥、凯丰皆病倒,不能参加会议,凯丰的妻子也因“经不起诱供、套供、连环战,得神经病死了”,\footnote{刘英:《在历史的激流中——刘英回忆录》,页~128。}王明早在~1941~年~10~月后就停止参加中央一切会议,只有博古与会接受批判。博古连续两次作检讨,进行自我鞭挞,仍未得到毛泽东的宽恕。毛疾言厉色,尽情发泄心中的怨恨之气,竟信口将王明、博古称之为“篡党”。致使博古一段时期精神极度压抑,甚至已作好最坏的准备。王明后来回忆说,当时博古曾被威胁道,若不检讨,将被逮捕枪毙,博古痛哭一整夜,才被迫写了交代材料——王明这段话是否真实,笔者不能确定,因为博古早已离世,特录之待考\footnote{博古虽然在~1943~年~9~月后的政治局整风会议上作了检讨,但仍未被毛泽东放过,毛在~1943~年~12~月~28~日发给各中央局、中央分局的电报中,将王明、博古捆在一起鞭打,该电报对王、博均不称“同志”,并说“现在除了王明、博古以外,一切领导同志都是团结一致的”。由此也可看出博古当时境况的险恶。参见王明:《中共五十年》,页~149。另参见李志英:《博古传》(北京:当代中国出版社,1994~年)页~453。}。

与博古相比,张闻天的境遇要稍好一些。早在~1941~年~9~月政治局扩大会议上,张闻天就开始检讨自己的“错误”,1943~年后,张闻天虽身为政治局委员,但是却没有担负任何实际领导工作,只是主持政治材料室,编辑国际、国内参考资料。1943~年~9~月政治局整风会议开始后,张闻天又向毛泽东面呈长达四万字的自我批判的“反省笔记”,获得了“表扬”。在这次会议上,张闻天又加大自我批判的力度,将自己从方方面面予以彻底否定,以显示他“跟真理走”的决心。

周恩来是首次参加这类点名道姓的高层检讨会议。1943~年夏返延安后,周恩来调阅了~1941~年~9~月政治局会议记录,至此他才明白当年会议的详情。从~1943~年~9~月至~1944~年春,周恩来写了大量的反省笔记——他当然知道,作为几个时期党的重要领导人,自己难逃干系。周恩来既然早已认清形势,于是只剩下反省检讨一途。1943~年~9~月~1~日,周恩来在政治局会议上报告三年来大后方工作,周借此向刘少奇表示敬意。周说,白区工作时期的暴露政策与跑街路线是错误路线,而刘少奇同志的言论是正确的。\footnote{《刘少奇年谱》,上卷,页~430、433。}周恩来多次检讨,对自己犯下的“经验主义”错误深切忏悔,但是仍遭到毛泽东、刘少奇等的严厉指责及与会者的批判。周恩来在会场的紧张、高压气氛中,仍然小心坚持一、两个阵地——周恩来顶住与会者的压力,为~1928~年在莫斯科召开的中共六大作了辩护。

在这些会议上,毛泽东每次都直接上阵,对所谓“两个宗派集团”左右开攻。刘少奇、康生则紧密配合,为毛摇旗呐喊。刘少奇在~10~月~24、25~日的会上详细讲述抗战以来党内以毛泽东为代表的正确路线与以王明为代表的投降主义路线的路线斗争。\footnote{《刘少奇年谱》,上卷,页~430、433。}康生当面责骂博古,攻击王明、周恩来、博古领导的武汉《新华日报》是国民党报纸。\footnote{《胡乔木回忆毛泽东》,页~286。}毛泽东并耸人听闻地宣称,王明、博古宗派至今还有破坏活动,党有被分裂的危险,威逼与会领导干部支持自己。

1943~年~12~月~28~日,毛泽东决定正式向全党高级干部公布有关王明、博古的“错误”,将对王、博“错误”的几个判断传达下去,以统一全党的认识。在这一天,中央政治局给各中央局、中央分局并各区党委发出关于研究王明、博古宗派机会主义路线错误的指示电,毛在这份电报中告诉全党高级干部:内战时期,王明、博古宗派左倾机会主义路线造成“白区损失十分之十,苏区及红军损失十分之九。抗战时期(1938~年),这个宗派的右倾机会主义(投降主义)造成项英的失败,华中、华北在其影响时期的损失。王明的主要错误是:一、主张速胜论,反对持久战;二、迷信国民党,反对统一战线的独立自主;三、主张运动战,反对游击战;四、在武汉形成事实上的第二中央,并提倡党内闹独立性,破坏党纪军纪。\footnote{《中央关于学习《反对统一战线中的机会主义》一文的指示》(1943~年~12~月~28~日),载中央档案馆编:《中共中央文件选编(1943~—~1944),第~14~册,页~142~—~43。}在政治局整风会议期间,毛泽东发出此电报,就是指望通过上下夹攻,逼使中央层的领导干部全部缴械投降。

1943~年~11~至~12~月,毛泽东等对王明、博古的进攻达到最高点。在这个时候,已经出现王明等是国民党“内奸”,王明是执行国民党“破坏”中共政策的代理人,以及王明在历史上曾被国民党逮捕,以后又被放出,其历史有疑点等各种论调,只是在接到季米特洛夫干预电报后,毛泽东才在表面上放松了对王明的指控。

在紧张、高压气氛下召开的~1943~年~9~月至~1944~年~4~月中央政治局整风会议,基本实现了毛泽东所要达到的目的,从周恩来开始,重要的领导干部一个接一个检讨、反省。然而,唯有彭德怀与众不同。

彭德怀于~1943~年~10~月上旬奉命返抵延安,此时距抗战初期出师山西、华北已近六年。彭德怀回到延安后,参加了~1943~年秋至~1944~年春由毛泽东主持召开的政治局整风会议。他虽然也作了“自我批评”,但是秉性倔强的彭德怀,并没有像其他高级领导人那样,做违心的自我鞭挞。对于这一切,毛都看在眼里,记在心上。毛泽东一向对彭德怀存有芥蒂,将彭的直言、坦荡、自尊视为对自己的冒犯,因此执意要将“火”烧到彭德怀的身上。

彭德怀与一般中共高级将须确实有所不同,他所关心的问题远远超出军事斗争的范围,还涉及到政治、经济、教育、妇女等广泛领域,并不时就某些涉及全党的重大战略问题向毛泽东提出自己的建读,颇有军人政治家的风采。1942~年~12~月~18~日,彭德怀就中共与国民党的斗争及根据地工作问题致电毛泽东,提出:国民党有长期历史影响,且有一定社会基础,战后我党与国民党的斗争仍是长期的。战后中国人民的政治动向是“自主自由”,“和平建国”,谁能满足这一愿望,谁将取得最后胜利。\footnote{《彭德怀年谱》,页~275、280~—~81、289~—~90、281、290、295。}1943~年~2~月~8~日,彭德怀又在中共北方局太行分局高干会议上发表有关民主教育问题的谈话,认为民主教育在今天中国来说,就是反对封建的教育,而民主革命的共同口号则是“自由、平等、博爱”。彭德怀认为中共应建立起一个完整的制度,在人与人之间发扬互爱、互敬、互助,“己所不欲,勿施于人”,以保障自由、平等、博爱成为合法的东西。目前则要进行“自由、平等、博爱”的启蒙教育,灌输科学精神,反对封建迷信。\footnote{《彭德怀年谱》,页~275、280~—~81、289~—~90、281、290、295。}在返回延安后,彭德怀又于~1944~年~5~月~4~日致信毛泽东,就边区财经问题发表意见,认为延安“对这方面还欠明确方向”,并写了一篇《论公营商店》的文章,希望毛修改后,“最好以社论名义发表”。\footnote{《彭德怀年谱》,页~275、280~—~81、289~—~90、281、290、295。}凡此种种,都引起毛的不快,在毛的眼中,彭德怀的这些言行足以说明他不守本份,有非分擅权之念,已对毛构成了“压迫”。故毛在~1943~年~6~月~6~日电示彭德怀,明确表示不同意他的有关民主教育问题的谈话,批评彭的讲话“从民主、自由、平等、博爱等的定义出发,而不从当前抗日斗争的政治需要出发”。\footnote{《彭德怀年谱》,页~275、280~—~81、289~—~90、281、290、295。}毛且将彭德怀的《论公营商店》一文转贾拓夫、高岗、贺龙、陈云讨论,结果是,“实行彭文方针似不可能”,\footnote{《彭德怀年谱》,页~275、280~—~81、289~—~90、281、290、295。}此文最后也未能正式发表。

但是如何处置这位为中共立下汗马功劳的彭德怀,毛泽东又颇为棘手。毛既要倚重彭德怀为自己打江山,又要削弱彭德怀在军队中的巨大影响,杀一下彭的傲气,以树立毛本人在军队中的绝对权威。于是从~1945~年~2~月~1~日至~7~月~25~日,以召开华北地方与军队工作座谈会的形式,时断时续地开会四十余次,对彭德怀进行了为期四十三天的斗争。

华北工作座谈会开始之初,薄一波被推举为会议主席,彭德怀在其所作的关于华北七年抗战的报告中,检讨了受到毛泽东批评的他在~1943~年~4~月发表于《新华日报》(华北版)上“关于民主、自由、平等、博爱”的讲话。彭德怀表示自己原先的观点基本是错误的,并对自己在华北工作的其它缺点也“进行了严格的白我批评”。彭德怀还回顾了他对毛泽东认识所经历的三个阶段:大哥——老师——领袖。他说,自己早已认识到毛泽东是中国人民的领袖,发展了马列主义,今后要向毛泽东学习。\footnote{《彭德怀年谱》,页~275、280~—~81、289~—~90、281、290、295。}尽管彭德怀已对毛表示了心悦诚服,毛却不准备让彭过关。在~1945~年~3~月后,对彭德怀的批评突然升级,为了打击彭德怀的威望,毛泽东有意将会议规模扩大。毛泽东派康生、李富春等十几位在中央机关和其它大区工作的负责人前来参加会议,对彭德怀的批评指责已突破华北的范围,变成了对彭德怀的总清算。

对彭德怀的指责包罗万象,从攻击彭德怀领导平江暴动是抱“入股”目的,“投机革命”,到内战期间拥护王明、博古。康生等更将斗争矛头指向彭德怀领导的华北军分会在~1937~至~1938~年制定的“运动游击战”方针,指责彭德怀执行王明“右倾投降主义路线”。康生声称彭德怀背着中央发动“百团大战”,从而“暴露我军力量,导致华北根据地遭到日军巨大压力,根据地大大缩小”。

显然没有毛泽东做后盾,康生绝不敢公然攻击像彭德怀这样声望卓著的军方重要将领,并重点指责由彭德怀领导的“百团大战”。其实由康生嘴里讲出的指责“百团大战”的话,最早也来源于毛泽东。毛泽东其人讲话经常前后不一,出尔反尔。1940~年~7~月~22~日,八路军总部以朱德、彭德怀、左权的名义下达战役准备命令,并报延安军委,8~月~20~日,战斗打响,毛闻捷报,大喜过望,给彭德怀打电报,谓之“百团大战真是令人兴奋,象这样的战斗是否还可组织一两次?”\footnote{《彭德怀自述》,页~238。}1940~年~12~月~22~日,毛泽东致电彭德怀,叮嘱“百团大战对外不要宣传结束”,因为中共要“利用百团大战的声势”去反对蒋介石的反共新高潮。\footnote{中国人民革命军事博物馆:《百团大战历史文献资料选编》(北京:解放军出版社,1991~年),页~14。}然而毛泽东说变脸就变脸,当~1943~年底,八路军主要领导人陆续返回延安后,毛泽东在和军方主要将领的谈话中就开始表达他对彭德怀领导的百团大战的不满。毛泽东指责对“百团大战”的宣传,“暴露了我们的力量,引起日本侵略军对我们力量的重新估计,使敌人集中力量来搞我们。同时,使得蒋介石增加了对我们的警惕”。\footnote{聂荣臻:《聂荣臻回忆录》,页~507。另参见薄一波:《回忆陈毅同志二、三事》(1988~年~6~月~30~日),载薄一波:《领袖·元帅·战友》(增订本)(北京:中共中央党校出版社,1992~年),页~139~—~40。}毛泽东在~1945~年重新挑起“百团大战”的话题,只不过是为了在更大的范围打击彭德怀的威望。

华北座谈会对彭德怀的斗争是有目的、有预谋和有组织的,彭德怀参加中共革命以来的历史及功绩,几乎被全盘否定,彭被指责为“一贯反对毛主席”(证据之一是彭德怀在讲话和文章中,很少引用毛泽东的话)。彭德怀的人格也受到恶毒的污辱,彭是中共及军队高级领导人中生活艰苦、节俭的典范,竟被指责为“虚伪”。彭德怀原名“彭得华”,也被罗织为其志在“得中华”,即意欲与毛泽东一争高低。\footnote{参见薄一波:《领袖·元帅·战友》(增订本),页~368~—~69。}由于得到毛泽东、刘少奇等的支持,许多与会者都参加了对彭德怀的斗争,罗瑞卿即是其中最积极的人之一。对于横加在自己头上的各种罪名和指责,彭德怀并未接受,他说,“毛泽东同志有百分之九十九点九是正确的,难道就没有百分之零点一的错误吗”?中共七大期间,彭德怀在大会发言中,对领导八路军的“缺点”、“错误”作了检讨,却仍然不被毛泽东放过。毛在和师哲谈话中,说彭德怀的检讨是“勉强的”——毛泽东对彭德怀确实非常了解,彭德怀只是为了党的团结才违心地做了检讨。毛泽东一语道出他对彭德怀的不满,\footnote{师哲:《在历史巨人身边——师哲回忆录》,页~269~—~70。}“此人刚愎自用,目空一切”。事实上所有对彭德怀的批判,斗争,就是因为彭不能像其他识时务的领导人那样,对毛诚惶诚恐,毕恭毕敬,因为毛泽东决不能忍受在中共军队内有彭德怀这样一个具有强烈自尊、且享有巨大威望的统帅人物。正是毛泽东的狭隘和嫉妒心理作祟,在“七大”闭幕以后,又继续对彭德怀进行斗争,直到日本投降前夕,中共面临国内局势的剧烈变化,必须全力对外,这一斗争才停止下来。当毛泽东打击了彭德怀的傲气,在军队领导干部中搞臭彭德怀的目的初步实现后,\footnote{1945~年彭德怀对薄一波说,他这个人是“高山上倒马桶,臭气出了名的”,参见薄一波:《七十年奋斗与思考》,上卷,《战争岁月》。页~367。}面对新的形势,毛仍不得不起用彭德怀,致使一些原先积极参加批彭的人都一度迷惑不解,“没想到彭德怀又起来了”。

在毛泽东要整肃的干部花名册中,除了彭德怀之外,陈毅是另一个需要“补课”的高级领导人。1944~年~3~月,陈毅从华中根据地返回延安,这是他自~1934~年与中央分别后,十年来第一次与昔日的同志会合,但是陈毅来延安后却承受着巨大的精神压力。1943~年~10~月,新四军政委饶漱石在新四军军部领导层,就陈毅在红军初期与毛泽东不和的旧事,对陈毅展开斗争,饶漱石等指责陈毅一贯反对毛泽东。饶漱石是刘少奇的亲信,1929~年刘少奇即与饶漱石相识。当时,担任满洲团省委书记的饶漱石曾陪同刘少奇同去哈尔滨指导工运。刘少奇在干部使用方面极重个人渊源,1938~年后,随着刘少奇在中共党内地位的上升,刘逐渐将历史上与其有旧谊的同志集合在自己的身边,予以提拔和重用。1942~年,刘少奇奉命前往延安之前,委派饶漱石接替他所担任的中共华中局书记和新四军政委两职。饶漱石在三十年代长期在共产国际工作,无论从党内资历和贡献看,饶漱石均不能与陈毅相比。刘少奇在新四军依靠文职干部出身的饶漱石掣肘陈毅等,这一切当然都被毛泽东看在眼里,毛乐得借刘少奇、饶漱石之手,打击当年曾与自己意见相左的陈毅。因此,饶漱石在对待陈毅的态度上,十分骄横,有恃无恐,显然没有刘少奇、毛泽东的明谕或暗示,饶漱石不可能也不敢对陈毅开展批判。在陈毅与饶漱石的矛盾、争论中,毛泽东非常明显地袒护饶。毛向陈毅明确表示,不愿听他谈与饶的争论,实际是在等待陈毅主动作出自我检讨,也就是向毛低头认错——~1929~年红四军第七次党代会上,由陈毅取代毛之前委书记的错误。为了当年这一“过错”,陈毅已付出沉重代价。1929~年后,陈毅一直受到毛泽东的冷遇。在~1930~至~1931~年的“肃~AB~团”的镇压中,毛泽东的亲信、红一方面军肃反负责人李韶九准备对陈毅下手,陈毅对此已有预感。不久,陈毅果真被扣押,并遭到殴打,硬要陈毅承认是“AB~团”,刚好被路过的毛泽东发现,才被救下。以后,毛泽东将打土豪得来的一些金子交陈毅保管,陈毅才知道自己已无生命之虞。\footnote{聂荣臻:《聂荣臻回忆录》,页~560~—~64。另参见薄一波:《回忆陈毅同志二、三事》(1988~年~6~月~30~日),载薄一波:《领袖·元帅·战友》(增订本)(北京:中共中央党校出版社,1992~年),页~139~—~40。}1932~年初,周恩来抵中央苏区后,陈毅受冷遇的情况也没有得到明显改善。周恩来为了安抚毛泽东,没让陈毅重回红一方面军,陈毅仅担任江西军区司令员,远离红一军团和红三军团,与毛泽东的关系也比较疏远。时隔十年后,对于毛泽东的冷淡,陈毅完全明白其中之原委,陈毅难抑心中的郁闷,多次向一些重要领导干部解释当年红军内部争论的内情,并且承认自己对此负有责任。陈毅性格豪爽,认为十年不见的毛泽东确实比其他中央领导人高明,心悦诚服地接受了毛泽东为中共领袖的事实。陈毅并努力去发现毛泽东对中共革命的贡献,1942~年陈毅远在盐城新四军军部时就发表文章,满怀激情地称颂毛的创造不仅对全民族具有伟大意义,甚至还可供其它各国所效法。返回延安后,在参与起草七大军事报告的过程中,陈毅提出了“毛泽东军事学派”的概念,并且相应批评红军时期某些重要军事领导人的“错误”,在~1945~年~3~月中共六届七中全会主席团讨论陈毅起草的《建军报告》时,一些被陈毅的报告所触及的军队领导干部,“有人潸然落泪,有人唏嘘”\footnote{《当代中国人物传记》丛书编辑部编:《陈毅传》(北京:当代中国出版社,1991~年),页~323。}。

毛泽东对于陈毅的态度有别于彭德怀。相比于性格孤傲的彭德怀,陈毅对于毛泽东显出尊崇、信服的态度。毛认为陈毅心直口快,而彭德怀则城府很深。所以,陈毅除了自我检讨外,没有遭到大规模的斗争。但是,毛泽东并没有完全信任陈毅,在陈毅与饶漱石的矛盾中,毛有意逼迫陈毅向饶低头。中共七大结束后,陈毅仍滞留延安,“天天下围棋”。当薄一波前去探望陈毅,问他为何不立即返回新四军时,陈毅回答道,“他们不让我走”。\footnote{薄一波:《领袖、元帅·战友》(增订本),页~141。}他们者,毛泽东、刘少奇也。毛、刘非逼陈毅表态服从饶漱石,才肯放陈毅返回华中,然而陈毅硬是不表这个态。1945~年~8~月,日本投降消息传来,陈毅表示愿去东北,并陈述在华中“没有事做,不起作用”,但陈毅的要求却不被毛泽东、刘少奇批准。毛命令陈毅前去华中,党内职务为华中局副书记,仍在饶漱石之下。毛此举除了有意造成陈毅与饶漱石之间的互相牵制,也还有疑忌陈毅的另一层涵义。是故,1949~年后,各大区军政委员会主席都是由各大军区司令员担任,只有华东例外,由饶漱石担任。

中共七大前后,毛泽东还策划召开了一系列中共各革命根据地、各部队历史问题座谈会,这些座谈会的宗旨只有一个:检查历史上谁反对毛泽东,谁支持毛泽东。在这类座谈会上,一些原党和军队领导人,如邓发、朱瑞、杨尚昆等均受到与会者的批评。在当时的紧张气氛下,一些战功卓著的红军将领被戴上“经验主义”或“教条主义”的帽子,遭受指责和冷遇。

毛泽东整肃内部,重建新权威,一路凯歌行进。现在,毛泽东在心理上已彻底征服了党和军队的领导人,毛泽东“改造中央”的前提——宣布中共政治路线在遵义会议前为左倾机会主义性质,已被党的领导人所接受。党和军队的领导人也纷纷作出检讨,下一步就是通过正式的会议,将这些用党的文件形式固定下来,毛终于决定召开中共七大的预备会议——中共六届七中全会。

\section{修订《历史决议》:建构以毛泽东为中心的中共党史体系}

1943~年~9~月至~1944~年春的中央政治局整风会议解决了中共领导层的问题,“两个宗派”已如毛泽东所愿,被彻底摧毁,毛泽东胸有成竹,1944~年~5~月,下令召开中共六届七中全会。

毛泽东自六届六中全会控制了中央后,这次全会距上届全会已有六年,召开中共例行会议就已经全无定例,何时开会,何时休会,只凭他个人说了算。现在,毛需要召开党的全会了,这次名曰六届七中全会的会议,竟一开就是十一个月。

毛泽东在这个历时将近一年的六届七中全会上所要解决的问题有两个:其一,将以全会的名义,正式通过对过去路线清算的决议案,把自己的历史地位及对手的“错误”,以中央文件的形式固定下来;其二,重建中共的领导机构。上述两件任务完成,召开中共七大的时机也就完全成熟了。

早在~1941~年~9~月政治局会议期间,毛泽东就主持起草了批判前中央错误的《关于四中全会以来中央领导路线问题结论草案》,该《结论草案》将遵义会议前的中央路线错误定性为“苏维埃运动后期的左倾机会主义错误”,却并没有否定六届四中全会,相反仍认为六届四中全会的路线基本是正确的。毛之所以这样做,乃是四中全会后的中央及中央派往江西苏区的代表团在到达江西后,全力支持了毛泽东镇压富田事变的方针,撤换了主张稳妥解决富田事变的项英的苏区中央局书记职务,由毛泽东替换之。如若否定四中全会,将无从解释以任弼时为首的中央代表团当年所采取行动的正当性。因此,毛在~1941~年秋的《结论草案》中,将错误路线的开端定为~1931~年~9~月~20~日,以该日中共中央所发布的一个有关反对日本帝国主义侵略的文件为标志。其理由是,读文件忽视日帝侵华将造成的国内阶级关系变化,仍一味盲目强调反资产阶级。然而更重要的原因乃是,到了~1931~年~11~月,以任弼时为首的中央代表团在赣南会议上对毛泽东有关土地政策方面的主张展开了批评。

他将有更大的权力,时间一晃三年过去了,毛泽东已牢牢控制了中央,根据自己的意志改写中共历史。在毛泽东看来,此事既重要,又迫切。

在毛泽东之前,中共党内也曾有人利用撰写党史进行党内斗争,甚至以此达到改组中央,使自己一跃进入中央核心层的政治目的,其中最典型的事例是~1930~年王明在上海秘密撰写批判李立三的小册子《两条路线》(又名《为中共更加布尔什维克化而斗争》)。王明、博古等以此作为自己的纲领,在共产国际的支持下,召开了中共六届四中全会,王明也因此成为“正确路线”的代表,跳过中央委员的台阶,进入了中央政治局。

尽管王明和毛泽东在利用修撰党史进行政治斗争方面具有不少相似之处,但细加研究,还是可以发现两者之间仍存有明显的差别。

首先王明的写作是一种个人的行为。王明在撰写《两条路线》时只是中共中央的一个工作人员,他的小册子表达的仅是他本人以及一批留苏学生对党内纷争的看法,虽然王明的观点以后被中共中央所接受,但是王明的小册子并没有经中央全会通过,也没有作为党的正式结论而下达。

其次,王明的小册子重点论述的是他个人对立三路线的批判,并不是对党的整个历史的总结。1938~年后,张闻天在延安马列学院讲授《中国现代革命运动史》(又名《中国革命运动史与中共党史》),张闻天开讲的这门课仅叙述到~1927~年国共分家,对~1927~年后党的历史甚少涉及。概言之,四十年代以前,中共还没有一本较正式的被党中央通过的党史范本,也没有一个全面总结党的历史经验教训的正式文件。

或许是受到王明小册子的启示,更或是毛泽东心理中对“名正言顺”的强烈要求,从遵义会议始,毛就极为重视在党的决议中表达自己的观点。《遵义会议决议》虽然由张闻天起草,但主要反映的是毛的观点,毛在肯定党的政治路线的同时,实际上已否定了这条政治路线的最重要的方面。1941~年,毛为了彻底颠覆原中央的政治合法性。精心编纂《六大以来》,到了~9~月政治局扩大会议,毛全面批评原中央路线,会后又亲自动手,起草了会议的《结论草案》,对修订党的历史决议显出极大的热情与关注。但是毛泽东知道,修订党史一事事关重大,要使自己的论点无懈可击,让党内高层心服口服,还有许多工作要做,其关键问题在于:9~月政治局扩大会议的《结论草案》只是对中共上层的斗争进行了初步总结,大区一级党的历史问题的讨论还未开展,如果在作中央结论之前,先对大区一级的党的历史进行总结,这就可为正式作好中央结论奠定基础。

1942~年~10~月~19~日至次年~1~月~14~日,在毛泽东的倡议下,由任弼时领导召开了中共西北局高干会议,这次会议历时近三个月,参加会议的并非仅是从事西北工作的干部,延安几乎所有重要机关、部队和学校的首长和各地前来延安参加七大的代表都列席了高干会议。中央领导人从毛泽东开始,刘少奇、任弼时、康生、陈云、彭真、高岗、李富春等依次在会上作重要报告。西北局高干会议的主题除了动员检查“两条心、一条心”、部署审干、讨论边区财政经济问题外,另一个重要内容就是检讨西北地区党史上的“路线是非”问题,而问题的重点在对~1935~年陕北“肃反扩大化”问题作出新的结论。

在任弼时、高岗的领导下,高干会议对原北方局派驻西北地区的代表朱理治和原边区党委书记郭洪涛进行了面对面的斗争。1942~年~11~月~17~日、18~日,高岗在高干会议上作《边区党的历史问题检讨》的报告,全面清算朱理治、郭洪涛的“左倾机会主义路线”,高岗在发言中还批判了“带着张国焘错误肃反路线影响来陕北的红二十五军主要领导者”。\footnote{高岗:《边区党的历史问题检讨》(1942~年~11~月~17、18~日),载郭华伦:《中共史论》,第~3~册,页~96。}会议作出决定:

一、修改~1935~年中央红军抵陕北后,中央党务委员会关于审查陕北肃反问题的决定,原决定在为刘志丹、高岗平反时仍批评了刘、高犯了右倾错误,现宣布原决定存在错误,将原错误责任人从戴季英(陕甘边政治保卫局局长)、聂洪钧,扩大到朱理治和郭洪涛,并且把高岗封为“正确路线”的代表。

二、对原陕甘负责人朱理治、郭洪涛作出处罚,将两人定为“错误路线”的代表。

这个新决定对毛泽东日后作出正式的党史决定具有重大意义:

一、它开创了用两分法——即以“正确路线”与“错误路线”的斗争为纲,总结党的历史的新思路与新方法。

二、它将西北历史问题置放于全党的路线斗争的框架不予以认识,换言之,西北党史上的路线斗争是全党路线斗争的一个缩影,两条路线的斗争不仅中央有,地方也有。一方面,中央的错误路线危害了地方的革命事业;另一方面,地方错误路线的代表又是中央错误路线的代理人和党内基础。

1942~年~11~月~1~日,毛泽东在西北局高干会议上作《布尔什维克化十二条》报告,对会议所作的历史结论表示满意,毛在谈到党的历史问题时讲到两个重要问题:一、错误路线(毛尚未正式提出“王明路线”的概念)搞光了十分之十,十分之九的苏区工作和十分之十的白区工作。二、西北的结论与全党的结论“是相同的性质”。\footnote{毛泽东:《布尔什维克化十二条》,载《毛泽东论党的历史》,页~36~—~37.}毛的这番话一方面已为日后的《历史决议》定下了调子,另一方面也表明他是将西北决议当作是全党决议的试点。

根据毛泽东的上述精神,1943~年~1~月,任弼时在高干会议上作《关于中央重新审查陕北“肃反”问题决定的两点解释和检讨历史的教训》的总结报告,通过对朱理治、郭洪涛的批判,为以后全党的《历史决议》对王明、博古的批判确定了几个重要的口径。任弼时指出:

一、从“九一八”至遵义会议之前为错误路线统治全党时期,陕北的情况只是全党的一个缩影。

二、朱理治、郭洪涛“品质恶劣”,“党性根坏,到了陕北这个区域,大摆钦差大臣的派头,有很高的领袖欲,是一种政治野心家,想篡夺陕北(包括陕甘边和陕北)党政军的领导,以满足当领袖的欲望,好来称王称霸”。在这里,只要将朱理治和郭洪涛的名字换上王明、博古,将陕北换成全党,就可成为对王、博的指控,日后对王明、博古、张闻天的批判几乎使用的是与此完全一样的调子。

三、“左倾机会主义”路线造成苏区工作损失十分之九。白区工作损失了十分之十。

四、路线斗争正确与否对革命胜利或失败具有决定性的作用,错误路线可以闹到亡国、亡党、亡头的地步\footnote{任弼时:《关于中央重新审查陕北“肃反”问题决定的两点解释和检讨历史的教训》,中国人民解放军第一炮校政教室翻印,1980~年~10~月。}。

1943~年~6~月~25~日,中共中央西北局作出《关于高干会议对边区党历史问题检讨的决定》,同意和批准~1942~年~11~月高干会检讨的基本精神及高岗对此问题所作的报告,并且要求边区各级组织、全体干部和党员,“应将高岗同志关于边区党历史所检讨的报告,作为整顿三风的重要文件之一,进行深入的研究的讨论”。\footnote{郭华伦:《中共史论》,第~3~册,页~120。}

由此可见,西北历史问题的结论实际上是毛泽东在正式作出全党结论前先行了一步,其基本思路与以后的《历史决议》完全一致,只是~1945~年的《历史决议》更具理论形态,逻辑论证更严密。

1943~年~9~月政治局整风会议及~11~月政治局会议开始以后,随着高层路线斗争的进一步展开,毛泽东对起草党的历史决议也愈加重视。任弼时在领导总结西北党的历史结论方面已获得丰富经验,与毛的认识也完全一致,因此毛放手让任弼时主持全党历史决议的写作班子。为了显示毛的光明正大,让犯“错误”同志对批判自己“心服口服”,毛特意吸取“错误路线”代表人物张闻天、博古参加《历史决议》的起草工作。在毛泽东的领导下,1944~年~5~月~10~日成立“党的历史问题决议准备委员会”,1945~年~4~月~20~日六届七中全会通过《历史决议》,再到~1945~年~8~月~9~日,七届一中全会第二次会议再通过修改后的《历史决议》。毛泽东对此文件,“精雕细刻”(毛之自谓也),前后共进行七次修改。现在,毛已全面否定六届四中全会,将错误路线的开端定为六届四中全会,毛将不再顾及中央代表团和中共中央在富田事变问题上对他的支持,绕开富田事变问题,全盘否定了任弼时、王稼祥、顾作霖的中央代表团。毛泽东在这份《历史决议》中,全面讨伐了以王明、博古为首的“教条主义集团”。\footnote{在~1945~年通过的《历史决议》中没有点王明、博古的名,1950~年~8~月~19~日,毛泽东指示对《历史决议》再作修改。点出王明、博古名字,作为附录收人《毛泽东选集》第~2~卷,以至人们长期以为王明、博古在~1945~年《历史决议》中就被公开点名批判。}同时,也对以周恩来为首的“经验主义宗派”进行了严厉的批判,不点名指责了周恩来等对王明的妥协、支持、拥护,和对毛泽东正确路线构成的危害。

在以毛泽东为中心的新党史中,如何反映作为白区工作“正确路线”代表的刘少奇也是一个至关重要的问题。显而易见,若要肯定刘少奇有关白区工作策略、方针的正确,就必须将其对立面:旧中央领导的三十年代的白区工作予以全盘的否定。1937~年春,刘少奇曾就包括党在白区工作在内的党的十年历史评价问题,公开向张闻天发起挑战,但是那次进攻因受到张闻天和其他从事白区工作许多党的干部的强烈反对,而未能取得成功。几年以后,在毛泽东的支持下,刘少奇在党内的地位得到加强,善于窥测风向的康生发现刘少奇的“政治行情”上涨,很快调整了与刘少奇的关系。三十年代初,康生曾经在党内攻击、指责过刘少奇有关白区工作的意见是“右倾机会主义”,但是到了~1941~年~9~月政治局扩大会议上,康生来了一个一百八十度的大转弯,竟对历史上反刘少奇的言行作了“自我批评”,康生表示当年自己反对刘少奇是错误的,是受到共产国际的影响云云。康生作为党在白区工作的重要领导人之一,以自己的“自我批评”,初步树立了刘少奇作为白区工作“正确路线”代表的形象。此时的毛泽东正需要刘少奇的鼎力协助,因而对扩大宣传刘少奇采取了完全支持的态度。到了~1944~年,刘少奇是党在白区工作“正确路线”代表的说法已广为流行,于是,一个苏区工作“正确路线”的代表毛泽东,再加上白区工作“正确路线”的代表刘少奇,新党史的框架基本就建立了起来。在这个新解释系统中,毛泽东的贡献是全局性的,因而是第一位的,刘少奇的贡献主要集中在白区工作方面,在党的“功劳簿”上名列第二。

然而,刘少奇在党内的威望毕竟无法与毛泽东相比,即使与周恩来等长期位居中枢的领导人相比,也显得逊色。将刘少奇树为白区工作“正确路线”的代表,首先就得让那些从事白区工作的干部承认“白区工作损失了百分之百”的观点,而做到这一点却并非容易。

1945~年中共七大后,在王世英等长期从事白区工作同志的要求下,在延安召开了由康生召集的白区工作总结会议,这次会议的目的就是为了统一对三十年代白区工作的认识,参加会议的有康生、黄火青(代表中组部)、潘汉年、王世英、王学文等十余人。康生在会上作了基调发言,强调白区工作出了大量的叛徒、特务,给党带来了很大损失,以此全盘否定中共三十年代的白区工作。但是康生的讲话当场受到王世英的驳斥,王世英详细列举了~1932~年以后中央军委(上海中央局机构)、中央特科等所开展的情报收集、秘密工作所取得的成就,以证明党在白区的工作并没失败。王世英的发言获得与会者的热烈掌声,但却使康生极为恼怒,不等会议结束康生就退出了会场。\footnote{段建国、贾岷岫著,罗青长审核:《王世英传奇》,页~200。}这个时期的王世英并不知道康生对白区工作的否定,是为了彰显刘少奇,是毛泽东整体战略的一部分。他也不知道,欲突出刘少奇就必须全盘否定党在三十年代的白区工作,尽管王世英在~1936~年就受刘少奇领导,与刘少奇有着较密切的工作与个人关系。\footnote{数十年后的~1968~年,王世英因“刘少奇叛徒案”,被康生、江青等折磨死于秦城监狱。}但是,王世英并不真正懂得“党的高级的政治生活”,他只是从他个人工作和观察的角度对康生的意见表示异议,其结果当然不可能改变党对三十年代白区工作的评价。白区工作总结会议后,毛泽东接见了王世英,没有正面谈论王的意见,只是表示白区工作是革命的一个组成部分。\footnote{段建国、贾岷岫著,罗青长审核:《王世英传奇》,页~201。}不久,1945~年~8~月~9~日,在再次修改通过的《历史决议》中,正式将刘少奇树为白区正确路线的代表,从而确立了刘少奇作为中共第二号人物的法理依据。这样,一个以毛泽东为中心、以党内两条路线斗争为经纬的中共党史体系基本建成。

1945~年春夏,斯大林领导的抗德战争已获全胜,斯大林在世界范围内的威望已达到顶峰。然而正在这个时候,毛泽东却一举打倒了斯大林在中共的代理人。因此毛泽东必须小心翼翼,尽量不触怒斯大林。

据师哲记述,1943~年后,毛泽东花了许多时间和精力,对弗拉基米洛夫(中文名孙平)进行“改造和培养”,“毛主席把孙平拉得紧紧的,目的是通过孙的嘴巴把我们的看法汇报给共产国际和斯大林”、试图“把他变成我们的朋友,让他宣传我们的观点”(从弗拉基米洛夫的《延安日记》看,他对毛的这个目的看得一清二楚)。1944~年夏至中共七大召开前夕,毛泽东几乎每周或隔一周与孙平长谈一次,“每次要花三、四个小时”。七大结束后,毛又召见孙平,向他介绍大会情况(孙平参加了中共七大),“让他照提纲向莫斯科汇报”,中心内容有三:“大会是团结的”,七大路线“得到了全党拥护”;“大会一致拥护毛泽东和刘少奇作为第一把手和第二把手”。\footnote{师哲:《在历史巨人身边——师哲回忆录》,页~220~—~22。}凡此种种,显示出毛泽东纵横捭阖之术已达化境。

毛泽东为了避免斯大林的猜忌,在《历史决议》中对原先已准备彻底否定的几个中央历史问题在评价方而作了改动。

其一、关于中共六大。1928~年在莫斯科召开的中共六大,是周恩来进入中央核心、成为中共事实上最高负责人的开端。在~1943~年秋至~1944~年春召开的中央政治局整风会议上,周恩来遭到与会者严厉指责,与此相联系,大多数与会者都主张否认中共六大。一段时间内,毛泽东放任这种对中共六大的指责——六大是在布哈林指导下召开的,而布哈林早已被斯大林处决,否认六大不致于得罪斯大林(将中共“机会主义”的思想渊源归之于所谓德波林学派——与布哈林关系密切的苏联理论家——成为延安一种流行的说辞)。但是,否定中共六大,遭到周恩来的抵制,周恩来搬出当年斯大林对中国革命的一些观点,强调六大开始注意红军和武装革命功不可没。毛泽东经过考虑,决定对六大基本予以肯定。毕竟斯大林在中共六大期间,多次接见周恩来,对中共六大极为关注,贸然否定六大,不可避免将引起斯大林的怀疑。于是,毛泽东为了显示自己的党内历史老资格,教训王明、博古等人,在许多场合多次宣称,自己是迄今仅有的六大选出的几名中央委员之一。

其二、六届四中全会和五中全会是否合法问题。在~1943~年秋以后,随着毛泽东、刘少奇、康生等对王明、博古、周恩来批判的调子愈来愈高,在中共领导层内,已出现王明、博古是内奸,是专门来破坏中共的论调,王明、博古似乎马上面临被捕的局面。在中央政治局整风会议期间,毛泽东尽性地斥骂王明、博古及其后台——~1938~年已被斯大林枪毙的原共产国际东方部部长米夫。在这种气氛下,由米夫一手策划召开的六届四中全会及由博古、周恩来在江西苏区召开的五中全会,就开始被认为是非法篡权会议。但是,在~1943~年~12~月~22~日季米特洛夫来电后,毛泽东经周密思考,最后决定,将王明、博古仍视为党内问题(不再认为是内奸),承认四中、五中全会均为合法会议。米夫虽被斯大林处死,并不意味着就可以借沟出水,将被共产国际批准的六届四中及五中全会乘机指为非法。如果把长期在莫斯科工作、斯大林对之较为熟悉的王明等人指为敌人,则又走得太远,特别是在眼下中共还需斯大林支持的时刻。

在毛泽东的精心指导下,《历史决议》圆满完成。从此,它成为毛泽东手中掌握的一个对付党内同僚的“紧箍咒”,在~1949~年后为毛的每一个政治斗争服务,一直到~1966~年刘少奇倒台,才从《毛泽东选集》中撤出《历史决议》。

《历史决议》的完成是毛泽东胜利大进军的一个路标,毛无惊无险,就将政治上的绝对优势化为新党史的样本。

毛泽东在加紧修订《历史决议》的同时,重建中央机构的工程也在抓紧进行。

重建中央领导机构的中心任务之一是调整中央书记处的格局,在延安冻结中央政治局和书记处的部分权力。整风运动中,毛泽东采用非常手段,1943~年~3~月,在周恩来等缺席的情况下,又对中央书记处实行重大改组,1937~年~12~月政治局会议——六届六中全会的书记处成员只有毛泽东一人继续保留书记职务,加上新任书记刘少奇、任弼时,中央书记处只有三人。对这种情况,尽管大多数中共领导人口不敢言,但长期以往,毕竟难以服众。尤其在周恩来等已返回延安、对以往过错作了全面检讨并表态全力拥护毛泽东后,再将周恩来排斥于最高领导层之外,似多有不妥。

调整中央书记处也和整风、审干已进入后期扫尾阶段有关。现在一度代行政治局、书记处功能的中央总学委已完成使命,随着甄别阶段的到来,客观上也要求恢复党的日常领导机构的正常运作。

最后,抗战即将胜利,中共将面临新的复杂形势,也迫使毛泽东不得不考虑建全党中央领导机构的问题。随着抗战期间中共力量的急剧增长和即将到来的抗战胜利,中共马上要迎来一个新的时期。新的环境和新的形势需要全党上下一心,一致对外,显然,1943~年的中央格局已不能适应目前党所面临的形势和新的任务。

正是基于上述因素,1944~年~5~月中共六届七中全会开幕之时宣布,由毛泽东、刘少奇、周恩来、朱德、任弼时五人组成主席团,代行政治局和书记处的职权,此五人主席团实际就是延安整风后新产生的中共最高领导核心,也是未来中共七大新组成的中央书记处组成人员。

在这五人领导层中,毛泽东是主宰全局的领袖,刘少奇是仅次于毛的党内第二号人物,周恩来位居第三,朱德作为红军和八路军总司令,其中央书记的职务在很大程度上只具象征意义,任弼时自~1940~年进入核心层,全力支持毛,1943~年成为三人书记处成员,现在继续保留中央书记的职务。

至此,延安整风运动所要达到的改组中共最高核心层的目标已经完成大半,下一步就是在中共七大上选出新一届中央委员会。

\section{中共七大召开及博古、张闻天等人的公开检讨}

1945~年~4~月,中共七大在延安杨家岭中央大礼堂召开,这次会议是在严格保密下进行的,所有代表的笔记本在当天会议结束后,都须编号上交中央办公厅保管。全程参加会议的外国人只有苏联驻延安观察员弗拉基米洛夫,日本共产党代表冈野进(即野坂参三)仅参加了部分会议。

中共七大以毛泽东的胜利而载入史册,毛泽东昔日的政治对手及一批党和军队的重要领导干部,在大会的讲台上相继对自己的错误向全党和毛泽东作出检讨。

毛泽东鼓动那些具有雄厚革命经历的党的负责人在全党面前公开认错,有极现实的意义:彼等的检讨一则证明毛之正确;二则用他们自己的嘴,清除或削弱彼等在党内之广泛影响,树立毛的领袖地位的绝对权威;第三,毛从此手握批评之主动权,可随时给“犯错误”的干部念“紧箍咒”,使其绝对服从自己的领导;第四,以此向斯大林表明,毛所作所为光明正大,被批判的干部已心悦诚服,毛是当之无愧的中共领袖。

王明本属应予检讨的头号人物,但因患重病未能参加全程的会议。王明本来要向大会请假,毛泽东亲自上门劝说,请王明务必参加大会的开幕式。于是,王明被抬着担架送入会场,以显示全党在毛泽东领导下的空前团结\footnote{王明:《中共五十年》,页~170~—~71、157。}。

在七大召开前夕的~1945~年~4~月~20~日,经毛泽东、刘少奇、朱德、周恩来、任弼时的“帮助”,王明向六届七中全会主席团交出长篇书面检讨,他表示接受《历史决议》对自己的全部批判,并且声称,将努力学习毛泽东思想,服从毛泽东的领导。

王明以后宣称,当年他的检讨是被迫的。他说,一些前去探望他的同志劝他作出检讨,并列举理由:共产国际已经解散,再也没有什么组织可以申诉自己的意见了……如果你拒绝承认七中全会的决议。反正七大也能通过类似的决议,如果那时你再不服从,就会把你开除出党,那时要进行斗争,就更加困难了……。\footnote{王明:《中共五十年》,页~170~—~71、157。}显然,王明是抱着“留得青山在,不怕没柴烧”的心理被迫承认“错误”的。

博古是在大会上作公开检讨的“教条宗派集团”的首要人物。1945~年~5~月~3~日,博古向大会做长篇检讨,他流着眼泪对自己作了全盘否定和鞭挞(在前不久的六届七中全会上,博古的“思想还搞不通”)。博古在大会发言中,以自己的错误、荒谬和对革命带来的危害来证明毛泽东的伟大与正确。

在博古之前,5~月~2~日,张闻天也在大会作公开检讨。张闻天作为“教条宗派集团”的骨干,自然是罪责难逃。张闻天的检讨采用对比法,以自己为错误的一方,以毛泽东为正确的一方,全盘否定白己,在对比中几乎用尽所有贬义词汇,痛责自己的愚蠢、狂妄、肤浅和食洋不化。张闻天表示从此要“以郑重与谨慎的态度来在实际行动中学习毛泽东同志的思想与作风”。

张闻天的检讨堪称“新我”战胜“旧我”,是在“灵魂深处爆发革命”的一个样本,也是将自我谴责与歌颂毛泽东相结合的一个范例。他首先将自己形容成一个对革命成事不足,败事有余,一贯给革命造成危害的小资产阶级分子,本没有资格担任领导,只是被“超级的提拔”,才进入中央领导机构。张闻天说:

\begin{quoting}
在这次整风运动中,首先使我深刻感觉到的,就是我过去自高自大,自以为是的骄傲态度,曾经妨碍了我认真学习毛泽东同志的思想与作风,……关于我过去教条主义,左倾机会主义、宗派主义等错误的尖锐与深刻的批评,使我的骄病有了转机。这里,我首先应该感谢毛泽东同志、刘少奇同志对于我的帮助。……为了真理,我曾经必须从我自己的身上撕去一切用虚假的“面子”与“威信”所织成的外衣,以赤裸裸的暴露我自己的一切醜相,我曾经必须打倒把我高悬在半空中的“地位”与“头衔”的支柱所搭成的空架子,使我从天上直摔到地下。……我的无产阶级的灵魂,就是这样,悄悄地在斗争中占了上风。
\end{quoting}

张闻天继续说:

\begin{quoting}
(毛泽东是一切方面的模范)他的思想与情感就是人民的思想与情感,他的痛苦、欢喜与愤怒,就是人民的痛苦、欢喜与愤怒,……他与人民的结合是如此的“密切”,因而分不出究竟他是人民,还是人民是他。……这是真正伟大的人格!……在这伟大的人格面前,我们感觉到真正的骄傲与光荣……而同时我们又感觉到我们自己是如何渺小呀!
\end{quoting}

张闻天表示,他是一个坏思想、坏作风根深蒂固的人,“你们如果轻易相信我,你们可能犯错误”。张闻天恳求毛泽东继续“帮助”“改造”他,他自己则要“赎罪于万一”\footnote{张闻天:《在中国共产党第七次代表大会上的发言》,《中共党史资料》第~53~辑,页~15~—~16、8~—~9。}!

在博古、张闻天之后,杨尚昆、朱瑞等被划入“教条宗派”的领导干部,在大会发言中,也都对自己以往所犯错误表示忏悔。

周恩来作为“经验宗派”的代表人物,在七大开幕式的演说和~4~月~30~日向大会所作的《论统一战线》的发言中,都对自己的“错误”作了自我批评,彭德怀、刘伯承、叶剑英也分别就自己所负责的工作中的错误,各自作了检讨。彭德怀在~4~月~30~日的大会发言中,还像其他高级领导人那样,将自我检讨与歌颂毛泽东结合起来。他说:华北抗战八年所取得的成绩,与毛泽东的正确路线和中央的许多具体指示规定是分不开的,与朱德的名字也是分不开的,同时也有赖于华北党的长期斗争历史和刘少奇对中共中央北方局的正确领导\footnote{《彭德怀年谱》,页~297。}。

中共领导干部在七大上做公开检讨是被精心安排的。做检讨的人无非是两类人,一类为留苏干部,即“教条宗派”分子,另一类为中共老干部,即“经验宗派”分子。毛泽东的亲信均不在做检讨之列。

康生在七大期间是一个备受代表们瞩目的人物,与会代表中一些人曾在审干、抢救运动中蒙受打击,现在他们都得到甄别,党也承认在抢救运动中出现过偏差,因此他们都迫切希望康生能在大会上对此问题作出检讨。然而这些干部的愿望注定要落空,因为康生的所作所为皆为毛泽东所批准,他不认为自己有何过错,毛泽东也不愿看到康生在七大会议上受到指责。

大会本来计划安排康生做审干、反特斗争的报告,后来毛泽东、刘少奇等以七大应解决党在当前斗争中的任务为由,取消了康生的报告,改由他在大会作发言。5~月~2~日,康生在七大第六次会议上作对毛泽东政治报告的认识和两年多反奸工作经验教训的发言,康生在发言中,未作一字的自我批评,引起与会者的强烈不满。毛、刘于是专门安排曾积极参与领导中央党校审干的原党校一部主任古大存在大会作专题发言。5~月~11~日,古大存在七大第十次会议上作广东党在开展武装斗争的经验教训的发言,他在发言中就势大谈审干的必要性。古大存强调,审干十分重要,有缺点错误在所难免,而这些错误是在正确路线上,并且已经改正,因此不必死抓住这点去做口实。古大存称,在抢救运动中,边区老百姓的政治警惕性提高了,国民党的特务政策破产了,这就是说党得到了很大的胜利。古大存在发言中就中共南方工作委员会、粤北省委被国民党破坏的教训(1944~年被国民党破获,南委副书记张文彬遭国民党逮捕并被杀害,南委负责人涂振农被捕叛变),强调必须加强对国民党特务政策的警惕。他批评从大后方撤退来延安的同志不能正确地对待党的审查,他说,有些同志因审干受到冲击而对审干表现得那样愤慨。古大存指责道,这些人拉拢对审干不满的同志,袒护嫌疑分子,这里骂,那里骂,他们为什么不痛恨国民党的特务政策,不去想一想那些牺牲的同志,却因个人情绪受一点刺激就永世不忘呢?!至于南委和粤北省委被破坏,我们在什么时候,什么地方听过大后方提起过这个沉痛的教训?使我们的同志们以后从这个教训里面去警惕国民党的特务政策呢?没有!\footnote{参阅杨立:《带刺的红玫瑰花——古大存沉冤录》,页~32~—~34。}——此时的古大存可能不知道他的这番话已经伤害到周恩来,因为大后方的党受周恩来领导,而中共南方工作委员会(南委)作为南方局的下属机构更是由周恩来直接领导,批评大后方的党不重视反奸斗争就是批评周恩来,且这种说法根本不符合事实。在抢救高潮中,周恩来曾主动为那些受到打击和怀疑的从大后方撤退来延安的同志作出证明和辩解,如果依照古大存的逻辑,周恩来的行为也称的上是“袒护嫌疑分子”和“拉拢对审干不满的同志”了。古大存虽是一位老党员,却不懂“党的高级政治生活”,他在七大的发言说出了毛泽东、刘少奇、康生、彭真想说又不便说的话,全面地表达了毛、刘、康、彭在抢救问题上的观点,客观上为康生等作了掩护。于是,大会期间,康生心安理得地欣赏和他地位不相上下的领导人,在全党面前自我羞辱,与会者虽然对康生强烈不满,但大家对他却无可奈何。

属于刘少奇系统的干部在中共七大上也受到保护。彭真在中央党校领导整风审干,曾伤害过许多干部,同样引起党内的不满,但是彭真等皆被划入刘少奇白区正确路线的大旗之下,他们的威信只能被提高,而不能像“教条宗派”分子和“经验宗派”分子那样受到打击。

\section{毛泽东的胜利与中共新的领导核心}

中共七大是毛泽东将其经过多年的努力与斗争而获得的胜利,以组织的形式予以体现和确认的会议。

在中共七大上全党正式接受以毛泽东思想作为中共的指导思想和行动方针,毛成为无可争议的中共最高领袖,毛昔日的政治对手在全党面前向毛缴械投降,毛以自己的意志彻底改造了中共,七大选出的中央委员会和七届一中全会选出的政治局,具体体现了毛泽东主宰下的中共新格局。

原“教条宗派”分子,前政治局候补委员凯丰,尽管自三十年代后期就积极靠拢毛泽东,甚至在延安整风初期还十分活跃,但终因在遵义会议上反毛泽东的历史旧帐,而未能进入中央委员会。

原“经验宗派”分子,前政治局候补委员,瑞金时代的国家政治保卫局局长邓发,因在江西时期负责肃反,结怨甚多,早在三十年代末就逐渐失势,邓发在~1936~年去苏联后,与王明的关系又颇为密切。在新疆任中共代表时,对毛泽东略有微辞,此次也在中央委员选举中落选。

抗战前期地位显赫,一度担任中共山东分局书记,与博古等关系密切的前中央候补委员朱瑞,和前中央候补委员、原北方局书记杨尚昆,未能和一般大区负责人一样,被选入新的中央委员会。

原“教条宗派”分子、前政治局候补委员王稼祥,只因毛泽东亲手点燃的反“教条宗派”的大火,差点烧到失控的地步,竟需要毛出面为王稼祥说几句好话,才被选为中央候补委员。

王明、博古作为“错误路线”的象征人物,被保留在中央委员会,但名列中央委员三十三人中的倒数第一和第二。

由毛泽东挂帅的新政治局也兼顾了历史与现实。七届政治局只保留一名原教条宗派分子张闻天,这也是对张闻天较早从王明集团中分化出来,长期配合、服从毛泽东的犒劳。

前政治局委员周恩来、陈云保留了他们原先的职务。在~1935~年末瓦窑堡会议补选为政治局委员的彭德怀,作为军队代表,继续担任政治局委员,以显示毛泽东公正、宽大,“惩前毖后,治病救人”。

将任弼时选为政治局委员是顺理成章,现在任弼时名列毛泽东、刘少奇、周恩来、朱德之后,成为中央书记处第五号人物。

新政治局的组成安排也反映了刘少奇力量的急速上升,刘少奇的老部下彭真跳过中央委员的台阶,一跃进入政治局。刘少奇在中共七大这次党内权力再分配的过程中,大大加强了自己在中共组织系统中支配性的影响,这主要表现为,在毛泽东的支持下,刘少奇的一批老部下,突破了因历史上曾被国民党关押于监狱和反省院而造成的干部使用方面的限制,顺利地进入了中央委员会。

1943~年~11~月,一批原从北平、天津等地国民党监狱和反省院出狱的中共干部相继来到延安。尽管~1937~年他们出狱曾经毛泽东、张闻天批准,但是在审干和七大代表的资格审查中,这批人当年出狱的问题又被重新提起。早先,由陈云主持的中央组织部曾倾向于将彼等的出狱视为政治历史问题,在干部使用方面予以一定的限制(1940~年就开始了对七大代表的资格审查)。现在,官司打到毛泽东那里,毛亲自接见薄一波等,询问有关情况,最后作出有利于薄一波等的安排,陈云原先担任的中央组织部长一职也由彭真正式接替。毛泽东此举固然与其当年知晓薄一波等出狱经过有关,更因薄一波、安子文、林枫等人皆属刘少奇的得力部下,且在抗战爆发后,从事领导华北敌后武装工作,在招兵买马方面功劳卓著。毛泽东不囿成规,对刘少奇系统的干部表示充分信任,并在组织安排上予以重用,以此作为对刘少奇支持自己的回报。1945~年,在中共七大召开期间,陈赓等部分七大代表再一次提出薄一波等因出狱问题不宜担任中央候补委员的意见(在酝酿名单中,薄一波被列入候补中委),陈赓向刘少奇、周恩来反映了自己的看法,毛泽东发话,将反对意见挡了回去。毛并说,将薄一波安排为候补中委本来就不妥,其意是薄一波有资格担任正式中央委员。\footnote{参见薄一波:《领袖·元帅·战友》(增订本),页~375。}结果,在七大中委选举中,薄一波、安子文等皆当选为中央委员。

高岗作为西北地方党和军队的代表,与彭真一样,也一步跃过中央委员的台阶,直接进入政治局,这个时期,高岗受到毛泽东的青睐,被认为是毛的亲信。

康生是老政治局委员,在七届一中全会上继续蝉连,随着形势巨变,中共急需一致对外,而对内整肃的任务已经完成,康生长期担任的中央社会部部长一职由李克农接任,康生成为挂名政治局委员,马上陷入“失业”的境地。此时毛泽东已暂时用不着康生,于是他只得主动请求去山西,以后又去山东渤海区,在两地搞了一场极左的土改。

中共七大召开及新的中央领导核心的建立,标志着毛泽东领导下的中共已取得空前的团结,毛声称,“教条宗派”与“经验宗派”已被打碎,不复存在了。经全党奋斗,几年努力,1949~年中共革命终于成功。但是,曾经一度失去踪影的“帽子”,数年后又忽隐忽显,飘浮在毛泽东那些同僚的头顶之上,康生又重新被从笼子里放出。自五十年代中后期始,毛泽东重又频念紧箍咒:1958~年周恩来被批评,1959~年庐山会议上,再掀历史老帐,斗争彭德怀、张闻天,指责朱德,延安整风时期的两顶帽子重新飞舞。而到了~1966~年文革爆发,毛更将他在整风期间提拔的刘少奇、彭真、陆定一等全部打翻在地,刘少奇等只能束手待毙,连招架之力也没有——怨怪毛泽东也不全对,难道不是因为他们才直接、间接造成了这一切!当年不正是刘少奇等将毛泽东捧为至尊至圣,使毛获得了予取于夺、凌驾于全党、无人能予制衡的绝对权力!现在刘少奇等意识到这一点已太晚,这正印证了毛泽东的一句名言:“搬起石头砸自己的脚”!