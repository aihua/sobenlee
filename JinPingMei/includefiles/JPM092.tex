%# -*- coding:utf-8 -*-
%%%%%%%%%%%%%%%%%%%%%%%%%%%%%%%%%%%%%%%%%%%%%%%%%%%%%%%%%%%%%%%%%%%%%%%%%%%%%%%%%%%%%


\chapter{陈敬济被陷严州府\KG 吴月娘大闹授官厅}


诗曰:

\[
猛虎冯其威,往往遭急缚。雷吼徒暴哮,枝撑已在脚。
忽看皮寝处,无复晴闪烁。人有甚于斯,尽以劝元恶。
\]

话说李衙内打了玉簪儿一顿,即时叫陶妈妈来领出,卖了八两银子,另买了个十八岁使女,名唤满堂儿上灶,不在话下。

却表陈敬济,自从西门大姐来家,交还了许多床帐妆奁,箱笼家伙,三日一场嚷,五日一场闹,问他娘张氏要本钱做买卖。他母舅张团练,来问他母亲借了五十两银子,复谋管事。被他吃醉了,往张舅门上骂嚷。他张舅受气不过,另问别处借了银子,干成管事,还把银子交还交来。他母亲张氏,着了一场重气,染病在身,日逐卧床不起,终日服药,请医调治。吃他逆殴不过,只得兑出三百两银子与他,叫陈定在家门首,打开两间房子开布铺,做买卖。敬济便逐日结交朋友陆三郎、杨大郎狐朋狗党,在铺中弹琵琶,抹骨牌,打双陆,吃半夜酒,看看把本钱弄下去了。陈定对张氏说他每日饮酒花费。张氏听信陈定言语,便不肯托他。敬济反说陈定染布去,克落了钱,把陈定两口儿撵出来外边居住,却搭了杨大郎做伙计。这杨大郎名唤杨光彦,绰号为铁指甲,专一粜风卖雨,架谎凿空。他许人话,如捉影捕风,骗人财,似探囊取物。这敬济问娘又要出二百两银子来添上,共凑了五百两银子,信着他往临清贩布去。

这杨大郎到家收拾行李,跟着敬济从家中起身,前往临清马头上寻缺货去。到了临清,这临清闸上是个热闹繁华大马头去处,商贾往来之所,车辆辐凑之地,有三十二条花柳巷,七十二座管弦楼。这敬济终是年小后生,被这杨大郎领着游娼楼,登酒店,货物到贩得不多。因走在一娼楼,见了一个粉头,名唤冯金宝,生的风流俏丽,色艺双全。问青春多少,鸨子说:“姐儿是老身亲生之女,止是他一人挣钱养活。今年青春才交二九一十八岁。”敬济一见,心目荡然,与了鸨子五两银子房金,一连和他歇了几夜。杨大郎见他爱这粉头,留连不舍,在旁花言说念,就要娶他家去。鸨子开口要银一百二十两,讲到一百两上,兑了银子,娶了来家。一路上用轿抬着,杨大郎和敬济都骑马,押着货物车走,一路扬鞭走马,那样欢喜。正是:

\[
多情燕子楼,马道空回首。
载得武陵春,陪作鸾凰友。
\]
张氏见敬济货到贩得不多,把本钱到娶了一个唱的来家,又着了口重气,呜呼哀哉,断气身亡。这敬济不免买棺装殓,念经做七,停放了一七光景,发送出门,祖茔合葬。他母舅张团练看他娘面上,亦不和他一般见识。这敬济坟上覆墓回来,把他娘正房三间,中间供养灵位,那两间收拾与冯金宝住,大姐到住着耳房。又替冯金宝买了丫头重喜儿伏侍。门前杨大郎开着铺子,家里大酒大肉买与唱的吃。每日只和唱的睡,把大姐丢着不去揪采。

一日,打听孟玉楼嫁了李知县儿子李衙内,带过许多东西去。三年任满,李知县升在浙江严州府做了通判,领凭起身,打水路赴任去了。这陈敬济因想起昔日在花园中拾了孟玉楼那根簪子,就要把这根簪子做个证儿,赶上严州去。只说玉楼先与他有了奸,与了他这根簪子,不合又带了许多东西,嫁了李衙内,都是昔日杨戬寄放金银箱笼,应没官之物。“那李通判一个文官,多大汤水!听见这个利害口声,不怕不叫他儿子双手把老婆奉与我。我那时娶将来家,与冯金宝做一对儿,落得好受用。”正是:计就月中擒月兔,谋成日里捉金乌。敬济不来到好,此一来,正是:失晓人家逢五道,溟泠饿鬼撞钟馗。有诗为证:

\[
赶到严州访玉人,人心难忖似石沉。
侯门一旦深似海,从此萧郎落陷坑。
\]

一日,陈敬济打点他娘箱中,寻出一千两金银,留下一百两与冯金宝家中盘缠,把陈定复叫进来看家,并门前铺子发卖零碎布匹。他与杨大郎又带了家人陈安,押着九百两银子,从八月中秋起身,前往湖州贩了半船丝绵绸绢,来到清江浦马头上,湾泊住了船只,投在个店主人陈二店内。交陈二杀鸡取酒,与杨大郎共饮。饮酒中间,和杨大郎说:“伙计,你暂且看守船上货物,在二郎店内略住数日。等我和陈安拿些人事礼物,往浙江严州府,看看家姐嫁在府中。多不上五日,少只三日就来。”杨大郎道:“哥去只顾去。兄弟情愿店中等候。哥到日,一同起身。”

这陈敬济千不合万不合和陈安身边带了些银两、人事礼物,有日取路径到严州府。进入城内,投在寺中安下。打听李通判到任一个月,家小船只才到三日。这陈敬济不敢怠慢,买了四盘礼物,四匹纻丝尺头,陈安押着。他便拣选衣帽齐整,眉目光鲜,径到府衙前,与门吏作揖道:“烦报一声,说我是通判老爹衙内新娶娘子的亲,孟二舅来探望。”这门吏听了,不敢怠慢,随即禀报进去。衙内正在书房中看书,听见是妇人兄弟,令左右先把礼物抬进来,一面忙整衣冠,道:“有请。”把陈敬济请入府衙厅上叙礼,分宾主坐下,说道:“前日做亲之时,怎的不会二舅?”敬济道:“在下因在川广贩货,一年方回。不知家姐嫁与府上,有失亲近。今日敬备薄礼,来看看家姐。”李衙内道:“一向不知,失礼,恕罪,恕罪。”须臾,茶汤已罢,衙内令左右:“把礼贴并礼物取进去,对你娘说,二舅来了。”孟玉楼正在房中坐的,只听小门子进来,报说:“孟二舅来了。”玉楼道:“再有那个舅舅,莫不是我二哥孟锐来家了,千山万水来看我?”只见伴当拿进礼物和贴儿来,上面写着:“眷生孟锐”,就知是他兄弟,一面道:“有请。”令兰香收拾后堂干净。

玉楼装点打扮,俟候出见。只见衙内让直来,玉楼在帘内观看,可霎作怪,不是他兄弟,却是陈姐夫。“他来做甚么?等我出去,见他怎的说话?常言,亲不亲,故乡人;美不美,乡中水。虽然不是我兄弟,也是我女婿人家。”一面整妆出来拜见。那敬济说道:“一向不知姐姐嫁在这里,没曾看得……”才说得这句,不想门子来请衙内,外边有客来了。这衙内分付玉楼款待二舅,就出去待客去了。玉楼见敬济磕下头去,连忙还礼,说道:“姐夫免礼,那阵风儿刮你到此?”叙毕礼数,上坐,叫兰香看茶出来。吃了茶,彼此叙了些家常话儿,玉楼因问:“大姐好么?”敬济就把从前西门庆家中出来,并讨箱笼的一节话告诉玉楼。玉楼又把清明节上坟,在永福寺遇见春梅,在金莲坟上烧纸的话告诉他。又说:“我那时在家中,也常劝你大娘,疼女儿就疼女婿,亲姐夫,不曾养活了外人。他听信小人言语,把姐夫打发出来。落后姐夫讨箱子,我就不知道。”敬济道:“不瞒你老人家说,我与六姐相交,谁人不知?生生吃他听奴才言语,把他打发出去,才吃武松杀了。他若在家,那武松有七个头八个胆,敢往你家来杀他?我这仇恨,结的有海来深。六姐死在阴司里,也不饶他。”玉楼道:“姐夫也罢,丢开手的事,自古冤仇只可解,不可结。”

说话中间,丫鬟放下桌儿,摆下酒来,杯盘肴品,堆满春台。玉楼斟上一杯酒,双手递与敬济说:“姐夫远路风尘,无可破费,且请一杯儿水酒。”这敬济用手接了,唱了喏,也斟一杯回奉妇人,叙礼坐下,因见妇人“姐夫长,姐夫短”叫他,口中不言,心内暗道:“这淫妇怎的不认犯,只叫我姐夫?等我慢慢的探他。”当下酒过三巡,肴添五道,无人在跟前,先丢几句邪言说入去,道:“我兄弟思想姐姐,如渴思浆,如热思凉,想当初在丈人家,怎的在一处下棋抹牌,同坐双双,似背盖一般。谁承望今日各自分散,你东我西。”玉楼笑道:“姐夫好说。自古清者清而浑者浑,久而自见。”这敬济笑嘻嘻向袖中取出一包双人儿的香茶,递与妇人,说:“姐姐,你若有情,可怜见兄弟,吃我这个香茶儿。”说着,就连忙跪下。那妇人登时一点红从耳畔起,把脸飞红了,一手把香茶包儿掠在地下,说道:“好不识人敬重!奴好意递酒与你吃,到戏弄我起来。”就撇了酒席往房里去了。敬济见他不理,一面拾起香茶来,就发话道:“我好意来看你,你到变了卦儿。你敢说你嫁了通判儿子好汉子,不采我了。你当初在西门庆家做第三个小老婆,没曾和我两个有首尾?”因向袖中取出旧时那根金头银簪子,拿在手内说:“这个是谁人的?你既不和我有奸,这根簪儿怎落在我手里?上面还刻着玉楼名字。你和大老婆串同了,把我家寄放的八箱子金银细软、玉带宝石东西,都是当朝杨戬寄放应没官之物,都带来嫁了汉子。我教你不要慌,到八字八\textuni{276CA}儿上和你答话!”

玉楼见他发话,拿的簪子委是他头上戴的金头莲瓣簪儿:“昔日在花园中不见,怎的落在这短命手里?”恐怕嚷的家下人知道,须臾变作笑吟吟脸儿,走将出来,一把手拉敬济,说道:“好阻夫,奴斗你耍子,如何就恼起来。”因观看左右无人,悄悄说:“你既有心,奴亦有意。”两个不由分说,搂着就亲嘴。这陈敬济把舌头似蛇吐信子一般,就舒到他口里交他咂,说道:“你叫我声亲亲的丈夫,才算你有我之心。”妇人道:“且禁声,只怕有人听见。”敬济悄悄向他说:“我如今治了半船货,在清江浦等候。你若肯下顾时,如此这般,到晚夕假扮门子,私走出来,跟我上船家去,成其夫妇,有何不可?他一个文职官,怕是非,莫不敢来抓寻你不成?”妇人道:“既然如此,也罢。”约会下:“你今晚在府墙后等着,奴有一包金银细软,打墙上系过去,与你接了,然后奴才扮做门子,打门里出来,跟你上船去罢。”看官听说,正是佳人有意,那怕粉墙高万丈;红粉无情,总然共坐隔千山。当时孟玉楼若嫁得个痴蠢之人,不如敬济,敬济便下得这个锹镢着;如今嫁这李衙内,有前程,又且人物风流,青春年少,恩情美满,他又勾你做甚?休说平日又无连手。这个郎君也是合当倒运,就吐实话,泄机与他,倒吃婆娘哄赚了。正是:

\[
花枝叶下犹藏刺,人心难保不怀毒。
\]

当下二人会下话,这敬济吃了几杯酒,告辞回去。李衙内连忙送出府门,陈安跟随而去。衙内便问妇人:“你兄弟住那里下处?我明日回拜他去,送些嗄程与他。”妇人便说:“那里是我兄弟,他是西门庆家女婿,如此这般,来勾搭要拐我出去。奴已约下他,今晚三更在后墙相等。咱不如将计就计,把他当贼拿下,除其后患如何?”衙内道:“叵耐这厮无端,自古无毒不丈夫,不是我去寻他,他自来送死。”一面走出外边,叫过左右伴当,心腹快手,如此这般预备去了。

这陈敬济不知机变,至半夜三更,果然带领家人陈安,来府衙后墙下,以咳嗽为号,只听墙内玉楼声音,打墙上掠过一条索子去,那边系过一大包银子。原来是库内拿的二百两赃罚银子。这敬济才待教陈安拿着走,忽听一阵梆子响,黑影里闪出四五条汉,叫声:“有贼了!”登时把敬济连陈安都绑了,禀知李通判,分付:“都且押送牢里去,明日问理。”

原来严州府正堂知府姓徐,名唤徐崶,系陕西临洮府人氏,庚戌进士,极是个清廉刚正之人。次早升堂,左右排两行官吏,这李通判上去,画了公座,库子呈禀贼情事,带陈敬济上去,说:“昨夜至一更时分,有先不知名今知名贼人二名:陈敬济、陈安,锹开库门锁钥,偷出赃银二百两,越墙而过,致被捉获,来见老爷。”徐知府喝令:“带上来!”把陈敬济并陈安揪采驱拥至当厅跪下。知府见敬济年少清俊,便问:“这厮是那里人氏?因何来我这府衙公廨,夜晚做贼,偷盗官库赃银,有何理说?”那陈敬济只顾磕头声冤。徐知府道:“你做贼如何声冤?”李通判在旁欠身便道:“老先生不必问他,眼见得赃证明白,何不回刑起来。”徐知府即令左右:“拿下去打二十板。”李通判道:“人是苦虫,不打不成。不然,这贼便要展转。”当下两边皂隶,把敬济、陈安拖番,大板打将下来。这陈敬济口内只骂:“谁知淫妇孟三儿陷我至此,冤哉!苦哉!”这徐知府终是黄堂出身官人,听见这一声,必有缘故,才打到十板上,喝令:“住了,且收下监去,明日再问。”李通判道:“老先生不该发落他,常言‘人心似铁,官法如炉’,从容他一夜不打紧,就翻异口词。”徐知府道:“无妨,吾自有主意。”当下狱卒把敬济、陈安押送监中去讫。

这徐知府心中有些疑忌,即唤左右心腹近前,如此这般,下监中探听敬济所犯来历,即便回报。这干事人假扮作犯人,和敬济晚间在一\textuni{3B71}上睡,问其所以:“我看哥哥青春年少,不是做贼的,今日落在此,打屈官司。”敬济便说:“一言难尽,小人本是清河县西门庆女婿,这李通判儿子新娶的妇人孟氏,是俺丈人的小,旧与我有奸的。今带过我家老爷杨戬寄放十箱金银宝玩之物来他家,我来此间问他索讨,反被他如此这般欺负,把我当贼拿了。苦打成招,不得见其天日,是好苦也!”这人听了,走来退厅告报徐知府。知府道:“如何?我说这人声冤叫孟氏,必有缘故。”

到次日升堂,官吏两旁侍立。这徐知府把陈敬济、陈安提上来,摘了口词,取了张无事的供状,喝令释放。李通判在旁不知,还再三说:“老先生,这厮贼情既的,不可放他。”反被徐知府对佐贰官尽力数说了李通判一顿,说:“我居本府正官,与朝廷干事,不该与你家官报私仇,诬陷平人作贼。你家儿子娶了他丈人西门庆妾孟氏,带了许多东西,应没官赃物,金银箱笼来。他是西门庆女婿,径来索讨前物,你如何假捏贼情,拿他入罪,教我替你家出力?做官养儿养女,也要长大,若是如此,公道何堪?”当厅把李通判数说的满面羞惭,垂首丧气而不敢言。陈敬济与陈安便释放出去了。良久,徐知府退堂。

这李通判回到本宅,心中十分焦燥。便对夫人大嚷大叫道:“养的好不肖子,今日吃徐知府当堂对众同僚官吏,尽力数落了我一顿,可不气杀我也!”夫人慌了,便道:“甚么事?”李通判即把儿子叫到跟前,喝令左右:“拿大板子来,气杀我也!”说道:“你拿得好贼,他是西门庆女婿。因这妇人带了许多妆奁、金银箱笼来,他口口声声称是当朝逆犯杨戬寄放应没官之物,来问你要。说你假盗出库中官银,当贼情拿他。我通一字不知,反被正堂徐知府对众数说了我这一顿。此是我头一日官未做,你照顾我的。我要你这不肖子何用?”即令左右雨点般大板子打将下来。可怜打得这李衙内皮开肉绽,鲜血迸流。夫人见打得不像模样,在旁哭泣劝解。孟玉楼立在后厅角门首,掩泪潜听。当下打了三十大板,李通判分付左右:“押着衙内,即时与我把妇人打发出门,令他任意改嫁,免惹是非,全我名节。”那李衙内心中怎生舍得离异,只顾在父母跟前啼哭哀告:“宁把儿子打死爹爹跟前,并舍不的妇人。”李通判把衙内用铁索墩锁在后堂,不放出去,只要囚禁死他。夫人哭道:“相公,你做官一场,年纪五十余岁,也只落得这点骨血。不争为这妇人,你囚死他,往后你年老休官,倚靠何人?”李通判道:“不然,他在这里,须带累我受人气。”夫人道:“你不容他在此,打发他两口儿回原籍真定府家去便了。”通判依听夫人之言,放了衙内,限三日就起身,打点车辆,同妇人归枣强县里攻书去了。

却表陈敬济与陈安出离严州府,到寺中取了行李,径往清江浦陈二店中来寻杨大郎。陈二说:“他三日前,说你有信来说不得来,他收拾了货船,起身往家中去了。”这敬济未信,还向河下去寻船只,扑了个空。说道:“这天杀的,如何不等我来就起身去了!”况新打监中出来,身边盘缠已无,和陈安不免搭在人船上,把衣衫解当,讨吃归家,忙忙似丧家之犬,急急如漏网之鱼,随行找寻杨大郎,并无踪迹。那时正值秋暮天气,树木凋零,金风摇落,甚是凄凉。有诗八句,单道这秋天行人最苦:

\[
栖栖芰荷枯,叶叶梧桐坠。蛩鸣腐草中,雁落平沙地。
细雨湿青林,霜重寒天气。不见路行人,怎晓秋滋味。
\]

有日敬济到家。陈定正在门首,看见敬济来家,衣衫褴褛,面貌黧黑,唬了一跳。接到家中,问货船到于何处。敬济气得半日不言,把严州府遭官司一节说了:“多亏正堂徐知府放了我,不然性命难保。今被杨大郎这天杀的,把我货物不知拐的往那里去了。”先使陈定往他家探听,他家说还不曾来家。敬济又亲去问了一遭,并没下落,心中着慌,走入房中。那冯金宝又和西门大姐首南面北,自从敬济出门,两个合气,直到如今。大姐便说:“冯金宝拿着银子钱,转与他鸨子去了。他家保儿成日来,瞒藏背掖,打酒买肉,在屋里吃。家中要的没有,睡到晌午,诸事儿不买,只熬俺们。”冯金宝又说:“大姐成日模草不拈,竖草不动,偷米换烧饼吃。又把煮的腌肉偷在房里,和丫头元宵儿同吃。”这陈敬济就信了,反骂大姐:“贼不是才料淫妇,你害馋痨谗痞了,偷米出去换烧饼吃,又和丫头打伙儿偷肉吃。”把元宵儿打了一顿,把大姐踢了几脚。这大姐急了,赶着冯金宝儿撞头,骂道:“好养汉的淫妇!你偷盗的东西与鸨子不值了,到学舌与汉子,说我偷米偷肉,犯夜的倒拿住巡更的了,教汉子踢我。我和你这淫妇兑换了罢,要这命做甚么!”这敬济道:“好淫妇,你换兑他,你还不值他几个脚指头儿哩。”也是合当有事,于是一把手采过大姐头发来,用拳撞脚踢、拐子打,打得大姐鼻口流血,半日苏醒过来。这敬济便归唱的房里睡去了。由着大姐在下边房里呜呜咽咽,只顾哭泣。元宵儿便在外间睡着了。可怜大姐到半夜,用一条索子悬梁自缢身死,亡年二十四岁。

到次日早辰,元宵起来,推里间不开。上房敬济和冯金宝还在被窝里,使他丫头重喜儿来叫大姐,要取木盆洗坐脚,只顾推不开。敬济还骂:“贼淫妇,如何还睡?这咱晚不起来!我这一跺开门进去,把淫妇鬓毛都拔净了。”重喜儿打窗眼内望里张看,说道:“他起来了,且在房里打秋千耍子儿哩。”又说:“他提偶戏耍子儿哩。”只见元宵瞧了半日,叫道:“爹,不好了,俺娘吊在床顶上吊死了。”这小郎才慌了,和唱的齐起来,跺开房门,向前解卸下来,灌救了半日,那得口气儿来。不知多咱时分,呜呼哀哉死了。正是:

\[
不知真性归何处,疑在行云秋水中。
\]

陈定听见大姐死了,恐怕连累,先走去报知月娘。月娘听见大姐吊死了,敬济娶唱的在家,正是冰厚三尺,不是一日之寒,率领家人小厮、丫鬟媳妇七八口,往他家来。见了大姐尸首吊的直挺挺的,哭喊起来,将敬济拿住,揪采乱打,浑身锥了眼儿也不计数。唱的冯金宝躲在床底下,采出来,也打了个臭死。把门窗户壁都打得七零八落,房中床帐妆奁都还搬的去了。归家请将吴大舅、二舅来商议。大舅说:“姐姐,你趁此时咱家人死了不到官,到明日他过不得日子,还来缠要箱笼。人无远虑,必有近忧。不如到官处断开了,庶杜绝后患。”月娘道:“哥见得是。”一面写了状子。

次日,月娘亲自出官,来到本县授官厅下,递上状去。原来新任知县姓霍,名大立,湖广黄冈县人氏,举人出身,为人鲠直。听见系人命重事,即升厅受状。见状上写着:

\[
告状人吴氏,年三十四岁,系已故千户西门庆妻。状告为恶婿欺凌孤孀,听信娼妇,熬打逼死女命,乞怜究治,以存残喘事。比有女婿陈敬济,遭官事投来氏家,潜住数年。平日吃酒行凶,不守本分,打出吊入。氏惧法逐离出门。岂期敬济怀恨,在家将氏女西门氏,时常熬打,一向含忍。不料伊又娶临清娼妇冯金宝来家,夺氏女正房居住,听信唆调,将女百般痛辱熬打,又采去头发,浑身踢伤,受忍不过,比及将死,于本年八月廿三日三更时分,方才将女上吊缢死。切思敬济,恃逞凶顽,欺氏孤寡,声言还要持刀杀害等语,情理难容。乞赐行拘到案,严究女死根由,尽法如律。庶凶顽知警,良善得以安生,而死者不为含冤矣。为此具状上告本县青天老爷施行。
\]
这霍知县在公座上看了状子,又见吴月娘身穿缟素,腰系孝裙,系五品职官之妻,生的容貌端庄,仪容闲雅。欠身起来,说道:“那吴氏起来,据我看,你也是个命官娘子,这状上情理,我都知了。你请回去,今后只令一家人在此伺候就是了。我就出牌去拿他。”那吴月娘连忙拜谢了知县,出来坐轿子回家,委付来昭厅下伺候。须臾批了呈状,委两个公人,一面白牌,行拘敬济、娼妇冯金宝,并两邻保甲,正身赴官听审。

这敬济正在家里乱丧事,听见月娘告下状来,县中差公人发牌来拿他,唬的魂飞天外,魄丧九霄。那冯金宝已被打得浑身疼痛,睡在床上。听见人拿他,唬的魂也不知有无。陈敬济没高低使钱,打发公人吃了酒饭,一条绳子连唱的都拴到县里。左邻范纲,右邻孙纪,保甲王宽。霍知县听见拿了人来,即时升厅。来昭跪在上首,陈敬济、冯金宝一行人跪在阶下。知县看了状子,便叫敬济上去说:“你这厮可恶!因何听信娼妇,打死西门氏,方令上吊,有何理说?”敬济磕头告道:“望乞青天老爷察情,小的怎敢打死他。因为搭伙计在外,被人坑陷了资本,着了气来家,问他要饭吃。他不曾做下饭,委被小的踢了两脚。他到半夜自缢身死了。”知县喝道:“你既娶下娼妇,如何又问他要饭吃?尤说不通。吴氏状上说你打死他女儿,方才上吊,你还不招认!”敬济说:“吴氏与小的有仇,故此诬陷小的,望老爷察情。”知县大怒,说:“他女儿见死了,还推赖那个?”喝令左右拿下去,打二十大板。提冯金宝上来,拶了一拶,敲一百敲。令公人带下收监。次日,委典史臧不息带领吏书、保甲、邻人等,前至敬济家,抬出尸首,当场检验。身上俱有青伤,脖项间亦有绳痕,生前委因敬济踢打伤重,受忍不过,自缢身死。取供具结,回报县中。知县大怒,又打了敬济十板。金宝褪衣,也是十板。问陈敬济夫殴妻至死者绞罪,冯金宝递决一百,发回本司院当差。

这陈敬济慌了,监中写出贴子,对陈定说,把布铺中本钱,连大姐头面,共凑了一百两银子,暗暗送与知县。知县一夜把招卷改了,止问了个逼令身死,系杂犯,准徒五年,运灰赎罪。吴月娘再三跪门哀告。知县把月娘叫上去,说道:“娘子,你女儿项上已有绳痕,如何问他殴杀条律?人情莫非忒偏向么?你怕他后边缠扰你,我这里替你取了他杜绝文书,令他再不许上你门就是了。”一面把陈敬济提到跟前,分付道:“我今日饶你一死,务要改过自新,不许再去吴氏家缠扰。再犯到我案下,决然不饶。即便把西门氏买棺装殓,发送葬埋来回话,我这里好申文书往上司去。”这敬济得了个饶,交纳了赎罪银子,归到家中,抬尸入棺,停放一七,念经送葬,埋城外。前后坐了半个月监,使了许多银两,唱的冯金宝也去了,家中所有都干净了,房儿也典了,刚刮剌出个命儿来,再也不敢声言丈母了。正是:祸福无门人自招,须知乐极有悲来。有诗为证:

\[
风波平地起萧墙,义重恩深不可忘。
水溢蓝桥应有会,三星权且作参商。
\]
