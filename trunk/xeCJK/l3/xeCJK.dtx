% \iffalse meta-comment
%<*internal>
\iffalse
%</internal>
%<*readme>

xeCJK is a package written for XeLaTeX which allows users to typeset
CJK scripts easily.


 - different default fonts for CJK and other characters;
 - spaces automatically ignored between CJK characters;
 - special effects on full-width CJK punctuation; and
 - automatic adjustment of the space between CJK and other characters.


This package is licensed in LPPL.

If you are interested in the process of development you
may observe

    http://code.google.com/p/ctex-kit/updates/list

- Sun Wenchang <sunwch@hotmail.com>

%</readme>
%<*internal>
\fi
%</internal>
%
%<*internal>
\begingroup
%</internal>
%<*batchfile>
\input docstrip.tex
\keepsilent
\preamble

 Version 3.0.2 (27-Apr-2012)

 Copyright (C) Wenchang Sun <sunwch@hotmail.com>

 Modified by Leo Liu <leoliu.pku@gmail.com>
         and Qing Lee  <sobenlee@gmail.com>

 This file may be distributed and/or modified under the
 conditions of the LaTeX Project Public License, either version 1.3
 of this license or (at your option) any later version.
 The latest version of this license is in
   http://www.latex-project.org/lppl.txt
 and version 1.3 or later is part of all distributions of LaTeX
 version 2005/12/01 or later.

\endpreamble
\postamble
\endpostamble
\askforoverwritefalse
\generate{\file{\jobname.sty}{\from{\jobname.dtx}{package}}}
\generate{\file{example-autofake.tex}{\from{\jobname.dtx}{ex-autofake}}}
\generate{\file{example-fallback.tex}{\from{\jobname.dtx}{ex-fallback}}}
\generate{\file{example-subCJKblock.tex}{\from{\jobname.dtx}{ex-block}}}
%</batchfile>
%<batchfile>\endbatchfile
%<*internal>
\generate{\file{\jobname.ins}{\from{\jobname.dtx}{batchfile}}}
\nopreamble\nopostamble
\generate{\file{README.txt}{\from{\jobname.dtx}{readme}}}
\endgroup
%</internal>
%
%<*driver>
\ProvidesFile{xeCJK.dtx}
%</driver>
%
%<*driver>
\documentclass[a4paper,full]{l3doc}
\usepackage{xeCJK}
\usepackage{metalogo}
\usepackage{enumitem}
\setlist{nosep}
\setmainfont[Ligatures=TeX]{TeX Gyre Pagella}
\setmonofont{Inconsolata}
\setCJKmainfont[BoldFont=SimHei]{SimSun}
\setCJKmonofont{FangSong}
\xeCJKDeclareSubCJKBlock{Ext-B} { "20000 -> "2A6DF }
\xeCJKDeclareSubCJKBlock{Kana}  { "3040 -> "309F, "30A0 -> "30FF, "31F0 -> "31FF, }
\xeCJKDeclareSubCJKBlock{Hangul}{ "1100 -> "11FF, "3130 -> "318F, "A960 -> "A97F, "AC00 -> "D7AF }
\setCJKmainfont[Ext-B]{SimSun-ExtB}
\setCJKmainfont[Kana,BoldFont=meiryob.ttc]{meiryo.ttc}
\setCJKmainfont[Hangul,BoldFont=malgunbd.ttf]{malgun.ttf}
\def\MacroFont{\small\normalfont\ttfamily\punctstyle{plain}}
\parindent=2em
\AtBeginDocument{\DeleteShortVerb{\"}\MakeShortVerb{\|}}
\makeatletter
\@addtoreset{CodelineNo}{section}
\makeatother
\begin{document}
  \DocInput{\jobname.dtx}
\end{document}
%</driver>
% \fi
%
% \GetFileInfo{\jobname.sty}
%
% \title{\pkg{xeCJK} 宏包}
% \author{孙文昌}
% \date{\filedate\qquad\fileversion}
% \maketitle
%
% \begin{documentation}
%
% \section{简介}
%
% \pkg{xeCJK} 是一个 \XeLaTeX 宏包,用于排版 CJK 文字,包括字体选择和标点控制
% 等。主要特点:
% \begin{enumerate}
% \item 分别设置 CJK 和英文字体;
% \item 自动忽略 CJK 文字间的空格而保留其它空格,允许在非标点汉字和英文
% 字母 (a--z, A--Z) 间断行;
% \item 提供多种标点处理方式: 全角式、半角式、开明式、行末半角式和 CCT 式;
% \item 自动调整中英文间空白。
% \end{enumerate}
%
%\end{documentation}
%
%
%
%\begin{implementation}
%
%\section{\pkg{xeCJK} 代码实现}
%
%    \begin{macrocode}
%<*package>
%    \end{macrocode}
%
%    \begin{macrocode}
\RequirePackage{xparse}
\RequirePackage{l3keys2e}
\ProvidesExplPackage {xeCJK} {2012/04/27} {3.0.2}
  {package for typesetting CJK scripts with XeLaTeX}
%    \end{macrocode}
%
% \begin{macro}
%  {\xeCJK_msg_new:nn ,\xeCJK_error:n,\xeCJK_error:nx,
%  \xeCJK_warning:nx,\xeCJK_info:nx,\xeCJK_info:nxx}
%    \begin{macrocode}
\cs_new_nopar:Npn \xeCJK_msg_new:nn { \msg_new:nnn     { xeCJK } }
\cs_new_nopar:Npn \xeCJK_error:n    { \msg_error:nn    { xeCJK } }
\cs_new_nopar:Npn \xeCJK_error:nx   { \msg_error:nnx   { xeCJK } }
\cs_new_nopar:Npn \xeCJK_warning:n  { \msg_warning:nn  { xeCJK } }
\cs_new_nopar:Npn \xeCJK_warning:nx { \msg_warning:nnx { xeCJK } }
\cs_new_nopar:Npn \xeCJK_info:nx    { \msg_info:nnx    { xeCJK } }
\cs_new_nopar:Npn \xeCJK_info:nxx   { \msg_info:nnxx   { xeCJK } }
%    \end{macrocode}
% \end{macro}
%
% \pkg{xeCJK} 必须使用 \XeTeX 引擎的支持。
%    \begin{macrocode}
\msg_new:nnn {xeCJK} { Require-XeTeX }
  {
    The~xeCJK~package~requires~XeTeX~to~function.\\\\
    You~must~change~your~typesetting~engine~to~"xelatex" \\
    instead~of~plain~"latex"~or~"pdflatex"~or~"lualatex".
  }
\xetex_if_engine:F { \msg_critical:nn { xeCJK } { Require-XeTeX } }
%    \end{macrocode}
%
% |\XeTeXglyphbouds| 可以得到一个字符的上下左右边距,用于标点压缩。
%    \begin{macrocode}
\xeCJK_msg_new:nn { XeTeX-too-old }
  {
    \token_to_str:N \XeTeXglyphbounds\c_space_tl is~not~defined.\\\\
    You~have~to~update~XeTeX~to~the~version~0.9995.0~or~later.
  }
\cs_if_free:NT \XeTeXglyphbounds
  {
    \xeCJK_error:n { XeTeX-too-old }
    \AtBeginDocument { \punctstyle { plain } }
  }
%    \end{macrocode}
%
% 抑制 |BOM|。
%    \begin{macrocode}
\char_set_catcode_ignore:n { "FEFF }
%    \end{macrocode}
%
% \begin{macro}{xeCJKactive}
% \pkg{xeCJK} 功能开关。
%    \begin{macrocode}
\keys_define:nn { xeCJK / options }
  {
    xeCJKactive .choice:,
    xeCJKactive / true  .code:n = { \XeTeXinterchartokenstate = \c_one  } ,
    xeCJKactive / false .code:n = { \XeTeXinterchartokenstate = \c_zero } ,
    xeCJKactive      .default:n = { true },
  }
%    \end{macrocode}
% \end{macro}
%
% \subsection{字符类别设定}
%
% \pkg{xeCJK} 需要以下字符类别用于字符输出。其中 |Default|、|CJK|、|FullLeft|、
% |FullRight|、|Boundary| 为 \XeTeX\ 中预定义的类别,\pkg{xeCJK} 新增加了\
% |HalfLeft|、|HalfRight|、|NormalSpace|。
% \begin{center}
% \begin{tabular}{cll}
% \hline
%   类别        & 说明                & 例子 \\ \hline
% |Default|     & 西文一般符号       & abc123 \\
% |CJK|         & CJK 表意符号       & 汉字ぁぃぅ \\
% |FullLeft|    & 全角左标点         & (《:“ \\
% |FullRight|   & 全角右标点         & ,。)》” \\
% |HalfLeft|    & 半角左标点         & ( [ \{ \\
% |HalfRight|   & 半角右标点         & , . ? ) ] \} \\
% |NormalSpace| & 前后原始间距的符号 & / \\
% |Boundary|    & 边界              & 空格 \\
% \hline
% \end{tabular}
% \end{center}
%
% \begin{macro}[int]
%   { \xeCJK_Default_class, \xeCJK_CJK_class, \xeCJK_FullLeft_class,
%     \xeCJK_FullRight_class, \xeCJK_Boundary_class }
% 这五类是 \XeTeX\ 预定义的类别。
%    \begin{macrocode}
\int_const:Nn \xeCJK_Default_class   {   0 }
\int_const:Nn \xeCJK_CJK_class       {   1 }
\int_const:Nn \xeCJK_FullLeft_class  {   2 }
\int_const:Nn \xeCJK_FullRight_class {   3 }
\int_const:Nn \xeCJK_Boundary_class  { 255 }
%    \end{macrocode}
% \end{macro}
%
% \begin{macro}{\xeCJK_new_class:n}
% 新建一个字符类别。
%    \begin{macrocode}
\cs_new_nopar:Nn \xeCJK_new_class:n
  {
    \exp_after:wN \newXeTeXintercharclass \cs:w xeCJK_#1_class\cs_end:
    \clist_gclear_new:c { g_xeCJK_#1_range_clist }
  }
%    \end{macrocode}
% \end{macro}
%
% \begin{macro}[int]
%   { \xeCJK_HalfLeft_class, \xeCJK_HalfRight_class, \xeCJK_NormalSpace_class }
% 新增西文半角左/右标点和前后原始间距的符号类。
%    \begin{macrocode}
\xeCJK_new_class:n { HalfLeft }
\xeCJK_new_class:n { HalfRight }
\xeCJK_new_class:n { NormalSpace }
%    \end{macrocode}
% \end{macro}
%
% \begin{macro}
%  {\g_xeCJK_base_class_clist,\g_xeCJK_CJK_class_clist,\g_xeCJK_CJK_class_prop}
%    \begin{macrocode}
\clist_set:Nn \g_xeCJK_base_class_clist
  {
    Default, CJK, FullLeft, FullRight, Boundary,
    HalfLeft, HalfRight, NormalSpace
  }
\clist_map_inline:Nn \g_xeCJK_base_class_clist
  { \clist_gclear_new:c { g_xeCJK_#1_range_clist } }
\clist_new:N \g_xeCJK_CJK_class_clist
\prop_new:N \g_xeCJK_CJK_class_prop
\cs_new_nopar:Nn \xeCJK_save_CJK_class:n
  {
    \clist_gput_right:Nx \g_xeCJK_CJK_class_clist {#1}
    \prop_gput:Nvx \g_xeCJK_CJK_class_prop { xeCJK_#1_class } {#1}
  }
\cs_generate_variant:Nn \prop_gput:Nnn { Nvx }
\clist_map_inline:nn { CJK, FullLeft, FullRight } { \xeCJK_save_CJK_class:n {#1} }
%    \end{macrocode}
% \end{macro}
%
% \begin{macro}{\xeCJK_class_num:n}
% |#1| 为字符类别名称,用于取得字符类别对应的编号。
%    \begin{macrocode}
\cs_new_nopar:Nn \xeCJK_class_num:n { \use:c { xeCJK_#1_class } }
%    \end{macrocode}
% \end{macro}
%
% \begin{macro}[pTF]{\tl_if_blank:x}
%     \begin{macrocode}
\prg_set_conditional:Nnn \tl_if_blank:x { p , T , F , TF }
  {
    \if_int_compare:w \pdftex_strcmp:D { \c_empty_tl } {#1} = \c_zero
      \prg_return_true:
    \else:
      \if_int_compare:w \pdftex_strcmp:D { \c_space_tl } {#1} = \c_zero
        \prg_return_true: \else: \prg_return_false: \fi:
    \fi:
  }
%    \end{macrocode}
% \end{macro}
%
% \begin{macro}{\xeCJKDeclareCharClass}
% \begin{macro}[aux]{\xeCJK_setcharclass_aux:Nn}
% 用于设置字符所属的类别,|#1| 为类别名称,|#2| 为字符的 |Unicode|,相邻字符用
% 半角逗号隔开,支持类似 |"1100 -> "11FF| 起止范围的使用方式。
%    \begin{macrocode}
\NewDocumentCommand \xeCJKDeclareCharClass { > { \TrimSpaces } m m }
  {
    \clist_gput_right:cx { g_xeCJK_#1_range_clist } {#2}
    \clist_map_inline:xn {#2}
      {
        \str_if_eq:xxF {##1} { -> }
          {
            \xeCJK_setcharclass_aux:Nn \xeCJKsetcharclass {##1}
            { \xeCJK_class_num:n {#1} }
          }
      }
  }
\NewDocumentCommand \xeCJK_setcharclass_aux:Nn
  { m > { \SplitArgument { 1 } { -> } } m } { #1 #2 }
\cs_generate_variant:Nn \clist_map_inline:nn { x }
%    \end{macrocode}
% \end{macro}
% \end{macro}
%
% \begin{macro}{\xeCJKsetcharclass}
% |#1| 和 |#2| 为字符类别起止的 |Unicode|,|#3| 为类别名称对应编号。
%    \begin{macrocode}
\NewDocumentCommand \xeCJKsetcharclass { m m m }
  {
    \xeCJK_check_num_range:nnNN {#1} {#2} \l_tmpa_int \l_tmpb_int
    \int_set:Nn \l_tmpc_int {#3}
    \prop_if_in:NVTF \g_xeCJK_CJK_class_prop \l_tmpc_int
      { \cs_set_eq:NN \xeCJK_set_catcode:n \char_set_catcode_other:n }
      { \cs_set_eq:NN \xeCJK_set_catcode:n \use_none:n }
    \loop
      \xeCJK_set_catcode:n \l_tmpa_int
      \XeTeXcharclass \l_tmpa_int = \l_tmpc_int
      \if_int_compare:w \l_tmpa_int < \l_tmpb_int
        \int_incr:N \l_tmpa_int
    \repeat
  }
%    \end{macrocode}
% \end{macro}
%
% \begin{macro}{\xeCJK_check_num_range:nnNN}
%    \begin{macrocode}
\cs_new_nopar:Nn \xeCJK_check_num_range:nnNN
  {
    \bool_if:nTF { \tl_if_blank_p:x {#1} || \tl_if_blank_p:x {#2} }
      {
        \int_set:Nn #3 { \tl_if_blank:xTF {#1} {#2} {#1} }
        \int_set_eq:NN #3 #4
      }
      {
        \int_set:Nn #3 { \int_min:nn {#1} { \IfNoValueTF {#2} {#1} {#2} } }
        \int_set:Nn #4 { \int_max:nn {#1} { \IfNoValueTF {#2} {#1} {#2} } }
      }
  }
%    \end{macrocode}
% \end{macro}
%
% \begin{macro}{\xeCJKResetPunctClass}
% 用于重置标点符号所属的字符类。
%    \begin{macrocode}
\NewDocumentCommand \xeCJKResetPunctClass { }
  {
    \xeCJKDeclareCharClass { HalfLeft }
      { "28 , "2D , "5B , "60 , "7B }
    \xeCJKDeclareCharClass { HalfRight }
      { "21 , "22 , "25 , "27 , "29 , "2C , "2E , "3A , "3B , "3F , "5D , "7D , }
    \xeCJKDeclareCharClass { FullLeft }
      {
        "2018 , "201C , "2116 , "3008 , "300A , "300C , "300E , "3010 , "3012 ,
        "3014 , "3016 , "3018 , "301A , "301D , "3036 , "E76C , "FE59 , "FE5B ,
        "FE5D , "FE5F , "FE60 , "FE69 , "FE6B , "FF03 , "FF04 , "FF08 , "FF20 ,
        "FF3B , "FF5B , "FFE0 , "FFE1 , "FFE5 , "FFE6 ,
      }
    \xeCJKDeclareCharClass { FullRight }
      {
        "00B7 , "2019 , "201D , "2014 , "2025 , "2026 , "2030 , "2500 , "3001 ,
        "3002 , "3005 , "3006 , "3009 , "300B , "300D , "300F , "3011 , "3015 ,
        "3017 , "3019 , "301B , "301E , "301F , "3041 , "3043 , "3045 , "3047 ,
        "3049 , "3063 , "3083 , "3085 , "3087 , "308E , "309B , "309C , "309D ,
        "309E , "30A1 , "30A3 , "30A5 , "30A7 , "30A9 , "30C3 , "30E3 , "30E5 ,
        "30E7 , "30EE , "30F5 , "30F6 , "30FB , "30FC , "30FD , "30FE , "FE50 ,
        "FE51 , "FE52 , "FE54 , "FE55 , "FE56 , "FE57 , "FE5A , "FE5C , "FE5E ,
        "FE6A , "FF01 , "FF05 , "FF09 , "FF0C , "FF0E , "FF1A , "FF1B , "FF1F ,
        "FF3D , "FF5D , "FF61 , "FF63 , "FF64 , "FF65 , "FF67 , "FF68 , "FF69 ,
        "FF6A , "FF6B , "FF6C , "FF6D , "FF6E , "FF6F , "FF70 , "FF9E , "FF9F ,
      }
  }
%    \end{macrocode}
% \end{macro}
%
% 设置 CJK 文字的范围。
%     \begin{macrocode}
\xeCJKDeclareCharClass { CJK }
  {
%    \end{macrocode}
% \begin{itemize}
% \item Hangul Jamo (谚文字母)
%    \begin{macrocode}
    "1100 -> "11FF ,
%    \end{macrocode}
% \item CJK Radicals Supplement (中日韩部首补充)
%    \begin{macrocode}
    "2E80 -> "2EFF ,
%    \end{macrocode}
% \item Kangxi Radicals (康熙部首)
%    \begin{macrocode}
    "2F00 -> "2FDF ,
%    \end{macrocode}
% \item Ideographic Description Characters (表意文字描述符)
%    \begin{macrocode}
    "2FF0 -> "2FFF ,
%    \end{macrocode}
% \item CJK Symbols and Punctuation (中日韩符号和标点)
%    \begin{macrocode}
    "3000 -> "303F ,
%    \end{macrocode}
% \item Hiragana (日文平假名)
%    \begin{macrocode}
    "3040 -> "309F ,
%    \end{macrocode}
% \item Katakana (日文片假名)
%    \begin{macrocode}
    "30A0 -> "30FF ,
%    \end{macrocode}
% \item Bopomofo (注音字母)
%    \begin{macrocode}
    "3100 -> "312F ,
%    \end{macrocode}
% \item Hangul Compatibility Jamo (谚文兼容字母)
%    \begin{macrocode}
    "3130 -> "318F ,
%    \end{macrocode}
% \item Kanbun (象形字注释标志)
%    \begin{macrocode}
    "3190 -> "319F ,
%    \end{macrocode}
% \item Bopomofo Extended (注音字母扩展)
%    \begin{macrocode}
    "31A0 -> "31BF ,
%    \end{macrocode}
% \item CJK Strokes (中日韩笔画)
%    \begin{macrocode}
    "31C0 -> "31EF ,
%    \end{macrocode}
% \item Katakana Phonetic Extensions (日文片假名语音扩展)
%    \begin{macrocode}
    "31F0 -> "31FF ,
%    \end{macrocode}
% \item Enclosed CJK Letters and Months (带圈中日韩字母和月份)
%    \begin{macrocode}
    "3200 -> "32FF ,
%    \end{macrocode}
% \item CJK Compatibility (中日韩兼容)
%    \begin{macrocode}
    "3300 -> "33FF ,
%    \end{macrocode}
% \item CJK Unified Ideographs Extension-A (中日韩统一表意文字扩展 A)
%    \begin{macrocode}
    "3400 -> "4DBF ,
%    \end{macrocode}
% \item Yijing Hexagrams Symbols (易经六十四卦符号)
%    \begin{macrocode}
    "4DC0 -> "4DFF ,
%    \end{macrocode}
% \item CJK Unified Ideographs (中日韩统一表意文字)
%    \begin{macrocode}
    "4E00 -> "9FFF ,
%    \end{macrocode}
% \item Yi Syllables (彝文音节)
%    \begin{macrocode}
    "A000 -> "A48F ,
%    \end{macrocode}
% \item Yi Radicals (彝文字根)
%    \begin{macrocode}
    "A490 -> "A4CF ,
%    \end{macrocode}
% \item Hangul Jamo Extended-A (谚文扩展 A)
%    \begin{macrocode}
    "A960 -> "A97F ,
%    \end{macrocode}
% \item Hangul Syllables (谚文音节)
%    \begin{macrocode}
    "AC00 -> "D7AF ,
%    \end{macrocode}
% \item Kana Supplement (日文假名增补)
%    \begin{macrocode}
    "B000 -> "B0FF ,
%    \end{macrocode}
% \item Hangul Jamo Extended-B (谚文扩展 B)
%    \begin{macrocode}
    "D7B0 -> "D7FF ,
%    \end{macrocode}
% \item CJK Compatibility Ideographs (中日韩兼容表意文字)
%    \begin{macrocode}
    "F900 -> "FAFF ,
%    \end{macrocode}
% \item CJK Compatibility Forms (中日韩兼容形式)
%    \begin{macrocode}
    "FE30 -> "FE4F ,
%    \end{macrocode}
% \item Halfwidth and Fullwidth Forms (半角及全角形式)
%    \begin{macrocode}
    "FF00 -> "FFEF ,
%    \end{macrocode}
% \item CJK Unified Ideographs Extension-B (中日韩统一表意文字扩展 B)
%    \begin{macrocode}
    "20000 -> "2A6DF ,
%    \end{macrocode}
% \item CJK Unified Ideographs Extension-C (中日韩统一表意文字扩展 C)
%    \begin{macrocode}
    "2A700 -> "2B73F ,
%    \end{macrocode}
% \item CJK Unified Ideographs Extension-D (中日韩统一表意文字扩展 D)
%    \begin{macrocode}
    "2B740 -> "2B81F ,
%    \end{macrocode}
% \item CJK Compatibility Ideographs Supplement (中日韩兼容表意文字增补)
%    \begin{macrocode}
    "2F800 -> "2FA1F ,
%    \end{macrocode}
% \end{itemize}
%    \begin{macrocode}
  }
%    \end{macrocode}
%
% 重置标点符号的字符类。
%    \begin{macrocode}
\xeCJKResetPunctClass
%    \end{macrocode}
%
% \begin{macro}{\normalspacedchars}
% 声明前后不加间距的字符。
%    \begin{macrocode}
\NewDocumentCommand \normalspacedchars { m }
  {
    \tl_map_inline:nn {#1}
      { \XeTeXcharclass `##1 = \xeCJK_class_num:n { NormalSpace } }
  }
\normalspacedchars{/}
%    \end{macrocode}
% \end{macro}
%
% \begin{macro}{\xeCJK_inter_class_toks:nnn,\xeCJK_inter_class_toks:nnc}
% 在相邻类别之间插入内容。
%    \begin{macrocode}
\cs_new_nopar:Nn \xeCJK_inter_class_toks:nnn
  { \XeTeXinterchartoks \xeCJK_class_num:n {#1} \xeCJK_class_num:n {#2} = {#3} }
\cs_generate_variant:Nn \xeCJK_inter_class_toks:nnn { nnc }
%    \end{macrocode}
% \end{macro}
%
% \begin{macro}{\xeCJK_clear_inter_class_toks:nn}
% 清除相邻类别之间的内容。
%    \begin{macrocode}
\cs_new_nopar:Nn \xeCJK_clear_inter_class_toks:nn
  {
    \XeTeXinterchartoks
    \xeCJK_class_num:n {#1} \xeCJK_class_num:n {#2} = { \c_empty_tl }
  }
%    \end{macrocode}
% \end{macro}
%
% \begin{macro}{\xeCJK_pre_inter_class_toks:nnn}
% 在相邻类别之间已有的内容前增加内容。
%    \begin{macrocode}
\cs_new_nopar:Nn \xeCJK_pre_inter_class_toks:nnn
  {
    \use:x
      {
        \XeTeXinterchartoks \xeCJK_class_num:n {#1} \xeCJK_class_num:n {#2} =
          {
            \exp_not:n {#3} \tex_the:D \XeTeXinterchartoks
              \xeCJK_class_num:n {#1} \xeCJK_class_num:n {#2}
          }
      }
  }
%    \end{macrocode}
% \end{macro}
%
% \begin{macro}{\xeCJK_app_inter_class_toks:nnn,\xeCJK_app_inter_class_toks:nnc}
% 在相邻类别之间已有的内容后追加内容。
%    \begin{macrocode}
\cs_new_nopar:Nn \xeCJK_app_inter_class_toks:nnn
  {
    \use:x
      {
        \XeTeXinterchartoks \xeCJK_class_num:n {#1} \xeCJK_class_num:n {#2} =
          {
            \tex_the:D \XeTeXinterchartoks \xeCJK_class_num:n {#1}
            \xeCJK_class_num:n {#2} \exp_not:n {#3}
          }
      }
  }
\cs_generate_variant:Nn \xeCJK_app_inter_class_toks:nnn { nnc }
%    \end{macrocode}
% \end{macro}
%
% \begin{macro}{\xeCJK_get_inter_class_toks:nn}
% 取出相邻类别之间的内容。
%    \begin{macrocode}
\cs_new_nopar:Nn \xeCJK_get_inter_class_toks:nn
  {
    \tex_the:D \XeTeXinterchartoks \xeCJK_class_num:n {#1} \xeCJK_class_num:n {#2}
  }
%    \end{macrocode}
% \end{macro}
%
% \begin{macro}{\xeCJK_copy_inter_class_toks:nnnn}
% 将 |#3| 和 |#4| 之间的内容复制到 |#1| 和 |#2| 之间。
%    \begin{macrocode}
\cs_new_nopar:Nn \xeCJK_copy_inter_class_toks:nnnn
  {
    \use:x
      {
        \xeCJK_inter_class_toks:nnn {#1} {#2}
          { \xeCJK_get_inter_class_toks:nn {#3} {#4} }
      }
  }
%    \end{macrocode}
% \end{macro}
%
% \begin{macro}{\xeCJK_clear_CJK_toks:}
% 清除边界与 CJK 文字、全角左右标点之间的内容。
%    \begin{macrocode}
\cs_new_nopar:Nn \xeCJK_clear_CJK_toks:
  {
    \clist_map_inline:Nn \g_xeCJK_CJK_class_clist
      { \xeCJK_clear_inter_class_toks:nn { Boundary } {##1} }
  }
%    \end{macrocode}
% \end{macro}
%
% \subsection{字符输出规则}
%
%    \begin{macrocode}
\clist_map_inline:nn { Default, HalfLeft, HalfRight }
  {
    \clist_map_inline:nn { CJK, FullLeft, FullRight }
      {
        \xeCJK_inter_class_toks:nnn {#1} {##1}
          {
            \c_group_begin_token
            \xeCJK_select_font:
            \xeCJK_clear_inter_class_toks:nn {#1} {##1}
            \xeCJK_clear_CJK_toks:
          }
      }
    \xeCJK_app_inter_class_toks:nnn {#1} { CJK } { \CJKsymbol }
    \clist_map_inline:nn { FullLeft, FullRight }
      { \xeCJK_app_inter_class_toks:nnc {#1} {##1} { xeCJK_CJK_and_##1:n } }
    \xeCJK_inter_class_toks:nnn { CJK } {#1} { \c_group_end_token }
  }
%    \end{macrocode}
%
%    \begin{macrocode}
\xeCJK_pre_inter_class_toks:nnn { Default } { CJK } { \CJKecglue }
%    \end{macrocode}
%
%    \begin{macrocode}
\clist_map_inline:nn { Default, HalfLeft, NormalSpace }
  {
    \xeCJK_inter_class_toks:nnn { Boundary } {#1}
      { \int_compare:nNnT \tex_lastkern:D = \c_one { \CJKecglue } }
    \str_if_eq:xxF {#1} { NormalSpace }
      { \xeCJK_app_inter_class_toks:nnn { CJK } {#1} { \CJKecglue } }
  }
%    \end{macrocode}
%
%    \begin{macrocode}
\clist_map_inline:nn { Default, HalfRight, NormalSpace }
  {
    \xeCJK_inter_class_toks:nnn {#1} { Boundary }
      {
        \peek_meaning:NTF \c_space_token
          { { \xeCJK_space_kern: } } { { \xeCJK_zero_kern: } }
      }
  }
%    \end{macrocode}
%
%    \begin{macrocode}
\xeCJK_inter_class_toks:nnn { Boundary } { CJK } { \xeCJK_Boundary_and_CJK: }
%    \end{macrocode}
%
% \begin{macro}{\xeCJK_Boundary_and_CJK:}
%    \begin{macrocode}
\cs_new_nopar:Nn \xeCJK_Boundary_and_CJK:
  {
    {
      \int_compare:nNnTF \tex_lastkern:D = \c_one
        { \CJKglue }
        {
          \bool_if:nTF
            {
              \int_compare_p:nNn \tex_lastkern:D = \c_two      ||
              \int_compare_p:nNn \etex_lastnodetype:D = \c_ten
            }
            { \CJKecglue }
            {
              \int_compare:nNnTF \etex_lastnodetype:D = \c_eleven
                {
                  \settowidth \l_tmpa_dim \CJKecglue
                  \bool_if:nT
                    {
                      \int_compare_p:nNn \tex_lastskip:D =
                        { \tex_fontdimen:D \c_two \tex_font:D }        ||
                      \int_compare_p:nNn \tex_lastskip:D = \l_tmpa_dim
                    }
                    {
                      \skip_set_eq:NN \l_tmpa_skip \tex_lastskip:D
                      \tex_unskip:D
                      \bool_if:nTF
                        {
                          \int_compare_p:nNn \tex_lastkern:D = \c_three    ||
                          \int_compare_p:nNn \etex_lastnodetype:D = \c_ten ||
                          \int_compare_p:nNn \tex_lastkern:D = \c_two      ||
                          \g_xeCJK_itcorr_bool
                        }
                        { \CJKecglue }
                        {
                          \bool_if:nT
                            {
                              ! ( \int_compare_p:nNn \tex_lastkern:D = \c_one ) ||
                              \l_xeCJK_space_bool
                            }
                            { \skip_horizontal:N \l_tmpa_skip }
                        }
                    }
                }
                { \bool_if:NT \g_xeCJK_itcorr_bool \CJKecglue }
            }
        }
      \bool_gset_false:N \g_xeCJK_itcorr_bool
    }
    \c_group_begin_token
    \xeCJK_clear_CJK_toks:
    \xeCJK_select_font:
    \CJKsymbol
  }
%    \end{macrocode}
% \end{macro}
%
%    \begin{macrocode}
\xeCJK_inter_class_toks:nnn { CJK } { Boundary }
  {
    \c_group_end_token
    { \xeCJK_CJK_kern: }
    \xeCJK_ignorespaces:
  }
\xeCJK_inter_class_toks:nnn { CJK } { CJK } { \xeCJK_CJK_and_CJK:n }
%    \end{macrocode}
%
% \begin{macro}{\xeCJK_CJK_and_CJK:n}
%    \begin{macrocode}
\cs_new_nopar:Nn \xeCJK_CJK_and_CJK:n { \CJKglue \CJKsymbol {#1} }
%    \end{macrocode}
% \end{macro}
%
%    \begin{macrocode}
\xeCJK_inter_class_toks:nnn { FullLeft } { CJK } { \nobreak \CJKsymbol }
%    \end{macrocode}
%
%    \begin{macrocode}
\xeCJK_inter_class_toks:nnn { FullRight } { CJK }
  { \xeCJK_after_FullRight: \CJKsymbol }
%    \end{macrocode}
%
%    \begin{macrocode}
\clist_map_inline:nn { Default, HalfLeft, HalfRight, Boundary }
  {
    \xeCJK_inter_class_toks:nnn { FullLeft } {#1} { \nobreak \c_group_end_token }
    \xeCJK_inter_class_toks:nnn { FullRight } {#1}
      { \xeCJK_after_FullRight: \c_group_end_token }
  }
%    \end{macrocode}
%
%    \begin{macrocode}
\clist_map_inline:nn { FullLeft, FullRight }
  { \xeCJK_app_inter_class_toks:nnn {#1} { Boundary } { \tex_ignorespaces:D } }
%    \end{macrocode}
%
%    \begin{macrocode}
\clist_map_inline:nn { CJK, FullLeft, FullRight }
  {
    \xeCJK_inter_class_toks:nnn { NormalSpace } {#1}
      {
        \c_group_begin_token
        \xeCJK_clear_CJK_toks:
        \xeCJK_select_font:
        \CJKsymbol
      }
    \xeCJK_inter_class_toks:nnn {#1} { NormalSpace } { \c_group_end_token }
  }
%    \end{macrocode}
%
%    \begin{macrocode}
\clist_map_inline:nn { FullLeft, FullRight }
  {
    \xeCJK_inter_class_toks:nnn { Boundary } {#1}
      {
        \c_group_begin_token
        \xeCJK_clear_CJK_toks:
        \xeCJK_select_font:
        \use:c { xeCJK_CJK_and_#1:n }
      }
  }
%    \end{macrocode}
%
% \begin{macro}{\xeCJK_punct_rule:n,\xeCJK_punct_rule:c}
% |#1| 为标点符号字体中的的实际左/右空白的负值,用于抹去标点符号的左/右空白。
%    \begin{macrocode}
\cs_new_nopar:Nn \xeCJK_punct_rule:n
  {
    \tex_vrule:D         \@width \dim_eval:n {#1}
    \@depth \c_zero_dim  \@height \c_zero_dim \scan_stop:
  }
\cs_generate_variant:Nn \xeCJK_punct_rule:n { c }
%    \end{macrocode}
% \end{macro}
%
% \begin{macro}{\xeCJK_punct_skip:n,\xeCJK_punct_skip:c}
% |#1| 为根据所选的标点处理方式在标点符号左/右增加的空白。
%    \begin{macrocode}
\cs_new_nopar:Nn \xeCJK_punct_skip:n
  {
    \skip_horizontal:n
      {
        \dim_eval:n {#1}   \@plus
        \dim_eval:n {#1/4} \@minus
        \dim_eval:n {#1/2}
      }
  }
\cs_generate_variant:Nn \xeCJK_punct_skip:n { c }
%    \end{macrocode}
% \end{macro}
%
% \begin{macro}{\xeCJK_punct_kern:n,\xeCJK_punct_kern:c}
% |#1| 为相邻两个标点之间的间距。
%    \begin{macrocode}
\cs_new_nopar:Nn \xeCJK_punct_kern:n { \tex_kern:D \dim_eval:n {#1} }
\cs_generate_variant:Nn \xeCJK_punct_kern:n { c }
%    \end{macrocode}
% \end{macro}
%
% \begin{macro}{\xeCJK_before_FullLeft:n}
%    \begin{macrocode}
\cs_new_nopar:Nn \xeCJK_before_FullLeft:n
  {
    \tl_set:Nx \l_xeCJK_lastpunct_tl {#1}
    \xeCJK_punct_rule:c { \l_xeCJK_current_punct_tl/rule/l/#1 }
    \CJKpunctsymbol {#1}
  }
%    \end{macrocode}
% \end{macro}
%
% \begin{macro}{\xeCJK_after_FullRight:}
%    \begin{macrocode}
\cs_new_nopar:Nn \xeCJK_after_FullRight:
  {
    \xeCJK_get_punct_bounds:nx { r } \l_xeCJK_lastpunct_tl
    \xeCJK_punct_rule:c { \l_xeCJK_current_punct_tl/rule/r/\l_xeCJK_lastpunct_tl }
    \xeCJK_punct_skip:c { \l_xeCJK_current_punct_tl/glue/r/\l_xeCJK_lastpunct_tl }
  }
%    \end{macrocode}
% \end{macro}
%
%    \begin{macrocode}
\clist_map_inline:nn { CJK, FullLeft, FullRight }
  {
    \clist_map_inline:nn { FullLeft, FullRight }
      { \xeCJK_inter_class_toks:nnc {#1} {##1} { xeCJK_#1_and_##1:n } }
  }
%    \end{macrocode}
%
% \begin{macro}{\xeCJK_CJK_and_FullLeft:n}
%    \begin{macrocode}
\cs_new_nopar:Nn \xeCJK_CJK_and_FullLeft:n
  {
    \xeCJK_get_punct_bounds:nn { l } {#1}
    \int_compare:nNnF \etex_lastnodetype:D = \c_one
      { \xeCJK_punct_skip:c { \l_xeCJK_current_punct_tl/glue/l/#1 } }
    \xeCJK_before_FullLeft:n {#1}
    \xeCJK_punct_if_middle:nT {#1}
      {
        \xeCJK_punct_rule:c { \l_xeCJK_current_punct_tl/rule/m/#1 }
        \xeCJK_punct_skip:c { \l_xeCJK_current_punct_tl/glue/l/#1 }
      }
  }
%    \end{macrocode}
% \end{macro}
%
% \begin{macro}{\xeCJK_CJK_and_FullRight:n}
%    \begin{macrocode}
\cs_new_nopar:Nn \xeCJK_CJK_and_FullRight:n
  {
    \xeCJK_get_punct_bounds:nn { r } {#1}
    \xeCJK_punct_if_long:nTF {#1} \CJKglue
      {
        \nobreak
        \xeCJK_punct_if_middle:nT {#1}
          {
            \xeCJK_punct_skip:c { \l_xeCJK_current_punct_tl/glue/r/#1 }
            \xeCJK_punct_rule:c { \l_xeCJK_current_punct_tl/rule/m/#1 }
          }
      }
    \tl_gset:Nx \l_xeCJK_lastpunct_tl {#1}
    \CJKpunctsymbol {#1}
  }
%    \end{macrocode}
% \end{macro}
%
% \begin{macro}{\xeCJK_FullLeft_and_FullLeft:n}
%    \begin{macrocode}
\cs_new_nopar:Nn \xeCJK_FullLeft_and_FullLeft:n
  {
    \nobreak
    \xeCJK_get_punct_bounds:nn { l } {#1}
    \xeCJK_get_kern:nn \l_xeCJK_lastpunct_tl {#1}
    \xeCJK_punct_kern:c
      { \l_xeCJK_current_punct_tl/kern/\l_xeCJK_lastpunct_tl-#1 }
    \xeCJK_before_FullLeft:n {#1}
  }
%    \end{macrocode}
% \end{macro}
%
% \begin{macro}{\xeCJK_FullLeft_and_FullRight:n}
%    \begin{macrocode}
\cs_new_nopar:Nn \xeCJK_FullLeft_and_FullRight:n
  {
    \nobreak
    \xeCJK_get_punct_bounds:nn { r } {#1}
    \xeCJK_get_kern:nn \l_xeCJK_lastpunct_tl {#1}
    \xeCJK_punct_kern:c
      { \l_xeCJK_current_punct_tl/kern/\l_xeCJK_lastpunct_tl-#1 }
    \nobreak
    \tl_set:Nx \l_xeCJK_lastpunct_tl {#1}
    \CJKpunctsymbol {#1}
  }
%    \end{macrocode}
% \end{macro}
%
% \begin{macro}{\xeCJK_FullRight_and_FullLeft:n}
%    \begin{macrocode}
\cs_new_nopar:Nn \xeCJK_FullRight_and_FullLeft:n
  {
    \xeCJK_punct_rule:c
      { \l_xeCJK_current_punct_tl/rule/r/\l_xeCJK_lastpunct_tl }
    \xeCJK_get_punct_bounds:nn { l } {#1}
    \xeCJK_get_kern:nn \l_xeCJK_lastpunct_tl {#1}
    \xeCJK_punct_kern:c
      { \l_xeCJK_current_punct_tl/kern/\l_xeCJK_lastpunct_tl-#1 }
    \xeCJK_punct_nobreak:
    \xeCJK_before_FullLeft:n {#1}
 }
%    \end{macrocode}
% \end{macro}
%
% \begin{macro}{\xeCJK_FullRight_and_FullRight:n}
%    \begin{macrocode}
\cs_new_nopar:Nn \xeCJK_FullRight_and_FullRight:n
  {
    \xeCJK_punct_rule:c
      { \l_xeCJK_current_punct_tl/rule/r/\l_xeCJK_lastpunct_tl }
    \xeCJK_get_punct_bounds:nn { r } {#1}
    \xeCJK_get_kern:nn \l_xeCJK_lastpunct_tl {#1}
    \xeCJK_punct_kern:c
      { \l_xeCJK_current_punct_tl/kern/\l_xeCJK_lastpunct_tl-#1 }
    \nobreak
    \tl_set:Nx \l_xeCJK_lastpunct_tl {#1}
    \CJKpunctsymbol {#1}
  }
%    \end{macrocode}
% \end{macro}
%
% \begin{macro}{\xeCJK_CJK_kern:, \xeCJK_zero_kern:, \xeCJK_space_kern:}
% 用于判断插入的各种 |kern|。
%     \begin{macrocode}
\cs_new_nopar:Nn \xeCJK_CJK_kern:   { \tex_kern:D -1 sp  \tex_kern:D 1 sp }
\cs_new_nopar:Nn \xeCJK_zero_kern:  { \tex_kern:D -2 sp  \tex_kern:D 2 sp }
\cs_new_nopar:Nn \xeCJK_space_kern: { \tex_kern:D -3 sp  \tex_kern:D 3 sp }
%    \end{macrocode}
% \end{macro}
%
% \begin{macro}{CJKglue}
% CJK 文字之间插入的 |glue|。
%    \begin{macrocode}
\keys_define:nn { xeCJK / options } { CJKglue .tl_set_x:N = \CJKglue }
%    \end{macrocode}
% \end{macro}
%
% \begin{macro}{CJKecglue}
% CJK 与西文和数学行内数学公式之间自动添加的空白。
%    \begin{macrocode}
\keys_define:nn { xeCJK / options } { CJKecglue .tl_set_x:N = \CJKecglue }
%    \end{macrocode}
% \end{macro}
%
% \begin{macro}{CJKspace,\CJKspace,\CJKnospace}
% 是否保留 CJK 文字间的空白,默认不保留。
%    \begin{macrocode}
\keys_define:nn { xeCJK / options }
  {
    CJKspace .choice:,
    CJKspace / true  .code:n = { \CJKspace   },
    CJKspace / false .code:n = { \CJKnospace },
    CJKspace      .default:n = { true },
  }
\NewDocumentCommand \CJKspace { }
  {
    \bool_set_true:N \l_xeCJK_space_bool
    \cs_set_eq:NN \xeCJK_ignorespaces: \xeCJK_peek_math:
  }
\NewDocumentCommand \CJKnospace { }
  {
    \bool_set_false:N \l_xeCJK_space_bool
    \cs_set_eq:NN \xeCJK_ignorespaces: \xeCJK_ignore_spaces:
  }
%    \end{macrocode}
% \end{macro}
%
% \begin{macro}{\xeCJK_ignore_spaces:,\xeCJK_peek_math:}
%     \begin{macrocode}
\cs_new_nopar:Nn \xeCJK_ignore_spaces:
  {
    \peek_meaning:NTF \c_space_token
      {
        \peek_catcode_ignore_spaces:NTF \c_math_toggle_token
          { \CJKecglue }
          { \bool_if:nT { \token_if_cs_p:N \l_peek_token } \CJKecglue }
      }
      {
        \token_if_math_toggle:NTF \l_peek_token \CJKecglue
          { \cs_if_exist:NF \l_peek_token \exp_not:N }
      }
  }
\cs_new_nopar:Nn \xeCJK_peek_math:
  {
    \peek_catcode:NTF \c_math_toggle_token \CJKecglue
      { \cs_if_exist:NF \l_peek_token \exp_not:N }
  }
%    \end{macrocode}
% \end{macro}
%
% \begin{macro}{CJKchecksingle,CJKchecksingleglue}
% 避免单个汉字单独占一行。
%    \begin{macrocode}
\keys_define:nn { xeCJK / options }
  {
    CJKchecksingle .choice:,
    CJKchecksingle / true  .code:n = { \xeCJKenablechecksingle  },
    CJKchecksingle / false .code:n = { \xeCJKdisablechecksingle },
    CJKchecksingle      .default:n = { true },
    CJKchecksingleglue .tl_set_x:N = { \CJKchecksingleglue },
  }
%    \end{macrocode}
% \end{macro}
%
% \begin{macro}{\xeCJKenablechecksingle,\xeCJKdisablechecksingle}
% 避免单个汉字单独占一行。
%    \begin{macrocode}
\bool_new:N \l_xeCJK_checksingle_bool
\NewDocumentCommand \xeCJKenablechecksingle { }
  {
    \bool_if:NF \l_xeCJK_checksingle_bool
      {
        \bool_set_true:N \l_xeCJK_checksingle_bool
        \cs_set_eq:NN \xeCJK_checksingle_save_CJKsymbol:n \xeCJK_CJK_and_CJK:n
        \cs_set_eq:NN \xeCJK_checksingle_save_CJKglue \CJKglue
        \cs_set_eq:NN \xeCJK_CJK_and_CJK:n \xeCJK_checksingle:n
        \cs_set_nopar:Npn \CJKglue { \CJKchecksingleglue }
      }
  }
\NewDocumentCommand \xeCJKdisablechecksingle { }
  {
    \bool_if:NT \l_xeCJK_checksingle_bool
      {
        \bool_set_false:N \l_xeCJK_checksingle_bool
        \cs_set_eq:NN \CJKglue \xeCJK_checksingle_save_CJKglue
        \cs_set_eq:NN \xeCJK_CJK_and_CJK:n \xeCJK_checksingle_save_CJKsymbol:n
      }
  }
%    \end{macrocode}
% \end{macro}
%
% \begin{macro}{\xeCJK_checksingle:n,\xeCJK_checksingle:nn,\xeCJK_checksingle:nnn}
%    \begin{macrocode}
\cs_new_nopar:Nn \xeCJK_checksingle:n
  {
    \peek_catcode:NTF \c_catcode_other_token
      { \xeCJK_checksingle:nn {#1} }
      {
        \bool_if:nTF
          {
            \tl_if_empty_p:x { \token_get_arg_spec:N \l_peek_token } &&
            \exp_args:No \tl_if_single_token_p:n \l_peek_token       &&
            \exp_after:wN \token_if_other_p:N \l_peek_token
          }
          { \exp_args:Nno \xeCJK_checksingle:nn {#1} }
          { \xeCJK_checksingle_save_CJKsymbol:n {#1} }
      }
  }
\cs_new_nopar:Nn \xeCJK_checksingle:nn
  {
    \peek_catcode:NTF \c_catcode_other_token
      { \xeCJK_checksingle_save_CJKsymbol:n {#1} #2 }
      {
        \bool_if:nTF
          {
            \token_if_space_p:N       \l_peek_token ||
            \token_if_math_toggle_p:N \l_peek_token ||
            \token_if_cs_p:N          \l_peek_token
          }
          {
            \bool_if:nTF { \token_if_space_p:N \l_peek_token }
              {
                \peek_catcode_ignore_spaces:NTF \c_catcode_other_token
                  { \xeCJK_checksingle_save_CJKsymbol:n {#1} #2 \c_space_token }
                  {
                    \bool_if:nTF
                      {
                        \token_if_math_toggle_p:N \l_peek_token ||
                        \token_if_cs_p:N          \l_peek_token
                      }
                      { \xeCJK_checksingle:nnn {#1} {#2} \c_space_token }
                      { \xeCJK_checksingle_save_CJKsymbol:n {#1} #2 \c_space_token }
                  }
              }
              { \xeCJK_checksingle:nnn {#1} {#2} { } }
          }
          { \xeCJK_checksingle_save_CJKsymbol:n {#1} #2 }
      }
  }
\cs_new_nopar:Nn \xeCJK_checksingle:nnn
  {
    \bool_if:nTF
      {
        \token_if_eq_meaning_p:NN \l_peek_token \par ||
        \token_if_eq_meaning_p:NN \l_peek_token \[ % \]
      }
      { \CJKsymbol {#1} \nobreak #2 }
      { \xeCJK_checksingle_env:nnn {#1} {#2} {#3} }
  }
%    \end{macrocode}
% \end{macro}
%
% \begin{macro}{\xeCJK_checksingle_env:nnn,\xeCJK_checksingle_env:nnnNn}
%    \begin{macrocode}
\cs_new_nopar:Nn \xeCJK_checksingle_env:nnn
  {
    \peek_catcode_remove:NTF \c_math_toggle_token
      {
        \peek_catcode:NTF \c_math_toggle_token
          { \CJKsymbol {#1} \nobreak #2 \c_math_toggle_token }
          { \xeCJK_checksingle_save_CJKsymbol:n {#1} #2#3 \c_math_toggle_token }
      }
      {
        \peek_meaning_remove:NTF \begin
          { \xeCJK_checksingle_env:nnnNn {#1} {#2} {#3} \begin }
          {
            \peek_meaning_remove:NTF \end
              { \xeCJK_checksingle_env:nnnNn {#1} {#2} {#3} \end }
              { \xeCJK_checksingle_save_CJKsymbol:n {#1} #2#3 }
          }
      }
  }
\cs_new_nopar:Nn \xeCJK_checksingle_env:nnnNn
  {
    \clist_if_in:NnTF \g_xeCJK_inline_env_clist {#5}
      { \xeCJK_checksingle_save_CJKsymbol:n {#1} #2#3 }
      { \CJKsymbol {#1} \nobreak #2 }
    #4 {#5}
  }
%    \end{macrocode}
% \end{macro}
%
% \begin{macro}{\g_xeCJK_inline_env_clist}
%    \begin{macrocode}
\clist_new:N \g_xeCJK_inline_env_clist
%    \end{macrocode}
% \end{macro}
%
% \begin{macro}{InlineEnv,InlineEnv+,InlineEnv-}
%    \begin{macrocode}
\keys_define:nn { xeCJK / options }
  {
    InlineEnv .clist_gset:N = \g_xeCJK_inline_env_clist ,
    InlineEnv+      .code:n =
      {
        \clist_gput_right:Nx \g_xeCJK_inline_env_clist {#1}
        \clist_gremove_duplicates:N \g_xeCJK_inline_env_clist
      },
    InlineEnv-      .code:n =
      {
        \clist_map_inline:xn {#1}
          { \clist_gremove_all:Nn \g_xeCJK_inline_env_clist {##1} }
      },
  }
%    \end{macrocode}
% \end{macro}
%
% \subsection{增加 CJK 子分区}
%
% \begin{macro}{\xeCJK_UL_subclass_patch_tl,\g_xeCJK_CJK_subclass_clist}
%    \begin{macrocode}
\tl_new:N \xeCJK_UL_subclass_patch_tl
\clist_new:N \g_xeCJK_CJK_subclass_clist
%    \end{macrocode}
% \end{macro}
%
% \begin{macro}{\xeCJK_erase_CJKsymbol:, \xeCJK_restore_CJKsymbol:}
%    \begin{macrocode}
\cs_new_nopar:Nn \xeCJK_erase_CJKsymbol:
  {
    \cs_gset_eq:NN \CJKsymbol_Block \CJKsymbol
    \cs_gset_eq:NN \CJKsymbol \prg_do_nothing:
  }
\cs_new_nopar:Nn \xeCJK_restore_CJKsymbol:
  {
    \cs_gset_eq:NN \CJKsymbol \CJKsymbol_Block
    \CJKsymbol
  }
%    \end{macrocode}
% \end{macro}
%
% \begin{macro}{\xeCJKDeclareSubCJKBlock}
% 声明 CJK 子区范围,|#1| 为自定义名称,|#2| 为子区的 |Unicode| 范围。
%    \begin{macrocode}
\NewDocumentCommand \xeCJKDeclareSubCJKBlock { m m }
  { \xeCJKDeclareSubCharClass { CJK } {#1} {#2} }
\@onlypreamble \xeCJKDeclareSubCJKBlock
%    \end{macrocode}
% \end{macro}
%
% \begin{macro}{\xeCJKCancelSubCJKBlock,\xeCJKRestoreSubCJKBlock}
% 取消和恢复对 CJK 子区的声明。
%    \begin{macrocode}
\bool_new:N \l_xeCJK_sub_restore_bool
\NewDocumentCommand \xeCJKCancelSubCJKBlock { m }
  {
    \bool_set_false:N \l_xeCJK_sub_restore_bool
    \xeCJK_sub_restore_or_cancel:n {#1}
  }
\NewDocumentCommand \xeCJKRestoreSubCJKBlock { m }
  {
    \bool_set_true:N \l_xeCJK_sub_restore_bool
    \xeCJK_sub_restore_or_cancel:n {#1}
  }
\cs_new_nopar:Nn \xeCJK_sub_restore_or_cancel:n
  {
    \clist_map_inline:xn {#1}
      {
        \cs_if_free:cTF { xeCJK_ CJK/##1 _class }
          { \xeCJK_error:nx { SubBlock-undefined } {##1} }
          {
            \xeCJKDeclareCharClass
              { CJK \bool_if:NT \l_xeCJK_sub_restore_bool { /##1 } }
              { \clist_use:c { g_xeCJK_CJK/##1_range_clist } }
          }
      }
  }
\xeCJK_msg_new:nn { SubBlock-undefined }
  {
    The~CJK~sub~block~'#1'~is~undefined.\\\\
    Try~to~use~\token_to_str:N \xeCJKDeclareSubCJKBlock \c_space_tl
    to~declare~it.
  }
%    \end{macrocode}
% \end{macro}
%
% \begin{macro}{\xeCJKDeclareSubCharClass}
%    \begin{macrocode}
\NewDocumentCommand \xeCJKDeclareSubCharClass
  { > { \TrimSpaces } m > { \TrimSpaces } m m }
  {
    \cs_if_free:cT { xeCJK_ #1/#2 _class }
      {
        \xeCJK_new_class:n { #1/#2 }
        \xeCJK_set_sub_class_toks:nn {#1} {#2}
        \xeCJK_new_sub_key:n {#2}
        \xeCJK_UL_subclass_patch:nn {#1} {#2}
      }
    \xeCJKDeclareCharClass { #1/#2 } {#3}
  }
\@onlypreamble \xeCJKDeclareSubCharClass
%    \end{macrocode}
% \end{macro}
%
% \begin{macro}{\xeCJK_set_sub_class_toks:nn}
%    \begin{macrocode}
\cs_new_nopar:Nn \xeCJK_set_sub_class_toks:nn
  {
    \clist_map_inline:Nn \g_xeCJK_base_class_clist
      {
        \xeCJK_copy_inter_class_toks:nnnn { #1/#2 } {##1} {#1}  {##1}
        \xeCJK_copy_inter_class_toks:nnnn {##1} { #1/#2 } {##1} {#1}
        \str_if_eq:xxTF {##1} { CJK }
          {
            \xeCJK_pre_inter_class_toks:nnn {##1} { #1/#2 }
              { \xeCJK_switch_font:nn {#1} {#2} }
          }
          {
            \xeCJK_pre_inter_class_toks:nnn {##1} { #1/#2 }
              { \xeCJK_erase_CJKsymbol: }
            \xeCJK_app_inter_class_toks:nnn {##1} { #1/#2 }
              { \xeCJK_switch_font:nn {#1} {#2} \xeCJK_restore_CJKsymbol: }
          }
      }
    \xeCJK_copy_inter_class_toks:nnnn { #1/#2 } { #1/#2 } {#1} {#1}
    \clist_if_empty:NF \g_xeCJK_CJK_subclass_clist
      {
        \clist_map_inline:Nn \g_xeCJK_CJK_subclass_clist
          {
            \xeCJK_copy_inter_class_toks:nnnn { #1/#2  } { #1/##1 } {#1} {#1}
            \xeCJK_copy_inter_class_toks:nnnn { #1/##1 } { #1/#2  } {#1} {#1}
            \xeCJK_pre_inter_class_toks:nnn { #1/#2 } { #1/##1 }
              { \xeCJK_switch_font:nn {#2} {##1} }
            \xeCJK_pre_inter_class_toks:nnn { #1/##1 } { #1/#2 }
              { \xeCJK_switch_font:nn {##1} {#2} }
          }
      }
    \clist_gput_right:Nn \g_xeCJK_CJK_subclass_clist {#2}
    \xeCJK_save_CJK_class:n { #1/#2 }
    \clist_map_inline:nn { CJK, FullLeft, FullRight }
      {
        \xeCJK_pre_inter_class_toks:nnn { #1/#2 } {##1}
          { \xeCJK_switch_font:nn {#2} {#1} }
      }
  }
%    \end{macrocode}
% \end{macro}
%
% \begin{macro}{\xeCJK_UL_subclass_patch:nn}
%    \begin{macrocode}
\cs_new_nopar:Nn \xeCJK_UL_subclass_patch:nn
  {
    \tl_put_right:Nn \xeCJK_UL_subclass_patch_tl
      {
        \clist_map_inline:nn { Default, HalfLeft, HalfRight }
          {
            \str_if_eq:xxTF {##1} { HalfLeft }
              { \xeCJK_inter_class_toks:nnn {#1} { #1/#2 } { \CJKecglue \CJKsymbol } }
              { \xeCJK_inter_class_toks:nnn {#1} { #1/#2 } { \CJKsymbol } }
            \xeCJK_inter_class_toks:nnn { Boundary } { #1/#2 }
              {
                \xeCJK_erase_CJKsymbol:
                \xeCJK_Boundary_and_CJK:
                \xeCJK_switch_font:nn {#1} {#2}
                \xeCJK_restore_CJKsymbol:
              }
          }
      }
  }
%    \end{macrocode}
% \end{macro}
%
% \subsection{标点处理}
%
% \begin{macro}{\g_xeCJK_punctstyle_clist}
% 标点处理方式。
%    \begin{macrocode}
\clist_new:N \g_xeCJK_punctstyle_clist
\clist_set:Nn \g_xeCJK_punctstyle_clist
  { CCT , halfwidth , fullwidth , marginkerning , mixedwidth , plain }
\clist_map_inline:Nn \g_xeCJK_punctstyle_clist
  { \tl_const:cn { c_xeCJK_ps_#1_tl } {#1} }
%    \end{macrocode}
% \end{macro}
%
% \begin{macro}{\keys_choices_generate:V, .generate_choices:V}
%    \begin{macrocode}
\cs_generate_variant:Nn \keys_choices_generate:n { V }
\cs_new_protected:cpn { \c_keys_props_root_tl .generate_choices:V } #1
  { \keys_choices_generate:V {#1} }
%    \end{macrocode}
% \end{macro}
%
% \begin{macro}
%    {CJKallowbreakbetweenpuncts, PunctStyle, KaiMingPunct, LongPunct, MiddlePunct}
% 相关选项声明。
%    \begin{macrocode}
\keys_define:nn { xeCJK / options }
  {
    CJKallowbreakbetweenpuncts .choice:,
    CJKallowbreakbetweenpuncts / true  .code:n = { \xeCJKallowbreakbetweenpuncts },
    CJKallowbreakbetweenpuncts / false .code:n = { \xeCJKnobreakbetweenpuncts },
    CJKallowbreakbetweenpuncts      .default:n = { true },
    KaiMingPunct  .code:n = { \xeCJK_set_special_punct:nn { mixedwidth } {#1} },
    KaiMingPunct+ .code:n = { \xeCJK_add_special_punct:nn { mixedwidth } {#1} },
    KaiMingPunct- .code:n = { \xeCJK_sub_special_punct:nn { mixedwidth } {#1} },
    LongPunct     .code:n = { \xeCJK_set_special_punct:nn { long } {#1} },
    LongPunct+    .code:n = { \xeCJK_add_special_punct:nn { long } {#1} },
    LongPunct-    .code:n = { \xeCJK_sub_special_punct:nn { long } {#1} },
    MiddlePunct+  .code:n = { \xeCJK_add_special_punct:nn { middle } {#1} },
    MiddlePunct   .code:n = { \xeCJK_set_special_punct:nn { middle } {#1} },
    MiddlePunct-  .code:n = { \xeCJK_sub_special_punct:nn { middle } {#1} },
    MiddlePunctWidth .dim_set:N = \l_xeCJK_middle_ps_wd_dim ,
    PunctStyle .choice_code:n =
      {
        \tl_set:Nx \l_xeCJK_punctstyle_tl { \l_keys_choice_tl }
        \tl_if_eq:NNT \l_xeCJK_punctstyle_tl \c_xeCJK_ps_plain_tl
                      { \xeCJKallowbreakbetweenpuncts }
      },
    PunctStyle .generate_choices:V = \g_xeCJK_punctstyle_clist,
    PunctStyle / banjiao       .meta:n = { PunctStyle = halfwidth },
    PunctStyle / quanjiao      .meta:n = { PunctStyle = fullwidth },
    PunctStyle / kaiming       .meta:n = { PunctStyle = mixedwidth },
    PunctStyle / hangmobanjiao .meta:n = { PunctStyle = marginkerning },
    PunctStyle / unknown .code:n =
      { \xeCJK_error:nx { PunctStyle-undefined } \l_keys_value_tl },
  }
\xeCJK_msg_new:nn { PunctStyle-undefined }
  {
    Punctstyle~"#1"~is~not~available. \\\\
    You~can~only~use~one~of\\\\
    "\g_xeCJK_punctstyle_clist".\\
  }
%    \end{macrocode}
% \end{macro}
%
% \begin{macro}{\xeCJKallowbreakbetweenpuncts, \xeCJKnobreakbetweenpuncts}
%     \begin{macrocode}
\NewDocumentCommand \xeCJKallowbreakbetweenpuncts { }
  { \cs_set_eq:NN \xeCJK_punct_nobreak: \xeCJK_zero_glue: }
\NewDocumentCommand \xeCJKnobreakbetweenpuncts { }
  { \cs_set_eq:NN \xeCJK_punct_nobreak: \nobreak }
\cs_new_protected_nopar:Nn \xeCJK_zero_glue: { \skip_horizontal:N \c_zero_skip }
%    \end{macrocode}
% \end{macro}
%
% \begin{macro}
% {\g_xeCJK_special_ps_clist,   \xeCJK_set_special_punct:nn,
%  \xeCJK_add_special_punct:nn, \xeCJK_sub_special_punct:nn}
%    \begin{macrocode}
\clist_set:Nn \g_xeCJK_special_ps_clist { mixedwidth , long , middle }
\clist_map_inline:Nn \g_xeCJK_special_ps_clist
  {
    \tl_new:c    { l_xeCJK_#1_punct_tl   }
    \prop_new:c  { l_xeCJK_#1_punct_prop }
  }
\cs_new_nopar:Nn \xeCJK_set_special_punct:nn
  {
    \tl_set:cx { l_xeCJK_#1_punct_tl } {#2}
    \prop_clear:c { l_xeCJK_#1_punct_prop }
    \tl_map_inline:cn { l_xeCJK_#1_punct_tl }
      { \prop_put:cnn  { l_xeCJK_#1_punct_prop } {##1} { } }
  }
\cs_new_nopar:Nn \xeCJK_add_special_punct:nn
  {
    \tl_put_right:cx { l_xeCJK_#1_punct_tl } {#2}
    \tl_map_inline:cn { l_xeCJK_#1_punct_tl }
      { \prop_put:cnn  { l_xeCJK_#1_punct_prop } {##1} { } }
  }
\cs_new_nopar:Nn \xeCJK_sub_special_punct:nn
  {
    \tl_map_inline:xn {#2}
      {
        \tl_remove_all:cn { l_xeCJK_#1_punct_tl } {##1}
        \prop_del:cn { l_xeCJK_#1_punct_prop } {##1}
      }
  }
\cs_generate_variant:Nn \tl_map_inline:nn { x }
%    \end{macrocode}
% \end{macro}
%
% \begin{macro}{\dim_set:Nv,\dim_add:Nv,\dim_set_min:Nv}
%    \begin{macrocode}
\cs_generate_variant:Nn \dim_set:Nn { Nv }
\cs_generate_variant:Nn \dim_add:Nn { Nv }
\cs_generate_variant:Nn \dim_set_min:Nn { Nv }
%    \end{macrocode}
% \end{macro}
%
% \begin{macro}{\l_xeCJK_current_punct_tl}
% 记录当前字体状态下的标点处理方式。
%    \begin{macrocode}
\tl_set:Nn \l_xeCJK_current_punct_tl
  { \l_xeCJK_font_coor_tl/\l_xeCJK_punctstyle_tl }
%    \end{macrocode}
% \end{macro}
%
% \begin{macro}{\xeCJK_get_punct_bounds:nn}
% |#1| 为 |l/r|,|#2| 为标点字符,返回标点的实际左/右空白的负值和根据标点处理方式
% 决定添加的左/右空白。全角方式直接添加标点的实际左/右空白;半角方式添加标点左右空白
% 中的最小值;开明方式空半角,只在开明标点再增加实际左/右空白的一半。
%     \begin{macrocode}
\cs_new_nopar:Nn \xeCJK_get_punct_bounds:nn
  {
    \cs_if_free:cT { \l_xeCJK_current_punct_tl/rule/#1/#2 }
      {
        \tl_if_eq:NNTF \l_xeCJK_punctstyle_tl \c_xeCJK_ps_plain_tl
          {
            \clist_map_inline:nn { glue , rule , space }
              { \tl_gset:cV { \l_xeCJK_current_punct_tl/##1/#1/#2 } \c_zero_dim }
            \tl_gset:cV { \l_xeCJK_current_punct_tl/rule/m/#2 } \c_zero_dim
          }
          {
            \cs_if_free:cT { \l_xeCJK_current_punct_tl/space/#1/#2 }
              { { \xeCJK_select_font: \xeCJK_get_punct_dimen:n {#2} } }
            \xeCJK_punct_if_long:nTF {#2}
              {
                \dim_zero:N \l_tmpa_dim
                \dim_zero:N \l_tmpb_dim
                \dim_zero:N \l_tmpc_dim
              }
              {
                \dim_set:Nv \l_tmpa_dim { \l_xeCJK_current_punct_tl/space/#1/#2 }
                \dim_set:Nv \l_tmpb_dim
                  { \l_xeCJK_current_punct_tl/space/\str_if_eq:xxTF {#1} lrl/#2 }
                \xeCJK_punct_if_middle:nTF {#2}
                  {
                    \dim_set:Nn \l_tmpc_dim
                      {
                        \dim_compare:nNnTF \l_xeCJK_middle_ps_wd_dim = \c_zero_dim
                          {
                            \prg_case_tl:Nnn \l_xeCJK_punctstyle_tl
                              {
                                \c_xeCJK_ps_halfwidth_tl  { .5 em }
                                \c_xeCJK_ps_mixedwidth_tl { .5 em }
                                \c_xeCJK_ps_CCT_tl        { .7 em }
                                \c_xeCJK_ps_fullwidth_tl  {  1 em }
                              }
                              { 1 em }
                          }
                          { \l_xeCJK_middle_ps_wd_dim }
                        - \tl_use:c { \l_xeCJK_current_punct_tl/dimen/#2 }
                      }
                    \dim_set:Nn \l_tmpc_dim { .5 \l_tmpc_dim }
                  }
                  {
                    \dim_set_eq:NN  \l_tmpc_dim \l_tmpa_dim
                    \dim_set_min:Nn \l_tmpc_dim \l_tmpb_dim
                    \prg_case_tl:Nnn \l_xeCJK_punctstyle_tl
                      {
                        \c_xeCJK_ps_halfwidth_tl { \prg_do_nothing: }
                        \c_xeCJK_ps_mixedwidth_tl
                          {
                            \xeCJK_punct_if_mixedwidth:nT {#2}
                              {
                                \dim_add:Nn \l_tmpc_dim { .5 \l_tmpa_dim }
                                \dim_set_min:Nn \l_tmpc_dim \l_tmpa_dim
                              }
                          }
                        \c_xeCJK_ps_CCT_tl
                          {
                            \xeCJK_punct_if_mixedwidth:nTF {#2}
                              { \dim_add:Nn \l_tmpc_dim { .5 \l_tmpa_dim } }
                              { \dim_add:Nn \l_tmpc_dim { .2 \l_tmpa_dim } }
                            \dim_set_min:Nn \l_tmpc_dim \l_tmpa_dim
                          }
                      }
                      { \dim_set_eq:NN \l_tmpc_dim \l_tmpa_dim }
                  }
                \dim_set_max:Nn \l_tmpc_dim \c_zero_dim
              }
            \tl_gset:cx { \l_xeCJK_current_punct_tl/rule/#1/#2 }
                        { - \dim_use:N \l_tmpa_dim }
            \tl_gset:cx { \l_xeCJK_current_punct_tl/rule/m/#2  }
                        { - \dim_use:N \l_tmpb_dim }
            \tl_gset:cV { \l_xeCJK_current_punct_tl/glue/#1/#2 } \l_tmpc_dim
          }
      }
  }
\cs_generate_variant:Nn \xeCJK_get_punct_bounds:nn { nx }
%    \end{macrocode}
% \end{macro}
%
% \begin{macro}[pTF]{\xeCJK_punct_if_right:n}
% \begin{macro}[TF]{\xeCJK_punct_if_long:n}
% \begin{macro}[TF]{\xeCJK_punct_if_middle:n}
% \begin{macro}[TF]{\xeCJK_punct_if_mixedwidth:n}
%    \begin{macrocode}
\prg_new_conditional:Nnn \xeCJK_punct_if_right:n { p , T , F , TF }
  {
    \cs_if_free:cTF { \l_xeCJK_current_punct_tl/glue/r/#1 }
      \prg_return_false: \prg_return_true:
  }
\cs_generate_variant:Nn \prg_new_conditional:Nnn { c }
\clist_map_inline:Nn \g_xeCJK_special_ps_clist
  {
    \prg_new_conditional:cnn { xeCJK_punct_if_#1:n } { p , T , F , TF }
      {
        \prop_if_in:coTF { l_xeCJK_#1_punct_prop } {##1}
          \prg_return_true: \prg_return_false:
      }
  }
%    \end{macrocode}
% \end{macro}
% \end{macro}
% \end{macro}
% \end{macro}
%
% \begin{macro}{\xeCJK_get_kern:nn}
% 标点挤压。行末半角方式,相邻标点不挤压;全角方式,相邻两个标点占 1.5 汉字宽度,
% 半角、开明和 CCT 方式相邻标点占一个汉字宽度。
%    \begin{macrocode}
\cs_new_nopar:Nn \xeCJK_get_kern:nn
  {
    \cs_if_free:cT { \l_xeCJK_current_punct_tl/kern/#1-#2 }
      {
        \dim_zero:N \l_tmpa_dim
        \prg_case_tl:Nnn \l_xeCJK_punctstyle_tl
          {
            \c_xeCJK_ps_plain_tl { \prg_do_nothing: }
            \c_xeCJK_ps_marginkerning_tl
              {
                \xeCJK_punct_if_right:nT {#1}
                  { \dim_add:Nv \l_tmpa_dim { \l_xeCJK_current_punct_tl/glue/r/#1 } }
                \xeCJK_punct_if_right:nF {#2}
                  { \dim_add:Nv \l_tmpa_dim { \l_xeCJK_current_punct_tl/glue/l/#2 } }
              }
          }
          {
            \tl_if_eq:NNTF \l_xeCJK_punctstyle_tl \c_xeCJK_ps_fullwidth_tl
              { \xeCJK_get_kern_aux:Nnnn \l_tmpa_dim { 1.5 em } {#1} {#2} }
              {
                \bool_if:nF
                  {
                    ( \xeCJK_punct_if_right_p:n {#1} &&
                      \xeCJK_punct_if_right_p:n {#2} )
                    || ( ! ( \xeCJK_punct_if_right_p:n {#1} ) &&
                         ! ( \xeCJK_punct_if_right_p:n {#2} ) )
                  }
                  {
                    \xeCJK_get_kern_aux:Nnnn \l_tmpa_dim { 1 em } {#1} {#2}
                    \dim_compare:nNnT \l_tmpa_dim < { .1 em }
                      {
                        \xeCJK_punct_if_right:nF {#2}
                          {
                            \dim_set:Nv \l_tmpa_dim
                              { \l_xeCJK_current_punct_tl/space/l/#1 }
                            \dim_set_min:Nv \l_tmpa_dim
                              { \l_xeCJK_current_punct_tl/space/r/#1 }
                          }
                      }
                    \xeCJK_punct_if_right:nT {#2}
                      {
                        \dim_set_min:Nv \l_tmpa_dim
                          { \l_xeCJK_current_punct_tl/glue/r/#2 }
                      }
                  }
              }
            \bool_if:nT
              { \xeCJK_punct_if_long_p:n {#1} || \xeCJK_punct_if_long_p:n {#2} }
              { \dim_set_max:Nn \l_tmpa_dim { .1 em } }
          }
        \dim_set_max:Nn \l_tmpa_dim \c_zero_dim
        \tl_gset:cV { \l_xeCJK_current_punct_tl/kern/#1-#2 } \l_tmpa_dim
      }
  }
%    \end{macrocode}
% \end{macro}
%
% \begin{macro}{\xeCJK_get_kern_aux:Nnnn}
% |#3| 和 |#4| 为相邻的两个标点,|#2| 为要确定的相邻两个标点总共占的宽度,|#1| 是
% 尺寸寄存器,用于记录根据所占宽度计算得出的相邻两个标点间距。
%    \begin{macrocode}
\cs_new_nopar:Nn \xeCJK_get_kern_aux:Nnnn
  {
    \dim_set:Nn #1
      {
        #2
        - \tl_use:c { \l_xeCJK_current_punct_tl/
                      \xeCJK_punct_if_right:nTF {#3} { space } { glue } /l/#3 }
        - \tl_use:c { \l_xeCJK_current_punct_tl/
                      \xeCJK_punct_if_right:nTF {#4} { glue } { space } /r/#4 }
        - \tl_use:c { \l_xeCJK_current_punct_tl/dimen/#3 }
        - \tl_use:c { \l_xeCJK_current_punct_tl/dimen/#4 }
      }
  }
%    \end{macrocode}
% \end{macro}
%
% \begin{macro}{\xeCJK_get_punct_dimen:n}
% 返回标点的左右空白和实际尺寸。对于破折号,计算两标点之间的空白,保证它中间不被断开。
%    \begin{macrocode}
\cs_new_nopar:Nn \xeCJK_get_punct_dimen:n
  {
    \xeCJK_get_glyph_bounds:nnN { left } {#1} \l_tmpa_dim
    \tl_gset:cV { \l_xeCJK_current_punct_tl/space/l/#1 } \l_tmpa_dim
    \dim_set_eq:NN \l_tmpb_dim \l_tmpa_dim
    \xeCJK_get_glyph_bounds:nnN { right } {#1} \l_tmpa_dim
    \tl_gset:cV { \l_xeCJK_current_punct_tl/space/r/#1 } \l_tmpa_dim
    \dim_add:Nn \l_tmpb_dim \l_tmpa_dim
    \tl_gset:cx { \l_xeCJK_current_punct_tl/dimen/#1 }
      { \dim_eval:n { \etex_fontcharwd:D \tex_font:D `#1 - \l_tmpb_dim } }
    \bool_if:nT
      {
        \xeCJK_punct_if_long_p:n {#1} &&
        ! ( \str_if_eq_p:xx {#1} { … } || \str_if_eq_p:xx {#1} { ‥ } )
      }
      {
        \clist_map_inline:Nn \g_xeCJK_punctstyle_clist
          {
            \str_if_eq:xxF \c_xeCJK_ps_plain_tl {##1}
              {
                \tl_gset:cx { \l_xeCJK_font_coor_tl/##1/kern/#1-#1 }
                  { - \dim_use:N \l_tmpb_dim }
              }
          }
      }
  }
%    \end{macrocode}
% \end{macro}
%
% \begin{macro}{\xeCJK_get_glyph_bounds:nnN}
% 用 |\XeTeXglyphbounds| 计算标点的上下左右空白。
%    \begin{macrocode}
\cs_new_nopar:Nn \xeCJK_get_glyph_bounds:nnN
  {
    \dim_set:Nn #3
      {
        \XeTeXglyphbounds
          \prg_case_str:xxn {#1}
            {
                { left   } { \c_one   }     { top    } { \c_two  }
                { right  } { \c_three }     { bottom } { \c_four }
            }
            { \c_three }
          \int_eval:n { \XeTeXcharglyph `#2 }
      }
  }
%    \end{macrocode}
% \end{macro}
%
% \begin{macro}{\xeCJKsetkern}
% 手动设置相邻标点的距离,仅在当前字体状态下有效。
%    \begin{macrocode}
\NewDocumentCommand \xeCJKsetkern  { m m m }
  { \tl_gset:cx { \l_xeCJK_current_punct_tl/kern/#1-#2 } { \dim_eval:n {#3} } }
%    \end{macrocode}
% \end{macro}
%
% \subsection{后备字体}
%
% \begin{macro}{AutoFallBack, FallBack}
% 后备字体的宏包选项声明。
%    \begin{macrocode}
\keys_define:nn { xeCJK / options }
  {
    AutoFallBack .choice:,
    AutoFallBack / true  .code:n = { \xeCJKenablefallback },
    AutoFallBack / false .code:n = { \xeCJKdisablefallback },
    AutoFallBack      .default:n = { true },
    FallBack    .meta:n = { AutoFallBack = #1 },
    fallback    .meta:n = { AutoFallBack = #1 },
    FallBack .default:n = { true },
    fallback .default:n = { true },
  }
%    \end{macrocode}
% \end{macro}
%
% \begin{macro}{\xeCJKenablefallback, \xeCJKdisablefallback}
% 后备字体的启用开关,默认关闭。
%     \begin{macrocode}
\bool_new:N \l_xeCJK_fallback_bool
\NewDocumentCommand \xeCJKenablefallback { }
  {
    \bool_if:NF \l_xeCJK_fallback_bool
      {
        \bool_set_true:N \l_xeCJK_fallback_bool
        \cs_set_eq:NN \xeCJK_fallback_save_CJKsymbol \CJKsymbol
        \cs_set_eq:NN \CJKsymbol \xeCJK_fallback_testsymbol:n
      }
  }
\NewDocumentCommand \xeCJKdisablefallback { }
  {
    \bool_if:NT \l_xeCJK_fallback_bool
      {
        \bool_set_false:N \l_xeCJK_fallback_bool
        \cs_set_eq:NN \CJKsymbol \xeCJK_fallback_save_CJKsymbol
      }
  }
%    \end{macrocode}
% \end{macro}
%
% \begin{macro}{\xeCJK_fallback_testsymbol:n}
% 测试当前字体中是否存在当前字符,如存在则直接输出,否则启用后备字体。
%     \begin{macrocode}
\cs_new_nopar:Nn \xeCJK_fallback_testsymbol:n
  {
    \font_glyph_if_exist:NnTF \tex_font:D {`#1}
      { \xeCJK_fallback_save_CJKsymbol {#1} }
      {
        \xeCJK_family_if_exist:nTF { \xeCJK@family/FallBack }
          { {
              \tl_put_right:Nn \xeCJK@family { /FallBack }
              \xeCJK_select_font:
              \xeCJK_fallback_testsymbol:n {#1}
          } }
          {
            \xeCJK_warning:nx { fallback } {#1}
            \xeCJK_fallback_save_CJKsymbol {囗}
          }
      }
  }
\xeCJK_msg_new:nn { fallback }
  {
    CJKfamily~'\xeCJK@family'~
    ( \prop_get:NV \g_xeCJK_family_prop \xeCJK@family )~
    does~not~contain~glyph~'#1'~(U+\int_to_hexadecimal:n {`#1}).\\
  }
%    \end{macrocode}
% \end{macro}
%
% \begin{macro}{\setCJKfallbackfamilyfont,\xeCJK_set_family_fallback:nnn}
%     \begin{macrocode}
\NewDocumentCommand \setCJKfallbackfamilyfont { m O{} m }
  {
    \exp_args:Nx \tl_if_in:nnTF {#3} { , }
      { \xeCJK_set_family_fallback:nnn {#1} {#2} {#3} }
      { \xeCJK_set_family:nnn { #1/FallBack } {#2} {#3} }
  }
\cs_new_nopar:Nn \xeCJK_set_family_fallback:nnn
  {
    \group_begin:
    \tl_set:Nx \l_xeCJK_family_fb_tl {#1}
    \prop_get:NVNF \g_xeCJK_family_prop \l_xeCJK_family_fb_tl \l_xeCJK_fontname_tl
      { \tl_clear:N \l_xeCJK_fontname_tl }
    \clist_map_inline:xn {#3}
      {
        \tl_put_right:Nn \l_xeCJK_family_fb_tl { /FallBack }
        \xeCJK_get_sub_features:nn \l_xeCJK_family_fb_tl {##1}
        \clist_put_left:cx { l_xeCJK_ \l_xeCJK_family_fb_tl _fontfeat_clist } {#2}
        \xeCJK_set_family:ncc \l_xeCJK_family_fb_tl
          { l_xeCJK_ \l_xeCJK_family_fb_tl _fontfeat_clist }
          { l_xeCJK_ \l_xeCJK_family_fb_tl _fontname_tl }
      }
    \group_end:
  }
%    \end{macrocode}
% \end{macro}
%
%
% \subsection{CJK 字体族声明方式}
%
%    \begin{macrocode}
\bool_new:N \g_xeCJK_AutoFakeBold_bool
\bool_new:N \g_xeCJK_AutoFakeSlant_bool
\fp_new:N \g_xeCJK_EmboldenFactor_fp
\fp_new:N \g_xeCJK_SlantFactor_fp
%    \end{macrocode}
%
% \begin{macro}
%    {AutoFakeBold, AutoFakeSlant, EmboldenFactor, SlantFactor, BoldFont, SlantFont}
% 伪粗体和伪斜体的宏包选项声明。
%    \begin{macrocode}
\keys_define:nn { xeCJK / options }
  {
    AutoFakeBold .choice:,
    AutoFakeBold / true    .code:n =
      { \bool_set_true:N  \g_xeCJK_AutoFakeBold_bool },
    AutoFakeBold / false   .code:n =
      { \bool_set_false:N \g_xeCJK_AutoFakeBold_bool },
    AutoFakeBold / unknown .code:n =
      {
        \bool_set_true:N  \g_xeCJK_AutoFakeBold_bool
        \fp_set:Nn \g_xeCJK_EmboldenFactor_fp { \l_keys_value_tl }
      },
    AutoFakeBold .default:n  = { true },
    AutoFakeSlant .choice:,
    AutoFakeSlant / true     .code:n =
      { \bool_set_true:N  \g_xeCJK_AutoFakeSlant_bool },
    AutoFakeSlant / false    .code:n =
      { \bool_set_false:N \g_xeCJK_AutoFakeSlant_bool },
    AutoFakeSlant / unknown  .code:n =
      {
        \bool_set_true:N  \g_xeCJK_AutoFakeSlant_bool
        \fp_set:Nn \g_xeCJK_SlantFactor_fp { \l_keys_value_tl }
      },
    AutoFakeSlant .default:n = { true },
    EmboldenFactor .fp_set:N = \g_xeCJK_EmboldenFactor_fp,
    SlantFactor    .fp_set:N = \g_xeCJK_SlantFactor_fp,
    BoldFont  .meta:n = { AutoFakeBold  = #1 },
    boldfont  .meta:n = { AutoFakeBold  = #1 },
    SlantFont .meta:n = { AutoFakeSlant = #1 },
    slantfont .meta:n = { AutoFakeSlant = #1 },
    BoldFont  .default:n = { true },
    boldfont  .default:n = { true },
    SlantFont .default:n = { true },
    slantfont .default:n = { true },
  }
%    \end{macrocode}
% \end{macro}
%
% \begin{macro}{\keys_set_known:nxN}
%    \begin{macrocode}
\cs_generate_variant:Nn \keys_set_known:nnN { nx }
%    \end{macrocode}
% \end{macro}
%
% \begin{macro}{\xeCJK_new_sub_key:n}
% \begin{macro}{\g_xeCJK_sub_key_clist}
% \begin{macro}[aux]{\xeCJK_gobble_brace:N}
% \begin{macro}[aux]{\xeCJK_get_sub_features:nn}
% \begin{macro}[aux]{\xeCJK_get_sub_features_aux:n}
% 用于定义 CJK 子区字体和备用字体的选项。
%     \begin{macrocode}
\clist_new:N \g_xeCJK_sub_key_clist
\cs_new_nopar:Nn \xeCJK_new_sub_key:n
  {
    \clist_gput_right:Nx \g_xeCJK_sub_key_clist {#1}
    \keys_define:nn { xeCJK / features }
      {
        #1 .code:n =
          {
            \tl_if_blank:xTF {##1}
              {
                \bool_set_false:c { l_xeCJK_copy_#1_bool }
                \bool_set_false:c { l_xeCJK_add_#1_bool }
                \tl_put_right:Nx \l_xeCJK_family_tl { /#1 }
              }
              {
                \clist_put_right:Nx \l_xeCJK_sub_key_clist {#1}
                \str_if_eq:xxTF {##1} *
                  {
                    \bool_set_true:c  { l_xeCJK_copy_#1_bool }
                    \bool_set_false:c { l_xeCJK_add_#1_bool }
                  }
                  {
                    \bool_set_false:c { l_xeCJK_copy_#1_bool }
                    \bool_set_true:c  { l_xeCJK_add_#1_bool }
                    \xeCJK_get_sub_features:nn {#1} {##1}
                  }
              }
          },
        #1 .default:n = \c_empty_tl,
      }
  }
\cs_generate_variant:Nn \xeCJK_new_sub_key:n { x }
\cs_new_nopar:Nn \xeCJK_get_sub_features:nn
  {
    \tl_set:Nx \l_tmpa_tl {#2}
    \clist_clear:N \l_xeCJK_sub_fontfeat_clist
    \xeCJK_gobble_brace:N \l_tmpa_tl
    \exp_args:No \tl_if_head_eq_charcode:nNTF \l_tmpa_tl [ % ]
      {
        \exp_after:wN \xeCJK_get_sub_features_aux:n \l_tmpa_tl \c_empty_tl
        \tl_if_blank:xT \l_xeCJK_sub_fontname_tl
          {
            \tl_set:Nx \l_xeCJK_sub_fontname_tl \l_tmpa_tl
            \clist_clear:N \l_xeCJK_sub_fontfeat_clist
          }
      }
      { \tl_set:Nx \l_xeCJK_sub_fontname_tl \l_tmpa_tl }
    \tl_if_blank:xTF \l_xeCJK_sub_fontname_tl
      { \tl_set:Nx \l_xeCJK_sub_fontname_tl \l_xeCJK_fontname_tl }
      { \tl_replace_all:Nnx \l_xeCJK_sub_fontname_tl * \l_xeCJK_fontname_tl }
    \clist_set:cx { l_xeCJK_#1_fontfeat_clist } \l_xeCJK_sub_fontfeat_clist
    \tl_set:cx { l_xeCJK_#1_fontname_tl } \l_xeCJK_sub_fontname_tl
  }
\cs_new_nopar:Nn \xeCJK_gobble_brace:N
  {
    \exp_args:No \tl_if_head_group:nT #1
      {
        \exp_after:wN \tl_set:Nn \exp_after:wN #1 #1
        \xeCJK_gobble_brace:N #1
      }
  }
\cs_new_nopar:Npn \xeCJK_get_sub_features_aux:n [#1] #2
  {
    \clist_set:Nx \l_xeCJK_sub_fontfeat_clist {#1}
    \tl_set:Nx \l_xeCJK_sub_fontname_tl {#2}
  }
%    \end{macrocode}
% \end{macro}
% \end{macro}
% \end{macro}
% \end{macro}
% \end{macro}
%
% \begin{macro}{FallBack}
%    \begin{macrocode}
\xeCJK_new_sub_key:n { FallBack }
%    \end{macrocode}
% \end{macro}
%
% \begin{macro}{.clist_set_x:N,.clist_set_x:c}
%    \begin{macrocode}
\cs_set_protected:cpn { \c_keys_props_root_tl .clist_set_x:N } #1
  { \keys_variable_set:NnN #1 { clist } x }
\cs_set_protected:cpn { \c_keys_props_root_tl .clist_set_x:c } #1
  { \keys_variable_set:cnN {#1} { clist } x }
%    \end{macrocode}
% \end{macro}
%
% \begin{macro}{\g_xeCJK_features_id_prop, BoldFont, ItalicFont, BoldItalicFont,
% SlantedFont, BoldSlantedFont,BoldFeatures, ItalicFeatures, BoldItalicFeatures,
% SlantedFeatures, BoldSlantedFeatures}
% 调用字体的属性声明,同 \pkg{fontspec} 宏包。
%    \begin{macrocode}
\cs_new_nopar:Nn \xeCJK_map_features_id:n
  { \prop_get:Nn \g_xeCJK_features_id_prop { #1 } }
\prop_new:N \g_xeCJK_features_id_prop
\prop_put:Nnn \g_xeCJK_features_id_prop { bf   } { Bold        }
\prop_put:Nnn \g_xeCJK_features_id_prop { it   } { Italic      }
\prop_put:Nnn \g_xeCJK_features_id_prop { bfit } { BoldItalic  }
\prop_put:Nnn \g_xeCJK_features_id_prop { sl   } { Slanted     }
\prop_put:Nnn \g_xeCJK_features_id_prop { bfsl } { BoldSlanted }
\prop_map_inline:Nn \g_xeCJK_features_id_prop
  {
    \keys_define:nn { xeCJK / features }
      {
        #2Font        .tl_set_x:c = { l_xeCJK_fontname_#1_tl },
        #2Features .clist_set_x:c = { l_xeCJK_fontfeat_#1_clist } ,
      }
  }
%    \end{macrocode}
% \end{macro}
%
% \begin{macro}{AutoFakeBold, AutoFakeSlant, Mono}
%    \begin{macrocode}
\keys_define:nn { xeCJK / features }
  {
    AutoFakeBold  .choice:,
    AutoFakeBold / false   .code:n =
      { \bool_set_false:N \l_xeCJK_AutoFakeBold_bool },
    AutoFakeBold / unknown .code:n =
      {
        \bool_set_true:N \l_xeCJK_AutoFakeBold_bool
        \fp_set:Nn \l_xeCJK_EmboldenFactor_fp { \l_keys_value_tl }
      },
    AutoFakeBold .default:n  = { \g_xeCJK_EmboldenFactor_fp },
    AutoFakeSlant  .choice:,
    AutoFakeSlant / false   .code:n =
      { \bool_set_false:N \l_xeCJK_AutoFakeSlant_bool },
    AutoFakeSlant / unknown .code:n =
      {
        \bool_set_true:N \l_xeCJK_AutoFakeSlant_bool
        \fp_set:Nn \l_xeCJK_SlantFactor_fp { \l_keys_value_tl }
      },
    AutoFakeSlant .default:n  = { \g_xeCJK_SlantFactor_fp },
    Mono .choice:,
    Mono / Exspace .code:n = { \xeCJK_set_monoexspace: },
    Mono / Scale   .code:n =
      {
        \xeCJK_set_monoscale:
        \tl_set:Nx \l_xeCJK_monoscale_tl
          { Scale = { \fp_use:N \g_xeCJK_monoscale_fp } }
      }
  }
%    \end{macrocode}
% \end{macro}
%
% \begin{macro}{\xeCJK_set_init:}
%    \begin{macrocode}
\cs_new_nopar:Nn \xeCJK_set_init:
  {
    \tl_clear:N \l_xeCJK_monoscale_tl
    \int_gincr:N \g_xeCJK_fontspec_int
    \prop_clear:N \l_xeCJK_add_font_prop
    \prop_map_inline:Nn \g_xeCJK_features_id_prop
      {
        \tl_clear:c    { l_xeCJK_fontname_##1_tl    }
        \clist_clear:c { l_xeCJK_fontfeat_##1_clist }
      }
    \clist_clear:N \l_xeCJK_pass_features_clist
    \clist_clear:N \l_xeCJK_sub_key_clist
    \clist_map_inline:Nn \g_xeCJK_sub_key_clist
      {
        \bool_set_false:c { l_xeCJK_copy_##1_bool }
        \bool_set_false:c { l_xeCJK_add_##1_bool  }
      }
    \bool_set_eq:NN \l_xeCJK_AutoFakeBold_bool   \g_xeCJK_AutoFakeBold_bool
    \bool_set_eq:NN \l_xeCJK_AutoFakeSlant_bool  \g_xeCJK_AutoFakeSlant_bool
    \fp_set_eq:NN \l_xeCJK_EmboldenFactor_fp \g_xeCJK_EmboldenFactor_fp
    \fp_set_eq:NN \l_xeCJK_SlantFactor_fp    \g_xeCJK_SlantFactor_fp
  }
%    \end{macrocode}
% \end{macro}
%
% \begin{macro}{\xeCJK_set_family:nnn}
% 设置一个 CJK 新字体族,与 |\newfontfamily| 类似,增加 |FallBack| 选项。
%    \begin{macrocode}
\cs_new_protected_nopar:Nn \xeCJK_set_family:nnn
  {
    \group_begin:
    \xeCJK_set_init:
    \tl_set:Nx \l_xeCJK_family_tl {#1}
    \clist_set:Nx \l_xeCJK_fontoptions_clist {#2}
    \tl_set:Nx \l_xeCJK_fontname_tl {#3}
    \keys_set_known:nxN { xeCJK / features }
      { \g_xeCJK_default_features_clist, #2 } \l_xeCJK_pass_features_clist
    \xeCJK_parse_features:
    \xeCJK_pass_features:
    \cs_gset_protected_nopar:cpx { xeCJK/family/\l_xeCJK_family_tl }
      {
        \exp_not:N \fontspec_select:nn
          { \l_xeCJK_pass_features_clist } { \l_xeCJK_fontname_tl }
        \prop_gput:NnV \exp_not:N \g_xeCJK_family_name_prop
          { \l_xeCJK_family_tl } \exp_not:N \l_fontspec_family_tl
        \cs_gset_protected_nopar:cpx { xeCJK/family/\l_xeCJK_family_tl }
        \exp_not:n
          { {
            \exp_not:N \fontencoding { \g_fontspec_encoding_tl }
            \exp_not:N \fontfamily   { \l_fontspec_family_tl }
            \exp_not:N \selectfont
          } }
      }
    \xeCJK_save_family_info:
    \xeCJK_add_sub_block:
    \group_end:
  }
\cs_generate_variant:Nn \xeCJK_set_family:nnn { ncc }
%    \end{macrocode}
% \end{macro}
%
% \begin{macro}
%  {\xeCJK_add_font:nn,\xeCJK_add_font_if_new:nn}
%    \begin{macrocode}
\cs_new_nopar:Nn \xeCJK_add_font:nn
  {
    \prop_put:Nxx \l_xeCJK_add_font_prop
      { \xeCJK_map_features_id:n {#1} Font } {#2}
  }
\cs_new_nopar:Nn \xeCJK_add_font_if_new:nn
  {
    \prop_put_if_new:Nxx \l_xeCJK_add_font_prop
      { \xeCJK_map_features_id:n {#1} Font } {#2}
  }
\cs_generate_variant:Nn \xeCJK_add_font:nn        { nx, nc, nV, nv }
\cs_generate_variant:Nn \xeCJK_add_font_if_new:nn { nx, nc, nV, nv }
%    \end{macrocode}
% \end{macro}
%
%    \begin{macrocode}
\cs_generate_variant:Nn \prop_put:Nnn        { Nxx }
\cs_generate_variant:Nn \prop_get:Nn         { NV  }
\cs_generate_variant:Nn \prop_get:NnNF       { Nx  }
\cs_generate_variant:Nn \prop_get:NnNTF      { Nx  }
\cs_generate_variant:Nn \prop_if_in:NnF      { Nx  }
\cs_generate_variant:Nn \prop_put_if_new:Nnn { Nxx }
%    \end{macrocode}
%
% \begin{macro}{\xeCJK_add_fake_bold:n,\xeCJK_add_fake_slant:n}
%    \begin{macrocode}
\cs_new_nopar:Nn \xeCJK_add_fake_bold:n
  {
    \clist_put_left:cx { l_xeCJK_fontfeat_#1_clist }
      { FakeBold = { \fp_use:N \l_xeCJK_EmboldenFactor_fp } }
  }
\cs_new_nopar:Nn \xeCJK_add_fake_slant:n
  {
    \clist_put_left:cx { l_xeCJK_fontfeat_#1_clist }
      { FakeSlant = { \fp_use:N \l_xeCJK_SlantFactor_fp } }
  }
%    \end{macrocode}
% \end{macro}
%
% \begin{macro}{\xeCJK_parse_features:,\xeCJK_parse_features:n}
%    \begin{macrocode}
\cs_new_nopar:Nn \xeCJK_parse_features:
  {
    \prop_map_inline:Nn \g_xeCJK_features_id_prop { \xeCJK_parse_features:n {##1} }
  }
\cs_new_nopar:Nn \xeCJK_parse_features:n
  {
    \xeCJK_if_font_select:nTF {#1}
      {
        \xeCJK_add_font:nv {#1} { l_xeCJK_fontname_#1_tl }
        \xeCJK_if_it_or_sl:nTF {#1}
          {
            \xeCJK_if_font_select:nF {bf#1}
              { \xeCJK_add_font_if_new:nv {bf#1} { l_xeCJK_fontname_#1_tl }  }
          }
          {
            \str_if_eq:xxT {#1} {bf}
              {
                \clist_map_inline:nn { it , sl }
                  {
                    \xeCJK_if_font_select:nF {bf##1}
                      { \xeCJK_add_font:nV {bf##1} \l_xeCJK_fontname_bf_tl }
                  }
              }
          }
      }
      { \xeCJK_set_fake:n {#1} }
  }
%    \end{macrocode}
% \end{macro}
%
% \begin{macro}[pTF]{\xeCJK_if_font_select:n}
% \begin{macro}[pTF]{\xeCJK_if_it_or_sl:n}
%    \begin{macrocode}
\prg_new_conditional:Nnn \xeCJK_if_font_select:n { p, T, F, TF }
  {
    \exp_args:Nc
    \tl_if_blank:xTF { l_xeCJK_fontname_#1_tl } \prg_return_false: \prg_return_true:
  }
\prg_new_conditional:Nnn \xeCJK_if_it_or_sl:n { p, T, F, TF }
  {
    \bool_if:nTF { \str_if_eq_p:xx { it } {#1} || \str_if_eq_p:xx { sl } {#1} }
      \prg_return_true: \prg_return_false:
  }
%    \end{macrocode}
% \end{macro}
% \end{macro}
%
% \begin{macro}{\xeCJK_set_fake:n}
%    \begin{macrocode}
\cs_new_nopar:Nn \xeCJK_set_fake:n
  {
    \str_if_eq:xxTF {#1} { bf }
      { \bool_if:NT \l_xeCJK_AutoFakeBold_bool { \xeCJK_add_fake_bold:n {#1} } }
      {
        \bool_if:NTF \l_xeCJK_AutoFakeSlant_bool
          { \xeCJK_add_fake_slant:n {#1} }
          { \xeCJK_if_it_or_sl:nT {#1} { \xeCJK_map_it_sl:n {#1} } }
        \bool_if:nT
          {      \l_xeCJK_AutoFakeBold_bool
            && ! ( \xeCJK_if_it_or_sl_p:n {#1} )
            && ! ( \xeCJK_if_font_select_p:n {bf} )
          }
          { \xeCJK_add_fake_bold:n {#1} }
      }
    \xeCJK_add_font_if_new:nn {#1} *
  }
%    \end{macrocode}
% \end{macro}
%
% \begin{macro}{\xeCJK_map_it_sl:n}
%    \begin{macrocode}
\cs_new_nopar:Nn \xeCJK_map_it_sl:n
  {
    \xeCJK_if_map_font_select:nT {#1}
      {
        \xeCJK_add_font:nx {#1} { \xeCJK_get_map_font:n {#1} }
        \xeCJK_if_font_select:nF {bf#1}
          { \xeCJK_add_font_if_new:nx {bf#1} { \xeCJK_get_map_font:n {#1} } }
      }
  }
%    \end{macrocode}
% \end{macro}
%
% \begin{macro}{\xeCJK_get_map_font:n}
%    \begin{macrocode}
\cs_new_nopar:Nn \xeCJK_get_map_font:n
  { \tl_use:c { l_xeCJK_fontname_\str_if_eq:xxTF {#1} {it} {sl} {it} _tl } }
%    \end{macrocode}
% \end{macro}
%
% \begin{macro}[pTF]{\xeCJK_if_map_font_select:n}
%    \begin{macrocode}
\prg_new_conditional:Nnn \xeCJK_if_map_font_select:n { p, T, F, TF }
  {
    \tl_if_blank:xTF { \xeCJK_get_map_font:n {#1} }
      \prg_return_false: \prg_return_true:
  }
%    \end{macrocode}
% \end{macro}
%
% \begin{macro}{\xeCJK_pass_features:}
%    \begin{macrocode}
\cs_new_nopar:Nn \xeCJK_pass_features:
  {
    \prop_map_inline:Nn \g_xeCJK_features_id_prop
      {
        \clist_if_empty:cF { l_xeCJK_fontfeat_##1_clist }
          {
            \clist_put_right:Nx \l_xeCJK_pass_features_clist
              { ##2Features = { \clist_use:c { l_xeCJK_fontfeat_##1_clist } } }
          }
      }
    \prop_map_inline:Nn \l_xeCJK_add_font_prop
      { \clist_put_right:Nx \l_xeCJK_pass_features_clist { ##1 = { ##2 } } }
    \tl_if_blank:xF \l_xeCJK_monoscale_tl
      { \clist_put_right:Nx \l_xeCJK_pass_features_clist \l_xeCJK_monoscale_tl }
  }
%    \end{macrocode}
% \end{macro}
%
% \begin{macro}
%  {\g_xeCJK_family_prop,\g_xeCJK_family_name_prop,\g_xeCJK_family_options_prop}
%    \begin{macrocode}
\prop_new:N \g_xeCJK_family_prop
\prop_new:N \g_xeCJK_family_name_prop
\prop_new:N \g_xeCJK_family_options_prop
%    \end{macrocode}
% \end{macro}
%
% \begin{macro}{\xeCJK_save_family_info:}
%    \begin{macrocode}
\cs_new_nopar:Nn \xeCJK_save_family_info:
  {
    \group_begin:
    \clist_map_inline:Nn \g_xeCJK_CJK_subclass_clist
      {
        \clist_remove_all:Nn \l_xeCJK_fontoptions_clist {##1}
        \clist_remove_all:Nn \l_xeCJK_fontoptions_clist { ##1 = }
      }
    \prop_gput:NVV \g_xeCJK_family_prop
                   \l_xeCJK_family_tl \l_xeCJK_fontname_tl
    \prop_gput:NVV \g_xeCJK_family_options_prop
                   \l_xeCJK_family_tl \l_xeCJK_fontoptions_clist
    \group_end:
  }
%    \end{macrocode}
% \end{macro}
%
% \begin{macro}{\xeCJK_add_sub_block:}
%    \begin{macrocode}
\cs_new_nopar:Nn \xeCJK_add_sub_block:
  {
    \clist_remove_duplicates:N \l_xeCJK_sub_key_clist
    \clist_map_inline:Nn \l_xeCJK_sub_key_clist
      {
        \tl_set:Nx \l_tmpa_tl { \l_xeCJK_family_tl/##1 }
        \bool_if:cT { l_xeCJK_copy_##1_bool }
          {
            \tl_map_inline:nn { \g_xeCJK_family_prop \g_xeCJK_family_options_prop }
              {
                \prop_get:NVNT ####1 \l_xeCJK_family_tl \l_tmpb_tl
                  { \prop_gput:NVV ####1 \l_tmpa_tl \l_tmpb_tl }
              }
            \prop_get:NVNT \g_xeCJK_family_options_prop \l_tmpa_tl \l_tmpa_clist
              {
                \clist_remove_all:Nn \l_tmpa_clist { ##1 = * }
                \prop_gput:NVV \g_xeCJK_family_options_prop \l_tmpa_tl \l_tmpa_clist
              }
            \cs_gset_protected_nopar:cpx { xeCJK/family/\l_tmpa_tl }
              {
                \exp_not:N \xeCJK_family_if_exist:nT { \l_xeCJK_family_tl }
                  {
                    \prop_get:NnNT \exp_not:N \g_xeCJK_family_name_prop
                      { \l_xeCJK_family_tl } \exp_not:N \l_tmpa_tl
                      {
                        \prop_gput:NnV \exp_not:N \g_xeCJK_family_name_prop
                          { \l_tmpa_tl } \exp_not:N \l_tmpa_tl
                        \cs_gset_protected_nopar:cpx { xeCJK/family/\l_tmpa_tl }
                        \exp_not:n
                          { {
                              \exp_not:N \fontencoding { \g_fontspec_encoding_tl }
                              \exp_not:N \fontfamily   { \l_tmpa_tl }
                              \exp_not:N \selectfont
                          } }
                      }
                  }
              }
          }
        \bool_if:cT { l_xeCJK_add_##1_bool }
          {
            \xeCJK_set_family:ncc \l_tmpa_tl
              { l_xeCJK_##1_fontfeat_clist } { l_xeCJK_##1_fontname_tl }
          }
      }
  }
%    \end{macrocode}
% \end{macro}
%
% \begin{macro}{\xeCJK_copy_family:nn}
%    \begin{macrocode}
\cs_new_nopar:Nn \xeCJK_copy_family:nn
  {
    \xeCJK_family_if_exist:nT {#2}
      {
        \tl_map_inline:nn
          { \g_xeCJK_family_prop \g_xeCJK_family_options_prop \g_xeCJK_family_name_prop }
          { \prop_get:NnNT ##1 {#2} \l_tmpa_tl { \prop_gput:NnV ##1 {#1} \l_tmpa_tl } }
        \cs_gset_eq:cc { xeCJK/family/#1 } { xeCJK/family/#2 }
      }
  }
\cs_generate_variant:Nn \xeCJK_copy_family:nn { xx }
%    \end{macrocode}
% \end{macro}
%
% \subsection{字体切换}
%
% \begin{macro}{\l_xeCJK_font_coor_tl,\xeCJK_select_font:,\xeCJK_switch_font:nn}
% 缓存当前字体的原始格式,以加速编译。
%    \begin{macrocode}
\tl_set:Nn \l_xeCJK_font_coor_tl { xeCJK/\xeCJK@family/\f@series/\f@shape/\f@size }
\cs_new_nopar:Nn \xeCJK_select_font:
  {
    \cs_if_exist_use:cF \l_xeCJK_font_coor_tl
      {
        \tl_set:Nx \l_xeCJK_current_coor_tl \l_xeCJK_font_coor_tl
        \cs_if_exist_use:cT { xeCJK/family/\xeCJK@family }
          { \exp_last_unbraced:NNV \cs_gset_eq:cN \l_xeCJK_current_coor_tl \font@name }
      }
  }
\cs_new_nopar:Nn \xeCJK_switch_font:nn
  {
    \str_if_eq:xxF {#1} {#2}
      {
        \xeCJK_info:nxx { CJK-block } {#1} {#2}
        \tl_remove_all:Nn \xeCJK@family { /#1 }
        \str_if_eq:xxF {#2} { CJK }
          {
            \tl_set:Nx \l_tmpa_tl \xeCJK@family
            \tl_put_right:Nx \xeCJK@family { /#2 }
            \xeCJK_family_if_exist:nF \xeCJK@family
              {
                \xeCJK_family_if_exist:nTF { \CJKfamilydefault/#2 }
                  { \xeCJK_copy_family:xx \xeCJK@family { \CJKfamilydefault/#2 } }
                  { \xeCJK_copy_family:xx \xeCJK@family \l_tmpa_tl }
              }
          }
      }
    \xeCJK_select_font:
  }
\xeCJK_msg_new:nn { CJK-block } { Switch~from~block~'#1'~to~'#2'. }
%    \end{macrocode}
% \end{macro}
%
% \begin{macro}[TF]{\xeCJK_family_if_exist:n}
%    \begin{macrocode}
\prg_new_conditional:Nnn \xeCJK_family_if_exist:n { T, F, TF }
  {
    \cs_if_free:cTF { xeCJK/family/#1 } \prg_return_false:
      {
        \prop_if_in:NxF
          \g_xeCJK_family_name_prop {#1} { \use:c { xeCJK/family/#1 } }
        \prg_return_true:
      }
  }
%    \end{macrocode}
% \end{macro}
%
% \begin{macro}{\CJKfamily}
% 用于切换 CJK 字体族。
%     \begin{macrocode}
\NewDocumentCommand \CJKfamily { m }
  {
    \xeCJK_family_if_exist:nTF {#1}
      { \tl_set:Nx \xeCJK@family {#1} }
      {
        \cs_if_free:cT { xeCJK_warned_aux_#1 }
          {
            \xeCJK_warning:nx { CJKfamily-Unknown } {#1}
            \tl_new:c { xeCJK_warned_aux_#1 }
          }
      }
  }
\xeCJK_msg_new:nn { CJKfamily-Unknown }
  {
    Unknown~CJK~family~'#1'~is~ignored.\\\\
    Try~to~use~\token_to_str:N \setCJKfamilyfont{#1}[...]{...}\c_space_tl
    to~define~it.
  }
%    \end{macrocode}
% \end{macro}
%
% \begin{macro}{\setCJKfamilyfont, \newCJKfontfamily, \CJKfontspec}
% 分别用于预声明 CJK 字体和随机调用 CJK 字体。
%    \begin{macrocode}
\NewDocumentCommand \setCJKfamilyfont { m O{} m }
  { \xeCJK_set_family:nnn {#1} {#2} {#3} }
\NewDocumentCommand \newCJKfontfamily { o m O{} m }
  {
    \IfNoValueTF {#1}
      { \tl_set:Nx \l_xeCJK_family_tl { \cs_to_str:N #2 } }
      { \tl_set:Nx \l_xeCJK_family_tl {#1} }
    \xeCJK_set_family:nnn \l_xeCJK_family_tl {#3} {#4}
    \cs_new_protected_nopar:Npx #2
      { \exp_not:N \CJKfamily { \l_xeCJK_family_tl } }
  }
\int_new:N   \g_xeCJK_fontspec_int
\int_gdecr:N \g_xeCJK_fontspec_int
\NewDocumentCommand \CJKfontspec { O{} m }
  {
    \tl_set:Nx \l_xeCJK_CJKfontspec_id_tl { CJKfontspec/#1/#2/id }
    \cs_if_free:cT \l_xeCJK_CJKfontspec_id_tl
      {
        \int_gincr:N \g_xeCJK_fontspec_int
        \tl_gset:Nx \g_xeCJK_family_spec_tl
          { CJKfontspec ( \int_use:N \g_xeCJK_fontspec_int ) }
        \xeCJK_set_family:nnn \g_xeCJK_family_spec_tl {#1} {#2}
        \tl_gset:cx \l_xeCJK_CJKfontspec_id_tl \g_xeCJK_family_spec_tl
      }
    \exp_args:Nv \CJKfamily \l_xeCJK_CJKfontspec_id_tl
    \tex_ignorespaces:D
  }
%    \end{macrocode}
% \end{macro}
%
% \begin{macro}{\defaultCJKfontfeatures, \addCJKfontfeatures}
% 分别用于设置 CJK 字体的默认属性和增加当前 CJK 字体的属性。
%    \begin{macrocode}
\clist_new:N \g_xeCJK_default_features_clist
\NewDocumentCommand \defaultCJKfontfeatures { m }
  { \clist_gset:Nn \g_xeCJK_default_features_clist {#1} }
\@onlypreamble \defaultCJKfontfeatures
\NewDocumentCommand \addCJKfontfeatures { m }
  {
    \prop_if_in:NVTF \g_xeCJK_family_prop \xeCJK@family
      {
        \group_begin:
        \clist_set:Nx \l_tmpa_clist {#1}
        \prop_get:NVN \g_xeCJK_family_prop \xeCJK@family \l_tmpa_tl
        \clist_map_inline:Nn \g_xeCJK_CJK_subclass_clist
          {
            \clist_if_in:NnT \l_tmpa_clist {##1}
              {
                \clist_remove_all:Nn \l_tmpa_clist {##1}
                \prop_get:NxNF \g_xeCJK_family_prop
                  { \xeCJK@family/##1 } \l_tmpb_tl
                  {
                    \prop_get:NxNF \g_xeCJK_family_prop
                      { \CJKfamilydefault/##1 } \l_tmpb_tl
                      { \tl_set:Nx \l_tmpb_tl \l_tmpa_tl }
                  }
                \clist_set:Nx \l_tmpa_clist
                  { ##1 = { [ \l_tmpa_clist ] { \l_tmpb_tl } } }
                \clist_map_break:
              }
          }
        \prop_get:NVNF \g_xeCJK_family_options_prop \xeCJK@family
          \l_tmpb_clist { \clist_clear:N \l_tmpb_clist }
        \clist_clear:N \l_tmpc_clist
        \clist_map_inline:Nn \l_tmpb_clist
          {
            \clist_if_in:NnF \l_tmpa_clist {##1}
              { \clist_put_right:Nn \l_tmpc_clist {##1} }
          }
        \clist_put_left:NV \l_tmpa_clist \l_tmpc_clist
        \use:x
          { \group_end: \exp_not:N \CJKfontspec [ \l_tmpa_clist ] { \l_tmpa_tl } }
      }
      { \xeCJK_warning:n { addCJKfontfeature-ignored } }
    \tex_ignorespaces:D
  }
\cs_new_eq:NN \addCJKfontfeature \addCJKfontfeatures
\xeCJK_msg_new:nn { addCJKfontfeature-ignored }
  {
    \tl_to_str:N \addCJKfontfeatures (s)~ignored.\\\\
    It~cannot~be~used~with~a~font~that~wasn't~selected~by~xeCJK.
  }
%    \end{macrocode}
% \end{macro}
%
% \begin{macro}
%   { \CJKrmdefault, \setCJKmainfont, \CJKsfdefault, \setCJKsansfont,
%     \CJKttdefault, \CJKfamilydefault }
% 设置文档的 CJK 普通字体、无衬线字体。
%    \begin{macrocode}
\cs_if_free:NT \CJKrmdefault { \tl_set:Nn \CJKrmdefault { rm } }
\NewDocumentCommand \setCJKmainfont { O{} m }
  { \xeCJK_set_family:nnn \CJKrmdefault {#1} {#2} }
\cs_new_eq:NN \setCJKromanfont \setCJKmainfont
\cs_if_free:NT \CJKsfdefault { \tl_set:Nn \CJKsfdefault { sf } }
\NewDocumentCommand \setCJKsansfont { O{} m }
  { \xeCJK_set_family:nnn \CJKsfdefault {#1} {#2} }
\cs_if_free:NT \CJKttdefault { \tl_set:Nn \CJKttdefault { tt } }
%    \end{macrocode}
% \end{macro}
%
% \begin{macro}{\CJKfamilydefault}
% CJK 默认字体族,作用于 |\normalfont|。
%    \begin{macrocode}
\cs_if_free:NT \CJKfamilydefault { \tl_set:Nn \CJKfamilydefault { \CJKrmdefault } }
%    \end{macrocode}
% \end{macro}
%
% 补丁 |\normalfont|、|\rmfamily|,|\sffamily| 和 |\ttfamily|,使其同时对
% CJK 字体族有效。
% \begin{macrocode}
\AtEndOfPackage
  {
    \etex_protected:D \tl_put_right:Nn \normalfont { \CJKfamily \CJKfamilydefault }
    \etex_protected:D \tl_put_right:Nn \rmfamily   { \CJKfamily \CJKrmdefault }
    \etex_protected:D \tl_put_right:Nn \sffamily   { \CJKfamily \CJKsfdefault }
    \etex_protected:D \tl_put_right:Nn \ttfamily   { \CJKfamily \CJKttdefault }
    \cs_set_eq:NN \reset@font \normalfont
  }
%    \end{macrocode}
%
% \begin{macro}{\setCJKmathfont}
% 设置 CJK 数学字体。
%    \begin{macrocode}
\NewDocumentCommand \setCJKmathfont { O{} m }
  { \xeCJK_set_family:nnn \c_xeCJK_math_tl {#1} {#2} }
%    \end{macrocode}
%
% \begin{macro}{CJKmath}
% 数学公式内的 CJK 字体。
%    \begin{macrocode}
\tl_const:Nn \c_xeCJK_math_tl { CJKmath }
\keys_define:nn { xeCJK / options } { CJKmath .bool_gset:N = \g_xeCJK_math_bool }
%    \end{macrocode}
% \end{macro}
%
% \begin{macro}{\xeCJKsetmathcode}
%    \begin{macrocode}
\NewDocumentCommand \xeCJKsetmathcode { m m m m }
  {
    \xeCJK_check_num_range:nnNN {#1} {#2} \l_tmpa_int \l_tmpb_int
    \int_set:Nn \l_tmpc_int { \xeCJK_math_type:n {#3} }
    \loop
      \XeTeXmathcode \l_tmpa_int = \l_tmpc_int \use:c { sym #4 } \l_tmpa_int
      \if_int_compare:w \l_tmpa_int < \l_tmpb_int
        \int_incr:N \l_tmpa_int
    \repeat
  }
\cs_set_eq:NN \xeCJK_math_type:n \mathchar@type
%    \end{macrocode}
% \end{macro}
%
% 当没有设置 CJK 数学字体时,使用 |\CJKfamilydefault| 作为数学字体。
%    \begin{macrocode}
\tl_put_left:Nn \document
  {
    \group_end:
    \bool_if:NT \g_xeCJK_math_bool
      {
        \xeCJK_family_if_exist:nTF \c_xeCJK_math_tl
          { \tl_set:Nx \l_tmpa_tl \c_xeCJK_math_tl }
          {
            \xeCJK_family_if_exist:nTF \CJKfamilydefault
              { \tl_set:Nx \l_tmpa_tl \CJKfamilydefault }
              { \tl_set_eq:NN \l_tmpa_tl \q_no_value }
          }
        \prop_get:NVNT \g_xeCJK_family_name_prop \l_tmpa_tl \l_xeCJK_math_family_tl
          {
            \DeclareSymbolFont \c_xeCJK_math_tl       \g_fontspec_encoding_tl
              \l_xeCJK_math_family_tl \mddefault \shapedefault
            \SetSymbolFont  \c_xeCJK_math_tl { bold } \g_fontspec_encoding_tl
              \l_xeCJK_math_family_tl \bfdefault \shapedefault
            \clist_map_inline:nn { CJK, FullLeft, FullRight  }
              {
                \clist_map_inline:cn { g_xeCJK_#1_range_clist }
                  {
                    \xeCJK_setcharclass_aux:Nn \xeCJKsetmathcode {##1}
                      \mathalpha \c_xeCJK_math_tl
                  }
              }
          }
      }
    \tl_set:Nx \xeCJK@family \CJKfamilydefault
    \normalfont
    \group_begin:
  }
%    \end{macrocode}
% \end{macro}
%
% 为了保证 CJK 数学字体的正确设置,应在导言区内设置 CJK 字体。
%    \begin{macrocode}
\@onlypreamble \setCJKmainfont
\@onlypreamble \setCJKmathfont
\@onlypreamble \setCJKsansfont
\@onlypreamble \setCJKmonofont
\@onlypreamble \setCJKromanfont
%    \end{macrocode}
%
% \subsection{处理等宽字体和抄录环境}
%
% \begin{macro}{\g_xeCJK_monoscale_fp,\g_xeCJK_exspace_dim}
%    \begin{macrocode}
\fp_new:N \g_xeCJK_monoscale_fp
\fp_gset_eq:NN \g_xeCJK_monoscale_fp \c_one_fp
\dim_new:N \g_xeCJK_exspace_dim
%    \end{macrocode}
% \end{macro}
%
% \begin{macro}{\setCJKmonoscale}
% \begin{macro}[aux]{\xeCJK_set_monoscale:}
%     \begin{macrocode}
\cs_new_nopar:Nn \xeCJK_set_monoscale:
  {
    \dim_gzero:N \g_xeCJK_exspace_dim
    \group_begin:
      \fontfamily \ttdefault \selectfont
      \fp_gset_from_dim:Nn \g_xeCJK_monoscale_fp
        { \c_two \tex_fontdimen:D \c_two \tex_font:D }
      \fp_gdiv:Nn \g_xeCJK_monoscale_fp \f@size
    \group_end:
  }
\NewDocumentCommand \setCJKmonoscale { }
  {
    \xeCJK_set_monoscale:
    \addCJKfontfeatures { Scale = \fp_use:N \g_xeCJK_monoscale_fp }
  }
%    \end{macrocode}
% \end{macro}
% \end{macro}
%
% \begin{macro}{\setCJKmonoscalefont}
%     \begin{macrocode}
\NewDocumentCommand \setCJKmonoscalefont { O{} m }
  { \xeCJK_set_family:nnn \CJKttdefault { Mono = Scale, #1 } {#2} }
%    \end{macrocode}
% \end{macro}
%
% \begin{macro}{\setCJKmonoexspace}
% \begin{macro}[aux]{\xeCJK_set_monoexspace:}
%     \begin{macrocode}
\cs_new_nopar:Nn \xeCJK_set_monoexspace:
  {
    \fp_gset_eq:NN \g_xeCJK_monoscale_fp \c_one_fp
    \group_begin:
      \fontfamily \ttdefault \selectfont
      \dim_gset:Nn \g_xeCJK_exspace_dim
        { \c_two \tex_fontdimen:D \c_two \tex_font:D - \f@size \p@ }
    \group_end:
  }
\NewDocumentCommand \setCJKmonoexspace { } { \xeCJK_set_monoexspace: }
%    \end{macrocode}
% \end{macro}
% \end{macro}
%
% \begin{macro}[aux]
%  { \xeCJK_fixed_ecglue:, \xeCJK_fixed_cjkglue:, \xeCJK_flexible_ecglue:,
%    \xeCJK_flexible_ecglue:, \xeCJK_flexible_cjkglue: }
%    \begin{macrocode}
\cs_new_nopar:Nn \xeCJK_fixed_ecglue:  { \skip_horizontal:n { .5\g_xeCJK_exspace_dim } }
\cs_new_nopar:Nn \xeCJK_fixed_cjkglue: { \skip_horizontal:N \g_xeCJK_exspace_dim }
\AtEndOfPackage
  {
    \tl_set_eq:NN \l_xeCJK_flexible_punctstyle_tl \l_xeCJK_punctstyle_tl
    \cs_new_eq:NN \xeCJK_flexible_ecglue:  \CJKecglue
    \cs_new_eq:NN \xeCJK_flexible_cjkglue: \CJKglue
  }
%    \end{macrocode}
% \end{macro}
%
% \begin{macro}{\CJKfixedspacing}
%     \begin{macrocode}
\NewDocumentCommand \CJKfixedspacing { }
  {
    \tl_if_eq:NNF \l_xeCJK_punctstyle_tl \c_xeCJK_ps_plain_tl
      {
        \tl_set_eq:NN \l_xeCJK_flexible_punctstyle_tl \l_xeCJK_punctstyle_tl
        \punctstyle { plain }
      }
    \cs_if_eq:NNF \CJKecglue \xeCJK_fixed_ecglue:
      {
        \cs_set_eq:NN \xeCJK_flexible_ecglue: \CJKecglue
        \cs_set_eq:NN \CJKecglue \xeCJK_fixed_ecglue:
      }
    \cs_if_eq:NNF \CJKglue \xeCJK_fixed_cjkglue:
      {
        \cs_set_eq:NN \xeCJK_flexible_cjkglue: \CJKglue
        \cs_set_eq:NN \CJKglue \xeCJK_fixed_cjkglue:
      }
  }
\AtBeginDocument { \tl_put_right:Nn \verbatim@font \CJKfixedspacing }
%    \end{macrocode}
% \end{macro}
%
% \begin{macro}{\CJKflexiblespacing}
%     \begin{macrocode}
\NewDocumentCommand \CJKflexiblespacing { m }
  {
    \cs_set_eq:NN \l_xeCJK_punctstyle_tl \l_xeCJK_flexible_punctstyle_tl
    \cs_set_eq:NN \CJKecglue \xeCJK_flexible_ecglue:
    \cs_set_eq:NN \CJKglue   \xeCJK_flexible_cjkglue:
  }
%    \end{macrocode}
% \end{macro}
%
% \begin{macro}{\setCJKmonoexspacefont}
%     \begin{macrocode}
\NewDocumentCommand \setCJKmonoexspacefont { O{} m }
  { \xeCJK_set_family:nnn \CJKttdefault { Mono = Exspace, #1 } {#2} }
%    \end{macrocode}
% \end{macro}
%
% \begin{macro}{\setCJKmonofont}
% 设置文档的 CJK 等宽字体族。
%    \begin{macrocode}
\NewDocumentCommand \setCJKmonofont { s t+ O{} m }
  {
    \IfBooleanTF {#1}
      { \setCJKmonoscalefont [#3] {#4} }
      {
        \IfBooleanTF {#2}
          { \setCJKmonoexspacefont [#3] {#4} }
          { \xeCJK_set_family:nnn \CJKttdefault {#3} {#4} }
      }
  }
%    \end{macrocode}
% \end{macro}
%
% \subsection{\pkg{xeCJK} 其它选项}
%
% \begin{macro}{CJKnumber, indentfirst, normalindentfirst}
% 是否启用 \pkg{CJKnumber} 宏包和首行是否缩进,其中 |CJKnumber| 选项仅
% 在 \pkg{xeCJK} 宏包调用的时候有效。并将 \pkg{xeCJK} 中未知的选项传递给
% \pkg{fontspec} 宏包。
%     \begin{macrocode}
\keys_define:nn { xeCJK / options }
  {
    CJKnumber .bool_set:N = \l_xeCJK_number_bool ,
    CJKnumber  .default:n = { true },
    indentfirst .choice: ,
    indentfirst / true  .code:n = { \cs_set_eq:NN \@afterindentfalse \prg_do_nothing: },
    indentfirst / false .code:n = { \cs_set_eq:NN \if@afterindent \if_false: },
    indentfirst      .default:n = { true },
    normalindentfirst   .meta:n = { indentfirst = false },
    quiet .code:n =
      {
        \msg_redirect_module:nnn { xeCJK } { warning } { info }
        \msg_redirect_module:nnn { xeCJK } { info }    { none }
        \PassOptionsToPackage { quiet } { fontspec }
      },
    silent .code:n =
      {
        \msg_redirect_module:nnn { xeCJK } { warning } { none }
        \msg_redirect_module:nnn { xeCJK } { info }    { none }
        \PassOptionsToPackage { silent } { fontspec }
      },
    unknown .code:n =
      {
        \cs_if_free:cTF {  ver@fontspec.sty }
          { \PassOptionsToPackage { \l_keys_key_tl } { fontspec } }
          { \xeCJK_error:nx { key-unknown } { \l_keys_key_tl } }
      },
  }
\xeCJK_msg_new:nn { key-unknown }
  {
    Sorry,~but~\l_keys_module_tl\c_space_tl does~not~have~a~key~called~'#1'.\\\\
    The~key~'#1'~is~being~ignored.
  }
%    \end{macrocode}
% \end{macro}
%
% \subsection{\pkg{xeCJK} 初始化设置}
%
% \pkg{xeCJK} 宏包的初始化设置。
%     \begin{macrocode}
\keys_set:nn { xeCJK / options }
  {
    xeCJKactive = true , CJKallowbreakbetweenpuncts = false ,
    indentfirst = true , CJKspace = false , CJKecglue = { \c_space_token } ,
    CJKglue = { \skip_horizontal:n { \c_zero_skip \@plus .08\baselineskip } },
    CJKchecksingleglue = \CJKglue , CJKchecksingle = false ,
    CJKnumber = false , CJKmath = true , PunctStyle = quanjiao ,
    EmboldenFactor = 4 , SlantFactor = .167 ,
    KaiMingPunct = { . 。? ! } , LongPunct = { — ‥ … ─ } ,
    MiddlePunct = { · ・ ・ } ,
  }
%    \end{macrocode}
%
% \begin{macro}{\CJKsymbol, \CJKpunctsymbol}
%    \begin{macrocode}
\cs_new_nopar:Npn \CJKsymbol      #1 {#1}
\cs_new_nopar:Npn \CJKpunctsymbol #1 {#1}
%    \end{macrocode}
% \end{macro}
%
% 执行宏包选项,并载入 \pkg{fontspec} 宏包。
%    \begin{macrocode}
\ProcessKeysOptions { xeCJK / options }
\RequirePackage { fontspec } [ 2011/09/13 ]
%    \end{macrocode}
%
% \begin{macro}{\xeCJKsetup}
% 在导言区或文档中设置 \pkg{xeCJK} 的接口。
%     \begin{macrocode}
\NewDocumentCommand \xeCJKsetup { m }
  {
    \keys_set:nn { xeCJK / options } {#1}
    \tex_ignorespaces:D
  }
%    \end{macrocode}
% \end{macro}
%
% \begin{macro}{\makexeCJKactive, \makexeCJKinactive}
%     \begin{macrocode}
\NewDocumentCommand \makexeCJKactive { }
  { \keys_set:nn { xeCJK / options } { xeCJKactive = true } }
\NewDocumentCommand \makexeCJKinactive { }
  { \keys_set:nn { xeCJK / options } { xeCJKactive = false } }
%    \end{macrocode}
% \end{macro}
%
% \begin{macro}{\DeclareKaiMingPunct, \DeclareLongPunct, \DeclareMiddlePunct}
%     \begin{macrocode}
\clist_map_inline:nn { KaiMing , Long , Middle }
  {
    \exp_args:Nc \NewDocumentCommand { Declare #1 Punct } { t+ t- m }
      {
        \IfBooleanTF {##1}
          { \keys_set:nn { xeCJK / options } { #1Punct+ = {##3} } }
          {
            \IfBooleanTF {##2}
              { \keys_set:nn { xeCJK / options } { #1Punct- = {##3} } }
              { \keys_set:nn { xeCJK / options } { #1Punct  = {##3} } }
          }
      }
  }
%    \end{macrocode}
% \end{macro}
%
% \begin{macro}
%   {\xeCJKsetEmboldenFactor, \xeCJKsetSlantFactor}
%    \begin{macrocode}
\NewDocumentCommand \xeCJKsetEmboldenFactor { m }
  { \keys_set:nn { xeCJK / options } { EmboldenFactor = {#1} } }
\NewDocumentCommand \xeCJKsetSlantFactor { m }
  { \keys_set:nn { xeCJK / options } { SlantFactor = {#1} } }
\cs_new_eq:NN \xeCJKsetslantfactor    \xeCJKsetEmboldenFactor
\cs_new_eq:NN \xeCJKsetemboldenfactor \xeCJKsetSlantFactor
%    \end{macrocode}
% \end{macro}
%
% \begin{macro}{\punctstyle, \xeCJKplainchr}
%    \begin{macrocode}
\NewDocumentCommand \punctstyle { m }
  { \keys_set:nn { xeCJK / options } { PunctStyle = {#1} } }
\cs_new_nopar:Npn \xeCJKplainchr { \punctstyle { plain } }
%    \end{macrocode}
% \end{macro}
%
% \begin{macro}{\CJKsetecglue}
%    \begin{macrocode}
\NewDocumentCommand \CJKsetecglue { m }
  { \keys_set:nn { xeCJK / options } { CJKecglue = {#1} } }
\cs_new_eq:NN \xeCJKsetecglue \CJKsetecglue
%    \end{macrocode}
% \end{macro}
%
% \subsection{兼容性修补}
%
% \begin{macro}{\/}
% \begin{macro}[aux]{\xeCJK_itcorr_aux}
% 修复倾斜校正。
%    \begin{macrocode}
\cs_new_eq:NN \xeCJK_itcorr_aux \/
\bool_new:N \g_xeCJK_itcorr_bool
\cs_set_protected_nopar:Npn \/
  {
    \scan_stop:
    \int_compare:nNnT \tex_lastkern:D = \c_two
      {
        \tex_unkern:D \tex_unkern:D
        \bool_gset_true:N \g_xeCJK_itcorr_bool
      }
    \xeCJK_itcorr_aux
  }
\cs_set_eq:NN \@@italiccorr \/
%    \end{macrocode}
% \end{macro}
% \end{macro}
%
% \begin{macro}{\xeCJK_patch:Nnn}
% 给已有宏内容前后附加补丁。
%    \begin{macrocode}
\cs_new_nopar:Nn \xeCJK_patch:Nnn
  { \tl_put_left:Nn  #1 {#2} \tl_put_right:Nn #1 {#3} }
%    \end{macrocode}
% \end{macro}
%
% 单独处理宽度有分歧的几个标点:包括省略号、破折号、间隔号、引号等中西文混用的
% 符号, 保证其命令形式输出的是西文字体。并对一些编码的符号宏包做特殊处理。
% \begin{macrocode}
\AtBeginDocument
  {
    \tl_map_inline:nn
      {
        \textellipsis  \textemdash     \textperiodcentered \textcentereddot
        \textquoteleft \textquoteright \textquotedblleft   \textquotedblright
      }
      { \xeCJK_patch:Nnn #1 { \group_begin: \makexeCJKinactive } { \group_end: } }
    \xeCJK_patch:Nnn \tipaencoding { \makexeCJKinactive } { }
    \cs_set_eq:NN \xeCJK_aux_r \r
    \cs_set_nopar:Npn \r #1 { { \makexeCJKinactive \xeCJK_aux_r {#1} } }
    \@ifpackageloaded { pifont }
      {
        \RenewDocumentCommand \Pifont { m }
          {
            \fontencoding {U} \fontfamily {#1} \fontseries {m} \fontshape {n}
            \selectfont \makexeCJKinactive
          }
      } { }
  }
%    \end{macrocode}
%
% 禁止在 \pkg{xeCJK} 宏包后再载入 \pkg{CJK} 宏包。
%    \begin{macrocode}
\tl_set:cn { ver@CJK.sty } { 2050/01/01 }
%    \end{macrocode}
%
% \begin{macro}{\xeCJKcaption}
% 可以使用 \pkg{CJK} 宏包中的 |.cpx| 文件。
%    \begin{macrocode}
\cs_if_free:NT \CJK@ifundefined
  { \cs_set_eq:NN \CJK@ifundefined \cs_if_free:NTF }
\NewDocumentCommand \xeCJKcaption { o m }
  {
    \IfValueT {#1} { \XeTeXdefaultencoding "#1" }
    \cs_set_nopar:Nx \xeCJK_reset_at_catcode:
      {
        \exp_not:n { \char_set_catcode:nn  { `\@ } }
                   { \char_value_catcode:n { `\@ } }
      }
    \char_set_catcode_letter:N \@
    \file_input:n { #2.cpx }
    \xeCJK_reset_at_catcode:
    \XeTeXdefaultencoding "UTF-8"
  }
%    \end{macrocode}
% \end{macro}
%
% \begin{macro}[aux]
%   {\xeCJK_ULprepunctchar:n, \xeCJK_ULpostpunctchar:n, \xeCJK_ULroutines:}
%    \begin{macrocode}
\cs_new_protected_nopar:Nn \xeCJK_ULprepunctchar:n
  {
    { \makexeCJKinactive \CJKpunctsymbol {#1} \nobreak }
    \tex_ignorespaces:D
  }
\cs_new_protected_nopar:Nn \xeCJK_ULpostpunctchar:n
  {
    { \makexeCJKinactive \CJKpunctsymbol {#1} }
    \xeCJK_ignorespaces:
  }
\cs_new_protected_nopar:Nn \xeCJK_ULroutines:
  {
    \xeCJK_inter_class_toks:nnn { Default   } { CJK } { \CJKecglue \CJKsymbol }
    \xeCJK_inter_class_toks:nnn { HalfLeft  } { CJK } { \CJKsymbol }
    \xeCJK_inter_class_toks:nnn { HalfRight } { CJK } { \CJKecglue \CJKsymbol }
    \xeCJK_inter_class_toks:nnn { Boundary  } { CJK } { \xeCJK_Boundary_and_CJK: }
    \clist_map_inline:nn { Default, HalfLeft, HalfRight, Boundary }
      {
        \xeCJK_inter_class_toks:nnn {##1} { FullLeft }  { \xeCJK_ULprepunctchar:n }
        \xeCJK_inter_class_toks:nnn {##1} { FullRight } { \xeCJK_ULpostpunctchar:n }
      }
    \xeCJK_UL_subclass_patch_tl
  }
%    \end{macrocode}
% \end{macro}
%
% 对 \pkg{ulem} 宏包打补丁,以支持 \pkg{CJKfntef} 宏包。
% \begin{macrocode}
\AtBeginDocument
  {
    \cs_if_exist:NT \UL@hook
      {
        \addto@hook \UL@hook
          {
            \cs_set_eq:NN \xeCJK_UL_CJKsymbol \CJKsymbol
            \cs_set_eq:NN \xeCJK_UL_CJKpunctsymbol \CJKpunctsymbol
            \cs_set_nopar:Npn \CJKsymbol #1
              {
                { \xeCJK_select_font: \xeCJK_UL_CJKsymbol {#1} }
                \xeCJK_CJK_kern: \xeCJK_ignorespaces:
              }
            \cs_set_nopar:Npn \CJKpunctsymbol #1
              { { \xeCJK_select_font: \xeCJK_UL_CJKpunctsymbol {#1} } }
            \xeCJK_ULroutines:
          }
      }
  }
%    \end{macrocode}
%
% 使用 \pkg{CJKnumb} 宏包。
%    \begin{macrocode}
\bool_if:NT \l_xeCJK_number_bool
  {
    \tl_set:Nn \CJK@UnicodeEnc { UTF8 }
    \cs_set_nopar:Npn \CJKaddEncHook #1#2 { \cs_set_nopar:cpn { xeCJK_enc_#1 } {#2} }
    \cs_set_nopar:Npn \Unicode #1#2
      { \tex_char:D \int_eval:n { #1 * \c_two_hundred_fifty_six + #2 } }
    \RequirePackage { CJKnumb }
    \use:c { xeCJK_enc_\CJK@UnicodeEnc }
    \tl_set:Nn \CJK@tenthousand    { 万 }
    \tl_set:Nn \CJK@hundredmillion { 亿 }
  }
%    \end{macrocode}
%
%    \begin{macrocode}
%</package>
%    \end{macrocode}
%
% \section{例子}
%
% \subsection{\pkg{example-autofake.tex}}
%
%    \begin{macrocode}
%<*ex-autofake>
%    \end{macrocode}
%
% \begin{macrocode}
\documentclass{article}
\usepackage[AutoFakeBold,AutoFakeSlant]{xeCJK}
\setCJKmainfont[BoldFont=simhei.ttf, ItalicFont=simkai.ttf]{simsun.ttc}
\setCJKsansfont[AutoFakeSlant=false,
  BoldFont=simhei.ttf, ItalicFont=simkai.ttf]{simsun.ttc}
\setCJKmonofont[ItalicFont=simkai.ttf]{simsun.ttc}
\begin{document}
\centering
\begin{tabular}{lllll}
\hline
 {\bf rm} & md & up & \verb|\rmfamily\mdseries\upshape| &
                      {\rmfamily\mdseries\upshape English 中文字体} \\
          & md & it & \verb|\rmfamily\mdseries\itshape| &
                      {\rmfamily\mdseries\itshape English 中文字体} \\
          & md & sl & \verb|\rmfamily\mdseries\slshape| &
                      {\rmfamily\mdseries\slshape English 中文字体} \\ \cline{2-5}
          & bf & up & \verb|\rmfamily\bfseries\upshape| &
                      {\rmfamily\bfseries\upshape English 中文字体} \\
          & bf & it & \verb|\rmfamily\bfseries\itshape| &
                      {\rmfamily\bfseries\itshape English 中文字体} \\
          & bf & sl & \verb|\rmfamily\bfseries\slshape| &
                      {\rmfamily\bfseries\slshape English 中文字体} \\ \hline
 {\bf sf} & md & up & \verb|\sffamily\mdseries\upshape| &
                      {\sffamily\mdseries\upshape English 中文字体} \\
          & md & it & \verb|\sffamily\mdseries\itshape| &
                      {\sffamily\mdseries\itshape English 中文字体} \\
          & md & sl & \verb|\sffamily\mdseries\slshape| &
                      {\sffamily\mdseries\slshape English 中文字体} \\ \cline{2-5}
          & bf & up & \verb|\sffamily\bfseries\upshape| &
                      {\sffamily\bfseries\upshape English 中文字体} \\
          & bf & it & \verb|\sffamily\bfseries\itshape| &
                      {\sffamily\bfseries\itshape English 中文字体} \\
          & bf & sl & \verb|\sffamily\bfseries\slshape| &
                      {\sffamily\bfseries\slshape English 中文字体} \\ \hline
 {\bf tt} & md & up & \verb|\ttfamily\mdseries\upshape| &
                      {\ttfamily\mdseries\upshape English 中文字体} \\
          & md & it & \verb|\ttfamily\mdseries\itshape| &
                      {\ttfamily\mdseries\itshape English 中文字体} \\
          & md & sl & \verb|\ttfamily\mdseries\slshape| &
                      {\ttfamily\mdseries\slshape English 中文字体} \\ \cline{2-5}
          & bf & up & \verb|\ttfamily\bfseries\upshape| &
                      {\ttfamily\bfseries\upshape English 中文字体} \\
          & bf & it & \verb|\ttfamily\bfseries\itshape| &
                      {\ttfamily\bfseries\itshape English 中文字体} \\
          & bf & sl & \verb|\ttfamily\bfseries\slshape| &
                      {\ttfamily\bfseries\slshape English 中文字体} \\ \hline
\end{tabular}
\end{document}
%    \end{macrocode}
%
%    \begin{macrocode}
%</ex-autofake>
%    \end{macrocode}
%
% \subsection{\pkg{example-fallback.tex}}
%
%    \begin{macrocode}
%<*ex-fallback>
%    \end{macrocode}
%
% \begin{macrocode}
\documentclass{article}
\usepackage[FallBack]{xeCJK}
\usepackage{CJKfntef}
\usepackage{array}
\setCJKmainfont[AutoFakeBold,AutoFakeSlant]{KaiTi_GB2312}
\setCJKfallbackfamilyfont{\CJKrmdefault}[BoldFont=SimHei]
  { [SlantedFont=FangSong]{SimSun} ,
    [BoldFont=*]          {SimSun-ExtB} }
\begin{document}
漢字源𣴑考

\textbf{漢字源𣴑考}

\textsl{漢字源𣴑考}

\CJKunderwave{漢字源𣴑考}
\begin{table}[ht]
\caption{生僻字测试}
\medskip\centering
\begin{tabular}{*4{|c>{\ttfamily U+}l}|}
㐀 & 3400  & 㐁 & 3401  & 㐂 & 3402  & 㐃 & 3403  \\
㐄 & 3404  & 㐅 & 3405  & 㐆 & 3406  & 㐇 & 3407  \\
㐈 & 3408  & 㐉 & 3409  & 㐊 & 340A  & 㐋 & 340B  \\
㐌 & 340C  & 㐍 & 340D  & 㐎 & 340E  & 㐏 & 341F  \\
㐐 & 3410  & 㐑 & 3411  & 㐒 & 3412  & 㐓 & 3413  \\
㐔 & 3414  & 㐕 & 3415  & 㐖 & 3416  & 㐗 & 3417  \\
㐘 & 3418  & 㐙 & 3419  & 㐚 & 341A  & 㐛 & 341B  \\
㐜 & 341C  & 㐝 & 341D  & 㐞 & 341E  & 㐟 & 341F  \\[1ex]
𠀀 & 20000 & 𠀁 & 20001 & 𠀂 & 20002 & 𠀃 & 20003 \\
𠀄 & 20004 & 𠀅 & 20005 & 𠀆 & 20006 & 𠀇 & 20007 \\
𠀈 & 20008 & 𠀉 & 20009 & 𠀊 & 2000A & 𠀋 & 2000B \\
𠀌 & 2000C & 𠀍 & 2000D & 𠀎 & 2000E & 𠀏 & 2000F \\
𠀐 & 20010 & 𠀑 & 20011 & 𠀒 & 20012 & 𠀓 & 20013 \\
𠀔 & 20014 & 𠀕 & 20015 & 𠀖 & 20016 & 𠀗 & 20017 \\
𠀘 & 20018 & 𠀙 & 20019 & 𠀚 & 2001A & 𠀛 & 2001B \\
𠀜 & 2001C & 𠀝 & 2001D & 𠀞 & 2001E & 𠀟 & 2001F \\
\end{tabular}
\end{table}
\end{document}
%    \end{macrocode}
%
%    \begin{macrocode}
%</ex-fallback>
%    \end{macrocode}
%
% \subsection{\pkg{example-SubCJKBlock.tex}}
%
%    \begin{macrocode}
%<*ex-block>
%    \end{macrocode}
%
% \begin{macrocode}
\documentclass{article}
\usepackage{xeCJK}
\usepackage{CJKfntef}
\usepackage{array}
\xeCJKDeclareSubCJKBlock{Ext-A} { "3400 -> "4DBF }
\xeCJKDeclareSubCJKBlock{Ext-B} { "20000 -> "2A6DF }
\xeCJKDeclareSubCJKBlock{Kana}  { "3040 -> "309F, "30A0 -> "30FF, "31F0 -> "31FF, }
\xeCJKDeclareSubCJKBlock{Hangul}{ "1100 -> "11FF, "3130 -> "318F, "A960 -> "A97F, "AC00 -> "D7AF }
\setCJKmainfont[Ext-A=SimHei,Ext-B=SimSun-ExtB]{SimSun}
\setCJKmainfont[Kana,BoldFont=meiryob.ttc]{meiryo.ttc}
\setCJKmainfont[Hangul,BoldFont=malgunbd.ttf]{malgun.ttf}
\parindent=2em
\begin{document}
\long\def\showtext{%
“中国是$x^2+y^2=z^2$位于亚洲东部的一个ABC地理区域范围,最早是指炎黄子孙在中原建立的国度,至现代国际体系成形后才开始作为国家的通称。作为其根基的中华文明是世界上古老的文明之一,对周边国家和民族的文化产生深远影响,形成了东亚/汉字文化圈。

「日本」\CJKunderwave{と$x^2+y^2=z^2$いABCう国号の表記は、太陽崇拝と相俟った自国中心的発想に基づくもの、また日本列島が中国大陸から見て東の果て、つまり「日の本(ひのもと)」に位}置することに由来して\addCJKfontfeatures{Kana=meiryob.ttc}いるのではないかとされる。憲法の表題に「日本国憲法」や「大日本帝国憲法」と示されているが、国号を「日本国」ないしは「日本」と直接かつ明確に規定した法令は存在しない。他に法律などで正式な国名を規定していない国としてはスペインなどが挙げられる。

\CJKspace
대한$x^2+y^2=z^2$민국은 동아시아의 한반ABC도 남반부에 자리한 공화국이다. 한일 병합 조약을 통하여 대한제국을 병합했던 일본 제국이 태평양 전쟁에서 연합국에게 패배하고, 이후 주요 연합국 중 하나인 미국이 한반도의 북위 38도 이남을 점령하면서 실시한 군정이 통치권을 이양하여, 1948년 8월 15일에 성립되었다.}
{\showtext}

\bigskip
\xeCJKCancelSubCJKBlock{Kana,Hangul}
{\showtext}

\bigskip
\xeCJKRestoreSubCJKBlock{Hangul}
{\showtext}

\begin{table}[ht]
\caption{生僻字测试}
\medskip\centering
\begin{tabular}{*4{|c>{\ttfamily U+}l}|}
㐀 & 3400  & 㐁 & 3401  & 㐂 & 3402  & 㐃 & 3403  \\
㐄 & 3404  & 㐅 & 3405  & 㐆 & 3406  & 㐇 & 3407  \\
㐈 & 3408  & 㐉 & 3409  & 㐊 & 340A  & 㐋 & 340B  \\
㐌 & 340C  & 㐍 & 340D  & 㐎 & 340E  & 㐏 & 341F  \\
㐐 & 3410  & 㐑 & 3411  & 㐒 & 3412  & 㐓 & 3413  \\
㐔 & 3414  & 㐕 & 3415  & 㐖 & 3416  & 㐗 & 3417  \\
㐘 & 3418  & 㐙 & 3419  & 㐚 & 341A  & 㐛 & 341B  \\
㐜 & 341C  & 㐝 & 341D  & 㐞 & 341E  & 㐟 & 341F  \\[1ex]
𠀀 & 20000 & 𠀁 & 20001 & 𠀂 & 20002 & 𠀃 & 20003 \\
𠀄 & 20004 & 𠀅 & 20005 & 𠀆 & 20006 & 𠀇 & 20007 \\
𠀈 & 20008 & 𠀉 & 20009 & 𠀊 & 2000A & 𠀋 & 2000B \\
𠀌 & 2000C & 𠀍 & 2000D & 𠀎 & 2000E & 𠀏 & 2000F \\
𠀐 & 20010 & 𠀑 & 20011 & 𠀒 & 20012 & 𠀓 & 20013 \\
𠀔 & 20014 & 𠀕 & 20015 & 𠀖 & 20016 & 𠀗 & 20017 \\
𠀘 & 20018 & 𠀙 & 20019 & 𠀚 & 2001A & 𠀛 & 2001B \\
𠀜 & 2001C & 𠀝 & 2001D & 𠀞 & 2001E & 𠀟 & 2001F \\
\end{tabular}
\end{table}
\end{document}
%    \end{macrocode}
%
%    \begin{macrocode}
%</ex-block>
%    \end{macrocode}
%
% \end{implementation}
% \Finale
\endinput
