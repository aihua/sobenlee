%# -*- coding:utf-8 -*-
%%%%%%%%%%%%%%%%%%%%%%%%%%%%%%%%%%%%%%%%%%%%%%%%%%%%%%%%%%%%%%%%%%%%%%%%%%%%%%%%%%%%%


\chapter{陈敬济失钥罚唱\KG 韩道国纵妇争锋}


词曰:

\[
衣染莺黄,爱停板驻拍,劝酒持觞。低鬟蝉影动,私语口脂香。檐滴露、竹风凉,拚剧饮琳琅。夜渐深笼灯就月,仔细端相。
\]

话说西门庆衙门中来家,进门就问月娘:“哥儿好些?使小厮请太医去。”月娘道:“我已叫刘婆子来了。吃了他药,孩子如今不洋奶,稳稳睡了这半日,觉好些了。”西门庆道:“信那老淫妇胡针乱灸,还请小儿科太医看才好。既好些了,罢。若不好,拿到衙门里去拶与老淫妇一拶子。”月娘道:“你恁的枉口拔舌骂人。你家孩儿现吃了他药好了,还恁舒着嘴子骂人!”说毕,丫鬟摆上饭来。西门庆刚才吃了饭,只见玳安儿来报:“应二爹来了。”西门庆教小厮:“拿茶出去,请应二爹卷棚内坐。”向月娘道:“把刚才我吃饭的菜蔬休动,教小厮拿饭出去,教姐夫陪他吃,说我就来。”月娘便问:“你昨日早晨使他往那里去?那咱才来。”西门庆便告说:“应二哥认的一个湖州客人何官儿,门外店里堆着五百两丝线,急等着要起身家去,来对我说要折些发脱。我只许他四百五十两银子。昨日使他同来保拿了两锭大银子作样银,已是成了来了,约下今日兑银子去。我想来,狮子街房子空闲,打开门面两间,倒好收拾开个绒线铺子,搭个伙计。况来保已是郓王府认纳官钱,教他与伙计在那里,又看了房儿,又做了买卖。”月娘道:“少不得又寻伙计。”西门庆道:“应二哥说他有一相识,姓韩,原是绒线行,如今没本钱,闲在家里,说写算皆精,行止端正,再三保举。改日领他来见我,写立合同。”说毕,西门庆在房中兑了四百五十两银子,教来保拿出来。陈敬济已陪应伯爵在卷棚内吃完饭,等的心里火发。见银子出来,心中欢喜,与西门庆唱了喏,说道:“昨日打搅哥,到家晚了,今日再扒不起来。”西门庆道:“这银子我兑了四百五十两,教来保取搭连眼同装了。今日好日子,便雇车辆搬了货来,锁在那边房子里就是了。”伯爵道:“哥主张的有理。只怕蛮子停留长智,推进货来就完了帐。”于是同来保骑头口,打着银子,迳到门外店中成交易去。谁知伯爵背地里与何官儿砸杀了,只四百二十两银子,打了三十两背工。对着来保,当面只拿出九两用银来,二人均分了。雇了车脚,即日推货进城,堆在狮子街空房内,锁了门,来回西门庆话。西门庆教应伯爵,择吉日领韩伙计来见。其人五短身材,三十年纪,言谈滚滚,满面春风。西门庆即日与他写立合同。同来保领本钱雇人染丝,在狮子街开张铺面,发卖各色绒丝。一日也卖数十两银子,不在话下。

光阴迅速,日月如梭,不觉八月十五日,月娘生辰来到,请堂客摆酒。留下吴大妗子、潘姥姥、杨姑娘并两个姑子住两日,晚夕宣唱佛曲儿,常坐到二三更才歇。那日,西门庆因上房有吴大妗子在这里,不方便,走到前边李瓶儿房中看官哥儿,心里要在李瓶儿房里睡。李瓶儿道:“孩子才好些儿,我心里不耐烦,往他五妈妈房里睡一夜罢。”西门庆笑道:“我不惹你。”于是走过金莲这边来。那金莲听见汉子进他房来,如同拾了金宝一般,连忙打发他潘姥姥过李瓶儿这边宿歇。他便房中高点银灯,款伸锦被,薰香澡牝,夜间陪西门庆同寝。枕畔之情,百般难述,无非只要牢宠汉子心,使他不往别人房里去。正是:鼓鬣游蜂,嫩蕊半匀春荡漾;餐香粉蝶,花房深宿夜风流。

李瓶儿见潘姥姥过来,连忙让在炕上坐的。教迎春安排酒菜果饼,晚夕说话,坐半夜才睡。到次日,与了潘姥姥一件葱白绫袄儿,两双缎子鞋面,二百文钱。把婆子欢喜的眉欢眼笑,过这边来,拿与金莲瞧,说:“这是那边姐姐与我的。”金莲见了,反说他娘:“好恁小眼薄皮的,什么好的,拿了他的来!”潘姥姥道:“好姐姐,人倒可怜见与我,你却说这个话。你肯与我一件儿穿?”金莲道:“我比不得他有钱的姐姐。我穿的还没有哩,拿什么与你!你平白吃了人家的来,等住回可整理几碟子来,筛上壶酒,拿过去还了他就是了。到明日少不的教人(石店)言试语,我是听不上。”一面分咐春梅,定八碟菜蔬,四盒果子,一锡瓶酒。打听西门庆不在家,教秋菊用方盒拿到李瓶儿房里,说:“娘和姥姥过来,无事和六娘吃杯酒。”李瓶儿道:“又教你娘费心。”少顷,金莲和潘姥姥来,三人坐定,把酒来斟。春梅侍立斟酒。

娘儿每说话间,只见秋菊来叫春梅,说:“姐夫在那边寻衣裳,教你去开外边楼门哩。”金莲分咐:“叫你姐夫寻了衣裳来这里喝瓯子酒去。”不一时,敬济寻了几家衣服,就往外走。春梅进来回说:“他不来。”金莲道:“好歹拉了他来。”又使出绣春去把敬济请来。潘姥姥在炕上坐,小桌儿摆着果盒儿,金莲、李瓶儿陪着吃酒。连忙唱了喏。金莲说:“我好意教你来吃酒儿,你怎的张致不来?就吊了造化了?呶了个嘴儿,教春梅:“拿宽杯儿来,筛与你姐夫吃。”敬济把寻的衣服放在炕上,坐下。春梅做定科范,取了个茶瓯子,流沿边斟上,递与他。慌的敬济说道:“五娘赐我,宁可吃两小锺儿罢。外边铺子里许多人等着要衣裳。”金莲道:“教他等着去,我偏教你吃这一大锺,那小锺子刁刁的不耐烦。”潘姥姥道:“只教哥哥吃这一锺罢,只怕他买卖事忙。”金莲道:“你信他!有什么忙!吃好少酒儿,金漆桶子吃到第二道箍上。”那敬济笑着拿酒来,刚呷了两口。潘姥姥叫春梅:“姐姐,你拿箸儿与哥哥。教他吃寡酒?”春梅也不拿箸,故意殴他,向攒盒内取了两个核桃递与他。那敬济接过来道:“你敢笑话我就禁不开他?”于是放在牙上只一磕,咬碎了下酒。潘姥姥道:“还是小后生家,好口牙。相老身,东西儿硬些就吃不得。”敬济道:“儿子世上有两椿儿——鹅卵石、牛犄角——吃不得罢了。”金莲见他吃了那锺酒,教春梅再斟上一锺儿,说:“头一锺是我的了。你姥姥和六娘不是人么?也不教你吃多,只吃三瓯子,饶了你罢。”敬济道:“五娘可怜见儿子来,真吃不得了。此这一锺,恐怕脸红,惹爹见怪。”金莲道:“你也怕你爹?我说你不怕他。你爹今日往那里吃酒去了?”敬济道:“后晌往吴驿丞家吃酒,如今在对门乔大户房子里看收拾哩。”金莲问:“乔大户家昨日搬了去,咱今日怎不与他送茶?”敬济道:“今早送茶去了。”李瓶儿问:“他家搬到那里住去了?”敬济道:“他在东大街上使了一千二百银子,买了所好不大的房子,与咱家房子差不多儿,门面七间,到底五层。”说话之间,敬济捏着鼻子又挨了一锺,趁金莲眼错,得手拿着衣服往外一溜烟跑了。迎春道:“娘你看,姐夫忘记钥匙去了。”那金莲取过来坐在身底下,向李瓶儿道:“等他来寻,你每且不要说,等我奈何他一回儿才与他。”潘姥姥道:“姐姐与他罢了,又奈何他怎的。”

那敬济走到铺子里,袖内摸摸,不见钥匙,一直走到李瓶儿房里寻。金莲道:“谁见你什么钥匙,你管着什么来?放在那里,就不知道?”春梅道:“只怕你锁在楼上了。”敬济道:“我记的带出来。”金莲道:“小孩儿家屁股大,敢吊了心!又不知家里外头什么人扯落的你恁有魂没识,心不在肝上。”敬济道:“有人来赎衣裳,可怎的样?趁爹不过来,免不得叫个小炉匠来开楼门,才知有没。”那李瓶儿忍不住,只顾笑。敬济道:“六娘拾了,与了我罢。”金莲道:“也没见这李大姐,不知和他笑什么,恰似我每拿了他的一般。”急得敬济只是牛回磨转,转眼看见金莲身底下露出钥匙带儿来,说道:“这不是钥匙!”才待用手去取,被金莲褪在袖内,不与他,说道:“你的钥匙儿,怎落在我手里?”急得那小伙儿只是杀鸡扯膝。金莲道:“只说你会唱的好曲儿,倒在外边铺子里唱与小厮听,怎的不唱个儿我听?今日趁着你姥姥和六娘在这里,只拣眼生好的唱个儿,我就与你这钥匙。不然,随你就跳上白塔,我也没有。”敬济道:“这五娘,就勒掯出人痞来。谁对你老人家说我会唱?”金莲道:“你还捣鬼?南京沈万三,北京枯树弯——人的名儿,树的影儿。”那小伙儿吃他奈何不过,说道:“死不了人,等我唱。我肚子里撑心柱肝,要一百个也有!”金莲骂道:“说嘴的短命!”自把各人面前酒斟上。金莲道:“你再吃一杯,盖着脸儿好唱。”敬济道:“我唱了慢慢吃。我唱个果子名《山坡羊》你听:

\[
初相交,在桃园儿里结义。相交下来,把你当玉黄李子儿抬举。人人说你在青翠花家饮酒,气的我把频波脸儿挝的粉粉的碎。我把你贼,你学了虎刺宾了,外实里虚,气的我李子眼儿珠泪垂。我使的一对桃奴儿寻你,见你在软枣儿树下就和我别离了去。气的我鹤顶红剪一柳青丝儿来呵,你海东红反说我理亏。骂了句生心红的强贼,逼的我急了,我在吊枝干儿上寻个无常,到三秋,我看你倚靠着谁?”
\]

唱毕,就问金莲要钥匙,说道:“五娘快与了我罢!伙计铺子里不知怎的等着我哩。只怕一时爹过来。”金莲道:“你倒自在性儿,说的且是轻巧。等你爹问,我就说你不知在那里吃了酒,把钥匙不见了,走来俺屋里寻。”敬济道:“爷嚛!五娘就是弄人的刽子手。”李瓶儿和潘姥姥再三旁边说道:“姐姐与他去罢。”金莲道:“若不是姥姥和你六娘劝我,定罚教你唱到天晚。头里骗嘴说一百个,才唱一个曲儿就要腾翅子?我手里放你不过。”敬济道:“我还有一个儿看家的,是银名《山坡羊》,亦发孝顺你老人家罢。”于是顿开喉音唱道:

\[
冤家你不来,白闷我一月,闪的人反拍着外膛儿细丝谅不彻。我使狮子头定儿小厮拿着黄票儿请你,你在兵部洼儿里元宝儿家欢娱过夜。我陪铜磬儿家私为焦心一旦儿弃舍,我把如同印箝儿印在心里愁无求解。叫着你把那挺脸儿高扬着不理,空教我拨着双火筒儿顿着罐子等到你更深半夜。气的奴花银竹叶脸儿咬定银牙来呵,唤官银顶上了我房门,随那泼脸儿冤家轻敲儿不理。骂了句煎彻了的三倾儿捣槽斜贼,空把奴一腔子暖汁儿真心倒与你,只当做热血。
\]

敬济唱毕,金莲才待叫春梅斟酒与他,忽有月娘从后边来,见奶子如意儿抱着官哥儿在房门首石基上坐,便说道:“孩子才好些,你这狗肉又抱他在风里,还不抱进去!”金莲问:“是谁说话?”绣春回道:“大娘来了。”敬济慌的拿钥匙往外走不迭。众人都下来迎接月娘。月娘便问:“陈姐夫在这里做什么来?”金莲道:“李大姐整治些菜,请俺娘坐坐。陈姐夫寻衣服,叫他进来吃一杯。姐姐,你请坐,好甜酒儿,你吃一杯。”月娘道:“我不吃。后边他大妗子和杨姑娘要家去,我又记挂着这孩子,迳来看看。李大姐,你也不管,又教奶子抱他在风里坐的。前日刘婆子说他是惊寒,人还不好生看他!”李瓶儿道:“俺陪着姥姥吃酒,谁知贼臭肉三不知抱他出去了。”月娘坐了半歇,回后边去了。一回,使小玉来,请姥姥和五娘、六娘后边坐。那潘金莲和李瓶儿匀了脸,同潘姥姥往后边来,陪大妗子、杨姑娘吃酒。到日落时分,与月娘送出大门,上轿去了。都在门里站立,先是孟玉楼说道:“大姐姐,今日他爹不在,往吴驿丞家吃酒去了,咱到好往对门乔大户家房里瞧瞧。”月娘问看门的平安儿:“谁拿着那边钥匙哩?”平安道:“娘每要过去瞧,开着门哩。来兴哥看着两个坌工的在那里做活。”月娘分咐:“你教他躲开,等俺每瞧瞧去。”平安儿道:“娘每只顾瞧,不妨事。他每都在第四层大空房拨灰筛土,叫出来就是了。”

当下月娘、李娇儿、孟玉楼、潘金莲、李瓶儿,都用轿子短搬抬过房子内。进了仪门,就是三间厅。第二层是楼。月娘要上楼去,可是作怪,刚上到楼梯中间,不料梯磴陡趄,只闻月娘哎了一声,滑下一只脚来,早是月娘攀住楼梯两边栏杆。慌了玉楼,便道:“姐姐怎的?”连忙搊住他一只胳膊,不曾跌下来。月娘吃了一惊,就不上去。众人扶了下来,唬的脸蜡查儿黄了。玉楼便问:“姐姐,怎么上来滑了脚,不曾扭着那里?”月娘道:“跌倒不曾跌着,只是扭了腰子,唬的我心跳在口里。楼梯子趄,我只当咱家里楼上来,滑了脚。早是攀住栏杆,不然怎了!”李娇儿道:“你又身上不方便,早知不上楼也罢了。”于是众姊妹相伴月娘回家。刚到家,叫的应就肚中疼痛。月娘忍不过,趁西门庆不在家,使小厮叫了刘婆子来看。婆子道:“你已是去经事来着伤,多是成不的了。”月娘道:“便了五个多月了,上楼着了扭。”婆子道:“你吃了我这药,安不住,下来罢了。”月娘道:“下来罢!”婆子于是留了两服大黑丸子药,教月娘用艾酒吃。那消半夜,吊下来了,在马桶里。点灯拨看,原来是个男胎,已成形了。正是:

\[
胚胎未能成性命,真灵先到杳冥天。
\]
幸得那日西门庆在玉楼房中歇了。

到次日,玉楼早晨到上房,问月娘:“身子如何?”月娘告诉:“半夜果然疼不住,落下来了,倒是小厮儿。”玉楼道:“可惜了!他爹不知道?”月娘道:“他爹吃酒来家,到我屋里才待脱衣裳,我说你往他们屋里去罢,我心里不自在。他才往你这边来了。我没对他说。我如今肚里还有些隐隐的疼。”玉楼道:“只怕还有些余血未尽,筛酒吃些锅脐灰儿就好了。”又道:“姐姐,你还计较两日儿,且在屋里不可出去。小产比大产还难调理,只怕掉了风寒,难为你的身子。”月娘道:“你没的说,倒没的唱扬的一地里知道,平白噪剌剌的抱什么空窝,惹的人动那唇齿。”以此就没教西门庆知道。此事表过不题。

且说西门庆新搭的开绒线铺伙计,也不是守本分的人,姓韩名道国,字希尧,乃是破落户韩光头的儿子。如今跌落下来,替了大爷的差使,亦在郓王府做校尉,见在县东街牛皮小巷居住。其人性本虚飘,言过其实,巧于词色,善于言谈。许人钱,如捉影捕风;骗人财,如探囊取物。自从西门庆家做了买卖,手里财帛从容,新做了几件虼蚤皮,在街上掇着肩膊儿就摇摆起来。人见了不叫他个韩希尧,只叫他做“韩一摇”。他浑家乃是宰牲口王屠妹子,排行六儿,生的长跳身材,瓜子面皮,紫膛色,约二十八九年纪。身边有个女孩儿,嫡亲三口儿度日。他兄弟韩二,名二捣鬼,是个耍钱的捣子,在外边另住。旧与这妇人有奸,赶韩道国不在家,铺中上宿,他便时常走来与妇人吃酒,到晚夕刮涎就不去了。不想街坊有几个浮浪子弟,见妇人搽脂抹粉,打扮的乔模乔样,常在门首站立睃人,人略斗他斗儿,又臭又硬,就张致骂人。因此街坊这些小伙子儿,心中有几分不愤,暗暗三两成群,背地讲论,看他背地与什么人有首尾。那消半个月,打听出与他小叔韩二这件事来。原来韩道国这间屋门面三间,房里两边都是邻舍,后门逆水塘。这伙人,单看韩二进去,或夜晚扒在墙上看觑,或白日里暗使小猴子在后塘推道捉蛾儿,单等捉奸。不想那日二捣鬼打听他哥不在,大白日装酒和妇人吃,醉了,倒插了门,在房里干事。不防众人睃见踪迹,小猴子扒过来,把后门开了,众人一齐进去,掇开房门。韩二夺门就走,被一少年一拳打倒拿住。老婆还在炕上,慌穿衣不迭。一人进去,先把裤子挝在手里,都一条绳子拴出来。须臾,围了一门首人,跟到牛皮街厢铺里,就哄动了那一条街巷。这一个来问,那一个来瞧,内中一老者见男妇二人拴做一处,便问左右看的人:“此是为什么事的?”旁边有多口的道:“你老人家不知,此是小叔奸嫂子的。”那老都点了点头儿说道:“可伤,原来小叔儿要嫂子的,到官,叔嫂通奸,两个都是绞罪。”那旁边多口的,认的他有名叫做陶扒灰,一连娶三个媳妇,都吃他扒了,因此插口说道:“你老人家深通条律,相这小叔养嫂子的便是绞罪,若是公公养媳妇的却论什么罪?”那老者见不是话,低着头一声儿没言语走了。正是:各人自扫檐前雪,莫管他人屋上霜。这里二捣鬼与妇人被捉不题。

单表那日,韩道国铺子里不该上宿,来家早,八月中旬天气,身上穿着一套儿轻纱软绢衣服,新盔的一顶帽儿,在街上阔行大步摇摆。但遇着人,或坐或立,口惹悬河,滔滔不绝。就是一回,内中遇着他两个相熟的人,一个是开纸铺的张二哥,一个是开银铺的白四哥,慌作揖举手。张好问便道:“韩老兄连日少见,闻得恭喜在西门大官府上,开宝铺做买卖,我等缺礼失贺,休怪休怪!”一面让他坐下。那韩道国坐在凳上,把脸儿扬着,手中摇着扇儿,说道:“学生不才,仗赖列位余光,与我恩主西门大官人做伙计,三七分钱。掌巨万之财,督数处之铺,甚蒙敬重,比他人不同。”白汝晃道:“闻老兄在他门下只做线铺生意。”韩道国笑道:“二兄不知,线铺生意只是名目而已。他府上大小买卖,出入资本,那些儿不是学生算帐!言听计从,祸福共知,通没我一时儿也成不得。大官人每日衙门中来家摆饭,常请去陪侍,没我便吃不下饭去。俺两个在他小书房里,闲中吃果子说话儿,常坐半夜他方进后边去。昨日他家大夫人生日,房下坐轿子行人情,他夫人留饮至二更方回。彼此通家,再无忌惮。不可对兄说,就是背地他房中话儿,也常和学生计较。学生先一个行止端庄,立心不苟,与财主兴利除害,拯溺救焚。凡百财上分明,取之有道。就是傅自新也怕我几分。不是我自己夸奖,大官人正喜我这一件儿。”刚说在热闹处,忽见一人慌慌张张走向前叫道:“韩大哥,你还在这里说什么,教我铺子里寻你不着。”拉到僻静处告他说:“你家中如此这般,大嫂和二哥被街坊众人撮弄了,拴到铺里,明早要解县见官去。你还不早寻人情理会此事?”这韩道国听了,大惊失色。口中只咂嘴,下边顿足,就要翅趫走。被张好问叫道:“韩老兄,你话还未尽,如何就去了?”这韩道国举手道:“大官人有要紧事,寻我商议,不及奉陪。”慌忙而去。正是:

\[
谁人挽得西江水,难洗今朝一面羞。
\]
