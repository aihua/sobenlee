% \iffalse meta-comment
% !TEX program  = XeLaTeX
%<*internal>
\ifnum\strcmp{\fmtname}{plain}=0 \else
  \expandafter\begingroup
\fi
%</internal>
%<*install>
\input docstrip.tex
\keepsilent
\askforoverwritefalse
\preamble

   Copyright (C) 2012 by Qing Lee <sobenlee@gmail.com>
--------------------------------------------------------------------------
   This work may be distributed and/or modified under the
   conditions of the LaTeX Project Public License, either version 1.3
   of this license or (at your option) any later version.
   The latest version of this license is in
     http://www.latex-project.org/lppl.txt
   and version 1.3 or later is part of all distributions of LaTeX
   version 2005/12/01 or later.

   This work has the LPPL maintenance status "maintained".
   The Current Maintainer of this work is Qing Lee.

\endpreamble
\postamble

   This package consists of the file  zhuyin.dtx,
                                      zhuyin-pinyin.cfg,
                and the derived files zhuyin.sty,
                                      zhuyin.pdf and
                                      zhuyin.ins.
\endpostamble
\usedir{tex/latex/zhuyin}
\generate{\file{\jobname.sty}{\from{\jobname.dtx}{package}}}
%</install>
%<install>\endbatchfile
%<*internal>
\usedir{source/latex/zhuyin}
\generate{\file{\jobname.ins}{\from{\jobname.dtx}{install}}}
\ifnum\strcmp{\fmtname}{plain}=0
  \expandafter\endbatchfile
\else
  \expandafter\endgroup
\fi
%</internal>
%
%<*driver|package>
\NeedsTeXFormat{LaTeX2e}
%<*driver>
\ProvidesFile{zhuyin.dtx}
%</driver>
\RequirePackage{expl3}
\GetIdInfo$Id$
          {package for typesetting Chinese phonetic alphabets}
%</driver|package>
%
%<*driver>
\documentclass{l3doc}
\usepackage{zhuyin}
\usepackage{fvrb-ex}
\usepackage{metalogo}
\hypersetup{pdfstartview=FitH}
\fvset{formatcom=\CJKfixedspacing}
\linespread{1.1}
\addtolength{\voffset}{-5\baselineskip}
\addtolength\textheight{8\baselineskip}
\setmainfont{TeX Gyre Pagella}
\setmonofont{Inconsolata}
\setCJKmainfont[BoldFont=Adobe Heiti Std,ItalicFont=Adobe Kaiti Std]{Adobe Song Std}
\setCJKmonofont{Adobe Kaiti Std}
\xeCJKsetup{PunctStyle=kaiming}
\newfontfamily\PinYinFont{FreeSans}
\zhuyinsetup{font=\PinYinFont}
\def\MacroFont{\small\normalfont\ttfamily}
\makeatletter
\let\orig@meta\meta
\def\meta#1{\orig@meta{\normalfont\itshape#1}}
\def\TF{true\orvar{}false}
\def\TTF{\defaultvar{true}\orvar{}false}
\def\TFF{true\orvar\defaultvar{false}}
\def\orvar{\char`\|}
\let\defaultvar\textbf
\def\argbrace#1{\char`\{#1\char`\}}
\makeatother
\def\indexname{代码索引}
\IndexPrologue{%
  \section*{\indexname}
  \markboth{\indexname}{\indexname}
  斜体的数字表示对应项说明所在的页码,下划线的数字表示定义所在的代码行号,而直立体的
  数字表示对应项使用时所在的行号。}
\begin{document}
  \DocInput{\jobname.dtx}
\end{document}
%</driver>
% \fi
%
% \GetFileInfo{\jobname.sty}
%
% \title{\bfseries\pkg{zhuyin} 宏包}
% \author{李清\\ \path{sobenlee@gmail.com}}
% \date{\filedate\qquad\fileversion}
% \maketitle
%
% \begin{documentation}
%
% \section{简介}
% \pkg{zhuyin} (注音)是一个 \XeLaTeX 宏包,提供了为汉字自动注音的功能。
%
% \pkg{zhuyin} 依赖 \pkg{xeCJK} 宏包。
%
% \section{使用方法}
%
% \begin{function}{zhuyinscope}
%   \begin{syntax}
%     \cs{begin}\argbrace{zhuyinscope}\oarg{options}
%     .....
%     \cs{end}\argbrace{zhuyinscope}
%   \end{syntax}
%   为 \env{zhuyinscope} 环境中的汉字自动注音。例如
%   \begin{Example}[frame=single,numbers=left,gobble=5]
%     \begin{zhuyinscope}
%     列位看官:你道此书从何而来?说起根由,虽近荒唐,细按则深有趣味。
%     待在下将此来历注明,方使阅者了然不惑。
%     \end{zhuyinscope}
%   \end{Example}
%   可选项 \meta{options} 用于局部设置拼音的格式,将在下面说明。
% \end{function}
%
% \begin{function}{\zhuyin}
%   \begin{syntax}
%     \cs{zhuyin} \oarg{options} \Arg{单个汉字} \Arg{拼音}
%     \cs{zhuyin*} \oarg{options} \Arg{文字}
%   \end{syntax}
%   对于多音字,可以使用 \cs{zhuyin} 为其设置拼音;而 \cs{zhuyin*} 相当于
%   \env{zhuyinscope} 环境的命令形式。\cs{zhuyin} 可以在 \env{zhuyinscope} 环境和
%   \cs{zhuyin*} 中使用。例如,\\[1ex]
%   \begin{SideBySideExample}[frame=single,numbers=left,xrightmargin=.5\linewidth,gobble=5]
%     \zhuyin{长}{chang1}\\
%     \zhuyin*{甄士隐梦幻识通灵}\\
%     \zhuyin*{\zhuyin{重}{zhong4}要}
%   \end{SideBySideExample}
% \end{function}
%
% \begin{function}{\pinyin}
%   \begin{syntax}
%     \cs{pinyin} \oarg{options} \Arg{拼音}
%   \end{syntax}
%   用于输出拼音,为了输入的方便 \texttt{\"u} 可以用 |v| 代替。例如,\\[1ex]
%   \begin{SideBySideExample}[frame=single,numbers=left,xrightmargin=.5\linewidth,gobble=5]
%     \pinyin{hao3 hao3 xve2 xi2}\\
%     \pinyin{nv3hai2zi}
%   \end{SideBySideExample}
% \end{function}
%
% \begin{function}{\setpinyin}
%   \begin{syntax}
%     \cs{setpinyin} \Arg{汉字} \Arg{拼音}
%   \end{syntax}
%   \pkg{zhuyin} 宏包的拼音数据(\file{zhuyin-pinyin.cfg})来源于 \texttt{Unicode 6.1.0}
%   的 \texttt{Unihan} 数据库\footnotemark 中的 \file{Unihan_Readings.txt} 文件。
%   对于多音字,一般来说这个文件选用的是常用读音。可以使用 \cs{setpinyin} 来设置多音字
%   的首选读音。
% \end{function}
%
% \footnotetext{\url{http://www.unicode.org/Public/UNIDATA/Unihan.zip}}
%
% \begin{function}{\zhuyinsetup}
%   \begin{syntax}
%     \cs{zhnumsetup} \argbrace{\meta{key1}=\meta{var1}, \meta{key2}=\meta{var2}, ...}
%   \end{syntax}
%   用于在导言区或文档中,设置拼音的格式。目前可以设置的 \meta{key} 如下介绍。
% \end{function}
%
% \begin{function}{ratio}
%   \begin{syntax}
%     ratio = \marg{number}
%   \end{syntax}
%   设置拼音字体大小与当前正文字体大小的比例,缺省值是 |0.4|。
% \end{function}
%
% \begin{function}{vsep}
%   \begin{syntax}
%     vsep = \marg{dimen}
%   \end{syntax}
%   设置拼音的基线与汉字顶部的间距,缺省值是 |0.25 ex|。
% \end{function}
%
% \begin{function}{hsep}
%   \begin{syntax}
%     hsep = \marg{skip}
%   \end{syntax}
%   设置注音汉字之间的间距,缺省值与 \pkg{xeCJK} 的 \cs{CJKglue} 的值相同。
%   为了断行时行末的对齐,设置的 \meta{skip} 最后有一定的弹性。例如
%   \begin{Example}[frame=single,numbers=left,gobble=5]
%     \zhuyin*[ratio={.7}, hsep={.5em plus .1em}, vsep={.5ex}]{贾雨村风尘怀闺秀}
%   \end{Example}
% \end{function}
%
% \begin{function}{pysep}
%   \begin{syntax}
%     pysep = \marg{glue}
%   \end{syntax}
%   设置 \cs{pinyin} 输出的相邻两个汉语拼音的空白,缺省值是一个空格。
% \end{function}
%
% \begin{function}{font}
%   \begin{syntax}
%     font = \marg{font}
%   \end{syntax}
%   设置拼音的字体,缺省值时 \cs{normalfont},即以正文西文字体相同。可以用
%   \pkg{fontspec} 宏包提供的 \cs{newfontfamily} 来选定字体。为了保证拼音能正确
%   输出,最好选用收字量较大的西文字体。
% \end{function}
%
% \begin{function}{format}
%   \begin{syntax}
%     format = \marg{format}
%   \end{syntax}
%   设置拼音的其它格式,例如颜色等,缺省值为空。
% \end{function}
%
% \end{documentation}
%
% \begin{implementation}
%
% \section{代码实现}
%
% \iffalse
%<*package>
% \fi
%
%    \begin{macrocode}
\ProvidesExplPackage{\ExplFileName}{\ExplFileDate}{1.0}{\ExplFileDescription}
\RequirePackage{xeCJK}
%    \end{macrocode}
%
% \begin{macro}[internal]{\zhuyin_make_box:Nn}
%    \begin{macrocode}
\cs_new_nopar:Nn \zhuyin_make_box:Nn
  { \zhuyin_save_CJKsymbol:N #1 \zhuyin_make_pinyin_box:Nn {#1} {#2} }
%    \end{macrocode}
% \end{macro}
%
% \begin{macro}[internal]{\zhuyin_make_pinyin_box:Nn}
%    \begin{macrocode}
\cs_new_nopar:Nn \zhuyin_make_pinyin_box:Nn
  {
    \hbox_overlap_left:n
      {
        \settowidth \l_tmpa_dim { \CJKglue }
        \hbox_set:Nn \l_tmpa_box
          {
            \makexeCJKinactive
            \int_set_eq:NN \XeTeXuseglyphmetrics \c_zero
            \xeCJK_select_font:
            #1
          }
        \hbox_set:Nn \l_tmpb_box
          { \zhuyin_select_font: \l_zhuyin_format_tl #2 }
        \dim_compare:nNnT
          { \box_wd:N \l_tmpb_box } > { \box_wd:N \l_tmpa_box + \l_tmpa_dim }
          {
            \box_resize:Nff \l_tmpb_box
              { \dim_eval:n { \box_wd:N \l_tmpa_box + \l_tmpa_dim } }
              { \dim_eval:n { \box_ht:N \l_tmpb_box + \box_dp:N \l_tmpb_box } }
          }
        \box_move_up:nn { \box_ht:N \l_tmpa_box + ( \l_zhuyin_vsep_tl ) }
          {
            \hbox_to_wd:nn { \box_wd:N \l_tmpa_box }
              { \tex_hss:D  \box_use_clear:N \l_tmpb_box \tex_hss:D }
          }
      }
    { \xeCJK_CJK_kern: }
  }
\cs_generate_variant:Nn \box_resize:Nnn { Nff }
%    \end{macrocode}
% \end{macro}
%
% \begin{macro}[internal]{\zhuyin_to_pinyin:N}
%    \begin{macrocode}
\cs_new_nopar:Nn \zhuyin_to_pinyin:N
  { \use:c { c_pinyin_ \int_to_hexadecimal:n {`#1} _tl } }
%    \end{macrocode}
% \end{macro}
%
% \begin{macro}[internal,var]{\l_zhuyin_coor_tl}
%    \begin{macrocode}
\tl_set:Nn \l_zhuyin_coor_tl
  { ( \cs_meaning:N \l_zhuyin_font_tl ) / \l_xeCJK_font_coor_tl / \l_zhuyin_ratio_tl }
%    \end{macrocode}
% \end{macro}
%
% \begin{macro}[internal]{\zhuyin_select_font:}
%    \begin{macrocode}
\cs_new_nopar:Nn \zhuyin_select_font:
  {
    \cs_if_exist_use:cF { \l_zhuyin_coor_tl }
      {
        \tl_set:Nx \l_zhuyin_current_coor_tl { \l_zhuyin_coor_tl }
        \dim_set:Nn \l_tmpa_dim { \f@size \p@ }
        \fontsize { \l_zhuyin_ratio_tl \l_tmpa_dim } \c_zero_skip
        \l_zhuyin_font_tl
        \selectfont
        \exp_last_unbraced:NNV
        \cs_gset_eq:cN \l_zhuyin_current_coor_tl \font@name
      }
  }
%    \end{macrocode}
% \end{macro}
%
% \begin{macro}[internal]{\zhuyin_CJKsymbol:N}
%    \begin{macrocode}
\cs_new_nopar:Nn \zhuyin_CJKsymbol:N
  { \zhuyin_make_box:Nn #1 { \zhuyin_to_pinyin:N #1 } }
%    \end{macrocode}
% \end{macro}
%
% \begin{macro}{zhuyinscope}
%    \begin{macrocode}
\NewDocumentEnvironment { zhuyinscope } { O{} }
  {
    \keys_set:nn { zhuyin } {#1}
    \tl_if_empty:NF \l_zhuyin_hsep_tl
      { \xeCJKsetup { CJKglue = \skip_horizontal:n { \l_zhuyin_hsep_tl } } }
    \zhuyin_replace_CJKsymbol:
  }
  { }
%    \end{macrocode}
% \end{macro}
%
% \begin{macro}{\zhuyin}
%    \begin{macrocode}
\NewDocumentCommand \zhuyin { s O{} m }
  {
    \IfBooleanTF {#1}
      {
        \group_begin:
        \keys_set:nn { zhuyin } {#2}
        \tl_if_empty:NF \l_zhuyin_hsep_tl
          { \xeCJKsetup { CJKglue = \skip_horizontal:n { \l_zhuyin_hsep_tl } } }
        \zhuyin_replace_CJKsymbol:
        #3
        \group_end:
      }
      {
        \group_begin:
        \keys_set:nn { zhuyin } {#2}
        \zhuyin_zhuyin_single_aux:Nn {#3}
      }
  }
%    \end{macrocode}
% \end{macro}
%
% \begin{macro}[internal]{\zhuyin_zhuyin_single_aux:Nn}
%    \begin{macrocode}
\cs_new_nopar:Nn \zhuyin_zhuyin_single_aux:Nn
  {
    \cs_if_eq:NNTF \CJKsymbol \zhuyin_CJKsymbol:N
      { \cs_set_eq:NN \CJKsymbol \use:n }
      { \cs_set_eq:NN \zhuyin_save_CJKsymbol:N \use:n }
    \zhuyin_make_box:Nn {#1} { \zhuyin_pinyin:n {#2} }
    \group_end:
  }
%    \end{macrocode}
% \end{macro}
%
% \begin{macro}{\zhuyin_replace_CJKsymbol:}
%    \begin{macrocode}
\cs_new_nopar:Nn \zhuyin_replace_CJKsymbol:
  {
    \cs_if_eq:NNF \CJKsymbol \zhuyin_CJKsymbol:N
      {
        \cs_set_eq:NN \zhuyin_save_CJKsymbol:N \CJKsymbol
        \cs_set_eq:NN \CJKsymbol \zhuyin_CJKsymbol:N
      }
  }
%    \end{macrocode}
% \end{macro}
%
% \begin{macro}{\pinyin}
%    \begin{macrocode}
\NewDocumentCommand \pinyin { O{} m }
  {
    \group_begin:
    \keys_set:nn { zhuyin } {#1}
    \l_zhuyin_font_tl
    \l_zhuyin_format_tl
    \selectfont
    \zhuyin_pinyin:n {#2}
    \group_end:
  }
%    \end{macrocode}
% \end{macro}
%
% \begin{macro}[internal]{\zhuyin_pinyin:n}
%    \begin{macrocode}
\cs_new_nopar:Nn \zhuyin_pinyin:n
  {
    \zhuyin_pinyin_init:
    \bool_set_true:N \l_zhuyin_first_bool
    \zhuyin_pinyin_aux:N #1 \q_recursion_tail
    \prg_break_point:n
      {
        \bool_if:NTF \l_zhuyin_first_bool {#1}
          { \tl_if_empty:NF \l_tmpc_tl { \l_zhuyin_pysep_tl \l_tmpc_tl } }
      }
  }
%    \end{macrocode}
% \end{macro}
%
% \begin{macro}[internal]{\zhuyin_pinyin_aux:N}
%    \begin{macrocode}
\cs_new_nopar:Nn \zhuyin_pinyin_aux:N
  {
    \quark_if_recursion_tail_break:N #1
    \zhuyin_if_number:NTF {#1}
      {
        \bool_if:NTF \l_zhuyin_first_bool
          { \bool_set_false:N \l_zhuyin_first_bool }
          { \l_zhuyin_pysep_tl }
        \l_tmpa_tl
        \zhuyin_accent:Vn \l_zhuyin_accent_tl {#1}
        \l_tmpb_tl
        \zhuyin_pinyin_init:
      }
      {
        \int_compare:nNnTF
          { 0 \use:c { c_zhuyin_ \l_zhuyin_accent_tl _tl } } >
          { 0 \use:c { c_zhuyin_ #1 _tl } }
          { \tl_put_right:Nx \l_tmpb_tl { \zhuyin_replace_v:N {#1} } }
          {
            \tl_set:Nn \l_zhuyin_accent_tl {#1}
            \tl_set_eq:NN \l_tmpa_tl \l_tmpc_tl
            \tl_clear:N \l_tmpb_tl
          }
        \tl_put_right:Nx \l_tmpc_tl { \zhuyin_replace_v:N {#1} }
      }
    \zhuyin_pinyin_aux:N
  }
%    \end{macrocode}
% \end{macro}
%
% \begin{macro}[internal]{\zhuyin_accent:Nn}
%    \begin{macrocode}
\cs_new_nopar:Nn \zhuyin_accent:Nn
  {
    \str_if_eq:nnTF {#1} { v }
      { \zhuyin_num_to_accent_v:n {#2} }
      { \zhuyin_num_to_accent:Nn {#1} {#2} }
  }
\cs_generate_variant:Nn \zhuyin_accent:Nn { V }
%    \end{macrocode}
% \end{macro}
%
% \begin{macro}[internal]{\zhuyin_replace_v:N}
%    \begin{macrocode}
\cs_new_nopar:Nn \zhuyin_replace_v:N
  { \str_if_eq:nnTF {#1} { v } { ü } {#1} }
%    \end{macrocode}
% \end{macro}
%
% \begin{macro}[internal]{\zhuyin_pinyin_init:}
%    \begin{macrocode}
\cs_new_nopar:Nn \zhuyin_pinyin_init:
  {
    \tl_clear:N \l_tmpa_tl
    \tl_clear:N \l_tmpb_tl
    \tl_clear:N \l_tmpc_tl
    \tl_clear:N \l_zhuyin_accent_tl
  }
%    \end{macrocode}
% \end{macro}
%
% \begin{macro}[internal,pTF]{\zhuyin_if_number:N}
%    \begin{macrocode}
\prg_new_conditional:Nnn \zhuyin_if_number:N { p , T , F , TF }
  {
    \if_int_compare:w \c_one < 1 #1 \exp_stop_f:
      \prg_return_true: \else: \prg_return_false: \fi:
  }
%    \end{macrocode}
% \end{macro}
%
% \begin{macro}[internal,var]{\l_zhuyin_first_bool}
%    \begin{macrocode}
\bool_new:N \l_zhuyin_first_bool
%    \end{macrocode}
% \end{macro}
%
% \begin{macro}[internal,var]
% {\c_zhuyin_a_tl,\c_zhuyin_o_tl,\c_zhuyin_e_tl,\c_zhuyin_i_tl,
% \c_zhuyin_u_tl,\c_zhuyin_v_tl}
%    \begin{macrocode}
\tl_const:Nn \c_zhuyin_a_tl { 3 }
\tl_const:Nn \c_zhuyin_o_tl { 2 }
\tl_const:Nn \c_zhuyin_e_tl { 2 }
\tl_const:Nn \c_zhuyin_i_tl { 1 }
\tl_const:Nn \c_zhuyin_u_tl { 1 }
\tl_const:Nn \c_zhuyin_v_tl { 1 }
%    \end{macrocode}
% \end{macro}
%
% \begin{macro}[internal]{\zhuyin_num_to_accent:Nn,\zhuyin_num_to_accent_v:n}
%    \begin{macrocode}
\cs_new_nopar:Nn \zhuyin_num_to_accent:Nn
  {
    \if_case:w \int_eval:w #2 \int_eval_end:
      #1  \or: \= #1  \or: \' #1  \or: \v #1  \or: \` #1
    \else: #1 \fi:
  }
\cs_new_nopar:Nn \zhuyin_num_to_accent_v:n
  {
    \if_case:w \int_eval:w #1 \int_eval_end:
       ü  \or: ǖ  \or: ǘ  \or: ǚ  \or: ǜ
    \else: ü \fi:
  }
%    \end{macrocode}
% \end{macro}
%
% \begin{macro}{\zhuyinsetup}
%    \begin{macrocode}
\NewDocumentCommand \zhuyinsetup { m } { \keys_set:nn { zhuyin } {#1} }
%    \end{macrocode}
% \end{macro}
%
% \begin{macro}{ratio,vsep,hsep,pysep,font,format}
%    \begin{macrocode}
\keys_define:nn { zhuyin }
  {
    ratio  .tl_set:N = \l_zhuyin_ratio_tl  ,
    vsep   .tl_set:N = \l_zhuyin_vsep_tl   ,
    hsep   .tl_set:N = \l_zhuyin_hsep_tl   ,
    pysep  .tl_set:N = \l_zhuyin_pysep_tl  ,
    font   .tl_set:N = \l_zhuyin_font_tl   ,
    format .tl_set:N = \l_zhuyin_format_tl ,
  }
\keys_set:nn { zhuyin }
  {
    ratio = .4 ,
    vsep  = .25 ex ,
    pysep = \c_space_tl ,
    font  = \normalfont
  }
%    \end{macrocode}
% \end{macro}
%
% \begin{macro}[internal]{zhuyin-pinyin.cfg}
%    \begin{macrocode}
\group_begin:
\char_set_catcode_active:N \U
\cs_set_nopar:Npn U+ #1 ~ #2 ~ { \tl_const:cn { c_pinyin_ #1 _ tl } {#2} }
\char_set_catcode_space:N \
\file_input:n {zhuyin-pinyin.cfg}
\group_end:
%    \end{macrocode}
% \end{macro}
%
% \begin{macro}{\setpinyin}
%    \begin{macrocode}
\NewDocumentCommand \setpinyin { m m }
  {
    \tl_set:cn
      { c_pinyin_ \int_to_hexadecimal:n {`#1} _ tl }
      { \zhuyin_pinyin:n {#2} }
  }
%    \end{macrocode}
% \end{macro}
%
%    \begin{macrocode}
\ProcessKeysOptions { zhuyin }
%    \end{macrocode}
%
% \iffalse
%</package>
% \fi
%
% \end{implementation}
%
% \PrintIndex
% \Finale
%
\endinput
