%# -*- coding:utf-8 -*-
%%%%%%%%%%%%%%%%%%%%%%%%%%%%%%%%%%%%%%%%%%%%%%%%%%%%%%%%%%%%%%%%%%%%%%%%%%%%%%%%%%%%%
%%  MAChapter2.tex'


\chapter{极\emspace 限}\label{ch:2}

高等数学与初等数学的差别,除了研究对象不同外,主要是研究方法上的不同。我们知道,初等数学的方法建立在有限观念上,而
高等数学的方法则是建立在无限观念之上。比如初等数学中要求一个数,通过有限步的代数运算,即可求出它的准确值。但在客观
上存在着这样一种数,若只进行有限步的代数运算,则无法求得其准确的值。例如圆的面积和周长,用有限步代数运算就不能求得
其准确值,必须通过无限步逼近,即所谓极限方法,才能求出它的准确值,这就是高等数学的方法。又比如初等数学要确定数的性
质(如是否是素数),理论上通过有限步运算,就能断定它是否具有这一性质;而高等数学要确定函数的性质,就要通过极限方法
才能确定它是否具有此性质。

所以理解极限概念、掌握极限方法,是能否学好高等数学的关键。只有掌握极限这把钥匙,才能打开通向微积分的大门,变门外汉
为驾驭微积分工具的主人。


\section{序列极限的定义}

\subsection{概念引入}

试求由抛物线~$y=x^2$、$x$~轴、直线~$x=1$~所围成的曲边三角形的面积。

\begin{wrapfigure}[9]{O}{0mm}
\somefigure
\caption{}\label{fig:sec2.1-1}
\end{wrapfigure}

我们会求直边形的面积,不会求曲边形的面积。但是我们可以分两步来求出曲边形面积:先是通过以直代曲,得到一系列越来越逼近曲边三
角形面积的近似值;然后考察这一系列近似值的变化趋势,从而确定出曲边三角形面积的准确值。为此,如\ref{fig:sec2.1-1}~所示,用分点
\[
  x=0,\enspace\frac1n,\enspace\frac2n,\dotsc\dotsc,\frac{n-1}n,1,
\]
把区间~$[0,1]$~分成~$n$~个小区间,相应的把曲边三角形分成~$n$~个细长的小曲边梯形。每一个小曲边梯形,用具有同底和左端点函数值
为高的矩形近似代替,因而得到曲边三角形面积的近似值
\begin{align*}
a\approx x_n&=\sum_{i=0}^{n-1}\Paren[\Big]{\frac in}^2\cdot\frac1n
 =\frac1{n^3}\sum_{i=0}^{n-1}i^2=\frac1{n^3}\frac{n(n-1)(2n-1)}6\\
&=\frac16\Paren[\Big]{1-\frac1n}\Paren[\Big]{2-\frac1n}。\footnotemark
\end{align*}
\footnotetext{利用\begin{align*}
\sum_{k=1}^mk^2
& =\sum_{k=1}^mk(k-1)+\sum_{k=1}^mk
  =\frac13\sum_{k=1}^{m+1}\Parenb{k(k-1)(k-2)-(k-1)(k-2)(k-3)}+\sum_{k=1}^mk\\
& =\frac13(m+1)m(m-1)+\frac12m(m+1)=\frac16m(m+1)(2m+1)。
\end{align*}}%
其中~$x_n$~表示图中阶梯形的面积,$n$~越大阶梯越多,近似程度也就越高,但不管~$n$~多大总是近似的。要想从曲边三角形面
积~$a$~的这一系列近似值~$x_n$~得到准确值~$a$,就必须考察~$n$~趋于无穷的过程。$n$~无限增大时,阶梯形面积~$x_n$~无限
逼近于所求的面积~$a$。令~$\ntoinf$,得到
\[
  x_n=\frac16\Paren[\Big]{1-\frac1n}\Paren[\Big]{2-\frac1n}\to\frac13。
\]
即通过考察~$n$~趋于无穷时,一系列近似值~$x_n$~的变化趋势,确定出曲边三角形的面积~$a=\sfrac13$。

在这个例子中,我们求出了值~$a=\sfrac13$,但我们目的不只是为了求~$a$~的值,也不只是关心如何以直代曲得到近似值~$x_n$~或~$x_n$~的
具体表示形式,而是要从考察~$n$~趋于无穷时,由~$x_n$~的变化趋势确定出值~$a$~这一步中,抽象出序列极限的定义。

\subsection{序列极限定义}

一个接一个排起来的一串数叫作\emph{序列}。例如
\begin{equation}\label{eq:ch1sec2.1-1}
1,2,3,4,\dotsc; \quad
1,\frac12,\frac13,\frac14,\dotsc;\quad
1,\frac12,\frac1{2^2},\frac1{2^3},\dotsc;\quad
1,-1,1,-1,\dotsc
\end{equation}
都是序列。我们只写出序列的前几项,后面用三点来表示未写出无穷多项。一般用
\[
  x_1,x_2,\dotsc,x_n,\dotsc
\]
来表示序列,简单记作~$\Brace{x_n}$,这里~$x_n$~称为序列的\emph{通项}。给定序列的通项公式,就可以写出序列的每一项;反之,给
定一个序列,即给出一个映射~$\map f\FN\FR$,一般来说可以归纳出通项公式。比如~\ref{eq:ch1sec2.1-1}~中第一个序列的通项公式
为~$x_n$;第二个序列为~$x_n=\sfrac1n$;第三个序列为~$x_n=\sfrac1{2^{n-1}}$;第四个序列为~$x_n=(-1)^{n-1}$。但序列
\[
  1.4,1.41,1.414,\dotsc
\]
的第~$n$~项是~$\sqrt2$~取小数点后~$n$~位的近似值。虽然通项的意义是明确的,但不能用明显的公式表示出来。

在求曲边三角形面积的例子中,我们遇到了通项为
\[
  x_n=\frac16\Paren[\Big]{1-\frac1n}\Paren[\Big]{2-\frac1n}
\]
的序列~$\Brace{x_n}$,并看出~$n$~趋于无穷时,通项~$x_n$~趋于~$a=\sfrac13$。在其它问题中,也要考察~$n$~趋于无穷时,序列
变化的趋势。为此我们引出序列极限的定性描述:

给定序列~$\Brace{x_n}$,当~$n$~无限增大时,若~$x_n$~无限的接近~$a$,则称~$a$~为~$n$~趋于无穷时序列的
极限,记作~$x_n\to a\,(\ntoinf)$,或
\[
  \lim_\ntoinf x_n=a。
\]

这种描述显然是很不够的,因为“无限增大”,“无限接近”并未给出确切的含义,只当作一般性用语来理解。要想给出极限的定义,必须
分析一下“无限增大”,“无限接近”的确切含义是什么。

再回到曲边三角形的例子。所谓~$n$~无限增大,$x_n$~无限接近于~$a$,是指阶梯形面积~$x_n$~与曲边三角形面积~$a$~的误差要多小就
能多小,只要~$n$~充分大。具体来说,给定误差~$\sfrac1{100}$,只要去~$n=100$,即把~$[0,1]$~区间~$100$~等分,每一个小曲边梯形
与矩形的误差加起来,就是~$x_n$~与~$a$~的误差。要估计小曲边梯形与矩形误差总和,只需如\ref{fig:sec2.1-1}~所示,将表示误差的
图形平移到以~$1$~为高,以~$\sfrac1{100}$~为底的该图左边的矩形上,显然误差总和不超过矩形面积的~$\sfrac1{100}$,所以
\[
  \abs{x_{100}-a}<\frac1{100}。
\]
若给定误差~$\num{1/1000}$,则只要取~$n=\num{1000}$,就有
\[
  \abs{x_{1000}-a}<\frac1{\num{1000}}。
\]
由此可见,$x_n$~与~$a$~的误差要多小就能多小,只要~$x_n$~充分大。用希腊字母~$\e$~表示误差,我们先试着给出极限的定义如下:

若给定~$\e>0$,总能找到自然数~$n$,使得
\[
  \abs{x_n-a}<\e,
\]
则称序列~$x_n$~的极限是~$a$,或~$a$~是序列的极限。

这个定义是否妥当呢?从字面上看,它只告诉我们序列有一项~$x_n$~与~$a$~的误差小于~$\e$,这项以后的各项与~$a$~的误差是
否小于~$\e$~呢?定义没有说,如果这项以后的项与~$a$~误差很大,则不能反映出~$x_n$~无限接近~$a$,所以定义应修改如下:

如果给定~$\e>0$,存在~$N$,使得当~$n>N$~时,就有
\[
  \abs{x_n-a}<\e,
\]
则称~$x_n$~的极限时~$a$。

这样修改后,保证自某一项以后各项与~$a$~的误差都小于~$\e$,究竟是否妥当了呢?可能想法是对的,但表达还有问题。给定~$\e>0$,只是
说对一个误差~$\e$~能找到~$N$,对别的误差能否找到~$N$~呢?定义反映不出来。如果对于更小的误差~$\e_1$,不能保证序列自某一项以后
各项与~$a$~的误差小于~$\e_1$,也不能反映出~$x_n$~无效接近与~$a$。因此,我们就得到了严格的极限定义。

\begin{definition}\label{def:sec2.1-1}
如果对于任意~$\e>0$,都存在非负整数~$N$,使得当~$n>N$~时,就有
\[
  \abs{x_n-1}<\e,
\]
则称序列~$\Brace{x_n}$~的\emph{极限}为~$a$,记作
\[
  \lim_\ntoinf x_n=a,
\]
或者~$x_n\to a\,(\ntoinf)$。
\end{definition}

极限的几何意义为:作以~$a$~为中心,以~$\e$~为半径的邻域~$(a-\e,a+\e)$,当~$n>N$~时,有
\[
  a-\e<x_n<a+\e,
\]
即随着~$n$~的变化,观察序列~$x_n$~在数轴上的变化时,发现开头有限项可能落在邻域外面,或一会儿落在邻域里边,一会儿又落在
邻域外边,但当标号大于~$N$~后,序列的项总在邻域里变动,再也不会落在邻域外边。从静态来看,以~$a$~为中心,以任意~$\e$~为
半径作一邻域,若序列在邻域外边只有有限项,则序列的极限是~$a$。

\begin{example}
证明~$\lim_\ntoinf\dfrac1n=0$。
\end{example}
\begin{proof}
对于任意~$\e>0$,则
\[
  \abs[\Big]{\frac1n}=\frac1n<\e\impliedby n>\frac1\e。
\]
因此,取~$N=\Floor[\Big]{\dfrac1\e}$,则当~$n>N$~时,就有~$\abs[\Big]{\dfrac1n}<\e$,即
\[
  \lim_\ntoinf\frac1n=0。\qedhere
\]
\end{proof}

这个例子说明~$N$~是依赖于~$\e$,$\e$~越小,$N$~越大。具体找~$N$~时,可以用分析法,即要使结论成立,看~$n$~应多大。

\begin{example}
证明~$\lim_\ntoinf q^n=0\,(0<q<1)$。
\end{example}
\begin{proof}
对于任意~$\e>0$,不妨假定~$\e<1$,则
\[
  \abs{q^n}=q^n<\e\impliedby n\ln q<\ln\e\impliedby n>\frac{\ln\e}{\ln q}。
\]
因此,取~$N=\Floor[\Big]{\dfrac{\ln\e}{\ln q}}$,则当~$n>N$~时,就有~$\abs{q^n}<\e$,即
\[
  \lim_\ntoinf q^n=0。\qedhere
\]
\end{proof}

\begin{example}
证明~$\lim_\ntoinf\dfrac{a^n}{n!}\,(a>1)$。
\end{example}
\begin{proof}
因为
\[
  \frac{a^n}{n!}=\frac a1\cdot\frac a2\dotsm\frac a{\Floor{a}}\cdot\frac a{\Floor a+1}\dotsm \frac an
  <\frac{a^{\Floor a}}{\Floor a!}\cdot\frac an\eqcolon c\cdot \frac an,
\]
所以对于任意~$\e>0$,有
\[
  \abs[\Big]{\frac{a^n}{n!}}=\frac{a^n}{n!}<\e\impliedby \frac{ca}n<\e\impliedby n>\frac{ca}\e。
\]
取~$N=\Floor[\Big]{\dfrac{ca}\e}$,则当~$n>N$~时,就有~$\abs[\Big]{\frac{a^n}{n!}}<\e$,即
\[
  \lim_\ntoinf\dfrac{a^n}{n!},\quad a>1。\qedhere
\]
\end{proof}

在极限定义中,我们关心的不是~$N$~的具体值,而是~$N$~的存在性。所以做题时,不一定直接去解不等式,可以用适当放大法来
找~$N$。例如从不等式
\[
  \frac {a^n}{n!}<\e
\]
出发解~$n$~大于什么很困难,而放大后求出~$N$~就很容易。要注意,运用放大法时,不要把含有变数~$n$~的因子移到不等式的右端,只
允许将左端逐步放大。还应注意反推法书写时,是由后面不等式成立,推出前面不等式成立;而不是前面不等式成立,推后面不等式成立。

\begin{example}
证明~$\lim_\ntoinf\sqrt[n]a=1\,(a>1)$。
\end{example}
\begin{proof*}
对于任意~$\e>0$,有
\begin{align*}
\abs{\sqrt[n]a-1}=\sqrt[n]a-1<\e
&\impliedby\sqrt[n]a<1+\e\impliedby\frac1n\ln a<\ln(1+\e)\\
&\impliedby n>\frac{\ln a}{\ln(1+\e)}。
\end{align*}
因此取~$N=\Floor[\Big]{\dfrac{\ln a}{\ln(1+\e)}}$,则当~$n>N$~时,就有~$\abs{\sqrt[n]a-1}<\e$,即
\[
  \lim_\ntoinf\sqrt[n]a=1,\quad a>1 。\qedhere
\]
\end{proof*}
\begin{proof*}
令~$\sqrt[n]a-1=h_n$,为了对~$h_n$~用适当放大法,我们先估计~$h_n$。因为
\[
  a=(1+h_n)^n=1+nh_n+\frac{n(n-1)}{2!}h_n^2+\dotsb+h_n^n>nh_n,
\]
所以
\[
  0<h_n<\frac an。
\]
因此,对任意~$\e>0$,
\[
  \abs{\sqrt[n]a-1}=h_n<\e\impliedby \frac an<\e。
\]
取~$N=\Floor[\Big]{\dfrac an}$,则当~$n>N$~时,就有~$\abs{\sqrt[n]a-1}<\e$。
\end{proof*}

\begin{exercise}
\item 用~$\e-N$~方法验证
\[
  \lim_\ntoinf\frac1{n^2-1}=0,
\]
并分别对~$\e=0.1,0.01$~确定相应的~$N$。
\item 用~$\e-N$~方法验证下列极限为零。
\begin{exlistcols}[3]
\item $\lim_\ntoinf\dfrac1{n^2+n}$;
\item $\lim_\ntoinf\dfrac1{n^4-n}$;
\item $\lim_\ntoinf\dfrac{\sqrt[3]{n^2}}{n-3}$;
\item $\lim_\ntoinf\Parenb{\sqrt{n+1}-\sqrt n}$;
\item $\lim_\ntoinf\dfrac{10^n}{n!}$;
\item $\lim_\ntoinf\dfrac{n!}{n^n}$;
\item $\lim_\ntoinf\dfrac n{a^n}\,(a>1)$;
\item $\lim_\ntoinf\Parenb{\ln(n+1)-\ln n}$。
\end{exlistcols}
\item 若~$\lim_\ntoinf x_n=a$,证明~$\lim_\ntoinf\abs{x_n}=\abs a$。
\item 设~$x_n>0\,(n\geq 1)$,且~$\lim_\ntoinf x_n=a$。证明~$\lim_\ntoinf\sqrt{x_n}=\sqrt a$。
\item 设~$\lim_\ntoinf x_n=a$,证明~$\lim_\ntoinf\sqrt[3]{x_n}=\sqrt[3]a$。
\item 设~$x_n\leq a\leq y_n\,(n\geq 1)$,且~$\lim_\ntoinf(y_n-x_n)=0$。证明~$\lim_\ntoinf x_n=\lim_\ntoinf y_n=a$。
\item 设~$n,p$~均为正整数。证明,
\begin{exlistcols}[2]
\item $n^p<\dfrac{(n+1)^{p+1}-n^p}{p+1}<(n+1)^p$;
\item $\sum_{k=1}^{n-1}k^p<\dfrac{n^{p+1}}{p+1}<\sum_{k=1}^nk^p$;
\item 求极限~$\lim_\ntoinf\dfrac1n\sum_{k=1}^n\Paren[\Big]{\dfrac kn}^p$。
\end{exlistcols}
\item 设~$\lim_\ntoinf x_n=a$,而~$\ell$~为确定的自然数。证明,$\lim_\ntoinf x_{n+\ell}=a$。试讨论相反的情况。
\item\begin{exlist}
  \item 证明~$\lim_\ntoinf x_n=a$~当且仅当~$\lim_\ntoinf x_{2n}=\lim_\ntoinf x_{2n+1}=a$;
  \item 已知~$\lim_\ntoinf x_{2n}$~和$\lim_\ntoinf x_{2n+1}$~都存在,试讨论~$\lim_\ntoinf x_n$~的存在性。
\end{exlist}
\item 在序列极限定义中,对于~$N$~请说明下列问题。
\begin{exlistcols}[2]
  \item $N$~是否唯一;
  \item $N$~是否是~$\e$~的函数;
  \item 前~$N$~项是否有~$\abs{x_n-a}\geq\e$。
\end{exlistcols}
\item 判断并说明序列极限定义改成下面形式是否可以。
\begin{exlist}
  \item 对任意~$\e>0$,存在~$N>0$,使得当~$n\geq N$~时,有~$\abs{x_n-a}<\e$;
  \item 对任意~$\e>0$,存在~$N>0$,使得当~$n> N$~时,有~$\abs{x_n-a}\leq\e$;
  \item 对任意~$\e>0$,存在~$N>0$,使得当~$n> N$~时,有~$\abs{x_n-a}<M\e$,这里~$M$~为某固定常数。
\end{exlist}
\item 若对任意~$N>0$,存在~$\e>0$,使得当~$n>N$~时,有~$\abs{x_n-a}<\e$,则序列~$\Brace{x_n}$~具有什么性质?
\item 若存在~$N>0$,对任意~$\e>0$,使得当~$n>N$~时,有~$\abs{x_n-a}<\e$,则序列~$\Brace{x_n}$~具有什么性质?
\item 判断并说明下述对~$\lim_\ntoinf\sqrt[n]n=1$~的证明的正确性。
\begin{exproof}
对任意~$\e>0$,有
\[
  \sqrt[n]n<1+\e\impliedby\frac1n\ln n<\ln(1+\e)\impliedby
  \frac1n<\frac{\ln(1+\e)}{\ln n}\leq\frac{\ln(1+\e)}{\ln 2}。
\]
取~$N=\Floor[\Big]{\dfrac{\ln 2}{\ln(1+\e)}}$,则当~$n>N$~时,有
\[
  1-\e<1<\sqrt[n]n<1+\e,
\]
由此即可得到~$\lim_\ntoinf\sqrt[n]n=1$。
\end{exproof}
\end{exercise}


\section{序列极限的性质与运算}

有了极限定义后,自然会问极限有什么性质和怎么求极限。为此,我们要讨论极限的唯一性、有界性,极限的四则运算和
极限不等式。

\begin{theorem}[唯一性]\label{thm:sec2.2-1}
若序列~$\Brace{x_n}$~的极限存在,则极限值是唯一的。
\end{theorem}

这个定理保证了不管用什么方法求极限,得到的极限值应该是一样的。

\begin{proof}
我们用反证法。假设序列极限不唯一,至少有两个不相等的极限值,设为
\[
  \lim_\ntoinf x_n=a,\quad \lim_\ntoinf x_n=b,
\]
且~$a\neq b$。不妨设~$a<b$,取~$\e=\dfrac{b-a}2>0$,由极限的定义,存在~$N_1$,使得当~$n>N_1$~时,有
\[
  \abs{x_n-a}<\e\implies x_n<a+\e=\frac{a+b}2。
\]
并且存在~$N_2$,使得当~$n>N_2$~时,有
\[
  \abs{x_n-b}<\e\implies x_n>a-\e=\frac{a+b}2。
\]
所以当~$n>\max\Brace{N_1,N_2}$~时,就有
\[
  x_n>\frac{a+b}2>x_n。
\]
显然,这是一个矛盾(见\ref{fig:sec2.1-2})。因此极限值确实是唯一的。
\end{proof}

\begin{wrapfigure}{O}{0mm}
\somefigure
\caption{}\label{fig:sec2.1-2}
\end{wrapfigure}

从函数观点来看,序列就是定义在自然数集合上的函数~$x_n=f(n)$,由函数有界定义,可以得到序列有界的定义。

若存在~$M>0$,使得对任意~$n$,都有~$\abs{x_n}\leq M$,则称序列~$\Brace{x_n}$~\emph{有界}。

\begin{theorem}[有界性]\label{thm:sec2.2-2}
若序列~$\Brace{x_n}$~有极限,则~$\Brace{x_n}$~有界。
\end{theorem}
\begin{proof}
设~$\lim_\ntoinf x_n=a$,取~$\e_0=1$,则存在~$N$,使得当~$n>N$~时,有~$\abs{x_n-a}<1$,那么
\[
  \abs{x_n}-\abs a\leq\abs{x_n-a}<1\implies\abs{x_n}<\abs a+1。
\]
令
\[
  M=\max\Braceb{1+\abs a,\abs{x_1},\abs{x_2},\dotsc,\abs{x_N}},
\]
则对任意~$n$,均有
\[
  \abs{x_n}\leq M,
\]
即序列~$\Brace{x_n}$~确实是有界的。
\end{proof}

在证明时必须分清什么时候取定~$\e$,什么时候任意给定~$\e$。上面证明中就必须取定~$\e$,不能任意给定~$\e$。如果
改为任意给定~$\e>0$,则~$N$~随~$\e$~在变,找到的~$M$~也随~$\e$~在变,这时的界~$M$~的意义就不明确,有界性就难以保证。

\begin{theorem}[极限的四则运算]\label{thm:sec2.2-3}
设~$\lim_\ntoinf x_n=a$,$\lim_\ntoinf y_n=b$,则
\begin{enumlistcols}
  \item $\lim_\ntoinf(x_n\pm y_n)=a\pm b$;\label{thm:sec2.2-3-item-1}
  \item $\lim_\ntoinf x_n\cdot y_n=a\cdot b$;\label{thm:sec2.2-3-item-2}
  \item 若~$n\neq0$,且~$y_n\neq0$,则~$\lim_\ntoinf\dfrac{x_n}{y_n}=\dfrac ab$。\label{thm:sec2.2-3-item-3}
\end{enumlistcols}
\end{theorem}
\begin{proof}\fixthmitem\begin{enumlist}
\item 任给~$\e>0$,由~$\lim_\ntoinf x_n=a$,存在~$N_1$,使得当~$n>N_1$~时,有
\[
  \abs{x_n-a}<\frac\e2;
\]
又由~$\lim_\ntoinf y_n=b$,存在~$N_2$,使得当~$n>N_2$~时,有
\[
  \abs{y_n-b}<\frac\e2。
\]
取~$N=\max\Brace{N_1,N_2}$,则当~$n>N$~时,就有
\[
  \absb{(x_n\pm y_n)-(a\pm b)}\leq\abs{x_n-a}+\abs{y_n-b}<\frac\e2+\frac\e2=\e,
\]
即
\[
  \lim_\ntoinf(x_n\pm y_n)=a\pm b。
\]
\item 要使得~$n>N$~时,$\abs{x_n\cdot y_n-ab}<\e$,常用的方法是插入一项,这项的两个因子,一个因子与第一项的
因子相同,另一个因子与第二项的因子相同。这样的话,
\[
  \abs{x_n\cdot y_n-ab}=\abs{x_n\cdot y_n-y_n\cdot a+y_n\cdot a-ab}
  \leq\abs{y_n}\cdot\abs{x_n-a}+\abs a\cdot\abs{y_n-b},
\]
所以要使结论成立,只要~$n>N$~时,利用~$\Brace{y_n}$~有界证明
\[
  \abs{y_n}\cdot\abs{x_n-a}<\frac\e2,\quad \abs a\cdot\abs{y_n-b}<\frac\e2
\]
即可。这样我们就得到了下面的证明。

由\ref{thm:sec2.2-2},存在~$M_1>0$,使得对任意~$n$,均有~$\abs{y_n}\leq M_1$,令
\[
  M=\max\Brace{M_1,\abs a}>0。
\]

对任意~$\e>0$,由~$\lim_\ntoinf x_n=a$,存在~$N_1>0$,使得当~$n>N_1$~时,有
\[
  \abs{x_n-a}<\frac\e{2M};
\]
又由~$\lim_\ntoinf y_n=b$,存在~$N_2>0$~,使得当~$n>N_2$~时,有
\[
  \abs{y_n-b}<\frac\e{2M}。
\]
取~$N=\max\Brace{N_1,N_2}$,则当~$n>N$~时,就有
\begin{align*}
\abs{x_n\cdot y_n-ab}
&\leq\abs{y_n}\cdot\abs{x_n-a}+\abs a\cdot\abs{y_n-b}\leq M\abs{x_n-a}+M\abs{y_n-b}\\
&<M\cdot\frac\e{2M}+M\cdot\frac\e{2M}=\e,
\end{align*}
即
\[
  \lim_\ntoinf x_n\cdot y_n=a\cdot b。
\]
\item 由~\ref{thm:sec2.2-3-item-2}~可知,只要证明~$\lim_\ntoinf\dfrac1{y_n}=\lim_\ntoinf\dfrac1b$~即可。要使~$n>N$~时
\[
  \abs[\Big]{\frac1{y_n}-\frac1b}=\abs[\Big]{\frac{y_n-b}{y_n\cdot b}}<\e,
\]
就要先估计~$\Brace{y_n}$~的下界。为此,对于~$\dfrac{\abs b}2>0$,由~$\lim_\ntoinf y_n=b$,存在~$N_1$,使得当~$n>N_1$~时,有
\[
  \abs b-\abs{y_n}\leq\abs{y_n-b}<\frac{\abs b}2\implies\frac{\abs b}2\leq\abs{y_n}。
\]

对任意~$\e>0$,仍由~$\lim_\ntoinf y_n=b$,存在~$N_2$,使得当~$n>N_2$~时,有
\[
  \abs{y_n-b}<\frac{\abs b^2}2\e。
\]
取~$N=\max\Brace{N_1,N_2}$,则当~$n>N$~时,就有
\[
  \abs[\Big]{\frac 1{y_n}-\frac 1b}=\abs[\Big]{\frac{y_n-b}{y_n\cdot b}}\leq\frac{2\abs{y_n-b}}{\abs b^2}
  <\frac2{\abs b^2}\cdot\frac{\abs b^2}2\e=\e,
\]
即
\[
  \lim_\ntoinf\frac1{y_n}=\frac1b。
\]
再由~~\ref{thm:sec2.2-3-item-2},
\[
  \lim_\ntoinf\frac{x_n}{y_n}=\lim_\ntoinf x_n\cdot\frac1{y_n}=a\cdot\frac1b=\frac ab。\qedhere
\]
\end{enumlist}
\end{proof}

\ref{thm:sec2.2-3}~的结论可以推广至有限个序列的情形。例如如果每个序列极限都存在,则有
\begin{Align*}
  \lim_\ntoinf\Parenb{x_n+y_n+z_n}&=\lim_\ntoinf x_n+\lim_\ntoinf y_n+\lim_\ntoinf z_n,\\
  \lim_\ntoinf\Parenb{x_n\cdot y_n\cdot z_n}&=\lim_\ntoinf x_n\cdot\lim_\ntoinf y_n\cdot\lim_\ntoinf z_n。
\end{Align*}

\begin{example}
求极限
\[
  \lim_\ntoinf\frac{4n^2-6n+1}{3n^2+n+7}。
\]
\end{example}
\begin{solution}
\[
\lim_\ntoinf\frac{4n^2-6n+1}{3n^2+n+7}=\lim_\ntoinf\frac{4-\dfrac6n+\dfrac1{n^2}}{3+\dfrac1n+\dfrac7{n^2}}
=\frac{\dps\lim_\ntoinf\Paren[\Big]{4-\frac6n+\frac1{n^2}}}{\dps\lim_\ntoinf\Paren[\Big]{3+\dfrac1n+\dfrac7{n^2}}}
=\dfrac43。\qedhere
\]
\end{solution}

\begin{example}
求极限~$\lim_\ntoinf\Parenb{1+a+\dotsb+a^n}\,(0<a<1)$。
\end{example}
\begin{solution}
\begin{align*}
\lim_\ntoinf\Parenb{1+a+\dotsb+a^n}
&=\lim_\ntoinf\frac{1-a^{n+1}}{1-a}=\lim_\ntoinf\frac1{1-a}-\lim_\ntoinf\frac{a^{n+1}}{1-a}\\
&=\frac1{1-a}。\qedhere
\end{align*}
\end{solution}

\begin{quiz*}\fixthmitem
\begin{enumlist}
\item 判断并说明下面推导是否正确。
\begin{align*}
\lim_\ntoinf\Paren[\Big]{1+\frac1n}^n
&=\lim_\ntoinf\Paren[\Big]{1+\frac1n}\cdot\lim_\ntoinf\Paren[\Big]{1+\frac1n}\cdot\dotsm\cdot
  \lim_\ntoinf\Paren[\Big]{1+\frac1n}\\
&=1\times1\times\dotsm\times1=1;
\end{align*}
\item 在证明\ref{thm:sec2.2-3}~的结论~\ref{thm:sec2.2-3-item-3}~时,令~$\lim_\ntoinf\dfrac{x_n}{y_n}=c$,则
\[
  a=\lim_\ntoinf x_n=\lim_\ntoinf\frac{x_n}{y_n}\cdot y_n=c\cdot b\implies c=\frac ab,
\]
判断并说明上述推导的正确性。
\end{enumlist}
\end{quiz*}

\begin{theorem}\label{thm:sec2.2-4}
给定序列~$\Brace{x_n}$~和~$\Brace{y_n}$,若~$x_n\leq y_n\,(n\geq1)$,且
\[
  \lim_\ntoinf x_n=a,\quad \lim_\ntoinf y_n=b,
\]
则~$a\leq b$。
\end{theorem}
\begin{proof}
用反证法。不妨假定~$a>b$,取
\[
  \e=\frac{a-b}2>0,
\]
由~$\lim_\ntoinf x_n=a$,存在~$N_1$,使得当~$n>N_1$~时,
\[
  \abs{x_n-a}<\e\implies x_n>a-\e=\frac{a+b}2;
\]
又由~$\lim_\ntoinf y_n=b$,存在~$N_2$,使得当~$n>N_2$~时,有
\[
  \abs{y_n-b}<\e\implies y_n<b+\e=\frac{a+b}2。
\]
因此,当~$n>\max\Brace{N_1,N_2}$~时,得到
\[
  x_n>\frac{a+b}2>y_n。
\]
这与条件矛盾(见\ref{fig:sec2.1-3})。所以~$a\leq b$。
\end{proof}

\thisfloatsetup{rowprecode=wrapfig}
\begin{wrapfigure}[10]{O}{0mm}
\somefigure
\caption{}\label{fig:sec2.1-3}
\end{wrapfigure}

下面我们来讨论极限的所谓夹逼原理。

\begin{theorem}\label{thm:sec2.2-5}
给定序列~$\Brace{x_n},\Brace{y_n}$~和~$\Brace{z_n}$,满足
\[
x_n\leq z_n\leq y_n,\quad n\geq1,
\]
且
\[
  \lim_\ntoinf x_n=a=\lim_\ntoinf y_n,
\]
则~$\lim_\ntoinf z_n=a$。
\end{theorem}
\begin{proof}
对任意~$\e>0$,由~$\lim_\ntoinf x_n=a$,存在~$N_1$,使得当~$n>N_1$~时,有
\[
  \abs{x_n-a}<\e\implies a-\e<x_n<a+\e;
\]
又由~$\lim_\ntoinf y_n=a$,存在~$N_2$,使得当~$n>N_2$~时,有
\[
  \abs{y_n-a}<\e\implies a-\e<y_n<a+\e 。
\]
取~$N=\max\Brace{N_1,N_2}$,则当~$n>N$~时,就有
\[
  a-\e<x_n\leq z_n\leq y_n<a+\e\implies\abs{z_n-a}<\e,
\]
即~$\lim_\ntoinf z_n=a$。
\end{proof}

\begin{example}
设~$a,b>0$,证明
\[
  \lim_\ntoinf\sqrt[n]{a^n+b^n}=\max\Brace{a,b}。
\]
\end{example}
\begin{proof}
因为
\[
  \Paren[\Big]{\Parenb{\max\Brace{a,b}}^n}^{\frac1n}\leq\Parenb{a^n+b^n}^{\frac1n}\leq
  \Paren[\Big]{2\Parenb{\max\Brace{a,b}}^n}^{\frac1n},
\]
即
\[
  \max\Brace{a,b}\leq\sqrt[n]{a^n+b^n}\leq\max\Brace{a,b}\cdot\sqrt[n]2。
\]
由~$\lim_\ntoinf\sqrt[n]2=1$~及\ref{thm:sec2.2-5}~即可得到
\[
  \lim_\ntoinf\sqrt[n]{a^n+b^n}=\max\Brace{a,b}。\qedhere
\]
\end{proof}

\begin{quiz*}\fixthmitem\begin{enumlist}
\item 在\ref{thm:sec2.2-4}~中若将条件改为~$x_n<y_n$,结论能否改为~$a<b$?
\item 考虑能否利用\ref{thm:sec2.2-4}~直接得到\ref{thm:sec2.2-5},即
\[
  a=\lim_\ntoinf x_n\leq\lim_\ntoinf z_n\leq\lim_\ntoinf y_n=a。
\]
\end{enumlist}
\end{quiz*}

作为序列极限的特殊情形,我们引入无穷小量的概念及其运算。

\begin{definition}
  极限为零的变量称为\emph{无穷小量}。
\end{definition}

若变量以序列形式表示,若序列极限为零,则称序列就叫作无穷小量。例如
\[
  \Brace[\Big]{\frac1n},\quad \Brace{q^n}\,(0<q<1),\quad
  \Brace[\Big]{\frac{a^n}{n!}}\,(a>0)
\]
都是无穷小量。所以无穷小量不是很小的量,而是极限值为零的变量。

\begin{corollary}\label{cor:sec2.2-1}
\fixthmitem\begin{enumlist}
  \item 无穷小量的绝对值是无穷小量;\label{cor:sec2.2-1-item-1}
  \item 无穷小量与有界变量的乘积是无穷小量。\label{cor:sec2.2-1-item-2}
\end{enumlist}
\end{corollary}
\begin{proof}\fixthmitem\begin{enumlist}
\item 设~$\lim_\ntoinf x_n=0$,即对任意~$\e>0$,存在~$N$,使得当~$n>N$~时,有
\[
  \abs{x_n-0}=\abs{x_n}<\e\implies\absb{\abs{x_n}-0}=\abs{x_n}<\e,
\]
故~$\lim_\ntoinf\abs{x_n}=0$。
\item 设~$\lim_\ntoinf x_n=0$,而~$\abs{y_n}\leq M$,则
\[
  \abs{x_ny_n}\leq M\abs{x_n}\implies
  -M\abs{x_n}\leq x_n\cdot y_n\leq M\abs{x_n}。
\]
由~\ref{cor:sec2.2-1-item-1}~可知~$\lim_\ntoinf M\abs{x_n}=\lim_\ntoinf(-M\abs{x_n})=0$,利用\ref{thm:sec2.2-5}~可得
\[
  \lim_\ntoinf x_n\cdot y_n=0。\qedhere
\]
\end{enumlist}
\end{proof}

\begin{corollary}\label{cor:sec2.2-2}
变量有极限~$a$~当且仅当变量可以分解成~$a$~与无穷小量之和。
\end{corollary}
\begin{proof}
设~$\lim_\ntoinf x_n=0$,则
\[
  \lim_\ntoinf(x_n-a)=0。
\]
令~$\alpha_n=x_n-a$,由定义可知~$\Brace{\alpha_n}$~是无穷小量,并有~$x_n=a+\alpha_n$。

反之,若~$x_n=a+\alpha_n$,且~$\lim_\ntoinf\alpha_n=0$,则
\[
  \lim_\ntoinf x_n=\lim_\ntoinf(a+\alpha_n)=a。\qedhere
\]
\end{proof}

\begin{example}
证明~$\lim_\ntoinf\sqrt[n]n=1$。
\end{example}
\begin{proof}
令~$\sqrt[n]n=1+h_n$,只要证明~$h_n$~是无穷小量。事实上,
\[
  n=(1+h_n)^n=1+nh_n+\frac{n(n-1)}2h_n^2+\dotsb+h_n^n\geq\frac{n(n-1)}2h_n^2,
\]
所以
\[
  0<h_n<\frac2{\sqrt{n-1}},\quad n>1,
\]
由~$\lim_\ntoinf\dfrac2{\sqrt{n-1}}=0$~以及\ref{thm:sec2.2-5},可得~$\lim_\ntoinf h_n=0$,再
由\ref{cor:sec2.2-1}~就可以得到
\[
\lim_\ntoinf\sqrt[n]n=1。\qedhere
\]
\end{proof}

\begin{exercise}
\item 若~$\lim_\ntoinf(y_n-x_n)=0$,且~$\lim_\ntoinf x_n=a$。证明~$\lim_\ntoinf y_n=a$。判断并说明下述证法的正确性。
\[
  0=\lim_\ntoinf(y_n-x_n)=\lim_\ntoinf y_n-\lim_\ntoinf x_n=\lim_\ntoinf y_n-a\implies\lim_\ntoinf y_n=a。
\]
\item 求下列极限。
\begin{exlistcols}
  \item $\lim_\ntoinf\Paren[\Big]{\dfrac1{n^2}+\dfrac2{n^2}+\dotsb+\dfrac n{n^2}}$;
  \item $\lim_\ntoinf\dfrac{a^n}{1+a+\dotsb+a^{n-1}}\;(a>0)$;
  \item $\lim_\ntoinf\Paren[\bigg]{\dfrac1{1\cdot2}+\dfrac1{2\cdot3}+\dotsb+\dfrac1{n(n+1)}}$;
  \item $\lim_\ntoinf\Parenb{\sqrt[3]{n+1}-\sqrt[3]n}$;
  \item $\lim_\ntoinf\sqrt[n]a\;(0<a<1)$;
  \item $\lim_\ntoinf\dfrac{n^k}{a^n}\;(a>1,k>0)$。
\end{exlistcols}
\item 设~$\lim_\ntoinf x_n=a$,$\lim_\ntoinf y_n=b$。证明,
\begin{exlistcols}
  \item $\lim_\ntoinf\max\Brace{x_n,y_n}=\max\Brace{a,b}$;
  \item $\lim_\ntoinf\min\Brace{x_n,y_n}=\min\Brace{a,b}$。
\end{exlistcols}
\item 证明序列~$\Brace{\cos n}$~的极限不存在。
\item 证明序列~$\Brace{\tan n}$~与~$\Brace{\sfrac1{\tan n}}$~的极限不存在。
\item 求下列极限。
\begin{exlistcols}
  \item $\lim_\ntoinf\dfrac{\sin n}{\sqrt[3]{1-n^2}}$;
  \item $\lim_\ntoinf\sqrt[n]{n\cdot\ln n}$;
  \item $\lim_\ntoinf\Paren[\bigg]{\dfrac{1\cdot3\cdot5\cdot\dotsm\cdot(2n-1)}{2\cdot4\cdot6\cdot\dotsm\cdot(2n)}}^{\frac1n}$;
  \item $\lim_\ntoinf\dfrac{1\cdot3\cdot5\cdot\dotsm\cdot(2n-1)}{2\cdot4\cdot6\cdot\dotsm\cdot(2n)}$;
  \item $\lim_\ntoinf\dfrac1{\sqrt[n]{n!}}$。
\end{exlistcols}
\item 若~$\abs{x_{n+1}}\leq k\abs{x_n}\,(0<k<1)$~对任意~$n$~成立。证明~$\lim_\ntoinf x_n=0$。
\item 设~$f(x)=\frac{x+2}{x+1}$,而~$x_0=1$,$x_{n+1}=f(x_n)\,(n\geq0)$。证明~$\lim_\ntoinf x_n=\sqrt2$。
\item 设~$x_1=1$,而~$x_{n+1}=\sqrt{2+x_n}\;(n\geq1)$。求~$\lim_\ntoinf x_n$。
\item 设~$x_0=1$,而~$x_{n+1}=1+\dfrac1{x_n}\,(n\geq0)$。求~$\lim_\ntoinf x_n$。
\item\begin{exlist}
  \item 设~$b>1$,$\lim_\ntoinf x_n=0$。证明,对任意~$\e$,$0<\e<1$,存在~$N$,使得当~$n>N$~时,有
  \[
    \log_b(1-\e)<x_n<\log_b(1+\e);
  \]
  \item 设~$b>1$,$\lim_\ntoinf x_n=0$,证明~$\lim_\ntoinf b^{x_n}=1$;
  \item 设~$0<b<1$,$\lim_\ntoinf x_n=0$,证明~$\lim_\ntoinf b^{x_n}=1$;
  \item 设~$b>0$,$\lim_\ntoinf x_n=a$,证明~$\lim_\ntoinf b^{x_n}=b^a$。
\end{exlist}
\item\begin{exlist}
  \item 设~$\lim_\ntoinf y_n=1$。证明,对任意~$\e$,$0<\e<1$,存在~$N$,使得当~$n>N$~时,有~$\me^{-\e}<y_n<\me^\e$;
  \item 设~$\lim_\ntoinf y_n=1$,证明~$\lim_\ntoinf\ln y_n=0$;
  \item 设~$\lim_\ntoinf y_n=b\,(b>0)$,证明~$\lim_\ntoinf\ln y_n=b$;
  \item 设~$\lim_\ntoinf y_n=b\,(b>0)$,$\lim_\ntoinf x_n=a$,证明~$\lim_\ntoinf y_n^{x_n}=b^a$。
\end{exlist}
\item\begin{exlist}
  \item 设~$x_n>0$,且~$\lim_\ntoinf\sqrt[n]{x_n}=q<1$,证明~$\lim_\ntoinf x_n=0$;
  \item 设~$x_n>0$,且~$\lim_\ntoinf\sqrt[n]{x_n}=\ell>1$,证明~$\lim_\ntoinf\dfrac1{x_n}=0$。
\end{exlist}
\item 设~$\Brace{x_n}$~为无穷小量。判断并说明,
\begin{exlistcols}
  \item $\Brace{x_n^n}$~是否为无穷小量;
  \item $\Brace{\!\sqrt[n]{x_n}}\,(x_n>0)$~是否为无穷小量。
\end{exlistcols}
\end{exercise}


\section{确界与单调有界序列}

在这一节中,我们将讨论极限的存在性。给定序列~$\Brace{x_n}$,怎么判定它有没有极限呢?用极限定义判定,就必须先要看出极限值,这
对稍微复杂的序列是办不到的,所以只能通过序列本身来判断它有没有极限。判定一般序列是否有极限,我们以后再讨论。对单调序列有简单
的判别准则。这个准则要用到所谓确界的概念,为此,我们先讨论集合的确界。

设有实数集合~$E\subset\FR$,若存在实数~$M$,使得对任意~$x\in E$,均有
\[
  x\leq M,
\]
则称~$M$~是集合~$E$~的一个\emph{上界}。集合~$E$~有一个上界,显然就有无穷个上界,那么这些上界中有没有最小上界呢?

若集合~$E$~由无限个元素组成,那么它的最大元素就是集合的最小上界。若~$E$~是无限集,这时~$E$~可以没有最大集合,那么它的最小上界
是什么意思呢?它有没有最小上界呢?首先我们给出最小上界的定义。

\begin{definition}
给定实数集合~$E$,若实数~$M$~满足
\begin{enumlist}
\item $M$~是集合~$E$~的上界;
\item 若~$M'$~是集合~$E$~的一个上界,则必有~$M\leq M'$,
\end{enumlist}
则称~$M$~是集合~$E$~的\emph{上确界}或\emph{最小上界},记作
\[
  M=\sup E\quad\text{或}\quad M=\sup_{x\in E}x。
\]
\end{definition}



\begin{exercise}
\item
\end{exercise}
\section{函数的极限}
\begin{exercise}
\item
\end{exercise}
\section{函数极限的推广}
\subsection{自变量趋于无穷的情形}
\subsection{无穷大量}
\subsection{单侧极限}
\subsection{极限存在性}
\subsection{复合函数求极限}
\begin{exercise}
\item
\end{exercise}
\section{两个重要极限}
\begin{exercise}
\item
\end{exercise}
\section{无穷小量的阶以及无穷大量的阶的比较}
\begin{exercise}
\item
\end{exercise}
\section{用肯定语气叙述极限不是某常数}
\subsection{极限不是某常数的肯定描述}
\subsection{序列极限与函数极限的关系}
\begin{exercise}
\item
\end{exercise}
\begin{exercise*}
\item
\end{exercise*}



\endinput
%%
%% End of file `MAChapter2.tex'.