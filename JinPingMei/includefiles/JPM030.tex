%# -*- coding:utf-8 -*-
%%%%%%%%%%%%%%%%%%%%%%%%%%%%%%%%%%%%%%%%%%%%%%%%%%%%%%%%%%%%%%%%%%%%%%%%%%%%%%%%%%%%%


\chapter{蔡太师擅恩锡爵\KG 西门庆生子加官}


词曰:

\[
十千日日索花奴,白马骄驼冯子都。今年新拜执金吾。侵幕露桃初结子,妒花娇鸟忽嗛雏。闺中姊妹半愁娱。
\]

话说西门庆与潘金莲两个洗毕澡,就睡在房中。春梅坐在穿廊下一张凉椅儿上纳鞋,只见琴童儿在角门首探头舒脑的观看。春梅问道:“你有甚话说?”那琴童见秋菊顶着石头跪在院内,只顾用手往来指。春梅骂道:“怪囚根子!有甚话,说就是了,指手画脚怎的?”那琴童笑了半日,方才说:“看坟的张安,在外边等爹说话哩。”春梅道:“贼囚根子!张安就是了,何必大惊小怪,见鬼也似!悄悄儿的,爹和娘睡着了。惊醒他,你就是死。你且叫张安在外边等等儿。”琴童儿走出来外边,约等勾半日,又走来角门首踅探,问道:“爹起来了不曾?”春梅道:“怪囚!失张冒势,唬我一跳,有要没紧,两头游魂哩!”琴童道:“张安等爹说了话,还要赶出门去,怕天晚了。”春梅道:“爹娘正睡的甜甜儿的,谁敢搅扰他,你教张安且等着去,十分晚了,教他明日去罢。”

正说着,不想西门庆在房里听见,便叫春梅进房,问谁说话。春梅道:“琴童说坟上张安儿在外边,见爹说话哩。”西门庆道:“拿衣我穿,等我起去。”春梅一面打发西门庆穿衣裳,金莲便问:“张安来说甚么话?”西门庆道:“张安前日来说,咱家坟隔壁赵寡妇家庄子儿连地要卖,价银三百两。我只还他二百五十两银子,教张安和他讲去。里面一眼井,四个井圈打水。若买成这庄子,展开合为一处,里面盖三间卷棚,三间厅房,叠山子花园、井亭、射箭厅、打毬场,耍子去处,破使几两银子收拾也罢。”妇人道:“也罢,咱买了罢。明日你娘每上坟,到那里好游玩耍子。”说毕,西门庆往前边和张安说话去了。

金莲起来,向镜台前重匀粉脸,再整云鬟。出来院内要打秋菊。那春梅旋去外边叫了琴童儿来吊板子。金莲问道:“叫你拿酒,你怎的拿冷酒与爹吃?原来你家没大了,说着,你还钉嘴铁舌儿的!”喝声:“叫琴童儿与我老实打与这奴才二十板子!”那琴童才打到十板子上,多亏了李瓶儿笑嘻嘻走过来劝住了,饶了他十板。金莲教与李瓶儿磕了头,放他起来,厨下去了。李瓶儿道:“老潘领了个十五岁的丫头,后边二姐姐买了房里使唤,要七两五钱银子。请你过去瞧瞧。”金莲遂与李瓶儿一同后边去了。李娇儿果问西门庆用七两银子买了,改名夏花儿,房中使唤,不在话下。

单表来保同吴主管押送生辰担,正值炎蒸天气,路上十分难行,免不得饥餐渴饮。有日到了东京万寿门外,寻客店安下。到次日,赍台驮箱礼物,迳到天汉桥蔡太师府门前伺候。来保教吴主管押着礼物,他穿上青衣,迳向守门官吏唱了个喏。那守门官吏问道:“你是那里来的?”来保道:“我是山东清河县西门员外家人,来与老爷进献生辰礼物。”官吏骂道:“贼少死野囚军!你那里便兴你东门员外、西门员外?俺老爷当今一人之下,万人之上,不论三台八位,不论公子王孙,谁敢在老爷府前这等称呼?趁早靠后!”内中有认的来保的,便安抚来保说道:“此是新参的守门官吏,才不多几日,他不认的你,休怪。你要禀见老爷,等我请出翟大叔来。”这来保便向袖中取出一包银子,重一两,递与那人。那人道:“我到不消。你再添一分,与那两个官吏,休和他一般见识。”来保连忙拿出三包银子来,每人一两,都打发了。那官吏才有些笑容儿,说道:“你既是清河县来的,且略等候,等我领你先见翟管家。老爷才从上清宝霄宫进了香回来,书房内睡。”良久,请将翟管家出来,穿着凉鞋净袜,青丝绢道袍。来保见了,忙磕下头去。翟管家答礼相还,说道:“前者累你。你来与老爷进生辰担礼来了?”来保先递上一封揭帖,脚下人捧着一对南京尺头,三十两白金,说道:“家主西门庆,多上覆翟爹,无物表情,这些薄礼,与翟爹赏人。前者盐客王四之事,多蒙翟爹费心。”翟谦道:“此礼我不当受。罢,罢,我且收下。”来保又递上太师寿礼帖儿,看了,还付与来保,分咐把礼抬进来,到二门里首伺候。原来二门西首有三间倒座,来往杂人都在那里待茶。须臾,一个小童拿了两盏茶来,与来保、吴主管吃了。

少顷,太师出厅。翟谦先禀知太师,然后令来保、吴主管进见,跪于阶下。翟谦先把寿礼揭帖呈递与太师观看,来保、吴主管各抬献礼物。但见:

\[
黄烘烘金壶玉盏,白晃晃减靸仙人。锦绣蟒衣,五彩夺目;南京纻缎,金碧交辉。汤羊美酒,尽贴封皮;异果时新,高堆盘盒。
\]
如何不喜,便道:“这礼物决不好受的,你还将回去。”慌的来保等在下叩头,说道:“小的主人西门庆,没甚孝意,些小微物,进献老爷赏人。”太师道:“既是如此,令左右收了。”旁边祗应人等,把礼物尽行收下去。太师又道:“前日那沧州客人王四等之事,我已差人下书,与你巡抚侯爷说了。可见了分上不曾?”来保道:“蒙老爷天恩,书到,众盐客就都放出来了。”太师又向来保说道:“累次承你主人费心,无物可伸,如何是好?你主人身上可有甚官役?”来保道:“小人的主人一介乡民,有何官役?”太师道:“既无官役,昨日朝廷钦赐了我几张空名告身扎付,我安你主人在你那山东提刑所,做个理刑副千户,顶补千户贺金的员缺,好不好?”来保慌的叩头谢道:“蒙老爷莫大之恩,小的家主举家粉首碎身,莫能报答!”于是唤堂候官抬书案过来,即时签押了一道空名告身扎付,把西门庆名字填注上面,列衔金吾卫衣左所副千户、山东等处提刑所理刑。又向来保道:“你二人替我进献生辰礼物,多有辛苦。”因问:“后边跪的是你甚么人?”来保才待说是伙计,那吴主管向前道:“小的是西门庆舅子,名唤吴典恩。”太师道:“你既是西门庆舅子,我观你倒好个仪表。”唤堂候官取过一张扎付:“我安你在本处清河县做个驿丞,倒也去的。”那吴典恩慌的磕头如捣蒜。又取过一张扎付来,把来保名字填写山东郓王府,做了一名校尉。俱磕头谢了,领了扎付。分咐明日早晨,吏、兵二部挂号,讨勘合,限日上任应役。又分咐翟谦西厢房管待酒饭,讨十两银子与他二人做路费,不在话下。

看官听说:那时徽宗,天下失政,奸臣当道,谗佞盈朝,高、杨、童、蔡四个奸党,在朝中卖官鬻狱,贿赂公行,悬秤升官,指方补价。夤缘钻刺者,骤升美任;贤能廉直者,经岁不除。以致风俗颓败,赃官污吏遍满天下,役烦赋兴,民穷盗起,天下骚然。不因奸臣居台辅,合是中原血染人。

当下翟谦把来保、吴主管邀到厢房管待,大盘大碗饱餐了一顿。翟谦向来保说:“我有一件事,央及你爹替我处处,未知你爹肯应承否?”来保道:“翟爹说那里话!蒙你老人家这等老爷前扶持看顾,不拣甚事,但肯分咐,无不奉命。”翟谦道:“不瞒你说,我答应老爷,每日止贱荆一人。我年将四十,常有疾病,身边通无所出。央及你爹,你那贵处有好人才女子,不拘十五六上下,替我寻一个送来。该多少财礼,我一一奉过去。”说毕,随将一封人事并回书付与来保,又送二人五两盘缠。来保再三不肯受,说道:“刚才老爷上已赏过了。翟爹还收回去。”翟谦道:“那是老爷的,此是我的,不必推辞。”当下吃毕酒饭,翟谦道:“如今我这里替你差个办事官,同你到下处,明早好往吏、兵二部挂号,就领了勘合,好起身。省的你明日又费往返了。我分咐了去,部里不敢迟滞你文书。”一面唤了个办事官,名唤李中友:“你与二位明日同到部里挂了号,讨勘合来回我话。”那员官与来保、吴典恩作辞,出的府门,来到天汉桥街上白酒店内会话。来保管待酒饭,又与了李中友三两银子,约定明日绝早先到吏部,然后到兵部,都挂号讨了勘合。闻得是太师老爷府里,谁敢迟滞,颠倒奉行。金吾卫太尉朱勔,即时使印,签了票帖,行下头司,把来保填注在本处山东郓王府当差。又拿了个拜帖,回翟管家。不消两日,把事情干得完备。有日雇头口起身,星夜回清河县来报喜。正是:

\[
富贵必因奸巧得,功名全仗邓通成。
\]

且说一日三伏天气,西门庆在家中聚景堂上大卷棚内,赏玩荷花,避暑饮酒。吴月娘与西门庆俱上坐,诸妾与大姐都两边列坐,春梅、迎春、玉箫、兰香,一般儿四个家乐在旁弹唱。怎见的当日酒席?但见:

\[
盆栽绿草,瓶插红花。水晶帘卷虾须,云母屏开孔雀。盘堆麟脯,佳人笑捧紫霞觞;盆浸冰桃,美女高擎碧玉斝。食烹异品,果献时新。弦管讴歌,奏一派声清韵美;绮罗珠翠,摆两行舞女歌儿。当筵象板撒红牙,遍体舞裙铺锦绣。消遣壶中闲日月,遨游身外醉乾坤。
\]

妻妾正饮酒中间,坐间不见了李瓶儿。月娘向绣春说道:“你娘往屋里做甚么哩?”绣春道:“我娘害肚里疼,\textuni{22C49}着哩。”月娘道:“还不快对他说去,休要\textuni{22C49}着,来这里听一回唱罢。”西门庆便问月娘:“怎的?”月娘道:“李大姐忽然害肚里疼,房里躺着哩。我使小丫头请他去了。”因向玉楼道:“李大姐七八临月,只怕搅撒了。”潘金莲道:“大姐姐,他那里是这个月?约他是八月里孩子,还早哩!”西门庆道:“既是早哩,使丫头请你六娘来听唱。”不一时,只见李瓶儿来到。月娘道:“只怕你掉了风冷气,你吃上锺热酒,管情就好了。”不一时,各人面前斟满了酒。西门庆分咐春梅:“你每唱个‘人皆畏夏日’我听。”那春梅等四个方才筝排雁柱,阮跨鲛绡,启朱唇,露皓齿,唱“人皆畏夏日”。那李瓶儿在酒席上,只是把眉头忔\textuni{3918}着,也没等的唱完,就回房中去了。月娘听了词曲,耽着心,使小玉房中瞧去。回来报说:“六娘害肚里疼,在炕上打滚哩。”慌了月娘道:“我说是时候,这六姐还强说早哩。还不唤小厮快请老娘去!”西门庆即令平安儿:“风跑!快请蔡老娘去!”于是连酒也吃不成,都来李瓶儿房中问他。

月娘问道:“李大姐,你心里觉的怎的?”李瓶儿回道:“大娘,我只心口连小肚子,往下鳖坠着疼。”月娘道:“你起来,休要睡着,只怕滚坏了胎。老娘请去了,便来也。”少顷,渐渐李瓶儿疼的紧了。月娘又问:“使了谁请老娘去了?这咱还不见来?”玳安道:“爹使来安去了。”月娘骂道:“这囚根子,你还不快迎迎去!平白没算计,使那小奴才去,有紧没慢的。”西门庆叫玳安快骑了骡子赶去。月娘道:“一个风火事,还象寻常慢条斯礼儿的。”那潘金莲见李瓶儿待养孩子,心中未免有几分气。在房里看了一回,把孟玉楼拉出来,两个站在西梢间檐柱儿底下那里歇凉,一处说话。说道:“耶嚛嚛!紧着热剌剌的挤了一屋子的人,也不是养孩子,都看着下象胆哩。”良久,只见蔡老娘进门,望众人道:“那位是主家奶奶?”李娇儿指着月娘道:“这位大娘哩。”那蔡老娘倒身磕头。月娘道:“姥姥,生受你。怎的这咱才来?请看这位娘子,敢待生养也?”蔡老娘向床前摸了摸李瓶儿身上,说道:“是时候了。”问:“大娘预备下绷接、草纸不曾?”月娘道:“有。”便叫小玉:“往我房中快取去!”

且说玉楼见老娘进门,便向金莲说:“蔡老娘来了,咱不往屋里看看去?”那金莲一面不是一面,说道:“你要看,你去。我是不看他。他是有孩子的姐姐,又有时运,人怎的不看他?头里我自不是,说了句话儿‘只怕是八月里的’,叫大姐姐白抢白相。我想起来好没来由,倒恼了我这半日。”玉楼道:“我也只说他是六月里孩子。”金莲道:“这回连你也韶刀了!我和你恁算:他从去年八月来,又不是黄花女儿,当年怀,入门养。一个婚后老婆,汉子不知见过了多少,也一两个月才生胎,就认做是咱家孩子?我说差了?若是八月里孩儿,还有咱家些影儿;若是六月的,踩小板凳儿糊险神道——还差着一帽头子哩!失迷了家乡,那里寻犊儿去?”正说着,只见小玉抱着草纸、绷接并小褥子儿来。孟玉楼道:“此是大姐姐自预备下他早晚用的,今日且借来应急儿。”金莲道:“一个是大老婆,一个是小老婆,明日两个对养,十分养不出来,零碎出来也罢。俺每是买了个母鸡不下蛋,莫不吃了我不成!”又道:“仰着合着,没的狗咬尿胞虚欢喜?”玉楼道:“五姐是甚么话!”以后见他说话不防头脑,只低着头弄裙带子,并不作声应答他。少顷,只见孙雪娥听见李瓶儿养孩子,从后边慌慌张张走来观看,不防黑影里被台基险些不曾绊了一交。金莲看见,教玉楼:“你看献勤的小妇奴才!你慢慢走,慌怎的?抢命哩!黑影子绊倒了,磕了牙也是钱!养下孩子来,明日赏你这小妇奴才一个纱帽戴!”良久,只听房里“呱”的一声养下来了。蔡老娘道:“对当家的老爹说,讨喜钱,分娩了一位哥儿。”吴月娘报与西门庆。西门庆慌忙洗手,天地祖先位下满炉降香,告许一百二十分清醮,要祈母子平安,临盆有庆,坐草无虞。这潘金莲听见生下孩子来了,合家欢喜,乱成一块,越发怒气,迳自去到房里,自闭门户,向床上哭去了。时宣和四年戊申六月念三日也。正是:

\[
不如意事常八九,可与人言无二三。
\]

蔡老娘收拾孩子,咬去脐带,埋毕衣胞,熬了些定心汤,打发李瓶儿吃了,安顿孩儿停当。月娘让老娘后边管待酒饭。临去,西门庆与了他五两一锭银子,许洗三朝来,还与他一匹缎子。这蔡老娘千恩万谢出门。

当日,西门庆进房去,见一个满抱的孩子,生的甚是白净,心中十分欢喜。合家无不欢悦。晚夕,就在李瓶儿房中歇了,不住来看孩儿。次日,巴天不明起来,拿十副方盒,使小厮各亲戚邻友处,分投送喜面。应伯爵、谢希大听见西门庆生了子,送喜面来,慌的两步做一步走来贺喜。西门庆留他卷棚内吃面。刚打发去了,正要使小厮叫媒人来寻养娘,忽有薛嫂儿领了个奶子来。原是小人家媳妇儿,年三十岁,新近丢了孩儿,不上一个月。男子汉当军,过不的,恐出征去无人养赡,只要六两银子卖他。月娘见他生的干净,对西门庆说,兑了六两银子留下,取名如意儿,教他早晚看奶哥儿。又把老冯叫来暗房中使唤,每月与他五钱银子,管顾他衣服。

正热闹一日,忽有平安报:“来保、吴主管在东京回还,见在门首下头口。”不一时,二人进来,见了西门庆报喜。西门庆问:“喜从何来?”二人悉把到东京见蔡太师进礼一节,从头至尾说道:“老爷见了礼物甚喜,说道:‘我累次受你主人之礼,无可补报。’朝廷钦赏了他几张空名诰身扎付,就与了爹一张,把爹名姓填注在金吾卫副千户之职,就委差在本处提刑所理刑,顶补贺老爷员缺。把小的做了铁铃卫校尉,填注郓王府当差。吴主管升做本县驿丞。”于是把一样三张印信扎付,并吏、兵二部勘合,并诰身都取出来,放在桌上与西门庆观看。西门庆看见上面衔着许多印信,朝廷钦依事例,果然他是副千户之职,不觉欢从额角眉尖出,喜向腮边笑脸生。便把朝廷明降,拿到后边与吴月娘众人观看,说:“太师老爷抬举我,升我做金吾卫副千户,居五品大夫之职。你顶受五花官诰,做了夫人。又把吴主管携带做了驿丞,来保做了郓王府校尉。吴神仙相我不少纱帽戴,有平地登云之喜,今日果然。不上半月,两椿喜事都应验了。”又对月娘说:“李大姐养的这孩子甚是脚硬,到三日洗了三,就起名叫做官哥儿罢。”来保进来,与月娘众人磕头,说了回话。分咐明日早把文书下到提刑所衙门里,与夏提刑知会了。吴主管明日早下文书到本县,作辞西门庆回家去了。

到次日,洗三毕,众亲邻朋友一概都知西门庆第六个娘子新添了娃儿,未过三日,就有如此美事,官禄临门,平地做了千户之职。谁人不来趋附?送礼庆贺,人来人去,一日不断头。常言:时来谁不来?时不来谁来!正是:

\[
时来顽铁有光辉,运退真金无颜色。
\]
