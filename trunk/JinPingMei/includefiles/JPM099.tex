%# -*- coding:utf-8 -*-
%%%%%%%%%%%%%%%%%%%%%%%%%%%%%%%%%%%%%%%%%%%%%%%%%%%%%%%%%%%%%%%%%%%%%%%%%%%%%%%%%%%%%


\chapter{刘二醉骂王六儿\KG 张胜窃听张敬济}


词曰:

\[
白云山,红叶树,阅尽兴亡,一似朝还暮。多少夕阳芳草渡,潮落潮生,还送人来去。阮公途,杨子路,九折羊肠,曾把车轮误。记得寒芫嘶马处,翠官银筝,夜夜歌楼曙。\named{右调《苏幕遮》}
\]

话说陈敬济,过了两日,到第三日,却是五月二十日他的生日,后厅整置酒肴,与他上寿,合家欢乐了一日。次日早辰,敬济说:“我一向不曾往河下去,今日没事,去走一遭,一者和主管算帐,二来就避炎暑,走走便回。”春梅分付:“你去坐一乘轿子,少要劳碌。”交两个军牢抬着轿子,小姜儿跟随,径往河下在酒楼店中来。

一路无词,午后时分到了,下轿进入里面。两个主管齐来参见,说:“官人贵体好些?”敬济道:“生受二位伙计挂心。”他一心只在韩爱姐身上,坐了一回便起身,分付主管:“查下帐目,等我来算。”就转身到后边。八老又早迎见,报与王六儿夫妇。韩爱姐正在楼上,凭栏盼望,挥毫作诗遣怀。忽报陈敬济来了,连忙轻移莲步,款蹙湘裙,走下楼来。母子面上堆下笑来迎接,说道:“官人,贵人难见面,那阵风儿吹你到俺这里?”敬济与他母子作了揖,同进阁儿内坐定。少顷,王六儿点茶上来。吃毕茶,爱姐道:“请官人到楼上奴房内坐。”敬济上的楼来,两个如鱼得水,似膝投胶,无非说些深情密意的话儿。爱姐砚台底下,露出一幅花笺,敬济取来观看。爱姐便说:“此是奴家盼你不来,作得一首诗,以消遣闷怀,恐污官人贵目。”敬济念了一遍,上写着:

\[
倦倚绣床愁懒动,闲垂锦帐鬓鬟低。
玉郎一去无消息,一日相思十二时。
\]
敬济看了,极口称羡不已。不一时,王六儿安排酒肴上楼,拨过镜架,就摆在梳妆卓上。两个并坐,爱姐筛酒一杯,双手递与敬济,深深道个万福,说:“官人一向不来,妾心无时不念。前八老来,又多谢盘缠,举家感之不尽。”敬济接酒在手,还了喏,说:“贱疾不安,有失期约,姐姐休怪。”酒尽,也筛一杯敬奉爱姐吃过,两个坐定,把酒来斟。王六儿、韩道国上来,也陪吃了几杯,各取方便下楼去了,教他二人自在吃几杯,叙些阔别话儿。良久,吃得酒浓时,情兴如火,免不得再把旧情一叙。交欢之际,无限恩情。穿衣起来,洗手更酌,又饮数杯。醉眼朦胧,余兴未尽。这小郎君,一向在家中不快,又心在爱姐,一向未与浑家行事。今日一旦见了情人,未肯一次即休。正是生死冤家,五百年前撞在一处,敬济魂灵都被他引乱。少顷,情窦复起,又干一度。自觉身体困倦,打熬不过,午饭也没吃,倒在床上就睡着了。

也是合当祸起,不想下边贩丝绵何官人来了,王六儿陪他在楼下吃酒。韩道国出去街上买菜蔬、肴品、果子来配酒。两个在下边行房。落后韩道国买将果菜来,三人又吃了几杯。约日西时分,只见洒家店坐地虎刘二,吃的酩酊大醉,軃开衣衫,露着一身紫肉,提着拳头走来酒楼下,大叫:“采出何蛮子来!”唬的两个主管见敬济在楼上睡,恐他听见,慌忙走出柜来,向前声诺,说道:“刘二哥,何官人并不曾来。”这刘二那里依听。大拔步撞入后边韩道国屋里,一手把门帘扯去半边,看见何官人正和王六儿并肩饮酒,心中大怒,便骂何官人:“贼狗男女,我肏你娘!那里没寻你,却在这里。你在我店中,占着两个粉头,几遭歇钱不与,又塌下我两个月房钱,却来这里养老婆!”那何官人忙出来道:“老二你休怪,我去罢。”那刘二骂道:“去你这狗入的!”不防飕的一拳来,正打在何官人面上,登时就青肿起来。那何官人也不顾,径夺门跑了。刘二将王六儿酒卓,一脚登翻,家活都打了。王六儿便骂道:“是那里少死的贼杀了!无事来老娘屋里放屁。娘不是耐惊耐怕儿的人!”被刘二向前一脚,跺了个仰八叉,骂道:“我入你淫妇娘!你是那里来的无名少姓私窠子?不来老爷手里报过,许你在这酒店内趁熟?还与我搬去!若搬迟,须吃我一顿好拳头。”那王六儿道:“你是那里来的光棍捣子?老娘就没了亲戚儿?许你便来欺负老娘,要老娘这命做甚么?”一头撞倒哭起来。刘二骂道:“我把淫妇肠子也踢断了,你还不知老爷是谁哩!”这里喧乱,两边邻舍并街上过往人,登时围看约有许多。有知道的旁边人说:“王六儿,你新来不知,他是守备老爷府中管事张虞候的小舅子,有名坐地虎刘二。在洒家店住,专一是打粉头的班头,降酒店的领袖。你让他些儿罢,休要不知利害。这地方人,谁敢惹他!”王六儿道:“还有大似他的,睬这杀才做甚么?”陆秉义见刘二打得凶,和谢胖子做好做歹,把他劝的去了。

陈敬济正睡在床上,听见楼下攘乱,便起来看,时天已日西时分,问:“那里攘乱?”那韩道国不知走的往那里去了,只见王六儿披发垢面上楼,如此这般告诉说:“那里走来一个杀才捣子,诨名唤坐地虎刘二,在洒家店住,说是咱府里管事张虞候小舅子。因寻酒店,无事把我踢打,骂了恁一顿去了。又把家活酒器都打得粉碎。”一面放声大哭起来。敬济就叫上两个主管去问。两个主管隐瞒不住,只得说:“是府中张虞候小舅子刘二,来这里寻何官人讨房钱,见他在屋里吃酒,不由分说,把帘子扯下半边来,打了何官人一拳,唬的何官人跑了。又和老韩娘子两个相骂,踢了一交,烘的满街人看。”敬济听了,便晓得是前番做道士,被他打的刘二了。欲要声张,又恐刘二泼皮行凶,一时斗他不过。又见天色晚了,因问:“刘二那厮如今在那里?”主管道:“被小人劝他回去了。”敬济安抚王六儿道:“你母子放心,有我哩,不妨事。你母子只情住着,我家去自有处置。”主管算了利钱银两递与他,打发起身上轿,伴当跟随。刚赶进城来,天已昏黑,心中甚恼。到家见了春梅,交了利息银两,归入房中。

一宿无话。到次日,心心念念要告春梅说,展转寻思:“且住,等我慢慢寻张胜那厮几件破绽,亦发教我姐姐对老爷说了,断送了他性命。叵耐这厮,几次在我身上欺心,敢说我是他寻得来,知我根本出身,量视我禁不得他。”正是:

\[
冤仇还报当如此,机会遭逢莫远图。
踏破铁鞋无觅处,得来全不费工夫。
\]

一日,敬济来到河下酒店内,见了爱姐母子,说:“外日吃惊。”又问陆主管道:“刘二那厮可曾走动?”陆主管道:“自从那日去了,再不曾来。”又问韩爱姐:“那何官人也没来行走?”爱姐道:“也没曾来。”这敬济吃了饭,算毕帐目,不免又到爱姐楼上。两个叙了回衷肠之话,干讫一度出来,因闲中叫过量酒陈三儿近前,如此这般,打听府中张胜和刘二几桩破绽。这陈三儿千不合,万不合,说出张胜包占着府中出来的雪娥,在洒家店做表子。刘二又怎的各处巢窝,加三讨利,举放私债,逞着老爷名坏事。这敬济听记在心,又与了爱姐二三两盘缠,和主管算了帐目,包了利息银两,作别骑头口来家。

闲话休题。一向怀意在心,一者也是冤家相凑,二来合当祸起。不料东京朝中徽宗天子,见大金人马犯边,抢至腹内地方,声息十分紧急。天子慌了,与大臣计议,差官往北国讲和,情愿每年输纳岁币,金银彩帛数百万。一面传位与太子登基,改宣和七年为靖康元年,宣帝号为钦宗。皇帝在位,徽宗自称太上道君皇帝,退居龙德宫。朝中升了李纲为兵部尚书,分部诸路人马。种师道为大将,总督内外军务。

一日,降了一道敕书来济南府,升周守备为山东都统制,提调人马一万,前往东昌府驻扎,会同巡抚都御史张叔夜,防守地方,阻挡金兵。守备领了敕书,不敢怠慢,一面叫过张胜、李安两个虞候近前分付,先押两车箱驮行李细软器物家去。原来在济南做了一年官,也撰得巨万金银。都装在行李驮箱内,委托二人押到家中:“交割明白,昼夜巡风仔细。我不日会同你巡抚张爷,调领四路兵马,打清河县起身。”二人当日领了钧旨,打点车辆,起身先行。一路无词。有日到了府中,交割明白,二人昼夜内外巡风,不在话下。

却说陈敬济见张胜押车辆来家,守备升了山东统制,不久将到,正欲把心腹中事要告诉春梅,等守备来家,发露张胜之事。不想一日因浑家葛翠屏往娘家回门住去了,他独自个在西书房寝歇,春梅蓦进房中看他。见丫鬟跟随,两个就解衣在房内云雨做一处。不防张胜摇着铃,巡风过来,到书院角门外,听见书房内仿佛有妇人笑语之声,就把铃声按住,慢慢走来窗下窃听。原来春梅在里面与敬济交媾。听得敬济告诉春梅说:“叵耐张胜那厮,好生欺压于我,说我当初亏他寻得来,几次在下人前败坏我。昨日见我在河下开酒店,一径使小舅子坐地虎刘二,来打我的酒店,把酒客都打散了。专一倚逞他在姐夫麾下,在那里开巢窝,放私债,又把雪娥隐占在外奸宿,只瞒了姐姐一人眼目。我几次含忍,不敢告姐姐说,趁姐夫来家,若不早说知,往后我定然不敢往河下做买卖去了。”春梅听了,说道:“这厮恁般无礼。雪娥那贱人,我卖了他,如何又留住在外?”敬济道:“他非是欺压我,就是欺压姐姐一般。”春梅道:“等他爷来家,交他定结果了这厮。”

常言道:“隔墙须有耳,窗外岂无人。”两个只管在内说,却不知张胜窗外听得明明白白,口中不言,心内暗道:“此时教他算计我,不如我先算计了他罢。”一面撇下铃,走到前边班房内,取了把解腕钢刀,说时迟,那时快,在石上磨了两磨,走入书院中来。不想天假其便,还是春梅不该死于他手。忽被后边小丫鬟兰花儿,慌慌走来叫春梅,报说:“小衙内金哥儿忽然风摇倒了,快请奶奶看去。”唬的春梅两步做一步走,奔了后房中看孩儿去了。刚进去了,那张胜提着刀子,径奔到书房内,不见春梅,只见敬济睡在被窝内。见他进来,叫道:“阿呀,你来做甚么?”张胜怒道:“我来杀你!你如何对淫妇说,倒要害我?我寻得你来不是了?反恩将仇报!常言“黑头虫儿不可救,救之就要吃人肉”,休走,吃我一刀子!明年今日是你死忌!”那敬济光赤条身子,没处躲,只搂着被,吃他拉过一边,向他身就扎了一刀子来。扎着软肋,鲜血就邈出来。这张胜见他挣扎,复又一刀去,攘着胸膛上,动弹不得了。一面采着头发,把头割下来,正是:

\[
三寸气在千般用,一日无常万事休。
\]

可怜敬济青春不上三九,死于非命。张胜提刀,绕屋里床背后,寻春梅不见,大拔步径望后厅走。走到仪门首,只见李安背着牌铃,在那里巡风。一见张胜凶神也似提着刀跑进来,便问:“那里去?”张胜不答,只顾走,被李安拦住。张胜就向李安戳一刀来。李安冷笑,说道:“我叔叔有名山东夜叉李贵,我的本事不用借。”早飞起右脚,只听忒楞的一声,把手中刀子踢落一边。张胜急了,两个就揪采在一处,被李安一个泼脚,跌番在地,解下腰间缠带登时绑了。嚷的后厅春梅知道,说:“张胜持刀入内,小的拿住了。”

那春梅方救得金哥苏醒,听言大惊失色。走到书院内,见敬济已被杀死在房中,一地鲜血横流,不觉放声大哭。一面使人报知浑家。葛翠屏慌奔家来,看见敬济杀死,哭倒在地,不省人事。被春梅扶救苏醒过来。拖过尸首,买棺材装殡。把张胜墩锁在监内,单等统制来家处治这件事。

那消数日,只见军情事务紧急,兵牌来催促。周统制调完各路兵马,张巡抚又早先往东昌府那里等候取齐。统制到家,春梅把杀死敬济一节说了。李安将凶器放在面前,跪禀前事。统制大怒,坐在厅上,提出张胜,也不问长短,喝令军牢,五棍一换,打一百棍,登时打死。随马上差旗牌快手,往河下捉拿坐地虎刘二,锁解前来。孙雪娥见拿了刘二,恐怕拿他,走到房中,自缢身死。旗牌拿刘二到府中,统制也分付打一百棍,当日打死。烘动了清河县,大闹了临清州。正是:

\[
平生作恶欺天,今日上苍报应。
\]
有诗为证:

\[
为人切莫用欺心,举头三尺有神明。
若还作恶无报应,天下凶徒人食人。
\]
当时统制打死二人,除了地方之害。分付李安将马头大酒店还归本主,把本钱收算来家。分付春梅在家,与敬济修斋做七,打发城外永福寺葬埋。留李安、周义看家,把周忠、周仁带去军门答应。春梅晚夕与孙二娘,置酒送饯,不觉簇地两行泪下,说:“相公此去,未知几时回还,出战之间,须要仔细。番兵猖獗,不可轻敌。”统制道:“你每自在家清心寡欲,好生看守孩儿,不必忧念。我既受朝廷爵禄,尽忠报国。至于吉凶存亡,付之天也。”嘱咐毕,过了一宿。次日,军马都在城外屯集,等候统制起程。一路无词。有日到了东昌府下,统制差一面令字蓝旗,打报进城。巡抚张叔夜,听见周统制人马来到,与东昌府知府达天道出衙迎接。至公厅叙礼坐下,商议军情,打听声息紧慢。驻马一夜,次日人马早行,往关上防守去了。不在话下。

却表韩爱姐母子,在谢家楼店中听见陈敬济已死,爱姐昼夜只是哭泣,茶饭都不吃,一心只要往城内统制府中,见敬济尸首一见,死也甘心。父母、旁人百般劝解不众。韩道国无法可处,使八老往统制府中打听,敬济灵柩已出了殡,埋在城外永福寺内。这八老走来,回了话。爱姐一心要到他坟上烧纸,哭一场,也是和他相交一场。做父母的只得依他。雇了一乘轿子,到永福寺中,问长老葬于何处。长老令沙弥引到寺后,新坟堆便是。这韩爱姐下了轿子,到坟前点着纸袋,道了万福,叫声:“亲郎我的哥哥!奴实指望和你同谐到老,谁想今日死了!”放声大哭,哭的昏晕倒了,头撞于地下,就死过去了。慌了韩道国和王六儿,向前扶救,叫姐姐,叫不应,越发慌了。

不想那日,正是葬的三日,春梅与浑家葛翠屏坐着两乘轿子,伴当跟随,抬三牲祭物,来与他暖墓烧纸。看见一个年小的妇人,穿着缟素,头戴孝髻,哭倒在地。一个男子汉和一中年妇人,搂抱他扶起来,又倒了,不省人事,吃了一惊。因问那男子汉是那里的,这韩道国夫妇向前施礼,把从前已往话,告诉了一遍:“这个是我的女孩儿韩爱姐。”春梅一闻爱姐之名,就想起昔日曾在西门庆家中会过,又认得王六儿。韩道国悉把东京蔡府中出来一节,说了一遍:“女孩儿曾与陈官人有一面之交,不料死了。他只要来坟前见他一见,烧纸钱,不想到这里,又哭倒了。”当下两个救了半日,这爱姐吐了口粘痰,方才苏醒,尚哽咽哭不出声来。痛哭了一场起来,与春梅、翠屏插烛也似磕了四个头,说道:“奴与他虽是露水夫妻,他与奴说山盟,言海誓,情深意厚,实指望和他同谐到老,谁知天不从人愿,一旦他先死了,撇得奴四脯着地。他在日曾与奴一方吴绫帕儿,上有四句情诗。知道宅中有姐姐,奴愿做小,倘不信——”向袖中取出吴绫帕儿来,上面写诗四句,春梅同葛翠屏看了。诗云:

\[
吴绫帕儿织回纹,洒翰挥毫墨迹新。
寄与多情韩五姐,永谐鸾凤百年情。
\]
爱姐道:“奴也有个小小鸳鸯锦囊,与他佩载在身边。两面都扣绣着并头莲,每朵莲花瓣儿一个字儿:寄与情郎陈君膝下。”春梅便问翠屏:“怎的不见这个香囊?”翠屏道:“在底裤子上拴着,奴替他装殓在棺椁内了。”

当下祭毕,让他母子到寺中摆茶饭,劝他吃了些。王六儿见天色将晚,催促他起身,他只顾不思动身。一面跪着春梅、葛翠屏哭说:“奴情愿不归父母,同姐姐守孝寡居。明日死,傍他魂灵,也是奴和他恩情一场,说是他妻小。”说着那泪如泉涌。翠屏只顾不言语。春梅便说:“我的姐姐,只怕年小青春,守不住,却不误了你好时光。”爱姐便道:“奶奶说那里话?奴既为他,虽刳目断鼻也当守节,誓不再配他人。”嘱付他父母:“你老公婆回去罢,我跟奶奶和姐姐府中去也。”那王六儿眼中垂泪,哭道:“我承望你养活俺两口儿到老,才从虎穴龙潭中夺得你来。今日倒闪赚了我。”那爱姐口里只说:“我不去了。你就留下我,到家也寻了无常。”那韩道国因见女儿坚意不去,和王六儿大哭一场,洒泪而别,回上临清店中去了。这韩爱姐同春梅、翠屏,坐轿子往府里来。那王六儿一路上悲悲切切,只是舍不的他女儿,哭了一场又一场。那韩道国又怕天色晚了,雇上两匹头口,望前赶路。正是:

\[
马迟心急路途穷,身似浮萍类转蓬。
只有都门楼上月,照人离恨各西东。
\]
