%# -*- coding:utf-8 -*-
%%%%%%%%%%%%%%%%%%%%%%%%%%%%%%%%%%%%%%%%%%%%%%%%%%%%%%%%%%%%%%%%%%%%%%%%%%%%%%%%%%%%%


\chapter{宇给事劾倒杨提督\KG 李瓶儿许嫁蒋竹山}


诗曰:

\[
早知君爱歇,本自无容妒;谁使恩情深,今来反相误。
愁眠罗帐晓,泣坐金闺暮;独有梦中魂,犹言意如故。
\]

话说五月二十日,帅府周守备生日。西门庆封五星分资、两方手帕,打选衣帽齐整,骑匹大白马,四个小厮跟随,往他家拜寿。席间也有夏提刑、张团练、荆千户、贺千户一班武官儿饮酒,鼓乐迎接,搬演戏文。玳安接了衣裳,回马来家。到日西时分,又骑马去接,走到西街口上,撞见冯妈妈,问道:“冯妈妈那里去?”冯妈妈道:“你二娘使我来请你爹。雇银匠整理头面完备,今日送来,请你爹那里瞧去。你二娘还和你爹说话哩!”玳安道:“俺爹今日在守备府周老爷处吃酒,我如今接去。你老人家回罢。等我到那里,对爹说就是了。”冯妈妈道:“累你好歹说声,你二娘等着哩!”这玳安打马迳到守备府。众官员正饮酒间,玳安走到西门庆席前,说道:“小的回马家来时,在街口撞遇冯妈妈,二娘使了来说,雇银匠送了头面来了,请爹瞧去,还要和爹说话哩。”西门庆听了,就要起身,那周守备那里肯放,拦门拿巨杯相劝。西门庆道:“蒙大人见赐,宁可饮一杯,还有些小事,不能尽情,恕罪,恕罪!”于是一饮而尽,辞周守备上马,迳到李瓶儿家。

妇人接着,茶汤毕,西门庆分付玳安回马家去,明日来接。玳安去了。李瓶儿叫迎春盒儿内取出头面来,与西门庆过目。黄烘烘火焰般一付好头面,收过去,单等二十四日行礼,出月初四日准娶。妇人满心欢喜,连忙安排酒来,和西门庆畅饮开怀。吃了一回,使丫鬟房中搽抹凉席干净。两个在纱帐之中,香焚兰麝,衾展鲛绡,脱去衣裳,并肩叠股,饮酒调笑。良久,春色横眉,淫心荡漾。西门庆先和妇人云雨一回,然后乘着酒兴,坐于床上,令妇人横躺于衽席之上,与他品箫。但见:

\[
不竹不丝不石,肉音别自唔咿。流苏瑟瑟碧纱垂,辨不出宫商角徵。一点樱桃欲绽,纤纤十指频移。深吞添吐两情痴,不觉灵犀味美。
\]
西门庆醉中戏问妇人:“当初花子虚在时,也和他干此事不干?”妇人道:“他逐日睡生梦死,奴那里耐烦和他干这营生!他每日只在外边胡撞,就来家,奴等闲也不和他沾身。况且老公公在时,和他另在一间房睡着,我还把他骂的狗血喷了头。好不好,对老公公说了,要打倘棍儿。奴与他这般顽耍,可不碜杀奴罢了!谁似冤家这般可奴之意,就是医奴的药一般。白日黑夜,教奴只是想你。”两个耍一回,又干了一回。傍边迎春伺候下一个小方盒,都是各样细巧果品,小金壶内满泛琼浆。从黄昏掌上灯烛,且干且歇,直耍到一更时分。只听外边一片声打的大门响,使冯妈妈开门瞧去,原来是玳安来了。西门庆道:“我分付明日来接,这咱晚又来做甚么?”因叫进来问他。那小厮慌慌张张走到房门首,因西门庆与妇人睡着,又不敢进来,只在帘外说道:“姐姐、姐夫都搬来了,许多箱笼在家中。大娘使我来请爹,快去计较话哩。”这西门庆听了,只顾犹豫:“这咱晚,端的有甚缘故?须得到家瞧瞧。”连忙起来。妇人打发穿上衣服,做了一盏暖酒与他吃。

打马一直到家,只见后堂中秉着灯烛,女儿女婿都来了,堆着许多箱笼床帐家伙,先吃了一惊,因问:“怎的这咱来家?”女婿陈敬济磕了头,哭说:“近日朝中,俺杨老爷被科道官参论倒了。圣旨下来,拿送南牢问罪。门下亲族用事人等,都问拟枷充军。昨日府中杨干办连夜奔来,透报与父亲知道。父亲慌了,教儿子同大姐和些家伙箱笼,且暂在爹家中寄放,躲避些时。他便起身往东京我姑娘那里,打听消息去了。待事宁之日,恩有重报,不敢有忘。”西门庆问:“你爹有书没有?”陈敬济道:“有书在此。”向袖中取出,递与西门庆。折开观看,上面写道:

\[
眷生陈洪顿首书奉大德西门庆亲家台览:余情不叙。兹因北虏犯边,抢过雄州地界,兵部王尚书不发救兵,失误军机,连累朝中杨老爷,俱被科道官参劾太重。圣旨恼怒,拿下南牢监禁,会同三法司审问。其门下亲族用事人等,俱照例发边卫充军。生一闻消息,举家惊惶,无处可投,先打发小儿、令爱,随身箱笼家活,暂借亲家府上寄寓。生即上京,投在姐夫张世廉处,打听示下。待事务宁帖之日,回家恩有重报,不敢有忘。诚恐县中有甚声色,生令小儿外具银五百两,相烦亲家费心处料,容当叩报没齿不忘。灯下草书,不宣。\named{仲夏二十日洪再拜}
\]

西门庆看了,慌了手脚,教吴月娘安排酒饭,管待女儿、女婿。就令家下人等,打扫厅前东厢房三间,与他两口儿居住。把箱笼细软都收拾月娘上房来。陈敬济取出他那五百两银子,交与西门庆打点使用。西门庆叫了吴主管来,与他五百两银子,教他连夜往县中承行房里,抄录一张东京行下来的文书邸报来看。上面端的写的是甚言语:

\[
兵科给事中宇文虚中等一本,恳乞宸断,亟诛误国权奸,以振本兵,以消虏患事:臣闻夷狄之祸,自古有之。周之猃狁,汉之匈奴,唐之突厥,迨及五代而契丹浸强,至我皇宋建国,大辽纵横中原者已非一日。然未闻内无夷狄而外萌夷狄之患者。语云:霜降而堂钟鸣,雨下而柱础润。以类感类,必然之理。譬若病夫,腹心之疾已久,元气内消,风邪外入,四肢百骸,无非受病,虽卢扁莫之能救,焉能久乎?今天下之势,正犹病夫兀羸之极矣。君犹元首也,辅臣犹腹心也,百官犹四肢也。陛下端拱于九重之上,百官庶政各尽职于下。元气内充,荣卫外扞,则虏患何由而至哉?今招夷虏之患者,莫如崇政殿大学士蔡京者:本以倹邪奸险之资,济以寡廉鲜耻之行,谗谄面谀,上不能辅君当道,赞元理化;下不能宣德布政,保爱元元。徒以利禄自资,希宠固位,树党怀奸,蒙蔽欺君,中伤善类。忠士为之解体,四海为之寒心。联翩朱紫,萃聚一门。迩者河湟失议,主议伐辽,内割三郡,郭药师之叛,卒使金虏背盟,凭陵中原。此皆误国之大者,皆由京之不职也。王黼贪庸无赖,行比俳优。蒙京汲引,荐居政府,未几谬掌本兵。惟事慕位苟安,终无一筹可展。乃者张达残于太原,为之张皇失散。今虏犯内地,则又挈妻子南下,为自全之计。其误国之罪,可胜诛戮?杨戬本以纨绔膏粱叨承祖荫,凭籍宠灵典司兵柄,滥膺阃外,大奸似忠,怯懦无比。此三臣者,皆朋党固结,内外蒙蔽,为陛下腹心之蛊者也。数年以来,招灾致异,丧本伤元,役重赋烦,生民离散,盗贼猖獗,夷虏犯顺,天下之膏腴已尽,国家之纲纪废弛,虽擢发不足以数京等之罪也。臣等待罪该科,备员谏职,徒以目击奸臣误国,而不为皇上陈之,则上辜君父之恩,下负平生所学。伏乞宸断,将京等一干党恶人犯,或下廷尉,以示薄罚;或致极典,以彰显戮;或照例枷号;或投之荒裔,以御魑魅。庶天意可回,人心畅快,国法以正,虏患自消。天下幸甚!臣民幸甚!奉圣旨:“蔡京姑留辅政。王黼、杨戬着拿送三法司,会问明白来说。钦此钦遵。”续该三法司会问过,并党恶人犯王黼、杨戬,本兵不职,纵虏深入,荼毒生民,损兵折将,失陷内地,律应处斩。手下坏事家人、书办、官掾、亲家董升、卢虎、杨盛、庞宣、韩宗仁、陈洪、黄玉、刘盛、赵弘道等,查出有名人犯,俱问拟枷号一个月,满日发边卫充军。
\]

西门庆不看,万事皆休;看了耳边厢只听飕的一声,魂魄不知往那里去了。就是:

\[
惊伤六叶连肝肺,吓坏三毛七孔心。
\]
当下即忙打点金银宝玩,驮装停当,把家人来保、来旺叫到卧房中,悄悄分付,如此这般:“雇头口星夜上东京打听消息。不消到你陈亲家老爹下处。但有不好声色,取巧打点停当,速来回报。”又与了他二人二十两银子。绝早五更雇脚夫起程,上东京去了,不在话下。

西门庆通一夜不曾睡着,到次日早,分付来昭、贲四,把花园工程止住,各项匠人都且回去,不做了。每日将大门紧闭,家下人无事亦不许往外去。西门庆只在房里走来走去,忧上加忧,闷上加闷,如热地蜒蚰一般,把娶李瓶儿的勾当丢在九霄云外去了。吴月娘见他愁眉不展,面带忧容,只得宽慰他,说道:“他陈亲家那边为事,各人冤有头债有主,你也不需焦愁如此。”西门庆道:“你妇人都知道些甚么?陈亲家是我的亲家,女儿、女婿两个孽障搬来咱家住着,平昔街坊邻舍恼咱的极多,常言:机儿不快梭儿快,打着羊驹驴战。倘有小人指搠,拔树寻根,你我身家不保。”正是:关门家里坐,祸从天上来。这里西门庆在家纳闷,不题。

且说李瓶儿等了一日两日,不见动静,一连使冯妈妈来了两遍,大门关得铁桶相似。等了半日,没一个人牙儿出来,竟不知怎的。看看到二十四日,李瓶儿又使冯妈妈送头面来,就请西门庆过去说话。叫门不开,立在对过房檐下等。少顷,只见玳安出来饮马,看见便问:“冯妈妈,你来做甚么?”冯妈妈说:“你二娘使我送头面来,怎的不见动静?请你爹过去说话哩。”玳安道:“俺爹连日有些事儿,不得闲。你老人家还拿头面去,等我饮马回来,对俺爹说就是了。”冯妈妈道:“好哥哥,我这在里等着,你拿进头面去和你爹说去。你二娘那里好不恼我哩!”这玳安一面把马拴下,走到里边,半日出来道:“对爹说了,头面爹收下了,教你上覆二娘,再待几日儿,我爹出来往二娘那里说话。”这冯妈妈一直走来,回了妇人话。妇人又等了几日,看看五月将尽,六月初旬,朝思暮盼,音信全无,梦攘魂劳,佳期间阻。正是:

\[
懒把蛾眉扫,羞将粉脸匀。
满怀幽恨积,憔悴玉精神。
\]

妇人盼不见西门庆来,每日茶饭顿减,精神恍惚。到晚夕,孤眠枕上展转踌蹰。忽听外边打门,仿佛见西门庆来到。妇人迎门笑接,携手进房,问其爽约之情,各诉衷肠之话。绸缪缱绻,彻夜欢娱。鸡鸣天晓,便抽身回去。妇人恍然惊觉,大呼一声,精魂已失。冯妈妈听见,慌忙进房来看。妇人说道:“西门他爹刚才出去,你关上门不曾?”冯妈妈道:“娘子想得心迷了,那里得大官人来?影儿也没有!”妇人自此梦境随邪,夜夜有狐狸假名抵姓,摄其精髓。渐渐形容黄瘦,饮食不进,卧床不起。冯妈妈向妇人说,请了大街口蒋竹山来看。其人年不上三十,生的五短身材,人物飘逸,极是轻浮狂诈。请入卧室,妇人则雾鬓云鬟,拥衾而卧,似不胜忧愁之状。茶汤已罢,丫鬟安放褥垫。竹山就床诊视脉息毕,因见妇人生有姿色,便开口说道:“学生适诊病源,娘子肝脉弦出寸口而洪大,厥阴脉出寸口久上鱼际,主六欲七情所致。阴阳交争,乍寒乍热,似有郁结于中而不遂之意也。似疟非疟,似寒非寒,白日则倦怠嗜卧,精神短少;夜晚神不守舍,梦与鬼交。若不早治,久而变为骨蒸之疾,必有属纩之忧矣。可惜,可惜!”妇人道:“有累先生,俯赐良剂。奴好了,重加酬谢。”竹山道:“学生无不用心,娘子若服了我的药,必然贵体全安。”说毕起身。这里送药金五星,使冯妈妈讨将药来。妇人晚间吃了药下去,夜里得睡,便不惊恐。渐渐饮食加添,起来梳头走动。那消数日,精神复旧。

一日,安排了一席酒肴,备下三两银子,使冯妈妈请过竹山来相谢。蒋竹山自从与妇人看病,怀觊觎之心已非一日。一闻其请,即具服而往。延之中堂,妇人盛妆出见,道了万福,茶汤两换,请入房中。酒肴已陈,麝兰香蔼。小丫鬟绣春在傍,描金盘内托出三两白金。妇人高擎玉盏,向前施礼,说道:“前日,奴家心中不好,蒙赐良剂,服之见效。今粗治了一杯水酒,请过先生来知谢知谢。”竹山道:“此是学生分内之事,理当措置,何必计较!”因见三两谢礼,说道:“这个学生怎么敢领?”妇人道:“些须微意,不成礼数,万望先生笑纳。”辞让了半日,竹山方才收了。妇人递酒,安下坐次。饮过三巡,竹山偷眼睃视妇人,粉妆玉琢,娇艳惊人,先用言以挑之,因道:“学生不敢动问,娘子青春几何?”妇人道:“奴虚度二十四岁。”竹山道:“似娘子这等妙年,生长深闺,处于富足,何事不遂,而前日有此郁结不足之病?”妇人听了,微笑道:“不瞒先生,奴因拙夫弃世,家事萧条,独自一身,忧愁思虑,何得无病!”竹山道:“原来娘子夫主殁了。多少时了?”妇人道:“拙夫从去岁十一月得伤寒病死了,今已八个月。”竹山道:“曾吃谁的药来?”妇人道:“大街上胡先生。”竹山道:“是那东街上刘太监房子住的胡鬼嘴儿?他又不是我太医院出身,知道甚么脉,娘子怎的请他?”妇人道:“也是因街坊上人荐举请他来看。还是拙夫没命,不干他事。”竹山又道:“娘子也还有子女没有?”妇人道:“儿女俱无。”竹山道:“可惜娘子这般青春妙龄之际,独自孀居,又无所出,何不寻其别进之路?甘为幽闷,岂不生病!”妇人道:“奴近日也讲着亲事,早晚过门。”竹山便道:“动问娘子与何人作亲?”妇人道:“是县前开生药铺西门大官人。”竹山听了道:“苦哉,苦哉!娘子因何嫁他?学生常在他家看病,最知详细。此人专在县中包揽说事,广放私债,贩卖人口,家中丫头不算,大小五六个老婆,着紧打倘棍儿,稍不中意,就令媒人领出卖了。就是打老婆的班头,坑妇女的领袖。娘子早是对我说,不然进入他家,如飞蛾投火一般,坑你上不上,下不下,那时悔之晚矣。况近日他亲家那边为事干连,在家躲避不出,房子盖的半落不合的,都丢下了。东京关下文书,坐落府县拿人。到明日他盖这房子,多是入官抄没的数儿。娘子没来由嫁他做甚?”一篇话把妇人说的闭口无言。况且许多东西丢在他家,寻思半晌,暗中跌脚:“嗔怪道一替两替请着他不来,他家中为事哩!”又见竹山语言活动,一团谦恭:“奴明日若嫁得恁样个人也罢了,不知他有妻室没有?”因说道:“既蒙先生指教,奴家感戴不浅,倘有甚相知人家,举保来说,奴无有个不依之理。”竹山乘机请问:“不知要何等样人家?学生打听的实,好来这里说。”妇人道:“人家到也不论大小,只要象先生这般人物的。”这蒋竹山不听便罢,听了此言,欢喜的满心痒,不知搔处,慌忙走下席来,双膝跪下告道:“不瞒娘子说,学生内帏失助,中馈乏人,鳏居已久,子息全无。倘蒙娘子垂怜,肯结秦晋之缘,足称平生之愿。学生虽衔环结草,不敢有忘。”妇人笑笑,以手携之,说道:“且请起,未审先生鳏居几时?贵庚多少?既要做亲,须得要个保山来说,方成礼数。”竹山又跪下哀告道:“学生行年二十九岁,正月二十七日卯时建生,不幸去年荆妻已故,家缘贫乏,实出寒微。今既蒙金诺之言,何用冰人之讲。”妇人笑道:“你既无钱,我这里有个妈妈姓冯,拉他做个媒证。也不消你行聘,择个吉日良时,招你进来,入门为赘。你意下若何?”这蒋竹山连忙倒身下拜:“娘子就如同学生重生父母,再长爹娘。夙世有缘,三生大幸矣!”一面两个在房中各递了一杯交欢酒,已成其亲事。竹山饮至天晚回家。

妇人这里与冯妈妈商议说:“西门庆如此这般为事,吉凶难保。况且奴家这边没人,不好了一场,险不丧了性命。为今之计,不如把这位先生招他进来,有何不可?”到次日,就使冯妈妈递信过去,择六月十八日大好日子,把蒋竹山倒踏门招进来,成其夫妻。过了三日,妇人凑了三百两银子,与竹山打开两间门面,店内焕然一新。初时往人家看病只是走,后来买了一匹驴儿骑着,在街上往来,不在话下。正是:

\[
一洼死水全无浪,也有春风摆动时。
\]
