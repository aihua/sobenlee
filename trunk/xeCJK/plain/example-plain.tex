
\input xeCJK
\xeCJK@numtrue
\input CJKfntef.sty\relax
\input xeCJK-base.sty\relax
\punctstyle{kaiming}
\makeatletter
\newdimen\paperwidth
\newdimen\paperheight
\paperwidth  = 210 mm
\paperheight = 297 mm

\begingroup
\edef\tempa{\meaning\pdfpagewidth}
\edef\tempb{\string\pdfpagewidth}
\ifx\tempa\tempb
  \endgroup
  \pdfpagewidth  = \paperwidth
  \pdfpageheight = \paperheight
\else
  \endgroup
  \special{papersize=\the\paperwidth,\the\paperheight}
\fi

\hoffset = 25 mm
\voffset = 25 mm
\hsize = \paperwidth
\vsize = \paperheight
\advance \hsize -2\hoffset
\advance \vsize -2\voffset
\advance \hoffset -1 in
\advance \voffset -1 in

\def\getzihaosize#1{%
  \ifnum#11>0 %
    \ifcase#1 %
        42.16 pt% 0
    \or 26.10 pt% 1
    \or 22.08 pt% 2
    \or 16.06 pt% 3
    \or 14.05 pt% 4
    \or 10.54 pt% 5
    \or  7.53 pt% 6
    \or  5.52 pt% 7
    \or  5.02 pt% 8
    \fi
  \else
    \ifcase-#1 %
        36.14 pt% -0
    \or 24.09 pt% -1
    \or 18.07 pt% -2
    \or 15.06 pt% -3
    \or 12.05 pt% -4
    \or  9.03 pt% -5
    \or  6.52 pt% -6
    \fi
  \fi
}

\DeclareFont{EU1/qpl/m/n}  {"TeX Gyre Pagella:mapping=tex-text;"()}
\DeclareFont{EU1/qpl/m/it} {"TeX Gyre Pagella/I:mapping=tex-text;+onum;"()}
\DeclareFont{EU1/qpl/m/sc} {"TeX Gyre Pagella:+smcp;+onum;"()}
\DeclareFont{EU1/qpl/m/sl} {"TeX Gyre Pagella:slant=.167;+onum;"()}
\DeclareFont{EU1/qpl/b/n}  {"TeX Gyre Pagella/B:mapping=tex-text;"()}
\DeclareFont{EU1/qpl/b/it} {"TeX Gyre Pagella/BI:mapping=tex-text;+onum;"()}
\DeclareFont{EU1/qpl/b/sc} {"TeX Gyre Pagella/B:+smcp;+onum;"()}
\DeclareFont{EU1/qpl/b/sl} {"TeX Gyre Pagella/B:slant=.167;+onum;"()}
\DeclareFont{EU1/cmss/m/n} {"[cmunss.otf]:mapping=tex-text;"()}
\DeclareFont{EU1/cmss/m/it}{"[cmunsi.otf]:mapping=tex-text;"()}
\DeclareFont{EU1/cmss/m/sc}{"[cmunss.otf]:+smcp;+onum;"()}
\DeclareFont{EU1/cmss/m/sl}{"[cmunss.otf]:slant=.167;"()}
\DeclareFont{EU1/cmss/b/n} {"[cmunsx.otf]:mapping=tex-text;"()}
\DeclareFont{EU1/cmss/b/it}{"[cmunso.otf]:mapping=tex-text;"()}
\DeclareFont{EU1/cmss/b/sc}{"[cmunsx.otf]:+smcp;+onum;"()}
\DeclareFont{EU1/cmss/b/sl}{"[cmunsx.otf]:slant=.167;"()}
\DeclareFont{EU1/cmtt/m/n} {"[cmuntt.otf]"()}
\DeclareFont{EU1/cmtt/m/it}{"[cmunit.otf]"()}
\DeclareFont{EU1/cmtt/m/sc}{"[cmuntt.otf]:+smcp;+onum;"()}
\DeclareFont{EU1/cmtt/m/sl}{"[cmunst.otf]"()}
\DeclareFont{EU1/cmtt/b/n} {"[cmuntb.otf]"()}
\DeclareFont{EU1/cmtt/b/it}{"[cmuntx.otf]"()}
\DeclareFont{EU1/cmtt/b/sc}{"[cmuntb.otf]:+smcp;+onum;"()}
\DeclareFont{EU1/cmtt/b/sl}{"[cmuntb.otf]:slant=.167;"()}

\DeclareFont{EU1/zhrm/m/n} {"[simsun.ttc]"()}
\DeclareFont{EU1/zhrm/m/it}{"[simkai.ttf]"()}
\DeclareFont{EU1/zhrm/m/sl}{"[simfang.ttf]"()}
\DeclareFont{EU1/zhrm/b/n} {"[simhei.ttf]"()}
\SubstFont{EU1/zhrm/b/it}{EU1/zhrm/b/n}
\SubstFont{EU1/zhrm/b/sl}{EU1/zhrm/b/n}
\DeclareFont{EU1/zhsf/m/n} {"[simyou.ttf]"()}
\SubstFont{EU1/zhsf/m/it}{EU1/zhsf/m/n}
\SubstFont{EU1/zhsf/m/sl}{EU1/zhsf/m/n}
\DeclareFont{EU1/zhsf/b/n} {"[simyou.ttf]:embolden=3"()}
\SubstFont{EU1/zhsf/b/it}{EU1/zhsf/b/n}
\SubstFont{EU1/zhsf/b/sl}{EU1/zhsf/b/n}
\DeclareFont{EU1/zhtt/m/n} {"[simfang.ttf]"()}
\SubstFont{EU1/zhtt/m/it}{EU1/zhtt/m/n}
\SubstFont{EU1/zhtt/m/sl}{EU1/zhtt/m/n}
\DeclareFont{EU1/zhtt/b/n} {"[simfang.ttf]:embolden=4"()}
\SubstFont{EU1/zhtt/b/it}{EU1/zhtt/b/n}
\SubstFont{EU1/zhtt/b/sl}{EU1/zhtt/b/n}
\DeclareFont{EU1/zhbiaosong/m/n}{"[simhei.ttf]"()}
\SubstFont{EU1/zhbiaosong/m/it}{EU1/zhbiaosong/m/n}
\SubstFont{EU1/zhbiaosong/m/sl}{EU1/zhbiaosong/m/n}
\SubstFont{EU1/zhbiaosong/b/n}{EU1/zhbiaosong/m/n}
\SubstFont{EU1/zhbiaosong/b/sl}{EU1/zhbiaosong/b/n}
\SubstFont{EU1/zhbiaosong/b/it}{EU1/zhbiaosong/b/n}

\DeclareFont{EU1/zhrm@FallBack/m/n} {"SimSun-ExtB"()}
\SubstFont{EU1/zhrm@FallBack/m/it}{EU1/zhrm@FallBack/m/n}
\SubstFont{EU1/zhrm@FallBack/m/sl}{EU1/zhrm@FallBack/m/n}
\SubstFont{EU1/zhrm@FallBack/b/n} {EU1/zhrm@FallBack/m/n}
\SubstFont{EU1/zhrm@FallBack/b/it}{EU1/zhrm@FallBack/b/n}
\SubstFont{EU1/zhrm@FallBack/b/sl}{EU1/zhrm@FallBack/b/n}

\def\declarezhfont#1#2{%
  \global\@namedef{xeCJK@font@#2}{#2}%
  \gdef#1{\CJKfamily{#2}}}

\declarezhfont\zhrmfamily{zhrm}
\declarezhfont\zhsffamily{zhsf}
\declarezhfont\zhttfamily{zhtt}
\declarezhfont\biaosong{zhbiaosong}
\declarezhfont\zhrmfamilyfallback{zhrm@FallBack}

\def\apptocmd#1#2{\edef#1{\unexpanded\expandafter{#1#2}}}
\apptocmd\rmfamily\zhrmfamily
\apptocmd\sffamily\zhsffamily
\apptocmd\ttfamily\zhttfamily
\def\upshape{\setfontshape{n}\selectfont}

\def\zihao#1{\setfontsize{\getzihaosize{#1}}\selectfont\ignorespaces}
\def\baselineskipscale{1.5}
\setfontencoding{EU1}
\setrmdefault{qpl}
\setfontsize{\getzihaosize{5}}
\rmfamily

\footline={\hss\zihao{-5}\folio\hss}

\raggedbottom
\topskip = 1 em
\parindent = 2 em

\def\emph#1{{\rmfamily\bfseries\upshape#1}}
\def\@ifblank#1{%
  \if\relax\detokenize{#1}\relax
    \expandafter\@firstoftwo
  \else
    \expandafter\@secondoftwo
  \fi}
\def\newpage{\par\vfil\break}
\def\chapter#1{\@chapter#1 \@nil}
\def\@chapter#1 #2\@nil{%
  \global\sectioncnt=\z@\relax
  \newpage
  \centerline{\rmfamily\bfseries\upshape\zihao{4}\biaosong#1\@ifblank{#2}{}{\quad#2}}%
  \bigbreak}
\def\date{\@ifstar{\s@date}{\@date}}
\def\s@date#1{%
  \medbreak
  \centerline{\rmfamily\mdseries\itshape\zihao{-5}#1}%
  \medbreak}
\def\@date#1{\@@date#1\@nil}
\def\@@date#1.#2.#3\@nil{%
  \s@date{\CJKdigits{#1}年\CJKnumber{#2}月\CJKnumber{#3}日}}
\newcount\sectioncnt
\def\section{\@ifstar{\s@section}{\@section}}
\def\s@section#1{%
  \goodbreak\medskip
  {\rmfamily\bfseries\upshape\zihao{-4}#1}%
  \vadjust{\nobreak}\medskip}
\def\@section#1{%
  \global \advance \sectioncnt \@ne
  \s@section{\CJKnumber{\number\sectioncnt}、#1}}

\newcount\parcnt
\def\parnum{%
  \advance\parcnt\@ne
  (\number\parcnt )}

\begingroup
\def\text{\textrm{ABC 文字}\quad\textit{ABC 文字}\quad\textsl{ABC 文字}\par}
\rmfamily\mdseries\text
\bfseries\text
\sffamily\mdseries\text
\bfseries\text
\ttfamily\mdseries\text
\bfseries\text
\endgroup

\begingroup
{\slshape 斜体 Slanted \bfseries  斜体 Slanted}
\long\def\sometexts{\par
 这是 English 中文 {\itshape Chinese} 中文    \TeX\
  间隔 \textit{Italic} 中文\textbf{字体} a 数学 $b$ 数学 $c$ $d$
\par
 这是English中文{\itshape Chinese}中文\TeX\
 间隔\textit{Italic}中文\textbf{字体}a数学$b$数学$c$ $d$\par
This is an example. 这是一个例子\bigskip
}
\CJKsetecglue{\hskip 0.15em plus 0.05em minus 0.05em}
\sometexts
\CJKsetecglue{ }
\sometexts

\CJKspace 这 是 一行 文字。

\CJKnospace 这 是 一行 文字。
\endgroup


\bigskip\bigskip

\begingroup
\hsize 120mm
\parskip 1ex
\parindent 0em
\long\def\sometexts{xeCJK   改进了中英文间距的处理,并可以避免单个汉字独占一段的最后一行。

xeCJK  改进了中英文间距的处理,并且可以避免单个汉字独占一段的最后一行。

}
\let\xeCJK@i@i\xeCJK@checksingle
\sometexts

\bigskip
不用CJKchecksingle的效果:

\def\xeCJK@i@i{\CJKglue\CJKsymbol}
\sometexts

\endgroup

\bigskip\bigskip

\CJKunderwave{\texttt{CJKfntef} 不正常,会加大字间距,原因不清楚。}

\bigskip\bigskip

\CJKnumber{1234567890}

\CJKdigits{0101011234567890}

\CJKdigits*{0101011234567890}

\begingroup

\bigskip\bigskip

\xeCJKenablefallback

\vbox{\tabskip=1em plus1em \offinterlineskip
\def\tablerule{\noalign{\hrule}}
\halign to .75\hsize {\strut\hfil#&\ttfamily U+#\hfil&\hfil#&
\ttfamily U+#\hfil&\hfil#&\ttfamily U+#\hfil&\hfil#&\ttfamily U+#\hfil\cr
\multispan8\hfil 漢字源𣴑考 \hfil\cr\noalign{\medskip}
𠀀 & 20000 & 𠀁 & 20001 & 𠀂 & 20002 & 𠀃 & 20003 \cr
𠀄 & 20004 & 𠀅 & 20005 & 𠀆 & 20006 & 𠀇 & 20007 \cr
𠀈 & 20008 & 𠀉 & 20009 & 𠀊 & 2000A & 𠀋 & 2000B \cr
𠀌 & 2000C & 𠀍 & 2000D & 𠀎 & 2000E & 𠀏 & 2000F \cr
𠀐 & 20010 & 𠀑 & 20011 & 𠀒 & 20012 & 𠀓 & 20013 \cr
𠀔 & 20014 & 𠀕 & 20015 & 𠀖 & 20016 & 𠀗 & 20017 \cr
𠀘 & 20018 & 𠀙 & 20019 & 𠀚 & 2001A & 𠀛 & 2001B \cr
𠀜 & 2001C & 𠀝 & 2001D & 𠀞 & 2001E & 𠀟 & 2001F \cr}}

\endgroup





\chapter{高举中国特色社会主义伟大旗帜 为夺取全面建设小康社会新胜利而奋斗}
\date{2007.10.15}

同志们!

现在,我代表第十六届中央委员会向大会作报告。

中国共产党第十七次全国代表大会,是在我国改革发展关键阶段召开的一次十分重要的大会。大会的主题是:高举中国特色社会主义伟大旗帜,以邓小平理论和“三个代表”重要思想为指导,深入贯彻落实科学发展观,继续解放思想,坚持改革开放,推动科学发展,促进社会和谐,为夺取全面建设小康社会新胜利而奋斗。

中国特色社会主义伟大旗帜,是当代中国发展进步的旗帜,是全党全国各族人民团结奋斗的旗帜。解放思想是发展中国特色社会主义的一大法宝,改革开放是发展中国特色社会主义的强大动力,科学发展、社会和谐是发展中国特色社会主义的基本要求,全面建设小康社会是党和国家到二〇二〇年的奋斗目标,是全国各族人民的根本利益所在。

当今世界正在发生广泛而深刻的变化,当代中国正在发生广泛而深刻的变革。机遇前所未有,挑战也前所未有,机遇大于挑战。全党必须坚定不移地高举中国特色社会主义伟大旗帜,带领人民从新的历史起点出发,抓住和用好重要战略机遇期,求真务实,锐意进取,继续全面建设小康社会、加快推进社会主义现代化,完成时代赋予的崇高使命。

\section{过去五年的工作}

十六大以来的五年是不平凡的五年。面对复杂多变的国际环境和艰巨繁重的改革发展任务,党带领全国各族人民,高举邓小平理论和“三个代表”重要思想伟大旗帜,战胜各种困难和风险,开创了中国特色社会主义事业新局面,开拓了马克思主义中国化新境界。

十六大确立“三个代表”重要思想的指导地位,作出全面建设小康社会的战略决策。为贯彻十六大精神,中央召开七次全会,分别就深化机构改革、完善社会主义市场经济体制、加强党的执政能力建设、制定“十一五”规划、构建社会主义和谐社会等关系全局的重大问题作出决定和部署,提出并贯彻科学发展观等重大战略思想,推动党和国家工作取得新的重大成就。

经济实力大幅提升。经济保持平稳快速发展,国内生产总值年均增长百分之十以上,经济效益明显提高,财政收入连年显著增加,物价基本稳定。社会主义新农村建设扎实推进,区域发展协调性增强。创新型国家建设进展良好,自主创新能力较大提高。能源、交通、通信等基础设施和重点工程建设成效显著。载人航天飞行成功实现。能源资源节约和生态环境保护取得新进展。“十五”计划胜利完成,“十一五”规划进展顺利。

改革开放取得重大突破。农村综合改革逐步深化,农业税、牧业税、特产税全部取消,支农惠农政策不断加强。国有资产管理体制、国有企业和金融、财税、投资、价格、科技等领域改革取得重大进展。非公有制经济进一步发展。市场体系不断健全,宏观调控继续改善,政府职能加快转变。进出口总额大幅增加,实施“走出去”战略迈出坚实步伐,开放型经济进入新阶段。

人民生活显著改善。城乡居民收入较大增加,家庭财产普遍增多。城乡居民最低生活保障制度初步建立,贫困人口基本生活得到保障。居民消费结构优化,衣食住行用水平不断提高,享有的公共服务明显增强。

民主法制建设取得新进步。政治体制改革稳步推进。人民代表大会制度、中国共产党领导的多党合作和政治协商制度、民族区域自治制度不断完善,基层民主活力增强。人权事业健康发展。爱国统一战线发展壮大。中国特色社会主义法律体系基本形成,依法治国基本方略切实贯彻。行政管理体制、司法体制改革不断深化。

文化建设开创新局面。社会主义核心价值体系建设扎实推进,马克思主义理论研究和建设工程成效明显。思想道德建设广泛开展,全社会文明程度进一步提高。文化体制改革取得重要进展,文化事业和文化产业快速发展,人民精神文化生活更加丰富。全民健身和竞技体育取得新成绩。

社会建设全面展开。各级各类教育迅速发展,农村免费义务教育全面实现。就业规模日益扩大。社会保障体系建设进一步加强。抗击非典取得重大胜利,公共卫生体系和基本医疗服务不断健全,人民健康水平不断提高。社会管理逐步完善,社会大局稳定,人民安居乐业。

国防和军队建设取得历史性成就。中国特色军事变革加速推进,裁减军队员额二十万任务顺利完成,军队革命化、现代化、正规化建设全面加强,履行新世纪新阶段历史使命能力显著提高。

港澳工作和对台工作进一步加强。香港、澳门保持繁荣稳定,与内地经贸关系更加紧密。两岸政党交流成功开启,人员往来和经济文化交流达到新水平。制定反分裂国家法,坚决维护国家主权和领土完整。

全方位外交取得重大进展。坚持独立自主的和平外交政策,各项外交工作积极开展,同各国的交流合作广泛加强,在国际事务中发挥重要建设性作用,为全面建设小康社会争取了良好国际环境。

党的建设新的伟大工程扎实推进。党的执政能力建设和先进性建设深入进行。理论创新和理论武装卓有成效。保持共产党员先进性教育活动取得重大成果。党内民主不断扩大。领导班子和干部队伍建设特别是干部教育培训取得重要进展,人才工作进一步加强,干部人事制度改革和组织制度创新不断深入。党风廉政建设和反腐败斗争成效明显。

在看到成绩的同时,也要清醒认识到,我们的工作与人民的期待还有不小差距,前进中还面临不少困难和问题,突出的是:经济增长的资源环境代价过大;城乡、区域、经济社会发展仍然不平衡;农业稳定发展和农民持续增收难度加大;劳动就业、社会保障、收入分配、教育卫生、居民住房、安全生产、司法和社会治安等方面关系群众切身利益的问题仍然较多,部分低收入群众生活比较困难;思想道德建设有待加强;党的执政能力同新形势新任务不完全适应,对改革发展稳定一些重大实际问题的调查研究不够深入;一些基层党组织软弱涣散;少数党员干部作风不正,形式主义、官僚主义问题比较突出,奢侈浪费、消极腐败现象仍然比较严重。我们要高度重视这些问题,继续认真加以解决。

总起来说,这五年,是改革开放和全面建设小康社会取得重大进展的五年,是我国综合国力大幅提升和人民得到更多实惠的五年,是我国国际地位和影响显著提高的五年,是党的创造力、凝聚力、战斗力明显增强和全党全国各族人民团结更加紧密的五年。实践充分证明,十六大和十六大以来中央作出的各项重大决策是完全正确的。

五年来的成就,是全党全国各族人民共同奋斗的结果。我代表中共中央,向全国各族人民,向各民主党派、各人民团体和各界爱国人士,向香港特别行政区同胞、澳门特别行政区同胞和台湾同胞以及广大侨胞,向一切关心和支持中国现代化建设的各国朋友,表示衷心的感谢!

\section{改革开放的伟大历史进程}

我们即将迎来改革开放三十周年。一九七八年,我们党召开具有重大历史意义的十一届三中全会,开启了改革开放历史新时期。从那时以来,中国共产党人和中国人民以一往无前的进取精神和波澜壮阔的创新实践,谱写了中华民族自强不息、顽强奋进新的壮丽史诗,中国人民的面貌、社会主义中国的面貌、中国共产党的面貌发生了历史性变化。

改革开放是党在新的时代条件下带领人民进行的新的伟大革命,目的就是要解放和发展社会生产力,实现国家现代化,让中国人民富裕起来,振兴伟大的中华民族;就是要推动我国社会主义制度自我完善和发展,赋予社会主义新的生机活力,建设和发展中国特色社会主义;就是要在引领当代中国发展进步中加强和改进党的建设,保持和发展党的先进性,确保党始终走在时代前列。

我们要永远铭记,改革开放伟大事业,是在以毛泽东同志为核心的党的第一代中央领导集体创立毛泽东思想,带领全党全国各族人民建立新中国、取得社会主义革命和建设伟大成就以及艰辛探索社会主义建设规律取得宝贵经验的基础上进行的。新民主主义革命的胜利,社会主义基本制度的建立,为当代中国一切发展进步奠定了根本政治前提和制度基础。

我们要永远铭记,改革开放伟大事业,是以邓小平同志为核心的党的第二代中央领导集体带领全党全国各族人民开创的。面对十年“文化大革命”造成的危难局面,党的第二代中央领导集体坚持解放思想、实事求是,以巨大的政治勇气和理论勇气,科学评价毛泽东同志和毛泽东思想,彻底否定“以阶级斗争为纲”的错误理论和实践,作出把党和国家工作中心转移到经济建设上来、实行改革开放的历史性决策,确立社会主义初级阶段基本路线,吹响走自己的路、建设中国特色社会主义的时代号角,创立邓小平理论,指引全党全国各族人民在改革开放的伟大征程上阔步前进。

我们要永远铭记,改革开放伟大事业,是以江泽民同志为核心的党的第三代中央领导集体带领全党全国各族人民继承、发展并成功推向二十一世纪的。从十三届四中全会到十六大,受命于重大历史关头的党的第三代中央领导集体,高举邓小平理论伟大旗帜,坚持改革开放、与时俱进,在国内外政治风波、经济风险等严峻考验面前,依靠党和人民,捍卫中国特色社会主义,创建社会主义市场经济新体制,开创全面开放新局面,推进党的建设新的伟大工程,创立“三个代表”重要思想,继续引领改革开放的航船沿着正确方向破浪前进。

十六大以来,我们以邓小平理论和“三个代表”重要思想为指导,顺应国内外形势发展变化,抓住重要战略机遇期,发扬求真务实、开拓进取精神,坚持理论创新和实践创新,着力推动科学发展、促进社会和谐,完善社会主义市场经济体制,在全面建设小康社会实践中坚定不移地把改革开放伟大事业继续推向前进。

新时期最鲜明的特点是改革开放。从农村到城市、从经济领域到其他各个领域,全面改革的进程势不可挡地展开了;从沿海到沿江沿边,从东部到中西部,对外开放的大门毅然决然地打开了。这场历史上从未有过的大改革大开放,极大地调动了亿万人民的积极性,使我国成功实现了从高度集中的计划经济体制到充满活力的社会主义市场经济体制、从封闭半封闭到全方位开放的伟大历史转折。今天,一个面向现代化、面向世界、面向未来的社会主义中国巍然屹立在世界东方。

新时期最显著的成就是快速发展。我们党实施现代化建设“三步走”战略,带领人民艰苦奋斗,推动我国以世界上少有的速度持续快速发展起来。我国经济从一度濒于崩溃的边缘发展到总量跃至世界第四、进出口总额位居世界第三,人民生活从温饱不足发展到总体小康,农村贫困人口从两亿五千多万减少到两千多万,政治建设、文化建设、社会建设取得举世瞩目的成就。中国的发展,不仅使中国人民稳定地走上了富裕安康的广阔道路,而且为世界经济发展和人类文明进步作出了重大贡献。

新时期最突出的标志是与时俱进。我们党坚持马克思主义的思想路线,不断探索和回答什么是社会主义、怎样建设社会主义,建设什么样的党、怎样建设党,实现什么样的发展、怎样发展等重大理论和实际问题,不断推进马克思主义中国化,坚持并丰富党的基本理论、基本路线、基本纲领、基本经验。社会主义和马克思主义在中国大地上焕发出勃勃生机,给人民带来更多福祉,使中华民族大踏步赶上时代前进潮流、迎来伟大复兴的光明前景。

事实雄辩地证明,改革开放是决定当代中国命运的关键抉择,是发展中国特色社会主义、实现中华民族伟大复兴的必由之路;只有社会主义才能救中国,只有改革开放才能发展中国、发展社会主义、发展马克思主义。

改革开放作为一场新的伟大革命,不可能一帆风顺,也不可能一蹴而就。最根本的是,改革开放符合党心民心、顺应时代潮流,方向和道路是完全正确的,成效和功绩不容否定,停顿和倒退没有出路。

在改革开放的历史进程中,我们党把坚持马克思主义基本原理同推进马克思主义中国化结合起来,把坚持四项基本原则同坚持改革开放结合起来,把尊重人民首创精神同加强和改善党的领导结合起来,把坚持社会主义基本制度同发展市场经济结合起来,把推动经济基础变革同推动上层建筑改革结合起来,把发展社会生产力同提高全民族文明素质结合起来,把提高效率同促进社会公平结合起来,把坚持独立自主同参与经济全球化结合起来,把促进改革发展同保持社会稳定结合起来,把推进中国特色社会主义伟大事业同推进党的建设新的伟大工程结合起来,取得了我们这样一个十几亿人口的发展中大国摆脱贫困、加快实现现代化、巩固和发展社会主义的宝贵经验。

改革开放以来我们取得一切成绩和进步的根本原因,归结起来就是:开辟了中国特色社会主义道路,形成了中国特色社会主义理论体系。高举中国特色社会主义伟大旗帜,最根本的就是要坚持这条道路和这个理论体系。

中国特色社会主义道路,就是在中国共产党领导下,立足基本国情,以经济建设为中心,坚持四项基本原则,坚持改革开放,解放和发展社会生产力,巩固和完善社会主义制度,建设社会主义市场经济、社会主义民主政治、社会主义先进文化、社会主义和谐社会,建设富强民主文明和谐的社会主义现代化国家。中国特色社会主义道路之所以完全正确、之所以能够引领中国发展进步,关键在于我们既坚持了科学社会主义的基本原则,又根据我国实际和时代特征赋予其鲜明的中国特色。在当代中国,坚持中国特色社会主义道路,就是真正坚持社会主义。

中国特色社会主义理论体系,就是包括邓小平理论、“三个代表”重要思想以及科学发展观等重大战略思想在内的科学理论体系。这个理论体系,坚持和发展了马克思列宁主义、毛泽东思想,凝结了几代中国共产党人带领人民不懈探索实践的智慧和心血,是马克思主义中国化最新成果,是党最可宝贵的政治和精神财富,是全国各族人民团结奋斗的共同思想基础。中国特色社会主义理论体系是不断发展的开放的理论体系。《共产党宣言》发表以来近一百六十年的实践证明,马克思主义只有与本国国情相结合、与时代发展同进步、与人民群众共命运,才能焕发出强大的生命力、创造力、感召力。在当代中国,坚持中国特色社会主义理论体系,就是真正坚持马克思主义。

实践永无止境,创新永无止境。全党同志要倍加珍惜、长期坚持和不断发展党历经艰辛开创的中国特色社会主义道路和中国特色社会主义理论体系,坚持解放思想、实事求是、与时俱进,勇于变革、勇于创新,永不僵化、永不停滞,不为任何风险所惧,不被任何干扰所惑,使中国特色社会主义道路越走越宽广,让当代中国马克思主义放射出更加灿烂的真理光芒。

\section{深入贯彻落实科学发展观}

在新的发展阶段继续全面建设小康社会、发展中国特色社会主义,必须坚持以邓小平理论和“三个代表”重要思想为指导,深入贯彻落实科学发展观。

科学发展观,是对党的三代中央领导集体关于发展的重要思想的继承和发展,是马克思主义关于发展的世界观和方法论的集中体现,是同马克思列宁主义、毛泽东思想、邓小平理论和“三个代表”重要思想既一脉相承又与时俱进的科学理论,是我国经济社会发展的重要指导方针,是发展中国特色社会主义必须坚持和贯彻的重大战略思想。

科学发展观,是立足社会主义初级阶段基本国情,总结我国发展实践,借鉴国外发展经验,适应新的发展要求提出来的。进入新世纪新阶段,我国发展呈现一系列新的阶段性特征,主要是:经济实力显著增强,同时生产力水平总体上还不高,自主创新能力还不强,长期形成的结构性矛盾和粗放型增长方式尚未根本改变;社会主义市场经济体制初步建立,同时影响发展的体制机制障碍依然存在,改革攻坚面临深层次矛盾和问题;人民生活总体上达到小康水平,同时收入分配差距拉大趋势还未根本扭转,城乡贫困人口和低收入人口还有相当数量,统筹兼顾各方面利益难度加大;协调发展取得显著成绩,同时农业基础薄弱、农村发展滞后的局面尚未改变,缩小城乡、区域发展差距和促进经济社会协调发展任务艰巨;社会主义民主政治不断发展、依法治国基本方略扎实贯彻,同时民主法制建设与扩大人民民主和经济社会发展的要求还不完全适应,政治体制改革需要继续深化;社会主义文化更加繁荣,同时人民精神文化需求日趋旺盛,人们思想活动的独立性、选择性、多变性、差异性明显增强,对发展社会主义先进文化提出了更高要求;社会活力显著增强,同时社会结构、社会组织形式、社会利益格局发生深刻变化,社会建设和管理面临诸多新课题;对外开放日益扩大,同时面临的国际竞争日趋激烈,发达国家在经济科技上占优势的压力长期存在,可以预见和难以预见的风险增多,统筹国内发展和对外开放要求更高。

这些情况表明,经过新中国成立以来特别是改革开放以来的不懈努力,我国取得了举世瞩目的发展成就,从生产力到生产关系、从经济基础到上层建筑都发生了意义深远的重大变化,但我国仍处于并将长期处于社会主义初级阶段的基本国情没有变,人民日益增长的物质文化需要同落后的社会生产之间的矛盾这一社会主要矛盾没有变。当前我国发展的阶段性特征,是社会主义初级阶段基本国情在新世纪新阶段的具体表现。强调认清社会主义初级阶段基本国情,不是要妄自菲薄、自甘落后,也不是要脱离实际、急于求成,而是要坚持把它作为推进改革、谋划发展的根本依据。我们必须始终保持清醒头脑,立足社会主义初级阶段这个最大的实际,科学分析我国全面参与经济全球化的新机遇新挑战,全面认识工业化、信息化、城镇化、市场化、国际化深入发展的新形势新任务,深刻把握我国发展面临的新课题新矛盾,更加自觉地走科学发展道路,奋力开拓中国特色社会主义更为广阔的发展前景。

科学发展观,第一要义是发展,核心是以人为本,基本要求是全面协调可持续,根本方法是统筹兼顾。

——必须坚持把发展作为党执政兴国的第一要务。发展,对于全面建设小康社会、加快推进社会主义现代化,具有决定性意义。要牢牢扭住经济建设这个中心,坚持聚精会神搞建设、一心一意谋发展,不断解放和发展社会生产力。更好实施科教兴国战略、人才强国战略、可持续发展战略,着力把握发展规律、创新发展理念、转变发展方式、破解发展难题,提高发展质量和效益,实现又好又快发展,为发展中国特色社会主义打下坚实基础。努力实现以人为本、全面协调可持续的科学发展,实现各方面事业有机统一、社会成员团结和睦的和谐发展,实现既通过维护世界和平发展自己、又通过自身发展维护世界和平的和平发展。

——必须坚持以人为本。全心全意为人民服务是党的根本宗旨,党的一切奋斗和工作都是为了造福人民。要始终把实现好、维护好、发展好最广大人民的根本利益作为党和国家一切工作的出发点和落脚点,尊重人民主体地位,发挥人民首创精神,保障人民各项权益,走共同富裕道路,促进人的全面发展,做到发展为了人民、发展依靠人民、发展成果由人民共享。

——必须坚持全面协调可持续发展。要按照中国特色社会主义事业总体布局,全面推进经济建设、政治建设、文化建设、社会建设,促进现代化建设各个环节、各个方面相协调,促进生产关系与生产力、上层建筑与经济基础相协调。坚持生产发展、生活富裕、生态良好的文明发展道路,建设资源节约型、环境友好型社会,实现速度和结构质量效益相统一、经济发展与人口资源环境相协调,使人民在良好生态环境中生产生活,实现经济社会永续发展。

——必须坚持统筹兼顾。要正确认识和妥善处理中国特色社会主义事业中的重大关系,统筹城乡发展、区域发展、经济社会发展、人与自然和谐发展、国内发展和对外开放,统筹中央和地方关系,统筹个人利益和集体利益、局部利益和整体利益、当前利益和长远利益,充分调动各方面积极性。统筹国内国际两个大局,树立世界眼光,加强战略思维,善于从国际形势发展变化中把握发展机遇、应对风险挑战,营造良好国际环境。既要总揽全局、统筹规划,又要抓住牵动全局的主要工作、事关群众利益的突出问题,着力推进、重点突破。

深入贯彻落实科学发展观,要求我们始终坚持“一个中心、两个基本点”的基本路线。党的基本路线是党和国家的生命线,是实现科学发展的政治保证。以经济建设为中心是兴国之要,是我们党、我们国家兴旺发达和长治久安的根本要求;四项基本原则是立国之本,是我们党、我们国家生存发展的政治基石;改革开放是强国之路,是我们党、我们国家发展进步的活力源泉。要坚持把以经济建设为中心同四项基本原则、改革开放这两个基本点统一于发展中国特色社会主义的伟大实践,任何时候都决不能动摇。

深入贯彻落实科学发展观,要求我们积极构建社会主义和谐社会。社会和谐是中国特色社会主义的本质属性。科学发展和社会和谐是内在统一的。没有科学发展就没有社会和谐,没有社会和谐也难以实现科学发展。构建社会主义和谐社会是贯穿中国特色社会主义事业全过程的长期历史任务,是在发展的基础上正确处理各种社会矛盾的历史过程和社会结果。要通过发展增加社会物质财富、不断改善人民生活,又要通过发展保障社会公平正义、不断促进社会和谐。实现社会公平正义是中国共产党人的一贯主张,是发展中国特色社会主义的重大任务。要按照民主法治、公平正义、诚信友爱、充满活力、安定有序、人与自然和谐相处的总要求和共同建设、共同享有的原则,着力解决人民最关心、最直接、最现实的利益问题,努力形成全体人民各尽其能、各得其所而又和谐相处的局面,为发展提供良好社会环境。

深入贯彻落实科学发展观,要求我们继续深化改革开放。要把改革创新精神贯彻到治国理政各个环节,毫不动摇地坚持改革方向,提高改革决策的科学性,增强改革措施的协调性。要完善社会主义市场经济体制,推进各方面体制改革创新,加快重要领域和关键环节改革步伐,全面提高开放水平,着力构建充满活力、富有效率、更加开放、有利于科学发展的体制机制,为发展中国特色社会主义提供强大动力和体制保障。要坚持把改善人民生活作为正确处理改革发展稳定关系的结合点,使改革始终得到人民拥护和支持。

深入贯彻落实科学发展观,要求我们切实加强和改进党的建设。要站在完成党执政兴国使命的高度,把提高党的执政能力、保持和发展党的先进性,体现到领导科学发展、促进社会和谐上来,落实到引领中国发展进步、更好代表和实现最广大人民的根本利益上来,使党的工作和党的建设更加符合科学发展观的要求,为科学发展提供可靠的政治和组织保障。

全党同志要全面把握科学发展观的科学内涵和精神实质,增强贯彻落实科学发展观的自觉性和坚定性,着力转变不适应不符合科学发展观的思想观念,着力解决影响和制约科学发展的突出问题,把全社会的发展积极性引导到科学发展上来,把科学发展观贯彻落实到经济社会发展各个方面。

\section{实现全面建设小康社会奋斗目标的新要求}

我们已经朝着十六大确立的全面建设小康社会的目标迈出了坚实步伐,今后要继续努力奋斗,确保到二〇二〇年实现全面建成小康社会的奋斗目标。

我们必须适应国内外形势的新变化,顺应各族人民过上更好生活的新期待,把握经济社会发展趋势和规律,坚持中国特色社会主义经济建设、政治建设、文化建设、社会建设的基本目标和基本政策构成的基本纲领,在十六大确立的全面建设小康社会目标的基础上对我国发展提出新的更高要求。

——增强发展协调性,努力实现经济又好又快发展。转变发展方式取得重大进展,在优化结构、提高效益、降低消耗、保护环境的基础上,实现人均国内生产总值到二〇二〇年比二〇〇〇年翻两番。社会主义市场经济体制更加完善。自主创新能力显著提高,科技进步对经济增长的贡献率大幅上升,进入创新型国家行列。居民消费率稳步提高,形成消费、投资、出口协调拉动的增长格局。城乡、区域协调互动发展机制和主体功能区布局基本形成。社会主义新农村建设取得重大进展。城镇人口比重明显增加。

——扩大社会主义民主,更好保障人民权益和社会公平正义。公民政治参与有序扩大。依法治国基本方略深入落实,全社会法制观念进一步增强,法治政府建设取得新成效。基层民主制度更加完善。政府提供基本公共服务能力显著增强。

——加强文化建设,明显提高全民族文明素质。社会主义核心价值体系深入人心,良好思想道德风尚进一步弘扬。覆盖全社会的公共文化服务体系基本建立,文化产业占国民经济比重明显提高、国际竞争力显著增强,适应人民需要的文化产品更加丰富。

——加快发展社会事业,全面改善人民生活。现代国民教育体系更加完善,终身教育体系基本形成,全民受教育程度和创新人才培养水平明显提高。社会就业更加充分。覆盖城乡居民的社会保障体系基本建立,人人享有基本生活保障。合理有序的收入分配格局基本形成,中等收入者占多数,绝对贫困现象基本消除。人人享有基本医疗卫生服务。社会管理体系更加健全。

——建设生态文明,基本形成节约能源资源和保护生态环境的产业结构、增长方式、消费模式。循环经济形成较大规模,可再生能源比重显著上升。主要污染物排放得到有效控制,生态环境质量明显改善。生态文明观念在全社会牢固树立。

到二〇二〇年全面建设小康社会目标实现之时,我们这个历史悠久的文明古国和发展中社会主义大国,将成为工业化基本实现、综合国力显著增强、国内市场总体规模位居世界前列的国家,成为人民富裕程度普遍提高、生活质量明显改善、生态环境良好的国家,成为人民享有更加充分民主权利、具有更高文明素质和精神追求的国家,成为各方面制度更加完善、社会更加充满活力而又安定团结的国家,成为对外更加开放、更加具有亲和力、为人类文明作出更大贡献的国家。

今后五年是全面建设小康社会的关键时期。我们要坚定信心,埋头苦干,为全面建成惠及十几亿人口的更高水平的小康社会打下更加牢固的基础。

\section{促进国民经济又好又快发展}

实现未来经济发展目标,关键要在加快转变经济发展方式、完善社会主义市场经济体制方面取得重大进展。要大力推进经济结构战略性调整,更加注重提高自主创新能力、提高节能环保水平、提高经济整体素质和国际竞争力。要深化对社会主义市场经济规律的认识,从制度上更好发挥市场在资源配置中的基础性作用,形成有利于科学发展的宏观调控体系。

(一)提高自主创新能力,建设创新型国家。这是国家发展战略的核心,是提高综合国力的关键。要坚持走中国特色自主创新道路,把增强自主创新能力贯彻到现代化建设各个方面。认真落实国家中长期科学和技术发展规划纲要,加大对自主创新投入,着力突破制约经济社会发展的关键技术。加快建设国家创新体系,支持基础研究、前沿技术研究、社会公益性技术研究。加快建立以企业为主体、市场为导向、产学研相结合的技术创新体系,引导和支持创新要素向企业集聚,促进科技成果向现实生产力转化。深化科技管理体制改革,优化科技资源配置,完善鼓励技术创新和科技成果产业化的法制保障、政策体系、激励机制、市场环境。实施知识产权战略。充分利用国际科技资源。进一步营造鼓励创新的环境,努力造就世界一流科学家和科技领军人才,注重培养一线的创新人才,使全社会创新智慧竞相迸发、各方面创新人才大量涌现。

(二)加快转变经济发展方式,推动产业结构优化升级。这是关系国民经济全局紧迫而重大的战略任务。要坚持走中国特色新型工业化道路,坚持扩大国内需求特别是消费需求的方针,促进经济增长由主要依靠投资、出口拉动向依靠消费、投资、出口协调拉动转变,由主要依靠第二产业带动向依靠第一、第二、第三产业协同带动转变,由主要依靠增加物质资源消耗向主要依靠科技进步、劳动者素质提高、管理创新转变。发展现代产业体系,大力推进信息化与工业化融合,促进工业由大变强,振兴装备制造业,淘汰落后生产能力;提升高新技术产业,发展信息、生物、新材料、航空航天、海洋等产业;发展现代服务业,提高服务业比重和水平;加强基础产业基础设施建设,加快发展现代能源产业和综合运输体系。确保产品质量和安全。鼓励发展具有国际竞争力的大企业集团。

(三)统筹城乡发展,推进社会主义新农村建设。解决好农业、农村、农民问题,事关全面建设小康社会大局,必须始终作为全党工作的重中之重。要加强农业基础地位,走中国特色农业现代化道路,建立以工促农、以城带乡长效机制,形成城乡经济社会发展一体化新格局。坚持把发展现代农业、繁荣农村经济作为首要任务,加强农村基础设施建设,健全农村市场和农业服务体系。加大支农惠农政策力度,严格保护耕地,增加农业投入,促进农业科技进步,增强农业综合生产能力,确保国家粮食安全。加强动植物疫病防控,提高农产品质量安全水平。以促进农民增收为核心,发展乡镇企业,壮大县域经济,多渠道转移农民就业。提高扶贫开发水平。深化农村综合改革,推进农村金融体制改革和创新,改革集体林权制度。坚持农村基本经营制度,稳定和完善土地承包关系,按照依法自愿有偿原则,健全土地承包经营权流转市场,有条件的地方可以发展多种形式的适度规模经营。探索集体经济有效实现形式,发展农民专业合作组织,支持农业产业化经营和龙头企业发展。培育有文化、懂技术、会经营的新型农民,发挥亿万农民建设新农村的主体作用。

(四)加强能源资源节约和生态环境保护,增强可持续发展能力。坚持节约资源和保护环境的基本国策,关系人民群众切身利益和中华民族生存发展。必须把建设资源节约型、环境友好型社会放在工业化、现代化发展战略的突出位置,落实到每个单位、每个家庭。要完善有利于节约能源资源和保护生态环境的法律和政策,加快形成可持续发展体制机制。落实节能减排工作责任制。开发和推广节约、替代、循环利用和治理污染的先进适用技术,发展清洁能源和可再生能源,保护土地和水资源,建设科学合理的能源资源利用体系,提高能源资源利用效率。发展环保产业。加大节能环保投入,重点加强水、大气、土壤等污染防治,改善城乡人居环境。加强水利、林业、草原建设,加强荒漠化石漠化治理,促进生态修复。加强应对气候变化能力建设,为保护全球气候作出新贡献。

(五)推动区域协调发展,优化国土开发格局。缩小区域发展差距,必须注重实现基本公共服务均等化,引导生产要素跨区域合理流动。要继续实施区域发展总体战略,深入推进西部大开发,全面振兴东北地区等老工业基地,大力促进中部地区崛起,积极支持东部地区率先发展。加强国土规划,按照形成主体功能区的要求,完善区域政策,调整经济布局。遵循市场经济规律,突破行政区划界限,形成若干带动力强、联系紧密的经济圈和经济带。重大项目布局要充分考虑支持中西部发展,鼓励东部地区带动和帮助中西部地区发展。加大对革命老区、民族地区、边疆地区、贫困地区发展扶持力度。帮助资源枯竭地区实现经济转型。更好发挥经济特区、上海浦东新区、天津滨海新区在改革开放和自主创新中的重要作用。走中国特色城镇化道路,按照统筹城乡、布局合理、节约土地、功能完善、以大带小的原则,促进大中小城市和小城镇协调发展。以增强综合承载能力为重点,以特大城市为依托,形成辐射作用大的城市群,培育新的经济增长极。

(六)完善基本经济制度,健全现代市场体系。坚持和完善公有制为主体、多种所有制经济共同发展的基本经济制度,毫不动摇地巩固和发展公有制经济,毫不动摇地鼓励、支持、引导非公有制经济发展,坚持平等保护物权,形成各种所有制经济平等竞争、相互促进新格局。深化国有企业公司制股份制改革,健全现代企业制度,优化国有经济布局和结构,增强国有经济活力、控制力、影响力。深化垄断行业改革,引入竞争机制,加强政府监管和社会监督。加快建设国有资本经营预算制度。完善各类国有资产管理体制和制度。推进集体企业改革,发展多种形式的集体经济、合作经济。推进公平准入,改善融资条件,破除体制障碍,促进个体、私营经济和中小企业发展。以现代产权制度为基础,发展混合所有制经济。加快形成统一开放竞争有序的现代市场体系,发展各类生产要素市场,完善反映市场供求关系、资源稀缺程度、环境损害成本的生产要素和资源价格形成机制,规范发展行业协会和市场中介组织,健全社会信用体系。

(七)深化财税、金融等体制改革,完善宏观调控体系。围绕推进基本公共服务均等化和主体功能区建设,完善公共财政体系。深化预算制度改革,强化预算管理和监督,健全中央和地方财力与事权相匹配的体制,加快形成统一规范透明的财政转移支付制度,提高一般性转移支付规模和比例,加大公共服务领域投入。完善省以下财政体制,增强基层政府提供公共服务能力。实行有利于科学发展的财税制度,建立健全资源有偿使用制度和生态环境补偿机制。推进金融体制改革,发展各类金融市场,形成多种所有制和多种经营形式、结构合理、功能完善、高效安全的现代金融体系。提高银行业、证券业、保险业竞争力。优化资本市场结构,多渠道提高直接融资比重。加强和改进金融监管,防范和化解金融风险。完善人民币汇率形成机制,逐步实现资本项目可兑换。深化投资体制改革,健全和严格市场准入制度。完善国家规划体系。发挥国家发展规划、计划、产业政策在宏观调控中的导向作用,综合运用财政、货币政策,提高宏观调控水平。

(八)拓展对外开放广度和深度,提高开放型经济水平。坚持对外开放的基本国策,把“引进来”和“走出去”更好结合起来,扩大开放领域,优化开放结构,提高开放质量,完善内外联动、互利共赢、安全高效的开放型经济体系,形成经济全球化条件下参与国际经济合作和竞争新优势。深化沿海开放,加快内地开放,提升沿边开放,实现对内对外开放相互促进。加快转变外贸增长方式,立足以质取胜,调整进出口结构,促进加工贸易转型升级,大力发展服务贸易。创新利用外资方式,优化利用外资结构,发挥利用外资在推动自主创新、产业升级、区域协调发展等方面的积极作用。创新对外投资和合作方式,支持企业在研发、生产、销售等方面开展国际化经营,加快培育我国的跨国公司和国际知名品牌。积极开展国际能源资源互利合作。实施自由贸易区战略,加强双边多边经贸合作。采取综合措施促进国际收支基本平衡。注重防范国际经济风险。

实现国民经济又好又快发展,必将进一步增强我国经济实力,彰显社会主义市场经济的强大生机活力。

\section{坚定不移发展社会主义民主政治}

人民民主是社会主义的生命。发展社会主义民主政治是我们党始终不渝的奋斗目标。改革开放以来,我们积极稳妥推进政治体制改革,我国社会主义民主政治展现出更加旺盛的生命力。政治体制改革作为我国全面改革的重要组成部分,必须随着经济社会发展而不断深化,与人民政治参与积极性不断提高相适应。要坚持中国特色社会主义政治发展道路,坚持党的领导、人民当家作主、依法治国有机统一,坚持和完善人民代表大会制度、中国共产党领导的多党合作和政治协商制度、民族区域自治制度以及基层群众自治制度,不断推进社会主义政治制度自我完善和发展。

深化政治体制改革,必须坚持正确政治方向,以保证人民当家作主为根本,以增强党和国家活力、调动人民积极性为目标,扩大社会主义民主,建设社会主义法治国家,发展社会主义政治文明。要坚持党总揽全局、协调各方的领导核心作用,提高党科学执政、民主执政、依法执政水平,保证党领导人民有效治理国家;坚持国家一切权力属于人民,从各个层次、各个领域扩大公民有序政治参与,最广泛地动员和组织人民依法管理国家事务和社会事务、管理经济和文化事业;坚持依法治国基本方略,树立社会主义法治理念,实现国家各项工作法治化,保障公民合法权益;坚持社会主义政治制度的特点和优势,推进社会主义民主政治制度化、规范化、程序化,为党和国家长治久安提供政治和法律制度保障。

(一)扩大人民民主,保证人民当家作主。人民当家作主是社会主义民主政治的本质和核心。要健全民主制度,丰富民主形式,拓宽民主渠道,依法实行民主选举、民主决策、民主管理、民主监督,保障人民的知情权、参与权、表达权、监督权。支持人民代表大会依法履行职能,善于使党的主张通过法定程序成为国家意志;保障人大代表依法行使职权,密切人大代表同人民的联系,建议逐步实行城乡按相同人口比例选举人大代表;加强人大常委会制度建设,优化组成人员知识结构和年龄结构。支持人民政协围绕团结和民主两大主题履行职能,推进政治协商、民主监督、参政议政制度建设;把政治协商纳入决策程序,完善民主监督机制,提高参政议政实效;加强政协自身建设,发挥协调关系、汇聚力量、建言献策、服务大局的重要作用。坚持各民族一律平等,保证民族自治地方依法行使自治权。推进决策科学化、民主化,完善决策信息和智力支持系统,增强决策透明度和公众参与度,制定与群众利益密切相关的法律法规和公共政策原则上要公开听取意见。加强公民意识教育,树立社会主义民主法治、自由平等、公平正义理念。支持工会、共青团、妇联等人民团体依照法律和各自章程开展工作,参与社会管理和公共服务,维护群众合法权益。

(二)发展基层民主,保障人民享有更多更切实的民主权利。人民依法直接行使民主权利,管理基层公共事务和公益事业,实行自我管理、自我服务、自我教育、自我监督,对干部实行民主监督,是人民当家作主最有效、最广泛的途径,必须作为发展社会主义民主政治的基础性工程重点推进。要健全基层党组织领导的充满活力的基层群众自治机制,扩大基层群众自治范围,完善民主管理制度,把城乡社区建设成为管理有序、服务完善、文明祥和的社会生活共同体。全心全意依靠工人阶级,完善以职工代表大会为基本形式的企事业单位民主管理制度,推进厂务公开,支持职工参与管理,维护职工合法权益。深化乡镇机构改革,加强基层政权建设,完善政务公开、村务公开等制度,实现政府行政管理与基层群众自治有效衔接和良性互动。发挥社会组织在扩大群众参与、反映群众诉求方面的积极作用,增强社会自治功能。

(三)全面落实依法治国基本方略,加快建设社会主义法治国家。依法治国是社会主义民主政治的基本要求。要坚持科学立法、民主立法,完善中国特色社会主义法律体系。加强宪法和法律实施,坚持公民在法律面前一律平等,维护社会公平正义,维护社会主义法制的统一、尊严、权威。推进依法行政。深化司法体制改革,优化司法职权配置,规范司法行为,建设公正高效权威的社会主义司法制度,保证审判机关、检察机关依法独立公正地行使审判权、检察权。加强政法队伍建设,做到严格、公正、文明执法。深入开展法制宣传教育,弘扬法治精神,形成自觉学法守法用法的社会氛围。尊重和保障人权,依法保证全体社会成员平等参与、平等发展的权利。各级党组织和全体党员要自觉在宪法和法律范围内活动,带头维护宪法和法律的权威。

(四)壮大爱国统一战线,团结一切可以团结的力量。促进政党关系、民族关系、宗教关系、阶层关系、海内外同胞关系的和谐,对于增进团结、凝聚力量具有不可替代的作用。要贯彻长期共存、互相监督、肝胆相照、荣辱与共的方针,加强同民主党派合作共事,支持民主党派和无党派人士更好履行参政议政、民主监督职能,选拔和推荐更多优秀党外干部担任领导职务。牢牢把握各民族共同团结奋斗、共同繁荣发展的主题,保障少数民族合法权益,巩固和发展平等团结互助和谐的社会主义民族关系。全面贯彻党的宗教工作基本方针,发挥宗教界人士和信教群众在促进经济社会发展中的积极作用。鼓励新的社会阶层人士积极投身中国特色社会主义建设。认真贯彻党的侨务政策,支持海外侨胞、归侨侨眷关心和参与祖国现代化建设与和平统一大业。

(五)加快行政管理体制改革,建设服务型政府。行政管理体制改革是深化改革的重要环节。要抓紧制定行政管理体制改革总体方案,着力转变职能、理顺关系、优化结构、提高效能,形成权责一致、分工合理、决策科学、执行顺畅、监督有力的行政管理体制。健全政府职责体系,完善公共服务体系,推行电子政务,强化社会管理和公共服务。加快推进政企分开、政资分开、政事分开、政府与市场中介组织分开,规范行政行为,加强行政执法部门建设,减少和规范行政审批,减少政府对微观经济运行的干预。规范垂直管理部门和地方政府的关系。加大机构整合力度,探索实行职能有机统一的大部门体制,健全部门间协调配合机制。精简和规范各类议事协调机构及其办事机构,减少行政层次,降低行政成本,着力解决机构重叠、职责交叉、政出多门问题。统筹党委、政府和人大、政协机构设置,减少领导职数,严格控制编制。加快推进事业单位分类改革。

(六)完善制约和监督机制,保证人民赋予的权力始终用来为人民谋利益。确保权力正确行使,必须让权力在阳光下运行。要坚持用制度管权、管事、管人,建立健全决策权、执行权、监督权既相互制约又相互协调的权力结构和运行机制。健全组织法制和程序规则,保证国家机关按照法定权限和程序行使权力、履行职责。完善各类公开办事制度,提高政府工作透明度和公信力。重点加强对领导干部特别是主要领导干部、人财物管理使用、关键岗位的监督,健全质询、问责、经济责任审计、引咎辞职、罢免等制度。落实党内监督条例,加强民主监督,发挥好舆论监督作用,增强监督合力和实效。

社会主义愈发展,民主也愈发展。在发展中国特色社会主义的历史进程中,中国共产党人和中国人民一定能够不断发展具有强大生命力的社会主义民主政治。

\section{推动社会主义文化大发展大繁荣}

当今时代,文化越来越成为民族凝聚力和创造力的重要源泉、越来越成为综合国力竞争的重要因素,丰富精神文化生活越来越成为我国人民的热切愿望。要坚持社会主义先进文化前进方向,兴起社会主义文化建设新高潮,激发全民族文化创造活力,提高国家文化软实力,使人民基本文化权益得到更好保障,使社会文化生活更加丰富多彩,使人民精神风貌更加昂扬向上。

(一)建设社会主义核心价值体系,增强社会主义意识形态的吸引力和凝聚力。社会主义核心价值体系是社会主义意识形态的本质体现。要巩固马克思主义指导地位,坚持不懈地用马克思主义中国化最新成果武装全党、教育人民,用中国特色社会主义共同理想凝聚力量,用以爱国主义为核心的民族精神和以改革创新为核心的时代精神鼓舞斗志,用社会主义荣辱观引领风尚,巩固全党全国各族人民团结奋斗的共同思想基础。大力推进理论创新,不断赋予当代中国马克思主义鲜明的实践特色、民族特色、时代特色。开展中国特色社会主义理论体系宣传普及活动,推动当代中国马克思主义大众化。推进马克思主义理论研究和建设工程,深入回答重大理论和实际问题,培养造就一批马克思主义理论家特别是中青年理论家。切实把社会主义核心价值体系融入国民教育和精神文明建设全过程,转化为人民的自觉追求。积极探索用社会主义核心价值体系引领社会思潮的有效途径,主动做好意识形态工作,既尊重差异、包容多样,又有力抵制各种错误和腐朽思想的影响。繁荣发展哲学社会科学,推进学科体系、学术观点、科研方法创新,鼓励哲学社会科学界为党和人民事业发挥思想库作用,推动我国哲学社会科学优秀成果和优秀人才走向世界。

(二)建设和谐文化,培育文明风尚。和谐文化是全体人民团结进步的重要精神支撑。要积极发展新闻出版、广播影视、文学艺术事业,坚持正确导向,弘扬社会正气。重视城乡、区域文化协调发展,着力丰富农村、偏远地区、进城务工人员的精神文化生活。加强网络文化建设和管理,营造良好网络环境。大力弘扬爱国主义、集体主义、社会主义思想,以增强诚信意识为重点,加强社会公德、职业道德、家庭美德、个人品德建设,发挥道德模范榜样作用,引导人们自觉履行法定义务、社会责任、家庭责任。加强和改进思想政治工作,注重人文关怀和心理疏导,用正确方式处理人际关系。动员社会各方面共同做好青少年思想道德教育工作,为青少年健康成长创造良好社会环境。深入开展群众性精神文明创建活动,完善社会志愿服务体系,形成男女平等、尊老爱幼、互爱互助、见义勇为的社会风尚。弘扬科学精神,普及科学知识。广泛开展全民健身运动。办好二〇〇八年奥运会、残奥会和二〇一〇年世博会。

(三)弘扬中华文化,建设中华民族共有精神家园。中华文化是中华民族生生不息、团结奋进的不竭动力。要全面认识祖国传统文化,取其精华,去其糟粕,使之与当代社会相适应、与现代文明相协调,保持民族性,体现时代性。加强中华优秀文化传统教育,运用现代科技手段开发利用民族文化丰厚资源。加强对各民族文化的挖掘和保护,重视文物和非物质文化遗产保护,做好文化典籍整理工作。加强对外文化交流,吸收各国优秀文明成果,增强中华文化国际影响力。

(四)推进文化创新,增强文化发展活力。在时代的高起点上推动文化内容形式、体制机制、传播手段创新,解放和发展文化生产力,是繁荣文化的必由之路。要坚持为人民服务、为社会主义服务的方向和百花齐放、百家争鸣的方针,贴近实际、贴近生活、贴近群众,始终把社会效益放在首位,做到经济效益与社会效益相统一。创作更多反映人民主体地位和现实生活、群众喜闻乐见的优秀精神文化产品。深化文化体制改革,完善扶持公益性文化事业、发展文化产业、鼓励文化创新的政策,营造有利于出精品、出人才、出效益的环境。坚持把发展公益性文化事业作为保障人民基本文化权益的主要途径,加大投入力度,加强社区和乡村文化设施建设。大力发展文化产业,实施重大文化产业项目带动战略,加快文化产业基地和区域性特色文化产业群建设,培育文化产业骨干企业和战略投资者,繁荣文化市场,增强国际竞争力。运用高新技术创新文化生产方式,培育新的文化业态,加快构建传输快捷、覆盖广泛的文化传播体系。设立国家荣誉制度,表彰有杰出贡献的文化工作者。

中华民族伟大复兴必然伴随着中华文化繁荣兴盛。要充分发挥人民在文化建设中的主体作用,调动广大文化工作者的积极性,更加自觉、更加主动地推动文化大发展大繁荣,在中国特色社会主义的伟大实践中进行文化创造,让人民共享文化发展成果。

\section{加快推进以改善民生为重点的社会建设}

社会建设与人民幸福安康息息相关。必须在经济发展的基础上,更加注重社会建设,着力保障和改善民生,推进社会体制改革,扩大公共服务,完善社会管理,促进社会公平正义,努力使全体人民学有所教、劳有所得、病有所医、老有所养、住有所居,推动建设和谐社会。

(一)优先发展教育,建设人力资源强国。教育是民族振兴的基石,教育公平是社会公平的重要基础。要全面贯彻党的教育方针,坚持育人为本、德育为先,实施素质教育,提高教育现代化水平,培养德智体美全面发展的社会主义建设者和接班人,办好人民满意的教育。优化教育结构,促进义务教育均衡发展,加快普及高中阶段教育,大力发展职业教育,提高高等教育质量。重视学前教育,关心特殊教育。更新教育观念,深化教学内容方式、考试招生制度、质量评价制度等改革,减轻中小学生课业负担,提高学生综合素质。坚持教育公益性质,加大财政对教育投入,规范教育收费,扶持贫困地区、民族地区教育,健全学生资助制度,保障经济困难家庭、进城务工人员子女平等接受义务教育。加强教师队伍建设,重点提高农村教师素质。鼓励和规范社会力量兴办教育。发展远程教育和继续教育,建设全民学习、终身学习的学习型社会。

(二)实施扩大就业的发展战略,促进以创业带动就业。就业是民生之本。要坚持实施积极的就业政策,加强政府引导,完善市场就业机制,扩大就业规模,改善就业结构。完善支持自主创业、自谋职业政策,加强就业观念教育,使更多劳动者成为创业者。健全面向全体劳动者的职业教育培训制度,加强农村富余劳动力转移就业培训。建立统一规范的人力资源市场,形成城乡劳动者平等就业的制度。完善面向所有困难群众的就业援助制度,及时帮助零就业家庭解决就业困难。积极做好高校毕业生就业工作。规范和协调劳动关系,完善和落实国家对农民工的政策,依法维护劳动者权益。

(三)深化收入分配制度改革,增加城乡居民收入。合理的收入分配制度是社会公平的重要体现。要坚持和完善按劳分配为主体、多种分配方式并存的分配制度,健全劳动、资本、技术、管理等生产要素按贡献参与分配的制度,初次分配和再分配都要处理好效率和公平的关系,再分配更加注重公平。逐步提高居民收入在国民收入分配中的比重,提高劳动报酬在初次分配中的比重。着力提高低收入者收入,逐步提高扶贫标准和最低工资标准,建立企业职工工资正常增长机制和支付保障机制。创造条件让更多群众拥有财产性收入。保护合法收入,调节过高收入,取缔非法收入。扩大转移支付,强化税收调节,打破经营垄断,创造机会公平,整顿分配秩序,逐步扭转收入分配差距扩大趋势。

(四)加快建立覆盖城乡居民的社会保障体系,保障人民基本生活。社会保障是社会安定的重要保证。要以社会保险、社会救助、社会福利为基础,以基本养老、基本医疗、最低生活保障制度为重点,以慈善事业、商业保险为补充,加快完善社会保障体系。促进企业、机关、事业单位基本养老保险制度改革,探索建立农村养老保险制度。全面推进城镇职工基本医疗保险、城镇居民基本医疗保险、新型农村合作医疗制度建设。完善城乡居民最低生活保障制度,逐步提高保障水平。完善失业、工伤、生育保险制度。提高统筹层次,制定全国统一的社会保险关系转续办法。采取多种方式充实社会保障基金,加强基金监管,实现保值增值。健全社会救助体系。做好优抚安置工作。发扬人道主义精神,发展残疾人事业。加强老龄工作。强化防灾减灾工作。健全廉租住房制度,加快解决城市低收入家庭住房困难。

(五)建立基本医疗卫生制度,提高全民健康水平。健康是人全面发展的基础,关系千家万户幸福。要坚持公共医疗卫生的公益性质,坚持预防为主、以农村为重点、中西医并重,实行政事分开、管办分开、医药分开、营利性和非营利性分开,强化政府责任和投入,完善国民健康政策,鼓励社会参与,建设覆盖城乡居民的公共卫生服务体系、医疗服务体系、医疗保障体系、药品供应保障体系,为群众提供安全、有效、方便、价廉的医疗卫生服务。完善重大疾病防控体系,提高突发公共卫生事件应急处置能力。加强农村三级卫生服务网络和城市社区卫生服务体系建设,深化公立医院改革。建立国家基本药物制度,保证群众基本用药。扶持中医药和民族医药事业发展。加强医德医风建设,提高医疗服务质量。确保食品药品安全。坚持计划生育的基本国策,稳定低生育水平,提高出生人口素质。开展爱国卫生运动,发展妇幼卫生事业。

(六)完善社会管理,维护社会安定团结。社会稳定是人民群众的共同心愿,是改革发展的重要前提。要健全党委领导、政府负责、社会协同、公众参与的社会管理格局,健全基层社会管理体制。最大限度激发社会创造活力,最大限度增加和谐因素,最大限度减少不和谐因素。妥善处理人民内部矛盾,完善信访制度,健全党和政府主导的维护群众权益机制。重视社会组织建设和管理。加强流动人口服务和管理。坚持安全发展,强化安全生产管理和监督,有效遏制重特大安全事故。完善突发事件应急管理机制。健全社会治安防控体系,加强社会治安综合治理,深入开展平安创建活动,改革和加强城乡社区警务工作,依法防范和打击违法犯罪活动,保障人民生命财产安全。完善国家安全战略,健全国家安全体制,高度警惕和坚决防范各种分裂、渗透、颠覆活动,切实维护国家安全。

和谐社会要靠全社会共同建设。我们要紧紧依靠人民,调动一切积极因素,努力形成社会和谐人人有责、和谐社会人人共享的生动局面。

\section{开创国防和军队现代化建设新局面}

国防和军队建设,在中国特色社会主义事业总体布局中占有重要地位。必须站在国家安全和发展战略全局的高度,统筹经济建设和国防建设,在全面建设小康社会进程中实现富国和强军的统一。

全面履行党和人民赋予的新世纪新阶段军队历史使命,必须坚持以毛泽东军事思想、邓小平新时期军队建设思想、江泽民国防和军队建设思想为指导,把科学发展观作为国防和军队建设的重要指导方针,贯彻新时期军事战略方针,加快中国特色军事变革,做好军事斗争准备,提高军队应对多种安全威胁、完成多样化军事任务的能力,坚决维护国家主权、安全、领土完整,为维护世界和平贡献力量。

军队革命化、现代化、正规化建设是统一的整体,必须全面加强、协调推进。要始终坚持党对军队绝对领导的根本原则和人民军队的根本宗旨,深入进行军队历史使命、理想信念、战斗精神和社会主义荣辱观教育,大力弘扬听党指挥、服务人民、英勇善战的优良传统。坚持科技强军,按照建设信息化军队、打赢信息化战争的战略目标,加快机械化和信息化复合发展,积极开展信息化条件下军事训练,全面建设现代后勤,加紧培养大批高素质新型军事人才,切实转变战斗力生成模式。坚持依法治军、从严治军,完善军事法规,加强科学管理。

适应世界军事发展新趋势和我国发展新要求,推进军事理论、军事技术、军事组织、军事管理创新。调整改革军队体制编制和政策制度,逐步形成一整套既有中国特色又符合现代军队建设规律的科学的组织模式、制度安排和运作方式。调整改革国防科技工业体制和武器装备采购体制,提高武器装备研制的自主创新能力和质量效益。建立和完善军民结合、寓军于民的武器装备科研生产体系、军队人才培养体系和军队保障体系,坚持勤俭建军,走出一条中国特色军民融合式发展路子。深入研究新的历史条件下建军治军特点规律和人民战争战略战术,繁荣和发展军事科学。

增强全民国防观念,完善国防动员体系,加强国防动员建设,提高预备役部队和民兵建设质量。加强人民武装警察部队建设,更好履行维护国家安全和社会稳定、保障人民安居乐业的职责使命。坚持拥军优属、拥政爱民,积极开展军民共建,巩固军政军民团结。各级党组织、政府和人民群众要一如既往支持国防和军队建设,军队要继续为经济社会发展作贡献。

\section{推进“一国两制”实践和祖国和平统一大业}

香港、澳门回归祖国以来,“一国两制”实践日益丰富。“一国两制”是完全正确的,具有强大生命力。按照“一国两制”实现祖国和平统一,符合中华民族根本利益。

保持香港、澳门长期繁荣稳定是党在新形势下治国理政面临的重大课题。我们将坚定不移地贯彻“一国两制”、“港人治港”、“澳人治澳”、高度自治的方针,严格按照特别行政区基本法办事;全力支持特别行政区政府依法施政,着力发展经济、改善民生、推进民主;鼓励香港、澳门各界人士在爱国爱港、爱国爱澳旗帜下和衷共济,促进社会和睦;加强内地与香港、澳门交流合作,实现优势互补、共同发展;积极支持香港、澳门开展对外交往,坚决反对外部势力干预香港、澳门事务。香港同胞、澳门同胞完全有智慧有能力管理好、建设好香港、澳门,香港、澳门已经并将继续为国家现代化建设发挥重要作用,伟大祖国永远是香港、澳门繁荣稳定的坚强后盾。

解决台湾问题、实现祖国完全统一,是全体中华儿女的共同心愿。我们将遵循“和平统一、一国两制”的方针和现阶段发展两岸关系、推进祖国和平统一进程的八项主张,坚持一个中国原则决不动摇,争取和平统一的努力决不放弃,贯彻寄希望于台湾人民的方针决不改变,反对“台独”分裂活动决不妥协,牢牢把握两岸关系和平发展的主题,真诚为两岸同胞谋福祉、为台海地区谋和平,维护国家主权和领土完整,维护中华民族根本利益。

坚持一个中国原则,是两岸关系和平发展的政治基础。尽管两岸尚未统一,但大陆和台湾同属一个中国的事实从未改变。中国是两岸同胞的共同家园,两岸同胞理应携手维护好、建设好我们的共同家园。台湾任何政党,只要承认两岸同属一个中国,我们都愿意同他们交流对话、协商谈判,什么问题都可以谈。我们郑重呼吁,在一个中国原则的基础上,协商正式结束两岸敌对状态,达成和平协议,构建两岸关系和平发展框架,开创两岸关系和平发展新局面。

十三亿大陆同胞和两千三百万台湾同胞是血脉相连的命运共同体。凡是对台湾同胞有利的事情,凡是对维护台海和平有利的事情,凡是对促进祖国和平统一有利的事情,我们都会尽最大努力做好。我们理解、信赖、关心台湾同胞,将继续实施和充实惠及广大台湾同胞的政策措施,依法保护台湾同胞的正当权益,支持海峡西岸和其他台商投资相对集中地区经济发展。两岸同胞要加强交往,加强经济文化交流,继续拓展领域、提高层次,推动直接“三通”,使彼此感情更融洽、合作更深化,为实现中华民族伟大复兴而共同努力。

当前,“台独”分裂势力加紧进行分裂活动,严重危害两岸关系和平发展。两岸同胞要共同反对和遏制“台独”分裂活动。中国主权和领土完整不容分割。任何涉及中国主权和领土完整的问题,必须由包括台湾同胞在内的全中国人民共同决定。我们愿以最大诚意、尽最大努力实现两岸和平统一,绝不允许任何人以任何名义任何方式把台湾从祖国分割出去。

两岸统一是中华民族走向伟大复兴的历史必然。海内外中华儿女紧密团结、共同奋斗,祖国完全统一就一定能够实现。

\section{始终不渝走和平发展道路}

当今世界正处在大变革大调整之中。和平与发展仍然是时代主题,求和平、谋发展、促合作已经成为不可阻挡的时代潮流。世界多极化不可逆转,经济全球化深入发展,科技革命加速推进,全球和区域合作方兴未艾,国与国相互依存日益紧密,国际力量对比朝着有利于维护世界和平方向发展,国际形势总体稳定。

同时,世界仍然很不安宁。霸权主义和强权政治依然存在,局部冲突和热点问题此起彼伏,全球经济失衡加剧,南北差距拉大,传统安全威胁和非传统安全威胁相互交织,世界和平与发展面临诸多难题和挑战。

共同分享发展机遇,共同应对各种挑战,推进人类和平与发展的崇高事业,事关各国人民的根本利益,也是各国人民的共同心愿。我们主张,各国人民携手努力,推动建设持久和平、共同繁荣的和谐世界。为此,应该遵循联合国宪章宗旨和原则,恪守国际法和公认的国际关系准则,在国际关系中弘扬民主、和睦、协作、共赢精神。政治上相互尊重、平等协商,共同推进国际关系民主化;经济上相互合作、优势互补,共同推动经济全球化朝着均衡、普惠、共赢方向发展;文化上相互借鉴、求同存异,尊重世界多样性,共同促进人类文明繁荣进步;安全上相互信任、加强合作,坚持用和平方式而不是战争手段解决国际争端,共同维护世界和平稳定;环保上相互帮助、协力推进,共同呵护人类赖以生存的地球家园。

当代中国同世界的关系发生了历史性变化,中国的前途命运日益紧密地同世界的前途命运联系在一起。不管国际风云如何变幻,中国政府和人民都将高举和平、发展、合作旗帜,奉行独立自主的和平外交政策,维护国家主权、安全、发展利益,恪守维护世界和平、促进共同发展的外交政策宗旨。

中国将始终不渝走和平发展道路。这是中国政府和人民根据时代发展潮流和自身根本利益作出的战略抉择。中华民族是热爱和平的民族,中国始终是维护世界和平的坚定力量。我们坚持把中国人民的利益同各国人民的共同利益结合起来,秉持公道,伸张正义。我们坚持国家不分大小、强弱、贫富一律平等,尊重各国人民自主选择发展道路的权利,不干涉别国内部事务,不把自己的意志强加于人。中国致力于和平解决国际争端和热点问题,推动国际和地区安全合作,反对一切形式的恐怖主义。中国奉行防御性的国防政策,不搞军备竞赛,不对任何国家构成军事威胁。中国反对各种形式的霸权主义和强权政治,永远不称霸,永远不搞扩张。

中国将始终不渝奉行互利共赢的开放战略。我们将继续以自己的发展促进地区和世界共同发展,扩大同各方利益的汇合点,在实现本国发展的同时兼顾对方特别是发展中国家的正当关切。我们将继续按照通行的国际经贸规则,扩大市场准入,依法保护合作者权益。我们支持国际社会帮助发展中国家增强自主发展能力、改善民生,缩小南北差距。我们支持完善国际贸易和金融体制,推进贸易和投资自由化便利化,通过磋商协作妥善处理经贸摩擦。中国决不做损人利己、以邻为壑的事情。

中国坚持在和平共处五项原则的基础上同所有国家发展友好合作。我们将继续同发达国家加强战略对话,增进互信,深化合作,妥善处理分歧,推动相互关系长期稳定健康发展。我们将继续贯彻与邻为善、以邻为伴的周边外交方针,加强同周边国家的睦邻友好和务实合作,积极开展区域合作,共同营造和平稳定、平等互信、合作共赢的地区环境。我们将继续加强同广大发展中国家的团结合作,深化传统友谊,扩大务实合作,提供力所能及的援助,维护发展中国家的正当要求和共同利益。我们将继续积极参与多边事务,承担相应国际义务,发挥建设性作用,推动国际秩序朝着更加公正合理的方向发展。我们将继续开展同各国政党和政治组织的交流合作,加强人大、政协、军队、地方、民间团体对外交往,增进中国人民和各国人民的相互了解和友谊。

中国发展离不开世界,世界繁荣稳定也离不开中国。中国人民将继续同各国人民一道,为实现人类的美好理想而不懈努力。

\section{以改革创新精神全面推进党的建设新的伟大工程}

中国特色社会主义事业是改革创新的事业。党要站在时代前列带领人民不断开创事业发展新局面,必须以改革创新精神加强自身建设,始终成为中国特色社会主义事业的坚强领导核心。

我们党已经成立八十六年,在全国执政五十八年,拥有七千多万党员,党的自身建设任务比过去任何时候都更为繁重。党领导的改革开放既给党注入巨大活力,也使党面临许多前所未有的新课题新考验。世情、国情、党情的发展变化,决定了以改革创新精神加强党的建设既十分重要又十分紧迫。必须把党的执政能力建设和先进性建设作为主线,坚持党要管党、从严治党,贯彻为民、务实、清廉的要求,以坚定理想信念为重点加强思想建设,以造就高素质党员、干部队伍为重点加强组织建设,以保持党同人民群众的血肉联系为重点加强作风建设,以健全民主集中制为重点加强制度建设,以完善惩治和预防腐败体系为重点加强反腐倡廉建设,使党始终成为立党为公、执政为民,求真务实、改革创新,艰苦奋斗、清正廉洁,富有活力、团结和谐的马克思主义执政党。

(一)深入学习贯彻中国特色社会主义理论体系,着力用马克思主义中国化最新成果武装全党。思想理论建设是党的根本建设,党的理论创新引领各方面创新。要按照建设学习型政党的要求,紧密结合改革开放和现代化建设的生动实践,深入学习马克思列宁主义、毛泽东思想、邓小平理论和“三个代表”重要思想,在全党开展深入学习实践科学发展观活动,坚持用发展着的马克思主义指导客观世界和主观世界的改造,进一步把握共产党执政规律、社会主义建设规律、人类社会发展规律,提高运用科学理论分析和解决实际问题能力。加强党员、干部理想信念教育和思想道德建设,使广大党员、干部成为实践社会主义核心价值体系的模范,做共产主义远大理想和中国特色社会主义共同理想的坚定信仰者、科学发展观的忠实执行者、社会主义荣辱观的自觉实践者、社会和谐的积极促进者。

(二)继续加强党的执政能力建设,着力建设高素质领导班子。党的执政能力建设关系党的建设和中国特色社会主义事业的全局,必须把提高领导水平和执政能力作为各级领导班子建设的核心内容抓紧抓好。要按照科学执政、民主执政、依法执政的要求,改进领导班子思想作风,提高领导干部执政本领,改善领导方式和执政方式,健全领导体制,完善地方党委领导班子配备改革后的工作机制,把各级领导班子建设成为坚定贯彻党的理论和路线方针政策、善于领导科学发展的坚强领导集体。以加强领导班子执政能力建设影响和带动全党,使党的全部工作始终符合时代要求和人民期待。

(三)积极推进党内民主建设,着力增强党的团结统一。党内民主是增强党的创新活力、巩固党的团结统一的重要保证。要以扩大党内民主带动人民民主,以增进党内和谐促进社会和谐。尊重党员主体地位,保障党员民主权利,推进党务公开,营造党内民主讨论环境。完善党的代表大会制度,实行党的代表大会代表任期制,选择一些县(市、区)试行党代表大会常任制。完善党的地方各级全委会、常委会工作机制,发挥全委会对重大问题的决策作用。严格实行民主集中制,健全集体领导与个人分工负责相结合的制度,反对和防止个人或少数人专断。推行地方党委讨论决定重大问题和任用重要干部票决制。建立健全中央政治局向中央委员会全体会议、地方各级党委常委会向委员会全体会议定期报告工作并接受监督的制度。改革党内选举制度,改进候选人提名制度和选举方式。推广基层党组织领导班子成员由党员和群众公开推荐与上级党组织推荐相结合的办法,逐步扩大基层党组织领导班子直接选举范围,探索扩大党内基层民主多种实现形式。全党同志要坚决维护党的集中统一,自觉遵守党的政治纪律,始终同党中央保持一致,坚决维护中央权威,切实保证政令畅通。

(四)不断深化干部人事制度改革,着力造就高素质干部队伍和人才队伍。坚持党管干部原则,坚持民主、公开、竞争、择优,形成干部选拔任用科学机制。规范干部任用提名制度,完善体现科学发展观和正确政绩观要求的干部考核评价体系,完善公开选拔、竞争上岗、差额选举办法。扩大干部工作民主,增强民主推荐、民主测评的科学性和真实性。加强干部选拔任用工作全过程监督。健全领导干部职务任期、回避、交流制度,完善公务员制度。健全干部双重管理体制。推进国有企业和事业单位人事制度改革,完善适合国有企业特点的领导人员管理办法。

坚持正确用人导向,按照德才兼备、注重实绩、群众公认原则选拔干部,提高选人用人公信度。加大培养选拔优秀年轻干部力度,鼓励年轻干部到基层和艰苦地区锻炼成长,提高年轻干部马克思主义理论素养和政治素质。重视培养选拔女干部、少数民族干部。格外关注长期在条件艰苦、工作困难地方努力工作的干部,注意从基层和生产一线选拔优秀干部充实各级党政领导机关。继续大规模培训干部,充分发挥党校、行政学院、干部学院作用,大幅度提高干部素质。全面做好离退休干部工作。贯彻尊重劳动、尊重知识、尊重人才、尊重创造的方针,坚持党管人才原则,统筹抓好以高层次人才和高技能人才为重点的各类人才队伍建设。创新人才工作体制机制,激发各类人才创造活力和创业热情,开创人才辈出、人尽其才新局面。

(五)全面巩固和发展先进性教育活动成果,着力加强基层党的建设。先进性是马克思主义政党的生命所系、力量所在,要靠千千万万高素质党员来体现。要扎实抓好党员队伍建设这一基础工程,坚持不懈地提高党员素质。认真学习和遵守党章,增强党员意识,建立党员党性定期分析制度,拓宽党员服务群众渠道,构建党员联系和服务群众工作体系,健全让党员经常受教育、永葆先进性长效机制,使党员真正成为牢记宗旨、心系群众的先进分子。加强和改进流动党员管理,加强进城务工人员中党的工作,建立健全城乡一体党员动态管理机制。提高发展党员质量,优化党员队伍结构,及时处置不合格党员。

党的基层组织是党执政的组织基础。要落实党建工作责任制,全面推进农村、企业、城市社区和机关、学校、新社会组织等的基层党组织建设,优化组织设置,扩大组织覆盖,创新活动方式,充分发挥基层党组织推动发展、服务群众、凝聚人心、促进和谐的作用。以党的基层组织建设带动其他各类基层组织建设。在党的基层组织和党员中深入开展创先争优活动。建立健全城乡党的基层组织互帮互助机制。在全国农村普遍开展党员干部现代远程教育。建立健全党内激励、关怀、帮扶机制,关心和爱护基层干部、老党员、生活困难党员。注重解决基层组织经费保障和活动场所等问题。

(六)切实改进党的作风,着力加强反腐倡廉建设。优良的党风是凝聚党心民心的巨大力量。要坚持人民是历史创造者的历史唯物主义观点,坚持全心全意为人民服务,坚持群众路线,真诚倾听群众呼声,真实反映群众愿望,真情关心群众疾苦,多为群众办好事、办实事,做到权为民所用、情为民所系、利为民所谋。以求真务实作风推进各项工作,多干打基础、利长远的事。加强调查研究,改进学风和文风,精简会议和文件,反对形式主义、官僚主义,反对弄虚作假。倡导勤俭节约、勤俭办一切事业,反对奢侈浪费。全党同志特别是领导干部都要讲党性、重品行、作表率。深入开展党风党纪教育,积极进行批评和自我批评,使领导干部模范遵守党纪国法,继承优良传统,弘扬新风正气,以优良的党风促政风带民风。

中国共产党的性质和宗旨,决定了党同各种消极腐败现象是水火不相容的。坚决惩治和有效预防腐败,关系人心向背和党的生死存亡,是党必须始终抓好的重大政治任务。全党同志一定要充分认识反腐败斗争的长期性、复杂性、艰巨性,把反腐倡廉建设放在更加突出的位置,旗帜鲜明地反对腐败。坚持标本兼治、综合治理、惩防并举、注重预防的方针,扎实推进惩治和预防腐败体系建设,在坚决惩治腐败的同时,更加注重治本,更加注重预防,更加注重制度建设,拓展从源头上防治腐败工作领域。严格执行党风廉政建设责任制。坚持深化改革和创新体制,加强廉政文化建设,形成拒腐防变教育长效机制、反腐倡廉制度体系、权力运行监控机制。健全纪检监察派驻机构统一管理,完善巡视制度。加强领导干部廉洁自律工作,提高党员干部拒腐防变能力。坚决纠正损害群众利益的不正之风,切实解决群众反映强烈的问题。坚决查处违纪违法案件,对任何腐败分子,都必须依法严惩,决不姑息!

同志们!我们党自诞生之日起就勇敢担当起带领中国人民创造幸福生活、实现中华民族伟大复兴的历史使命。为了完成这个历史使命,一代又一代中国共产党人前赴后继,无数革命先烈献出了宝贵生命。当代中国共产党人必须继续承担好这个历史使命。我们党正在带领全国各族人民进行的改革开放和社会主义现代化建设,是新中国成立以后我国社会主义建设伟大事业的继承和发展,是近代以来中国人民争取民族独立、实现国家富强伟大事业的继承和发展。抚今追昔,我们深感肩负的使命神圣而光荣。展望未来,我们对实现推进现代化建设、完成祖国统一、维护世界和平与促进共同发展这三大历史任务充满信心。

全党同志必须清醒认识到,实现全面建设小康社会的目标还需要继续奋斗十几年,基本实现现代化还需要继续奋斗几十年,巩固和发展社会主义制度则需要几代人、十几代人甚至几十代人坚持不懈地努力奋斗。要奋斗就会有困难有风险。我们一定要居安思危、增强忧患意识,始终保持对马克思主义、对中国特色社会主义、对实现中华民族伟大复兴的坚定信念;一定要戒骄戒躁、艰苦奋斗,牢记社会主义初级阶段基本国情,为党和人民事业不懈努力;一定要刻苦学习、埋头苦干,不断创造经得起实践、人民、历史检验的业绩;一定要加强团结、顾全大局,自觉维护全党的团结统一,保持党同人民群众的血肉联系,巩固全国各族人民的大团结,加强海内外中华儿女的大团结,促进中国人民同世界各国人民的大团结,为战胜一切艰难险阻、推动党和人民事业取得新的更大胜利提供强大力量。

让我们高举中国特色社会主义伟大旗帜,更加紧密地团结在党中央周围,万众一心,开拓奋进,为夺取全面建设小康社会新胜利、谱写人民美好生活新篇章而努力奋斗!

\bye
