%# -*- coding:utf-8 -*-
%%%%%%%%%%%%%%%%%%%%%%%%%%%%%%%%%%%%%%%%%%%%%%%%%%%%%%%%%%%%%%%%%%%%%%%%%%%%%%%%%%%%%


\chapter{西门庆乘醉烧阴户\KG 李瓶儿带病宴重阳}


词曰:

\[
蛩声泣露惊秋枕,泪湿鸳鸯锦。独卧玉肌凉,残更与恨长。 
阴风翻翠幌,雨涩灯花暗。毕竟不成眠,鸦啼金井寒。
\]

话说一日,韩道国铺中回家,睡到半夜,他老婆王六儿与他商议道:“你我被他照顾,挣了恁些钱,也该摆席酒儿请他来坐坐。况他又丢了孩儿,只当与他释闷,他能吃多少!彼此好看。就是后生小郎看着,到明日南边去,也知财主和你我亲厚,比别人不同。”韩道国道:“我心里也是这等说。明日初五日是月忌,不好。到初六日,安排酒席,叫两个唱的,具个柬帖,等我亲自到宅内,请老爹散闷坐坐。我晚夕便往铺子里睡去。”王六儿道:“平白又叫甚么唱的?只怕他酒后要来这屋里坐坐,不方便。隔壁乐三嫂家,常走的一个女儿申二姐,年纪小小的,且会唱,他又是瞽目的,请将他来唱唱罢。要打发他过去还容易。”韩道国道:“你说的是。”一宿晚景题过。

到次日,韩道国走到铺子里,央及温秀才写了个请柬儿,亲见西门庆,声喏毕,说道:“明日,小人家里治了一杯水酒,无事请老爹贵步下临,散闷坐一日。”因把请柬递上去。西门庆看了,说道:“你如何又费此心。我明日倒没事,衙门中回家就去。”韩道国作辞出门。到次早,拿银子叫后生胡秀买嗄饭菜蔬,一面叫厨子整理,又拿轿子接了申二姐来,王六儿同丫鬟伺候下好茶好水,单等西门庆来到。等到午后,只见琴童儿先送了一坛葡萄酒来,然后西门庆坐着凉轿,玳安、王经跟随,到门首下轿,头戴忠靖冠,身穿青水纬罗直身,粉头皂靴。韩道国迎接入内,见毕礼数,说道:“又多谢老爹赐将酒来。”正面独独安放一张交椅,西门庆坐下。

不一时,王六儿打扮出来,与西门庆磕了四个头,回后边看茶去了。须臾,王经拿出茶来,韩道国先取一盏,举的高高的奉与西门庆,然后自取一盏,旁边相陪。吃毕,王经接了茶盏下去,韩道国便开言说道:“小人承老爹莫大之恩,一向在外,家中小媳妇承老爹看顾,王经又蒙抬举,叫在宅中答应,感恩不浅。前日哥儿没了,虽然小人在那里,媳妇儿因感了些风寒,不曾往宅里吊问的,恐怕老爹恼。今日,一者请老爹解解闷,二者就恕俺两口儿罪。”西门庆道:“无事又教你两口儿费心。”说着,只见王六儿也在旁边坐下。因向韩道国道:“你和老爹说了不?”道国道:“我还不曾说哩。”西门庆问道:“是甚么?”王六儿道:“他今日要内边请两位姐儿来伏侍老爹,我恐怕不方便,故不去请。隔壁乐家常走的一个女儿,叫做申二姐,诸般大小时样曲儿,连数落都会唱。我前日在宅里,见那一位郁大姐唱的也中中的,还不如这申二姐唱的好。教我今日请了他来,唱与爹听。未知你老人家心下何如?若好,到明日叫了宅里去,唱与他娘每听。”西门庆道:“既是有女儿,亦发好了。你请出来我看看。”不一时,韩道国叫玳安上来:“替老爹宽去衣服。”一面安放桌席,胡秀拿果菜案酒上来。王六儿把酒打开,烫热了,在旁执壶,道国把盏,与西门庆安席坐下,然后才叫出申二姐来。西门庆睁眼观看,见他高髻云鬟,插着几枝稀稀花翠,淡淡钗梳,绿袄红裙,显一对金莲趫趫;桃腮粉脸,抽两道细细春山。望上与西门庆磕了四个头。西门庆便道:“请起。你今青春多少?”申二姐道:“小的二十一岁了。”又问:“你记得多少唱?”申二姐道:“大小也记百十套曲子。”西门庆令韩道国旁边安下个坐儿与他坐。申二姐向前行毕礼,方才坐下。先拿筝来唱了一套《秋香亭》,然后吃了汤饭,添换上来,又唱了一套《半万贼兵》。落后酒阑上来,西门庆吩咐:“把筝拿过去,取琵琶与他,等他唱小词儿我听罢。”那申二姐一迳要施逞他能弹会唱。一面轻摇罗袖,款跨鲛绡,顿开喉音,把弦儿放得低低的,弹了个《四不应·山坡羊》。唱完了,韩道国教浑家满斟一盏,递与西门庆。王六儿因说:“申二姐,你还有好《锁南枝》,唱两个与老爹听。”那申二姐就改了调儿,唱《锁南枝》道:

初相会,可意人,年少青春,不上二旬。黑\textuni{29B79}\textuni{29B79}两朵乌云,红馥馥一点朱唇,脸赛夭桃如嫩笋。若生在画阁兰堂,端的也有个夫人分。可惜在章台,出落做下品。但能够改嫁从良,胜强似弃旧迎新。

初相会,可意娇,月貌花容,风尘中最少。瘦腰肢一捻堪描,俏心肠百事难学,恨只恨和他相逢不早。常则怨席上樽前,浅斟低唱相偎抱。一觑一个真,一看一个饱。虽然是半霎欢娱,权且将闷解愁消。

西门庆听了这两个《锁南枝》,正打着他初请了郑月儿那一节事来,心中甚喜。王六儿满满的又斟上一盏,笑嘻嘻说道:“爹,你慢慢儿的饮,申二姐这个才是零头儿,他还记的好些小令儿哩。到明日闲了,拿轿子接了,唱与他娘每听,管情比郁大姐唱的高。”西门庆因说:“申二姐,我重阳那日,使人来接你,去不去?”申二姐道:“老爹说那里话,但呼唤,怎敢违阻!”西门庆听见他说话伶俐,心中大喜。

不一时,交杯换盏之间,王六儿恐席间说话不方便,叫他唱了几套,悄悄向韩道国说:“教小厮招弟儿,送过乐三嫂家歇去罢。”临去拜辞,西门庆向袖中掏出一包儿三钱银子,赏他买弦。申二姐连忙嗑头谢了。西门庆约下:“我初八日使人请你去。”王六儿道:“爹只使王经来对我说,等我这里教小厮请他去。”说毕,申二姐往隔壁去了。韩道国与老婆说知,也就往铺子里睡去了。只落下老婆在席上,陪西门庆掷骰饮酒。吃了一回,两个看看吃的涎将上来,西门庆推起身更衣,就走入妇人房里,两个顶门顽耍。王经便把灯烛拿出来,在前半间和玳安、琴童儿做一处饮酒。

那后生胡秀,在厨下偷吃了几碗酒,打发厨子去了,走在王六儿隔壁供养佛祖先堂内,地下铺着一领席,就睡着了。睡了一觉起来,忽听见妇人房里声唤,又见板壁缝里透过灯亮来,只道西门庆去了,韩道国在房中宿歇。暗暗用头上簪子刺破板缝中糊的纸,往那边张看。见那边房中亮腾腾点着灯烛,不想西门庆和老婆在屋里正干得好。伶伶俐俐看见,把老婆两只腿,却是用脚带吊在床头上,西门庆上身止着一件绫袄儿,下身赤露,就在床沿上一来一往,一动一静,扇打的连声响亮,老婆口里百般言语都叫将出来。良久,只听老婆说:“我的亲达!你要烧淫妇,随你心里拣着那块只顾烧,淫妇不敢拦你。左右淫妇的身子属了你,怕那些儿了!”西门庆道:“只怕你家里的嗔是的。”老婆道:“那忘八七个头八个胆,他敢嗔!他靠着那里过日子哩?”西门庆道:“你既一心在我身上,等这遭打发他和来保起身,亦发留他长远在南边,做个买手置货罢。”老婆道:“等走过两遭儿,却教他去。省的闲着在家做甚么?他说倒在外边走惯了,一心只要外边去。你若下顾他,可知好哩!等他回来,我房里替他寻下一个,我也不要他,一心扑在你身上,随你把我安插在那里就是了。我若说一句假,把淫妇不值钱身子就烂化了。”西门庆道:“我儿,你快休赌誓!”两个一动一静,都被胡秀听了个不亦乐乎。

韩道国先在家中不见胡秀,只说往铺子里睡去了。走到缎子铺里,问王显、荣海,说他没来。韩道国一面又走回家,叫开门,前后寻胡秀,那里得来,只见王经陪玳安、琴童三个在前边吃酒。胡秀听见他的语音来家,连忙倒在席上,又推睡了。不一时,韩道国点灯寻到佛堂地下,看见他鼻口内打鼾睡,用脚踢醒,骂道:“贼野狗死囚,还不起来!我只说先往铺子里睡去,你原来在这里挺得好觉儿。还不起来跟我去!”那胡秀起来,推揉了揉眼,楞楞睁睁跟道国往铺子里去了。

西门庆弄老婆,直弄够有一个时辰,方才了事。烧了王六儿心口里并\textuni{23B48}盖子上、尾亭骨儿上共三处香。老婆起来穿了衣服,教丫头打发舀水净了手,重筛暖酒,再上佳肴,情话攀盘。又吃了几钟,方才起身上马,玳安、王经、琴童三个跟着。到家中已有二更天气,走到李瓶儿房中。李瓶儿睡在床上,见他吃的酣酣儿的进来,说道:“你今日在谁家吃酒来?”西门庆道:“韩道国家请我。见我丢了孩子,与我释闷。他叫了个女先生申二姐来,年纪小小,好不会唱!又不说郁大姐。等到明日重阳,使小厮拿轿子接他来家,唱两日你每听,就与你解解闷。你紧心里不好,休要只顾思想他了。”说着,就要叫迎春来脱衣裳,和李瓶儿睡。李瓶儿道:“你没的说!我下边不住的长流,丫头替我煎药哩。你往别人屋里睡去罢。你看着我成日好模样儿罢了,只有一口游气儿在这里,又来缠我起来。”西门庆道:“我的心肝!我心里舍不的你。只要和你睡,如之奈何?”李瓶儿瞟了他一眼,笑了笑儿:“谁信你那虚嘴掠舌的。我倒明日死了,你也舍不的我罢!”又道:“亦发等我好好儿,你再进来和我睡也不迟。”西门庆坐了一回,说道:“罢,罢。你不留我,等我往潘六儿那边睡去罢。”李瓶儿道:“原来你去,省的屈着你那心肠儿。他那里正等的你火里火发,你不去,却忙惚儿来我这屋里缠。”西门庆道:“你恁说,我又不去了。”李瓶儿微笑道:“我哄你哩,你去罢。”于是打发西门庆过去了。李瓶儿起来,坐在床上,迎春伺候他吃药。拿起那药来,止不住扑簌簌香腮边滚下泪来,长吁了一口气,方才吃了那盏药。正是:

\[
心中无限伤心事,付与黄鹂叫几声。
\]

不说李瓶儿吃药睡了,单表西门庆到于潘金莲房里。金莲才叫春梅罩了灯上床睡下。忽见西门庆推开门进来便道:“我儿,又早睡了?”金莲道:“稀幸!那阵风儿刮你到我这屋里来!”因问:“你今日往谁家吃酒去来?”西门庆道:“韩伙计打南边来,见我没了孩子,一者与我释闷,二者照顾他外边走了这遭,请我坐坐。”金莲道:“他便在外边,你在家又照顾他老婆了。”西门庆道:“伙计家,那里有这道理?”妇人道:“伙计家,有这个道理!齐腰拴着根线儿,只怕\textuni{34B2}过界儿去了。你还捣鬼哄俺每哩,俺每知道的不耐烦了!你生日,贼淫妇他没在这里?你悄悄把李瓶儿寿字簪子,黄猫黑尾偷与他,却叫他戴了来施展。大娘、孟三儿,这一家子那个没看见?吃我问了一句,他把脸儿都红了,他没告诉你?今日又摸到那里去,贼没廉耻的货,一个大摔瓜长淫妇,乔眉乔样,描的那水鬓长长的,搽的那嘴唇鲜红的——倒象人家那血\textuni{23B48}。甚么好老婆,一个大紫腔色黑淫妇,我不知你喜欢他那些儿!嗔道把忘八舅子也招惹将来,一早一晚教他好往回传话儿。”西门庆坚执不认,笑道:“怪小奴才儿,单管只胡说,那里有此勾当?今日他男子汉陪我坐,他又没出来。”妇人道:“你拿这个话儿来哄我?谁不知他汉子是个明忘八,又放羊,又拾柴,一径把老婆丢与你,图你家买卖做,要赚你的钱使。你这傻行货子,只好四十里听铳响罢了!”西门庆脱了衣裳,坐在床沿上,妇人探出手来,把裤子扯开,摸见那话软叮当的,托子还带在上面,说道:“可又来,你腊鸭子煮到锅里——身子儿烂了,嘴头儿还硬。见放着不语先生在这里,强盗和那淫妇怎么弄耸,耸到这咱晚才来家?弄的恁个样儿,嘴头儿还强哩!你赌个誓,我叫春梅舀一瓯子凉水,你只吃了,我就算你好胆子。论起来,盐也是这般咸,醋也是这般酸,秃子包网中——饶这一抿子儿也罢了。若是信着你意儿,把天下老婆都耍遍了罢。贼没羞的货,一个大眼里火行货子!你早是个汉子,若是个老婆,就养遍街,\textuni{34B2}遍巷。”几句说的西门庆睁睁的,只是笑。

上的床来,叫春梅筛热了烧酒,把金穿心盒儿内药拈了一粒,放在口里咽下去,仰卧在枕上,令妇人:“我儿,你下去替你达品,品起来是你造化。”那妇人一径做乔张致,便道:“好干净儿!你在那淫妇窟窿子里钻了来,教我替你咂,可不臜杀了我!”西门庆道:“怪小淫妇儿,单管胡说白道的,那里有此勾当?”妇人道:“那里有此勾当?你指着肉身子赌个誓么!”乱了一回,教西门庆下去使水,西门庆不肯下去,妇人旋向袖子里掏出个汗巾来,将那话抹展了一回,方才用朱唇裹没。呜咂半晌,咂弄的那话奢棱跳脑,暴怒起来,乃骑在妇人身上,纵麈柄自后插入牝中,两手兜其股,蹲踞而摆之,肆行扇打,连声响亮。灯光之下,窥玩其出入之势,妇人倒伏在枕畔,举股迎凑者久之。西门庆兴犹不惬,将妇人仰卧朝上,那话上使了粉红药儿,顶入去,执其双足,又举腰没棱露脑掀腾者将二三百度。妇人禁受不的,瞑目颤声,没口子叫:“达达,你这遭儿只当将就我,不使上他也罢了。”西门庆口中呼叫道:“小淫妇儿,你怕我不怕?再敢无礼不敢?”妇人道:“我的达达,罢么,你将就我些儿,我再不敢了!达达慢慢提,看提散了我的头发。”两个颠鸳倒凤,足狂了半夜,方才体倦而寝。

话休饶舌,又早到重阳令节。西门庆对吴月娘说:“韩伙计前日请我,一个唱的申二姐,生的人材又好,又会唱。我使小厮接他来,留他两日,教他唱与你每听。”又吩咐厨下收拾肴馔果酒,在花园大卷棚聚景堂内,安放大八仙桌,合家宅眷,庆赏重阳。

不一时,王经轿子接的申二姐到了。入到后边,与月娘众人磕了头。月娘见他年小,生的好模样儿。问他套数,也会不多,诸般小曲儿倒记的有好些。一面打发他吃了茶食,先教在后边唱了两套,然后花园摆下酒席。那日,西门庆不曾往衙门中去,在家看着栽了菊花。请了月娘、李娇儿、孟玉楼、潘金莲、李瓶儿、孙雪娥并大姐,都在席上坐的。春梅、玉箫、迎春、兰香在旁斟酒伏侍。申二姐先拿琵琶在旁弹唱。那李瓶儿在房中,因身上不方便,请了半日才来。恰似风儿刮倒的一般,强打着精神陪西门庆坐,众人让他酒儿也不大吃。西门庆和月娘见他面带忧容,眉头不展,说道:“李大姐,你把心放开,教申二姐弹唱曲儿你听。”玉楼道:“你说与他,教他唱甚么曲儿,他好唱。”李瓶儿只顾不说。正饮酒中间,忽见王经走来说道:“应二爹、常二叔来了。”西门庆道:“请你应二爹、常二叔在小卷棚内坐,我就来。”王经道:“常二叔教人拿了两个盒子在外头。”西门庆向月娘道:“此是他成了房子,买礼来谢我的意思。”月娘道:“少不的安排些甚么管待他,怎好空了他去!你陪他坐去,我这里吩咐看菜儿。”西门庆临出来,又叫申二姐:“你唱个好曲儿,与你六娘听。”一直往前边去了。金莲道:“也没见这李大姐,随你心里说个甚么曲儿,教申二姐唱就是了,辜负他爹的心!为你叫将他来,你又不言语。”催逼的李瓶儿急了,半日才说出来:“你唱个‘紫陌红尘’罢。”那申二姐道:“这个不打紧,我有。”于是取过筝来,顿开喉音,细细唱了一套。唱毕,吴月娘道:“李大姐,好甜酒儿,你吃上一钟儿。”李瓶儿又不敢违阻,拿起钟儿来咽了一口儿,又放下了。坐不多时,下边一阵热热的来,又往屋里去了,不题。

且说西门庆到于小卷棚翡翠轩,只见应伯爵与常峙节在松墙下正看菊花。原来松墙两边,摆放二十盆,都是七尺高,各样有名的菊花,也有大红袍、状元红、紫袍金带、白粉西、黄粉西、满天星、醉杨妃、玉牡丹、鹅毛菊、鸳鸯花之类。西门庆出来,二人向前作揖。常峙节即唤跟来人,把盒儿掇进来。西门庆一见便问:“又是甚么?”伯爵道:“常二哥蒙哥厚情,成了房子,无可酬答,教他娘子制造了这螃蟹鲜并两只炉烧鸭儿,邀我来和哥坐坐。”西门庆道:“常二哥,你又费这个心做甚么?你令正病才好些,你又禁害他!”伯爵道:“我也是恁说。他说道别的东西儿来,恐怕哥不稀罕。”西门庆令左右打开盒儿观看:四十个大螃蟹,都是剔剥净了的,里边酿着肉,外用椒料姜蒜米儿团粉裹就,香油煠,酱油醋造过,香喷喷,酥脆好食。又是两大只院中炉烧熟鸭。西门庆看了,即令春鸿、王经掇进去,吩咐拿五十文钱赏拿盒人,因向常峙节谢了。

琴童在旁掀帘,请入翡翠轩坐。伯爵只顾夸奖不尽好菊花,问:“哥是那里寻的?”西门庆道:“是管砖厂刘太监送的。这二十盆,就连盆都送与我了。”伯爵道:“花到不打紧,这盆正是官窑双箍邓浆盆,都是用绢罗打,用脚跐过泥,才烧造这个物儿,与苏州邓浆砖一个样儿做法。如今那里寻去!”夸了一回。西门庆唤茶来吃了,因问:“常二哥几时搬过去?”伯爵道:“从兑了银子三日就搬过去了。昨见好日子,买了些杂货儿,门首把铺儿也开了。就是常二嫂兄弟,替他在铺里看银子儿。”西门庆道:“俺每几时买些礼儿,休要人多了,再邀谢子纯你三四位,我家里整理菜儿抬了去——休费烦常二哥一些东西——叫两个妓者,咱每替他暖暖房,耍一日。”常峙节道:“小弟有心也要请哥坐坐,算计来不敢请。地方儿窄狭,只怕亵渎了哥。”西门庆道:“没的扯淡,那里又费你的事起来。如今使小厮请将谢子纯来,和他说说。”即令琴童儿:“快请你谢爹去!”伯爵因问:“哥,你那日叫那两个去?”西门庆笑道:“叫将郑月儿和洪四儿去罢。”伯爵道:“哥,你是个人,你请他就不对我说声,我怎的也知道了?比李挂儿风月如何?”西门庆道:“通色丝子女不可言!”伯爵道:“他怎的前日你生日时,那等不言语,扭扭的,也是个肉佞贼小淫妇儿。”西门庆道:“等我到几时再去着,也携带你走走。你月娘会打的好双陆,你和他打两贴双陆。”伯爵道:“等我去混那小淫妇儿,休要放了他!”西门庆道:“你这歪狗才,不要恶识他便好。”正说着,谢希大到了,声诺毕,坐下。西门庆道:“常二哥如此这般,新有了华居,瞒着俺每,已搬过去了。咱每人随意出些分资,休要费烦他丝毫。我这里整治停当,教小厮抬到他府上,我还叫两个妓者,咱耍一日何如?”谢希大道:“哥吩咐每人出多少分资,俺每都送到哥这里来就是了。还有那几位?”西门庆道:“再没人,只这三四个儿,每人二星银子就够了。”伯爵道:“十分人多了,他那里没地方儿。”

正说着,只见琴童来说:“吴大舅来了。”西门庆道:“请你大舅这里来坐。”不一时,吴大舅进入轩内,先与三人作了揖,然后与西门庆叙礼坐下。小厮拿茶上来,同吃了茶,吴大舅起身说道:“请姐夫到后边说句话儿。”西门庆连忙让大舅到后边月娘房里。月娘还在卷棚内与众姊妹吃酒听唱,听见说:“大舅来了,爹陪着在后边说话哩。”一面走到上房,见大舅道了万福,叫小玉递上茶来。大舅向袖中取出十两银子递与月娘,说道:“昨日府里才领了三锭银子,姐夫且收了这十两,余者待后次再送来。”西门庆道:“大舅,你怎的这般计较?且使着,慌怎的!”大舅道:“我恐怕迟了姐夫的。”西门庆因问:“仓廒修理的也将完了?”大舅道:“还得一个月终完。”西门庆道:“工完之时,一定抚按有些奖励。”大舅道:“今年考选军政在迩,还望姐夫扶持,大巡上替我说说。”西门庆道:“大舅之事,都在于我。”

说毕话,月娘道:“请大舅前边同坐罢。”大舅道:“我去罢,只怕他三位来有甚么话说。”西门庆道:“没甚么话。常二哥新近问我借了几两银子,买下了两间房子,已搬过去了,今日买了些礼儿来谢我,节间留他每坐坐。大舅来的正好。”于是让至前边坐了。月娘连忙叫厨下打发莱儿上去。琴童与王经先安放八仙桌席端正,西门庆旋教开库房,拿出一坛夏提刑家送的菊花酒来。打开碧靛清,喷鼻香,未曾筛,先搀一瓶凉水,以去其蓼辣之性,然后贮于布甑内,筛出来醇厚好吃,又不说葡萄酒。叫王经用小金钟儿斟一杯儿,先与吴大舅尝了,然后,伯爵等每人都尝讫,极口称羡不已。须臾,大盘大碗摆将上来,众人吃了一顿。然后才拿上酿螃蟹并两盘烧鸭子来,伯爵让大舅吃。连谢希大也不知是甚么做的,这般有味,酥脆好吃。西门庆道:“此是常二哥家送我的。”大舅道:“我空痴长了五十二岁,并不知螃蟹这般造作,委的好吃!”伯爵又问道:“后边嫂子都尝了尝儿不曾?”西门庆道:“房下每都有了。”伯爵道:“也难为我这常嫂子,真好手段儿!”常峙节笑道:“贱累还恐整理的不堪口,教列位哥笑话。”

吃毕螃蟹,左右上来斟酒,西门庆令春鸿和书童两个,在旁一递一个歌唱南曲。应伯爵忽听大卷棚内弹筝歌唱之声,便问道:“哥,今日李桂姐在这里?不然,如何这等音乐之声?”西门庆道:。“你再听,看是不是?”伯爵道:“李桂姐不是,就是吴银儿。”西门庆道:“你这花子单管只瞎诌。倒是个女先生。”伯爵道:“不是郁大姐?”西门庆道:“不是他,这个是申二姐。年小哩,好个人材,又会唱。”伯爵道:“真个这等好?哥怎的不牵出来俺每瞧瞧?就唱个儿俺每听。”西门庆道:“今日你众娘每大节间,叫他来赏重阳顽耍,偏你这狗才耳朵尖,听的见!”伯爵道:“我便是千里眼,顺风耳,随他四十里有蜜蜂儿叫,我也听见了。”谢希大道:“你这花子,两耳朵似竹签儿也似,愁听不见!”两个又顽笑了一回,伯爵道:“哥,你好歹叫他出来,俺每见见儿,俺每不打紧,教他只当唱个与老舅听也罢了。休要就古执了。”西门庆吃他逼迫不过,一面使王经领申二姐出来唱与大舅听。不一时,申二姐来,望上磕了头起来,旁边安放交床儿与他坐下。伯爵问申二姐:“青春多少?”申二姐回道:“属牛的,二十一岁了。”又问:“会多少小唱?”申二姐道:“琵琶筝上套数小唱,也会百十来套。”伯爵道:“你会许多唱也够了。”西门庆道:“申二姐,你拿琵琶唱小词儿罢,省的劳动了你。说你会唱‘四梦八空’,你唱与大舅听。”吩咐王经、书童儿,席间斟上酒。那申二姐款跨鲛绡,微开檀口,慢慢唱着,众人饮酒不题。

且说李瓶儿归到房中,坐净桶,下边似尿的一般,只顾流将起来,登时流的眼黑了。起来穿裙子,忽然一阵旋晕,向前一头撞倒在地。饶是迎春在旁搊扶着,还把额角上磕伤了皮。和奶子搊到炕上,半日不省人事。慌了迎春,忙使绣春:“快对大娘说去!”绣春走到席上,报与月娘众人。月娘撇了酒席,与众姐妹慌忙走来看视。见迎春、奶子两个搊扶着他坐在炕上,不省人事。便问:“他好好的进屋里,端的怎么来就不好了?”迎春揭开净桶与月娘瞧,把月娘唬了一跳。说道:“他刚才只怕吃了酒,助赶的他血旺了,流了这些。”玉楼、金莲都说:“他几曾大吃酒来!”一面煎灯心姜汤灌他。半晌苏醒过来,才说出话儿来。月娘问:“李大姐,你怎的来?”李瓶儿道:“我不怎的。坐下桶子起来穿裙子,只见眼儿前黑黑的一块子,就不觉天旋地转起来,由不的身子就倒了。”月娘便要使来安儿:“请你爹进来——对他说,教他请任医官来看你。”李瓶儿又嗔教请去:“休要大惊小怪,打搅了他吃酒。”月娘吩咐迎春:“打铺教你娘睡罢。”月娘于是也就吃不成酒了,吩咐收拾了家伙,都归后边去了。

西门庆陪侍吴大舅众人,至晚归到后边月娘房中。月娘告诉李瓶儿跌倒之事,西门庆慌走到前边来看视。见李瓶儿睡在炕上,面色蜡查黄了,扯着西门庆衣袖哭泣。西门庆问其所以,李瓶儿道:“我到屋里坐杩子,不知怎的,下边只顾似尿也一般流将起来,不觉眼前一块黑黑的。起来穿裙子,天旋地转,就跌倒了。”西门庆见他额上磕伤一道油皮,说道,“丫头都在那里,不看你,怎的跌伤了面貌?”李瓶儿道:“还亏大丫头都在跟前,和奶子搊扶着我,不然,还不知跌的怎样的。”西门庆道:“我明早请任医官来看你。”当夜就在李瓶儿对面床上睡了一夜。

次日早晨,往衙门里去,旋使琴童请任医官去了。直到晌午才来。西门庆先在大厅上陪吃了茶,使小厮说进去。李瓶儿房里收拾干净,熏下香,然后请任医官进房中。诊毕脉,走出外边厅上,对西门庆说:“老夫人脉息,比前番甚加沉重,七情伤肝,肺火太旺,以致木旺土虚,血热妄行,犹如山崩而不能节制。若所下的血紫者,犹可以调理;若鲜红者,乃新血也。学生撮过药来,若稍止,则可有望;不然,难为矣。”西门庆道:“望乞老先生留神加减,学生必当重谢!”任医官道:“是何言语!你我厚间,又是明用情分,学生无不尽心。”西门庆待毕茶,送出门,随即具一匹杭绢、二两白金,使琴童儿讨将药来,名曰“归脾汤”,乘热吃下去,其血越流之不止。西门庆越发慌了,又请大街口胡太医来瞧。胡太医说是气冲血管,热入血室,亦取将药来。吃下去,如石沉大海一般。

月娘见前边乱着请太医,只留申二姐住了一夜,与了他五钱银子、一件云绢比甲儿并花翠,装了个盒于,就打发他坐轿子去了。花子由自从那日开张吃了酒去,听见李瓶儿不好,使了花大嫂,买了两盒礼来看他。见他瘦的黄恹恹儿,不比往时,两个在屋里大哭了一回。月娘后边摆茶请他吃了。韩道国说:“东门外住的一个看妇人科的赵太医,指下明白,极看得好。前岁,小媳妇月经不通,是他看来。老爹请他来看看六娘,管情就好哩。”西门庆听了,就使琴童和王经两个叠骑着头口,往门外请赵太医去了。

西门庆请了应伯爵来,和他商议道:“第六个房下,甚是不好的重,如之奈何?”伯爵失惊道:“这个嫂子贵恙说好些,怎的又不好起来?”西门庆道:“自从小儿没了,着了忧戚,把病又发了。昨日重阳,我接了申二姐,与他散闷顽耍,他又没好生吃酒,谁知走到屋中就晕起来,一交跌倒,把脸都磕破了。请任医官来看,说脉息比前沉重。吃了药,倒越发血盛了。”伯爵道:“你请胡太医来看,怎的说?”西门庆道:“胡大医说,是气冲了血管,吃了他的,也不见动静。今日韩伙计说,门外一个赵太医,名唤赵龙岗,专科看妇女,我使小厮请去了。把我焦愁的了不的。生生为这孩子不好,白日黑夜思虑起这病来了。妇女人家,又不知个回转,劝着他,又不依你,叫我无法可处。”

正说着,平安来报:“乔亲家爹来了。”西门庆一面让进厅上,同伯爵叙礼坐下。乔大户道:“闻得六亲家母有些不安,特来候问。”西门庆道:“便是。一向因小儿没了,着了忧戚,身上原有些不调,又发起来了。蒙亲家挂念。”乔大户道:“也曾请人来看不曾?”西门庆道:“常吃任后溪的药,昨日又请大街胡先生来看,吃药越发转盛。今日又请门外专看妇人科赵龙岗去了。”乔大户道:“咱县门前住的何老人,大小方脉俱精。他儿子何歧轩,见今上了个冠带医士。亲家何不请他来看看亲家母?”西门庆道:“既是好,等赵龙岗来,来过再请他来看看。”乔大户道:“亲家,依我愚见,不如先请了何老人来,再等赵龙岗来,叫他两个细讲一讲,就论出病原来了。然后下药,无有不效之理。”西门庆道:“亲家说的是。”一面使玳安拿拜帖儿和乔通去请。

那消半晌,何老人到来,与西门庆、乔大户等作了揖,让于上面坐下。西门庆举手道:“数年不见你老人家,不觉越发苍髯皓首。”乔大户又问:“令郎先生肄业盛行?”何老人道:“他逐日县中迎送,也不得闲,倒是老拙常出来看病。”伯爵道:“你老人家高寿了,还这等健朗。”何老人道:“老拙今年痴长八十一岁。”叙毕话,看茶上来吃了,小厮说进去。须臾,请至房中,就床看李瓶儿脉息,旋搊扶起来,坐在炕上,形容瘦的十分狼狈了。但见他——

\[
面如金纸,体似银条。看看减褪丰标,渐渐消磨精彩。隐隐耳虚闻磐响,昏昏眼暗觉萤飞。六脉细沉,一灵缥缈,丧门吊客已临身,扁鹊卢医难下手。
\]

何老人看了脉息,出到厅上,向西门庆、乔大户说道:“这位娘子,乃是精冲了血管起,然后着了气恼。气与血相搏,则血如崩。不知当初起病之由是也不是?”西门庆道:“是便是,却如何治疗?”正论间,忽报:“琴童和王经请了赵先生来了。”何老人便问:“是何人?”西门庆道:“也是伙计举来一医者,你老人家只推不知,待他看了脉息,你老人家和他讲一讲,好下药。”不一时,赵大医从外而入,西门庆与他叙礼毕,然后与众人相见。何、乔二老居中,让他在左,伯爵在右,西门庆主位相陪。吃了茶,赵太医便问:“列位尊长贵姓?”乔大户道:“俺二人一姓何,一姓乔。”伯爵道:“在下姓应。老先想就是赵龙岗先生了。”赵太医答道:“龙岗是贱号。在下以医为业,家祖见为太医院院判,家父见充汝府良医,祖传三辈,习学医术。每日攻习王叔和、东垣勿听子《药性赋》、《黄帝素问》、《难经》、《活人书》、《丹溪纂要》、《丹溪心法》、《洁古老脉诀》、《加减十三方》、《千金奇效良方》、《寿域神方》、《海上方》,无书不读。药用胸中活法,脉明指下玄机。六气四时,辨阴阳之标格;七表八里,定关格之沉浮。风虚寒热之症候,一览无余;弦洪芤石之脉理,莫不通晓。小人拙口钝吻,不能细陈。”何老人听了,道:“敢问看病当以何者为先?”赵太医道:“古人云,望闻问切,神圣功巧。学生先问病,后看脉,还要观其气色。就如子平兼五星一般,才看得准,庶乎不差。”何老人道:“既是如此,请先生进去看看。”西门庆即令琴童:“后边说去,又请了赵先生来了。”

不一时,西门庆陪他进入李瓶儿房中。那李瓶儿方才睡下安逸一回,又搊扶起来,靠着枕褥坐着。这赵太医先诊其左手,次诊右手,便教:“老夫人抬起头来,看看气色。”那李瓶儿真个把头儿扬起来。赵太医教西门庆:“老爹,你问声老夫人,我是谁?”西门庆便教李瓶儿:“你看这位是谁?”那李瓶儿抬头看了一眼,便低声说道:“他敢是太医?”赵先生道:“老爹,不妨事,还认的人哩。”西门庆道:“赵先生,你用心看,我重谢你。”一面看视了半日,说道:“老夫人此病,休怪我说,据看其面色,又诊其脉息,非伤寒,只为杂症,不是产后,定然胎前。”西门庆道:“不是此疾。先生你再仔细诊一诊。”赵先生又沉吟了半晌道:“如此面色这等黄,多管是脾虚泄泻,再不然定是经水不调。”西门庆道:“实说与先生,房下如此这般,下边月水淋漓不止,所以身上都瘦弱了。有甚急方妙药,我重重谢你。”赵先生道:“如何?我就说是经水不调。不打紧处,小人有药。”

西门庆一面同他来到前厅,乔大户、何老人问他甚么病源,赵先生道:“依小人讲,只是经水淋漓。”何老人道:“当用何药治之?”赵先生道:“我有一妙方,用着这几味药材,吃下去管情就好。听我说:

\[
甘草甘遂与碙砂,黎芦巴豆与芫花,姜汁调着生半夏,用乌头杏仁天麻。这几味儿齐加,葱蜜和丸只一挝,清晨用烧酒送下。”
\]
何老人听了,便道:“这等药恐怕太狠毒,吃不得。”赵先生道:“自古毒药苦口利于病。怎么吃不得?”西门庆见他满口胡说,因是韩伙计举保来,不好嚣他,称二钱银子,也不送,就打发他去了。因向乔大户说:“此人原来不知甚么。”何老人道:“老拙适才不敢说,此人东门外有名的赵捣鬼,专一在街上卖杖摇铃,哄过往之人,他那里晓的甚脉息病源!”因说:“老夫人此疾,老拙到家撮两帖药来,遇缘,若服毕经水少减,胸口稍开,就好用药。只怕下边不止,就难为矣。”说毕,起身。

西门庆封白金一两,使玳安拿盒儿讨将药来,晚夕与李瓶儿吃了,并不见分毫动静。吴月娘道:“你也省可与他药吃。他饮食先阻住了,肚腹中有甚么儿,只是拿药淘碌他。前者,那吴神仙算他三九上有血光之灾,今年却不整二十七岁了。你还使人寻这吴神仙去,叫替他打算算那禄马数上如何。只怕犯着甚么星辰,替他禳保禳保。”西门庆听了,旋差人拿帖儿往周守备府里问去。那里回说:“吴神仙云游之人,来去不定。但来,只在城南土地庙下。今岁从四月里,往武当山去了。要打数算命,真武庙外有个黄先生打的好数,一数只要三钱银子,不上人家门。”西门庆随即使陈敬济拿三钱银子,迳到北边真武庙门首黄先生家。门上贴着:“抄算先天易数,每命卦金三钱。”陈敬济向前作揖,奉上卦金,说道:“有一命烦先生推算。”写与他八字:女命,年二十七岁,正月十五日午时。这黄先生把算子一打,就说:“这个命,辛未年庚寅月辛卯日甲午时,理取印绥之格,借四岁行运。四岁己未,十四岁戊午,二十四岁丁巳,三十四岁丙辰。今年流年丁酉,比肩用事,岁伤日干,计都星照命,又犯丧门五鬼,灾杀作炒。夫计都者,阴晦之星也。其象犹如乱丝而无头,变异无常。大运逢之,多主暗昧之事,引惹疾病,主正、二、三、七、九月病灾有损,小口凶殃,小人所算,口舌是非,主失财物。或是阴人大为不利。”抄毕数,敬济拿来家。西门庆正和应伯爵、温秀才坐的,见抄了数来,拿到后边,解说与月娘听。见命中多凶少吉,不觉——

\[
眉间搭上三黄锁,腹内包藏一肚愁。
\]
