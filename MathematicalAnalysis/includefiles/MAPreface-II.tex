%# -*- coding:utf-8 -*-
%%%%%%%%%%%%%%%%%%%%%%%%%%%%%%%%%%%%%%%%%%%%%%%%%%%%%%%%%%%%%%%%%%%%%%%%%%%%%%%%%%%%%
%%  `MAPreface-II.tex'

\begin{preface}
本书是北京大学数学系沈燮昌教授,方企勤、廖可人、李正元副教授合编的《数学分析》一书中的第二册。第一册由方企勤副教授执
笔,第二册主要由沈燮昌教授执笔,第三册由廖可人、李正元副教授执笔。全书三册有我统一看过一遍,并做了一些修改。

本册内容包括定积分,定积分的应用,实数空间,广义积分,数值级数,函数项级数,幂级数和~Fourier~级数等共八章,其中实数空
间这一章由方企勤副教授执笔。

正如第一册前言所说,为了使难点分散和便于理解,我们商定讲极限分成两大部分来讲,在第一册中介绍极限基础。这里,即第二册中
进一步介绍实数空间;从直观的有理数的分割法开始引入实数,然后研究实数是一个有序域,进而证明它是一个实数空间。此外还引入
了连通性,紧性以及完备性等重要概念,并说明这些概念本质上用到些什么内容,这对于今后进一步学习一些抽象空间是有启发的。

对于~Riemann~积分,我们还是用通常直观的方法引入其定义,但是我们说明了,为了保证积分存在,就需要研究函数在每一个小的分割
区间上的偏差性质。从而引入了积分的大和、小和以及它们的下确界及上确界,而~Riemann~和正是位于这两者之间,因此自然地引入
上、下积分的概念。这样一来,我们就能给出~Riemann~积分存在的另两个等价的定理,这给具体使用带来了较大的方便。

对于一些概念的引进,我们尽量给予直观的解释以利于读者理解这些概念。例如在讲有理数分划能确定一个实数时,我们用形象“排队”
的说法,只要知道前面是什么人,而后面又是什么人后就可以确定自身的位置。对于~Abel~变换,除了给出这个变换的分析表达式以
外,还给出了对面积进行不同的分法而得到的同一个结果的解释。

在本册中还渗透了无穷小量阶的思想,这对研究级数和广义积分的收敛与发散性更能看清其本质,而且也易于判别。

此外,还给出定积分的几种近似计算方法并利用简洁的方法给出了误差估计式。考虑到目前是广泛地使用电子计算机的时代,初步了解
一些计算方法以及知道误差估计的重要性,这对学生来说无疑是有好处的。

本册还给出了很多例题,由易而难,有些例子不仅是较有趣,而且也给出一些重要的结果。这些例题可以使学生加深对理论的理解并且
对如何灵活地使用学到的理论会起到重要的示范作用。

这里还介绍了两个逼近定理,过去在常见的教科书中,总是用~Bernstein~多项式来实现多项式逼近,这个多项式对于初学者来说是很难
理解的。这里介绍了多项式逼近定理的~Lebesgue~证明,首先用折线来逼近连续函数,然后再利用幂级数展开的方法来逼近折线函数。%
这样的证明比较直观、易懂,且也是所学过的方法的灵活运用。

作者在书写本书过程中深深地感到,对于像这类基本内容都已经比较成熟的教科书,如何进行改革,一方面要使学生容易接受,能够通过
学习掌握一些最基本的知识且在能力上有所提高,另一方面又能适当地现代化,这是一件很不容易做到的事情。希望广大读者多多地提出
宝贵意见。

作者感谢李正元副教授,他仔细地阅读了原稿,并提出了很多宝贵的意见。作者也感谢欧阳光中副教授、董延闿教授仔细地审阅了原稿,%
并提出了很多改进意见。
\Sign{1985~年~3~月}
\end{preface}

\endinput
%%
%% End of file `MAPreface-II.tex'.