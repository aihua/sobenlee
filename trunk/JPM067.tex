%# -*- coding:utf-8 -*-
%%%%%%%%%%%%%%%%%%%%%%%%%%%%%%%%%%%%%%%%%%%%%%%%%%%%%%%%%%%%%%%%%%%%%%%%%%%%%%%%%%%%%


\chapter{西门庆书房赏雪\KG 李瓶儿梦诉幽情}


词曰:

\[
朔风天,琼瑶地。冻色连波,波上寒烟砌。山隐彤云云接水,衰草无情,想在彤云内。黯香魂,追苦意。夜夜除非,好梦留人睡。残月高楼休独倚,酒入愁肠,化作相思泪。
\]

话说西门庆归后边,辛苦的人,直睡至次日日高还未起来。有来兴儿进来说:“搭彩匠外边伺候,请问拆棚。”西门庆骂了来兴儿几句,说:“拆棚教他拆就是了,只顾问怎的!”搭彩匠一面卸下席绳松条,送到对门房子里堆放不题。玉箫进房说:“天气好不阴的重。”西门庆令他向暖炕上取衣裳穿,要起来。月娘便说:“你昨日辛苦了一夜,天阴,大睡回儿也好。慌的老早爬起去做甚么?就是今日不往衙门里去也罢了。”西门庆道:“我不往衙门里去,只怕翟亲家那人来讨书。”月娘道:“既是恁说,你起去,我去叫丫鬟熬下粥等你吃。”西门庆也不梳头洗面,披着绒衣,戴着毡巾,径走到花园里书房中。

原来自从书童去了,西门庆就委王经管花园书房,春鸿便收拾大厅前书房。冬月间,西门庆只在藏春阁书房中坐。那里烧下地炉暖炕,地平上又放着黄铜火盆,放下油单绢暖帘来。明间内摆着夹枝桃,各色菊花,清清瘦竹,翠翠幽兰,里面笔砚瓶梅,琴书潇洒。西门庆进来,王经连忙向流金小篆炷爇龙涎。西门庆使王经:“你去叫来安儿请你应二爹去。”王经出来吩咐来安儿请去了。只见平安走来对王经说:“小周儿在外边伺候。”王经走入书房对西门庆说了,西门庆叫进小周儿来,磕了头,说道:“你来得好,且与我篦篦头,捏捏身上。”因说:“你怎一向不来?”小周儿道:“小的见六娘没了,忙,没曾来。”西门庆于是坐在一张醉翁椅上,打开头发教他整理梳篦。只见来安儿请的应伯爵来了,头戴毡帽,身穿绿绒袄子,脚穿一双旧皂靴棕套,掀帘子进来唱喏。西门庆正篦头,说道:“不消声喏,请坐。”伯爵拉过一张椅子来,就着火盆坐下。西门庆道:“你今日如何这般打扮?”伯爵道:“你不知,外边飘雪花儿哩,好不寒冷。昨日家去,鸡也叫了,今日白爬不起来。不是大官儿去叫,我还睡哩。哥,你好汉,还起的早。若是我,成不的。”西门庆道:“早是你看着,我怎得个心闲!自从发送他出去了,又乱着接黄太尉,念经,直到如今。今日房下说:‘你辛苦了,大睡回起去。’我又记挂着翟亲家人来讨回书,又看着拆棚,二十四日又要打发韩伙计和小价起身。丧事费劳了人家,亲朋罢了,士大夫官员,你不上门谢谢孝,礼也过不去。”伯爵道:“正是,我愁着哥谢孝这一节。少不的只摘拨谢几家要紧的,胡乱也罢了。其余相厚的,若会见,告过就是了。谁不知你府上事多,彼此心照罢。”

正说着,只见画童儿拿了两盏酥油白糖熬的牛奶子。伯爵取过一盏,拿在手内,见白潋潋鹅脂一般酥油飘浮在盏内,说道:“好东西,滚热!”呷在口里,香甜美味,那消气力,几口就喝没了。西门庆直待篦了头,又教小周儿替他取耳,把奶子放在桌上,只顾不吃。伯爵道:“哥且吃些不是?可惜放冷了。象你清晨吃恁一盏儿,倒也滋补身子。”西门庆道:“我且不吃,你吃了,停会我吃粥罢。”那伯爵得不的一声,拿在手中,又一吸而尽。西门庆取毕耳,又叫小周儿拿木滚子滚身上,行按摩导引之术。伯爵问道:“哥滚着身子,也通泰自在么?”西门庆道:“不瞒你说,象我晚夕身上常发酸起来,腰背疼痛,不着这般按捏,通了不得!”伯爵道:“你这胖大身子,日逐吃了这等厚味,岂无痰火!”西门庆道:“任后溪常说:‘老先生虽故身体魁伟,而虚之太极。’送了我一罐儿百补延龄丹,说是林真人合与圣上吃的,教我用人乳常清晨服。我这两日心上乱,也还不曾吃。你们只说我身边人多,终日有此事,自从他死了,谁有甚么心绪理论此事!”

正说着,只见韩道国进来,作揖坐下,说:“刚才各家都来会了,船已雇下,准在二十四日起身。”西门庆吩咐:“甘伙计攒下帐目,兑了银子,明日打包。”因问:“两边铺子里卖下多少银两?”韩道国说:“共凑六千余两。”西门庆道:“兑二千两一包,着崔本往湖州买绸子去。那四千两,你与来保往松江贩布,过年赶头水船来。你每人先拿五两银子,家中收拾行李去。”韩道国道:“又一件:小人身从郓王府,要正身上直,不纳官钱如何处?”西门庆道:“怎的不纳官钱?象来保一般也是郓王差事,他每月只纳三钱银子。”韩道国道:“保官儿那个,亏了太师老爷那边文书上注过去,便不敢缠扰。小人乃是祖役,还要勾当余丁。”西门庆道:“既是如此,你写个揭帖,我央任后溪到府中替你和王奉承说,把你名字注销,常远纳官钱罢。你每月只委人打米就是了。”韩伙计作揖谢了。伯爵道:“哥,你替他处了这件事,他就去也放心。”少顷,小周滚毕身上,西门庆往后边梳头去了,吩咐打发小周儿吃点心。

良久,西门庆出来,头戴白绒忠靖冠,身披绒氅,赏了小周三钱银子。又使王经:“请你温师父来。”不一时,温秀才峨冠博带而至。叙礼已毕,左右放桌儿,拿粥来,伯爵与温秀才上坐,西门庆关席,韩道国打横。西门庆吩咐来安儿:“再取一盏粥、一双筷儿,请姐夫来吃粥。”不一时,陈敬济来到,头戴孝巾,身穿白绸道袍,与伯爵等作揖,打横坐下。须臾吃了粥,收下家火去,韩道国起身去了。西门庆因问温秀才:“书写了不曾?”温秀才道:“学生已写稿在此,与老先生看过,方可誊真。”一面袖中取出,递与西门庆观看。其书曰:

\[
寓清河眷生西门庆端肃书复大硕德柱国云峰老亲丈大人先生台下:自从京邸邂逅,不觉违越光仪,倏忽半载。生不幸闺人不禄,特蒙亲家远致赙仪,兼领悔教,足见为我之深且厚也。感刻无任,而终身不能忘矣。但恐一时官守责成有所疏陋之处,企仰门墙有负荐拔耳,又赖在老爷钧前常为锦覆。则生始终蒙恩之处,皆亲家所赐也。今因便鸿谨候起居,不胜驰恋,伏惟照亮,不宣。外具扬州绉纱汗巾十方、色绫汗巾十方、拣金挑牙二十付、乌金酒钟十个,少将远意,希笑纳。
\]

西门庆看毕,即令陈敬济书房内取出人事来,同温秀才封了,将书誊写锦笺,弥封停当,印了图书。另外又封五两白银与下书人王玉,不在话下。

一回见雪下的大了,西门庆留下温秀才在书房中赏雪。揩抹桌儿,拿上案酒来。只见有人在暖帘外探头儿,西门庆问是谁,王经说:“是郑春。”西门庆叫他进来。那郑春手内拿着两个盒儿,举的高高的,跪在当面,上头又搁着个小描金方盒儿,西门庆问是甚么,郑春道:“小的姐姐月姐,知道昨日爹与六娘念经辛苦了,没甚么,送这两盒儿茶食儿来,与爹赏人。”揭开,一盒果馅顶皮酥、一盒酥油泡螺儿。郑春道:“此是月姐亲手拣的。知道爹好吃此物,敬来孝顺爹。”西门庆道:“昨日多谢你家送茶,今日你月姐费心又送这个来。”伯爵道:“好呀!拿过来,我正要尝尝!死了我一个女儿会拣泡螺儿,如今又是一个女儿会拣了。”先捏了一个放在口内,又拈了一个递与温秀才,说道:“老先儿,你也尝尝。吃了牙老重生,抽胎换骨。眼见希奇物,胜活十年人。”温秀才呷在口内,入口而化,说道:“此物出于西域,非人间可有。沃肺融心,实上方之佳味。”西门庆又问:“那小盒儿内是甚么?”郑春悄悄跪在西门庆跟前,递上盒儿,说:“此是月姐捎与爹的物事。”西门庆把盒子放在膝盖儿上,揭开才待观看,早被伯爵一手挝过去,打开是一方回纹锦同心方胜桃红绫汗巾儿,里面裹着一包亲口嗑的瓜仁儿。伯爵把汗巾儿掠与西门庆,将瓜仁两把喃在口里都吃了。比及西门庆用手夺时,只剩下没多些儿,便骂道:“怪狗才,你害馋痨馋痞!留些儿与我见见儿,也是人心。”伯爵道:“我女儿送来,不孝顺我,再孝顺谁?我儿,你寻常吃的够了。”西门庆道:“温先儿在此,我不好骂出来,你这狗才,忒不象模样!”一面把汗巾收入袖中,吩咐王经把盒儿掇到后边去。

不一时,杯盘罗列,筛上酒来。才吃了一巡酒,玳安儿来说:“李智、黄四关了银子,送银子来了。”西门庆问多少,玳安道:“他说一千两,余者再一限送来。”伯爵道:“你看这两个天杀的,他连我也瞒了不对我说。嗔道他昨日你这里念经他也不来,原来往东平府关银子去了。你今收了,也少要发银子出去了。这两个光棍,他揽的人家债多了,只怕往后后手不接。昨日,北边徐内相发恨,要亲往东平府自家抬银子去。只怕他老牛箍嘴箍了去,却不难为哥的本钱!”西门庆道:“我不怕他。我不管甚么徐内相李内相,好不好把他小厮提在监里坐着,不怕他不与我银子。”一面教陈敬济:“你拿天平出去收兑了他的就是了。我不出去罢。”

良久,陈敬济走来回话说:“银子已兑足一千两,交入后边,大娘收了。黄四说,还要请爹出去说句话儿。”西门庆道:“你只说我陪着人坐着哩。左右他只要捣合同,教他过了二十四日来罢。”敬济道:“不是。他说有桩事儿要央烦爹。”西门庆道:“甚么事?等我出去。”一面走到厅上,那黄四磕头起来,说:“银子一千两,姐夫收了。余者下单我还。小人有一桩事儿央烦老爹。”说着磕在地下哭了。西门庆拉起来道:“端的有甚么事,你说来。”黄四道:“小的外父孙清,搭了个伙计冯二,在东昌府贩绵花。不想冯二有个儿子冯淮,不守本分,要便锁了门出去宿娼。那日把绵花不见了两大包,被小人丈人说了两句,冯二将他儿子打了两下。他儿子就和俺小舅子孙文相厮打起来,把孙文相牙打落了一个,他亦把头磕伤。被客伙中解劝开了。不想他儿子到家,迟了半月,破伤风身死。他丈人是河西有名土豪白五,绰号白千金,专一与强盗做窝主,教唆冯二,具状在巡按衙门朦胧告下来,批雷兵备老爹问。雷老爹又伺候皇船,不得闲,转委本府童推官问。白家在童推官处使了钱,教邻见人供状,说小人丈人在旁喝声来。如今童推官行牌来提俺丈人。望乞老爹千万垂怜,讨封书对雷老爹说,宁可监几日,抽上文书去,还见雷老爹问,就有生路了。他两人厮打,委的不管小人丈人事,又系歇后身死,出于保辜限外。先是他父冯二打来,何必独赖孙文相一人身上?”西门庆看了说帖,写着:“东昌府见监犯人孙清、孙文相,乞青目。”因说:“雷兵备前日在我这里吃酒,我只会了一面,又不甚相熟,我怎好写书与他?”黄四就跪下哭哭啼啼哀告说:“老爹若不可怜见,小的丈人子父两个就都是死数了。如今随孙文相出去罢了,只是分豁小人外父出来,就是老爹莫大之恩。小人外父今年六十岁,家下无人,冬寒时月再放在监里,就死罢了。”西门庆沉吟良久,说:“也罢,我转央钞关钱老爹和他说说去——与他是同年,都是壬辰进士。”黄四又磕下头去,向袖中取出“一百石白米”帖儿递与西门庆,腰里就解两封银子来。西门庆不接,说道:“我那里要你这行钱!”黄四道:“老爹不稀罕,谢钱老爹也是一般。”西门庆道:“不打紧,事成我买礼谢他。”

正说着,只见应伯爵从角门首出来,说:“哥,休替黄四哥说人情。他闲时不烧香,忙时抱佛腿。昨日哥这里念经,连茶儿也不送,也不来走走儿,今日还来说人情!”那黄四便与伯爵唱喏,说道:“好二叔,你老人家杀人哩!我因这件事,整走了这半月,谁得闲来?昨日又去府里领这银子,今日一来交银子,就央说此事,救俺丈人。老爹再三不肯收这礼物,还是不下顾小人。”伯爵看见一百两雪花官银放在面前,因问:“哥,你替他去说不说?”西门庆道:“我与雷兵备不熟,如今要转央钞关钱主政替他说去。到明日,我买分礼谢老钱就是了,又收他礼做甚么?”伯爵道:“哥,你这等就不是了。难道他来说人情,哥你倒陪出礼去谢人?也无此道理。你不收,恰似嫌少的一般。你依我收下。虽你不稀罕,明日谢钱公也是一般。黄四哥在这里听着:看你外父和你小舅子造化,这一回求了书去,难得两个都没事出来。你老爹他恒是不稀罕你钱,你在院里老实大大摆一席酒,请俺们耍一日就是了。”黄四道:“二叔,你老人家费心,小人摆酒不消说,还叫俺丈人买礼来,磕头酬谢你老人家。不瞒说,我为他爷儿两个这一场事,昼夜替他走跳,还寻不出个门路来。老爹再不可怜怎了!”伯爵道:“傻瓜,你搂着他女儿,你不替他上紧谁上紧?”黄四道:“房下在家只是哭。”西门庆被伯爵说着,把礼帖收了,说礼物还令他拿回去。黄四道:“你老人家没见好大事,这般多计较!”就往外走。伯爵道:“你过来,我和你说:你书几时要?”黄四道:“如今紧等着救命,望老爹今日写了书,差下人,明早我使小儿同去走遭。不知差那位大官儿去,我会他会。”西门庆道:“我就替你写书。”因叫过玳安来吩咐:“你明日就同黄大官一路去。”

那黄四见了玳安,辞西门庆出门。走到门首,问玳安要盛银子的褡裢。玳安进入后边,月娘房里正与玉箫、小玉裁衣裳,见玳安站着等褡裢,玉箫道:“使着手,不得闲誊。教他明日来与他就是了。”玳安道:“黄四等紧着明日早起身东昌府去,不得来了,你誊誊与他罢。”月娘便说:“你拿与他就是了,只教人家等着。”玉箫道:“银子还在床地平上掠着不是?”走到里间,把银子往床上只一倒,掠出褡裢来,说:“拿了去!怪囚根子,那个吃了他这条褡裢,只顾立叮蚂蝗的要!”玳安道:“人家不要,那个好来取的!”于是拿了出去,走到仪门首,还抖出三两一块麻姑头银子来。原来纸包破了,怎禁玉箫使性子那一倒,漏下一块在褡裢底内。玳安道:“且喜得我拾个白财。”于是褪入袖中。到前边递与黄四,约会下明早起身。

且说西门庆回到书房中,即时教温秀才修了书,付与玳安不题。一面觑那门外下雪,纷纷扬扬,犹如风飘柳絮,乱舞梨花相似。西门庆另打开一坛双料麻姑酒,教春鸿用布甑筛上来,郑春在旁弹筝低唱,西门庆令他唱一套“柳底风微”。正唱着,只见琴童进来说:“韩大叔教小的拿了这个帖儿与爹瞧。”西门庆看了,吩咐:“你就拿往门外任医官家,替他说说去。央他明日到府中承奉处替他说说,注销差事。”琴童道:“今日晚了,小的明早去罢。”西门庆道:“明早去也罢。”不一时,来安儿用方盒拿了八碗下饭,又是两大盘玫瑰鹅油烫面蒸饼,连陈敬济共四人吃了。西门庆教王经盒盘儿拿两碗下饭、一盘点心与郑春吃,又赏了他两大钟酒。郑春跪禀:“小的吃不的。”伯爵道:“傻孩子,冷呵呵的,你爹赏你不吃。你哥他怎的吃来?”郑春道:“小的哥吃的,小的本吃不的。”伯爵道:“你只吃一钟罢,那一钟我教王经替你吃罢。”王经说道:“二爹,小的也吃不的。”伯爵道:“你这傻孩儿,你就替他吃些儿也罢。休说一个大分上,自古长者赐,少者不敢辞。”一面站起来说:“我好歹教你吃这一杯。”那王经捏着鼻子,一吸而饮。西门庆道:“怪狗才,小行货子他吃不的,只恁奈何他!”还剩下半盏,应伯爵教春鸿替他吃了,就要令他上来唱南曲。西门庆道:“咱每和温老先儿行个令,饮酒之时教他唱便有趣。”于是教王经取过骰盆儿,“就是温老先儿先起。”温秀才道:“学生岂敢僭,还从应老翁来。”因问:“老翁尊号?”伯爵道:“在下号南坡。”西门庆戏道:“老先生你不知,他孤老多,到晚夕桶子掇出来,不敢在左近倒,恐怕街坊人骂,教丫头直掇到大南首县仓墙底下那里泼去,因起号叫做‘南泼’。”温秀才笑道:“此‘坡’字不同。那‘泼’字乃点水边之‘发’,这‘坡’字却是‘土’字旁边着个‘皮’字。”西门庆道:“老先儿倒猜得着,他娘子镇日着皮子缠着哩。”温秀才笑道:“岂有此说?”伯爵道:“葵轩,你不知道,他自来有些快伤叔人家。”温秀才道:“自古言不亵不笑。”伯爵道:“老先儿,误了咱每行令,只顾和他说甚么,他快屎口伤人!你就在手,不劳谦逊。”温秀才道:“掷出几点,不拘诗词歌赋,要个‘雪’字,就照依点数儿上。说过来,饮一小杯;说不过来,吃一大盏。”温秀才掷了个幺点,说道:“学生有了:雪残鸂鶒亦多时。”推过去,该应伯爵行,掷出个五点来。伯爵想了半日,想不起来,说:“逼我老人家命也!”良久,说道:“可怎的也有了。”说道:“雪里梅花雪里开。——好不好?”温秀才道:“南老说差了,犯了两个‘雪’字,头上多了一个‘雪’字。”伯爵道:“头上只小雪,后来下大雪来了。”西门庆道:“这狗才,单管胡说。”教王经斟上大钟,春鸿拍手唱南曲《驻马听》:

\[
寒夜无茶,走向前村觅店家。这雪轻飘僧舍,密洒歌楼,遥阻归槎。江边乘兴探梅花,庭中欢赏烧银蜡。一望无涯,有似灞桥柳絮满天飞下。
\]

伯爵才待拿起酒来吃,只见来安儿后边拿了几碟果食,内有一碟酥油泡螺,又一碟黑黑的团儿,用桔叶裹着。伯爵拈将起来,闻着喷鼻香,吃到口犹如饴蜜,细甜美味,不知甚物。西门庆道:“你猜?”伯爵道:“莫非是糖肥皂?”西门庆笑道:“糖肥皂那有这等好吃。”伯爵道:“待要说是梅酥丸,里面又有核儿。”西门庆道:“狗才过来,我说与你罢,你做梦也梦不着。是昨日小价杭州船上捎来,名唤做衣梅。都是各样药料和蜜炼制过,滚在杨梅上,外用薄荷、桔叶包裹,才有这般美味。每日清晨噙一枚在口内,生津补肺,去恶味,煞痰火,解酒克食,比梅酥丸更妙。”伯爵道:“你不说,我怎的晓得。”因说:“温老先儿,咱再吃个儿。”教王经:“拿张纸儿来,我包两丸儿,到家捎与你二娘吃。”又拿起泡螺儿来问郑春:“这泡螺儿果然是你家月姐亲手拣的?”郑春跪下说:“二爹,莫不小的敢说谎?不知月姐费了多少心,只拣了这几个儿来孝顺爹。”伯爵道:“可也亏他,上头纹溜,就象螺蛳儿一般,粉红、纯白两样儿。”西门庆道:“我儿,此物不免使我伤心。惟有死了的六娘他会拣,他没了,如今家中谁会弄他!”伯爵道:“我头里不说的,我愁甚么?死了一个女儿会拣泡螺儿孝顺我,如今又钻出个女儿会拣了。偏你也会寻,寻的都是妙人儿。”西门庆笑的两眼没缝儿,赶着伯爵打,说:“你这狗才,单管只胡说。”温秀才道:“二位老先生可谓厚之至极。”伯爵道:“老先儿你不知,他是你小侄人家。”西门庆道:“我是他家二十年旧孤老。”陈敬济见二人犯言,就起身走了。那温秀才只是掩口而笑。

须臾,伯爵饮过大钟,次该西门庆掷骰儿。于是掷出个七点来,想了半日说:“我说《香罗带》上一句唱:‘东君去意切,梨花似雪。’”伯爵道:“你说差了,此在第九个字上了,且吃一大钟。”于是流沿儿斟了一银衢花钟,放在西门庆面前,教春鸿唱,说道:“我的儿,你肚子里裹枣核解板儿——能有几句!”春鸿又拍手唱了一个。看看饮酒至昏,掌烛上来。西门庆饮过,伯爵道:“姐夫不在,温老先生你还该完令。”温秀才拿起骰儿,掷出个幺点,想了想,见壁上挂着一幅吊屏,泥金书一联:“风飘弱柳平桥晚;雪点寒梅小院春。”就说了末后一句。伯爵道:“不算,不算,不是你心上发出来的。该吃一大钟。”春鸿斟上,那温秀才不胜酒力,坐在椅上只顾打盹,起来告辞。伯爵还要留他,西门庆道:“罢罢!老先儿他斯文人,吃不的。”令画童儿:“你好好送你温师父那边歇去。”温秀才得不的一声,作别去了。伯爵道:“今日葵轩不济,吃了多少酒儿?就醉了。”于是又饮够多时,伯爵起身说:“地下滑,我也酒够了。”因说:“哥,明日你早教玳安替他下书去。”西门庆道:“你不见我交与他书,明日早去了。”伯爵掀开帘子,见天阴地下滑,旋要了个灯笼,和郑春一路去。西门庆又与了郑春五钱银子,盒内回了一罐衣梅,捎与他姐姐郑月儿吃。临出门,西门庆因戏伯爵:“你哥儿两个好好去。”伯爵道:“你多说话。父子上山,各人努力。好不好,我如今就和郑月儿那小淫妇儿答话去。”说着,琴童送出门去了。

西门庆看收了家伙,扶着来安儿,打灯笼入角门,从潘金莲门首过,见角门关着,悄悄就往李瓶儿房里来。弹了弹门,绣春开了门,来安就出去了。西门庆进入明间,见李瓶儿影,就问:“供养了羹饭不曾?”如意儿就出来应道:“刚才我和姐供养了。”西门庆椅上坐了,迎春拿茶来吃了。西门庆令他解衣带,如意儿就知他在这房里歇,连忙收拾床铺,用汤婆熨的被窝暖洞洞的,打发他歇下。绣春把角门关了,都在明间地平上支着板凳,打铺睡下。西门庆要茶吃,两个已知科范,连忙撺掇奶子进去和他睡。老婆脱衣服钻入被窝内,西门庆乘酒兴服了药,那话上使了托子,老婆仰卧炕上,架起腿来,极力鼓捣,没高低扇磞,扇磞的老婆舌尖冰冷,淫水溢下,口中呼“达达”不绝。夜静时分,其声远聆数室。西门庆见老婆身上如绵瓜子相似,用一双胳膊搂着他,令他蹲下身子,在被窝内咂\textuni{23B20}\textuni{23B36},老婆无不曲体承奉。西门庆说:“我儿,你原来身体皮肉也和你娘一般白净,我搂着你,就如和他睡一般。你须用心伏侍我,我看顾你。”老婆道:“爹没的说,将天比地,折杀奴婢!奴婢男子汉已没了,爹不嫌丑陋,早晚只看奴婢一眼儿就够了。”西门庆便问:“你年纪多少?”老婆道:“我今年属免的,三十一岁了。”西门庆道:“你原来小我一岁。”见他会说话儿,枕上又好风月,心下甚喜。早晨起来,老婆伏侍拿鞋袜,打发梳洗,极尽殷勤,把迎春、绣春打靠后。又问西门庆讨葱白绸子:“做披袄子,与娘穿孝。”西门庆一一许他。就教小厮铺子里拿三匹葱白绸来:“你每一家裁一件。”瞒着月娘,背地银钱、衣服、首饰,甚么不与他!

次日,潘金莲就打听得知,走到后边对月娘说:“大姐姐,你不说他几句!贼没廉耻货,昨日悄悄钻到那边房里,与老婆歇了一夜。饿眼见瓜皮,甚么行货子,好的歹的揽搭下。不明不暗,到明日弄出个孩子来算谁的?又象来旺儿媳妇子,往后教他上头上脸,甚么张致!”月娘道:“你们只要栽派教我说,他要了死了的媳妇子,你每背地都做好人儿,只把我合在缸底下。我如今又做傻子哩!你每说只顾和他说,我是不管你这闲帐。”金莲见月娘这般说,一声儿不言语,走回房去了。

西门庆早起见天晴了,打发玳安往钱主事家下书去了。往衙门回来,平安儿来禀:“翟爹人来讨书。”西门庆打发书与他,因问那人:“你怎的昨日不来取?”那人说:“小的又往巡抚侯爷那里下书来,耽搁了两日。”说毕,领书出门。西门庆吃了饭就过对门房子里,看着兑银、打包、写书帐。二十四日烧纸,打发韩伙计、崔本并后生荣海、胡秀五人起身往南边去。写了一封书捎与苗小湖,就谢他重礼。

看看过了二十五六,西门庆谢毕孝,一日早晨,在上房吃了饭坐的。月娘便说:“这出月初一日,是乔亲家长姐生日,咱也还买份礼儿送了去。常言先亲后不改,莫非咱家孩儿没了,就断礼不送了?”西门庆道:“怎的不送!”于是吩咐来兴买四盒礼,又是一套妆花缎子衣服、两方销金汗巾、一盒花翠。写帖儿,叫王经送了去。这西门庆吩咐毕,就往花园藏春阁书房中坐的。只见玳安下了书回来回话,说:“钱老爹见了爹的帖子,随即写书差了一吏,同小的和黄四儿子到东昌府兵备道下与雷老爹。雷老爹旋行牌问童推官催文书,连犯人提上去从新问理。连他家儿子孙文相都开出来,只追了十两烧埋钱,问了个不应罪名,杖七十,罚赎。复又到钞关上回了钱老爹话,讨了回帖,才来了。”西门庆见玳安中用,心中大喜。拆开回帖观看,原来雷兵备回钱主事帖子都在里面。上写道:

\[
来谕悉已处分,但冯二已曾责子在先,何况与孙文相忿殴,彼此俱伤,歇后身死,又在保辜限外,问之抵命,难以平允。量追烧埋钱十两给与冯二,相应发落。谨此回覆。
\]
下书:“年侍生雷启元再拜。”

西门庆看了欢喜,因问:“黄四舅子在那里?”玳安道:“他出来都往家去了。明日同黄四来与爹磕头。黄四丈人与了小的一两银子。”西门庆吩咐置鞋脚穿,玳安磕头而出。西门庆就\textuni{22C49}在床炕上眠着了。王经在桌上小篆内炷了香,悄悄出来了。良久,忽听有人掀的帘儿响,只见李瓶儿蓦地进来,身穿糁紫衫、白绢裙,乱挽乌云,黄恹恹面容,向床前叫道:“我的哥哥,你在这里睡哩,奴来见你一面。我被那厮告了一状,把我监在狱中,血水淋漓,与秽污在一处,整受了这些时苦。昨日蒙你堂上说了人情,减我三等之罪。那厮再三不肯,发恨还要告了来拿你。我待要不来对你说,诚恐你早晚暗遭毒手。我今寻安身之处去也,你须防范他。没事少要在外吃夜酒,往那去,早早来家。千万牢记奴言,休要忘了!”说毕,二人抱头而哭。西门庆便问:“姐姐,你往那去?对我说。”李瓶儿顿脱,撒手却是南柯一梦。西门庆从睡梦中直哭醒来,看见帘影射入,正当日午,由不的心中痛切。正是:

\[
花落土埋香不见,镜空鸾影梦初醒。
\]
有诗为证:

\[
残雪初晴照纸窗,地炉灰烬冷侵床。
个中邂逅相思梦,风扑梅花斗帐香。
\]

不想早晨送了乔亲家礼,乔大户娘子使了乔通来送请帖儿,请月娘众姊妹。小厮说:“爹在书房中睡哩。”都不敢来问。月娘在后边管待乔通,潘金莲说:“拿帖儿,等我问他去。”于是蓦地推开书房门,见西门庆\textuni{22C49}着,他一屁股就坐在旁边,说:“我的儿,独自个自言自语,在这里做甚么?嗔道不见你,原来在这里好睡也!”一面说话,一面看着西门庆,因问:“你的眼怎生揉的恁红红的?”西门庆道:“想是我控着头睡来。”金莲道:“到只象哭的一般。”西门庆道:“怪奴才,我平白怎的哭?”金莲道:“只怕你一时想起甚心上人儿来是的。”西门庆道:“没的胡说,有甚心上人、心下人?”金莲道:“李瓶儿是心上的,奶子是心下的,俺们是心外的人,入不上数。”西门庆道:“怪小淫妇儿,又六说白道起来。”因问:“我和你说正经话——前日李大姐装椁,你每替他穿了甚么衣服在身底下来?”金莲道:“你问怎的?”西门庆道:“不怎的,我问声儿。”金莲道:“你问必有缘故。上面穿两套遍地金缎子衣服,底下是白绫袄、黄绸裙,贴身是紫绫小袄、白绢裙、大红小衣。”西门庆点了点头儿。金莲道:“我做兽医二十年,猜不着驴肚里病?你不想他,问他怎的?”西门庆道:“我才方梦见他来。”金莲道:“梦是心头想,喷涕鼻子痒。饶他死了,你还这等念他。象俺每都是可不着你心的人,到明日死了,苦恼也没那人想念!”西门庆向前一手搂过他脖子来,就亲个嘴,说:“怪小油嘴,你有这些贼嘴贼舌的。”金莲道:“我的儿,老娘猜不着你那黄猫黑尾的心儿!”两个又咂了一回舌头,自觉甜唾溶心,脂满香唇,身边兰麝袭人。西门庆于是淫心辄起,搂他在怀里。他便仰靠梳背,露出那话来,叫妇人品箫。妇人真个低垂粉头,吞吐裹没,往来鸣咂有声。西门庆见他头上戴金赤虎分心,香云上围着翠梅花钿儿,后鬓上珠翘错落,兴不可遏。正做到美处,忽见来安儿隔帘说:“应二爹来了。”西门庆道:“请进来。”慌的妇人没口子叫:“来安儿贼囚,且不要叫他进来,等我出去着。”来安儿道:“进来了,在小院内。”妇人道:“还不去教他躲躲儿!”那来安儿走去,说:“二爹且闪闪儿,有人在屋里。”这伯爵便走到松墙旁边,看雪培竹子。王经掀着软帘,只听裙子响,金莲一溜烟后边走了。正是:

\[
雪隐鹭鸶飞始见,柳藏鹦鹉语方知。
\]

伯爵进来,见西门庆,唱喏坐下。西门庆道:“你连日怎的不来?”伯爵道:“哥,恼的我要不的在这里。”西门庆问道:“又怎的恼?你告我说。”伯爵道:“紧自家中没钱,昨日俺房下那个,平白又桶出个孩儿来。白日里还好挝挠,半夜三更,房下又七痛八病。少不得扒起来收拾草纸被褥,叫老娘去。打紧应保又被俺家兄使了往庄子上驮草去了。百忙挝不着个人,我自家打灯笼叫了巷口邓老娘来。及至进门,养下来了。”西门庆问:“养个甚么?”伯爵道:“养了个小厮。”西门庆骂道:“傻狗才,生了儿子倒不好,如何反恼?是春花儿那奴才生的?”伯爵笑道:“是你春姨。”西门庆道:“那贼狗掇腿的奴才,谁教你要他来?叫叫老娘还抱怨!”伯爵道:“哥,你不知,冬寒时月,比不的你们有钱的人家,又有偌大前程,生个儿子锦上添花,便喜欢。俺们连自家还多着个影儿哩,要他做甚么!家中一窝子人口要吃穿,巴劫的魂也没了。应保逐日该操当他的差事去了,家兄那里是不管的。大小女便打发出去了,天理在头上,多亏了哥你。眼见的这第二个孩儿又大了,交年便是十三岁。昨日媒人来讨帖儿。我说:‘早哩,你且去着。’紧自焦的魂也没了,猛可半夜又钻出这个业障来。那黑天摸地,那里活变钱去?房下见我抱怨,没奈何,把他一根银挖儿与了老娘去了。明日洗三,嚷的人家知道了,到满月拿甚么使?到那日我也不在家,信信拖拖到那寺院里且住几日去罢。”西门庆笑道:“你去了,好了和尚来赶热被窝儿。你这狗才,到底占小便益儿。”又笑了一回,那应伯爵故意把嘴谷都着不做声。西门庆道:“我的儿,不要恼,你用多少银子,对我说,等我与你处。”伯爵道:“有甚多少?”西门庆道:“也够你搅缠是的。到其间不够了,又拿衣服当去。”伯爵道:“哥若肯下顾,二十两银子就够了,我写个符儿在此。费烦的哥多了,不好开口的,也不敢填数儿,随哥尊意便了。”西门庆也不接他文约,说:“没的扯淡,朋友家,什么符儿!”正说着,只见来安儿拿茶进来。西门庆叫小厮:“你放下盏儿,唤王经来。”不一时,王经来到。西门庆吩咐:“你往后边对你大娘说,我里间床背阁上,有前日巡按宋老爹摆酒两封银子,拿一封来。”王经应诺,不多时拿了银子来。西门庆就递与应伯爵,说:“这封五十两,你都拿了使去。原封未动,你打开看看。”伯爵道:“忒多了。”西门庆道:“多的你收着,眼下你二令爱不大了?你可也替他做些鞋脚衣裳,到满月也好看。”伯爵道:“哥说的是。”将银子拆开,都是两司各府倾就分资,三两一锭,松纹足色,满心欢喜,连忙打恭致谢,说道:“哥的盛情,谁肯!真个不收符儿?”西门庆道:“傻孩儿,谁和你一般计较?左右我是你老爷老娘家,不然你但有事就来缠我?这孩子也不是你的孩子,自是咱两个分养的。实和你说,过了满月,把春花儿那奴才叫了来,且答应我些时儿,只当利钱不算罢。”伯爵道:“你春姨这两日瘦的象你娘那样哩!”两个戏了一回,伯爵因问:“黄四丈人那事怎样了?”西门庆说:“钱龙野书到,雷兵备旋行牌提了犯人上去从新问理,把孙文相父子两个都开出来,只认了十两烧埋钱。”伯爵道:“造化他了。他就点着灯儿,那里寻这人情去!你不受他的,干不受他的。虽然你不稀罕,留送钱大人也好。别要饶了他,教他好歹摆一席大酒,里边请俺们坐一坐。你不说,等我和他说。饶了他小舅一个死罪,当别的小可事儿!”这里说话不题。

且说月娘在上房,只见孟玉楼走来,说他兄弟孟锐:“不久又起身往川广贩杂货去。今来辞辞他爹,在我屋里坐着哩。他在那里?姐姐使个小厮对他说声儿。”月娘道:“他在花园书房和应二坐着哩。又说请他爹哩,头里潘六姐到请的好!乔通送帖儿来,等着讨个话儿,到明日咱们好去不去。我便把乔通留下,打发吃茶,长等短等不见来,熬的乔通也去了。半日,只见他从前边走将来,教我问他:‘你对他说了不曾?’他没的话回,只哕了一声:‘我就忘了。’帖子还袖在袖子里。原来是恁个没尾巴行货子!不知前头干甚么营生,那半日才进来,恰好还不曾说。吃我讧了两句,往前去了。”少顷,来安进来,月娘使他请西门庆,说孟二舅来了。西门庆便起身,留伯爵:“你休去了,我就来。”走到后边,月娘先把乔家送帖来请说了。西门庆说:“那日只你一人去罢。热孝在身,莫不一家子都出来!”月娘说:“他孟二舅来辞辞你,一两日就起身往川广去。在三姐屋里坐着哩。”又问:“头里你要那封银子与谁?”西门庆道:“应二哥房里春花儿,昨晚生了个儿子,问我借几两银子使。告我说,他第二个女儿又大,愁的要不的。”月娘道:“好,好。他恁大年纪,也才见这个孩子,应二嫂不知怎的喜欢哩!到明日,咱也少不的送些粥米儿与他。”西门庆道:“这个不消说。到满月,不要饶花子,奈何他好歹发帖儿,请你们往他家走走去,就瞧瞧春花儿怎么模样。”月娘笑道:“左右和你家一般样儿,也有鼻儿也有眼儿,莫不差别些儿!”一面使来安请孟二舅来。

不一时,孟玉楼同他兄弟来拜见。叙礼已毕,西门庆陪他叙了回话,让至前边书房内与伯爵相见。吩咐小厮看菜儿,放桌儿筛酒上来,三人饮酒。西门庆教再取双钟箸:“对门请温师父陪你二舅坐。”来安不一时回说:“温师父不在,望倪师父去了。”西门庆说:“请你姐夫来坐坐。”良久,陈敬济来,与二舅见了礼,打横坐下。西门庆问:“二舅几时起身,去多少时?”孟锐道:“出月初二日准起身。定不的年岁,还到荆州买纸,川广贩香蜡,着紧一二年也不止。贩毕货就来家了。此去从河南、陕西、汉州去,回来打水路从峡江、荆州那条路来,往回七八千里地。”伯爵问:“二舅贵庚多少?”孟锐道:“在下虚度二十六岁。”伯爵道:“亏你年小小的,晓的这许多江湖道路,似俺们虚老了,只在家里坐着。”须臾添换上来,杯盘罗列,孟二舅吃至日西时分,告辞去了。

西门庆送了回来,还和伯爵吃了一回。只见买了两座库来,西门庆委付陈敬济装库。问月娘寻出李瓶儿两套锦衣,搅金银钱纸装在库内。因向伯爵说:“今日是他六七,不念经,烧座库儿。”伯爵道:“好快光阴,嫂子又早没了个半月了。”西门庆道:“这出月初五日是他断七,少不的替他念个经儿。”伯爵道:“这遭哥念佛经罢了。”西门庆道:“大房下说,他在时,因生小儿,许了些《血盆经忏》,许下家中走的两个女僧做首座,请几众尼僧,替他礼拜几卷忏儿罢了。”说毕,伯爵见天晚,说道:“我去罢。只怕你与嫂子烧纸。”又深深打恭说:“蒙哥厚情,死生难忘!”西门庆道:“难忘不难忘,我儿,你休推梦里睡哩!你众娘到满月那日,买礼都要去哩。”伯爵道:“又买礼做甚?我就头着地,好歹请众嫂子到寒家光降光降。”西门庆道:“到那日,好歹把春花儿那奴才收拾起来,牵了来我瞧瞧。”伯爵道:“你春姨他说来,有了儿子,不用着你了。”西门庆道:“不要慌,我见了那奴才和他答话。”伯爵笑的去了。

西门庆令小厮收了家伙,走到李瓶儿房里。陈敬济和玳安已把库装封停当。那日玉皇庙、永福寺、报恩寺都送疏来。西门庆看着迎春摆设羹饭完备,下出匾食来,点上香烛,使绣春请了吴月娘众人来。西门庆与李瓶儿烧了纸,抬出库去,教敬济看着,大门首焚化。正是:

\[
芳魂料不随灰死,再结来生未了缘。
\]
