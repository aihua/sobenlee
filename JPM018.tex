%# -*- coding:utf-8 -*-
%%%%%%%%%%%%%%%%%%%%%%%%%%%%%%%%%%%%%%%%%%%%%%%%%%%%%%%%%%%%%%%%%%%%%%%%%%%%%%%%%%%%%


\chapter{赂相府西门脱祸\KG 见娇娘敬济销魂}


词曰:

\[
有个人人,海棠标韵,飞燕轻盈。酒晕潮红,羞蛾一笑生春。为伊无限伤心,更说甚巫山楚云!斗帐香销,纱窗月冷,着意温存。
\]

话分两头。不说蒋竹山在李瓶儿家招赘,单表来保、来旺二人上东京打点,朝登紫陌,暮践红尘,一日到东京,进了万寿门,投旅店安歇。到次日,街前打听,只听见街谈巷议,都说兵部王尚书昨日会问明白,圣旨下来,秋后处决。止有杨提督名下亲族人等,未曾拿完,尚未定夺。来保等二人把礼物打在身边,急来到蔡府门首。旧时干事来了两遍,道路久熟,立在龙德街牌楼底下,探听府中消息。少顷,只见一个青衣人,慌慌打府中出来,往东去了。来保认得是杨提督府里亲随杨干办,待要叫住问他一声事情如何,因家主不曾分付,以此不言语,放过他去了。迟了半日,两个走到府门前,望着守门官深深唱个喏:“动问一声,太师老爷在家不在?”那守门官道:“老爷朝中议事未回。你问怎的?”来保又问道:“管家翟爷请出来,小人见见,有事禀白。”那官吏道:“管家翟叔也不在了。”来保见他不肯实说,晓得是要些东西,就袖中取出一两银子递与他。那官吏接了便问:“你要见老爷,要见学士大爷?老爷便是大管家翟谦禀,大爷的事便是小管家高安禀,各有所掌。况老爷朝中未回,止有学士大爷在家。你有甚事,我替你请出高管家来,禀见大爷也是一般。”这来保就借情道:“我是提督杨爷府中,有事禀见。”官吏听了,不敢怠慢,进入府中。良久,只见高安出来。来保慌忙施礼,递上十两银子,说道:“小人是杨爷的亲,同杨干办一路来见老爷讨信。因后边吃饭,来迟了一步,不想他先来了。所以不曾赶上。”高安接了礼物,说道:“杨干办只刚才去了,老爷还未散朝。你且待待,我引你再见见大爷罢。”一面把来保领到第二层大厅傍边,另一座仪门进去。坐北朝南三间敞厅,绿油栏杆,朱红牌额,石青镇地,金字大书天子御笔钦赐“学士琴堂”四字。

原来蔡京儿子蔡攸,也是宠臣,见为祥和殿学士兼礼部尚书、提点太乙宫使。来保在门外伺候,高安先入,说了出来,然后唤来保入见,当厅跪下。蔡攸深衣软巾,坐于堂上,问道:“你是那里来的?”来保禀道:“小人是杨爷的亲家陈洪的家人,同府中杨干办来禀见老爷讨信。不想杨干办先来见了,小人赶来后见。”因向袖中取出揭帖递上。蔡攸见上面写着“白米五百石”,叫来保近前说道:“蔡老爷亦因言官论列,连日回避。阁中之事并昨日三法司会问,都是右相李爷秉笔。杨老爷的事,昨日内里有消息出来,圣上宽恩,另有处分了。其手下用事有名人犯,待查明问罪。你还到李爷那里去说。”来保只顾磕头道:“小的不认的李爷府中,望爷怜悯,看家杨老爷分上。”蔡攸道:“你去到天汉桥边北高坡大门楼处,问声当朝右相、资政殿大学士兼礼部尚书讳邦彦的你李爷,谁是不知道!也罢,我这里还差个人同你去。”即令祗候官呈过一缄,使了图书,就差管家高安同去见李爷,如此替他说。

那高安承应下了,同来保去了府门,叫了来旺,带着礼物,转过龙德街,迳到天汉桥李邦彦门首。正值邦彦朝散才来家,穿大红绉纱袍,腰系玉带,送出一位公卿上轿而去,回到厅上,门吏禀报说:“学士蔡大爷差管家来见。”先叫高安进去说了回话,然后唤来保、来旺进见,跪在厅台下。高安就在傍边递了蔡攸封缄,并礼物揭帖,来保下边就把礼物呈上。邦彦看了说道:“你蔡大爷分上,又是你杨老爷亲,我怎么好受此礼物?况你杨爷,昨日圣心回动,已没事。但只手下之人,科道参语甚重,一定问发几个。”即令堂候官取过昨日科中送的那几个名字与他瞧。上面写着:“王黼名下书办官董升,家人王廉,班头黄玉,杨戬名下坏事书办官卢虎,干办杨盛,府掾韩宗仁、赵弘道,班头刘成,亲党陈洪、西门庆、胡四等,皆鹰犬之徒,狐假虎威之辈。乞敕下法司,将一干人犯,或投之荒裔以御魍魉,或置之典刑,以正国法。”来保见了,慌的只顾磕头,告道:“小人就是西门庆家人,望老爷开天地之心,超生性命则个!”高安又替他跪禀一次。邦彦见五百两金银,只买一个名字,如何不做分上?即令左右抬书案过来,取笔将文卷上西门庆名字改作贾廉,一面收上礼物去。邦彦打发来保等出来,就拿回帖回学士,赏了高安、来保、来旺一封五两银子。

来保路上作辞高管家,回到客店,收拾行李,还了房钱,星夜回清河县。来家见西门庆,把东京所干的事,从头说了一遍。西门庆听了,如提在冷水盆内,对月娘说:“早时使人去打点,不然怎了!”正是,这回西门庆性命有如——

\[
落日已沉西岭外,却被扶桑唤出来。
\]
于是一块石头方才落地。过了两日,门也不关了,花园照旧还盖,渐渐出来街上走动。

一日,玳安骑马打狮子街过,看见李瓶儿门首开个大生药铺,里边堆着许多生熟药材。朱红小柜,油漆牌匾,吊着幌子,甚是热闹。归来告与西门庆说——还不知招赘蒋竹山一节,只说:“二娘搭了个新伙计,开了个生药铺。”西门庆听了,半信不信。

一日,七月中旬,金风淅淅,玉露泠泠。西门庆正骑马街上走着,撞见应伯爵、谢希大。两人叫住,下马唱喏,问道:“哥,一向怎的不见?兄弟到府上几遍,见大门关着,又不敢叫,整闷了这些时。端的哥在家做甚事?嫂子娶进来不曾?也不请兄弟们吃酒。”西门庆道:“不好告诉的。因舍亲陈宅那边为些闲事,替他乱了几日。亲事另改了日期了。”伯爵道:“兄弟们不知哥吃惊。今日既撞遇哥,兄弟二人肯空放了?如今请哥同到里边吴银姐那里吃三杯,权当解闷。”不由分说,把西门庆拉进院中来。正是:

\[
高榭樽开歌妓迎,漫夸解语一含情。
纤手传杯分竹叶,一帘秋水浸桃笙。
\]

当日西门庆被二人拉到吴银儿家,吃了一日酒。到日暮时分,已带半酣,才放出来。打马正走到东街口上,撞见冯妈妈从南来,走得甚慌。西门庆勒住马,问道:“你那里去?”冯妈妈道:“二娘使我往门外寺里鱼篮会,替过世二爷烧箱库去来。”西门庆醉中道:“你二娘在家好么?我明日和他说话去。”冯妈妈道:“还问甚么好?把个见见成成做熟了饭的亲事,吃人掇了锅儿去了。”西门庆听了失声惊问道:“莫不他嫁人去了?”冯妈妈道:“二娘那等使老身送过头面,往你家去了几遍不见你,大门关着。对大官儿说进去,教你早动身,你不理。今教别人成了,你还说甚的?”西门庆问:“是谁?”冯妈妈悉把半夜三更妇人被狐狸缠着,染病看看至死,怎的请了蒋竹山来看,吃了他的药怎的好了,某日怎的倒踏门招进来,成其夫妇,见今二娘拿出三百两银子与他开了生药铺,从头至尾说了一遍。这西门庆不听便罢,听了气的在马上只是跌脚,叫道:“苦哉!你嫁别人,我也不恼,如何嫁那矮王八!他有甚么起解?”于是一直打马来家。

刚下马进仪门,只见吴月娘、孟玉楼、潘金莲并西门大姐四个,在前厅天井内月下跳马索儿耍子。见西门庆来家,月娘、玉楼、大姐三个都往后走了。只有金莲不去,且扶着庭柱兜鞋,被西门庆带酒骂道:“淫妇们闲的声唤,平白跳甚么百索儿?”赶上金莲踢了两脚。走到后边,也不往月娘房中去脱衣裳,走在西厢一间书房内,要了铺盖,那里宿歇。打丫头,骂小厮,只是没好气。众妇人同站在一处,都甚是着恐,不知是那缘故。吴月娘埋怨金莲:“你见他进门有酒了,两三步叉开一边便了。还只顾在跟前笑成一块,且提鞋儿,却教他蝗虫蚂蚱一例都骂着。”玉楼道:“骂我们也罢,如何连大姐姐也骂起淫妇来了?没槽道的行货子!”金莲接过来道:“这一家子只是我好欺负的!一般三个人在这里,只踢我一个儿。那个偏受用着甚么也怎的?”月娘就恼了,说道:“你头里何不叫他连我踢不是?你没偏受用,谁偏受用?恁的贼不识高低货!我到不言语,你只顾嘴头子哗哩薄喇的!”金莲见月娘恼了,便把话儿来摭,说道:“姐姐,不是这等说。他不知那里因着甚么头由儿,只拿我煞气。要便睁着眼望着俺叫,千也要打个臭死,万也要打个臭死!”月娘道:“谁教你只要嘲他来?他不打你,却打狗不成!”玉楼道:“大姐姐,且叫小厮来问他声,今日在谁家吃酒来?早晨好好出去,如何来家恁个腔儿!”不一时,把玳安叫到跟前,月娘骂道:“贼囚根子!你不实说,教大小厮来拷打你和平安儿,每人都是十板。”玳安道:“娘休打,待小的实说了罢。爹今日和应二叔们都在院里吴家吃酒,散了来在东街口上,撞遇冯妈妈,说花二娘等爹不去,嫁了大街住的蒋太医了。爹一路上恼的要不的。”月娘道:“信那没廉耻的歪淫妇,浪着嫁了汉子,来家拿人煞气。”玳安道:“二娘没嫁蒋太医,把他倒踏门招进去了。如今二娘与他本钱,开了好不兴的生药铺。我来家告爹说,爹还不信。”孟玉楼道:“论起来,男子汉死了多少时儿?服也还未满,就嫁人,使不得的!”月娘道:“如今年程,论的甚么使的使不的。汉子孝服未满,浪着嫁人的,才一个儿?淫妇成日和汉子酒里眠酒里卧的人,他原守的甚么贞节!”看官听说:月娘这一句话,一棒打着两个人——孟玉楼与潘金莲都是孝服不曾满再醮人的,听了此言,未免各人怀着惭愧归房,不在话下。正是:

\[
不如意事常八九,可与人言无二三。
\]

却说西门庆当晚在前边厢房睡了一夜。到次日早,把女婿陈敬济安在他花园中,同贲四管工记帐,换下来招教他看守大门。西门大姐白日里便在后边和月娘众人一处吃酒,晚夕归到前边厢房中歇。陈敬济每日只在花园中管工,非呼唤不敢进入中堂,饮食都是内里小厮拿出来吃。所以西门庆手下这几房妇人都不曾见面。一日,西门庆不在家,与提刑所贺千户送行去了。月娘因陈敬济一向管工辛苦,不曾安排一顿饭儿酬劳他,向孟玉楼、李娇儿说:“待要管,又说我多揽事;我待欲不管,又看不上。人家的孩儿在你家,每日早起睡晚,辛辛苦苦,替你家打勤劳儿,那个与心知慰他一知慰儿也怎的?”玉楼道:“姐姐,你是个当家的人,你不上心谁上心!”月娘于是分付厨下,安排了一桌酒肴点心,午间请陈敬济进来吃一顿饭。这陈敬济撇了工程教贲四看管,迳到后边参见月娘,作揖毕,旁边坐下。小玉拿茶来吃了,安放桌儿,拿蔬菜按酒上来。月娘道:“姐夫每日管工辛苦,要请姐夫进来坐坐,白不得个闲。今日你爹不在家,无事,治了一杯水酒,权与姐夫酬劳。”敬济道:“儿子蒙爹娘抬举,有甚劳苦,这等费心!”月娘陪着他吃了一回酒。月娘使小玉:“请大姑娘来这里坐。”小玉道:“大姑娘使着手,就来。”少顷,只听房中抹得牌响。敬济便问:“谁人抹牌?”月娘道:“是大姐与玉箫丫头弄牌。”敬济道:“你看没分晓,娘这里呼唤不来,且在房中抹牌。”一不时,大姐掀帘子出来,与他女婿对面坐下,一周饮酒。月娘便问大姐:“陈姐夫也会看牌不会?”大姐道:“他也知道些香臭儿。”月娘只知敬济是志诚的女婿,却不道这小伙子儿诗词歌赋,双陆象棋,拆牌道字,无所不通,无所不晓。正是:

\[
自幼乖滑伶俐,风流博浪牢成。爱穿鸭绿出炉银,双陆象棋帮衬。琵琶笙筝箫管,弹丸走马员情。只有一件不堪闻:见了佳人是命。
\]

月娘便道:“既是姐夫会看牌,何不进去咱同看一看?”敬济道:“娘和大姐看罢,儿子却不当。”月娘道:“姐夫至亲间,怕怎的?”一面进入房中,只见孟玉楼正在床上铺茜红毡看牌,见敬济进来,抽身就要走。月娘道:“姐夫又不是别人,见个礼儿罢。”向敬济道:“这是你三娘哩。”那敬济慌忙躬身作揖,玉楼还了万福。当下玉楼、大姐三人同抹,敬济在傍边观看。抹了一回,大姐输了下来,敬济上来又抹。玉楼出了个天地分;敬济出了个恨点不到;吴月娘出了个四红沉八不就,双三不搭两么儿,和儿不出,左来右去配不着色头。只见潘金莲掀帘子进来,银丝鬒髻上戴着一头鲜花儿,笑嘻嘻道:“我说是谁,原来是陈姐夫在这里。”慌的陈敬济扭颈回头,猛然一见,不觉心荡目摇,精魂已失。正是:五百年冤家相遇,三十年恩爱一旦遭逢。月娘道:“此是五娘,姐夫也只见个长礼儿罢。”敬济忙向前深深作揖,金莲一面还了万福。月娘便道:“五姐你来看,小雏儿倒把老鸦子来赢了。”这金莲近前一手扶着床护炕儿,一只手拈着白纱团扇儿,在傍替月娘指点道:“大姐姐,这牌不是这等出了,把双三搭过来,却不是天不同和牌?还赢了陈姐夫和三姐姐。”众人正抹牌在热闹处,只见玳安抱进毡包来,说:“爹来家了。”月娘连忙撺掇小玉送姐夫打角门出去了。

西门庆下马进门,先到前边工上观看了一遍,然后踅到潘金莲房中来。金莲慌忙接着,与他脱了衣裳,说道:“你今日送行去来的早。”西门庆道:“提刑所贺千户新升新平寨知寨,合卫所相知都郊外送他来,拿帖儿知会我,不好不去的。”金莲道:“你没酒,教丫鬟看酒来你吃。”不一时,放了桌儿饮酒,菜蔬都摆在面前。饮酒中间,因说起后日花园卷棚上梁,约有许多亲朋都要来递果盒酒挂红,少不得叫厨子置酒管待。说了一回,天色已晚。春梅掌灯归房,二人上床宿歇。西门庆因起早送行,着了辛苦,吃了几杯酒就醉了。倒下头鼾睡如雷,齁齁不醒。那时正值七月二十头天气,夜间有些余热,这潘金莲怎生睡得着?忽听碧纱帐内一派蚊雷,不免赤着身子起来,执烛满帐照蚊。照一个,烧一个。回首见西门庆仰卧枕上,睡得正浓,摇之不醒。其腰间那话,带着托子,累垂伟长,不觉淫心辄起,放下烛台,用纤手扪弄。弄了一回,蹲下身去,用口吮之。吮来吮去,西门庆醒了,骂道:“怪小淫妇儿,你达达睡睡,就掴昆死了。”一面起来,坐在枕上,亦发叫他在下尽着吮咂;又垂首玩之,以畅其美。正是:

\[
怪底佳人风性重,夜深偷弄紫箫吹。
\]
又有蚊子双关《踏莎行》词为证:

\[
我爱他身体轻盈,楚腰腻细。行行一派笙歌沸。黄昏人未掩朱扉,潜身撞入纱厨内。款傍香肌,轻怜玉体。嘴到处,胭脂记。耳边厢造就百般声,夜深不肯教人睡。
\]

妇人顽了有一顿饭时,西门庆忽然想起一件事来,叫春梅筛酒过来,在床前执壶而立。将烛移在床背板上,教妇人马爬在他面前,那话隔山取火,托入牡中,令其自动,在上饮酒取乐。妇人骂道:“好个刁钻的强盗!从几时新兴出来的例儿,怪剌剌教丫头看答着,甚么张致!”西门庆道:“我对你说了罢,当初你瓶姨和我常如此干,叫他家迎春在傍执壶斟酒,到好耍子。”妇人道:“我不好骂出来的,甚么瓶姨鸟姨,题那淫妇做甚,奴好心不得好报。那淫妇等不的,浪着嫁汉子去了。你前日吃了酒来家,一般的三个人在院子里跳百索儿,只拿我煞气,只踢我一个儿,倒惹的人和我辨了回子嘴。想起来,奴是好欺负的!”西门庆问道:“你与谁辨嘴来?”妇人道:“那日你便进来了,上房的好不和我合气,说我在他跟前顶嘴来,骂我不识高低的货。我想起来为甚么?养虾蟆得水虫儿病,如今倒教人恼我!”西门庆道:“不是我也不恼,那日应二哥他们拉我到吴银儿家,吃了酒出来,路上撞见冯妈妈子,这般告诉我,把我气了个立睁。若嫁了别人,我到罢了。那蒋太医贼矮忘八,那花大怎不咬下他下截来?他有甚么起解?招他进去,与他本钱,教他在我眼面前开铺子,大剌剌的做买卖!”妇人道:“亏你脸嘴还说哩!奴当初怎么说来?先下米儿先吃饭。你不听,只顾来问大姐姐。常言:信人调,丢了瓢。你做差了,你埋怨那个?”西门庆被妇人几句话,冲得心头一点火起,云山半壁通红,便道:“你由他,教那不贤良的淫妇说去。到明日休想我理他!”看官听说:自古谗言罔行,君臣、父子、夫妇、昆弟之间,皆不能免。饶吴月娘恁般贤淑,西门庆听金莲衽席睥睨之间言,卒致于反目,其他可不慎哉!自是以后,西门庆与月娘尚气,彼此觌面,都不说话。月娘随他往那房里去,也不管他;来迟去早,也不问他;或是他进房中取东取西,只教丫头上前答应,也不理他。两个都把心冷淡了。正是:

\[
前车倒了千千辆,后车到了亦如然。
分明指与平川路,却把忠言当恶言。
\]

且说潘金莲自西门庆与月娘尚气之后,见汉子偏听,以为得志。每日抖擞着精神,妆饰打扮,希宠市爱。因为那日后边会着陈敬济一遍,见小伙儿生的乖猾伶俐,有心也要勾搭他。但只畏惧西门庆,不敢下手。只等西门庆往那里去,便使了丫鬟叫进房中,与他茶水吃,常时两个下棋做一处。一日西门庆新盖卷棚上梁,亲友挂红庆贺,递果盒。许多匠作,都有犒劳赏赐。大厅上管待客官,吃到午晌,人才散了。西门庆因起得早,就归后边睡去了。陈敬济走来金莲房中讨茶吃。金莲正在床上弹弄琵琶,道:“前边上梁,吃了这半日酒,你就不曾吃些甚么,还来我屋里要茶吃?”敬济道:“儿子不瞒你老人家说,从半夜起来,乱了这一五更,谁吃甚么来!”妇人问道:“你爹在那里?”敬济道:“爹后边睡去了。”妇人道:“你既没吃甚么,”叫春梅:“拣籹里拿我吃的那蒸酥果馅饼儿来,与你姐夫吃。”这小伙儿就在他炕桌儿上摆着四碟小菜,吃着点心。因见妇人弹琵琶,戏问道:“五娘,你弹的甚曲儿?怎不唱个儿我听。”妇人笑道:“好陈姐夫,奴又不是你影射的,如何唱曲儿你听?我等你爹起来,看我对你爹说不说!”那敬济笑嘻嘻,慌忙跪着央及道:“望乞五娘可怜见,儿子再不敢了!”那妇人笑起来了。自此这小伙儿和这妇人日近日亲,或吃茶吃饭,穿房入屋,打牙犯嘴,挨肩擦背,通不忌惮。月娘托以儿辈,放这样不老实的女婿在家,自家的事却看不见。正是:

\[
只晓采花成酿蜜,不知辛苦为谁甜。
\]
