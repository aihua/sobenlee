%# -*- coding:utf-8 -*-
%%%%%%%%%%%%%%%%%%%%%%%%%%%%%%%%%%%%%%%%%%%%%%%%%%%%%%%%%%%%%%%%%%%%%%%%%%%%%%%%%%%%%

\chapter{毛泽东与原中共中央历史上分歧的由来}

\section{“农民党”、“军党”和毛泽东的“书记独裁”问题}

发端于~1942~年春的延安整风运动,在一定意义上是毛泽东长久以来对原中共中央不满的一个总爆发。整风运动的起步有一个漫长的历史演变过程,它的近期的准备和酝酿,虽然可从~1935~年遵义会议后毛泽东出台的一系列措施和部署中寻找到踪迹,但其根源则可追溯到苏维埃运动时期。在长达七、八年的时间里,毛泽东与中共中央最高层之间积累了大量的矛盾和恩怨,双方既有过合作,但更多的却是互相猜疑和防范。在毛与原中共中央的矛盾中,政见分歧占突出地位,但其它因素——由毛的个性和工作作风而引起的对毛的反感和排斥也占据一定的比重。正是基于这些原因,中共中央对于毛泽东,一直是欲用不能、欲弃不舍。

毛泽东之在中国成为远近闻名的人物,始自于~1927~年秋率众上了井冈山,最先走上武装反抗国民党的道路,从此成为中共武装革命的著名领导人。在国民党方面,毛固然为其心腹大患的“朱毛赤匪”的首领;而在中共及莫斯科方面,毛则是创建了可使中共赖以生存的红色苏区的头等功臣。但在一个相当长的时期中,毛却因其思想、行为中的“异端”色彩不大见容于莫斯科及中共中央。毛泽东的“异端”在不同时期有不同的内容与表现。1927~—~1930~年,是毛“异端”萌生的阶段,在这个阶段,莫斯科和中共中央密切注视着毛泽东在江西的活动,尽管对毛的部分观点存在疑虑,对江西共产党区域的政策也有一些不同的看法,但总的说来,对毛泽东的意见和毛在红军、根据地内的领导地位是承认和尊重的。

共产国际和中共中央看到了毛泽东对中共所作出的重大贡献,这就是在~1927~年国共分裂后最严重的形势下,毛以极大的勇气和智能开辟了一块中共领导的根据地,发展了一支由共产党领导的红色军队,使中共在国民党统治的汪洋大海中有了可以立足、并实现其政纲的地盘。1928~年~6~月,在毛泽东未出席的情况下,在莫斯科召开的中共六大仍选毛为中央委员。在处理毛泽东与其他重要军政领导人的关系问题上,中共中央也极为谨慎,一般都以维护毛泽东的威信为重。1929~年~9~月,周恩来指导起草的著名的“九月来信”,在毛泽东与朱德间就红四军中前委与军委的权限关系而发生的争论中,支持毛泽东的意见,帮助恢复了毛对红四军的领导\footnote{中共中央文献研究室编:《周恩来年谱(1898~—~1949)》(北京:中央文献出版社、人民出版社,1990~年),页~169。以下称《周恩来年谱》。}。

1927~—~1930~年毛泽东主要以军事领导人闻名于中共,其活动基本上也是围绕军事武装问题而展开,理论活动只是其军事活动的一个侧面。中共中央认为毛的理论观点仍在共产国际和中共中央路线的框架之内,毛并没有要求中共中央修正以城市为中心的总路线。

这一时期,毛泽东在江西苏维埃区域和红军中享有实际的最高权威,中共中央对江西根据地的指示基本是通过毛泽东来贯彻和实行的。对于共产国际和中共中央的指示,毛根据现实和自己的需要加以灵活的解释,因而远在莫斯科的斯大林和上海的中共中央对毛并不构成直接和具体的约束。毛所领导的红四军是维系根据地军队、党、苏维埃政权生存的唯一重要的柱石,而毛所担任的前委书记一职是红四军的最高职务。江西苏维埃区域各级党、政机构的多数负责人,和主力红军的各级领导人中的大部分,都是跟随毛上井冈山,或较早参加井冈山和赣南、闽西斗争的老同志。这些人虽然对共产国际和中共中央充满尊敬,但其中的大部分人,在情感和知识背景等方面却和共产国际及在上海的中共中央存在着隔膜,他们对共产国际的尊崇和服从是以尊崇、服从毛泽东来实现的,因此,在这一时期,共产国际和中共中央只有通过毛泽东才能具体影响江西苏维埃区域,而这种影响大体也处在毛的控制之下。

但是随着~1930~年后中共中央对江西苏维埃区域的日益重视,中共中央的工作重心逐渐向江西倾斜,从莫斯科学习归来的干部陆续被派往江西以加强根据地的各项工作,毛泽东与中共中央的关系渐渐微妙起来。

中共中央原先为了中共的发展和红军力量的加强,一度隐忍了对毛泽东某些“异端”观点和行为的不满,现在,从江西苏区不断传来一些令人不安的消息,中共中央对毛逐渐形成了某些消极看法。

一、“农民党”的问题

“农民党”的问题是由中共湖南省委派往湘赣边界巡视的杨克敏于~1929~年~2~月~25~日在给湖南省委的报告中提出来的。杨克敏就中共在边界地区的组织状况写道:“因为根本是个农民区域,所以农民党的色彩很浓厚”\footnote{《杨克敏关于湘赣边苏区情况的综合报告》(1929~年~2~月~25~日),载江西省档案馆编:《井冈山革命根据地史料选编》(南昌:江西人民出版社,1986~年),页~136。}。毛泽东在给中共中央的报告中也谈到类似的情况:“边界各县的党,几乎完全是农民成份的党”\footnote{毛泽东:《井冈山的斗争》,载《毛泽东选集》,第~1~卷(北京:人民出版社,1951~年),页~82。}。“农民党”问题的严重性表现为:

首先,江西共产党区域的各级基层组织的成员绝大部分都是农民。

其次,参加中共党组织的农民中还包含许多“帮会”分子。据杨克敏的报告,酃县中共党员三、四百人,“且多洪会中人”\footnote{《杨克敏关于湘赣边苏区情况的综合报告》(1929~年~2~月~25~日),载江西省档案馆编:《井冈山革命根据地史料选编》(南昌:江西人民出版社,1986~年),页~134。}。

第三,地方党组织的家族化。由于根据地只能存在于偏僻的乡村,而湘赣边界的山地又基本上处于封闭的自给自足的自然经济环境下,家族——宗族组织就成为维系当地百姓社会生活的唯一重要纽带,中共在乡村的组织不可避免与这种家族——宗族结合在一起。一姓一族的成员组成党支部成为一个普遍的现象,“支部会议简直同时就是家族会议”。

第四,由于党组织成员几乎全为农民,文化程度很低,许多人甚至是文盲,“斗争的布尔什维克党的建设,真是难得很”。党的思想训练在实施中遇到极大的困难,许多党员和基层党组织表现出严重的“地方观念、保守思想、自由散漫的劣根性”。一些最基本的党的知识也难以接受,对此毛泽东极为感慨:“说共产党不分国界、省界的话,他们不大懂,不分县界、区界、乡界的话,他们也是不大懂得的”\footnote{毛泽东:《井冈山的斗争》,载《毛泽东选集》,第~1~卷(北京:人民出版社,1951~年),页~79。}。

杨克敏的观察和毛的看法基本一致:“农民做梦也想不到机器工业是一个什么样儿,是一回什么事,帝国主义到底是一回什么事。”\footnote{《杨克敏关于湘赣边苏区情况的综合报告》(1929~年~2~月~25~日),载江西省档案馆编:《井冈山革命根据地史料选编》(南昌:江西人民出版社,1986~年),页~136。}

尽管毛泽东与杨克敏在对党的“农民化”问题上的分析大致相同,但是两人在对问题的性质及处理方法的认识上却存在明显的差别:

毛泽东只是提出党的农民成份居多的事实,而杨克敏则认为边界的党组织是“农民党”。毛认为,可以通过给农民灌输通俗的革命知识将农民改造成布尔什维克;杨克敏则认为,由于农民文化如此低下,政治知识极度缺乏,“实在很难使农民有进步的思想发生”。杨克敏的看法实际上反映的是共产国际和中共中央的正统观点,即认为,只有通过党的工人阶级化才能克服“农民化”对党的危害。

毛泽东虽然在~1926~年~9~月就曾表述过“农民最具革命性”的思想,但在~1927~—~1928~年,共产国际和中共中央的概念对毛仍具有相当影响。由此出发,毛一度对党的“农民化”表现出某种忧虑,但是经过在湘赣边界一年多的游击战争和对农村经济社会状况了解的加深,毛逐渐消除了他对党的“农民化”趋向的担心。毛认为尽管农民知识低下,但政治教育可以发挥作用;至于农民文化知识少,正可避免第二国际错误思想的影响。而更现实的问题是,在江西根据地几乎不存在工人阶级,即使勉强地把所有小作坊的工匠和店员列为工人,和农民相比,在人数上也只占极小的比例。因此,毛很快就改变了对边界党组织“农民化”的批评,转而致力于对农民党员的思想训练。

对于毛泽东的这种通过政治训练改造农民党员的观点,中共中央很难提出任何正式批评,一则因为毛泽东并未否认工人阶级的革命领导作用;二则如果按严格的苏共标准衡量,苏区的中共党组织将不成其为共产党;面对现实的唯一方法只能是依靠思想灌输使农民党员布尔什维克化。但是,中共中央对毛泽东在马克思主义解释方面的灵活性却有所不安,毛在一般肯定工人阶级革命作用的同时,愈加突出强调农民对中国革命的意义,在中共中央看来,毛已开始表现出“离经叛道”的趋向。

二、“军党”的问题

1927~年~10~月,毛泽东率秋收暴动队伍上井冈山以后,军队就成了维持苏区存在的唯一重要的力量,在紧张的战争环境下,党与红军已融为一体,军队实质上已成为中共党组织的化身。

在红军中建立党组织是毛泽东吸取中共在大革命失败的教训、为加强党对红军的领导而采取的一个重大措施。毛认为中共在~1927~年的失败原因之一即是“我们在国民党军中的组织,完全没有抓住士兵,即在叶挺部也还是每团只有一个支部,故经不起严重的考验”。1927~年~10~月,毛在永新县三湾确立了“支部建在连上”的制度,10~月中旬又在酃县亲自主持了六个士兵党员的人党宣誓仪式。从此,在红军中大力发展党员,成为中共一项最基本的制度。

与“支部建在连上”相配套的是继续采用仿效苏联红军模式的北伐时代的党代表制度。自~1929~年起,红军中的党代表改称政治委员。连的政治委员从~1931~年起改称政治指导员,此称谓一直沿用至今。军队的重要作用尤其体现在它实际上是地方党的保姆和守护者。1928~年~4~月之前,中共湘赣边界的地方党组织基本处在分散和工作全面停顿的状态,5~月中旬,毛泽东在宁岗茅坪主持召开了中共湘赣边界第一次代表大会,此次会议正式确定“军队帮助地方党发展”的方针,选举了以毛为书记的边界第一届特委会,毛泽东从此一身兼军队和地方的最高领导。然而军队主力一旦转移,地方党的生存马上就发生危机。1928~年~8~月,边界各县党遵照湖南省委指示,朱德率红军主力进攻湖南,导致“八月失败”,组织和政权大部分解体。而一旦主力红军于~9~月重新占领该地区,所有的中共组织和政权即迅速重建起来。军队的作用如此显著,地方党组织隶属军队系统的领导也就逐渐被认为是顺理成章了。

对于中共军队在根据地对党组织所起的支配性作用,中共中央的态度是矛盾和复杂的。一方面,党的领导人完全支持在军队中建立党组织,也深知军队领导地方党是艰苦恶劣环境下的必然产物;另一方面,又对中共军队的农民化、军队对地方党组织支配性的关系深感忧虑。

1928~年~6~月~4~日,中共中央在《致朱德、毛泽东并前委信》中虽然承认了毛泽东统领湘赣边界红军与地方党的最高权威,但对湘赣边界党和军队的工作也提出了一系列批评。中共中央对红军中农民成份的急剧增长表示严重的忧虑,认为“无论在政权机关或党的指导机关中都有很深厚的小资产阶级的农民意识的影响”,要求毛泽东注意在红军“成份上尽可能增加工农和贫农的成份,减少流氓的成份”,指示毛泽东必须“彻底地改造各级党部及指导机关,多提拔积极的工农分子特别是工人分子参加各级党部的指导机关”。中共中央还批评湘赣边界的“苏维埃政权,多是上层的委派的而无下层选举的基础”,责令毛泽东改变方式,“禁止党部和军队委派苏维埃”,“绝对防止党命令苏维埃的毛病”\footnote{《中央致朱德、毛泽东并前委信》(1928~年~6~月~4~日),载中央档案馆编:《中共中央文件选集》(1928),第~4~册(北京:中共中央党校出版社,1989~年),页~248、253、256、250、252。}。

中共中央关于改变红军成份,调整党、军队与地方苏维埃关系的训令,实际上是一种仿效苏共经验的一厢情愿的空想。1928~年湘赣边界共产党的状况与~1917~—~1918~年的俄共与苏俄红军的情形不啻相距万里。在湘赣边界形成的由农民组成的红军,和在红军指导下建立的以农民为主体的共产党组织以及苏维埃政权的三位一体,是客观历史环境的产物。这个以军队为核心的三位一体是作为苏共模式一个分支的中国共产革命的一个基本形态,只是当时,它正处在刚刚成长的萌芽状态,而不被正统的中共中央所理解。

三、毛泽东的“书记独裁”的问题

毛泽东既是湘赣红军的创始人,也是湘赣边界党组织的领导者,由于军队对边界共产党的存在有着举足轻重的影响,毛兼军队与地方党负责人于一身也就十分自然了。但是随着毛身兼二职,全部权力逐渐集中到毛泽东手中,逐渐出现了对毛大权独揽的议论。

曾经担任中共湖南省委巡视员的杜修经在给上级的报告中指出:

\begin{quoting}
现在边界特委工作日益扩大,实际上一切工作与指导,都集中在泽东同志身上,而泽东同志又负军党代表责,个人精力有限,怎理得这多\footnote{杜修经:《向中共湖南省委的报告》(1928~年),载江西省档案馆编:《井冈山革命根据地史料选编》,页~20。}?
\end{quoting}

一度担任边界特委书记的杨开明亦有同感:

\begin{quoting}
特委的事总是书记一个人处理,个人专政,书记独裁,成为边界的通弊。首先泽东为特委书记时,特委就在泽东一个人荷包里,后来(杨)开明代理书记,特委又是开明一个人的独脚戏。……党员崇拜领袖,信仰英雄,而不大认得党的组织\footnote{《杨克敏关于湘赣边苏区情况的综合报告》(1929~年~2~月~25~日),载江西省档案馆编:《井冈山革命根据地史料选编》(南昌:江西人民出版社,1986~年),页~132、136。}。
\end{quoting}

和杜修经、杨开明议论毛泽东“书记独裁”相联系,在中共中央~1928~年~6~月~4~日《致朱德、毛泽东并前委信》中,要求朱、毛取消红四军的党代表制,建立政治部体制,也包含了分散作为红四军总党代表毛泽东权力的意思。1929~年红四军内部围绕朱德与毛泽东的权限范围的问题终于爆发了一场严重争论,虽然毛泽东的一系列重大战略和战术方针比较接近实际,但是由于他的专断作风也十分明显,红四军多数干部对毛都存有不同程度的意见\footnote{参见《龚楚将军回忆录》(香港:明报月刊社,1978~年),页~171、205~—~207、348、357。},毛泽东曾一度被迫离开红四军,前往地方工作。最后,中共中央出于红四军统一和发展江西根据地的战略考虑,决定在朱、毛之间支持毛作为红四军最高领导,才解决了这场领导机关内部的危机,但是对毛泽东“书记独裁”的不安并没有真正消除。

杜修经、杨开明对毛泽东“书记独裁”的议论不是偶发的,它来源于中共中央,是处于转折年代中共路线、方针和工作方式急剧变化的产物。这个时期,中共中央在理念和党的作风上还受到俄国十月革命初期党内民主化思想及其实践的影响,因此对江西根据地的“书记独裁”现象颇为不满,所谓“群众只认识个人不认识党”,显然是指只认识毛泽东而不认识党。对此,中共中央别无他法,只能再次重申“增加工人领导力量,严格防止农民党的倾向”,“反对个人领导,一切权力集中于前委”一类的意见。

对于有关“书记独裁”一类的议论和指责,毛泽东很不以为然。1927~年“八七会议”后,中共中央领导机关大改组,瞿秋白曾建议毛去上海党中央工作,毛泽东表示“不愿跟你们去住高楼大厦”,主动前往艰苦的农村,为共产党开辟一条新路。毛在湘赣边界也经常向中央汇报工作,反观上海中央领导人,论资历不及昔日的陈独秀,论工作成绩也乏善可陈,却在上海的“洋房”里指手划脚,只能徒增毛泽东对中共中央的反感。

综上所述,1927~—~1930~年,毛泽东与共产国际、中共中央在重大路线方面并无大的矛盾,但已隐藏着若干不协调的因素:“农民党”的问题,“军党”的问题,以及毛泽东“书记独裁”的问题,上述三个方面的问题。在日后又发展为其它一系列新问题,导致中共中央与毛泽东的关系进一步复杂化。

\section{毛泽东在“肃~AB~团”问题上的极端行为与中共中央的反应}

在中共中央与毛泽东的诸多矛盾中,有关肃反问题引起的对毛的反感和猜疑占有突出的地位,但是这个问题又十分敏感,无论是毛,抑或是中共中央,都未将各自在肃反问题上的责任言深说透。相反,毛和中共中央出自不同的原因,在这个问题上还说了许多言不及义的话,造成了大量假话的流行。事实上,毛泽东是中共历史上厉行肃反的始作俑者,毛出于极左的肃反观和复杂的个人目的,直接参与领导了~1930~—~1931~年镇压“AB~团”的行动。在极困难的形势下,苏区中央局书记项英作了许多努力,试图纠正毛的错误,但是中共中央却从左的理念出发,否定了项英的意见,全力支持毛泽东,从而形成了毛与六届四中全会后中共中央的合作。以后随着肃反事态的进一步恶化,中共中央才着手调整政策,而与毛的合作也走到了尾声。

江西苏区的“肃反”运动渊源久远,它最早萌发于~1928~年~9~月毛泽东领导的湘赣边界的“洗党”。以清除“投机分子”为目标的“洗党”是中国共产党历史上最早的一次整党运动,它创造了将整党与肃反相结合、以党内出身地富家庭的知识分子为清洗对象的整党肃反模式的最初形态。

“洗党”将打击矛头集中在党内出身地富家庭的知识分子身上,根据若于资料记载,清洗对象除了叛变、投降国民党者外,主要为知识分子党员:“凡是农民党员都发了党员证,知识分子入党不发(须上级批准)”\footnote{刘克犹:《回忆宁冈县的党组织》,载余怕流、夏道汉编:《井冈山革命根据地研究》南昌:江西人民出版社,1978~年),页~308。}。“凡是有亲戚在国民党反动派办事的、当兵的、不服从指挥的、不愿干革命的、社会关系不好的,就尽量洗刷。洗刷的党员不宣布也不通知,开会不叫他参加,重新立过党员花名册。对犯错误的党员有几种处分:警告、留党察看、开除党籍。”\footnote{朱开卷:《宁冈区乡政权和党的建设情况》,载《井冈山革命根据地研究》,页~307。}

如果说~1928~年~9~月在井冈山地区开展的“洗党”规模较小,为时较短,那么~1930~年~2~月以后席卷赣西南的“肃~AB~团”则是一场大规模的残酷的党内清洗运动,它直接导致了~1930~年~12~月震惊苏区的富田事变的爆发。

一、毛泽东的“肃~AB~团”与富田事变

1929~年古田会议后,毛泽东在江西苏区的权威已经初步形成。促成毛领导权威形成的两个最重要条件都已具备:一、中共中央对毛的明确支持为毛的权威提供了法理基础;二、毛的事功突出,在他的领导下,根据地地盘扩大,人口增加,一度与毛意见相左的朱德,因军事失利、威望有所损失,毛的军事成功为毛的权威提供了事实基础。作为毛领导权威的具体体现,1930~年,毛担任了红一方面军总政委和总前委书记,在统一的苏区党领导机构尚未建立的形势下,毛所领导的红一方面军总前委成为江西苏区最高领导机构。然而毛毕竟不是中共中央,在苏区内部仍有部分红军和党组织援引中共中央来消极对抗毛的新权威。

二十年代末至三十年代初是中共武装革命的草创年代,一时豪雄四起,在反抗国民党的大目标下,革命阵营内部的矛盾被暂时掩盖。但根据地内,外来干部与本地干部的矛盾、留苏干部与国内干部的矛盾、知识分子干部与农民出身的干部的矛盾仍然存在,使之可以凝聚的唯一力量来自于中共中央的权威,包括中央提供的意识形态义理系统的理论权威。只是此时的中共中央远离乡村,城市中央对根据地的领导必须通过毛泽东来体现,因此,毛个人的识见、智能、人格因素和作风态度就显得特别重要了。

毛泽东在江西苏区所有武装同志中最具政治眼光,意志最为刚强,且最善用兵打仗,但其人作风专断。1929~年~7~月陈毅赴上海汇报请示中央对朱、毛纷争的意见,中共中央明确表态支持毛,陈毅返赣后,亲自请毛出山,朱德、陈毅为忠诚的共产党员,一切服从中央,重新理顺了与毛的关系,使红四军内部的分歧和矛盾得以化解。但是,毛与赣西南地方红军和党组织的分歧却因各种原因而尖锐化了。

正是在这种情况下才发生了毛泽东“打~AB~团”的大清洗。这场事变的直接原因是,毛在江西苏区的权威刚刚建立,却遭到以李文林为首的赣西南地方红军和党组织的挑战,毛不能容忍在他鼻子底下有任何违抗自己权威与意志的有组织的反对力量,而不管这种反对力量来自红军内部或是地方党组织。为了维护自己在根据地的权威,毛一举挣脱党道德和党伦理的约束,不惜采用极端手段镇压被他怀疑为异己力量的党内同志。

毛泽东用流血的超常规手段解决党内纷争,究竟要达到什么目标?一言以蔽之,毛要做江西苏区的列宁,由于此时毛尚未成为中国党的列宁,不具号令全党的法理权威,他才不惜采用极端手段镇压党内的反对派。

将大规模的恐怖施之于党内和军内,与党道德和党伦理存在巨大冲突,如何解决这个矛盾?毛泽东自有办法。他声称:以李文林为首的赣西南党和红军已被机会主义和富农路线所控制,为了挽救革命,必须对之进行彻底改造,这样,毛的镇压就有了思想的旗帜。

李文林是知识分子出身的赣西南党与红军的创始人之一,1929~年~2~月,毛、朱根据湘赣国民党军队正着手对井冈山进行第三次“会剿”的紧急形势,决定撤离井冈山向赣南发展,在被誉为“东井岗”的东固与李文林部会合。

毛、朱与李文林部会合之初,双方关系亲密。但是自~1929~年下半年到~1930~年~2~月初,赣西南出现了复杂的局面。随着~1929~年毛泽东率红四军两度进出赣西南和彭德怀所率的红五军于~1930~年初分兵在赣西南游击,经历多次组合的江西地方红军和赣西南党团机构,在若干问题上与毛泽东产生了意见分歧,和毛泽东的关系也日趋紧张。

赣西南方面与毛泽东的分歧主要集中在两个问题上:

一、土改政策问题。赣西南方面主张执行中共六大关于“没收豪绅地主土地”的决定,反对毛泽东提出的“没收一切土地”的主张。

二、军队及地方党机构的归属及人员配置问题。1929~年~11~月底,毛泽东提出合并赣西、湘赣边界两特委,成立新的赣西特委。毛并决定将李文林领导的红二、四团合并到彭德怀部,另成立红六军。赣西南方面则认为此决定须经中共中央及中共江西省委的批准才能生效。1930~年~1~月,毛泽东任命红四军干部刘士奇、曾山组成赣西特委,作为领导赣西南等地的最高机构,但是却受到赣西南方面的抵制。

为了解决与赣西南地方红军、党团机构的矛盾,毛泽东于~1930~年~2~月~6~至~9~日,在赣西特委所在地吉安县陂头村主持召开了由红四军前委、红五、六军军委,及其下属各行委、中心区委、苏维埃党团负责人参加的联席会议,江汉波也以中共江西省委巡视员身份参加会议,刘士奇、曾山作为助手配合毛泽东组织了这次会议。

在“二七”会议上,两个月前经中共中央批准恢复了对红四军领导权的毛泽东,在刘士奇、曾山的帮助下,发动了一场对赣西南地方红军和党团机构负责人的严厉斗争,这场斗争为日后掀起“肃~AB~团”运动埋下了火种。

毛泽东等列举的赣西南地方红军和党团机构负责人的“严重政治错误”主要有两项:

一、毛等批评赣西南负责人江汉波、李文林提出的仅“没收豪绅地主的土地”的主张,是“完全走向农村资产阶级(富农)的路线”,指出“由此发展下去势必根本取消工人阶级争取农民的策略,走上托拉茨基陈独秀的道路,根本取消土地革命全部”。

二、毛等指责江汉波等“用非政治的琐碎话,煽动同志反对正确路线的党的领袖”\footnote{《前委开除江汉波党籍决议》(1930~年~4~月~4~日),载江西省档案馆、《中央革命根据地史料选编》,上册(南昌:江西人民出版社,1983~年),页~576~—~77。}——这里所提到的“党的领袖”是指由毛泽东派任赣西特委书记的刘士奇。

由毛泽东主持的“二七”会议将赣西南方面的负责人扣上“富农分子”的帽子,宣布开除江汉波的党籍,李文林被调出主力部队,转任赣西南特委常委兼军委书记及赣西南苏维埃政府党团书记。2~月~16~日,由毛任书记的红四军总前委发出前委通告第一号,正式宣布开展“肃清地主富农”的斗争,通告指出:

\begin{quoting}
赣西南党内有一严重的危机,即地主富农充塞党的各级地方指导机关,党的政策完全是机会主义的政策,若不彻底肃清,不但不能执行党的伟大的政治任务,而且革命根本要遭失败。联席会议号召党内革命同志起来,打倒机会主义的政治领导,开除地主富农出党,使党迅速的布尔塞维克化\footnote{《前委通告第一号》(1930~年~2~月~16~日),载《中央革命根据地史料选编》中册,页~173。}。
\end{quoting}

以此通告发布为标志,江西苏区开始了持续两年的“肃~AB~团”的斗争,很快“肃~AB~团”的野火就迅速在赣西南蔓延开来。

江西苏区的“肃~AB~团”运动前后历经两个阶段。第一阶段:1930~年春至~1931~年~1~月;第二阶段:1931~年~5~月至~1932~年初,“富田事变”则发生在第一个阶段的后期。

“打倒机会主义领导”在此前还是一个党内斗争的概念,这个~1927~年“八七会议”前后始出现的概念,也只是宣布改变陈独秀的路线并中止其在中央的领导职务。在这之后,中央虽已愈益强调思想统一,但党内还保留了若干大革命时期民主传统的痕迹。依那时的党道德和党伦理,不同意见仍可在党内争论,中共中央或莫斯科共产国际总部则拥有最终裁判权,未闻有将持不同意见的党内同志予以肉体消灭的事例,但是到了~1930~年,毛却将“打倒机会主义领导”与肉体消灭结合起来。

将一个党内斗争的概念转变成一个对敌斗争的概念,这中间需要过渡和转换,毛泽东轻而易举就找到了这个中介环节。他宣布,党内机会主义领导本身就是地富反革命分子,从而将对敌斗争的口号——打倒国民党、消灭地富反革命,与党内斗争的概念“打倒机会主义领导”顺利地衔接起来,一举获得了镇压的正当性,所谓镇压有据,消灭有理。1930~年春,在苏区内已流传国民党~AB~团进行渗透破坏、其组织已被陆续破获的消息,根据地的对敌警惕气氛空前高涨,在这样的形势下,毛完全可以用“镇压反革命”来要求红军和根据地的党组织、苏维埃政权全力支持和服从镇压“AB~团”的政策。

“二七”会议后,革命恐怖的气氛在江西苏区已逐渐形成,赣西南特委在书记刘士奇的领导下,遵照红四军前委关于“各级领导机关已充塞地主富农”、“打倒机会主义的政治领导”的(第一号通告)的精神,率先发动“肃~AB~团”的宣传攻势,6~月~25~日,中共赣西南特委西路行委印发了《反改组派~AB~团宣传大纲》,命令各级组织:

\begin{quoting}
如发现群众中有动摇表现不好的分子,应捉拿交苏(维埃)政府究办,凡出来生疏的经过赤色区域必须严格检查,如有嫌疑应即拘捕交苏维埃政府,赤色区域内的民众流通应持所属苏维埃通行条子。……工农群众只有阶级之分,不要顾至亲戚朋友关系,凡是来到自己家里或发现其它地方有行动不对的人不论亲戚朋友,应报告苏维埃拿办。
\end{quoting}
《大纲》同时号召“实行赤色清乡”和“赤色恐怖”以“肃清红旗下的奸细”:

\begin{quoting}
现在各级苏维埃应加紧肃清反革命的工作,捕杀豪绅地主反动富农分子以示警戒,但是杀人要有反动事实可证,严禁误杀\footnote{《中央革命根据地史料选编》,下册,页~634~—~635。}。
\end{quoting}

这份《大纲》虽然提到杀人要有证据,严禁误杀,但是恐怖大门一经打开,事态很快就失去了控制。

1930~年~7、8~月间,“肃~AB~团”迅速从基层清洗转移到上级机关。8~月,李文林出任根据李立三攻打大城市命令而新成立的江西省行动委员会的书记,在“打~AB~团”的积极性方面李文林并不输于前任特委书记刘士奇,甚至有过之而无不及。

赣西南特委首先选择“工作消极,言论行动表现不好”的团特委发行科工作人员朱家浩作突破口,据赣西南特委~1930~年~9~月~24~日印发的《紧急通告第~20~号——动员党员群众彻底肃清~AB~团》文件透露,朱家浩被拘押后,“特委即把他拿起审讯”,起初他坚决不肯承认,后“采用软硬兼施的办法严审他,才供出来,红旗社、列宁青年社、赣西南政府,都有~AB~团的小组……所有混入在党团特委的赣西南政府的~AB~团分子全部破获,并将各县区的组织通通供报出来了”。

这份《紧急通告》还详细规定了“肃~AB~团”的审讯方法和处决人犯的基本原则,这就是鼓动采用肉刑逼供和对“AB~团分子”实行“杀无赦”。《紧急通告》说:

\begin{quoting}
AB~团非常阴险狡猾奸诈强硬,非用最残酷拷打,决不肯招供出来,必须要用软硬兼施的办法,去继续不断的严形(刑)审问忖度其说话的来源,找出线索,跟迹追问,主要的要使供出~AB~团组织以期根本消灭。
\end{quoting}
一经发现“AB~团”分子,下一步就是枪毙。《紧急通告》要求:

\begin{quoting}
对于首领当然采取非常手段处决,但须注意,在群众大会中由群众斩杀。……富农小资产阶级以上和流氓地痞的~AB~团杀无赦。……工农分子加入~AB~团有历史地位,而能力较活动的杀无赦\footnote{《中央革命根据地史料选编》,下册,页~646、648~—~49。}。
\end{quoting}

赣西南特委厉行“肃~AB~团”,到了~10~月,在赣西南三万多共产党员中已“开除的地主富农有一千多人”(“二七”会议要求把“党内代表富农分子不论其阶如何及过去工作如何,无情地坚决地开除出党”),消灭了一千多“AB~团”\footnote{《中央革命根据地史料选编》,上册,页~626、631。}。赣西南苏维埃政府工作人员的四分之一被打成“AB~团”也大多被杀\footnote{《中央革命根据地史料选编》,下册,页~110。}。

在赣西南特委大张旗鼓“肃~AB~团”时,毛泽东因忙于主持军中事务,并没有直接插手地方的“打~AB~团”,进入~10~月后,毛的态度发生转变。1930~年~10~月~14~日,毛泽东在江西吉安县城给中共中央写了一封信,通报他对赣西南党的状况的看法及准备采取的措施。毛在这封信中继续发展了他在“二七”会议期间对赣西南党团机构的基本观点,指出:“近来赣西南党全般的呈非常严重的危机,全党完全是富农路线领导,……党团两特委机关、赣西南苏维埃政府、红军学校发现大批~AB~团分子,各级指导机关,无论内外多数为~AB~团富农所充塞领导机关”。毛申明,为“挽救这一危机”,决定进行一场以“打~AB~团”为号召的肃反运动,以对赣西南的党团机构“来一番根本改造”\footnote{中共中央文献研究室编:《毛泽东年谱(1893~—~1919)》,上卷(北京:中央文献出版社、人民出版社,1993~年),页~319。以下称《毛泽东年谱》。另参见戴向青、罗惠兰:《AB~团与富田事变始末》(郑州:河南人民出版社,1994~年)页~90。}。

毛泽东真的相信有如此众多的“AB~团”吗?答案是似信非信。1927~年后,为生存而奋斗的中共,长期处在被封锁和剿杀的极端残酷的环境下,作为一种自卫反应,毛习惯将国民党的反共行为给予严重的估计,在激烈的国共斗争中,毛已形成一种思维定式:即对于国民党在共产党区域的活动,宁可信其有,不可信其无。若从“警惕性”方面而言,中共高级领导人当中未有超过毛泽东者。在另一方面,连李文林等人也大打“AB~团”,毛就更没有理由不相信有大批“AB~团”了。

然而毛泽东又绝对是一个现实主义者,他十分清楚,苏区不可能有这么多的“AB~团”,既然恐怖闸门已开,正可顺势引导,将所有公开和潜在的反对派一并镇压下去。毛作出了一个惊人之举:在镇压李文林等赣西南领导人之前,率先在自己指挥的红一方面军(红一、三军团)进行大清洗,开展“打~AB~团”运动。

就在赣西南肃“AB~团”的大背景下,1930~年~11~月,红一方面军总前委在毛泽东的主持下,也在红一方面军(红一、三军团)大开杀戒,开展了“打~AB~团”运动。

1930~年~10~月,毛泽东率红一方面军攻占吉安,旋又退出,毛并动员彭德怀在打下长沙后退出。毛的这些举措在部分红军指战员中引起不满,军中一时思想混乱。为了消除军中的不稳定情绪,毛在率军退出吉安后,于~11~月下旬至~12~月中旬在一方面军迅速发动“快速整军”——其主要内容就是在师、团、营、连、排成立肃反组织,捕杀军中地富出身的党员和牢骚不满分子。在不到一个月的时间内,在四万多红军中肃出四千四百余名“AB~团”分子\footnote{毛泽东:《总前委答辩的一封信》(1930~年~12~月~20~日),载中国人民解放军政治学院编:《中共党史教学参考资料》(北京:中国人民解放军政治学院印行,1985~年);第~14~册,页~634。},其中有“几十个总团长”(指~AB~团总团长),这些人都遭处决。

红一方面军内部的“打~AB~团”极为惨烈,地富或知识分子出身的党员,过去曾与毛泽东意见相左的同志,人人自危,朝不保夕。

黄克诚当时任红三军团第三师政委,该师组织科长、政务科长都被当作~AB~团被肃掉,宣传科长何笃才在大革命时期加入共产党,参加过南昌暴动,随朱德上井冈山后,曾在危急形势下将被上司叛变拉走的队伍重新拉了回来。只因何笃才在古田会议前的朱、毛争论中,站在朱德一边,从此便不被毛重用,不久即将其调出红一军团,在黄克诚手下当个宣传科长。何笃才与黄克诚有友谊,“两人在一起无所不谈”。何笃才认为,毛泽东这个人了不起,论本事,还没有一个人能超过毛泽东,他的政治主张毫无疑问是最正确的。但毛的组织路线不对头,“毛泽东过于信用顺从自己的人,对持不同意见的人不能一视同仁,不及朱老总宽厚坦诚。”何笃才并举例说,一些品质很坏的人就是因为会顺从,受到毛的重用,被赋予很大的权力,干坏了事情也不被追究\footnote{《黄克诚自述》(北京:人民出版社,1994~年),页~100~—~101.}。果其不然,这位聪颖过人、毫无过错的何笃才不久就被扣之以“AB~团”的恶名遭到杀害。

据萧克回忆,在“肃~AB~团”达到高潮的~1930~年~11~月底至~12~月初,他所在的师“没干什么别的事,主要精力就是打~AB~团”,杀了六十人。十几天后,该师又决定再杀六十多人,后经军政委罗荣桓的援救,萧克迅速赶往刑场,予以制止,才救出了二十几人,“但还是杀了二十多人”\footnote{《萧克谈中央苏区初期的肃反运动》,载中国革命博物馆编:《党史研究资料》,1982~年第~5~期。}。

毛泽东既然对自己直接率领的红一方面军也大开杀戒,那么对一贯与自己唱反调的赣西南地方红军就更不会有丝毫顾惜了。如前所述,李文林在“肃~AB~团”问题上态度十分坚决,但是到了~1930~年~10~月,随着“肃~AB~团”中暴露出的乱打乱杀现象日益严重,李文林的态度开始转趋冷静,10~月底,省行委已着手部署纠偏措施,但令人回味的是,当李文林的肃反狂热降温时,毛泽东却开始升温了。

1930~年~11~月,毛泽东“根本改造”的利刃终于刺向中共江西省行动委员会及其所辖的赣西南地方红军。此次行动更因李文林等坚持执行李立三中央的路线,反对弃攻南昌的主张而更加惨烈。

1930~年~5~月,赣西南代表李文林赴上海参加了由李立三主持的全国苏维埃区域代表会议,此次会议要求集中攻打大城市,争取一省、数省的首先胜利。李文林返回后,于~8~月上旬主持召开了赣西南特委第二次全体委员会议,部署贯彻李立三的指示。“二全会”不指名地指责了毛泽东的一系列观点和做法,撤销了拥护毛的主张、被毛派至赣西南特委任书记的刘士奇的职务,并建议上海中央开除其党籍。这一切引致毛的极大不满,毛早已习惯以中央在江西苏区最高代表行事,岂能容忍有人在自己眼皮下以拥护中央为名反对自己?此时毛尚不知“立三路线”这个词语,于是认定“二全会”是“AB~团取消派”的会议,并将参加“二全会”的人一律视为“AB~团”分子,列入应予“扑灭”的范围。

1930~年~10~月,红一方面军攻下吉安,在国民党地方当局的文件中发现了一张据称是李文林的地主父亲用真名签字的便条\footnote{参见《龚楚将军回忆录》,页~353;但据~1987~年中共吉水县党史办的调查报告称,李文林的父亲只是富裕中农,且早在~1927~年~5~月就病故。参阅《关于李文林被错杀情况的调查》,载中共江西省党史资料征集委员会、中共江西省党史研究室编:《江西党史资料》,第~1~辑,页~326。},这张字条究竟是何内容不得而知,然而将李文林与“AB~团”联系在一起已经有了所谓证据。10~月中旬和月底,李文林又在峡江会议和罗坊会议上公开反对毛泽东提出的“诱敌深人”的军事作战方针,主张执行李立三有关攻打大城市的指示,与毛的冲突全面激化,由此毛认定李文林就是“AB~团”首领。1930~年~11~月底,李文林在宁都县黄陂被拘押,紧接着一批与李文林有工作联系的人相继被捕。红一方面军总前委根据犯人被刑讯后的口供,于~1930~年~12~月~3~日写信给改组后的江西省行委(李文林被捕后,由毛的老部下曾山领导)——(此信其实是毛泽东所写,大陆学者为避毛讳,有意隐去毛的名字),认定段良弼(省行委常委,赣西南团特委书记)、李白芳(省行委秘书长)等为~AB~团分子,命令“捕捉李白芳等并严搜赣西南的反革命线索,给以全部扑灭”。毛在这封信中要求省行委接到此信后“务必会同李同志(即李韶九)立即执行扑灭反革命的任务,不可有丝毫的犹豫”,对“各县各区须大捉富农流氓动摇分子,并大批把他们杀戮。凡那些不捉不杀的区域,那个区域的党与政府必是~AB~团,就可以把那地方的负责人捉了讯办”\footnote{转引自戴向青、罗惠兰:《AB~团与富田事变始末》,页~98。}。

李韶九携毛泽东指示信于~12~月~3~日前往富田,5~日毛又将第二封指示信派两位红军战士给已出发的李韶九和省行委送去。毛在信中指示彼等要从已被捉的人的线索中“找得更重要的人”,为了督促贯彻两封指示信,毛又派出总前委秘书长古柏赶往富田“协助肃反”。

12~月~7~日下午,时任红一方面军总政治部秘书长兼肃反委员会主席的李韶九代表总前委,到达江西省苏维埃政府所在地富田,指导江西省行委实施总前委关于肃反的部署。李韶九向曾山(江西省苏维埃政府主席)和陈正人面交了毛泽东的指示信,随即将省行委和红二十军八个主要领导人段良弼、李白芳、金万邦(省苏维埃政府军事部长)、周冕(省苏维埃政府财政部长)、谢汉昌(红二十军政治部主任)、刘万清、任心达、马铭等人予以逮捕。李韶九对这批同志施用了“打地雷公烧香火”等多种刑法,被打同志“皆体无完肤”,“手指折断,满身烧烂行动不得”,有的人被当场折磨致死,而每一次用刑都有李韶九在现场指挥。据当时资料记载,被害同志“哭声震天,不绝于耳,残酷严刑无所不用其极”。12~月~8~日,李白芳、马铭、周冕的妻子来看被拘押中的丈夫,也被当作“AB~团”抓起来,被施以严刑“用地雷公打手,香火烧身,烧阴户,用小刀割乳”\footnote{《省行委紧急通告第九号》(1930~年~12~月~15~日),转引自戴向青、罗惠兰:《AB~团与富田事变始未,页~105。}。在惨酷的刑讯下,段良弼供出李文林、金万邦、刘敌、周冕、马铭、任心达、丛允中、段起风等“是~AB~团首颌,并供出红军学校有大批~AB~团”。对于这次刑讯逼供,萧克将军在~1982~年曾回忆道,“即便过了半个世纪,也不能不令人惨然一叹。我们这些‘过来人’也觉不堪回首。”\footnote{《萧克谈中央苏区初期的肃反运动》,载中国革命博物馆编:《党史研究资料》,1982~年第~5~期。}

12~月~7~日至~12~日晚,在短短的五天时间里,李韶九(于~9~日离开富田)、省苏维埃主席曾山和总前委秘书长古柏(于~8~日到达)坐镇富田,厉行肃反。曾山亲自审讯段良弼,所得结果是抓出“AB~团”一百二十多名,要犯几十名,先后处决四十余人\footnote{曾山:《为“富田事变”宣言》(1931~年~1~月~14~日),转引自戴向青、罗惠兰:《AB~团与富田事变始未》,页~105~—~106。},其中李韶九在动身前往东固前亲自布置将二十五人处决。在这之前的~12~月~9~日,当总前委派来“协助肃反”的古柏赶到富田后,李韶九带一个排警卫,押着被捕的红二十军政治部主任谢汉昌离开富田,于~10~日到达红二十军军部所在地东固,立即与军长刘铁超、政委曾炳春研究如何贯彻毛泽东两封指示信,“找得线索来一个大的破获”。李韶九、刘铁超、曾炳春根据段良弼、谢汉昌被刑讯后的口供,认定红二十军~174~团政委刘敌是~AB~团分子。但因李韶九与刘敌是同乡,李韶九虽然在谈话中已点出刘是~AB~团分子,但是并没有立即将刘敌逮捕,只是“采用软硬兼施的方法”,迫使刘敌自己供认。刘敌在富田事变后,写给中央的信中承认他在同李韶九谈话后,即有了发动事变的念头。刘敌并在信中陈述说,他一向知道“李韶九是素来观念不正确,无产阶级意识很少的一个惯用卑鄙手腕,制造纠纷”的人。为了躲过马上就要降临的刑讯逼供,刘敌改用长沙口音对李韶九说,“我是你老人家的部下,……现在幸喜你老人家来了,我只有尽量接受政治教育,承认错误,我相信毛泽东同志总不是~AB~团,军长总不是~AB~团,我总为你三位是追是随,我个人还有什么呢。”李韶九见刘敌这番表态,就放刘敌回去了\footnote{刘敌:《给中共中央的信》(1931~年~1~月~11~日),转引自戴向青、罗惠兰:《AB~团与富田事变始末》,页~107~—~108。}。

12~月~12~日,富田事变爆发。这天上午早饭后,刘敌同独立营长张兴、政委梁学贻秘密开会商量应对李韶九的措施。三人一致认为,红一方面军总前委来抓~AB~团是打击赣西南党的干部“阴谋计划的组成部分”,为了防止阴谋得逞,决定立即逮捕李韶九和红二十军军长刘铁超等人。会后,刘敌即至独立营对战士进行鼓动,率领全营红军战士包围了军部,逮捕了军长刘铁超,释放了谢汉昌等人,李韶九被捉后逃走,军政委曾炳春也跑回家乡躲避。傍晚,谢汉昌、刘敌率红二十军军部直属独立营冲到富田,包围了省行委和省苏维埃政府,释放了在押的段良弼、李白芳等“AB~团分子”七十余人。中央提款委员易尔士(刘作抚)也被捉了起来(次日即被释放,并被邀请在群众大会上发表演讲)。省苏维埃政府主席曾山趁乱逃出富田,跑回家乡。古柏也从“肃反机关跑了出来”\footnote{曾山:《赣西南苏维埃时期革命斗争历史的回忆》(1959~年~6~月~12~日),载陈毅、萧华等:《回忆中央苏区》(南昌:江西人民出版社,1981~年),页~21~—~23。}。古柏之妻曾碧漪、陈正人之妻彭儒均趁黑夜逃走。这就是震惊江西苏区的富田事变。

富田事变发生后,谢汉昌、刘敌等把所率领的红二十军带到赣江以西湘赣苏区永新、莲花、安福一带,继续展开土地革命,并在吉安县永阳成立了“江西省行动委员会”和“江西省苏维埃政府”,谢汉昌、刘敌采取了四项紧急措施:

一、派遣段良弼携二百斤黄金紧急赶往上海(实际带到上海中央的只有“几十两”)\footnote{参见张国焘:《我的回忆》,第~2~册(北京:现代史料编刊社,1980~年),页~484;另参见何盛明:《陈刚》,载《中共党史人物传》,第~34~卷(西安:陕西人民出版社,1987~年),页~211。},向中共中央汇报赣西南“肃~AB~团”及富田事变经过,请中共中央裁决。

二、通缉曾山、陈正人、古柏、李韶九。省行委认为,曾山等配合李韶九滥抓“AB~团”,有不可推卸的责任,应缉拿归案。

三、争取赣西南特委下属的湘赣苏区西路行动委员会书记王怀的同情与支持(1930~年~12~月~9~日,毛泽东的老部下、原省行委常委、宣传部长陈正人率一个排兵力去西路行委,贯彻总前委两封信的精神,准备逮捕行委书记王怀,但未果)。在王怀领导下的河西苏区、红二十军的反毛行动受到普遍同情,王怀的观点——红二十军行动不是反革命行为,而是“工人阶级路线与农民路线两条路线斗争”,被迅速传播开来。

富田事变当事人之一的曾山,在几十年后对此还记忆犹新。他说,当时“河西苏区党员和群众的思想极端混乱,甚至还影响到河东苏区部分人民和部分党员的认识也逐渐模糊起来”\footnote{曾山:《赣西南苏维埃时期革命斗争历史的回忆》(1959~年~6~月~12~日),载陈毅、萧华等:《回忆中央苏区》(南昌:江西人民出版社,1981~年),页~21~—~23。}。由此可见,当年毛泽东的极端行为造成的影响是何等广泛。

四、公开打出反毛泽东旗帜,并试图争取朱德、彭德怀、黄公略、滕代远的支持。谢汉昌、刘敌在向赣江西边转移途中,张贴大量标语和《告同志和民众书》,指责毛泽东有“党皇帝思想”,宣称“党内大难到了”并提出“打倒毛泽东,拥护朱、彭、黄”的口号。12~月~20~日,谢汉昌、李白芳、丛允中等在永阳镇写了《致朱德、彭德怀、黄公略、滕代远信》,这封信一方面谴责李韶九大抓~AB~团,对同志滥捕滥杀,同时又抨击总前委偏护李韶九,还附上了伪造的《毛泽东给古柏的信》,离间毛与朱、彭、黄的关系。

《毛泽东给古柏的信》普遍被认为是一封伪造信,当事人彭德怀的证据可能最有说服力。数十年后,彭德怀在狱中写交待材料回忆此事时说:“这封信是富田事变的头子丛允中写的,他平日学毛体字,学得比较像,但是露出了马脚——毛泽东同志写信,年、月、日也是用汉字,不用罗马字和阿拉伯字。”

这封伪造的毛泽东致古柏的信,自~1930~年代后,一直未予公开,直到~1985~年江西出版的一本有关中央苏区的历史著作中才首次予以全文披露:

\begin{quoting}
古柏同志:据目前各方形势的转变,及某方来信,我们的计划更要赶快的实现,我们决定捕杀军队~CP~与地方~CP,同时并进,并于捕杀后,即以我们的布置出去,仅限三日内将赣西及省行委任务完成,于拷问段(指段良弼——引者注,下同)、李(指李白芳)、王(指王怀)等中坚干部时,须特别注意勒令招出朱、彭、黄、滕系红军中~AB~团主犯,并与某方白军接洽等罪状,送来我处,以便早日捕杀,迅速完成我们的计划,此信要十分秘密,除曾(指曾山)、李(指李韶九)、陈(指陈正人)三人,任何人不准告之~10/12~毛泽东\footnote{见戴向青:《中央革命根据地研究》(南昌:江西人民出版社,1985~年),页~252。}。
\end{quoting}

朱德、彭德怀、黄公略闻知此信有不同的反应。朱德随红一方面军总前委驻在黄陂,没有直接领军,因此“离间”能否成功,关键在于手上握有一万兵力的红三军团司令员彭德怀及其副手黄公略。

1930~年~12~月中旬,彭德怀接到谢汉昌等人的信并《毛泽东给古柏的信》,当即作出判断,认定此是“分裂党、分裂红军的险恶阴谋”,彭德怀迅速草拟一份“不到二百宇的简单宣言”,宣称“富田事变是反革命性质的”,表示三军团“拥护毛泽东同志,拥护总前委领导”。

至于黄公略的态度则较为暧昧,据彭德怀回忆:“我讲这段话时(指彭分析《毛泽东给古柏的信》是伪造的假信)黄公略来了,大概听了十来分钟就走了。会后我问邓萍同志,公略来干吗?邓说,他没说别的。只说:‘老彭还是站在毛这边的。’他就走了。”\footnote{《彭德怀自述》(北京:人民出版社,1981~年),页~166。}

在彭德怀的解释和说服下,红三军团的“情绪转变过来了,把愤恨转到富田事变上”,彭又把部队开到距黄陂总前委所在地十五里的小布,亲自请毛泽东来三军团干部会上讲话,以表示对毛泽东的坚决支持。

在富田事变后的紧张形势下,彭德怀及三军团对毛泽东的支持具有极重要意义,此举巩固了毛泽东已遭动摇的地位。但是事变领导人到处散布的反毛的舆论毕竟已严重损害了毛的声望,毛泽东为了反驳赣西南方面的抨击,亲自出马,毫无愧作,于~1930~年~12~月~20~日草写了《总前委答辩的一封信》。

在这封答辩信中,毛泽东坚持认为“肃~AB~团”均是有根有据的。他说:红军中“AB~团”要犯的口供“多方证明省行委内安了江西~AB~团省总团部,段良弼、李白芳、谢汉昌为其首要,总前委为挽救赣西南的革命危机,派李韶九同志前往富田捕捉”。毛认定段良弼等为“AB~团”首犯乃是证据确凿,他说:“如果段、李、金、谢等是忠实革命的同志,纵令其一时受屈;总有洗冤的一天,为什么要乱供陷害其他同志呢?别人还可以乱供,段、李、金、谢这样负省行委及军政治部主任责任的为什么可以呢?”\footnote{毛泽东:《总前委答辩的一封信》(1930~年~12~月~20~日),载中国人民解放军政治学院编:《中共党史教学参考资料》(北京:中国人民解放军政治学院印行,1985~年);第~14~册,页~634。}毛明知将段等定为“AB~团”全靠刑讯逼供,却对刑讯逼供无只字批评,反而指责段等不能为革命一时受屈,而不能为革命受屈,就一定是心中有鬼,照毛的逻辑,只要段良弼等自己承认是“AB~团”头子,即可证明彼等系货真价实的“AB~团”——毛的这种逻辑和思维方式,成为日后极左的审干肃反的常规思路,是造成冤假错案连绵不断的最重要的思想根源。在这样的思路下,毛坚持“肃~AB~团”不仅无错,反而是对革命的巨大贡献。他说,“AB~团已在红军中设置了~AB~团的总指挥、总司令、军师团长,五次定期暴动,制好了暴动旗,假设不严厉扑灭,恐红军早已不存在了。”毛声称富田事变将“叛逆的原形完全现出来了”,号召对事变进行坚决镇压\footnote{毛泽东:《总前委答辩的一封信》(1930~年~12~月~20~日),载中国人民解放军政治学院编:《中共党史教学参考资料》(北京:中国人民解放军政治学院印行,1985~年);第~14~册,页~634。}。

1930~年~12~月中下旬,毛泽东以中国工农革命委员会的名义起草了一份六言体的讨伐富田事变的布告:

\begin{quoting}
段谢刘李等逆,叛变起于富田。

带了红军反水,不顾大敌当前。

分裂革命势力,真正罪恶滔天。

破坏阶级决战,还要乱造谣言。

进攻省苏政府,推翻工农政权。

赶走曾山主席,捉起中央委员。

实行拥蒋反共,反对彻底分田。

妄想阴谋暴动,破坏红军万千。

要把红色区域,变成黑暗牢监。

AB~取消两派,乌龟王八相联。

口里喊得革命,骨子是个内奸。

扯起红旗造反,教人不易看穿。

这是蒋逆毒计,大家要做宣传。

这是斗争紧迫,阶级反叛必然。

不要恐慌奇怪,只有团结更坚。

打倒反革命派,胜利就在明天。\footnote{《黄克诚自述》(北京:人民出版社,1994~年),页~85。}
\end{quoting}

毛泽东理直气壮乃是他认定自己就是红军和党的象征。毛就是根据地的中央,就是共产国际在中国的代表,反毛即是“AB~团”,所杀的皆是反革命,何愧之有!在毛的眼里,只要目标崇高——扑灭“AB~团”就是保卫革命,即使手段严厉一些,也无关紧要。在大恐怖中,总前委和毛的个人权威得到完全确立,毛就在大恐怖中成了江西苏区的列宁!

二、历时四个月的项英对毛泽东的纠偏

1931~年~1~月~15~日,中共苏维埃区域中央局在宁都小布宣布成立,项英正式走马上任,根据中共中央政治局的决定,由项英担任代理书记,取消总前委和由毛泽东担任的总前委书记的职务,毛泽东、朱德等参加中央局。在苏区中央局宣布成立的同时,还建立了归苏区中央局领导的中央革命军事委员会,统领江西和全国红军,项英兼任主席,朱德、毛泽东任副主席。至此,从党的法理上,项英已取代了毛泽东的地位,成为江西苏区中共党、军队的最高领导人。

项英前来苏区及苏区中央局的建立,是处于转折年头的中共实现其将工作重心向苏区转移的重大战略步骤,是落实斯大林及共产国际有关指示的具体行动。1930~年~7~月下旬,斯大林在莫斯科接见前来汇报工作的周恩来,指示中共应把红军问题放在中国革命的第一位。7~月~23~日,共产国际执委会政治秘书处发出《关于中国问题的决议案》,指出,组织苏维埃中央政府和建立完全有战斗力和政治坚定的红军,“在现时中国的特殊条件之下,是第一等任务”\footnote{《周恩来年谱》,页~183。}。

中共工作重心向江西苏区转移,首当其冲的问题是如何协调中央与毛泽东的关系,及如何评价毛在江西的工作。从这一时期周恩来的言论中可以看出,中共中央对毛并不尽然满意,但是周恩来却常以自我批评的口吻谈论这类问题。1930~年~8~月~22~日,周在中共中央临时政治局会议上发言说:“我们过去一方面屡屡批评农民保守观念的错误,另一方面反对单纯军事游击式的策略,中央还特别提出割据的错误,对于根据地确实注意得比较少,这是工作中的缺点”\footnote{《周恩来年谱》,页~185。}。

既然已经发现了问题的症结所在,下一步的措施就必然是加强中央对苏区的领导和红军中党的领导。在~9~月~29~日政治局会议上,周恩来要求中央派自己前往苏区工作。次日,周恩来又在中共中央军委扩大会议上强调在红军中党的领导要有最高权威。

1930~年~10~月~3~日,六届三中全会后的全党最高核心——中共中央政治局三人常委会成立(由向忠发、周恩来、徐锡根组成,周恩来为实际负责人),初步决定由周恩来、项英、毛泽东、余飞、袁炳辉、朱德和当地一人组成苏区中央局,派项英先行前往江西。10~月~17~日中央政治局最后确定组成以周恩来为书记的苏区中央局,在周恩来未到达之前,由项英代理书记一职,以苏区中央局为苏区党、军、政最高领导机构。10~月~29~日周恩来起草中共中央致红一、红三军团前委的指示信,通知毛泽东:“苏区中央局在项英未到前,可先行成立,暂以毛泽东代理书记,朱德为红一、红三军团总司令。目前一切政治军事领导统一集中到中央局。”\footnote{《周恩来年谱》,页~192。}

至~1930~年~10~月,中共中央为贯彻斯大林和共产国际指示的具体措施已经落实就绪。为配合中共中央向苏区的转移,周恩来在~9、10~月采取了更为细致的部署:

在上海举办从苏联返国准备前往苏区的军事训练班,一批军政干部如张爱萍、黄火青等参加学习后被派往江西苏区。

安排从苏联学习返国的刘伯承、叶剑英、傅锺、李卓然等把苏联红军步兵战斗条令和政治工作条例译成中文,并送往苏区。

主持打通了比较固定和安全的由上海前往江西苏区的秘密交通线,成立了以吴德峰为局长的中共中央交通局。

积极筹备建立自莫斯科共产国际总部至上海共产国际远东局的大功率秘密电台和上海中共中央机关至江西苏区的无线电联系,莫斯科——上海——江西苏区的通讯渠道即将全部打通。

项英就是在这样的背景下,以中共江西苏区最高负责人的身份,肩负加强中共中央对江西红军领导的重大使命,沿地下秘密交通线于~1930~年底抵达江西苏区。

项英是中共党内少数出身产业工人的高级领导人之一,他于~1921~年在武汉加入中共后,长期从事工人运动,曾在~1928~年赴莫斯科参加中共六大,是~1925~年中共四大后的历届中央委员会委员,在中共六届一中全会上又被选为中央政治局委员、常委。项英对从苏联学来的马列理论有着坚定的信仰,对斯大林和苏联的“感情”较深,个人性格和作风则比较拘谨和严肃。

1930~年~11~月下旬,项英从上海出发,当他刚一抵达江西苏区就闻知不久前在赣西南红军内部爆发了一场矛头直指红四军总前委书记毛泽东的富田事变。

项英领导的苏区中央局成立后的第一项工作就是处理富田事变。1931~年~1~月~16~日,发出《苏区中央局通告第二号——对富田事变的决议》,一方面表示“完全同意总前委对富田事变所采取的斗争路线”;另一方面,又在相当程度上冲淡了毛泽东等对富田事变性质的看法,主张采取较为缓和的、有区别的政策,以缓和苏区内部的紧张关系,避免红军的分裂。

《决议》的矛盾性和含混性集中体现在对富田事变性质的看法上。项英认为“江西省行委中之段良弼、李白芳及二十军政治部主任谢汉昌等均系~AB~团要犯”,彼等发动“富田事变”是“分裂革命势力”、“分裂红军”的“反党行为”,并决定“将富田事变的首领段良弼、李白芳、谢汉昌、刘敌、金万邦等开除党籍”;但与此同时,项英又声称富田事变不是~AB~团领导的反革命暴动,而是“无原则的派别斗争”。并责令赣西南特委和红二十军党委,停止党内互相攻击,听候中央局调查处理。

如果说,项英在对富田事变性质的认识上采取了折衷主义的立场,那么,项英针对“肃~AB~团”扩大化的尖锐批评,几乎就是直接指向毛泽东了。《决议》重点批评了“过去反~AB~团取消派斗争中的缺点和错误”,并列举其主要表现:“第一非群众路线,许多地方由红军或上级机关代打”,“第二是盲动,没有标准,一咬便打”。项英强调:今后“必须根据一定事实和情形,绝对不能随便乱打乱杀”,“也不能随便听人乱供乱咬加以逮捕”;“党在每个斗争中都应以教育方式来教育全党党员。这样才能使党走上布尔什维克的道路”\footnote{《苏区中央局通告第二号——对富田事变的决议》(1931~年~1~月~16~日),载《中共党史教学参考资料》,第~14~册,页~639~—~42。}。

项英的上述态度与他对毛泽东的复杂的观感密切相关。项英在大革命时期虽与毛泽东有过一些工作接触,但他在未抵江西之前,对江西苏区的认识全凭在上海中央机关所看到的来自苏区的零散的报告和周恩来的介绍。项英在性格上较为直露和坦率,与毛泽东是完全不同的两类人。因此,一经发现富田事变的原委,项英很快就掩饰不住对毛的不满。但是,项英毕竟是一位老共产党员,十分了解毛在~1927~年后对党与红军的贡献及毛在江西苏区所拥有的举足轻重的地位,自己又甫抵苏区,深知不能公开指责毛泽东,所以在对富田事变性质的判断及处理方法上煞费苦心,既要考虑到维护毛泽东的威信,又要坚决地制止、纠正毛在“肃~AB~团”问题上的错误。然而随着项英逐渐熟悉江西苏区的内情,他原先对富田事变性质的看法进一步发生变化,对毛泽东的批评也日趋尖锐。

1931~年~2~月~4~日,项英以苏区中央局的名义发出《中央局给西路同志信》:“飞函王怀、丛允中等同志及各党部派一人及有关系诸同志(如陈正人,红军学校等)来中央局讨论,将一切得到一个最后的解决。”项英在这封信中还明确表示那种认为“二全会”是“AB~团”会议的看法是错误的,\footnote{毛泽东在《总前委答辩的一封信》中强调:“二全会议主要反对二七会议,开除刘士奇就是反对二七会议,反对毛泽东”。}显示出与毛完全不同的态度。项英这封信表明他已着手准备富田事变的全部善后处理工作。1931~年~2~月~19~日,中共苏区中央局发出第十一号通告,事实上修正了~1~月~16~日《决议》关于富田事变是段良弼等人领导的“反党反革命”行动的看法:

\begin{quoting}
中央局根据过去赣西南党的斗争的历史和党组织基础以及富田事变客观行动事实,不能得出一个唯心的结论,肯定说富田事变即是~AB~团取消派的暴动,更不能有事实证明领导富田事变的全部人纯粹是~AB~团取消派,或者说他们是自觉的与~AB~团取消派即公开联合战线来反党反革命,这种分析和决议正是马克思列宁主义唯物辩证论的运用,是铁一般的正确\footnote{转引自戴向青、余伯流、夏道汉、陈衍森:《中央革命根据地史稿》(上海:上海人民出版社,1986~年),页~311。}。
\end{quoting}

《通告》宣布开除李韶九、段良弼等五人的党籍,对其他人,只要“证明未加入反动组织(AB~团),承认参加富田事变的错误,绝对服从党的决议的条件之下,应允许他们重新回到党的领导下来”。

2~月~19~日后,项英把主要精力放在动员红二十军返回河东,毛泽东尽管感到项英的压力,却因身系事件中心,一时明显处于下风,难以有所作为,只能暂取观望之态,遂把全部身心投入指挥和国民党“围剿”部队的作战中。

项英首先责成富田事变时躲回家乡的红二十军政委曾炳春回到河西红二十军中去作说服动员工作。并随带中央局指示,通知赣西特委负责人和参加事变的领导人回苏区中央局开会,并委派干部去永阳解散由谢汉昌等成立的江西省行动委员会。

是否去苏区中央局开会,这是关系到领导事变负责人的人身安全的关键性问题,在这个节骨眼上,项英的个人威望起决定性的作用。据曾山回忆,谢汉昌等对项英抱很大希望,“估计项英同志是支持他们的”,在这种预期心理的作用下,1931~年~4~月间,富田事变的主要领导者谢汉昌、刘敌、李白芳等及西路行委书记王怀,遵照项英和苏区中央局的指示,回到宁都黄陂苏区中央局驻地参加会议,“向党承认错误,请党教育”,只有段良弼一人因去上海中央汇报富田事变而未前往。红二十军的官兵也遵照苏区中央局的指示,“努力向泰和、固江北路歼灭各地地主武装,夺回被迫反水群众”,但是等待他们的命运却是他们和项英都未曾料到的:中共中央否决了项英对富田事变性质的评价及其处理方法,谢汉昌、刘敌、李白芳等及红二十军、赣西南大批党员干部的出路只有一条:被枪毙!

三、“肃~AB~团”烽火再起:中央代表团贬斥项英,支持毛泽东

根据迄今披露的资料显示,1931~年~2~月~13~日中共中央第一次对富田事变作出反应。

1930~年~11~月至~1931~年~1~月,是中共历史上的一个特殊时期;上海中央内部围绕“纠正立三路线错误”的问题发生了激烈的争论。从莫斯科中山大学返回,原先在中共党内地位较低的陈绍禹(王明)、秦邦宪(博古)、王稼祥等要求召开紧急会议,改组在和“立三路线”斗争中“犯了调和主义错误”的以周恩来、瞿秋白为核心的中共中央;而以何孟雄和罗章龙分别为首的“江苏省委派”和“全总派”,在一度与陈绍禹等联络反中央后,又转而反对陈绍禹新提出的召开中共六届四中全会的主张。党内各派别的争论使中共濒于分裂,最后,在~1930~年~12~月中旬秘密抵达上海的共产国际代表米夫的亲自主持下,中共中央于~1931~年~1~月~7~日召开了扩大的六届四中全会,强行统一了全党的认识。会议改选了中央政治局,陈绍禹在米夫的支持下进入了中央政治局,由周恩来、向忠发、张国焘组成中央常委会,仍由向忠发担任总书记一职,但从此中共中央实际由陈绍禹、周恩来掌握。1~月~27~日,中共中央政治局举行会议,一致通过开除继续反对中央的罗章龙的中央委员及党籍,至此,开始了中共党史上被称之为“王明左倾路线统治时期”。在这次会议后,原有的党内纷争基本结束,中央政治局的工作走上轨道。

一经解决了党内的分裂危机,新成立的中央政治局所处理的第一件大事就是讨论富田事变问题。

1931~年~2~月~13~日,中共中央召开政治局会议,会议的中心议题是讨论富田事变。在这里,有若干问题仍存有疑点:中共中央是如何知道富田事变发生消息的?1931~年~1~—~2~月上海中央与江西苏区的电讯联系还未建立,直到同年秋,才开通了上海中央经香港与江西苏区的电讯联络。有资料显示,富田事变后,中央政治局曾要求毛向中央报告富田事变真相,\footnote{参见张国焘:《我的回忆》,第~2~册(北京:现代史料编刊社,1980~年),页~484;另参见何盛明:《陈刚》,载《中共党史人物传》,第~34~卷(西安:陕西人民出版社,1987~年),页~486。}毛是否对此作过反应?毛泽东在富田事变后,曾写有《总前委答辩的一封信》,这封信是否是给上海中央的?据八十年代后期披露的权威性资料反映,富田事变中被扣押的中央提款委员易尔士(刘作抚)在事变后即被段良弼开释,携在苏区筹集的千两黄金很快返回上海向中央汇报。另有资料透露,1931~年~2~至~3~月,段良弼及江西省团委共三人去上海汇报富田事变经过,博古等会见了他们,并向中央常委会作过报告。博古判定,赣西南来人及其口头叙述与少共中央所收到的赣西南控告毛的文件大体都是真实的。\footnote{参见张国焘:《我的回忆》,第~2~册(北京:现代史料编刊社,1980~年),页~484;另参见何盛明:《陈刚》,载《中共党史人物传》,第~34~卷(西安:陕西人民出版社,1987~年),页~484。}尽管周恩来没有接待过赣西南来人,但有一个问题基本可以确定,这就是在~1931~年~2~月~13~日前,周恩来等已得知富田事变的有关情况,此时的周恩来已意识到在赣西南所发生事件的严重性质,并决定采取相应的组织措施。

周恩来在~2~月~13~日的政治局会议上作了两项决定:第一、立即给江西发一中央训令“停止争论,一致向敌人作战”;第二,重新调整中共苏区中央局人选,决定项英、任弼时、毛泽东、王稼祥为常委。经过这次调整,毛泽东在苏区中央局第二号人物的角色将由任弼时担任,而刚刚在六届四中全会担任中央委员的留苏派王稼祥则进入了苏区中央局最高领导机构。\footnote{《周恩来年谱》,页~203~—~204。}

2~月~16~日,中央政治局又举行会议,决定由周恩来、任弼时、王稼祥组成委员会,研究富田事变的性质及处理意见。2~月~20~日,中央政治局专门开会讨论三人委员会的意见,周恩来代表三人委员会发言:“赣西南的~AB~团是反革命组织,但是尚有动摇的和红军中的不坚定分子,在客观上也可为~AB~团所利用”\footnote{《周恩来年谱》,页~205。}。

会议决定:根据周恩来这一结论,由任弼时起草一信,要江西苏区停止争论,集中一切力量对付敌人,派遣中央代表团前往苏区处理富田事变,中央代表团有全权解决的权限。

2~月~23~日,中共中央发出由任弼时起草的致红一方面军总前委、江西省委、各特委、各地方党部的信,信中指出:

\begin{quoting}
不幸的富田事变,恰恰发生于敌人加紧向我们进攻而红军与群众正在与敌人艰苦作战的当儿,无论如何,总是便利于敌人而削弱我们自己的。中央特决定立即派出代表团前往苏区组织中央局,并委托代表团以全权调查与解决这一问题。在中央代表团没有到达以前,从总前委起,江西省委、各特委、各红军党部一直到各地党的支部都要立即停止这一争论,无条件地服从总前委的统一领导,一致的向敌人进行残酷的战争\footnote{中央档案馆编:《中共中央文件选集》(1931),第~7~册,页~141;另见《周恩来年谱》,页~205。}。
\end{quoting}

任弼时代表中共中央起草的这封信,在两个关键性的问题上,推翻了~1930~年~10~月中央政治局原先作出的决定,沉重地打击了项英。

第一,否认了经中央政治局批准(中央六届三中全会后的政治局)而刚刚成立的中共苏区中央局的合法性,剥夺了项英在江西苏区的最高领导权。

第二,明确规定,在中央代表团抵达之前,毛泽东在江西苏区享有指挥一切的最高权威,重新恢复了被取消的红一方面军总前委,否认了~1~月中旬刚成立的项英领导的中共中央军委的合法性(事实上,1931~年~1~月~30~日,六届四中全会后的中央政治局已决定重新组成由周恩来为书记的新的七人中央军委)。

3~月~4~日,中央政治局常委会议决定,组成由任弼时、王稼祥、顾作霖组成的中央代表团,立即动身前往江西苏区。作为中央六届四中全会后的中共中央派往江西的第一个高级代表团,任弼时、王稼祥、顾作霖的直接使命是代表中央政治局处理富田事变问题,享有明确而全面的授权。为策安全,议定任弼时于~3~月~5~日出发,王稼祥~3~月~7~日启程。

关于共产国际对富田事变的态度,至今没有详尽资料。1931~年春,共产国际常驻中国的机构是设在上海的远东局,负责人罗伯特系德国人,其人在共产国际地位较低,他的意见经常不被中共中央领导人重视和接受。早在~1930~年春由于中共中央与远东局在“富农问题”等意见上的分歧,周恩来曾专程去莫斯科共产国际总部汇报。1931~年夏秋,由于罗伯特向莫斯科汇报了李立三试图把苏联拉入中国内战的情报,以及远东局对李立三的抵制,罗伯特在共产国际的地位有所提高,但仍不具备足够的权威,以至于共产国际专门派遣米夫秘密来华主持召开中共六届四中全会。据有关资料透露,米夫于~1930~年~12~月抵华后,曾在上海秘密逗留半年时间,但迄今也未发现有关米夫对富田事变发表看法的任何资料。

只有一两份资料间接透露了有关共产国际对富田事变的态度。根据《周恩来年谱(1898~—~1949)》一书透露:1931~年~3~月~27~日,中共中央政治局在上海召开了会议,周恩来在会上提出了共产国际远东局的意见。至于远东局意见的具体内容如何,该书未作任何披露。但笔者根据周恩来在~3~月~27~日中央政治局会议上的发言,和次日发表的《中央政治局关于富田事变的决议》,判断共产国际远东局对富田事变的大概意见是:一、富田事变是反革命行动。二、不应夸大敌人在内部进攻的力量。

笔者的这个判断可从另一份资料中得到证实。由中共中央文献研究室编写的《任弼时传》透露:在~1931~年~2~月~20~日讨论富田事变的政治局会议后,与中央政治局的意见相异,共产国际远东局不同意贸然肯定总前委反“AB~团”的行动,因此在由任弼时代表中央政治局起草的~2~月~23~日的信中,没有写上总前委反~AB~团“一般是正确的”这句话。但是到了~3~月~27~日,远东局改变了原有的看法,认定富田事变“是反革命的暴动,前委领导是对的”,甚至要求政治局与远东局联名发表对富田事变表态的决议。\footnote{中共中央文献研究室编:《任弼时传》(北京:中央文献出版社、人民出版社,1994~年),页~209。}这就是次日发出的《中央政治局关于富田事变的决议》。

1931~年~3~月~28~日中共中央发出的《决议》究竟是谁起草的,迄今仍无直接资料予以证实,笔者分析周恩来起草的可能性最大。周恩来在政治局内分工负责苏区与红军的工作,从~1931~年~1~月起,周恩来起草中共六届四中全会第一号通告后,周恩来代表中共中央起草了约七份有关涉及全党政治路线、红军与苏区工作,以及致共产国际执委会的指示信和电报。举凡全局性的、最重要的文件均由周恩来参与起草。《决议》体现了周恩来所特有的虽具强烈倾向性、但仍含折衷色彩的思维及行事方式的风格,与周恩来在讨论富田事变的~2~月~20~日政治局会议上的发言精神基本一致。《决议》指出:“(富田事变)实质上毫无疑问的是阶级敌人以及他的斗争机关~AB~团所准备所执行的反革命行动”,“在泽东同志领导下的总前委坚决反对阶级敌人的斗争路线,实质上是正确的。这种坚决与革命敌人斗争的路线在任何时候都应执行”。《决议》又说,“同时过份的估量反革命组织力量及它在群众中的欺骗影响而减弱我们有群众力量有正确路线可以战胜阶级敌人的坚强信心,这也是一种危险”。\footnote{中共中央书记处编:《六大以来——党内秘密文件》(上)(北京:人民出版社,1980~年),页~126;另参见《周恩来年谱》,页~208。}1931~年~2~月以后,中共中央及周恩来在对富田事变定性问题上一直持强硬态度,以任弼时为首的赴苏区的中央代表团忠实地执行了周恩来的方针,而根本不知道随后不久中共中央及周恩来等对富田事变的看法又发生了新的变化。而具体改变肃反政策及纠正毛在肃反问题上的错误,则是在~1931~年底周恩来进入中央苏区后,此时,数千名红军将士和地方干部早已被冤杀。

1931~年~4~月中旬,任弼时、王稼祥、顾作霖率领的中央代表团带着六届四中全会的文件,经闽西到达赣南,和项英领导的苏区中央局会合。在任弼时等未抵达江西苏区之前的~3~月~18~日,项英曾主持召开了中共苏区中央局第一次扩大会议。这次会议本是项英为加强苏区内部团结而开的一次会议,也是巩固其在苏区领导权威的一个重要行动。会议的主要议题是传达刚刚收到的~1930~年~10~月共产国际的来信,具体讨论的问题包括富田事变和“一、三军团过去工作的检阅”等。项英在谈到苏区中央局处理富田事变问题时,进一步重申:“用教育的方法是对的,我们应该清楚认识所有参加富田事变的人不一定个个都是~AB~团取消派,如果否认这一点是错误的。\footnote{中国革命博物馆编:《党史研究资料》,1990~年第~6~期。}”

然而,项英的意见在中央代表团抵达后立即被推翻。任弼时、王稼祥、顾作霖下车伊始,马上召开苏区中央局扩大会议,传达六届四中全会文件和中共中央对富田事变的意见,作为“第一次扩大会议的继续”。4~月~17~日,由任弼时主持在宁都的青塘举行中央局扩大会议,毛泽东、项英等参加了会议。这次会议通过了中央代表团起草的《关于富田事变的决议》,进一步肯定了富田事变的“反革命”性质:“富田事变是~AB~团领导的,以立三路线为旗帜的反革命暴动,更清楚的说,富田事变是~AB~团领导的与立三路线的一部分拥护者所参加的反革命暴动。”

《决议》批评苏区中央局是在三中全会“调和路线”下成立的,指责项英为首的苏区中央局“解决富田事变的路线完全是错误的”:

\begin{quoting}
(项英)根本没有指出富田事变是~AB~团领导的反革命暴动,反而肯定富田事变不是~AB~团的暴动,这完全是模糊了富田事变的反革命性质。又说富田事变是由无原则派别斗争演进而成的,更是大错特错。
\end{quoting}

由于推翻了项英对富田事变的分析和处理意见,中央代表团与毛泽东在思想一致的基础上建立起密切的友好合作关系。1931~年~5~月,重新恢复了红一方面军总前委的建制,仍由毛泽东担任书记。8~月毛泽东担任苏区中央局书记,1931~年~10~月~11~日,苏区中央局致电中共中央,通报由毛泽东正式取代项英,主持苏区中央局:

\begin{quoting}
项英解决富田事变,完全错误,认为是派别斗争,工作能力不够领导。因此丧失信仰,中央局决定以毛泽东代理书记,请中央批准\footnote{《苏区中央局致中共临时中央的电报》(1931~年~10~月~11~日),转引自《任弼时传》,页~212。}。
\end{quoting}

与重新确立毛泽东领导权相同步,是重新逮捕、审讯响应中央局通知回到中央局开会的富田事变的主要领导人。在苏区中央局的直接领导下,成立了以周以栗(1930~年长江局派去红一方面军总前委的代表,随后与毛泽东结成密切关系)为首的审判委员会,“首先把富田事变头子刘敌执行枪决”,然后,依次“公审”谢汉昌、李白芳、金万邦、周冕、丛允中等,也一并处死。事隔三十年后,当年参加“公审”的曾山回忆了这次公审,他说:

\begin{quoting}
在公审中,毫无逼供现象,被审的人完全可以自由自在地谈他自己的意见。他们不承认是反革命组织,而肯定是一个反毛团的组织。\footnote{曾山:《赣西南苏维埃时期革命斗争历史的回忆》(1959~年~6~月~12~日),载陈毅、萧华等:《回忆中央苏区》(南昌:江西人民出版社,1981~年),页~21~—~23。}
\end{quoting}

处决富田事变的主要领导人并不意味“肃~AB~团”运动已告“胜利完成”,相反,它标志着更大的“打~AB~团”风暴的袭来。1931~年~7~月间,原在河西坚持游击战争的红二十军在政委曾炳春和继刘铁超之后任军长的萧大鹏的说服教育下,服从中央局决定,回到苏维埃中心区域的河东于都县,但是等待他们的并不是欢迎和鞭炮,而是大逮捕和大处决。苏区中央局命令解散红二十军,扣押军长萧大鹏、政委曾炳春,直到副排长的全体干部,“士兵被分编到四军、三军团去”。被扣押的红二十军干部,大部分被当作“AB~团取消派”受到~r~处置”(即枪决)。

在地方,“赣西南地区的干部百分之九十以上被打成~AB~团分子”,“有的被错误处置,有的被监禁或停止工作”\footnote{《萧克谈中央苏区初期的肃反运动》,载中国革命博物馆编:《党史研究资料》,1982~年第~5~期。}。继毛泽东在~1930~年发动“打~AB~团”运动后,江西苏区的“肃~AB~团”只因项英的坚决制止才停顿了四个月,又在~1931~年~4~月后如火如荼全面开展起来,并在五、六、七三个月达到最高潮。

为了贯彻落实苏区中央局~4~月~17~日《关于富田事变的决议》中提出的对“AB~团”分子要“软硬兼施,穷追细问”的精神,加紧了对所谓“AB~团”分子的刑讯逼供。“所有~AB~团的破获完全是根据犯人的口供去破获的,……审犯人的技术,全靠刑审”。对犯人普遍采用“软硬兼施”的方法:所谓“软”,“就是用言语骗出犯人口供,……所谓硬的方法,通常着双手吊起,人向悬空,用牛尾竹扫子去打,如仍坚持不供的,则用香火或洋油烧身,甚至有用洋钉将手钉在桌上,用篾片插入手指甲内,在各县的刑法种类无奇不有,有所谓炸刑(万泰),打地雷公,坐轿子,坐飞机(各县皆然),坐快活椅子,虾蟆喝水,猴子牵缰,用枪通条烧红通肛门(胜利县)……等。就胜利(县)说,刑法计有一百廿种之多……”。\footnote{《江西苏区中共省委工作总结报告》(1932~年~5~月),载《中央革命根据地史料选编》上册,页~477~—~78、480。}在运动中,被审人因经不住酷刑乱供乱咬,“AB~团取消派”便越打越多,“凡打~AB~团不毒辣的,都认为与~AB~团有关系,有被扣留的可能”。而肃反机关则捕风捉影,“甚至于公开地说,宁肯杀错一百,不肯放过一个之谬论”,使得“人人自危,噤若寒蝉,因之提拔干部,调动工作,大部分人都是啼啼哭哭,不愿意去……,在打~AB~团最激烈的时候,两人谈话,都可被疑为~AB~团”\footnote{《江西苏区中共省委工作总结报告》(1932~年~5~月),载《中央革命根据地史料选编》上册,页~477~—~78、480。}。

当时在中央苏区的邓小平对此惨剧也有过评论。他说,“我对总前委之反~AB~团的方式亦觉有超越组织的错误,这种方法事实上引起了党的恐怖现象,同志不敢说话”\footnote{邓小平:《七军工作报告》~1931~年~4~月~29~日),载中共中文献研究室、中央档案馆编:《党的文献》~1989~年第~3~期。}。

然而,在“肃~AB~团”的基础上,中央代表团和毛泽东结成的友好合作关系,仅维持了七个月左右,一经解决了毛泽东与项英在“肃~AB~团”问题上的矛盾后,中央代表团和毛之间又逐渐产生了新的矛盾。除了政策分歧外,毛的个人权力与中央代表团权限的不明确,也加剧了双方关系的紧张。至少在法理上,毛泽东是江西苏区党、军队、苏维埃政权的最高领袖,而中央代表团的地位则不甚明确。从中共中央授权讲,任弼时应是江西苏区最高负责人,但毛泽东已就任苏区中央局代理书记一职,因此,无论是从实力基础或是从苏区中央局书记的法理权限讲,毛已是苏区最有权力的人,而中央代表团虽具权威,但只是处在一个监督者的地位。于是,在诸多矛盾的作用下,毛与六届四中全会后中共中央的蜜月终于在~1931~年~11~月初宣告结束,从此双方开始了长达三年零两个月的对抗和冲突。

\section{周恩来与毛泽东在苏区肃反问题上的异同点}

长时期以来,关于苏区肃反“扩大化”的问题,在中共党史编纂学中是一个被严重搞乱的问题。根据传统的解释,造成苏区“肃反”灾祸的所有罪责,皆在王明与王明路线的身上,毛泽东与此毫无关联。不仅如此,毛还被描绘成是与王明“左倾”肃反路线斗争的英雄。然而历史的真实却与此相反,毛是苏区极端的肃反政策与实践的始作俑者。

其实在肃反问题上,毛与中共中央并无原则上的分歧,双方都一致肯定肃反的必要性,但是随着周恩来等较深人地了解到苏区肃反的真相,中共中央开始调整肃反政策,并采取了一系列纠偏的措施。同时,原先对毛个人专权的怀疑也在逐渐增长,中共中央加强了对毛的防范,并果断中止了针对革命阵营内部的大规模的肉体消灭行动。

1931~年~3~月,以任弼时为首的中央代表团启程赴赣后,中共中央在继续强调富田事变“反革命性质”的同时,开始提及防止肃反“过火化”的问题。1931~年~7~月下旬,随任弼时等同赴江西苏区的中央巡视员欧阳钦返回上海,他完全接受了苏区存在大量“AB~团”的论断,并将此情况向周恩来作了汇报。1931~年~8~月~30~日,周恩来在听取欧阳钦的汇报后,起草了《中共中央致苏区中央局并红军总前委指示信》,在肯定中央苏区“反~AB~团斗争是绝对正确的而必要”的同时,批评了在反“AB~团”斗争中存在的“简单化”和“扩大化”的错误,强调:“不是每一个地主残余或富农分子便一定是~AB~团”,“不是每一个党的错误路线的执行者和拥护者,每一个落后的农民,每一个犯有错误倾向或行动的党员或群众便一定是~AB~团”\footnote{《周恩来年谱》,页~212。}。

周恩来起草的这封信在延安整风运动期间被略去周的名字,作为王明路线的代表作受到严厉批判,直至八十年代中期仍被指责。\footnote{参见孔永松、林太乙、戴金生:《中央革命根据地史要》(南昌:江西人民出版社,1985~年),页~258~—~59。}

周恩来的这封信之所以使毛泽东不能忘怀,盖因为这封信对中央苏区第一次党代表大会(即赣南会议)起到重要的指导作用,任弼时等依据这封指示信中有关纠正“富农路线”的精神开始了对毛的不指名批评。

任弼时作为中央代表团团长,在赣南会议上传达了周恩来指示信的精神,但因任弼时已深深地卷人到“肃~AB~团”运动,因此赣南会议把重点放在检讨土地政策方面,而没有深入检讨肃反工作。虽然在赣南会议的《政治决议案》和~1931~年~12~月~5~日苏区中央局致各级党组织的指示信中,都传达了中共中央对苏区肃反“扩大化”的批评,并且提出了“坚决的反对极有害的极错误的‘肃反中心论’”的口号,但中央苏区的乱打乱杀并没有得到有效的遏止。

中央苏区大规模的“肃~AB~团”运动是在~1931~年底周恩来进入江西苏区后才真正得以停止。由于苏区肃反与中共中央的“反右倾”路线及与苏区领导层内部的斗争有着千丝万缕的联系。周恩来不得不采取较为缜密的措施,一方面避免与毛泽东发生直接的对抗;另一方面大大加强了苏区中央局的权威,才将中央苏区从肃反大恐怖中解脱出来。

周恩来具体了解肃反所造成的惨祸是他在~1931~年~12~月中旬从闽西进人到赣南的途中。此时闽西正在轰轰烈烈开展一场与赣南“肃~AB~团”平行的肃反运动——“肃社民党”,这场斗争的残酷性及对闽西苏区造成的巨大破坏,促使周恩来采取紧急措施,对苏区肃反进行急刹车。

发生在闽西的“肃社会民主党”事件起始于~1931~年初,到了~3~月,迅速走向高潮,在运动规模、肃反手段及残酷程度方面,都与赣南的“肃~AB~团”难分伯仲。在近一年的时间里,大批红军干部、地方领导人及普通士兵、群众被扣之以“社党分子”的罪名被镇压,遇害者总数达~6352~人。\footnote{《闽西“肃清社会民主党”历史冤案已平反昭雪》,载中共中央党史研究室编《党史通讯》~1986~年第~5~期。}由此引发了闽西苏维埃政府财政部长傅伯翠脱离共产党,拥兵自守的事件,并造成与富田事变类同的~1931~年~5~月~27~日的“坑口事变”。经这次肃反,闽西苏区元气大伤,党员人数由原先的八千人,减至五千人。\footnote{《中共苏区中央局致闽粤赣省委并转省代表大会的信》(1932~年~2~月~19~日),转引自蒋伯英:《闽西革命根据地史》(福州:福建人民出版社,1988~年),页~196。}

对于闽西的“肃社民党”事件,中共中央、中央军委书记周恩来、中央代表邓发、闽西地方领导人张鼎丞、和中央代表团成员任弼时,各有其不同的责任。

受到共产国际“反右倾”路线影响,中共中央对待闽西“肃社民党”的态度和对“肃~AB~团”完全一样:先期全力支持;到了~1931~年~8~月后,在继续肯定“肃社民党”的同时,重点转向防止肃反的“过火化”、“简单化”。

1931~年~4~月~4~日,经周恩来修改的《中央对福建目前工作决议》发出,要求福建省委“依据国际路线和四中全会的决议在实际工作中进行全部的彻底的转变”。\footnote{《周恩来年谱》,页~209。}中共中央同日根据闽西给中央的报告,发出致闽粤赣特委信,对肃反工作作了如下的指示:“普遍的白色恐怖积极的打入到党的组织内和红军中来从事破坏(闽西的所谓社会民主党、江西的~AB~团以及其它地方的改组派等),从蒋介石到傅伯翠都有整个的联系和计划的”,要求各级党组织应对他们采取“最严厉的手段来镇压”。\footnote{《中央给闽粤赣特委的信——闽粤赣目前形势和任务》(1931~年~4~月~4~日),转引自蒋伯英:《闽西革命根据地史》,页~193。}

中共中央的~4~月~4~日来信,对闽西的肃反起到火上浇油的恶劣作用。这封信究竟是谁起草的,大陆方面至今仍未公布。根据有关线索分析,周恩来起草的可能性最大。如前所述,周恩来在政治局内分管苏区和军事工作,凡涉及苏区及军事方面问题的中央指示信,一般皆由周恩来草拟,就在发出给闽粤赣特委信的当天,周恩来修改的中央对福建工作指示信也一并发出。同日,周恩来还出席了讨论湘鄂赣边苏区问题的政治局常委会。作为中央负责人,周恩来对闽西肃反的“过火化”、“扩大化”负有一定的领导责任。

1931~年夏,周恩来对苏区肃反问题的认识发生明显变化,在批评“肃~AB~团”问题上的“简单化”、“扩大化”的同时,周恩来也对闽西的“肃社民党”中暴露出的问题提出了较为直接的批评。

1931~年~9~月中旬前后,由周恩来于~8~月~29~日起草的中央致《闽粤赣苏区省委的信》送达闽西。这封信既肯定了“社民党在闽西和其它地方是存在的”,又对闽西肃反提出了一系列的疑问:

\begin{quoting}
(社党分子)既然能广泛的深入我们的党团和红军中去,经迭次破获和逮捕以后,仍然时常发现他们在我们组织中活动?为什么一部分被欺骗的群众抱着观望的态度不能自动的自首,甚至害怕加入共产党?这些问题非常值得我们注意,从你们一系列文件中找不到这些问题的最圆满的回答\footnote{《中央致闽粤赣苏区省委的信》(1931~年~8~月~29~日)载中央档案馆编:《中共中央文件选集》(1931)第~7~册,页~349。}。
\end{quoting}

周恩来在肃反问题上的这种新态度,为他在抵达苏区后对恶性肃反进行紧急纠偏,提供了必要的思想基础。然而,在当时具有像周恩来这样有灵活眼光的苏区领导人少之又少,闽粤赣党的最高负责人邓发,就缺少周恩来的学养和眼光,他在主持闽西肃反时的狂热态度,直接酿成了闽西肃反惨祸。

1930~年~12~月,受六届三中全会后的中共中央的派遣,新增选的中央委员邓发到达闽西的龙岩,担任新成立的中共闽粤赣特委书记。从隶属关系上,邓发应直接受苏区中央局领导,但由于当时闽西与赣南尚未打通(1931~年~9~月,闽西才与赣南苏区打通,连成一片),邓发实际上享有工作中的高度自主权。邓发抵闽西后,即和当地干部邓子恢、张鼎丞、林一株、罗寿春等组成了新的党与苏维埃领导机构;全面负责起闽西苏区的工作。

邓发和项英都是在中共六届三中全会后,为加强苏区工作,被中共中央派往苏区的。项英在前往赣西南的途中路经龙岩,曾与先期抵达的邓发见面。项英抵达赣西南后,立即集中精力处理富田事变的后遗问题,未曾过问闽西的工作。

邓发与项英是中共党内少数出身无产阶级的领导人,在中共早期历史上,两人都曾声名显赫。邓发更因在省港大罢工期间担任过工人纠察队队长,对“群众专政”的一套较为熟悉。邓发进入闽西苏区后,一时颇看不惯在农村根据地中盛行的“流氓现象”和“流氓作风”,当邓发看到苏维埃文化部里,竟有干部抱着两个女人睡觉,就凭直觉做出判断,闽西党和苏维埃机关里,已混人大量的反革命分子。而当~1931~年~1~月初红十二军(由罗炳辉任军长,谭震林任政委)部分指战员在大会上呼错口号的事件发生后(1931~年~1~月初,红十二军召开纪念李卜克内西、卢森堡、列宁大会,有十几名红军指战员由于不了解第二国际与第三国际的区别,在会上呼喊“拥护第二国际”、“社会民主党万岁”),邓发便毫不犹豫地发动了“肃社民党”运动。同是六届三中全会派往苏区的中央代表,邓发缺乏项英所具有的对复杂事物进行缜密分析和慎重判断的能力,邓发的激烈的性格和狂热的革命气质导致闽西肃反的规模不断扩大。

由邓发主导的闽西肃反具有革命绞肉机的全部特征,指称社党分子,全凭肉刑和逼供;肃反的唯一手段就是处决;恐怖机器一经开动,就产生了自我驱动的内在动力,使其疯狂运转,不断依次进入更高阶段,结果是纵火者也与之俱焚——杀人者终被杀!

1931~年~3~月~2~日,由处决原红十二军第~100~团政委林梅汀等十七人而拉开了闽西苏区肃反大恐怖的帷幕,肃反狂潮迅速波及红军、党和苏维埃各级机构,以及共青团、少先队、儿童团系统,结果导致地方红军中大部分排以上干部,闽西苏维埃政府三十五名执行委员和候补委员中的~50\%,\footnote{参见蒋伯英:《闽西革命根据地史》,页~194。}段奋夫等一批闽西农民暴动的领导者,和永定、龙岩、杭武等县区的负责人都尽行被消灭。被害者中大多为二十几岁的青年,闽西肃反第一个牺牲者林梅汀被杀时,年仅二十四岁。在被害者中,也有不少少先队、儿童团员,最小的只有十六岁。\footnote{杭武县苏肃反委员会:《革命法庭》(1931~年~6~月~1~日),转引自蒋伯英:《闽西革命根据地史》,页~194。}许多五花八门的罪名,诸如参加了“社民党”的“十毫子运动”、“食烟大同盟”、“姑娘姐妹团”、“找爱团”、“膳食委员会”,都成为被处决的理由。

在闽西肃反的牺牲者中,地富家庭出身的党员干部占有相当的比例,这也反映了苏区肃反运动的一个带有共性的特征,即在所有清洗运动中地富出身的党员干部都是首当其冲的整肃对象。在~1931~年~3~月~2~日召开的闽西第一次公审处决大会上,闽西肃反委员会主席林一株明确宣布惩处“社党分子”的三项原则,其中最重要的一条,即出身不好者处以死刑,其依据是,“地主富农子弟,在斗争中必然会背叛革命”。\footnote{《共青团闽西特委对肃清社会民主党工作的决议》(1931~年~4~月~6~日),转引自蒋伯英:《闽西革命根据地史》,页~191、193。}

由“肃社民党”造成的空前“红色恐怖”使闽西苏区的党员、干部和普通群众陷入一片惊恐之中,许多干部和战士被迫逃亡,有的甚至飘洋过海以求避难,更多的人则纷纷逃往由傅伯翠控制的上杭古蛟区。

傅伯翠是蛟洋农民暴动的领导人,曾任红四军四纵队司令员和闽西苏维埃政府财政部长。傅因在其家乡古蛟区实行“共家制度”受到闽西党组织的批评,其后,又因拒绝出席党的会议和拒不服从工作调动,在~1930~年~10~月,被党组织指称为“第三党观点”而受到留党察看的处分。邓发担任中共闽粤赣特委书记后,在~1931~年~2~月宣布开除傅伯翠的党籍,并派红军攻打傅的家乡古蛟区,逼使傅伯翠走上拥兵反抗的道路。

1931~年~3~月~6~日,闽西苏维埃政府发布第二十三号通告,宣布傅伯翠为闽西“社民党”首领,古蛟区为“社会民主党巢穴”。在大恐怖中,傅伯翠控制的古蛟区成为大批红军干部战士逃避捕杀的避难所\footnote{傅伯翠脱离中共后,曾接受国民党委任的(上)杭、(龙)岩、连(城)边界保安总队队长职务。1934~年~10~月,中央红军撤出苏区后,傅伯翠曾接济过在赣、粤、闽坚持游击战争的共产党游击武装。1949~年~5~月,傅伯翠率所部三千余人归顺中共。1985~年~5~月~14~日,中共福建省委发出通知,为傅伯翠平反,宣布其为“同志”,推翻傅身上的“社会民主党领袖”的不实之词。}。

1931~年春夏之交,闽西大规模的红色恐怖已发展到动摇共产党社会基础的危险地步——在闽西政府所辖之下,风声鹤唳,人人自危,根据地的社会秩序受到严重破坏。在傅伯翠拥兵反抗之后,1931~年~5~月~27~日,又爆发了在中共历史上鲜为人知的“坑口兵变”。

“坑口兵变”的发生与被镇压,几乎与赣西南的“富田事变”如出一辙。

在闽西大清洗的高潮阶段,闽西杭武县第三区(现属上杭县溪口乡,太拔乡)区委书记何登南、县武装第三大队政委陈锦玉等二百人被控以“社党分子”的罪名,被拘押在坑口和白砂(县苏维埃政府所在地)。5~月~27~日,县武装第三大队大队长李真,副政委张纯铭,副大队长丘子庭等率众扣押了正在此巡视的闽西苏维埃政府秘书长罗寿春,迫其书写手令释放被扣人员。当晚,李真等率领三大队包围区苏维埃政府,放出被捕人员。同时又派出一部分人员前往白砂,以罗寿春的手令,将被关押的第三区人员带回释放。

以邓发为书记的中共闽粤赣省委(1931~年~5~月,原特委易名为省委)得知“坑口事变”消息,立即认定属于“反革命暴动性质”,随即抽调新红十二军进攻杭武第三区,至~5~月~29~日,除少数人逃亡外,第三大队的大部分人员被缴械逮捕,两天前刚被释放的人员又再次被捕。同日,闽粤赣省委作出决议,指示:“对于已经被捕的社党,应多方审讯以破获其整个组织,同时要很快地处决”\footnote{《中共闽粤赣省委关于杭武第三区事变决议》(1931~年~5~月~29~日)转引自蒋伯英:《闽西革命根据地史》,页~197。}。于是,李真、何登南、丘子庭及第三大队绝大多数被捕干部、战士尽被处决。

5~月~29~日的镇压虽然极为严厉,但是并没有完全压下闽西苏区军民对肃反的极度愤怒。6~月~1~日,杭武县第二区部分干部与地方武装又发动反抗,在此前后,永定的溪南和虎岗,也发生类似自发的反抗行动,但全部遭到镇压\footnote{参见蒋伯英:《闽西革命根据地史》,页~197。}。

邓发作为中共闽西苏区党的最高领导人,对于所发生的这一切极端行为,应负有直接的、第一位的责任。在闽西肃反问题上,担任闽西苏维埃政府主席的张鼎丞与充满肃反狂热的邓发相比,其态度要相对温和一些,但是他最终还是屈从于邓发的意志。

张鼎丞是闽西党和苏维埃政权的主要创始者,极为熟悉闽西革命历史和干部状况,是闽西地方干部的代表人物。邓发抵达闽西后,张鼎丞作为邓发的副手,有责任向邓发介绍他所了解的闽西干部的真实情况,并在肃反袭来时尽全力保护干部。但是迄今为止,这类事例还很少披露。相反,所能发现的,尽是以闽西苏维埃政府主席张鼎丞的名义所发布的肃反通告。在这类文告中,尤以张鼎丞在~1931~年~2~月~21~日发布的“裁字”第一号、第二号给闽西造成的危害最为严重。

在发动“肃反”之初,张鼎丞曾在文告中规定,社党主要负责人,应扣留严办,一般成员在交待其行为后,令其自首,处以禁闭和警告\footnote{《闽西苏维埃政府通告第二十号》(裁字第二号)(1931~年~2~月~21~日)转引自蒋伯英:《闽西革命根据地史》,页~188。}。闽西政府还曾公布《反动政治犯自首条例》,明文规定凡在半个月内自首者,不论其职务高低,概行免去处罚。然而,这些规定并没有真正实行,随着处决权迅速下放,这类政策条令形同一纸废文。

1931~年~3~月~18~日,闽西政府发出第二十五号通告,修改了处决人犯需报请闽西政府批准的规定。明确宣布,“如有迫不及待要先处决的”,可先行刑,再“补报到本政府追认”\footnote{《闽西苏维埃政府通告第二十五号(裁字第四号)》(1931~年~3~月~18~日),转引自蒋伯英:《闽西革命根据地史》,页~191。}。这个新规定,造成大处决迅速蔓延,苏区各级组织甚至包括医院,都有权随意逮捕、处决“社党分子”。而在当时的狂热气氛下,指称“社党分子”全凭肉刑和逼供,结果被捕者屈打成招,胡乱招供,形成恐怖的“瓜蔓抄”,甚至在少先队、儿童团也多次破获“社党”。

自~1931~年~3~月处决权下放后,在近一年的时间里,肃反成了闽西一切工作的中心。闽西政府要求各地在两个月内肃清“社党”。在上级号召的推动下,各级组织均以捕人愈多,处决愈快为革命最坚决的标准,一些对运动稍有怀疑的干部,迅即被草率处决。永定县委负责人谢献球、卢肇西、曾牧村等因“对特委将社会民主党名单要他拘捕,完全犹疑不坚决的态度”,而被冠之以“社党”罪名处死。\footnote{参见《中共闽粤赣特委常委第一次扩大会议决议》(1931~年~2~月~27~日),载《中央革命根据地史料选编》,中册,页~286。}为自保性命,各机关实际上展开了一场杀“社党”的大竞赛。一旦开了杀戒,杀一人与杀一百人都一样,肃反干部的疯狂与内心恐惧交织在一起,只有通过杀更多的“社党”才能舒缓心理的失衡。于是,“肃反”野火越烧越旺,一发不可收拾,一直到周恩来抵达闽西后才被扑灭。

张鼎丞对闽西肃反惨祸应负的另一份责任还在于他对主持肃反大计的林一株没有发挥应有的约束力。林一株为闽西地方干部,是闽粤赣特委的主要负责人之一,在闽西肃反中,担任权势极大的闽西政府肃反裁判部部长,是一个令无数人闻之色变的人物。有论著称,林一株“在处理一系列重大案件时,完全背着闽西苏维埃政府主席张鼎丞”,\footnote{蒋伯英:《闽西革命根据地史》,页~194。}这种说法有其一定的真实性,因为林一株直接听命于邓发,且有擅权之恶名。但是作为闽西党的元老,张鼎丞应对本地干部出身的林一株具有一定的影响力和约束力。将闽西肃反惨祸的全部责任推到邓发和林一株身上,似乎张鼎丞与此毫无关系,显然是说不通的。因为在闽西肃反中,张鼎丞始终处在安全和有权的地位。而在肃反高潮中,许多受害者都曾寄希望于张鼎丞能对林一株发挥某种约束力。

在收到周恩来起草的批评闽西肃反扩大化的~8~月~29~日来信后,闽西最高领导对林一株的约束力立时就显现出来。邓发等把肃反干将林一株等抛出来,送上断头台,双手沾满无辜者鲜血的林一株在肃反传送带上终于走到了最后一站。9~月~29~日闽西苏维埃政府发出第九十七号通告,宣布林一株是闽西“社党”特委书记,同时指称罗寿春(闽西政府秘书长)、张丹川(闽西政府文化部长)、熊炳华(闽西政府劳动监察部长)等八人为闽西“社党”核心人员,分别予以处死。

张鼎丞在闽西肃反问题上所持的立场和态度,应是受到赣西南“肃~AB~团”的严重影响。这个时期,闽西与赣西南的交通虽未打通。但两地一直有着密切的联系,张鼎丞与毛泽东早在~1929~年就相识。率先在闽西打“社党”的闽西地方部队红十二军的主要领导人谭震林、罗炳辉都是毛泽东领导的红四军派来支持闽西的。在赣西南发起“肃~AB~团”,尤其在富田事变爆发后,张鼎丞的思想受到波动,继而“头脑发热”应是不奇怪的。

对于闽西肃反惨祸,任弼时也有其间接的责任。1931~年~3~月~15~日,正在闽西肃反走向高潮之际,任弼时率领的中央代表团在前往赣西南途中路经永定县的虎岗,任弼时向邓发等传达了六届四中全会精神,要求闽西“集中火力反右倾”。闽西本来就左祸严重,如今又再“反右倾”,只能使左祸连天。任弼时在对待邓发与项英的态度上也完全不同,任弼时在抵达赣西南后,不满项英对肃反的消极态度,下令免去项英的苏区中央局书记一职。而邓发则继续担任闽西最高负责人的职务,这助长了本来就够左的邓发,使其在极左的道路上越走越远。

毛泽东与闽西肃反有无关联,这仍是一个有待研究的问题,但从时间上判断,项英进入赣西南后,毛泽东被免去苏区中央局书记一职,毛并很快领导红一方面军与进攻苏区的国民党军作战,毛似无机会过问闽西肃反一事。

在另一方面,闽西肃反又是一件发生在赣西南眼皮底下、震动苏区全局的事件,毛绝无可能不知道。1931~年~4~月后,中央代表团支持毛,批判项英,赣西南的“肃~AB~团”运动再掀高潮,而此时,闽西“打社党”运动正方兴未艾,此恰可证明开展“打~AB~团”的合理性。毛没有任何理由反对这场与“肃~AB~团”平行展开的“肃社党”运动。

毛泽东深深卷人赣西南的“肃~AB~团”,没有或较少涉人闽西“打社民党”的事件,这些都决定了毛在对待这两个事件的态度上,有着明显的差别。

1931~年~9、10~月后,周恩来起草的、包含有对“打~AB~团”过火化批评内容的中共中央~8~月~30~日指示信已经传到江西苏区,毛开始受到以任弼时为首的中央代表团的冷遇,毛逐渐调正自己的姿态,以摆脱不必要的干系。1931~年~11~月,张鼎丞在瑞金参加第一次全国工农兵代表大会期间,向毛汇报闽西肃反工作,毛指示张鼎丞,必须立即纠正肃反扩大化的错误,并拨款五千银元,作为善后救济费。

毛泽东在遵义会议后,逐渐削夺与周恩来关系较为密切的邓发的权力。在延安整风运动期间,又利用闽西肃反“扩大化”一案,进一步打击邓发。对于自己未曾直接卷人的闽西“肃社党”事件,毛采取的方法是,肯定肃反之必要性,将其问题定性为“扩大化”。

邓发作为此案的直接当事人,在~1945~年就曾明确表示,“今天来看,不仅当时全国没有什么社会民主党,连傅伯翠本人是不是也难说\footnote{《邓发同志在闽西党史座谈会上的发言记录》(1945~年~2~月~23~日)转引自蒋伯英:《闽西革命根据地史》页~189。}。”然而毛泽东却不愿直接承认闽西“肃社党”是一件冤案。

在~1945~年~5~月~31~日中共七大会议上,毛泽东在讲话中提到:“肃反,走了极痛苦的道路。反革命应当反对,党未成熟时,在这个问题上走了弯路,犯了错误。”\footnote{参见中共中央党史研究室一室:《中国共产党历史(上卷)若干问题说明》,(北京:中共党史出版社,1991),页~121。}在这里,毛泽东虽然提到了肃反的痛苦性,但没有正面涉及为“打~AB~团”和“肃社党”冤死者平反的问题,尤其回避了他自己的个人责任问题。即使这样,毛泽东的这段话也长期未予公布。

毛泽东长期不为“肃社党”案平反,其根本原因乃是赣西南“打~AB~团”与闽西“打社党”有极大的关联,如果为“肃社党”全面平反,势必牵扯到为“打~AB~团”翻案,从而有损自己的声誉。

1954~年,中共福建党组织根据中共中央有关处理历史遗留问题的意见,对在闽西肃反中被错杀的~3728~人予以平反昭雪,并追认为烈士\footnote{《闽西“肃清社会民主党”历史冤案已平反昭雪》,载中共中央党史研究室编《党史通讯》~1986~年第~5~期。}。但在根本问题上,即闽西苏区是否有“社党”,“肃社党”是否是冤案,则全部维持~1931~年的结论。直到~1985~年,在毛泽东去世九年后,这个问题才最终得以解决。中共福建省委在大量调查的基础上得出结论:闽西根本没有“社会民主党”,闽西“肃社党”运动不是什么“扩大化”问题,而纯属历史冤案。1985~年,原被定为闽西“社党首领”的傅伯翠也得到平反。

赣西南的“肃~AB~团”案也是在八十年代隐去了毛泽东历史责任后,才得到澄清。

勿庸置疑,周恩来对于苏区肃反造成严重后果方面,应负有一定的领导责任,但是在肃反问题上,周与毛的态度有着显著的差别。

种种迹象表明,周恩来是从推行共产国际“反右倾”的理念出发而支持苏区肃反,而无任何个人的动机;毛的行为则很难摆脱利用肃反剪灭异己的嫌疑。

周恩来正是因为从理念出发,当发生了赣西南“肃~AB~团”和富田事变后,在未深入了解实情的情况下,就匆匆发出中央指示信,客观上助长了苏区内已经蔓延的左祸。然而,毛则是极端的肃反运动的始作俑者,是毛发动在前,周支持在后。

周恩来在~1931~年~8~月就已把重点转到纠正肃反扩大化方面,在进入中央苏区后,用了几乎三个月的时间,才使疯狂运转的肃反机器停了下来,毛则鲜有类似的表现。正是因为苏区肃反问题牵涉面广,涉及到领导人的过失责任等敏感问题,周恩来小心翼翼,既要显出纠偏的决心,又随时作出妥协,尽最大努力来维持党的团结。

1931~年~12~月~18~日,周恩来在目睹了闽西肃反惨祸、从永定赴长汀的途中,致信中共中央,要求中央立即作一有力决议,制止闽西的恶性肃反。周在信中说,“我入苏区虽只三日,但沿途所经,见到闽西解决社党所得恶果非常严重”,“目前问题已很严重,转变非常困难”。周表示,自己决心“与此严重问题斗争”\footnote{《伍豪自中区来信》(1931~年~12~月~18~日),载《周恩来书信选集》(北京:中央文献出版社,1988~年),页~76~—~77。}。

1932~年~1~月~7~日,周主持就任苏区中央局书记后的第一次中央局会议,会议通过《苏区中央局关于苏区肃反问题工作决议案》,严厉批评“总前委领导时期”在“肃~AB~团”问题上滥用刑法、“以杀人为儿戏”的严重错误,强调纠正“肃反工作中的路线错误”\footnote{中央档案馆编:《中共中央文件选集》(1932~—~1933)第~8~册,页~18。}

在收到周恩来的信后,上海中央于~1932~年~1~月~21~日就肃反问题给闽粤赣省委发出一封与周意见一致的指示信,责令邓发领导的省委必须深刻检查“过去在肃反的问题上所犯的不可宽恕的”错误。苏区中央局还在~1932~年~2~月~29~日致信闽粤赣省委和即将召开的省党代表大会,再次批评闽西“在肃反工作中的严重错误”。周恩来并派任弼时代表中央局前往长汀指导在~3~月初召开的闽粤赣省委第二次代表大会,又派李克农具体负责纠正赣西南、闽西及红一方面军的肃反冤案。在周恩来的艰苦努力下,中央苏区的大规模肃反在~1932~年~3~月才告基本停止。

周恩来虽在肃反紧急刹车方面措施有力,但在处理有关责任人时,态度却极为谨慎。1932~年春,对闽西肃反惨祸负有直接责任的邓发被调至瑞金,担任权力极大的中华苏维埃共和国政治保卫局局长一职。\footnote{1931~年~11~月,中华苏维埃共和国在江西瑞金正式成立,邓发被任命为国家政治保卫局局长,但正式组建机关是在周恩来抵达江西瑞金之后,时间约在~1932~年~1~至~2~月间。国家政治保卫局的工作制度是在周恩来的指导下建立的。}任弼时则在周赴任后,出任苏区中央局副书记,在党内的地位仅次于周恩来。邓发与任弼时的过失,也许被视为是“好心办坏事”,因为对于这两人而言,都不存在利用肃反剪除异己的不良动机,因此与共产党的党道德和党伦理并无冲突。在大敌当前的形势下,不宜开展过份的党内斗争,这或许是周恩来对任命邓发、任弼时新职的考虑。

至于毛泽东,问题则比较复杂。周恩来小心翼翼,不去触及毛泽东,而是将苏区中央局、闽西省委(前闽粤赣省委)和红一方面军总前委放在一起进行批评。在苏区中央局会议上,周严厉批评了上述单位在肃反问题上所犯下的严重错误。1932~年~5~月,国家政治保卫局将毛泽东的老对头、原赣西南党和地方红军负责人李文林处死。1932~年~1~月~25~日,周恩来主持召开苏区中央局会议,作出《关于处罚李韶九同志过去错误的决议》,周知道李韶九是毛泽东的老部下,是造成赣西南肃反惨祸的祸首之一,\footnote{1933~年夏之前,李韶九曾被任命为汀州连城分区司令员,之后,李韶九被调往赣东北,担任职务及最后结局不详。}但只给予李韶九留党察看六个月的极温和的处分。周恩来主持的所有这类纠偏会议和主持制定的文件,均未直接批评毛,对曾山、陈正人等基本上也没有触及。1932~年初,因原先担任江西省委书记的陈正人患病,苏区中央局任命李富春接任,曾山继续担任江西省苏维埃政府主席,毛泽东的老部下周兴虽“有助长李韶九错误的事实”,也只是由江西省委给其“留党察看”的处分,\footnote{《江西苏区中共省委工作总结报告》(1932~年~5~月),载《中央革命根据地史料选编》上册,页~481。}张鼎丞也在~1932~年~3~月后,继续留任福建省苏维埃政府主席一职。

尽管如此,周恩来实际上对于毛泽东已有了新的认识,而这种认识以后又成为迁移至瑞金的中共中央一班核心人物彼此心照不宣的看法。1932~年春,周恩来派自己的老部下、前中共中央特科成员李克农、钱壮飞、胡底、李一氓到国家政治保卫局工作,李克农、李一氓先后都担任过政治保卫局执行部部长,李克农、钱壮飞还先后任红一方面军保卫局局长。

在被称之为王明路线占统治地位的~1932~—~1934~这几年,由国家政治保卫局承担中央苏区内部的肃反事务,不再由各机关、单位和军队自己大搞肃反。国家政治保卫局在~1932~年~5~月~30~日处决了李文林、曾炳春、王怀等一批“AB~团”首犯,以后又杀了二百多名“反革命分子”,\footnote{《红色中华》,1932~年~11~月~7~日。}但总的“工作情况比较平稳”。\footnote{李一氓:《模糊的荧屏——李一氓回忆录》(北京:人民出版社,1992~年),页~159。}中央苏区再没有开展过像“肃~AB~团”、“肃社会民主党”一类大规模的肃反运动。

尽管中央苏区大规模的“肃反”在~1932~年后已经基本停止,但是在苏区中央局机关内部仍然时断时续地开展“反右倾”、“反托派”的斗争,1932~年~6~月后发生在瑞金的“工农剧社事件”即是一起典型的事例。

1932~年~6~月,瑞金红军学校内的一些党员知识分子发起组织了“工农剧社”,因在剧社章程中有“在总的社会主义革命任务下,配合红军目前的伟大胜利”几句话,很快被苏区中央局指控为进行“托派”活动。8~月~13~日,邓颖超代表中央局主持反托派斗争大会,判定“工农剧社”偷运托洛茨基的“私货”,因为所谓“社会主义”云云,就是否认了中国革命现阶段是资产阶级民权革命,全是托派的说法。邓颖超还说,剧社章程没提农民问题,这也是从托陈取消派的观念出发的等等。在这天的斗争会上,对工农剧社党团会干事张爱萍等人开展了严厉的批斗。邓颖超指责张爱萍“在反对反革命政治派别托陈取消派的斗争中,他表现消沉不积极”,“非但未在党的领导下,去深刻认识自己的错误的严重性,去深刻揭发并改正自己的错误。相反的,在会后不久,……对中央局将此事通知红军学校政治部表示不满,……企图转移斗争的中心”。邓颖超还指控张爱萍与“有重大嫌疑的人们(危拱之、王观澜)接近”,并说这是“他对托洛斯基主义犯了自由主义错误的根源……。”\footnote{邓颖超:《火力向着反革命的托洛斯基主义与对它的腐朽的自由主义》,载中共苏区中央局组织部编:《党的建设》,第~5~期,1932~年~10~月~25~日,转引自曹伯一:《江西苏维埃之建立及其崩溃(1931~—~1934)(台北:国立政治大学东》亚研究所,1969~年),页~438~—~41。}在这次批斗会后,少共中央局于~8~月~17~日给张爱萍书面严重警告处分,12~月,苏区中央局宣布开除危拱之\footnote{《中央局关于开除郭化玉危拱之罗欣然等党籍与处分左权张爱萍同志的决议案》(1932~年~12~月~11~日),载中共苏区中央局组织部编:《党的建设》,第~6~期,1932~年~12~月~30~日,转引自曹伯一:《江西苏维埃之建立及其崩溃(1931~—~1934)》,页~442。}等人的党籍,给左权、张爱萍留党察看一年的处分。所幸张爱萍、王观澜、危拱之等人犯事的时候已是~1932~年,如果早一年,他们一定会因此而命丧黄泉。1932~年后,中央苏区的肃反已用较缓和的方式进行,但是在鄂豫皖、湘鄂西,类似“肃~AB~团”、“肃社会民主党”的大肃反运动仍继续进行,造成了极为严重的后果。

由张国焘领导的鄂豫皖苏区,和由夏曦、贺龙领导的湘鄂西苏区,是两个具有高度自主性的战略根据地,“天高皇帝远”,中共中央对两地的领导必须通过张国焘、夏曦来实现。而此时的中央并没有威权十足、足以号令四方的“皇帝”。加之张国焘也是具有某种枭雄气质的领导人,一旦“肃反”成为其消灭异己、树立自己权威的有利工具,他自不会轻易放弃使用。夏曦原是湘省一激进青年学生,全凭杀人树威,才建立起他在湘鄂西的地位,当夏曦尝到肃反的甜头,已犹如鸦片上瘾,非一般手段就可以让其自行终止。

鄂豫皖(继而在川陕根据地)、湘鄂西等地的恶性肃反不能得到有效制止的最终原因,是中共中央在这个问题上的妥协立场。中共中央是在肯定肃反的前题下,提出“扩大化”及“纠偏”问题的,因而使张国焘等有机可乘。1932~年~10~月后,张国焘率部突出国民党军的包围,从鄂豫皖根据地向川北作大规模战略转移,中央对张国焘更是鞭长莫及。到了~1933~年,中央苏区的军事形势也频频告急,打破国民党军的围剿,成为博古、周恩来等考虑的第一位问题,从而再难关注到对张国焘部及湘鄂西肃反的“纠偏”。

远在莫斯科的王明对苏区肃反问题的复杂性和微妙性一无所知,却从阶级斗争的理念出发,大谈苏区“肃~AB~团”斗争所取得的“伟大胜利”。王明甚至认为~1932~年后中央肃反已不如过去那般坚决有力了,批评苏区中央局“对于反对反革命组织及其活动底斗争和警惕性有减弱的倾向”\footnote{王明:《革命,战争和武装干涉与中国共产党的任务》(1933~年~12~月),载《王明言论选辑》(北京:人民出版社,1982~年),页~364。}。如果是在~1930~—~1931~年,王明的这番话肯定会引起毛泽东的好感,只是现在形势已大变。1932~年后,毛泽东不再位居中央苏区核心决策层,他已不需要为中央的政策承担责任,而党内对肃反不满的情绪却依然存在。几年后,毛泽东把这股情绪导引至王明、博古、周恩来,以及邓发、张国焘、夏曦的身上,俨然自己一身清白。当毛将党和军队大权牢牢掌握后,知情人纷纷三缄其口,最终,肃反问题反而成为毛泽东打击王明等的一根大棒。

\section{在土地政策方面的分歧}

在毛泽东与中共中央的诸多分歧中,有关土地政策方面的分歧占有突出的地位,在~1931~年~11~月~1~至~5~日于瑞金召开的苏区党的第一次代表大会上,以任弼时为首的中央代表团不指名的批评了毛的有关土地政策方面的主张,从而结束了中央代表团与毛长达半年的密切合作。

毛泽东关于土地问题的思想与实践在~1927~—~1931~年几经周折,发生过多次变化,其间毛曾一度制定过比中共中央的土地政策还要激进的土地分配方案,又在~1930~年后适时作出调整,转而采取较为务实的现实主义方针。但是,毛有关土地政策的思想演变过程十分复杂,即便在毛的思想发生转变之后,他在对待富农问题上的态度仍然十分激进,极左的色彩与共产国际不相上下(在中共党史编纂学中,只强调毛泽东与共产国际、中共中央土地政策的冲突和抵制的一面。完全不提毛在土地与富农问题上曾持有的极左立场)。

标志毛泽东土地政策从较左的立场,转向较具现实主义立场的文件,是~1930~年~2~月~7~日由毛亲自制定的《二七土地法》。

《二七土地法》是一部具有求实风格的革命土地法。毛修改了在此前制定的《井冈山土地法》(1928~年~12~月)、《兴国土地法》(1929~年~4~月)中的过左内容,明确宣布所有农民皆可分得土地,地主及其家属也可得到土地。后一条规定,是对共产国际有关主张的重大突破。

问题的复杂性在于:毛泽东在修正了共产国际某些极左方针的同时,又坚持了共产国际另一部分极左的方针,而与当时相对务实的中共中央的有关方针发生了冲突。

毛泽东制定的《二七土地法》提出“没收一切土地”的口号,包括没收农民土地;而中共六大关于农村土地问题的决定,则提出只没收豪绅地主的土地,并不主张没收农民土地予以重新分配。导致毛严厉镇压赣西南党的原因之一,即在于赣西南方面坚持中共六大有关“没收豪绅地主土地”的立场。

毛泽东的上述主张恰又与共产国际的精神相符合。1930~年~8~月,共产国际东方部提出《关于中国苏区土地农民问题决议案草案》,明确宣布平分一切土地,包括平分农民的私有土地。毛泽东在土地与农民问题上的某些相对务实的主张,往往与更激进的极左主张相依相存。毛泽东对富农问题的看法,就具有这种特征。

1929~年,共产国际开始推行“反富农”的新方针,随即迅速传至中国。毛泽东在苏区积极贯彻并发展了共产国际这项《反富农》方针。

1930~年~6~月,毛泽东主持制定了《富农问题决议案》,虽然批评了违背农民意愿兴办“模范农场”的错误,强调了“抽多补少,抽肥补瘦”的分田原则,但是在对待富农的问题上,毛的态度与共产国际的有关方针并无任何差别,甚至言辞更为激烈。

毛泽东猛烈抨击富农,宣称“富农的剥削比较地主更加残酷”,“这个阶级自始至终是反革命的”。毛甚至还将打击的矛头指向那些未出租土地、也不雇工、自己耕种土地的富裕中农,指称他们是“第三种”富农,号召要像对“第一种”(“半地主性的”)、“第二种”(“资本主义性的”)富农一样,“坚决赞成群众没收他们的土地,废除他们的债务”。对于债务问题,仅仅四个月前制定的《二七土地法》还规定,工农贫民之间的债务仍然有效,到了此时,毛竟修改了原先的主张,提出“废除一切债务”的口号。毛并认为“废除高利贷”的口号是错误的。更有甚者,毛居然发明了“富农共产党员”的概念,把赞成中共六大“没收豪绅地主土地”的同志,一律视为“党内的富农成份”,要求将他们从党的队伍中“洗刷出来”,“无条件地开除富农及一切富农路线的人出党”\footnote{《富农问题——~1930~年~6~月前委闽西特委联席会议决议》,载《中央革命根据地史料选编》,下册,页~398~—~99、400、402、404、410、413。}。

对于毛泽东在富农问题上的极左主张,六届四中全会前的中共中央并不完全赞成,但是到了~1931~年初中共六届四中全会后,中共中央的土地政策向极左的方向急剧转变。此时的中共中央已全盘接受了共产国际的新路线,改变了中共六大原先在土地方面的政策。1931~年~8~月~21~日,根据共产国际新制定的关于中国土地问题的决议,和王明于~1931~年~3~月起草的《土地法令草案》的精神,由任弼时主持,苏区中央局通过了《关于土地问题的决读案》,开始贯彻比毛更激进的土地政策。该决议案抽象肯定了毛的“抽多补少,抽肥补瘦”的土地分配原则,但随即又将其贴上“非阶级的”标签。同时严厉批评给地主分田的方针“离开了土地革命的观点”,宣布今后一律不再分配土地给地主,富农只能分坏田。\footnote{《苏区中央局关于士地问题的决议案》(1931~年~8~月~21~日),载中央革命根据地史料选编》,下册,页~445、448。}同年~8~月~30~日,周恩来起草的给苏区中央局和红一方面军总前委的信,批评中央苏区犯了“富农路线的错误”(“抽肥补瘦,抽多补少”和没有没收富农的剩余农具)\footnote{《中央给苏区中央局并红军总前委的指示信——关于中央苏区存在的问题及今后的中心任务》(1931~年~8~月~30),载中央档案馆编:《中共中央文件选集》(1931),第~7~册,页~361、357。}。

接着,在瑞金召开的苏区第一次党代表大会上,再次按照周恩来起草的中央指示信精神,对毛泽东进行了不点名的批评。会议通过的《政治决议案》批评毛制定的“抽多补少,抽肥补瘦”的土地分配原则,“犯了富农路线的错误”。随后举行的第一次全国工农兵代表大会,根据王明起草的土地法草案,正式通过《中华苏维埃共和国土地法》,地主不分田、富农分坏田的土地政策开始在中央苏区全面推行。

从~1932~年~3~月开始,在中央苏区又展开了“土地检查运动”,1933~年,转入查田运动。毛泽东虽被中央局责成领导这场运动,但毛因参与指挥战争,继而在宁都会议上被剥夺了军事指挥权,随后长时间告假休养,直到~1933~年春毛才着手领导查田运动。

毛泽东领导查田运动的得力助手是王观澜。1931~年王观澜自苏联返国进入中央苏区,不久,被任命为《红色中华》主编,但在~1932~年秋,王观澜被苏区中央局指控有“重大托派嫌疑”,而被免去《红色中华》主编职务,调到中央临时政府,协助毛泽东领导查田运动。毛派王观澜深人到叶坪乡(苏区中央局和临时中央政府所在地)进行调查研究,为毛提供了许多生动具体的统计资料,把运动“搞得有声有色”\footnote{参见蔡孝干:《江西苏区·红军西窜回忆》(台北:中共研究杂志社。1970~年),页~103。}。

在已被削弱了部分权力的新形势下,毛泽东对中共中央有关“地主不分田,富农分坏田”的政策,采取了灵活的态度,他没有公开反对这项政策,因而,毛的若干主张也随之被中央所接受。1933~年~6~月~2~日,苏区中央局发表了毛起草的《怎样分析农村阶级》一文。在以划分阶级成分为重点的查田运动全面推开后的七、八、九三个月,中央苏区一共补查出一万三千多名“地主”、“富农”\footnote{毛泽东:《查田运动的初步总结》,载《斗争》,第~24~期,1933~年~8~月~29~日。},其中有相当多的中农,甚至是贫农、雇农被错划为地富分子。

然而毛泽东毕竟长期在农村战斗,对农村状况远比博古、周恩来等人熟悉。在查田运动中,毛较多注意防止“过火”的倾向。由于运动遭到群众的“冷淡”,中央局在不违背“地主不分田,富农分坏田”的原则下,接受了毛的意见,对查田运动作局部调整。1933~年~10~月~10~日,中华苏维埃共和国临时中央政府正式颁布毛的《怎样分析农村阶级》和由毛主持制定的《关于土地斗争中一些问题的决定》,开始纠正查田运动中的“过火”行为。例如,胜利县在毛的两份文件下达后,就改正了错划地富共~94~户\footnote{王观澜:《继续开展查田运动与无情的镇压地主富农的反攻》,载《红色中华》,1934~年~3~月~20~日。}。

但是毛泽东对查田运动所作的调整很快就被中央局加以扭转,中央局怀疑毛的调整已危及中央的路线。1934~年~3~月~15~日,新任人民委员会主席张闻天发布了“训令中字第一号”——《关于继续开展查田运动的问题》,其基本精神是“坚决打击以纠正过去‘左’的倾向为借口,而停止查田运动的右倾机会主义”。训令规定,“不论地主、富农提出任何证据,不得翻案,已翻案者无效”,“地主、富农利用决定上的任何条文作为翻案的武器,必须防止。他们的一切反革命活动应该受到最严厉的苏维埃法律的制裁”。随着这个训令的贯彻,苏区各地阶级成分已经改正的农民纷纷又被戴上“地主”、“富农”的帽子。胜利县在二十多天内,就把已经改变阶级成份的~1512~户中的~890~户,重新划为“翻案的地主、富农”,而且又“新查出地主、富农八十三家”\footnote{高自立:《继续查田运动的初步检查》,载《红色中华》,1934~年~5~月~7~日。}。

1934~年~2~月以后的查田运动,随着中央苏区军事形势日益恶化,更趋极端。被定为“地主”、“富农”的人遭到极为严厉的处置,地主一律被编人“永久劳役队”,富农则编人“暂时劳役队”,地、富家属“一律驱逐出境”。许多农民惧于“红色恐怖”,“成群结队整村整乡”地逃往国民党统治区域\footnote{《人民委员会为万太群众逃跑问题给万太县苏主席团的指示信》,载《红色中华》,1934~年~4~月~10~日。}。到了~1934~年~7~月,情况甚至发展到“造成一种削弱苏维埃政权的无政府状态”,以至人民委员会主席张闻天撰文号召“反对小资产阶级的极‘左’主义”\footnote{参见程中原:《张闻天传》(北京:当代中国出版杜,1993~年),页~178~—~79。},但一切已为时晚矣。此时,中共中央忙于部署大规模的战略转移,查田运动终告结束。然而,在土地政策方面,毛与中共中央的分歧,并没有得到任何解决。

\section{在军事战略方针方面的分歧}

1931~年~11~月后,毛泽东迫于中共中央的压力,被迫表示服从中央的路线,但是这种“服从”只是表面的,在他最熟悉、最具优势的军事作战领域,毛曾多次公开表示自己不同于中央的意见。

毛泽东不是军人,但是在长期的武装斗争中,逐渐形成了一套有关红军作战的战略战术。事实证明,在红军处于劣势的条件下,毛的这些主张对保存、发展红军实力极为有用。但由于毛个性专断、“处事独裁”,在用人方面有较强的宗派色彩,以至在一段时期内,毛在红军中的口碑远低于作风民主的朱德\footnote{《龚楚将军回忆录》,页~207、357;另参见《黄克诚自述》,页~100~—~101。}。

中共中央不满于毛的军事作战方针始于赣南会议期间,在这次会议上,曾经提出过红军中存在着有待纠正的“狭隘经验论”及“忽视阵地战、白刃战”的“游击主义的传统”的问题\footnote{《毛泽东年谱》,上卷,页~359。}。但是赣南会议的主题是批评毛的土地政策,而未及全面检讨苏区的军事战略问题。中央代表团对毛的军事方针的批评只是隐约其辞,任弼时、王稼祥等当时还是标准的文职党干部,让他们去讨论自己未曾经历过的军事作战问题,显然没有像研讨具有理论色彩的土地政策问题那样得心应手。

在中共中央首席军事专家周恩来,及一批在苏联学习军事的干部陆续进入中央苏区后,中共中央与毛在军事战略方面的分歧就逐渐显现了出来,由攻打赣州而引发的有关军事战略问题上的激烈争论,使毛与上海中央的冲突几近白热化。

攻打赣州的决策是远在上海的临时中央政治局作出的。1932~年~1~月~9~日,临时政治局通过了由张闻天起草的《关于争取革命在一省与数省首先胜利的决议》,号召“为占领几个中心城市以开始革命在一省数省首先胜利而争斗”,同日,电令苏区中央局“急攻赣州”。

周恩来对于是否执行攻打赣州的计划,曾经有过数次变化。在未到苏区之前,周是主攻派;在抵达赣南与毛交换意见后,周接受了毛的看法。周并向上海发电表示,在目前形势下,攻打中心城市存在困难。临时政治局覆电坚持原有意见,周就又接受了上海的指示,于~1~月~10~日发出训令,决定攻打赣州\footnote{《周恩来年谱》,页~216~—~17。}。

攻打赣州之役最后遭到失败。1932~年~3~月中旬,周恩来主持召开总结攻赣经验教训的苏区中央局会议。在这次会议上,毛泽东提出集中兵力向北发展,打通赣东北的主张,但遭到否定\footnote{《彭德怀自述》,页~175~—~76;另见《周恩来年谱》,页~218。}。会议采纳了周恩来等多数人的意见,以赣江附近为中心,由彭德怀率红三军团组成西路军向赣江西岸出击,争取打通湘赣苏区;由毛泽东率一、五军团组成东路军,向闽西发展。4~月~20~日,毛率军攻占了闽南重镇漳洲,其役是~1932~年中央苏区在军事上的最大胜利。

漳洲战役的胜利,暂时减缓了苏区中央局对毛泽东的不满,但是随着~1932~年~6~月蒋介石对中央苏区发动第四次围剿,和上海中央强令贯彻“进攻路线”,苏区中央局和毛泽东围绕军事方针上的分歧重新尖锐起来,最终导致毛泽东的军权被剥夺。

蒋介石对中央苏区发动的第四次围剿改变了以往的战略:先打湘鄂西和鄂豫皖苏区,以扫清中央苏区的外围;继而重点进攻中央苏区。为应付这种新的变化,1932~年~6~月上旬,中央临时政治局和苏区中央局决定恢复赣南会议上撤消的红一方面军建制,任命周恩来为总政委;紧接着又仿苏联内战体制,在人民委员会下成立了以周恩来为主席的劳动与战争委员会,作为战争动员和指挥作战名义上的最高机关。7~月中旬,周以中央局代表的身份赶赴前方与毛泽东会合,后方则由任弼时代理中央局书记\footnote{《周恩来年谱》,页~223。}。

此时的毛泽东正集中全力指挥战事,但权责并不明确,毛只是以中央政府主席、中革军委委员的身份随军行动。为此,周恩来与毛泽东、朱德、王稼祥联名,于~1932~年~7~月~25~日致电中央局,建议以毛为红一方面军总政委,其职权范围为指挥作战,行动方针的决定权则由周恩来掌握。对于周等的建议,中央局拒绝予以批准,坚持应由周担任总政委一职。7~月~29~日,周恩来致信中央局,再次坚持原有意见。周在信中说,如由他本人兼总政委,将“弄得多头指挥,而且使政府主席将无事可做。泽东的经验与长处,还须尽量使他发展而督使他改正错误”\footnote{《周恩来年谱》,页~223~—~24。}。在周恩来的一再恳求下,中央局才照准周的提议,8~月~8~日,任命毛为总政委。

毛泽东在二十年代末、三十年代初逐渐形成了具有自己特色的军事战略方针,这就是一切以保存、壮大实力为前提,绝不与敌打消耗战;集中优势兵力进攻敌之薄弱环节,“与其伤敌十指,不如断其一指”——在实际运用中,毛的这套作战原则,经常表现为在敌进攻前,军队进行大幅度后退,这些又恰恰被临时中央政治局视为是毛的“右倾机会主义”和“等待主义”的集中体现。

1932~年~4~月~4~日,张闻天(洛甫)发表著名的《在争取中国革命在一省数省的首先胜利中中国共产党内机会主义的动摇》一文,用“中央苏区的同志”的代名,不点名指责毛泽东的“诱敌深入”战略“表现出浓厚的等待主义”,仅“把‘巩固苏区根据地’当作符咒一样的去念”\footnote{《六大以来——党内秘密文件》(上),页~216。}。4~月~11~日,项英从江西秘密抵达上海,向临时中央政治局常委会汇报苏区工作,几个常委在发言中都对苏区工作提出严厉批评。有的常委认为,中央苏区领导在革命基本问题的看法是“民粹派的观点”,“中央区的领导是脱离了布尔什维克的路线的”。有的常委在发言中认为“狭隘经验论”的实质是“机会主义障碍路线的执行”\footnote{参见中共中央文献研究室编:《毛泽东传(1893~—~1949)》(北京:中央文献出版社,1996~年),页~290。}。正是在这种背景下,4~月~14~日,上海中央致信苏区,批评苏区中央局“不了解红军的积极行动的必要而陷入庸俗的保守主义”,命令对右倾“做最坚决无情的争斗”\footnote{《周恩来年谱》,页~219。}。

毛泽东对于临时中央政治局~4~月~14~日来信极不以为然,他在~5~月~3~日覆电苏区中央局,明确表示“中央的政治估量和军事战略,是完全错误的”\footnote{《毛泽东年谱》,上卷,页~379。}。但是,周恩来对于上海中央一向尊重并言听计从,在收到中央来信后,于~1932~年~5~月~11~日主持中央局会议,表示接受中央的批评,随即宣称,“立即实行彻底的转变”,“彻底纠正中央局过去的右倾机会主义错误”\footnote{《毛泽东年谱》,页~375;另参见《周恩来年谱》,页~220。}。上海中央一不做,二不休,于~5~月~20~日再给苏区中央局发出指示信,直接点名批评毛泽东,将其军事主张定为“游击主义”和“纯粹防御路线”。宣称“泽东及其他纯粹防御路线的指挥者”的“消极态度”,是当前“极大的危险”,要求苏区中央局:

\begin{quoting}
以说服的态度,设法争取他(指毛泽东——引者注)赞成积极斗争的路线,使他在红军及群众中宣传积极路线,争取党和红军的干部说服他的纯粹防御路线的错误与危险,公开讨论泽东的观点\footnote{中共中央~1930~年~5~月~20~日给苏区中央局的指示电最早收于~1932~年~7~月~1~日苏区中央局出版的《为实现一省数省革命首先胜利与反机会主义的动摇而斗争》小册子中,1941~年~11~月中共中央书记处编辑出版的《六大以来》也收录了此文,1991~年复被收入《中共中央文件选集》(1932~—~1933),第~8~册,但是在以上所有版本中,没有批评毛泽东的文字。据分析,似因~1932~年~7~月需公开出版小册子,此电文中有关涉及毛泽东的内容已被苏区中央局所删节,上述有关批评毛的电文转引自《中央革命根据地史要》,页~305。该书引用的这段电文没有标明原始出处。}。
\end{quoting}

两个月后,上海中央又给苏区中央局及闽赣两省委发出指示信。继续批评中央局“没有及时采取进攻的策略”,再次敦促苏区中央局“进行彻底的转变”\footnote{《周恩来年谱》,页~223。}。

面对上海方面的再三催促,周恩来不得不兴兵作战。1932~年~8~月初,周恩来在兴国召开中央局会议,决定发动乐安、宜黄战役,以威胁南昌,吸引围剿鄂豫皖的国民党军队。红一方面军攻占乐安、宜黄后,于~8~月~24~日进抵南昌城近郊,周、毛发现守敌强大,当即放弃攻城。从~8~月下旬至~9~月下旬,周、毛、朱率军分兵在赣江、抚河之间转战月余,这样又受到中央局的严厉指责。

中央局坚持红一方面军应打永平,周、毛、朱、王稼祥则认为在目前形势下红军应以“促起敌情变化”为方针,避免“急于求战而遭不利”\footnote{《周恩来年谱》,页~228。}。双方电报往来十余次,互不相让。9~月~29~日中央局致电周、毛、朱、王,批评彼等的意见“完全是离开了原则,极危险的布置”\footnote{《周恩来年谱》,页~230。},决定立即在前方开中央局会议。这次会议即是~1932~年~10~月在宁都召开的由周恩来主持的苏区中央局会议,史称“宁都会议”。

宁都会议对毛泽东的指责十分激烈。会议对毛进行了面对面的批评,尤其指责毛“不尊重党领导机关与组织观念上的错误”。虽然这次会议的主题是讨论评估攻打赣州以来的几次重大战役,但是问题最后都集中到批评毛对党机关的态度上,毛的比较正确的军事主张被会议否定,与此有密切关系。在~1931~年赣南会议后,毛的自傲一直是中央局与毛关系紧张的一个重要因素。苏区中央局利用军事战略问题的争论,一举剥夺了毛的军事指挥权。10~月~12~日,中革军委发布通令,调毛泽东“暂回中央政府主持一切工作”,10~月~26~日,临时中央正式任命周恩来兼任红一方面军总政委。

在宁都会议上,毛泽东受到任弼时、项英、邓发、顾作霖等与会绝大多数苏区中央局成员的批评与指责,唯有周恩来的态度较为温和,在一些问题上为毛泽东作了辩护和开脱。

对毛泽东批评最尖锐的是在后方瑞金主持苏区中央局的代书记任弼时和苏区中央局成员项英。任、项根据临时中央~2~月以来有关加速反“右倾”的历次决议,尤其依据~5~月~20~日和~7~月~21~日中央政治局两次对苏区中央局批评信的精神,尖锐指责毛泽东的“诱敌深人”军事方针和向赣东北发展的主张是“专等待敌人进攻”的保守的“等待观念”\footnote{《苏区中央局宁都会议经过简报》(1932~年~10~月~21~日),中央档案馆编:《中共中央文件选集》(1932~—~1933),第~8~册,页~530。}。

朱德、王稼祥由于一直随周恩来、毛泽东在前方指挥作战,同属四人最高军事会议,客观上也必须分担苏区中央局对毛的批评,因此在宁都会议上并不积极,只是一般地同意、附和了任弼时、项英等的看法。

周恩来作为前方四人最高军事会议主席和前方负最后决定权的苏区中央局代表,他的看法和态度是至关重要的。周恩来在宁都会议上的表现,反映了他一贯的作风和风格:即一方面接受中共中央和苏区中央局的意见;另一方面又从维护党、红军的愿望出发,对毛泽东表示充分尊重。

周恩来在发言中承认在前方的最高军事会议,“确有以准备为中心的观念”,认为后方中央局任弼时、项英等所强调的“集中火力反对等待倾向是对的”;同时也批评后方同志对敌人大举进攻认识不足,指出他们对毛泽东的批评过份。周强调“泽东往年的经验多偏于作战,他的兴趣亦在主持战争”,“如在前方则可吸引他贡献不少意见,对战争有帮助”,坚持毛泽东应留在前方。为此,周恩来提出两种办法供中央局选择:“一种是由我负指挥战争全责,泽东仍留前方助理;另一种是泽东负指挥战争全责,我负监督行动方针的执行”\footnote{《周恩来年谱》,页~231。}。但与会“大多数同志认为毛同志承认与了解错误不够,如他主持战争,在政治上与行动方针上容易发生错误”。毛泽东也因不能取得中央局的全权信任,坚决不赞成后一种办法。会议于是通过周恩来提出的第一种办法,并“批准毛同志暂时请病假,必要时到前方”\footnote{《苏区中央局宁都会议经过简报》(1932~年~10~月~21~日),中央档案馆编:《中共中央文件选集》(1932~—~1933),第~8~册,页~530。}。

宁都会议后,毛泽东已被剥夺了军事指挥权,但周恩来却因其在宁都会议上的折衷态度受到苏区中央局成员项英、顾作霖等的批评。1932~年~11~月~12~日,后方苏区中央局成员任弼时、项英、顾作霖、邓发联名致电临时中央,报告宁都会议经过和争论情况,其中涉及到对周恩来的看法:“恩来同志在会议前与前方其他同志意见没有什么明显的不同,在报告中更未提到积极进攻,以准备为中心的精神来解释中央指示电”,“对泽东的批评,当时项英发言中有过份的地方,但他(指周恩来——引者注)在结论中不给泽东错误以明确的批评,反而有些地方替他解释掩护,这不能说只是态度温和的问题——我们认为恩来在斗争中不坚决,这是他个人最大的弱点,他应该深刻了解此弱点加以克服”\footnote{《周恩来年谱》,页~233~—~34;另参见《任弼时传》,页~245。}。

同一日,周恩来致电上海临时中央,为自己在宁都会议上的表现进行辩解:“我承认在会议上我对泽东同志的批评是采取了温和的态度,对他的组织观念错误批评得不足,另外却指正了后方同志对他的过份批评”,但“会后顾、项等同志认为未将这次斗争局面展开,是调和,是模糊了斗争战线,我不能同意”\footnote{《周恩来年谱》,页~233;另参见《任弼时传》,页~244。}。

根据现存资料,毛泽东在宁都会议上并没有接受苏区中央局对他的指责,11~月~26~日,苏区中央局在给上海中央的电报中也称,毛“仍表现有以准备为中心的意见”\footnote{《毛泽东年谱》,上卷,页~391。},然而根据临时中央~11~月给苏区中央局的覆电却又看出,毛在压力下,在会议上曾被迫作出承认“错误”的表态:

\begin{quoting}
泽东同志在会议上已承认自己的错误,必须帮助泽东同志迅速彻底地改正自己的观点与吸引他参加积极的工作\footnote{《周恩来年谱》,页~234。}。
\end{quoting}

毛泽东既然已在一定程度上承认自己犯下“错误”并离开了军事指挥岗位,周恩来也是在肯定毛有错误的前提下,主张对毛采取较为宽和的态度,那么继续维护以周恩来为核心的苏区中央局的团结就是当下最重要的任务了。1932~年~11~月,临时中央覆电苏区中央局,指出:“恩来同志在(宁都)会议上的立场是正确的,一部分同志责备恩来为调和派是不正确的”。“为击破敌之‘围剿’,领导一致是目前最重要的”\footnote{《周恩来年谱》,页~234。}。

继赣南会议批评毛泽东的土地政策,1931~年底至~1932~年初,周恩来集中纠正毛的肃反偏差,现在又在宁都会议上集中批评了毛的军事作战方针,毛在中央苏区的权势被一步步削夺。宁都会议结束后,苏区中央局书记仍由周恩来担任,当周在前方指挥作战时,苏区中央局书记一职则继续由在后方的任弼时代理。1933~年~1~月~27~日,博古抵达瑞金后,虽然局部调整了中央苏区的领导机构,但是宁都会议后形成的权力格局基本没有变动。

宁都会议后,毛泽东因在政治上蒙受打击和患严重疟疾,在长汀医院休养达半年之久,周恩来曾数次请张闻天、博古劝毛回瑞金工作\footnote{《周恩来年谱》,页~245。}。1933~年~3~月,共产国际执委会致电瑞金,指示“要运用诱敌深人,各个击破,瓦解敌军和消耗敌人的战术”,同时要求“对毛泽东必须采取尽量忍让的态度和运用同志式的影响,使他完全有可能在党中央或中央局领导下做负责工作”\footnote{《共产国际执委会致中国共产党中央委员会电》(1933~年~3~月),载《中共党史研究》,1988~年第~2~期;另参见《毛泽东年谱》,上卷,页~398。}。共产国际对毛的关照,对毛处境的改善有所作用,1933~年春夏间毛返回瑞金,开始主管查田运动。在~6~月上旬出席中央政治局会议时,毛对一年前的宁都会议提出批评,认为自己受到了不公正的对待。但是,博古把毛的批评挡了回去,重申宁都会议是正确的,并说没有第一次宁都会议,就没有第四次反围剿的胜利\footnote{《毛泽东年谱》,上卷,页~403。}。从此,毛未再予闻中央苏区的军事指挥事宜。只是到了~1933~年~10~月,陈铭枢、蒋光鼐、蔡廷楷派代表前来苏区联络商谈双方停战之事时,毛才被允许参与某些重要军事决策的讨论。

根据~1935~年后的毛泽东的解释,在关于是否援助第十九路军的讨论中,毛提议红军应向以江浙为中心的苏浙皖赣出击,以调动围赣之敌,打破国民党军对中央苏区的“围剿”,同时支持福建人民政府\footnote{参见毛泽东:《中国革命战争的战略问题》,载《毛泽东选集》,第~1~卷,页~236。}。博古等人却拒绝了毛的正确意见,导致中央苏区在进行第五次反“围剿”战争时陷于孤立,成为被迫长征的主要原因。

然而,毛泽东在福建事变期间的态度远比上述解释复杂的多。毛的有关红军出击苏浙皖赣的意见被否定,确实使打破“围剿”失去了一个重要机会,但是拒绝与十九路军合作则肯定加剧了中央苏区军事形势的危机。

福建事变发生后,毛泽东在公开场合是坚决主张对陈铭枢等采取孤立政策的。1934~年~1~月~24~至~25~日,毛在中华苏维埃共和国第二次全国代表大会上所作的政府报告中说:

\begin{quoting}
至于福建所谓人民革命政府,一位同志说他有一点革命的性质,不完全是反动的,这种观点是错误的。……人民革命政府只不过是部分统治阶级以及在共产主义和反动政治之间用“第三条道路”的虚伪口号来欺骗人民的鬼把戏而已\footnote{《红色中华》,1934~年~8~月~1~日。}。
\end{quoting}

如果说毛泽东的上述言论是在公众场合依照中央路线而发表的,不足以表明他的真实想法,那么在内部讨论时,毛的态度又是怎样的呢?据与毛关系一度十分密切、曾担任赣南军区参谋长的龚楚回忆,在领导层讨论陈铭枢、蔡廷楷等人提出的要求与红军联合行动的会议上,毛主张采取谨慎的方法,提出“派不重要的代表到福建去与李济深等先进行试探的会商”,反对周恩来等提出的“立即派大员到福州去举行正式谈判”的意见\footnote{龚楚:《我与红军》(香港:香港南风出版社,1954~年),页~364。关于龚楚和他的回忆录《我与红军》一书的史料价值问题,杨尚昆在~1984~年~7~月~9~日的一次内部谈话中曾说:“有一个叫龚楚的,在井冈山时期就跟毛主席在一起,在中央苏区时是作战处长。此人在三年游击战争中被捕叛变,还带敌人去抓陈毅。后来,龚楚写了《我与红军》一书,在台湾、香港广为流传,书中写了他是怎样参加红军和在红军中做了些甚么工作。建国后我看了这本书,曾问过陈老总(指陈毅——引者注),他说龚楚的历史就是那个样子,叛变前的那一段历史基本上是确实的。”见杨尚昆:《在全军党史资料征集工作座谈会上的讲话》(1984~年~7~月~9~日),此讲话稿经杨尚昆修订,并征得杨本人同意后发表于中共中央党史研究室编辑出版的《党史通讯》~1984~年第~11~期。}。

龚楚的口述回忆只是提供了一种说法,是否完全确实,还有待新资料的发现和证实\footnote{龚楚在~1978~年出版的《龚楚将军回忆录》中修正了他在《我与红军》一书中有关苏区核心层对福建事变争论的叙述。龚楚称他前书有误,“是因当时记忆错误所致”。在《龚楚将军回忆录》中,毛泽东被改为“主张立即派大员到福州举行正式谈判”。笔者认为,龚楚的更正应予以重视,但他在~1954~年出版的《我与红军》中的有关叙述,的确可从另外的资料得到证实,故本书倾向于接受前一书的论断,并认为此一问题的彻底澄清还有赖于新资料的发现。参见《龚楚将军回忆录》,页~513、515。}。由于毛泽东当时不处于核心决策层,即使毛反对援助陈铭枢、蔡廷楷,这个决定仍需博古、李德和周恩来作出。

根据现有资料分析,导致与福建人民政府的合作不能实现,红军丧失最后机会,是由于共产国际及在沪的代表的错误指导,以及博古、周恩来的犹疑不决。

博古原是主张联合蔡廷楷的,并得到周恩来的支持。博、周认为,应该抓住这个难得的机会,去实现~1933~年~1~月~7~日中共提出的建立广泛民族统一战线的新方针\footnote{参见奥托·布劳恩(李德):《中国纪事(1932~—~1939)》(北京:现代史料编刊社,1980~年),页~84、85;另参见龚楚:《我与红军》,页~364。}。1933~年~11~月~24~日,周恩来致电瑞金的中革军委,催促早为决定红三、五军团是否参加侧击向福建前进的蒋介石入闽部队\footnote{《周恩来年谱》,页~254。}。周并经中央同意,派出潘汉年作为红军代表,与十九路军谈判并签订了“反日反蒋的初步协定”。

博古、周恩来的意见,并没有获得中共核心层的一致支持\footnote{参见奥托·布劳恩(李德):《中国纪事(1932~—~1939)》,页~85。},而在否定周、博意见的过程中,共产国际及驻华代表、中共上海局则起到决定性作用。

1933~年~10~月~25~日,共产国际给瑞金来电,提出:“为着战斗的行动的目的,应该争取下层统一战略的策略,国民党的广东派,以反日的武断空话(护)符,隐蔽地为英帝国主义的奴仆,这种假象是应该揭穿的。”\footnote{引自方长明:《试述共产国际与我党对闽变的策略》,载《党史资料与研究》(福建),1983~年第~3~期。}这份电报对中共中央转变对福建事变的态度有重要影响。在沪的共产国际驻华代表阿瑟·尤尔特、军事总顾问弗雷德和中共上海局忠实执行莫斯科的指示,他们一致认为,蔡廷楷与蒋介石之间不过是军阀间的一般斗争,中央应最大限度地利用这种斗争来加强自己在内战中的地位,不给蔡廷惜以实际的军事援助\footnote{参见奥托·布劳恩(李德):《中国纪事(1932~—~1939)》,页~85~—~86。}。

周恩来、博古等对于来自上海的意见并非从一开始就全盘接受,双方曾互相争论,电报往来不绝。10~月~30~日,中共中央曾给福州中心市委与福建全体同志发了一封指示信,要求反对关门主义的左倾思想,但是中共中央在收到共产国际~10~月~25~日来电后,马上转变立场,在~11~月~18~日再次给福州党的书记发了一封与前信内容完全相反的信,该信大骂十九路军,声称他们与中共的停战合作只是“一个大的武断宣传的阴谋”\footnote{福建省档案馆编:《福建事变档案资料》(福州:福建人民出版社;1984~年),页~120~—~21、133。}。显然,在共产国际的压力下,周恩来、博古作了让步,接受了上海方面的意见\footnote{据当时任中共驻十九路军联络员张云逸的回忆,在他去福州前,博古曾指示他,“此行的目的是设法争取拉点队伍过来”,全不提如何出兵配合作战。参见张云逸:《一次重大的失策》,载《福建事变档案资料》,页~226。}。至于李德,据他称,虽对他的顶头上司弗雷德的计划有不同看法,但李德自己也承认,他还是贯彻了弗雷德的指示。

在中共决策层围绕福建事变而发生的争论中,共产国际驻中国代表所发挥的决定性作用,反映了这一时期中共对莫斯科的严重依赖和中共组织结构的不健全。1933~年~1~月,博古抵达瑞金,与周恩来等会合,成立了中共中央局,而在上海的留守干部盛忠亮、李竹声等也组成了中共上海中央局。从法理上讲,在瑞金的中央局即应是中共中央,但在~1934~年六届五中全会召开之前,上海中央局也经常以中共中央的名义活动,而共产国际驻华军事总顾问弗雷德·施特恩经常以中共中央的名义,从上海向瑞金发指示电。周恩来在前线收到弗雷德以中共临时中央名义发来的有关军事作战计划的第一份电报,是在~1933~年~4~月~14~日\footnote{《周恩来年谱》,页~245。}。在~1933~年~9~月李德抵达瑞金之前,弗雷德就曾向瑞金发出四份干预苏区军事行动计划的电报\footnote{《周恩来年谱》,页~245~—~46、249~—~51。}。

弗雷德是李德的直接上司,1933~年春才抵达上海,以后在西班牙内战中以“克勒贝尔将军”之名而著称。李德以后报怨他是替弗雷德受过,似乎弗雷德更应为苏区军事失败而负责,而回避了他自己所应负的重大责任。

至于毛泽东,有关福建事变的争论却改善了他的处境。在这一时期,毛所扮演的角色是微妙的:一方面,毛不处于有权的地位;另一方面,由于苏区军事形势日紧,毛的有关意见又逐渐被重视,毛的作用比~1932~—~1933~年明显增强,这为他一年多后的复出埋下了伏笔。

\section{党权高涨,全盘俄化及毛泽东被冷遇}

自~1931~年~11~月中央代表团举行赣南会议至~1934~年~10~月中共中央机关、中央红军被迫撤出江西,是中共党权大张的时期。在这一时期,党的领导机关的权威得到完全确立和巩固,没有任何军事阅历、文职党干部出身的博古、张闻天等在以周恩来为代表的中共老干部派的支持下,基本控制了原先由毛泽东领导的军队。中央苏区弥漫着“以俄为师”、全盘俄化的气氛,而苏区的创造者、党与军队的元老毛泽东则倍受压抑和冷落。

在原先由毛泽东一人说了算的江西苏区,中共中央迅速在组织上建立起党对毛泽东的优势。1931~年春,继任弼时、王稼祥、顾作霖之后,大批干部被陆续派往江西,其中许多人为留苏返国干部,计有刘伯承、叶剑英、朱瑞、杨尚昆、凯丰(何克全)、李伯钊、伍修权、肖劲光、刘伯坚等,林伯渠、董必武、聂荣臻、阮啸仙等一批老干部也在这一时期被调往江西,他们分别担任了党、军、政、青等机构的领导职务。1933~年~1~月,临时中央负责人博古抵达瑞金,在此前后,张闻天、刘少奇、陈云、罗迈(李维汉)、瞿秋白等也先后到达。博古、张闻天等到达后,和以周恩来为首的苏区中央局会合,于~1933~年~6~月,组成了中共中央局,实际上起着中央政治局的作用。

以博古为首的中共中央得以在江西苏区顺利地确立起领导权威,是与周恩来等的配合、协助分不开的。在中共中央局中,周恩来的力量举足轻重,缺乏苏区经验的博古、张闻天等,离开周的支持是很难维持下去的。由于周恩来与留苏派形成了实际上的联盟,毛泽东在中共中央局中明显处于劣势。

以周恩来为代表的老干部派与国际派的政治结合,在中共六届五中全会上得到进一步的加强,在中央常委(书记处成员)中,国际派的博古、张闻天与老干部派的周恩来、项英达成了权力平衡,毛泽东则未能进入这四人权力核心。

中共六届五中全会还进一步削弱了毛泽东的权力基础。毛所长期担任的政府主席一职被分割为中央执委会主席与人民委员主席两职。由张闻天担任人民委员主席,使得毛的政府主席一职,几乎成为一个虚职。

毛泽东之成为“毛主席”源自~1931~年~11~月~7~日,他开始担任新成立的中华苏维埃共和国中央临时政府中央执委会主席。在以后的几年中,中央执委会下虽设立了人民委员会,但实际上是两块牌子,一套机构,毛基本上以中央执委会主席的名义行事。经中央局同意,毛陆续安排了一些自己的老部下,如邓子恢、王观澜\footnote{王观澜~1931~年自苏联返国进入中央苏区,长期在毛泽束的领导下工作,与毛私交其笃,毛称其为“真正改造好的知识分子”参见赵来群:《毛泽东与王观澜》,载《党的文献》,1996~年第~6~期。}、高自立以及何叔衡等在政府内担任人民委员或副人民委员,但是在博古等发动的反“罗明路线”斗争中,财政人民委员邓子恢、工农检察人民委员何叔衡都被批判和撤职。毛的老部下张鼎丞被撤去福建省苏维埃主席,谭震林也被调离福建军区司令员和政委。

以博古为首的中共中央对防范毛泽东在军队中的影响,给予了高度的注意。1933~年初,博古甫抵中央苏区,在未抵达瑞金前即曾向一些高级军事干部了解对朱、毛的看法。龚楚曾直接向博古反映,毛虽具领导政治斗争的才智和对军事战略的卓见,但其领导方式多有独裁倾向\footnote{龚楚:《我与红军》(香港:香港南风出版社,1954~年),页~356~—~57。}。博古到了瑞金后,为消除所谓“游击主义”对红军的影响,指示周恩来等依照苏联红军的建制,对中央红军的作战训练、军事教育开始了有系统的改造。过去,中革军委主席一职虽长期由朱德担任,但掌握军队实权的则是副主席周恩来。博古抵达瑞金后,周对军队的领导权开始受到限制,1933~年~5~月~8~日,博古、项英参加中革军委,由项英代理中革军委主席,前方军事行动的决定权,改由后方的中央局直接掌握。李德抵达后,周的决策影响力进一步缩小,红军最高决策权又被转移到李德手中。周被削弱军权,也许与周对毛的温和态度有关。在周被削军权的同时,对毛态度冷淡的项英被允许参与军队的决策,毛则完全被排斥于军委之外,甚至连军委委员也不是。在军委总参谋部,正副总参谋长也分别由曾留学苏联的刘伯承与叶剑英担任。刘、叶与毛在~1931~年以前几乎没有接触,而与周恩来却有较深的历史渊源。

中共中央还利用自己在干部上的优势,在中央苏区建立起意识形态宣传教育系统。在~1931~年以前,苏区的所有宣传鼓动工作全部在毛泽东领导的红一方面军统辖之下,毛享有充分的发言权和解释权。留苏干部进入苏区后,出现了解释权转移的明显趋势。1933~年初,临时中央迁至瑞金后,国际派迅速在自己的强势领域——宣传解释马列方面行动起来,建立起一系列机构和学校。张闻天担任了中共中央局宣传部长(1934~年~1~月后为中共中央宣传部长)、中央局党校校长,中央党报委员会主任等职。中华苏维埃政府机关报《红色中华》的主编也从王观澜改由沙可夫担任。国际派还创办了《青年实话》、《苏区反帝画报》等一系列报刊。由国际派控制的党刊,在配合对毛的影射攻击方面发挥了重要的作用,在反“邓、毛、谢、古”的斗争中,中央局党刊《斗争》直接批判《毛泽覃同志的三国志热》,明显影射毛泽东。凡此种种,皆是触发毛在几年后攻读马列、继而夺回解释权的动因。

在共产国际的强大影响下,苏联之外的另一个苏式社会在江西苏区建成并初具规模。中共在中央苏区建立起一套直接脱胎于苏俄的政治、经济、军事动员及意识形态体制。在中央苏区内,党的领导机构——由苏区中央局、中共中央局演变而来的中共中央政治局,处于主宰一切的地位。在政治局常委会下,设立了党的军事决策指挥机构中革军委,下辖中国工农红军总部。中共中央建立起自己的直属机关:中央组织部、中央宣传部、中央党务委员会、中央审查委员会,以及中共中央党校(马克思共产主义学校)和中央机关刊物《斗争》编辑部。中共中央还直接指导共青团中央——少共中央局。在少共中央局下,另设有领导少年儿童的组织——少年先锋队中央总队部。

政府系统的创设也依照于苏联的体制。中央执委会主席与人民委员会主席的权限范围完全类似于苏联:中央执委会主席毛泽东的地位,犹如苏联名誉元首加里宁;张闻天的人民委员会主席一职,如同莫洛托夫所担任的苏联人民委员会主席。在人民委员会之下,也盲目仿效苏联,画床架屋湘赣省、设置了十七个人民委员部及有关委员会。中央苏区管辖的江西省、福建省和闽粤赣省,也依此例设立了名目繁多,而实际上只是徒具形式的机构。

在中华苏维埃共和国时期,中央苏区弥漫着浓厚的俄化气氛,许多机构的名称都有鲜明的俄式色彩。在党的教育系统,有马克思共产主义学校;在军队内,有少共国际师、工人师和红军大学。以后又为了纪念被控参与指挥广州暴动而遭国民党杀害的苏联驻广州副领事郝西史,将红军大学易名为“工农红军郝西史大学”;在肃反保卫系统,有国家政治保卫局;在政府教育系统,有沈泽民苏维埃大学、高尔基戏剧学校、高级列宁师范学校、初级列宁师范学校,和众多的列宁小学。在中央苏区,还有“苏区反帝总同盟”和号称拥有六十万成员的“苏联之友会”。每逢列宁诞辰、十月革命纪念日、国际劳动节等众多的国际共运纪念日,都要举行各种形式的纪念活动。在一些重要的政治集会上,不仅要组成大会主席团,选出国际共运和苏联著名人物作“名誉主席”(如~1934~年~1~月在瑞金召开的第二次苏维埃全国代表大会就把斯大林、莫洛托夫、加里宁、合尔曼、片山潜、高尔基等都列为大会的名誉主席),还要发出“致苏联工人和集体农庄农民电”。

1934~年~9~月中旬,中央苏区的形势已极端危急,中华苏维埃共和国临时中央主席毛泽东已被完全排挤出核心决策层,他“日夜忧思,对时局放心不下”,在得到中央同意后,来到南线的于都视察\footnote{《毛泽东年谱》,上卷;页~433。},在这里他会见了井冈山时期的老部下、时任赣南军区参谋长的龚楚。毛泽东对龚楚说:“龚同志!现在不是我们井冈山老同志的世界了!我们只好暂时忍耐吧!”说至此,毛竟凄然泪下\footnote{参见《龚楚将军回忆录》(香港:明报月刊社,1978~年),页~550。}!

所有这些表明,力图在中共党内贯彻共产国际路线的博古等留苏派,在政治上已经取得了对毛泽东的完全优势。然而,博古等的成功仅是一种虚幻的假象,留苏派最缺乏的是军事方面的成功。在国民党军队大举围剿下,博古等不能取得实质性的军事胜利,其一切成功都尤如建筑在沙滩上的楼阁,一遇风浪,终将被摧毁。
