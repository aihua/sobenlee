%# -*- coding:utf-8 -*-
%%%%%%%%%%%%%%%%%%%%%%%%%%%%%%%%%%%%%%%%%%%%%%%%%%%%%%%%%%%%%%%%%%%%%%%%%%%%%%%%%%%%%
%%  MAChapter14.tex'


\chapter{Fourier~级数}\label{ch:14}
\section{基本三角函数系}
\begin{exercise}

\end{exercise}
\section{周期函数的~Fourier~级数}
\begin{exercise}

\end{exercise}
\section{Fourier~级数的收敛性}
\subsection{Fourier~级数的部分和}
\subsection{Fourier~级数部分和的极限问题}
\subsection{Fourier~级数的收敛性判别法——~Dini~判别法}
\subsection{Fourier~级数收敛的~Dirichlet~判别法}
\begin{exercise}

\end{exercise}
\section{任意区间上的~Fourier~级数}
\subsection{周期是~$2\ell$~的情形}
\subsection{非周期函数的情形}
\subsection{函数的奇延拓与偶延拓}
\begin{exercise}

\end{exercise}
\section{Fourier~级数的平均收敛性}
\subsection{平方平均偏差与它的最小值}
\subsection{用三角多项式逼近函数}
\begin{exercise}

\end{exercise}
\section{Fourier~级数的复数形式与频谱分析}
\begin{exercise*}

\end{exercise*}




\endinput
%%
%% End of file `MAChapter14.tex'.