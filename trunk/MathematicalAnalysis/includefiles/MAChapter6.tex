%# -*- coding:utf-8 -*-
%%%%%%%%%%%%%%%%%%%%%%%%%%%%%%%%%%%%%%%%%%%%%%%%%%%%%%%%%%%%%%%%%%%%%%%%%%%%%%%%%%%%%
%%  MAChapter6.tex'


\chapter{不定积分}\label{ch:6}
\section{不定积分概念}
\begin{exercise}

\end{exercise}
\section{积分表与线性性质}
\begin{exercise}

\end{exercise}
\section{换元法}
\subsection{第一换元法}
\subsection{第二换元法}
\begin{exercise}

\end{exercise}
\section{分部积分法}
\begin{exercise}

\end{exercise}
\section{有理函数的积分}
\begin{exercise}

\end{exercise}
\section{三角函数有理式的积分}
\begin{exercise}

\end{exercise}
\section{无理函数的积分}
\subsection{$\displaystyle\int R\biggl(x,\kern-1ex\LEFTROOT{-5}\UPROOT{12}\SQRT[m]{\frac{\smash[b]{ax+b}}{\smash[b]{cx+d}}}\enspace\biggr)\,\mathrm dx$~型积分}
\subsection{二项式微分式积分}
\subsection{$\displaystyle\int R\bigl(x,\sqrt{ax^2+bx+c}\bigr)\,\mathrm dx$~型积分}
\begin{exercise}

\end{exercise}
\begin{exercise*}

\end{exercise*}


\endinput
%%
%% End of file `MAChapter6.tex'.