%# -*- coding:utf-8 -*-
%%%%%%%%%%%%%%%%%%%%%%%%%%%%%%%%%%%%%%%%%%%%%%%%%%%%%%%%%%%%%%%%%%%%%%%%%%%%%%%%%%%%%
%%  `MAPreface-III.tex'

\begin{preface}
北京大学数学系编的《数学分析》分三册出版。第一册由方企勤副教授编写,第二册由沈燮昌教授编写,本书是第三册,内容包括多
元函数的微分学和积分学,还有含参变量的积分。

本书是我们在北京大学数学系多年讲授《数学分析》课的基础上编写成的。我们力图使本书在微积分教学现代化方面有较充分而恰当
的体现,使它既适应于现代科学技术和数学发展的需要,又切合我国的实际教学情况。

本套《数学分析》教材仍按原来的传统,先讲一元函数微积分(第一册和第二册),然后再讲多元函数微积分学,由浅入深,便于学
生掌握。但是,对于多元函数,我们没有按照过去的习惯讲法,从二元、三元讲起,再过渡到~$m\,(m\geq 4)$~元,而是直接
讲~$m\,(m\geq 2)$~元。不仅多元微分学是这样,多元积分学也是这样。实践证明,只要对~$m\,(m\geq 2)$~维~Euclid~空间的
完备性,空间中点集的列紧性、紧致性和连通性等性质作了比较深入的阐述,使学生对~$m\,(m\geq 2)$~维与一维~Euclid~空间之
间的异同有一个比较清楚的认识,直接讲~$m\,(m\geq 2)$~元是可行的。这不仅节省了教学时间,更重要的是适应了发展的前景,%
使读者更容易从~Euclid~空间过渡到度量空间以及更一般的拓扑空间中去。

本书对向量空间极为重视,不仅增加了这方面的内容,而且将其应用贯穿于本书之中。在向量微分学这一章(第十七章)里,用线性
变换定义向量函数的导数,这是古典微积分中导数概念的深化和发展,有利于学生掌握导数概念的实质和应用。在此基础上,我们用
压缩映射原理证明反函数的存在和可微定理,并由此推出隐函数的存在和可微定理。显然,这一部分内容相对来说较为抽象,不够直
观,难度稍大一些,初学者不容易掌握。为此,在讲授这些内容之前,在第十六章里先讨论了多元数值函数的微分学,还用几何直观较
强的方法证明了由一个方程式确定的隐函数存在和可微定理,为学生学习向量函数微分学铺设阶梯和桥梁,循序渐进,逐步提高到预
定的理论高度。这样做,免不了有一些重复,但这不是简单重复,而是螺旋式地上升。

在多元积分学这一章(第二十章)里,我们简要地介绍了~Jordan~测度及其主要性质。之所以要增加这一部分的内容,一方面是
由于我们直接讲~$m\,(m\geq 2)$~重积分,Jordan~测度是不可少的;另一方面也是由于~Jordan~测度本身有许多可取之处。Jordan~
测度是各种测度中最简单而直观的一种,容易为初学者接受和掌握。应用~Jordan~测度,可以明确给出重积分中许多定理成立的条件,%
并简化定理的证明。先在数学分析课中讲~Jordan~测度,在实变函数等后续课程再讲~Lebesgue~测度和抽象测度,学生就有可能对各种测
度之间进行分析对比,加深对测度概念实质的认识,了解各种积分之间差异的由来。

对于~$m\,(m\geq 2)$~重积分的变量替换公式,我们给出了公式成立的两个充分条件,并给予严格证明。这两个定理的证明是在向
量函数微分学的基础上进行的。我们先介绍正则变换及其性质,然后证明在最简单的正则变换下重积分的变量替换公式成立,再利用
一般的正则变换可以局部地分解为最简单的正则变换的复合来证明一般的变量替换公式。这样做,一方面通过运用可以巩固和加强向
量函数微分学的基础;另一方面可以明确认识作变量替换时要满足的条件,灵活而又准确地运用公式计算~$m$~重积分。显然,如果
在讲授这两个变量替换定理之前,先介绍单位分解这一近代分析的重要概念,利用单位分解定理来证明这两个定理,不仅可以简化这
两个定理的证明,而且更接近于流形上的积分的处理方法,只是考虑到教学大纲的要求和学时的限制,我们没有这样做。

按照教学大纲的要求,我们不讲流形及流形上的微积分,仍按传统,只讲~$\FR^3$~中的曲线积分和曲面积分,但是场论在数
学分析课中是十分重要的内容,需要加强。我们在从物理意义引进场论的散度和旋度概念时作了比较深入细致的分析。为了突出
~Gauss~公式和~Stokes~公式的物理意义,这两个公式都放在场论中讨论,都是先从直观引出,再作严格的证明。考虑到~$\FR^2$~
中~Green~公式的证明是比较典型的,这一公式又是~Stokes~公式证明的基础,我们在曲线积分这一章(第二十一章)里先讲这一公
式,让学生预先熟悉这一公式的证明和应用。

本书在加强多元微积分理论的同时,对多元微积分的计算也给予高度的重视,选取了一定数量的例题,细致分析解题的典型方法和技
巧,作为示范。

考虑到一些不学流形上的微积分的读者阅读近代文献资料的需要,我们特编写了《微分形式和~Stokes~公式》作为附录。

使用本书进行教学时,可以根据学时和学生的水平灵活掌握要求,根据实际情况对内容进行删节和改变讲法。例如,凡是有~$\ast$~
标记的章节可以略去;对于隐函数存在定理的证明,可以重点讲由一个式子确定的隐函数;对于重积分变量替换定理,可以只讲二重
积分替换公式的证明;对于重积分的计算,可以只限于二、三重积分等等。

本书的初稿,第十五章至第十九章由李正元编写,第二十章至第二十三章由廖可人编写。附录的初稿由方企勤提供,廖可人改编和补
充。

方企勤副教授除了提供附录的初稿之外,还对本书正文作了修改,沈燮昌教授具体负责本书编写的组织和审定工作,对本书提过不少
宝贵意见,欧阳光中副教授和董延闿教授在审阅本书时也提出许多宝贵意见,高等教育出版社的责任编辑文小西同志为本书的出版做
了许多深入细致的工作。我们在此谨向他们表示衷心的谢意。

由于编者水平所限,书中难免有不妥或错误之处,敬请读者赐教指正。
\Sign{1985~年~9~月}
\end{preface}

\endinput
%%
%% End of file `MAPreface-III.tex'.