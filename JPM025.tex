%# -*- coding:utf-8 -*-
%%%%%%%%%%%%%%%%%%%%%%%%%%%%%%%%%%%%%%%%%%%%%%%%%%%%%%%%%%%%%%%%%%%%%%%%%%%%%%%%%%%%%


\chapter{吴月娘春昼秋千\KG 来旺儿醉中谤仙}


词曰:

\[
蹴罢秋千,起来整顿纤纤手。露浓花瘦,薄汗轻衣透。见客入来,袜刬金钗溜。和羞走,倚门回首,却把青梅嗅。
\]

话说灯节已过,又早清明将至。西门庆有应伯爵早来邀请,说孙寡嘴作东,邀了郊外耍子去了。

先是吴月娘花园中,扎了一架秋千。这日见西门庆不在家,闲中率众姊妹游戏,以消春困。先是月娘与孟玉楼打了一回,下来教李娇儿和潘金莲打。李娇儿辞说身体沉重,打不的,却教李瓶儿和金莲打。打了一回,玉楼便叫:“六姐过来,我和你两个打个立秋千。”分咐:“休要笑。”当下两个玉手挽定彩绳,将身立于画板之上。月娘却教蕙莲、春梅两个相送。正是:

\[
红粉面对红粉面,玉酥肩并玉酥肩。
两双玉腕挽复挽,四只金莲颠倒颠。
\]

那金莲在上面笑成一块。月娘道:“六姐你在上头笑不打紧,只怕一时滑倒,不是耍处。”说着,不想那画板滑,又是高底鞋,跐不牢,只听得滑浪一声把金莲擦下来,早是扶住架子不曾跌着,险些没把玉楼也拖下来。月娘道:“我说六姐笑的不好,只当跌下来。”因望李娇儿众人说道:“这打秋千,最不该笑。笑多了,一定腿软了,跌下来。咱在家做女儿时,隔壁周台官家花园中扎着一座秋千。也是三月佳节,一日他家周小姐和俺一般三四个女孩儿,都打秋千耍子,也是这等笑的不了,把周小姐滑下来,骑在画板上,把身子喜抓去了。落后嫁与人家,被人家说不是女儿,休逐来家,今后打秋千,先要忌笑。”金莲道:“孟三儿不济,等我和李大姐打个立秋千。”月娘道:“你两个仔细打。”却教玉箫、春梅在旁推送。才待打时,只见陈敬济自外来,说道:“你每在这里打秋千哩。”月娘道:“姐夫来的正好,且来替你二位娘送送儿。丫头每气力少。”这敬济老和尚不撞钟——得不的一声,于是拨步撩衣,向前说:“等我送二位娘。”先把金莲裙子带住,说道:“五娘站牢,儿子送也。”那秋千飞在半空中,犹若飞仙相似。李瓶儿见秋千起去了,唬的上面怪叫道:“不好了,姐夫你也来送我送儿。”敬济道:“你老人家到且性急,也等我慢慢儿的打发将来。这里叫,那里叫,把儿子手脚都弄慌了。”于是把李瓶儿裙子掀起,露着他大红底衣,推了一把。李瓶儿道:“姐夫,慢慢着些!我腿软了!”敬济道:“你老人家原来吃不得紧酒。”金莲又说:“李大姐,把我裙子又兜住了。”两个打到半中腰里,都下来了。却是春梅和西门大姐两个打了一回。然后,教玉箫和蕙莲两个打立秋千。这蕙莲手挽彩绳,身子站的直屡屡的,脚跐定下边画板,也不用人推送,那秋千飞在半天云里,然后忽地飞将下来,端的却是飞仙一般,甚可人爱。月娘看见,对玉楼、李瓶儿说:“你看媳妇子,他倒会打。”这里月娘众人打秋千不题。

话分两头。却表来旺儿往杭州织造蔡太师生辰衣服回来,押着许多驮垛箱笼船上,先走来家。到门首,下了头口,收卸了行李,进到后边。只见雪娥正在堂屋门首,作了揖。那雪娥满面微笑,说道:“好呀,你来家了。路上风霜,多有辛苦!几时没见,吃得黑胖了。”来旺因问:“爹娘在那里?”雪娥道:“你爹今日被应二众人,邀去门外耍子去了。你大娘和大姐,都在花园中打秋千哩。”来旺儿道:“啊呀,打他则甚?”雪娥便倒了一盏茶与他吃,因问:“媳妇子在灶上,怎的不见?”那雪娥冷笑了一声,说道:“你的媳妇子,如今还是那时的媳妇儿哩?好不大了!他每日只跟着他娘每伙儿里下棋,挝子儿,抹牌顽耍。他肯在灶上做活哩!”正说着,小玉走到花园中,报与月娘。月娘自前边走来,来旺儿向前磕了头,立在旁边。问了些路上往回的话,月娘赏了两瓶酒。吃一回,他媳妇宋蕙莲来到。月娘道:“也罢,你辛苦了,且往房里洗洗头面,歇宿歇宿去。等你爹来,好见你爹回话。”那来旺儿便归房里。蕙莲先付钥匙开了门,又舀些水与他洗脸摊尘,收拾褡裢去,说道:“贼黑囚,几时没见,便吃得这等肥肥的。”又替他换了衣裳,安排饭食与他吃。睡了一觉起来,已是日西时分。

西门庆来家,来旺儿走到跟前参见,说道:“杭州织造蔡太师生辰的尺头并家中衣服,俱已完备,打成包裹,装了四箱,搭在官船上来家,只少雇夫过税。”西门庆满心欢喜,与了他赶脚银两,明日早装载进城。又赏银五两,房中盘缠;又教他管买办东西。这来旺儿私已带了些人事,悄悄送了孙雪娥两方绫汗巾,两只装花膝裤,四匣杭州粉,二十个胭脂。雪娥背地告诉来旺儿说:“自从你去了四个月,你媳妇怎的和西门庆勾搭,玉箫怎的做牵头,金莲屋里怎的做窝窠。先在山子底下,落后在屋里,成日明睡到夜,夜睡到明。与他的衣服、首饰、花翠、银钱,大包带在身边。使小厮在门首买东西,见一日也使二三钱银子。”来旺道:“怪道箱子里放着衣服、首饰!我问他,他说娘与他的。”雪娥道:“那娘与他?到是爷与他的哩!”这来旺儿遂听记在心。

到晚夕,吃了几锺酒,归到房中。常言酒发顿腹之言,因开箱子,看见一匹蓝缎子,甚是花样奇异,便问老婆:“是那里的缎子?谁人与你的?趁上实说。”老婆不知就里,故意笑着,回道:“怪贼囚,问怎的?此是后边见我没个袄儿,与了这匹缎子,放在箱中,没工夫做。端的谁肯与我?”来旺儿骂道:“贼淫妇!还捣鬼哩!端的是那个与你的?”又问:“这些首饰是那里的?”妇人道:“呸!怪囚根子,那个没个娘老子,就是石头罅剌儿里迸出来,也有个窝巢儿,为人就没个亲戚六眷?此是我姨娘家借来的钗梳。是谁与我的!”被来旺儿一拳,险不打了一交,说:“贼淫妇,还说嘴哩!有人亲看见你和那没人伦的猪狗有首尾!玉箫丫头怎的牵头,送缎子与你,在前边花园内两个干,落后吊在潘家那淫妇屋里明干,成日\textuni{34B2}的不值了。贼淫妇,你还要我手里吊子曰儿。”那妇人便大哭起来,说道:“贼不逢好死的囚根子!你做甚么来家打我?我干坏了你甚么事来?你恁是言不是语,丢块砖瓦儿也要个下落。是那个嚼舌根的,没空生有,调唆你来欺负老娘?我老娘不是那没根基的货!教人就欺负死,也拣个干净地方。你问声儿,宋家的丫头,若把脚略趄儿,把‘宋’字儿倒过来!你这贼囚根子,得不个风儿就雨儿。万物也要个实。人教你杀那个人,你就杀那个人?”几句说的来旺儿不言语了。妇人又道:“这匹蓝缎子,越发我和你说了罢,也是去年十一月里三娘生日,娘见我上穿着紫袄,下边借了玉箫的裙子穿着,说道:‘媳妇子怪剌剌的,甚么样子?’才与了我这匹缎子。谁得闲做他?那个是不知道!就纂我恁一遍舌头。你错认了老娘,老娘不是个饶人的。明日我咒骂个样儿与他听。破着我一条性命,自恁寻不着主儿哩。”来旺儿道:“你既没此事,平白和人合甚气?快些打铺我睡。”这妇人一面把铺伸下,说道:“怪倒路的囚根子,噇了那黄汤,挺你那觉!平白惹老娘骂。”把来旺掠翻在炕上,鼾声如雷。看官听说:但凡世上养汉的婆娘,饶他男子汉十八分精细,吃他几句左话儿右说,十个九个都着了道儿。正是:东净里砖儿——又臭又硬。

这宋蕙莲窝盘住来旺儿,过了一宿。到次日,往后边问玉箫,谁人透露此事,终莫知其所由,只顾海骂。一日,月娘使小玉叫雪娥,一地里寻不着。走到前边,只见雪娥从来旺儿房里出来,只猜和他媳妇说话,不想走到厨下,蕙莲又在里面切肉,良久,西门庆前边陪着乔大户说话,只为扬州盐商王四峰,被按抚使送监在狱中,许银二千两,央西门庆对蔡太师讨人情释放。刚打发大户去了,西门庆叫来旺,来旺从他屋里跑出来。正是:

\[
雪隐鹭莺飞始见,柳藏鹦鹉语方知。
\]
以此都知雪娥与来旺儿有尾首。

一日,来旺儿吃醉了,和一般家人小厮在前边恨骂西门庆,说怎的我不在家,使玉箫丫头拿一匹蓝缎子,在房里哄我老婆。把他吊在花园奸耍,后来潘金莲怎的做窝主:“由他,只休要撞到我手里。我教他白刀子进去,红刀子出来。好不好,把潘家那淫妇也杀了,也只是个死。你看我说出来做的出来。潘家那淫妇,想着他在家摆死了他汉子武大,他小叔武松来告状,多亏了谁替他上东京打点,把武松垫发充军去了?今日两脚踏住平川路,落得他受用,还挑拨我的老婆养汉。我的仇恨,与他结的有天来大。常言道,一不做,二不休,到跟前再说话。破着一命剐,便把皇帝打!”这来旺儿自知路上说话,不知草里有人,不想被同行家人来兴儿听见。这来兴儿在家,西门庆原派他买办食用撰钱过日,只因与来旺媳妇勾搭,把买办夺了,却教来旺儿管领。来兴儿就与来旺不睦,听见发此言语,就悄悄走来潘金莲房里告诉。

金莲正和孟玉楼一处坐的,只见来兴儿掀帘子进来,金莲便问来兴儿:“你来有甚事?你爹今日往谁家吃酒去了?”来兴道:“今日俺爹和应二爹往门外送殡去了。适有一件事,告诉老人家,只放在心里,休说是小的来说。”金莲道:“你有甚事,只顾说,不妨事!”来兴儿道:“别无甚事,叵耐来旺儿,昨日不知那里吃的醉稀稀的,在前边大吆小喝,指猪骂狗,骂了一日。又逻着小的厮打,小的走来一边不理,他对着家中大小,又骂爹和五娘。”潘金莲就问:“贼囚根子,骂我怎的?”来兴说:“小的不敢说。三娘在这里,也不是别人。那厮说爹怎的打发他不在家,耍了他的老婆,说五娘怎的做窝主,赚他老婆在房里和爹两个明睡到夜,夜睡到明。他打下刀子,要杀爹和五娘,白刀子进去,红刀子出来。又说,五娘那咱在家,毒药摆杀了亲夫,多亏了他上东京去打点,救了五娘一命。说五娘恩将仇报,挑拨他老婆养汉。小的穿青衣抱黑住,先来告诉五娘说声,早晚休吃那厮暗算。”玉楼听了,如提在冷水盆内一般,吃了一惊。这金莲不听便罢,听了,粉面通红,银牙咬碎,骂道:“这犯死的奴才!我与他往日无冤近日无仇,他主子要了他的老婆,他怎的缠我?我若教这奴才在西门庆家,永不算老婆!怎的我亏他救活了性命?”因分咐来兴儿:“你且去,等你爹来家问你时,你也只照恁般说。”来兴儿说:“五娘说那里话!小的又不赖他,有一句说一句。随爹怎的问,也只是这等说。”说毕,往前边去了。

玉楼便问金莲:“真个他爹和这媳妇子有?”金莲道:“你问那没廉耻的货!甚的好老婆,也不枉了教奴才这般挟制了。在人家使过了的奴才淫妇,当初在蔡通判家,和大婆作弊养汉,坏了事,才打发出来,嫁了蒋聪。岂止见过一个汉子儿?有一拿小米数儿,甚么事儿不知道!贼强人瞒神吓鬼,使玉箫送缎子儿与他做袄儿穿。一冬里,我要告诉你,没告诉你。那一日,大姐姐往乔大户家吃酒,咱每都不在前边下棋?只见丫头说他爹来家,咱每不散了?落后我走到后边仪门首,见小玉立在穿廊下,我问他,小玉望着我摇手儿。我刚走到花园前,只见玉箫那狗肉在角门首站立,原来替他观风。我还不知,教我径往花园里走。玉箫拦着我,不教我进去,说爹在里面。教我骂了两句。我到疑影和他有些甚么查子帐,不想走到里面,他和媳妇子在山洞里干营生。媳妇子见我进去,把脸飞红的走出来了。他爹见了我,讪讪的,吃我骂了两句没廉耻。落后媳妇子走到屋里,打旋磨跪着我,教我休对他娘说。落后正月里,他爹要把淫妇安托在我屋里过一夜儿,吃我和春梅折了两句,再几时容他傍个影儿!贼万杀的奴才,没的把我扯在里头。好娇态的奴才淫妇,我肯容他在那屋里头弄碜儿?就是我罢了,俺春梅那小肉儿,他也不肯容他。”玉楼道:“嗔道贼臭肉在那里坐着,见了俺每意意似似,待起不起的,谁知原来背地有这本帐!论起来,他爹也不该要他。那里寻不出老婆来,教奴才在外边倡扬,甚么样子?”金莲道:“左右的皮靴儿没番正,你要奴才老婆,奴才暗地里偷你的小娘子,彼此换着做!贼小妇奴才,千也嘴头子嚼说人,万也嚼说,今日打了嘴,也不说的!”玉楼向金莲道:“这椿事,咱对他爹说好,不说好?大姐姐又不管。倘忽那厮真个安心,咱每不言语,他爹又不知道,一时遭了他手怎了?六姐,你还该说说。”金莲道:“我若是饶了这奴才,除非是他\textuni{34B2}出我来。”正是:

\[
平生不作皱眉事,世上应无切齿人。
\]

西门庆至晚来家,只见金莲在房中云鬟不整,睡揾香腮,哭的眼坏坏的。问其所以,遂把来旺儿醉酒发言,要杀主之事诉说一遍:“见有来兴儿亲自听见,思想起来,你背地图他老婆,他便背地要你家小娘子。你的皮靴儿没番正。那厮杀你便该当,与我何干?连我一例也要杀!趁早不为之计,夜头早晚,人无后眼,只怕暗遭他毒手。”西门庆因问:“谁和那厮有首尾?”金莲道:“你休来问我,只问小玉便知。”又说:“这奴才欺负我,不是一遭儿了。说我当初怎的用药摆杀汉子,你娶了我来,亏他寻人情搭救我性命来。在外边对人揭条。早是奴没生下儿没长下女,若是生下儿女,教贼奴才揭条着好听?敢说:‘你家娘当初在家不得地时,也亏我寻人情救了他性命。’恁说在你脸上也无光了!你便没羞耻,我却成不的,要这命做甚么?”西门庆听了妇人之言,走到前边,叫将来兴儿到无人处,问他始末缘由。这小厮一五一十说了一遍。又走到后边,摘问了小玉口词,与金莲所说无差:委的某日,亲眼看见雪娥从来旺儿屋里出来,他媳妇儿不在屋里,的有此事。这西门庆心中大怒,把孙雪娥打了一顿,被月娘再三劝了,拘了他头面衣服,只教他伴着家人媳妇上灶,不许他见人。此事表过不题。

西门庆在后边,因使玉箫叫了宋蕙莲,背地亲自问他。这婆娘便道:“啊呀,爹,你老人家没的说,他是没有这个话。我就替他赌了大誓。他酒便吃两锺,敢恁七个头八个胆,背地里骂爹?又吃纣王水土,又说纣王无道!他靠那里过日子?爹,你不要听人言语。我且问爹,听见谁说这个话来?”那西门庆被婆娘一席话儿,闭口无言。问的急了,说:“是来兴儿告诉我说的。”蕙莲道:“来兴儿因爹叫俺这一个买办,说俺每夺了他的,不得赚些钱使,结下这仇恨儿,平空拿这血口喷他,爹就信了。他有这个欺心的事,我也不饶他。爹你依我,不要教他在家里,与他几两银子本钱,教他信信脱脱,远离他乡,做买卖去。他出去了,早晚爹和我说句话儿也方便些。”西门庆听了满心欢喜,说道:“我的儿,说的是。我有心要叫他上东京,与盐商王四峰央蔡太师人情,回来,还要押送生辰担去,只因他才从杭州来家,不好又使他的,打帐叫来保去。既你这样说,我明日打发他去便了。回来,我教他领一千两银子,同主管往杭州贩买绸绢丝线做买卖。你意下如何?”老婆心中大喜,说道:“爹若这等才好。”正说着,西门庆见无人,就搂他过来亲嘴。婆娘忙递舌头在他口里,两个咂做一处。妇人道:“爹,你许我编\textuni{4BFC}髻,怎的还不替我编?恁时候不戴到几时戴?只教我成日戴这头发壳子儿?”西门庆道:“不打紧,到明日将八两银子,往银匠家替你拔丝去。”西门庆又道:“怕你大娘问,怎生回答?”妇人道:“不打紧,我自有话打发他,只说问我姨娘家借来戴戴,怕怎的?”当下二人说了一回话,各自分散了。

到了次日,西门庆在厅上坐着,叫过来旺儿来:“你收拾衣服行李,赶明日三月二十八日起身,往东京央蔡太师人情。回来,我还打发你杭州做买卖去。”这来旺心中大喜,应诺下来,回房收拾行李,在外买人事。来兴儿打听得知,就来告报金莲知道。金莲打听西门庆在花园卷棚内,走到那里,不见西门庆,只见陈敬济在那里封礼物。金莲便道:“你爹在那里?你封的是甚么?”敬济道:“爹刚才在这里,往大娘那边兑盐商王四峰银子去了。我封的是往东京央蔡太师的礼。”金莲问:“打发谁去?”敬济道:“我听见昨日爹分咐来旺儿去。”这金莲才待下台基,往花园那条路上走,正撞见西门庆拿了银子来。叫到屋里,问他:“明日打发谁往东京去?”西门庆道:“来旺儿和吴主管二人同去。因有盐商王四峰一千干事的银两,以此多着两个去。”妇人道:“随你心下,我说的话儿你不依,到听那奴才淫妇一面儿言语。他随问怎的,只护他的汉子。那奴才有话在先,不是一日儿了。左右破着老婆丢与你,坑了你这银子,拐的往那头里停停脱脱去了,看哥哥两眼儿空哩。你的白丢了罢了,难为人家一千两银子,不怕你不赔他。我说在你心里,也随你。老婆无故只是为他。不争你贪他这老婆,你留他在家里也不好,你就打发他出去做买卖也不好。你留他在家里,早晚没这些眼防范他。你打发他外边去,他使了你本钱,头一件你先说不得他。你若要他这奴才老婆,不如先把奴才打发他离门离户。常言道:剪草不除根,萌芽依旧生;剪草若除根,萌芽再不生。就是你也不耽心,老婆他也死心塌地。”一席话儿,说得西门庆如醉方醒。正是:

\[
数语拨开君子路,片言提醒梦中人。
\]
