%# -*- coding:utf-8 -*-
%%%%%%%%%%%%%%%%%%%%%%%%%%%%%%%%%%%%%%%%%%%%%%%%%%%%%%%%%%%%%%%%%%%%%%%%%%%%%%%%%%%%%


\chapter{雪娥唆打陈敬济\KG 金莲解渴王潮儿}


诗曰:

\[
雨打梨花倍寂寥,几回肠断泪珠抛。
睽违一载犹三载,情绪千丝与万条。
好句每从秋里得,离魂多自梦中消。
香罗重解知何日,辜负巫山几暮朝。
\]

话说潘金莲自从春梅去后,房中纳闷,不题。单表陈敬济,次日上饭时出去,假作讨帐,骑头口到于薛嫂儿家。薛嫂儿正在屋里,一面让进来坐。敬济拴了头口,进房坐下,点茶吃了。薛嫂故意问:“姐夫来有何话说?”敬济道:“我往前街讨帐,竟到这里。昨晚大小姐出来了,和他说句话儿。”薛嫂故作乔张致,说:“好姐夫,昨日你家丈母好不分付我,因为你每通同作弊,弄出丑事来,才把他打发出门,教我防范你们,休要与他会面说话。你还不趁早去哩,只怕他一时使将小厮来看见,到家学了,又是一场儿。倒没的弄的我也上不的门。”那敬济便笑嘻嘻袖中拿出一两银子来:“权作一茶,你且收了,改日还谢你。”那薛嫂见钱眼开,便道:“好姐夫,自恁没钱使,将来谢我!只是我去年腊月,你铺子当了人家两付扣花枕顶,将有一年来,本利该八钱银子,你寻与我罢。”敬济道:“这个不打紧,明日就寻与你。”

这薛嫂儿一面请敬济里间房里去,与春梅厮见,一面叫他媳妇金大姐定菜儿,“我去买茶食点心。”又打了一壶酒,并肉鲊之类,教他二人吃。这春梅看见敬济,说道:“姐夫,你好人儿,就是个弄人的刽子手!把俺娘儿两个弄的上不上下不下,出丑惹人嫌,到这步田地。”敬济道:“我的姐姐,你既出了他家门,我在他家也不久了。‘妻儿赵迎春,各自寻投奔’。你教薛妈妈替你寻个好人家去罢,我‘腌韭菜——已是入不的畦”了。我往东京俺父亲那里去计较了回来,把他家女儿休了,只要我家寄放的箱子。”说毕,不一时,薛嫂买将茶食酒菜来,放炕桌儿摆了,两个做一处饮酒叙话。薛嫂也陪他吃了两盏,一递一句,说了回月娘心狠:“宅里恁个出色姐儿出来,通不与一件儿衣服簪环。就是往人家上主儿去,装门面也不好看。还要旧时原价。就是清水,这碗里倾倒那碗内,也抛撒些儿。原来这等夹脑风。临时出门,倒亏了小玉丫头做了个分上,教他娘拿了两件衣服与他。不是,往人家相去,拿甚么做上盖?”比及吃得酒浓时,薛嫂教他媳妇金大姐抱孩子,躲去人家坐的,教他两个在里间自在坐个房儿。正是:

\[
云淡淡天边鸾凤,水沉沉波底鸳鸯。
写成今世不休书,结下来生欢喜带。
\]
两个干讫,一度作别,比时难割难舍。薛嫂恐怕月娘使人来瞧,连忙撺掇敬济出港,骑上头口来家。

迟不上两日,敬济又稍了两方销金汗巾,两双膝裤与春梅,又寻枕头出来与薛嫂儿。又拿银子打酒,在薛嫂儿房内正和春梅吃酒,不想月娘使了来安小厮来催薛嫂儿:“怎的还不上主儿?”看见头口拴在门首,来安儿到家学了舌,说:“姐夫也在那里来。”月娘听了,心中大怒,使人一替两替叫了薛嫂儿去,尽力数说了一遍,道:“你领了奴才去,今日推明日,明日推后日,只顾不上紧替我打发,好窝藏着养汉挣钱儿与你家使。若是你不打发,把丫头还与我领了来,我另教冯妈妈子卖,你再休上我门来。”这薛嫂儿听了,到底还是媒人的嘴,说道:“天么天么!你老人家怪我差了。我赶着增福神着棍打?你老人家照顾我,怎不打发?昨日也领着走了两三个主儿,都出不上,你老人家要十六两原价,俺媒人家那里有这些银子陪上。”月娘又道:“小厮说陈家种子今日在你家和丫头吃酒来。”薛嫂慌道:“耶嚛!耶嚛!又是一场儿。还是去年腊月,当了人家两付枕顶,在咱狮子街铺内,银子收了,今日姐夫送枕顶与我。我让他吃茶,他不吃,忙忙就上头口来了。几时进屋里吃酒来!原来咱家这大官儿,恁快捣谎驾舌!”月娘吃他一篇,说的不言语了,说道:“我只怕一时被那种子设念随邪,差了念头。”薛嫂道:“我是三岁小孩儿?岂可恁些事儿不知道。你那等分付了我,我长吃好,短吃好?他在那里也没的久停久坐,与了我枕头,茶也没吃就来了。几曾见咱家小大姐面儿来!万物也要个真实,你老人家就上落我起来。既是如此,如今守备周老爷府中,要他图生长,只出十二两银子。看他若添到十三两上,我兑了银子来罢。说起来,守备老爷前者在咱家酒席上,也曾见过小大姐来。因他会这几套唱,好模样儿,才出这几两银子。又不是女儿,其余别人出不上。”薛嫂当下和月娘砸死了价钱。

次日,早把春梅收拾打扮,妆点起来,戴着围发云髻儿,满头珠翠,穿上红段袄儿,蓝段裙子,脚上双鸾尖翘翘,一顶轿子送到守备府中。周守备见了春梅生的模样儿,比旧时越又红又白,身段儿不短不长,一双小脚儿,满心欢喜,就兑出五十两一锭元宝来,这薛嫂儿拿出家,凿下十三两银子,往西门庆家交与月娘,另外又拿出一两来,说:“是周爷赏我的喜钱,你老人家这边不与我些儿?”那吴月娘免不过,只得又秤出五钱银子与他,恰好他还禁了三十七两五钱银子。十个九个媒人,都是如此赚钱养家。

却表陈敬济见卖了春梅,又不得往金莲那边去,见月娘凡事不理他,门户都严禁,到晚夕亲自出来,打灯笼前后照看,上了锁,方才睡去,因此弄不得手脚。敬济十分急了,先和西门大姐嚷了两场,淫妇前淫妇后骂大姐:“我在你家做女婿,不道的雌饭吃,吃伤了!你家收了我许多金银箱笼,你是我老婆,不顾赡我,反说我雌你家饭吃!我白吃你家饭来?”骂的大姐只是哭涕。

十一月念七日,孟玉楼生日。玉楼安排了几碗酒菜点心,好意教春鸿拿出前边铺子,教敬济陪傅伙计吃。月娘便拦说:“他不是才料。休要理他。要与傅伙计,自与傅伙计自家吃就是了,不消叫他。”玉楼不肯。春鸿拿出来,摆在水柜上。一大壶酒都吃了,不勾,又使来巡儿后边要去。傅伙计便说:“姐夫不消要酒去了,这酒勾了,我也不吃了。”敬济不肯,定要来安要去。等了半晌,来安儿出来,回说没了酒了。这陈敬济也有半酣酒儿在肚内,又使他要去,那来安不动。又另拿钱,打了酒来吃着。骂来安儿:“贼小奴才儿,你别要慌!你主子不待见我,连你这奴才每也欺负我起来了,使你使儿不动。我与你家做女婿,不道的酒肉吃伤了,有爹在怎么行来?今日爹没了,就改变了心肠,把我来不理,都乱来挤撮我。我大丈母听信奴才言语,凡事托奴才,不托我。由他,我好耐凉耐怕儿!”傅伙计劝道:“好姐夫,快休舒言。不敬奉姐夫,再敬奉谁?想必后边忙。怎不与姐夫吃?你骂他不打紧,墙有缝,壁有耳,恰似你醉了一般。”敬济道:“老伙计,你不知道,我酒在肚里,事在心头。俺丈母听信小人言语,骂我一篇是非。就算我\textuni{34B2}了人,人没\textuni{34B2}了我?好不好我把这一屋子里老婆都刮剌了,到官也只是后丈母通奸,论个不应罪名。如今我先把你家女儿休了,然后一纸状子告到官。再不,东京万寿门进一本,你家见收着我家许多金银箱笼,都是杨戬应没官赃物。好不好把你这几间业房子都抄没了,老婆便当官办卖。我不图打鱼,只图混水耍子。会事的把俺女婿收笼着,照旧看待,还是大家便益。”傅伙计见他话头儿来的不好,说道:“姐夫,你原来醉了。王十九,只吃酒,且把散话革起。”这敬济眼瞅着傅伙计,骂道:“老贼狗,怎的说我散话!揭跳我醉了,吃了你家酒来?我不才是他家女婿娇客,你无故只是他家行财,你也挤撮我起来!我教你这老狗别要慌,你这几年赚的俺丈人钱勾了,饭也吃饱了,心里要打伙儿把我疾发了去,要夺权儿做买卖,好禁钱养家。我明日本状也带你一笔。教他打官司!”那傅伙计最是个小胆儿的人,见头势不好,穿上衣裳,悄悄往家一溜烟走了。小厮收了家活,后边去了,敬济倒在炕上睡下,一宿晚景题过。

次日,傅伙计早辰进后边,见月娘把前事具诉一遍,哭哭啼啼,要告辞家去,交割帐目,不做买卖了。月娘便劝道:“伙计,你只安心做买卖,休要理那泼才料,如臭屎一般丢着他。当初你家为官事投到俺家来权住着,有甚金银财宝?也只是大姐几件妆奁,随身箱笼。你家老子便躲上东京去了,那时恐怕小人不足,教俺家昼夜耽心。你来时才十六七岁,黄毛团儿也一般。也亏在丈人家养活了这几年,调理的诸般买卖儿都会。今日翅膀毛儿干了,反恩将仇报,一扫帚扫的光光的。小孩儿家说话欺心,恁没天理,到明日只天照看他!伙计,你自安心做你买卖,休理他便了。他自然也羞。”一面把傅伙计安抚住了不题。

一日,也是合当有事,印了铺挤着一屋里人赎讨东西。只见奶子如意儿,抱着孝哥儿送了一壶茶来与傅伙计吃,放在桌上。孝哥儿在奶子怀里,哇哇的只管哭。这陈敬济对着那些人,作耍当真说道:“我的哥哥,乖乖儿,你休哭了。”向众人说:“这孩子倒相我养的,依我说话,教他休哭,他就不哭了。”那些人就呆了。如意儿说:“姐夫,你说的好妙话儿,越发叫起儿来了,看我进房里说不说。”这陈敬济赶上踢了奶子两脚,戏骂道:“怪贼邋遢,你说不是!我且踢个响屁股儿着。”那奶子抱孩子走到后边,如此这般向月娘哭说:“姐夫对众人将哥儿这般言语发出来。”这月娘不听便罢,听了此言,正在镜台边梳着头,半日说不出话来,往前一撞,就昏倒在地,不省人事。但见:

\[
荆山玉损,可惜西门庆正室夫妻;宝鉴花残,枉费九十日东君匹配。花容掩淡,犹如西园芍药倚朱栏;檀口无言,一似南海观音来入定。小园昨日春风急,吹折江梅就地花。
\]
慌了小玉,叫将家中大小,扶起月娘来炕上坐的。孙雪娥跳上炕,撅救了半日,舀姜汤灌下去,半日苏醒过来。月娘气堵心胸,只是哽咽,哭不出声来。奶子如意儿对孟玉楼、孙雪娥,将敬济对众人将哥儿戏言之事,说了一遍:“我好意说他,又赶着我踢了两脚,把我也气的发昏在这里。”雪娥扶着月娘,待的众人散去,悄悄在房中对月娘说:“娘也不消生气,气的你有些好歹,越发不好了。这小厮因卖了春梅,不得与潘家那淫妇弄手脚,才发出话来。如今一不做,二不休,大姐已是嫁出女,如同卖出田一般,咱顾不得他这许多。常言养虾蟆得水蛊儿病,只顾教那小厮在家里做甚么!明日哄赚进后边,下老实打与他一顿,即时赶离门,教他家去。然后叫将王妈妈子来,把那淫妇教他领了去,变卖嫁人,如同狗臭尿,掠将出去,一天事都没了。平空留着他在家里做甚么!到明日,没的把咱们也扯下水去了。”月娘道:“你说的也是。”当下计议已定了。

到次日,饭时已后,月娘埋伏了丫鬟媳妇七八个人,各拿短棍棒槌。使小厮来安儿请进陈敬济来后边,只推说话。把仪门关了,教他当面跪下,问他:“你知罪么?”那陈敬济也不跪,转把脸儿高扬,佯佯不采。月娘大怒,于是率领雪娥并来兴儿媳妇、来昭妻一丈青、中秋儿、小玉、绣春众妇人,七手八脚,按在地下,拿棒槌短棍打了一顿。西门大姐走过一边,也不来救。打的这小伙儿急了,把裤子脱了,露出那直竖一条棍来。唬的众妇人看见,却丢下棍棒乱跑了。月娘又是那恼,又是那笑,口里骂道:“好个没根基的王八羔子!”敬济口中不言,心中暗道:“若不是我这个法儿,怎得脱身。”于是扒起来,一手兜着裤子,往前走了。月娘随令小厮跟随,教他算帐,交与傅伙计。敬济自知也立脚不定,一面收拾衣服铺盖,也不作辞,使性儿一直出离西门庆家,径往他母舅张团练家,他旧房子自住去了。正是:

\[
唯有感恩并积恨,万年千载不生尘。
\]

潘金莲在房中,听见打了陈敬济,赶离出门去了,越发忧上加忧,闷上添闷。一日,月娘听信雪娥之言,使玳安儿去叫了王婆来。那王婆自从他儿子王潮跟淮上客人,拐了起车的一百两银子来家,得其发迹,也不卖茶了,买了两个驴儿,安了盘磨,一张罗柜,开起磨房来。听见西门庆宅里叫他,连忙穿衣就走,到路上问玳安说:“我的哥哥,几时没见你,又早笼起头去了,有了媳妇儿不曾?”玳安道:“还不曾有哩。”王婆子道:“你爹没了,你家谁人请我做甚么?莫不是你五娘养了儿子了,请我去抱腰?”玳安道:“俺五娘倒没养儿子,倒养了女婿。俺大娘请你老人家,领他出来嫁人。”王婆子道:“天么,天么,你看么!我说这淫妇,死了你爹,怎守的住。只当狗改不了吃屎,就弄碜儿来了。就是你家大姐那女婿子?他姓甚么?”玳安道:“他姓陈,名唤陈敬济。”王婆子道:“想着去年,我为何老九的事,去央烦你爹。到宅内,你爹不在,贼淫妇他就没留我房里坐坐儿,折针也迸不出个来,只叫丫头倒一钟清茶我吃了,出来了。我只道千年万岁在他家,如何今日也还出来!好个浪蹄子淫妇,休说我是你个媒王,替你作成了恁好人家,就是闲人进去,也不该那等大意。”玳安道:“为他和俺姐夫在家里炒嚷作乱,昨日差些儿没把俺大娘气杀了哩。俺姐夫已是打发出去了,只有他老人家,如今教你领他去哩。”王婆子道:“他原是轿儿来,少不得还叫顶轿子。他也有个箱笼来,这里少不的也与他个箱子儿。”玳安道:“这个少不的,俺大娘自有个处。”

两个说话间,到了门首。进入月娘房里,道了万福坐下,丫鬟拿茶吃了。月娘便道:“老王,无事不请你来。”悉把潘金莲如此这般,上项说了一遍:“今来是是非人,去是非者。一客不烦二王,还起动你领他出去,或聘嫁,或打发,叫他吃自在饭去罢。我男子汉已是没了,招揽不过这些人来。说不的当初死鬼为他丢了许多钱底那话了,就打他恁个人儿也有。如今随你聘嫁,多少儿交得来,我替他爹念个经儿,也是一场勾当。”王婆道:“你老人家,是稀罕这钱的?只要把祸害离了门就是了。我知道,我也不肯差了。”又道:“今日好日,就出去罢。又一件,他当初有个箱笼儿,有顶轿儿来,也少不的与他顶轿儿坐了去。”月娘道:“箱子与他一个,轿子不容他坐。”小玉道:“俺奶奶气头上便是这等说,到临岐,少不的雇顶轿儿。不然街坊人家看着,抛头露面的,不吃人笑话?”月娘不言语了,一面使丫鬟绣春,前边叫金莲来。

这金莲一见王婆子在房里,就睁了,向前道了万福,坐下。王婆子开言便道:“你快收拾了。刚才大娘说,教我今日领你出去哩。”金莲道:“我汉子死了多少时儿,我为下甚么非,作下甚么歹来?如何平空打发我出去?”王婆道:“你休稀里打哄,做哑装聋!自古蛇钻窟窿蛇知道,各人干的事儿,各人心里明。金莲你休呆里撒奸,说长道短,我手里使不的巧语花言,帮闲钻懒。自古没个不散的筵席,出头椽儿先朽烂,人的名儿,树的影儿。苍蝇不钻没缝儿蛋,你休把养汉当饭,我如今要打发你上阳关。”金莲见势头不好,料难久住,便也发话道:“你打人休打脸,骂人休揭短!有势休要使尽了,赶人不可赶上。我在你家做老婆,也不是一日儿,怎听奴才淫妇戳舌,便这样绝情绝义的打发我出去!我去不打紧,只要大家硬气,守到老没个破字儿才好。”当下金莲与月娘乱了一回。月娘到他房中,打点与了他两个箱子,一张抽替桌儿,四套衣服,几件钗梳簪环,一床被褥。其余他穿的鞋脚,都填在箱内。把秋菊叫到后边来,一把锁就把房门锁了。金莲穿上衣服,拜辞月娘,在西门庆灵前大哭了一回。又走到孟玉楼房中,也是姊妹相处一场,一旦分离,两个落了一回眼泪。玉楼瞒着月娘,悄悄与了他一对金碗簪子,一套翠蓝段袄、红裙子,说道:“六姐,奴与你离多会少了,你看个好人家,往前进了罢。自古道,千里长篷,也没个不散的筵席。你若有了人家,使个人来对我说声,奴往那里去,顺便到你那里看你去,也是姐妹情肠。”于是洒泪而别。临出门,小玉送金莲,悄悄与了金莲两根金头簪儿。金莲道:“我的姐姐,你倒有一点人心儿在我。”王婆又早雇人把箱笼桌子抬的先去了。独有玉楼、小玉送金莲到门首,坐了轿子才回。正是:

\[
世上万般哀苦事,无非死别共生离。
\]

却说金莲到王婆家,王婆安插他在里间,晚夕同他一处睡。他儿子王潮儿,也长成一条大汉,笼起头去了,还未有妻室,外间支着床睡。这潘金莲次日依旧打扮,乔眉乔眼在帘下看人。无事坐在炕上,不是描眉画眼,就是弹弄琵琶。王婆不在,就和王潮儿斗叶儿、下棋。那王婆自去扫面,喂养驴子,不去管他。朝来暮去,又把王潮儿刮剌上了。晚间等的王婆子睡着了,妇人推下炕溺尿,走出外间床上,和王潮儿两个干,摇的床子一片响声。被王婆子醒来听见,问那里响。王潮儿道:“是柜底下猫捕老鼠响。”王婆子睡梦中,喃喃呐呐,口里说道:“只因有这些麸面在屋里,引的这扎心的半夜三更耗爆人,不得睡。”良久,又听见动旦,摇的床子格支支响,王婆又问那里响。王潮道:“是猫咬老鼠,钻在炕洞下嚼的响。”婆子侧耳,果然听见猫在炕洞里咬的响,方才不言语了。妇人和小厮干完事,依旧悄悄上炕睡去了。有几句双关,说得这老鼠好:

\[
你身躯儿小,胆儿大,嘴儿尖,忒泼皮。见了人藏藏躲躲,耳边厢叫叫唧唧,搅混人半夜三更不睡。不行正人伦,偏好钻穴隙。更有一桩儿不老实,到底改不的偷馋抹嘴。
\]

有日,陈敬济打听得潘金莲出来,还在王婆家聘嫁,因提着两吊铜钱,走到王婆家来。婆子正在门前扫驴子撒的粪。这敬济向前深深地唱个喏。婆子问道:“哥哥,你做甚么?”敬济道:“请借里边说话。”王婆便让进里面。敬济便道:“动问西门大官人宅内,有一位娘子潘六姐,在此出嫁?”王婆便道:“你是他甚么人?”那敬济嘻嘻笑道:“不瞒你老人家说,我是他兄弟,他是我姐姐。”那王婆子眼上眼下,打量他一回,说:“他有甚兄弟,我不知道,你休哄我。你莫不是他家女婿姓陈的,在此处撞蠓子,我老娘手里放不过。”敬济笑向腰里解下两吊铜钱来,放在面前,说:“这两吊钱权作王奶奶一茶之费,教我且见一面,改日还重谢你老人家。”婆子见钱,越发乔张致起来,便道:“休说谢的话。他家大娘子分付将来,不许教闲杂人来看他。咱放倒身说话,你既要见这雌儿一面,与我五两银子,见两面与我十两。你若娶他,便与我一百两银子,我的十两媒人钱在外。我不管闲帐。你如今两串钱儿,打水不浑的,做甚么?”敬济见这虔婆口硬,不收钱,又向头上拔下一对金头银脚簪子,重五钱,杀鸡扯腿跪在地下,说道:“王奶奶,你且收了,容日再补一两银子来与你,不敢差了。且容我见他一面,说些话儿则个。”那婆子于是收了簪子和钱,分付:“你进去见他,说了话就与我出来。不许你涎眉睁目,只顾坐着。所许那一两头银子,明日就送来与我。”于是掀帘,放敬济进里间。妇人正坐在炕上,看见敬济,便埋怨他道:“你好人儿!弄的我前不着村,后不着店,有上稍,没下稍,出丑惹人嫌。你就影儿也不来看我看儿了。我娘儿们好好的,拆散的你东我西,皆是为谁来?”说着,扯住敬济,只顾哭泣。王婆又嗔哭,恐怕有人听见。敬济道:“我的姐姐,我为你剐皮剐肉,你为我受气耽羞,怎不来看你?昨日到薛嫂儿家,已知春梅卖在守备府里去了,才打听知你出离了他家门,在王奶奶这边聘嫁。今日特来见你一面,和你计议。咱两个恩情难舍,拆散不开,如之奈何?我如今要把他家女儿休了,问他要我家先前寄放金银箱笼。他若不与我,我东京万寿门一本一状进下来,那里他双手奉与我还是迟了。我暗地里假名托姓,一顶轿子娶到你家去,咱两个永远团圆,做上个夫妻,有何不可?”妇人道:“现今王干娘要一百两银子,你有这些银子与他?”敬济道:“如何人这许多?”婆子说道:“你家大丈母说,当初你家爹,为他打个银人儿也还多,定要一百两银子,少一丝毫也成不的。”敬济道:“实不瞒你老人家说,我与六姐打得热了,拆散不开,看你老人家下顾,退下一半儿来,五六十两银子也罢,我往母舅那里典上两三间房子,娶了六姐家去,也是春风一度。你老人家少转些儿罢。”婆子道:“休说五六十两银子,八十两也轮不到你手里了。昨日湖州贩绸绢何官人,出到七十两;大街坊张二官府,如今见在提刑院掌刑,使了两个节级来,出到八十两上,拿着两卦银子来兑,还成不的,都回去了。你这小孩儿家,空口来说空话,倒还敢奚落老娘,老娘不道的吃伤了哩!”当下一直走出街上,大吆喝说:“谁家女婿要娶丈母,还来老娘屋里放屁!”敬济慌了,一手扯进婆子来,双膝跪下央及:“王奶奶噤声,我依王奶奶价值一百两银子罢。争奈我父亲在东京,我明日起身往东京取银子去。”妇人道:“你既为我一场,休与干娘争执,上紧取去,只恐来迟了,别人娶了奴去,就不是你的人了。”敬济道:“我雇头口连夜兼程,多则半月,少则十日就来了。”婆子道:“常言先下米先食饭,我的十两银子在外,休要少了,我先与你说明白着。”敬济道:“这个不必说,恩有重报,不敢有忘。”说毕,敬济作辞出门,到家收拾行李,次日早雇头口,上东京取银子去。此这去,正是:

\[
青龙与白虎同行,吉凶事全然未保。
\]
