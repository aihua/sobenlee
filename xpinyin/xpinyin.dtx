% \iffalse meta-comment
% !TEX program  = XeLaTeX
%<*internal>
\ifnum\strcmp{\fmtname}{plain}=0 \else
  \expandafter\begingroup
\fi
%</internal>
%<*install>
\input docstrip.tex
\keepsilent
\askforoverwritefalse
\preamble

   Copyright (C) 2012 by Qing Lee <sobenlee@gmail.com>
--------------------------------------------------------------------------
   This work may be distributed and/or modified under the
   conditions of the LaTeX Project Public License, either version 1.3
   of this license or (at your option) any later version.
   The latest version of this license is in
     http://www.latex-project.org/lppl.txt
   and version 1.3 or later is part of all distributions of LaTeX
   version 2005/12/01 or later.

   This work has the LPPL maintenance status "maintained".
   The Current Maintainer of this work is Qing Lee.

\endpreamble
\postamble

   This package consists of the file  xpinyin.dtx,
                                      xpinyin-map.cfg,
                and the derived files xpinyin.sty,
                                      xpinyin.pdf and
                                      xpinyin.ins.
\endpostamble
\usedir{tex/latex/xpinyin}
\generate{\file{\jobname.sty}{\from{\jobname.dtx}{package}}}
%</install>
%<install>\endbatchfile
%<*internal>
\usedir{source/latex/xpinyin}
\generate{\file{\jobname.ins}{\from{\jobname.dtx}{install}}}
\ifnum\strcmp{\fmtname}{plain}=0
  \expandafter\endbatchfile
\else
  \expandafter\endgroup
\fi
%</internal>
%
%<*driver|package>
\NeedsTeXFormat{LaTeX2e}
%<*driver>
\ProvidesFile{xpinyin.dtx}
%</driver>
\RequirePackage{expl3}
\GetIdInfo$Id$
          {package for typesetting Chinese phonetic alphabets}
%</driver|package>
%
%<*driver>
\documentclass{l3doc}
\usepackage{xeCJK}
\usepackage{xpinyin}
\usepackage{fvrb-ex}
\usepackage{metalogo}
\hypersetup{pdfstartview=FitH}
\fvset{formatcom=\CJKfixedspacing}
\linespread{1.1}
\addtolength{\voffset}{-5\baselineskip}
\addtolength\textheight{8\baselineskip}
\setmainfont{TeX Gyre Pagella}
\setmonofont{Inconsolata}
\setCJKmainfont[BoldFont=Adobe Heiti Std,ItalicFont=Adobe Kaiti Std]{Adobe Song Std}
\setCJKmonofont{Adobe Kaiti Std}
\xeCJKsetup{PunctStyle=kaiming}
\newfontfamily\PinYinFont{FreeSans}
\xpinyinsetup{font=\PinYinFont,multiple=\color{red}}
\def\MacroFont{\small\normalfont\ttfamily}
\makeatletter
\let\orig@meta\meta
\def\meta#1{\orig@meta{\normalfont\itshape#1}}
\def\TF{true\orvar{}false}
\def\TTF{\defaultvar{true}\orvar{}false}
\def\TFF{true\orvar\defaultvar{false}}
\def\orvar{\char`\|}
\let\defaultvar\textbf
\def\argbrace#1{\char`\{#1\char`\}}
\makeatother
\def\indexname{代码索引}
\IndexPrologue{%
  \section*{\indexname}
  \markboth{\indexname}{\indexname}
  斜体的数字表示对应项说明所在的页码,下划线的数字表示定义所在的代码行号,而直立体的
  数字表示对应项使用时所在的行号。}
\begin{document}
  \DocInput{\jobname.dtx}
\end{document}
%</driver>
% \fi
%
% \GetFileInfo{\jobname.sty}
%
% \title{\bfseries\pkg{xpinyin} 宏包}
% \author{李清\\ \path{sobenlee@gmail.com}}
% \date{\filedate\qquad\fileversion}
% \maketitle
%
% \begin{documentation}
%
% \section{简介}
%
% \pkg{xpinyin} 是一个 \LaTeX 宏包,提供了为汉字自动注音的功能。
%
% \section{基本用法}
%
% \pkg{xpinyin} 只支持采用 |UTF-8| 编码的 \TeX 源文件。如果使用 \LaTeX 或 pdf\LaTeX 的
% 编译方式,则 \pkg{xpinyin} 依赖 \pkg{CJKutf8} 宏包^^A
% \footnote{\url{http://mirrors.ctan.org/language/chinese/CJK/}};
% 如果使用 \XeLaTeX{},则依赖 \pkg{xeCJK} 宏包^^A
% \footnote{\url{http://mirrors.ctan.org/macros/xetex/latex/xecjk/}}。
% \pkg{xpinyin} 不会自动载入 \pkg{CJKutf8} 或 \pkg{xeCJK} 宏包,应该在它之前或之后
% 自行载入。
%
% 需要注意的是,\pkg{xinyin} 缺省将拼音的字体设置为与文档的主字体(\cs{normalfont})相同,
% 所以为了保证声调字母的正确输出,应该选用合适的西文主字体。也可以通过将在下一节介绍的
% \meta{font} 选项来单独设置拼音的字体。
%
% \begin{minipage}[t]{\dimexpr(\linewidth-\parindent-2em)/2\relax}
% \XeLaTeX 下的简单示例:
% \begin{verbatim}[frame=single,gobble=3]
%   \documentclass{article}
%   \usepackage{xeCJK}
%   \usepackage{xpinyin}
%   \setmainfont{CMU Serif}
%   \setCJKmainfont{SimSun}
%
%   \begin{document}
%   \xpinyin*{汉语拼音示例}
%   \end{document}
% \end{verbatim}
% \end{minipage}\qquad
% \begin{minipage}[t]{\dimexpr(\linewidth-\parindent-2em)/2\relax}
% (pdf)\LaTeX 下的简单示例:
% \begin{verbatim}[frame=single,gobble=3]
%   \documentclass{article}
%   \usepackage{CJKutf8}
%   \usepackage{xpinyin}
%   \usepackage[T1]{fontenc}
%   \usepackage{lmodern}
%
%   \begin{document}
%   \begin{CJK}{UTF8}{gbsn}
%   \xpinyin*{汉语拼音示例}
%   \end{CJK}
%   \end{document}
% \end{verbatim}
% \leavevmode\hrule height \dp\strutbox width 0pt \relax
% \end{minipage}
%
% 运行上述示例要求系统安装了设置的字体,源文件用 |UTF-8| 编码保存,使用相应的编译方式。
%
% \pkg{xinyin} 可以与 \pkg{ctex} 宏包或文档类^^A
% \footnote{\url{http://mirrors.ctan.org/language/chinese/ctex/}} 共同使用,使用
% 方式与上面类似。
%
% \section{用户手册}
%
% \begin{function}{pinyinscope}
%   \begin{syntax}
%     \cs{begin}\argbrace{pinyinscope}\oarg{options}
%     .....
%     \cs{end}\argbrace{pinyinscope}
%   \end{syntax}
%   为 \env{pinyinscope} 环境中的汉字自动注音。例如
%   \begin{Example}[frame=single,numbers=left,gobble=5]
%     \begin{pinyinscope}
%     列位看官:你道此书从何而来?说起根由,虽近荒唐,细按则深有趣味。
%     待在下将此来历注明,方使阅者\xpinyin{了}{liao3}然不惑。
%     \end{pinyinscope}
%   \end{Example}
%   可选项 \meta{options} 用于局部设置拼音的格式,将在下面说明。
% \end{function}
%
% \begin{function}{\xpinyin}
%   \begin{syntax}
%     \cs{xpinyin} \oarg{options} \Arg{单个汉字} \Arg{拼音}
%     \cs{xpinyin*} \oarg{options} \Arg{文字}
%   \end{syntax}
%   对于多音字,可以使用 \cs{xpinyin} 为其设置拼音;而 \cs{xpinyin*} 相当于
%   \env{pinyinscope} 环境的命令形式。\cs{xpinyin} 可以在 \env{pinyinscope} 环境和
%   \cs{xpinyin*} 中使用。例如,\\[1ex]
%   \begin{SideBySideExample}[frame=single,numbers=left,xrightmargin=.5\linewidth,gobble=5]
%     \xpinyin{长}{chang2}\\
%     \xpinyin*{甄士隐梦幻识通灵}\\
%     \xpinyin*{\xpinyin{重}{zhong4}要}
%   \end{SideBySideExample}
% \end{function}
%
% \begin{function}{\pinyin}
%   \begin{syntax}
%     \cs{pinyin} \oarg{options} \Arg{拼音}
%   \end{syntax}
%   用于输出拼音,为了输入的方便 \texttt{\"u} 可以用 |v| 代替。例如,\\[1ex]
%   \begin{SideBySideExample}[frame=single,numbers=left,xrightmargin=.5\linewidth,gobble=5]
%     \pinyin{hao3 hao3 xve2 xi2}\\
%     \pinyin{nv3hai2zi}
%   \end{SideBySideExample}
% \end{function}
%
% \begin{function}{\setpinyin}
%   \begin{syntax}
%     \cs{setpinyin} \Arg{汉字} \Arg{拼音}
%   \end{syntax}
%   \pkg{xpinyin} 宏包的拼音数据(\file{xpinyin-map.cfg})来源于 \texttt{Unicode 6.1.0}
%   的 \texttt{Unihan} 数据库\footnotemark 中的 \file{Unihan_Readings.txt} 文件。对于多
%   音字,一般来说这个文件选用的是常用读音。可以使用 \cs{setpinyin} 来设置多音字的首选读音。
% \end{function}
%
% \footnotetext{\url{http://www.unicode.org/Public/UNIDATA/Unihan.zip}}
%
% \begin{function}{\xpinyinsetup}
%   \begin{syntax}
%     \cs{xpinyinsetup} \argbrace{\meta{key1}=\meta{var1}, \meta{key2}=\meta{var2}, ...}
%   \end{syntax}
%   用于在导言区或文档中,设置拼音的格式。目前可以设置的 \meta{key} 如下介绍。
% \end{function}
%
% \begin{function}{ratio}
%   \begin{syntax}
%     ratio = \marg{number}
%   \end{syntax}
%   设置拼音字体大小与当前正文字体大小的比例,缺省值是 |0.4|。
% \end{function}
%
% \begin{function}{vsep}
%   \begin{syntax}
%     vsep = \marg{dimen}
%   \end{syntax}
%   设置拼音的基线与汉字基线的间距,缺省值是 |1 em|。
% \end{function}
%
% \begin{function}{hsep}
%   \begin{syntax}
%     hsep = \marg{skip}
%   \end{syntax}
%   设置注音汉字之间的间距,缺省值与 \cs{CJKglue} 的值相同。为了断行时行末的对齐,设置
%   的 \meta{skip} 最后有一定的弹性。例如
%   \begin{Example}[frame=single,numbers=left,gobble=5]
%     \xpinyin*[ratio={.7},hsep={.5em plus .1em},vsep={1.1em}]{贾雨村风尘怀闺秀}
%   \end{Example}
% \end{function}
%
% \begin{function}{pysep}
%   \begin{syntax}
%     pysep = \marg{glue}
%   \end{syntax}
%   设置 \cs{pinyin} 输出的相邻两个汉语拼音的空白,缺省值是一个空格。
% \end{function}
%
% \begin{function}{font}
%   \begin{syntax}
%     font = \marg{font}
%   \end{syntax}
%   设置拼音的字体,缺省值是 \cs{normalfont},即以正文西文字体相同。为了保证拼音能正确
%   输出,最好选用收字量较大的西文字体。
% \end{function}
%
% \begin{function}{format}
%   \begin{syntax}
%     format = \marg{format}
%   \end{syntax}
%   设置拼音的其它格式,例如颜色等,缺省值为空。
% \end{function}
%
% \begin{function}{multiple}
%   \begin{syntax}
%     multiple = \marg{format}
%   \end{syntax}
%   设置多音字拼音的其它格式,缺省值为空。可以通过这个选项来提醒校正多音字的拼音。例如
%   本文档设置多音字拼音的颜色是红色:
%   \begin{verbatim}[frame=single,gobble=5]
%     \xpinyinsetup{multiple={\color{red}}}
%   \end{verbatim}
% \end{function}
%
% \end{documentation}
%
%
% \begin{implementation}
%
% \section{代码实现}
%
% \iffalse
%<*package>
% \fi
%
%    \begin{macrocode}
\ProvidesExplPackage{\ExplFileName}{\ExplFileDate}{1.0}{\ExplFileDescription}
%    \end{macrocode}
%
%    \begin{macrocode}
\msg_new:nnn  { xpinyin } { no-LuaTeX }
  {
    The~xpinyin~package~is~not~supported~in~LuaTeX.\\\\
    You~must~change~your~typesetting~engine~to\\
    "xelatex"~or~"pdflatex"~or~"latex"~instead~of~"lualatex".
  }
\luatex_if_engine:T { \msg_critical:nn { xpinyin } { no-LuaTeX } }
%    \end{macrocode}
%
%    \begin{macrocode}
\RequirePackage{xparse}
\RequirePackage{l3keys2e}
%    \end{macrocode}
%
% \begin{macro}[internal]{\xpinyin_make_box:nn}
%    \begin{macrocode}
\cs_new_nopar:Nn \xpinyin_make_box:nn
  { \xpinyin_save_CJKsymbol:n {#1} \xpinyin_make_pinyin_box:nn {#1} {#2} }
%    \end{macrocode}
% \end{macro}
%
% \begin{macro}[internal]{\xpinyin_make_pinyin_box:nn}
%    \begin{macrocode}
\cs_new_nopar:Nn \xpinyin_make_pinyin_box:nn
  {
    \hbox_overlap_left:n
      {
        \hbox_set:Nn \l_tmpa_box
          { \xpinyin_CJKsymbol_hook: \xpinyin_save_CJKsymbol:n {#1} }
        \hbox_set:Nn \l_tmpb_box
          {
            \color_group_begin: \color_ensure_current:
            \xpinyin_select_font:
            \l_xpinyin_format_tl
            \clist_if_exist:cT
              { c_xpinyin_multiple_ \xpinyin_CJKsymbol_to_unicode:n {#1} _clist }
              { \l_xpinyin_multiple_tl }
            {#2}
            \color_group_end:
          }
        \dim_compare:nNnT
          { \box_wd:N \l_tmpb_box } > { \box_wd:N \l_tmpa_box + \l_xpinyin_CJKglue_dim }
          {
            \box_resize:Nff \l_tmpb_box
              { \dim_eval:n { \box_wd:N \l_tmpa_box + \l_xpinyin_CJKglue_dim } }
              { \dim_eval:n { \box_ht:N \l_tmpb_box + \box_dp:N \l_tmpb_box } }
          }
        \box_move_up:nn { \l_xpinyin_vsep_tl }
          {
            \hbox_to_wd:nn { \box_wd:N \l_tmpa_box }
              { \tex_hss:D  \box_use_clear:N \l_tmpb_box \tex_hss:D }
          }
      }
    { \xpinyin_CJK_node: }
  }
\cs_generate_variant:Nn \box_resize:Nnn { Nff }
%    \end{macrocode}
% \end{macro}
%
% \begin{macro}[internal]{\xpinyin_CJKsymbol:n}
%    \begin{macrocode}
\cs_new_nopar:Nn \xpinyin_CJKsymbol:n
  { \xpinyin_make_box:nn {#1} { \xpinyin_to_pinyin:n {#1} } }
%    \end{macrocode}
% \end{macro}
%
% \begin{macro}[internal]{\xpinyin_to_pinyin:n}
%    \begin{macrocode}
\cs_new_nopar:Nn \xpinyin_to_pinyin:n
  { \use:c { c_xpinyin_ \xpinyin_CJKsymbol_to_unicode:n {#1} _tl } }
%    \end{macrocode}
% \end{macro}
%
% \begin{macro}{pinyinscope}
%    \begin{macrocode}
\NewDocumentEnvironment { pinyinscope } { O{} }
  {
    \keys_set:nn { xpinyin } {#1}
    \tl_if_empty:NF \l_xpinyin_hsep_tl
      { \cs_set_nopar:Npn \CJKglue { \skip_horizontal:n { \l_xpinyin_hsep_tl } } }
    \settowidth \l_xpinyin_CJKglue_dim { \CJKglue }
    \xpinyin_replace_CJKsymbol:
  }
  { \cs_gset_eq:NN \CJKsymbol \xpinyin_save_CJKsymbol:n }
%    \end{macrocode}
% \end{macro}
%
% \begin{macro}{\xpinyin}
%    \begin{macrocode}
\NewDocumentCommand \xpinyin { s O{} m }
  {
    \IfBooleanTF {#1}
      {
        \group_begin:
        \keys_set:nn { xpinyin } {#2}
        \tl_if_empty:NF \l_xpinyin_hsep_tl
          { \cs_set_nopar:Npn \CJKglue { \skip_horizontal:n { \l_xpinyin_hsep_tl } } }
        \settowidth \l_xpinyin_CJKglue_dim { \CJKglue }
        \xpinyin_replace_CJKsymbol:
        #3
        \group_end:
      }
      {
        \group_begin:
        \keys_set:nn { xpinyin } {#2}
        \settowidth \l_xpinyin_CJKglue_dim { \CJKglue }
        \xpinyin_xpinyin_single_aux:nn {#3}
      }
  }
%    \end{macrocode}
% \end{macro}
%
% \begin{macro}[internal,var]{\l_xpinyin_CJKglue_dim}
%    \begin{macrocode}
\dim_new:N \l_xpinyin_CJKglue_dim
%    \end{macrocode}
% \end{macro}
%
% \begin{macro}[internal]{\xpinyin_xpinyin_single_aux:nn}
%    \begin{macrocode}
\cs_new_nopar:Nn \xpinyin_xpinyin_single_aux:nn
  {
    \xpinyin_xpinyin_single_hook:n
      { \cs_set_eq:NN \xpinyin_save_CJKsymbol:n \use:n }
    \cs_set_eq:NN \xpinyin_CJKsymbol_to_unicode:n \xpinyin_CJKchar_to_unicode:n
    \xpinyin_make_box:nn {#1} { \xpinyin_pinyin:n {#2} }
    \group_end:
  }
%    \end{macrocode}
% \end{macro}
%
% \begin{macro}[internal]{\xpinyin_replace_CJKsymbol_aux:}
%    \begin{macrocode}
\cs_new_nopar:Nn \xpinyin_replace_CJKsymbol_aux:
  {
    \cs_if_eq:NNF \CJKsymbol \xpinyin_CJKsymbol:n
      {
        \cs_set_eq:NN \xpinyin_save_CJKsymbol:n \CJKsymbol
        \cs_set_eq:NN \CJKsymbol \xpinyin_CJKsymbol:n
      }
  }
%    \end{macrocode}
% \end{macro}
%
% \begin{macro}[internal]{\xpinyin_xpinyin_single_hook_aux:n}
%    \begin{macrocode}
\cs_new_nopar:Npn \xpinyin_xpinyin_single_hook_aux:n
  {
    \cs_if_eq:NNTF \CJKsymbol \xpinyin_CJKsymbol:n
      {
        \cs_set_eq:NN \CJKsymbol \xpinyin_save_CJKsymbol:n
        \cs_set_eq:NN \xpinyin_save_CJKsymbol:n \use:n
      }
  }
%    \end{macrocode}
% \end{macro}
%
% \begin{macro}[internal]{\xpinyin_select_font_xetex:}
%    \begin{macrocode}
\cs_new_nopar:Nn \xpinyin_select_font_xetex:
  {
    \cs_if_exist_use:cF { \l_xpinyin_coor_tl }
      {
        \tl_set:Nx \l_xpinyin_current_coor_tl { \l_xpinyin_coor_tl }
        \xpinyin_select_font_aux:
        \int_compare:nNnT { \XeTeXfonttype \tex_font:D } > \c_zero
          {
            \exp_last_unbraced:NNV
            \cs_gset_eq:cN \l_xpinyin_current_coor_tl \font@name
          }
      }
  }
%    \end{macrocode}
% \end{macro}
%
% \begin{macro}[internal]{\xpinyin_select_font_aux:}
%    \begin{macrocode}
\cs_new_nopar:Nn \xpinyin_select_font_aux:
  {
    \dim_set:Nn \l_tmpa_dim { \f@size \p@ }
    \fontsize { \l_xpinyin_ratio_tl \l_tmpa_dim } \c_zero_dim
    \l_xpinyin_font_tl
    \selectfont
  }
%    \end{macrocode}
% \end{macro}
%
% \begin{macro}[internal]{\xpinyin_CJKsymbol_to_unicode_xetex:n}
%    \begin{macrocode}
\cs_new_nopar:Nn \xpinyin_CJKsymbol_to_unicode_xetex:n { \int_to_hexadecimal:n {`#1} }
%    \end{macrocode}
% \end{macro}
%
% \begin{macro}[internal]
% {\xpinyin_CJKsymbol_to_unicode_pdftex:n,\xpinyin_CJKchar_to_unicode_pdftex:n}
%    \begin{macrocode}
\cs_new_nopar:Nn \xpinyin_CJKsymbol_to_unicode_pdftex:n
  { \int_to_hexadecimal:n { \int_from_hexadecimal:V \CJK@plane * "100 + #1 } }
\cs_new_nopar:Nn \xpinyin_CJKchar_to_unicode_pdftex:n
  { \int_to_hexadecimal:n { \xpinyin_UTF_viii_to_unicode:NNNw #1 \q_stop } }
\cs_new_nopar:Npn \xpinyin_UTF_viii_to_unicode:NNNw #1#2#3#4 \q_stop
  {
    \tl_if_empty:nTF {#4}
      { ( `#1 - "E0 ) * "1000 + ( `#2 - "80 ) * "40 + ( `#3 - "80 ) }
      { ( `#1 - "F0 ) * "4000 + ( `#2 - "80 ) * "1000 + ( `#3 - "80 ) * "40 + ( `#4 - "80 ) }
  }
\cs_generate_variant:Nn \int_from_hexadecimal:n { V }
%    \end{macrocode}
% \end{macro}
%
% \begin{macro}[internal]{\xpinyin_adjust_xeCJK_hook:}
%    \begin{macrocode}
\cs_new_nopar:Nn \xpinyin_adjust_xeCJK_hook:
  {
    \cs_new_eq:NN \xpinyin_select_font:           \xpinyin_select_font_xetex:
    \cs_new_eq:NN \xpinyin_CJKsymbol_to_unicode:n \xpinyin_CJKsymbol_to_unicode_xetex:n
    \cs_new_eq:NN \xpinyin_CJKchar_to_unicode:n   \xpinyin_CJKsymbol_to_unicode:n
    \cs_new_eq:NN \xpinyin_replace_CJKsymbol:     \xpinyin_replace_CJKsymbol_aux:
    \cs_new_eq:NN \xpinyin_CJK_node:              \xeCJK_CJK_kern:
    \tl_gset:Nn \l_xpinyin_coor_tl
      { (\cs_meaning:N \l_xpinyin_font_tl)/\l_xeCJK_font_coor_tl/\l_xpinyin_ratio_tl }
    \cs_new_nopar:Nn \xpinyin_CJKsymbol_hook: { \makexeCJKinactive \xeCJK_select_font: }
    \cs_new_nopar:Npn \xpinyin_xpinyin_single_hook:n
      { \cs_if_eq:NNTF \CJKsymbol \xpinyin_CJKsymbol:n { \cs_set_eq:NN \CJKsymbol \use:n } }
  }
%    \end{macrocode}
% \end{macro}
%
% \begin{macro}[internal]{\xpinyin_adjust_CJK_hook:}
%    \begin{macrocode}
\cs_new_nopar:Nn \xpinyin_adjust_CJK_hook:
  {
    \cs_new_eq:NN \xpinyin_select_font:           \xpinyin_select_font_aux:
    \cs_new_eq:NN \xpinyin_CJKsymbol_to_unicode:n \xpinyin_CJKsymbol_to_unicode_pdftex:n
    \cs_new_eq:NN \xpinyin_CJKchar_to_unicode:n   \xpinyin_CJKchar_to_unicode_pdftex:n
    \cs_new_eq:NN \xpinyin_CJKsymbol_hook:        \prg_do_nothing:
    \cs_new_eq:NN \xpinyin_CJK_node:              \CJK@CJK
    \@ifpackageloaded { CJKpunct }
      { \xpinyin_adjust_CJKpunct_hook: }
      {
        \cs_new_eq:NN \xpinyin_replace_CJKsymbol:   \xpinyin_replace_CJKsymbol_aux:
        \cs_new_eq:NN \xpinyin_xpinyin_single_hook:n \xpinyin_xpinyin_single_hook_aux:n
      }
    \prop_map_function:NN \g_xpinyin_tone_prop \DeclareUnicodeCharacter
  }
%    \end{macrocode}
% \end{macro}
%
% \begin{macro}[internal]{\xpinyin_adjust_CJKpunct_hook:}
%    \begin{macrocode}
\cs_new_nopar:Nn \xpinyin_adjust_CJKpunct_hook:
  {
    \cs_new_nopar:Nn \xpinyin_replace_CJKsymbol:
      {
        \int_compare:nNnTF { \CJKpunct@punctstyle } = { \CJKpunct@ps@plain }
          { \xpinyin_replace_CJKsymbol_aux: }
          {
            \cs_if_eq:NNF \CJKosymbol \xpinyin_CJKsymbol:n
              {
                \cs_set_eq:NN \xpinyin_save_CJKsymbol:n \CJKosymbol
                \cs_set_eq:NN \CJKosymbol \xpinyin_CJKsymbol:n
              }
          }
      }
    \cs_new_nopar:Npn \xpinyin_xpinyin_single_hook:n
      {
        \int_compare:nNnTF { \CJKpunct@punctstyle } = { \CJKpunct@ps@plain }
          { \xpinyin_xpinyin_single_hook_aux:n }
          {
            \cs_if_eq:NNTF \CJKosymbol \xpinyin_CJKsymbol:n
              {
                \cs_set_eq:NN \CJKosymbol \xpinyin_save_CJKsymbol:n
                \cs_set_eq:NN \xpinyin_save_CJKsymbol:n \use:n
              }
          }
      }
  }
%    \end{macrocode}
% \end{macro}
%
% \begin{macrocode}
\AtBeginDocument
  {
    \@ifpackageloaded { xeCJK }
      { \xpinyin_adjust_xeCJK_hook: }
      {
        \@ifpackageloaded { CJKutf8 }
          { \xpinyin_adjust_CJK_hook: }
          { \msg_warning:nn { xpinyin } { invalid } }
      }
  }
\msg_new:nnn { xpinyin } { invalid }
  {
    If~you~want~to~use~xpinyin~in~the~right~way,~you\\
    should~load~the~\xetex_if_engine:TF { xeCJK } { CJKutf8 }~
    package~in~the~preamble.\\
  }
%    \end{macrocode}
%
% \begin{macro}{\pinyin}
%    \begin{macrocode}
\NewDocumentCommand \pinyin { O{} m }
  {
    \group_begin:
    \keys_set:nn { xpinyin } {#1}
    \l_xpinyin_font_tl
    \l_xpinyin_format_tl
    \selectfont
    \xpinyin_pinyin:n {#2}
    \group_end:
  }
%    \end{macrocode}
% \end{macro}
%
% \begin{macro}[internal]{\xpinyin_pinyin:n}
%    \begin{macrocode}
\cs_new_nopar:Nn \xpinyin_pinyin:n
  {
    \xpinyin_xpinyin_init:
    \bool_set_true:N \l_xpinyin_first_bool
    \xpinyin_xpinyin_aux:N #1 \q_recursion_tail
    \prg_break_point:n
      {
        \bool_if:NTF \l_xpinyin_first_bool {#1}
          { \tl_if_empty:NF \l_tmpc_tl { \l_xpinyin_pysep_tl \l_tmpc_tl } }
      }
  }
%    \end{macrocode}
% \end{macro}
%
% \begin{macro}[internal]{\xpinyin_xpinyin_aux:N}
%    \begin{macrocode}
\cs_new_nopar:Nn \xpinyin_xpinyin_aux:N
  {
    \quark_if_recursion_tail_break:N #1
    \xpinyin_if_number:NTF {#1}
      {
        \bool_if:NTF \l_xpinyin_first_bool
          { \bool_set_false:N \l_xpinyin_first_bool }
          { \l_xpinyin_pysep_tl }
        \l_tmpa_tl
        \xpinyin_tone:Vn \l_xpinyin_tone_tl {#1}
        \l_tmpb_tl
        \xpinyin_xpinyin_init:
      }
      {
        \int_compare:nNnTF
          { 0 \use:c { c_xpinyin_ \l_xpinyin_tone_tl _tl } } >
          { 0 \use:c { c_xpinyin_ #1 _tl } }
          { \tl_put_right:Nx \l_tmpb_tl { \xpinyin_replace_v:N {#1} } }
          {
            \tl_set:Nn \l_xpinyin_tone_tl {#1}
            \tl_set_eq:NN \l_tmpa_tl \l_tmpc_tl
            \tl_clear:N \l_tmpb_tl
          }
        \tl_put_right:Nx \l_tmpc_tl { \xpinyin_replace_v:N {#1} }
      }
    \xpinyin_xpinyin_aux:N
  }
%    \end{macrocode}
% \end{macro}
%
% \begin{macro}[internal]{\xpinyin_tone:Nn}
%    \begin{macrocode}
\cs_new_nopar:Nn \xpinyin_tone:Nn
  { \use:c { xpinyin_num_to_tone_ #1 :Nn } {#1} {#2} }
\cs_generate_variant:Nn \xpinyin_tone:Nn { V }
%    \end{macrocode}
% \end{macro}
%
% \begin{macro}[internal]{\xpinyin_replace_v:N}
%    \begin{macrocode}
\cs_new_nopar:Nn \xpinyin_replace_v:N
  { \str_if_eq:nnTF {#1} { v } { \exp_not:n { ü } } { \exp_not:n {#1} } }
%    \end{macrocode}
% \end{macro}
%
% \begin{macro}[internal]{\xpinyin_xpinyin_init:}
%    \begin{macrocode}
\cs_new_nopar:Nn \xpinyin_xpinyin_init:
  {
    \tl_clear:N \l_tmpa_tl    \tl_clear:N \l_tmpb_tl
    \tl_clear:N \l_tmpc_tl    \tl_clear:N \l_xpinyin_tone_tl
  }
%    \end{macrocode}
% \end{macro}
%
% \begin{macro}[internal,pTF]{\xpinyin_if_number:N}
%    \begin{macrocode}
\prg_new_conditional:Nnn \xpinyin_if_number:N { p , T , F , TF }
  {
    \if_int_compare:w \c_one < 1 #1 \exp_stop_f:
      \prg_return_true: \else: \prg_return_false: \fi:
  }
%    \end{macrocode}
% \end{macro}
%
% \begin{macro}[internal,var]{\l_xpinyin_first_bool}
%    \begin{macrocode}
\bool_new:N \l_xpinyin_first_bool
%    \end{macrocode}
% \end{macro}
%
% \begin{macro}[internal,var]
% {\c_xpinyin_a_tl,\c_xpinyin_o_tl,\c_xpinyin_e_tl,\c_xpinyin_i_tl,
% \c_xpinyin_u_tl,\c_xpinyin_v_tl}
%    \begin{macrocode}
\tl_const:Nn \c_xpinyin_a_tl { 3 }
\tl_const:Nn \c_xpinyin_o_tl { 2 }
\tl_const:Nn \c_xpinyin_e_tl { 2 }
\tl_const:Nn \c_xpinyin_i_tl { 1 }
\tl_const:Nn \c_xpinyin_u_tl { 1 }
\tl_const:Nn \c_xpinyin_v_tl { 1 }
%    \end{macrocode}
% \end{macro}
%
% \begin{macro}[internal]{\xpinyin_num_to_tone:Nn}
%    \begin{macrocode}
\cs_new_nopar:Nn \xpinyin_num_to_tone:Nn
  {
    \if_case:w \int_eval:w #2 - \c_one \int_eval_end:
      \= {#1}  \or: \'{#1}  \or: \v {#1}  \or: \` {#1}  \else: #1 \fi:
  }
\tl_map_inline:nn { a o e u }
  { \cs_new_eq:cN { xpinyin_num_to_tone_ #1 :Nn } \xpinyin_num_to_tone:Nn }
\cs_new_nopar:Nn \xpinyin_num_to_tone_i:Nn
  {
    \if_case:w \int_eval:w #2 - \c_one \int_eval_end:
      ī  \or: í  \or: ǐ  \or: ì \else: i \fi:
  }
\cs_new_nopar:Nn \xpinyin_num_to_tone_v:Nn
  {
    \if_case:w \int_eval:w #2 - \c_one \int_eval_end:
      ǖ  \or: ǘ  \or: ǚ  \or: ǜ \else: ü \fi:
  }
%    \end{macrocode}
% \end{macro}
%
% \begin{macro}[internal]{\g_xpinyin_tone_prop,\xpinyin_prop_put_aux:n}
%    \begin{macrocode}
\prop_new:N \g_xpinyin_tone_prop
\cs_new_nopar:Nn \xpinyin_prop_put_aux:n { \prop_gput:Nnn \g_xpinyin_tone_prop #1 }
\clist_map_function:nN
  {
    {0101}{\=a} ,     {00E1}{\'a} ,     {01CE}{\v{a}} ,   {00E0}{\`a} ,
    {014D}{\=o} ,     {00F3}{\'o} ,     {01D2}{\v{o}} ,   {00F2}{\`o} ,
    {0113}{\=e} ,     {00E9}{\'e} ,     {011B}{\v{e}} ,   {00E8}{\`e} ,
    {012B}{\={\i}} ,  {00ED}{\'{\i}} ,  {01D0}{\v{\i}} ,  {00EC}{\`{\i}} ,
    {016B}{\=u} ,     {00FA}{\'u} ,     {01D4}{\v{u}} ,   {00F9}{\`u} ,
    {00FC}{\"u} ,
    {01D6}{\={\"u}} , {01D8}{\'{\"u}} , {01DA}{\v{\"u}} , {01DC}{\`{\"u}}
  }
  \xpinyin_prop_put_aux:n
%    \end{macrocode}
% \end{macro}
%
% \begin{macro}{\xpinyinsetup}
%    \begin{macrocode}
\NewDocumentCommand \xpinyinsetup { m } { \keys_set:nn { xpinyin } {#1} }
%    \end{macrocode}
% \end{macro}
%
% \begin{macro}{ratio,vsep,hsep,pysep,font,format,multiple}
%    \begin{macrocode}
\clist_map_inline:nn
  { ratio , vsep , hsep , pysep , font , format , multiple }
  { \keys_define:nn { xpinyin } { #1 .tl_set:c = { l_xpinyin_ #1 _tl } } }
\keys_set:nn { xpinyin }
  {
    ratio   = .4 ,
    vsep    = 1 em ,
    pysep   = \c_space_tl ,
    font    = \normalfont ,
  }
%    \end{macrocode}
% \end{macro}
%
% \begin{macro}[internal]{xpinyin-map.cfg}
%    \begin{macrocode}
\group_begin:
\char_set_catcode_active:N \U
\char_set_catcode_active:N \V
\cs_set_nopar:Npn U+ #1 ~ #2 ~ { \tl_gset:cn { c_xpinyin_ #1 _tl } {#2} }
\cs_set_nopar:Npn V+ #1 ~ #2 ~ { \clist_gset:cn { c_xpinyin_multiple_ #1 _clist } {#2} }
\char_set_catcode_space:N \
\file_input:n {xpinyin-map.cfg}
\group_end:
%    \end{macrocode}
% \end{macro}
%
% \begin{macro}{\setpinyin}
%    \begin{macrocode}
\NewDocumentCommand \setpinyin { m m }
  {
    \tl_set:cn
      { c_xpinyin_ \xpinyin_CJKchar_to_unicode:n {#1} _tl }
      { \xpinyin_pinyin:n {#2} }
  }
%    \end{macrocode}
% \end{macro}
%
%    \begin{macrocode}
\ProcessKeysOptions { xpinyin }
%    \end{macrocode}
%
% \iffalse
%</package>
% \fi
%
% \end{implementation}
%
% \PrintIndex
% \Finale
%
\endinput
