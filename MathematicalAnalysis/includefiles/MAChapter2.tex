%# -*- coding:utf-8 -*-
%%%%%%%%%%%%%%%%%%%%%%%%%%%%%%%%%%%%%%%%%%%%%%%%%%%%%%%%%%%%%%%%%%%%%%%%%%%%%%%%%%%%%
%%  MAChapter2.tex'


\chapter{极\emspace 限}\label{ch:2}
\section{序列极限的定义}
\subsection{概念引入}
\subsection{序列极限定义}
\begin{exercise}
\item
\end{exercise}
\section{序列极限的性质与运算}
\begin{exercise}
\item
\end{exercise}
\section{确界与单调有界序列}
\begin{exercise}
\item
\end{exercise}
\section{函数的极限}
\begin{exercise}
\item
\end{exercise}
\section{函数极限的推广}
\subsection{自变量趋于无穷的情形}
\subsection{无穷大量}
\subsection{单侧极限}
\subsection{极限存在性}
\subsection{复合函数求极限}
\begin{exercise}
\item
\end{exercise}
\section{两个重要极限}
\begin{exercise}
\item
\end{exercise}
\section{无穷小量的阶以及无穷大量的阶的比较}
\begin{exercise}
\item
\end{exercise}
\section{用肯定语气叙述极限不是某常数}
\subsection{极限不是某常数的肯定描述}
\subsection{序列极限与函数极限的关系}
\begin{exercise}
\item
\end{exercise}
\begin{exercise*}
\item
\end{exercise*}



\endinput
%%
%% End of file `MAChapter2.tex'.