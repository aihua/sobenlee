%# -*- coding:utf-8 -*-
%%%%%%%%%%%%%%%%%%%%%%%%%%%%%%%%%%%%%%%%%%%%%%%%%%%%%%%%%%%%%%%%%%%%%%%%%%%%%%%%%%%%%
%%  MAChapter9.tex'


\chapter{实数空间}\label{ch:9}
\section{实数定义}
\subsection{为什么要定义实数}
\subsection{实数的定义}
\section{实数空间}
\subsection{实数的计算}
\subsection{实数集是域}
\subsection{实数集是全序域}
\subsection{实数集的连通性}
\subsection{实数的表示}
\subsection{实数集的公理系统}
\begin{exercise}
\item
\end{exercise}
\section{确界存在原理与区间套定理}
\subsection{确界存在原理}
\subsection{区间套定理}
\begin{exercise}
\item
\end{exercise}
\section{紧性定理}
\subsection{有限覆盖定理}
\subsection{聚点原理}
\subsection{子序列与~Bolzano~定理}
\begin{exercise}
\item
\end{exercise}
\section{完备性定理}
\subsection{序列极限的~Cauchy~准则}
\subsection{函数极限的~Cauchy~准则}
\begin{exercise}
\item
\end{exercise}
\section{连续函数性质证明}
\begin{exercise}
\item
\end{exercise}
\section{压缩映射原理}
\begin{exercise}
\item
\end{exercise}
\section{上极限与下极限}
\subsection{序列的上、下极限的定义}
\subsection{上、下极限的性质}
\subsection{上、下极限的等价形式}
\subsection{函数的上、下极限}
\begin{exercise}
\item
\end{exercise}
\begin{exercise*}
\item
\end{exercise*}




\endinput
%%
%% End of file `MAChapter9.tex'.