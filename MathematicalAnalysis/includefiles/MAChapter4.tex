%# -*- coding:utf-8 -*-
%%%%%%%%%%%%%%%%%%%%%%%%%%%%%%%%%%%%%%%%%%%%%%%%%%%%%%%%%%%%%%%%%%%%%%%%%%%%%%%%%%%%%
%%  MAChapter4.tex'


\chapter{导数与微分}\label{ch:4}
\section{导数概念}
\begin{exercise}
\item
\end{exercise}
\section{导数的几何意义与极限}
\begin{exercise}
\item
\end{exercise}
\section{导数的四则运算}
\begin{exercise}
\item
\end{exercise}
\section{复合函数求导}
\subsection{复合函数求导}
\subsection{隐函数微分法}
\subsection{对数微分法}
\begin{exercise}
\item
\end{exercise}
\section{反函数与参数式求导}
\subsection{反函数求导}
\subsection{参数式求导}
\subsection{极坐标式求导}
\begin{exercise}
\item
\end{exercise}
\section{微\emspace 分}
\subsection{微分定义}
\subsection{微分与导数}
\subsection{微分的几何意义}
\subsection{一阶微分形式的不变形}
\subsection{微分的应用}
\begin{exercise}
\item
\end{exercise}
\section{高阶导数与高阶微分}
\subsection{高阶导数}
\subsection{Leibniz~公式}
\subsection{其它函数关系的高阶导数}
\subsection{高阶微分}
\begin{exercise}
\item
\end{exercise}
\begin{exercise*}
\item
\end{exercise*}


\endinput
%%
%% End of file `MAChapter4.tex'.