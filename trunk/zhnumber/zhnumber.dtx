% \iffalse meta-comment
% !TEX program  = XeLaTeX
%<*internal>
\iffalse
%</internal>
%<*readme>
Introduction
------------
The zhnumber package provides commands to typeset Chinese representations of
numbers. The main difference between this package and 'CJKnumb' is that commands
provided by this package is expandable in the proper way. So, it seems that
zhnumber is a good alternative to CJKnumb package.

It may be distributed and/or modified under the conditions of the
LaTeX Project Public License (LPPL), either version 1.3c of this license or
(at your option) any later version. The latest version of this license is in

   http://www.latex-project.org/lppl.txt

and version 1.3 or later is part of all distributions of LaTeX version
2005/12/01 or later.

This work has the LPPL maintenance status "maintained".
The Current Maintainer of this work is Qing Lee.

This work consists of the file  zhnumber.dtx,
          and the derived files zhnumber.pdf,
                                zhnumber.sty,
                                zhnumber-utf8.cfg,
                                zhnumber-gbk.cfg,
                                zhnumber-big5.cfg,
                                zhnumber.ins and
                                README (this file).

Basic Usage
-----------
The package provides the following macros:

  \zhnumber{number}
    Convert `number' to a full Chinese representation.

  \zhnum{counter}
    Similar to \arabic{counter}, but representation of 'counter' as Chinese numerals.

  \zhdigits{number}
  \zhdigits*{number}
    Handle `number' as a string of digits and convert each of them into the
    corresponding Chinese digit. The starred version uses the Chinese circle glyph
    for digit zero; the unstarred version uses the traditional glyph.

You can read the package manual (in Chinese) for more detailed explanations.

Author
------
Qing Lee
Email: sobenlee@gmail.com

If you are interested in the process of development you may observe

    http://code.google.com/p/ctex-kit/

Installation
------------
The package is supplied in dtx format and as a pre-extracted zip file,
zhnumber.tds.zip. The later is most convenient for most users: simply
unzip this in your local texmf directory and run texhash to update the
database of file locations. If you want to unpack the dtx yourself, please
ensure that the "iconv" program is installed and working properly, then
running "xetex -shell-escape zhnumber.dtx" will extract the package whereas
"xelatex -shell-escape zhnumber.dtx" will extract it and also typeset the
documentation.

The package requires LaTeX3 support as provided in the l3kernel and l3packages
bundles. Both of these are available on CTAN as ready-to-install zip files.
Suitable versions are available in the latest version of MiKTeX and TeX Live
(updating the relevant packages online may be necessary).

To compile the documentation without error, you will need the xeCJK package
and some specific Chinese Simplified fonts (TrueType or OpenType).
%</readme>
%<*internal>
\fi
\ifnum\strcmp{\fmtname}{plain}=0 \else
  \expandafter\begingroup
\fi
%</internal>
%<*batchfile>
\input docstrip.tex
\keepsilent
\askforoverwritefalse
\preamble

   Copyright (C) 2012 by Qing Lee <sobenlee@gmail.com>
--------------------------------------------------------------------------
   This work may be distributed and/or modified under the
   conditions of the LaTeX Project Public License, either version 1.3
   of this license or (at your option) any later version.
   The latest version of this license is in
     http://www.latex-project.org/lppl.txt
   and version 1.3 or later is part of all distributions of LaTeX
   version 2005/12/01 or later.

   This work has the LPPL maintenance status "maintained".
   The Current Maintainer of this work is Qing Lee.

\endpreamble
\postamble

   This package consists of the file  zhnumber.dtx,
                and the derived files zhnumber.pdf,
                                      zhnumber.sty,
                                      zhnumber-utf8.cfg,
                                      zhnumber-gbk.cfg,
                                      zhnumber-big5.cfg,
                                      zhnumber.ins and
                                      README.
\endpostamble
\ifnum\shellescape=1 \else
  \errmessage{
    Shell escape is disabled. Please use ^^J^^J
    xe(la)tex -shell-escape \jobname.ins(dtx)^^J}
  \expandafter\endbatchfile
\fi
\usedir{tex/latex/zhnumber}
\generate{\file{\jobname.sty}{\from{\jobname.dtx}{package}}}
\usedir{tex/latex/zhnumber/config}
\generate{
  \file{\jobname-utf8.cfg}{\from{\jobname.dtx}{config,utf8}}
  \file{\jobname-big5.cfg}{\from{\jobname.dtx}{config,big5}}
  \file{\jobname-gbk.cfg}{\from{\jobname.dtx}{config,gbk}}}
\immediate\write18{iconv -f utf-8 -t big-5 \jobname-big5.cfg > \jobname-big5.temp}
\immediate\write18{iconv -f utf-8 -t gbk \jobname-gbk.cfg > \jobname-gbk.temp}
\immediate\write18{mv -f \jobname-big5.temp \jobname-big5.cfg}
\immediate\write18{mv -f \jobname-gbk.temp \jobname-gbk.cfg}
%</batchfile>
%<batchfile>\endbatchfile
%<*internal>
\usedir{source/latex/zhnumber}
\generate{\file{\jobname.ins}{\from{\jobname.dtx}{batchfile}}}
\nopreamble\nopostamble
\usedir{doc/latex/zhnumber}
\generate{\file{README.txt}{\from{\jobname.dtx}{readme}}}
\ifnum\strcmp{\fmtname}{plain}=0
  \expandafter\endbatchfile
\else
  \expandafter\endgroup
\fi
%</internal>
%
%<*driver|package>
\NeedsTeXFormat{LaTeX2e}
%<*driver>
\ProvidesFile{zhnumber.dtx}
%</driver>
\RequirePackage{expl3}
%</driver|package>
%<package|config>\GetIdInfo$Id$
%<package>          {package for typesetting numbers with Chinese glyphs}
%<config&utf8>          {Chinese numerals with UTF8 encoding}
%<config&big5>          {Chinese numerals with Big5 encoding}
%<config&gbk>          {Chinese numerals with Big5 encoding}
%<package>\ProvidesExplPackage{\ExplFileName}{\ExplFileDate}{1.5}{\ExplFileDescription}
%<config>\ProvidesExplFile
%<config&utf8>  {\ExplFileName-utf8.cfg}
%<config&big5>  {\ExplFileName-big5.cfg}
%<config&gbk>  {\ExplFileName-gbk.cfg}
%<config>  {\ExplFileDate} {1.5} {\ExplFileDescription}
%
%<*driver>
\RequirePackage{filehook}
\AtEndOfPackageFile{hypdoc}{\let\SaveHDtheCodelineNo\theCodelineNo}
\documentclass{l3doc}
\usepackage{amsmath}
\usepackage{xeCJK}
\usepackage{zhnumber}
\usepackage{fvrb-ex}
\usepackage{metalogo}
\hypersetup{pdfstartview=FitH}
\setlist{nosep}
\linespread{1.1}
\addtolength{\voffset}{-5\baselineskip}
\addtolength\textheight{8\baselineskip}
\setmainfont{TeX Gyre Pagella}
\setmonofont{Inconsolata}
\xeCJKDeclareCharClass{CJK}{ "25CB }
\setCJKmainfont[BoldFont=Adobe Heiti Std,ItalicFont=Adobe Kaiti Std]{Adobe Song Std}
\setCJKmonofont{Adobe Kaiti Std}
\xeCJKsetup{PunctStyle=kaiming}
\def\MacroFont{\small\normalfont\ttfamily}
\makeatletter
\let\orig@meta\meta
\def\meta#1{\orig@meta{\normalfont\itshape#1}}
\def\TF{true\orvar{}false}
\def\TTF{\defaultvar{true}\orvar{}false}
\def\TFF{true\orvar\defaultvar{false}}
\def\orvar{\char`\|}
\let\defaultvar\textbf
\def\argbrace#1{\char`\{#1\char`\}}
\@addtoreset{CodelineNo}{section}
\ExplSyntaxOn
\let\SaveLDtheCodelineNo\theCodelineNo
\tl_replace_once:Nnn \SaveHDtheCodelineNo
 { \HDorg@theCodelineNo } { \SaveLDtheCodelineNo }
\let \theCodelineNo \SaveHDtheCodelineNo
\pretocmd { \doc_print_macroname:n } { \HD@target } { }{ }
\pretocmd { \doc_special_main_index:n } { \HD@target } { } { }
\patchcmd { \doc_special_main_index:n }
  { hdpgindex{\thepage} } { hdclindex{\the\c@HD@hypercount} } { } { }
\ExplSyntaxOff
\makeatother
\def\indexname{代码索引}
\IndexPrologue{%
  \section*{\indexname}
  \markboth{\indexname}{\indexname}
  斜体的数字表示对应项说明所在的页码,下划线的数字表示定义所在的代码行号,而直立体的
  数字表示对应项使用时所在的行号。}
\begin{document}
  \DocInput{\jobname.dtx}
\end{document}
%</driver>
% \fi
%
% \GetFileInfo{\jobname.sty}
%
% \title{\bfseries\pkg{zhnumber} 宏包}
% \author{李清\\ \path{sobenlee@gmail.com}}
% \date{\filedate\qquad\fileversion}
% \maketitle
%
% \begin{documentation}
%
% \section{简介}
% \pkg{zhnumber} 宏包用于将阿拉伯数字按照中文格式输出。相比于 \pkg{CJKnumb},它提供
% 的三个格式转换命令 |\zhnumber|,|\zhdigits| 和 |\zhnum| 都是可以适当展开的,可以
% 正常使用于 |PDF| 书签和交叉引用。
%
% \pkg{zhnumber} 支持 |GBK|,|Big5| 和 |UTF8| 编码,依赖 \LaTeX~3 项目的 \pkg{expl3},
% \pkg{xparse} 和 \pkg{l3keys2e} 宏包。
%
% \section{使用方法}
%
% \begin{function}[updated=2012-5-25]{encoding}
%   \begin{syntax}
%     encoding = \meta{GBK\orvar{}Big5\orvar{}UTF8}
%   \end{syntax}
%   用于指定编码的宏包选项,可以在调用宏包的时候设定,也可以用 |\zhnumsetup| 在导言区内设定。
%   对于 \XeLaTeX 和 \LuaLaTeX ,不用指定编码,宏包将自动使用 |UTF8| 编码。只有 \LaTeX
%   和 pdf\LaTeX 需要指定编码,如果没有指定,默认将使用 |GBK|,并且此时 \pkg{zhnumber}
%   宏包应该在 \pkg{CJK} 或 \pkg{CJKutf8} 宏包之后载入。
% \end{function}
%
% \begin{function}{\zhnumber}
%   \begin{syntax}
%     \cs{zhnumber} \marg{number}
%   \end{syntax}
%   以中文格式输出数字。这里的数字可以是整数、小数和分数。例如\\[1ex]
%   \begin{SideBySideExample}[frame=single,numbers=left,xrightmargin=.6\linewidth,gobble=5]
%     \zhnumber{2012020120}\\
%     \zhnumber{2 012 020 120}\\
%     \zhnumber{2,012,020,120}\\
%     \zhnumber{2012.020120}\\
%     \zhnumber{2012.}\\
%     \zhnumber{.2012}\\
%     \zhnumber{20120/20120}\\
%     \zhnumber{/2012}\\
%     \zhnumber{2012/}\\
%     \zhnumber{201;2020/120}
%   \end{SideBySideExample}
% \end{function}
%
% \begin{function}{\zhdigits}
%   \begin{syntax}
%     \cs{zhdigits} \marg{number}
%   \end{syntax}
%   将阿拉伯数字转换为中文数字串。缺省状态下,|\zhdigits| 将 0 映射为〇,如果需要
%   将其映射为零,可以使用 |\zhdigits*|。例如\\[1ex]
%   \begin{SideBySideExample}[frame=single,numbers=left,xrightmargin=.6\linewidth,gobble=5]
%     \zhdigits{2012020120}\\
%     \zhdigits*{2012020120}
%   \end{SideBySideExample}
% \end{function}
%
% \begin{function}{\zhnum}
%   \begin{syntax}
%     \cs{zhnum} \marg{counter}
%   \end{syntax}
%   与 |\roman| 等类似,用于将 \LaTeX 计数器的值转换为中文数字。例如\\[1ex]
%   \begin{SideBySideExample}[frame=single,numbers=left,xrightmargin=.6\linewidth,gobble=5]
%     \zhnum{section}
%   \end{SideBySideExample}
% \end{function}
%
% \begin{function}[added=2012-5-25]{\zhweekday}
%   \begin{syntax}
%     \cs{zhweekday} \Arg{yyyy/mm/dd}
%   \end{syntax}
%   输出日期当天的星期。例如\\[1ex]
%   \begin{SideBySideExample}[frame=single,numbers=left,xrightmargin=.6\linewidth,gobble=5]
%     \zhweekday{2012/5/20}
%   \end{SideBySideExample}
% \end{function}
%
% \begin{function}[added=2012-5-25]{\zhdate}
%   \begin{syntax}
%     \cs{zhdate}  \Arg{yyyy/mm/dd}
%     \cs{zhdate*} \Arg{yyyy/mm/dd}
%   \end{syntax}
%   以中文格式输出日期,其中带 |*| 的命令还输出星期。例如\\[1ex]
%   \begin{SideBySideExample}[frame=single,numbers=left,xrightmargin=.6\linewidth,gobble=5]
%     \zhdate{2012/5/21}\\
%     \zhdate*{2012/5/21}
%   \end{SideBySideExample}
% \end{function}
%
% \begin{function}[added=2012-5-25]{\zhtoday}
%   与 |\today| 类似,以中文输出当天的日期。例如\\[1ex]
%   \begin{SideBySideExample}[frame=single,numbers=left,xrightmargin=.6\linewidth,gobble=5]
%     \zhtoday
%   \end{SideBySideExample}
% \end{function}
%
% \begin{function}[added=2012-5-25]{\zhtime}
%   \begin{syntax}
%     \cs{zhtime} \Arg{hh:mm}
%   \end{syntax}
%   以中文格式输出时间。例如\\[1ex]
%   \begin{SideBySideExample}[frame=single,numbers=left,xrightmargin=.6\linewidth,gobble=5]
%     \zhtime{23:56}
%   \end{SideBySideExample}
% \end{function}
%
% \begin{function}[added=2012-5-25]{\zhcurrtime}
%   输出当前的时间。例如\\[1ex]
%   \begin{SideBySideExample}[frame=single,numbers=left,xrightmargin=.6\linewidth,gobble=5]
%     \zhcurrtime
%   \end{SideBySideExample}
% \end{function}
%
% \begin{function}[added=2012-5-25]{\zhnumExtendScaleMap}
%   \begin{syntax}
%     \cs{zhnumExtendScaleMap} \oarg{character} \argbrace{\meta{character 1}, \meta{character 2}, ..., \meta{character n}}
%   \end{syntax}
%   缺省状态下 |\zhnumber| 能正确中文格式化的最大整数是 $10^{48}-1$,|\zhditits| 不受
%   这个大小的限制。可以通过 |\zhnumExtendScaleMap| 来扩展 |\zhnumber|。
%   \meta{character} $i$ 设置 $10^{4(i+11)}$。若给出可选项 \meta{character},则当
%   数字大于 $10^{4(n+12)}-1$ 时,统一用 \meta{character} 设置输出数字的进位。
% \end{function}
%
% \begin{function}{\zhnumsetup}
%   \begin{syntax}
%     \cs{zhnumsetup} \argbrace{\meta{key1}=\meta{var1}, \meta{key2}=\meta{var2}, ...}
%   \end{syntax}
%   用于在导言区或文档中,设置中文数字的输出格式。目前可以设置的 \meta{key} 如下介绍。
% \end{function}
%
% \begin{function}[added=2012-5-25]{time}
%   \begin{syntax}
%     time =  \argbrace{\meta{Arabic}\orvar\meta{Chinese}}
%   \end{syntax}
%   设置日期和时间的数字格式,\meta{Arabic} 为阿拉伯数字,而 \meta{Chinese} 为中文数字。
%   默认使用阿拉伯数字。例如\\[1ex]
%   \begin{SideBySideExample}[frame=single,numbers=left,xrightmargin=.6\linewidth,gobble=5]
%     \zhnumsetup{time=Chinese}
%     \zhtoday\zhcurrtime
%   \end{SideBySideExample}
% \end{function}
%
% \begin{function}[updated=2012-5-25]{style}
%   \begin{syntax}
%     style = \argbrace{\meta{Simplified}\orvar\meta{Traditional}\orvar\meta{Normal}\orvar\meta{Financial}\orvar\meta{Ancient}}
%   \end{syntax}
%   意义分别为
%   \begin{itemize}[font=\ttfamily,labelsep=1em]
%     \item[Simplified]  以简体中文输出数字(对 |Big5| 编码无效);
%     \item[Traditional] 以繁体中文输出数字(对 |Big5| 编码无效);
%     \item[Normal] 以小写形式输出中文数字;
%     \item[Financial]  以大写形式输出中文数字;
%     \item[Ancient] 以廿输出 20,以卅输出 30,以卌输出 40,以皕输出 200。
%   \end{itemize}
%   可以设置 |style| 为其中一个,也可以是前三个与后两个的适当组合,默认是简体小写。例如\\[1ex]
%   \begin{SideBySideExample}[frame=single,numbers=left,xrightmargin=.4\linewidth,gobble=5]
%     \zhnumsetup{style={Traditional,Financial}}
%     \zhnumber{62012.3}\\
%     \zhnumsetup{style=Ancient}
%     \zhnumber{21}
%   \end{SideBySideExample}
% \end{function}
%
% \begin{function}{null}
%   \begin{syntax}
%     null = \meta{\TFF}
%   \end{syntax}
%   缺省状态下,除了 |\zhdigits| 外,其它的格式转换命令,将 0 映射成零,如果需要将 0 映射
%   成〇,可以使用这个选项。\strut
% \end{function}
%
% \pkg{zhnumber} 提供下列选项来控制阿拉伯数字的中文映射。
% \begin{verbatim}[frame=single]
%   - -0 0 1 2 3 4 5 6 7 8 9 10 20 30 40 200
%   E2 E3 E4 E8 E12 E16 E20 E24 E28 E32 E36 E40 E44
%   F0 F1 F2 F3 F4 F5 F6 F7 F8 F9 F10 FE2 FE3
%   dot and parts
%   year month day hour minute weekday mon tue wed thu fri sat sun
% \end{verbatim}
% 其中 |-| 设置负,|-0| 设置〇,|dot| 设置小数的点,|and| 和 |parts| 分别设置分数
% 的“又”和“分之”,|E|$n$ 设置 $10^n$,而 |F|$n$ 设置数字 $n$ 的大写。其它的选项同
% 字面意思,不再赘述。例如
% \begin{verbatim}[frame=single]
%   \zhnumsetup{2={两}}
% \end{verbatim}
% 可以将 2 映射成两。需要说明的是,\pkg{zhnumber} 将优先使用这里的设置,所以可能会影响
% 到 |style| 选项。如果要恢复 |style| 的功能,可以使用 |reset| 选项。
%
% \begin{function}[updated=2012-5-25]{reset}
%   \begin{syntax}
%     reset
%   \end{syntax}
%   用于恢复 \pkg{zhnumber} 对阿拉伯数字的初始化映射。\pkg{zhnumber} 的中文数字初始化
%   设置见源代码(第 \ref{sec:zhnum-map} 节)。
% \end{function}
%
% \begin{function}{\zhnumber,\zhdigits,\zhnum}
%   \begin{syntax}
%     \cs{zhnumber} \oarg{options} \marg{number}
%     \cs{zhdigits} \oarg{options} \marg{number}
%     \cs{zhnum} \oarg{options} \marg{counter}
%   \end{syntax}
%   如果只改变当前数字的中文输出格式,可以使用带选项的格式转换命令,其中 \meta{options}
%   与 |\zhnumsetup| 的参数相同,如上所介绍。这些带了选项的命令是不可展开的,在某些场合使
%   用时要小心。
% \end{function}
%
% \end{documentation}
%
%
% \begin{implementation}
%
% \section{\pkg{zhnumber} 宏包代码实现}
%
%    \begin{macrocode}
%<*package>
%    \end{macrocode}
%
%    \begin{macrocode}
\msg_new:nnn { zhnumber } { l3-too-old }
  {
    Support~package~'expl3'~too~old. \\\\
    Please~update~an~up~to~date~version~of~the~bundles\\\\
    'l3kernel'~and~'l3packages'\\\\
    using~your~TeX~package~manager~or~from~CTAN.
  }
\@ifpackagelater { expl3 } { 2012/02/19 } { }
  { \msg_error:nn  { zhnumber }  { l3-too-old } }
%    \end{macrocode}
%
%    \begin{macrocode}
\RequirePackage{xparse}
\RequirePackage{l3keys2e}
%    \end{macrocode}
%
% \begin{macro}{\zhnumber}
% 用于将输入的数字按照中文格式输出。
%    \begin{macrocode}
\DeclareExpandableDocumentCommand \zhnumber { o m }
  { \zhnum_number_aux:nNn {#1} \zhnum_number:n {#2} }
\cs_new_nopar:Nn \zhnum_number:n { \zhnum_number:w #1 . \q_nil . \q_stop }
%    \end{macrocode}
% \end{macro}
%
% \begin{macro}[internal]{\zhnum_number_aux:nNn}
%    \begin{macrocode}
\cs_new_nopar:Nn \zhnum_number_aux:nNn
  {
    \IfNoValueTF {#1} { #2 {#3} }
      { \group_begin: \zhnumsetup {#1} #2 {#3} \group_end: }
  }
%    \end{macrocode}
% \end{macro}
%
% \begin{macro}[internal]{\zhnum_number:w}
% 先判断输入的是小数还是分数。
%    \begin{macrocode}
\cs_new_nopar:Npn \zhnum_number:w #1.#2.#3 \q_stop
  {
    \quark_if_nil:nTF {#2}
      { \zhnum_integer_or_fraction:w #1 / \q_nil / \q_stop }
      { \zhnum_decimal:nn {#1} {#2} }
  }
%    \end{macrocode}
% \end{macro}
%
% \begin{macro}[internal]{\zhnum_integer_or_fraction:w}
% 判断是否输入的是分数。
%    \begin{macrocode}
\cs_new_nopar:Npn \zhnum_integer_or_fraction:w #1/#2/#3 \q_stop
  {
    \quark_if_nil:nTF {#2}
      { \zhnum_integer:f {#1} }
      { \zhnum_fraction:w #2 \q_mark #1 ; \q_nil ; \q_stop }
  }
%    \end{macrocode}
% \end{macro}
%
% \begin{macro}[internal]{\zhnum_fraction:w}
% 对分数进行预处理。
%    \begin{macrocode}
\cs_new_nopar:Npn \zhnum_fraction:w #1 \q_mark #2;#3;#4 \q_stop
  {
    \quark_if_nil:nTF {#3}
      {
        \zhnum_blank_to_zero:f {#1} \c_zhnum_parts_tl
        \zhnum_blank_to_zero:f {#2}
      }
      {
        \tl_if_blank:fF {#2} { \zhnumber {#2} \c_zhnum_and_tl }
        \zhnum_blank_to_zero:f {#1} \c_zhnum_parts_tl
        \zhnum_blank_to_zero:f {#3}
      }
  }
\cs_generate_variant:Nn \tl_if_blank:nF { f }
%    \end{macrocode}
% \end{macro}
%
% \begin{macro}[internal]{\zhnum_decimal:nn}
% 对小数进行预处理。
%    \begin{macrocode}
\cs_new_nopar:Nn \zhnum_decimal:nn
  {
    \zhnum_blank_to_zero:f {#1} \c_zhnum_dot_tl
    \tl_if_blank:fTF {#2} { \c_zhnum_zero_tl } { \zhdigits * {#2} }
  }
\cs_generate_variant:Nn \tl_if_blank:nTF { f }
%    \end{macrocode}
% \end{macro}
%
% \begin{macro}[internal]{\zhnum_blank_to_zero:n}
% 输出小数的整数位。
%    \begin{macrocode}
\cs_new_nopar:Nn \zhnum_blank_to_zero:n
  { \tl_if_blank:nTF {#1} { \zhnum_digit_map:n \c_zero } { \zhnumber {#1} } }
\cs_generate_variant:Nn \zhnum_blank_to_zero:n { f }
%    \end{macrocode}
% \end{macro}
%
% \begin{macro}{\zhnum}
% 用于将 \LaTeX 计数器按中文格式输出。
%    \begin{macrocode}
\DeclareExpandableDocumentCommand \zhnum { o m }
  { \zhnum_number_aux:nNn {#1} \zhnum_counter:n {#2} }
\cs_new_nopar:Nn \zhnum_counter:n
  {
    \exp_args:Nc \token_if_int_register:NTF { c@#1 }
      { \zhnum_integer:v { c@#1 } }
      { \@nocounterr {#1} }
  }
%    \end{macrocode}
% \end{macro}
%
% \begin{macro}[internal]{\zhnum_integer:n,\zhnum_integer_aux:n}
% 对整数的处理。
%    \begin{macrocode}
\cs_new_nopar:Nn \zhnum_integer:n
  { \zhnum_integer_aux:f { \zhnum_erase_separator:n {#1} } }
\cs_new_nopar:Nn \zhnum_integer_aux:n
  {
    \int_compare:nNnT { \int_get_sign:n {#1} \c_one } < \c_zero { \c_zhnum_minus_tl }
    \zhnum_parse_number:f { \zhnum_trim_zeros:f { \int_get_digits:n {#1} } }
  }
\cs_generate_variant:Nn \zhnum_integer:n { f , v , V }
\cs_generate_variant:Nn \zhnum_integer_aux:n { f }
%    \end{macrocode}
% \end{macro}
%
% \begin{macro}[internal]{\zhnum_erase_separator:n,\zhnum_trim_zeros:n}
% 去掉分隔符和多余的 $0$。
%    \begin{macrocode}
\cs_new_nopar:Nn \zhnum_erase_separator:n
  { \cs_to_str:c { \tl_map_function:nN {#1} \zhnum_erase_separator_aux:N } }
\cs_new_nopar:Nn \zhnum_erase_separator_aux:N { \zhnum_if_number:NT {#1} {#1} }
\cs_new_nopar:Nn \zhnum_trim_zeros:n
  {
    \tl_if_empty:nTF {#1} \c_zero
      {
        \int_compare:nNnTF { \tl_head:n {#1} } = \c_zero
          { \zhnum_trim_zeros:o { \use_none:n #1 } } {#1}
      }
  }
\prg_new_conditional:Nnn \zhnum_if_number:N { p , T , F , TF }
  {
    \if_int_compare:w \c_one < 1 #1 \exp_stop_f:
      \prg_return_true: \else: \prg_return_false: \fi:
  }
\cs_generate_variant:Nn \zhnum_trim_zeros:n { f , o }
\cs_generate_variant:Nn \cs_to_str:N        { c }
%    \end{macrocode}
% \end{macro}
%
% \begin{macro}[internal]
% {\zhnum_parse_number:n,\zhnum_parse_number:nn,\zhnum_parse_number:nnn}
%    \begin{macrocode}
\cs_new_nopar:Nn \zhnum_parse_number:n
  { \zhnum_parse_number:nf {#1} { \tl_length:n {#1} } }
\cs_new_nopar:Nn \zhnum_parse_number:nn
  {
    \int_compare:nNnTF {#2} < \c_five
      {
        \int_compare:nNnTF {#1} = \c_zero
          { \c_zhnum_zero_tl }
          { \zhnum_process_number:NNn \c_true_bool \c_true_bool {#1} }
      }
      {
        \int_compare:nNnTF { \int_mod:nn {#2} \c_four } = \c_zero
          {
            \zhnum_split_number:nNNnfn {#1} \c_true_bool \c_true_bool { \c_zero }
              { \int_eval:n { (#2) / \c_four - \c_one } }
              { \c_zero }
          }
          {
            \zhnum_parse_number:nnf {#1} {#2}
              {
                \use:c
                  {
                    zhnum_use_
                    \int_to_roman:n { \int_mod:nn {#2} \c_four }
                    _delimit_by_q_stop:w
                  }
                #1 \q_stop
              }
          }
      }
  }
\cs_new_nopar:Nn \zhnum_parse_number:nnn
  {
    \zhnum_process_number:NNn \c_true_bool \c_true_bool {#3}
    \zhnum_scale_map:n { \int_div_truncate:nn {#2} \c_four }
    \int_compare:nNnTF { \int_mod:nn {#3} \c_ten } = \c_zero
      { \zhnum_split_number:nNNffn {#1} \c_false_bool \c_true_bool }
      { \zhnum_split_number:nNNffn {#1} \c_true_bool \c_false_bool }
    { \int_mod:nn {#2} \c_four }
    { \int_eval:n { \int_div_truncate:nn {#2} \c_four - \c_one } }
    { \c_zero }
  }
\cs_generate_variant:Nn \zhnum_parse_number:n   {   f }
\cs_generate_variant:Nn \zhnum_parse_number:nn  {  nf }
\cs_generate_variant:Nn \zhnum_parse_number:nnn { nnf }
\cs_new_nopar:Npn \zhnum_use_i_delimit_by_q_stop:w   #1#2     \q_stop {#1}
\cs_new_nopar:Npn \zhnum_use_ii_delimit_by_q_stop:w  #1#2#3   \q_stop {#1#2}
\cs_new_nopar:Npn \zhnum_use_iii_delimit_by_q_stop:w #1#2#3#4 \q_stop {#1#2#3}
%    \end{macrocode}
% \end{macro}
%
% \begin{macro}[internal]{\zhnum_split_number:nNNnnn}
% 将输入的整数由低位到高位,以四位为一段进行处理。
%    \begin{macrocode}
\cs_new_nopar:Nn \zhnum_split_number:nNNnnn
  {
    \exp_args:Nf \zhnum_split_number_aux:nnnnnnn
      { \zhnum_number_item:nn {#1} { \c_one + #4 + #6 * \c_four } }
      {#1} {#2} {#3} {#4} {#5} {#6}
  }
\cs_new_nopar:Nn \zhnum_split_number_aux:nnnnnnn
  {
    \int_compare:nNnTF {#1} = \c_zero { \use_i_ii:nnn }
      {
        \bool_if:NF #3 { \c_zhnum_zero_tl }
        \zhnum_process_number:NNn {#3} {#4} {#1}
        \zhnum_scale_map:n { #6 - #7 }
        \int_compare:nNnTF { \int_mod:nn {#1} \c_ten } = \c_zero
          { \use_i_ii:nnn } { \zhnum_use_i_iii:nnn }
      }
    { \int_compare:nNnF {#6} = {#7} }
    { {
        \zhnum_split_number:nNNnnf
          {#2} \c_false_bool \c_true_bool {#5} {#6} { \int_eval:n { #7 + \c_one } }
    } }
    { {
        \zhnum_split_number:nNNnnf
          {#2} \c_true_bool \c_false_bool {#5} {#6} { \int_eval:n { #7 + \c_one } }
    } }
  }
\cs_new_nopar:Nn \zhnum_use_i_iii:nnn {#1#3}
\cs_generate_variant:Nn \zhnum_split_number:nNNnnn { nNNnf , nNNff , nNNnnf }
%    \end{macrocode}
% \end{macro}
%
% \begin{macro}[internal]{\zhnum_number_item:nn,\zhnum_number_item_aux:nN}
% 截取整数的其中四位数。
%    \begin{macrocode}
\cs_new_nopar:Nn \zhnum_number_item:nn
  {
    \zhnum_number_item_aux:nN {#2} #1
    \q_recursion_tail
    \prg_break_point:n { }
  }
\cs_new_nopar:Nn \zhnum_number_item_aux:nN
  {
    \quark_if_recursion_tail_break:n {#2}
    \int_compare:nNnTF {#1} = \c_one
      { \zhnum_recursion_break:NNNNw #2 }
      { \zhnum_number_item_aux:fN { \int_eval:n { #1 - \c_one } } }
  }
\cs_generate_variant:Nn \zhnum_number_item_aux:nN { f }
\cs_new_nopar:Npn \zhnum_recursion_break:NNNNw #1#2#3#4#5 \prg_break_point:n #6 {#1#2#3#4}
%    \end{macrocode}
% \end{macro}
%
% \begin{macro}[internal]{\zhnum_process_number:NNn,\zhnum_process_number:NNNNNN}
% 对四位数字按情况进行处理。
%    \begin{macrocode}
\cs_new_nopar:Nn \zhnum_process_number:NNn
  {
    \zhnum_process_number:ffffNN
      { \int_mod:nn {#3} \c_ten }
      { \int_mod:nn { \int_div_truncate:nn {#3} \c_ten } \c_ten }
      { \int_mod:nn { \int_div_truncate:nn {#3} \c_one_hundred } \c_ten }
      { \int_div_truncate:nn {#3} \c_one_thousand }
      {#1} {#2}
  }
\cs_new_nopar:Nn \zhnum_process_number:NNNNNN
  {
    \int_compare:nNnTF {#4} = \c_zero
      { \bool_if:NF #6 { \c_zhnum_zero_tl } }
      { \zhnum_digit_map:n {#4} \c_zhnum_thousand_tl }
    \int_compare:nNnTF {#3} = \c_zero
      { \int_compare:nNnT { #4 * (#2#1) } > \c_zero { \c_zhnum_zero_tl } }
      {
        \bool_if:nTF
          { \l_zhnum_ancient_bool && \int_compare_p:nNn {#3} = \c_two }
          { \zhnum_digit_map:n { #3 00 } }
          { \zhnum_digit_map:n {#3} \c_zhnum_hundred_tl }
      }
    \int_compare:nNnTF {#2} = \c_zero
      { \int_compare:nNnT { #3 * #1 } > \c_zero { \c_zhnum_zero_tl } }
      {
        \bool_if:nF
          {
            \int_compare_p:nNn {#2}   = \c_one  &&
            \int_compare_p:nNn {#4#3} = \c_zero && #6 && #5
          }
          {
            \bool_if:nTF
              {
                \l_zhnum_ancient_bool                   &&
                ( \int_compare_p:nNn {#2} = \c_two   ||
                  \int_compare_p:nNn {#2} = \c_three ||
                  \int_compare_p:nNn {#2} = \c_four )
              }
              { \zhnum_digit_map:n { #2 0 } \use_none:n }
              { \zhnum_digit_map:n {#2} }
          }
        \c_zhnum_ten_tl
      }
    \int_compare:nNnF {#1} = \c_zero { \zhnum_digit_map:n {#1} }
  }
\cs_generate_variant:Nn \zhnum_process_number:NNn    { NNf  }
\cs_generate_variant:Nn \zhnum_process_number:NNNNNN { ffff }
%    \end{macrocode}
% \end{macro}
%
% \begin{macro}{\zhdigits}
% 将输入的数字输出为中文数字串输出。
%    \begin{macrocode}
\DeclareExpandableDocumentCommand \zhdigits { s o m }
  {
    \IfBooleanTF {#1}
      { \zhnum_digits_aux:nnN {#2} {#3} \zhnum_digits_zero_aux:N }
      { \zhnum_digits_aux:nnN {#2} {#3} \zhnum_digits_null_aux:N }
  }
\cs_new_nopar:Nn \zhnum_digits_aux:nnN
  {
    \IfNoValueTF {#1} { \tl_map_function:fN {#2} #3 }
      { \group_begin: \zhnumsetup {#1} \tl_map_function:fN {#2} #3 \group_end: }
  }
\cs_new_nopar:Nn \zhnum_digits_null:n { \tl_map_function:fN {#1} \zhnum_digits_null_aux:N }
\cs_generate_variant:Nn \zhnum_digits_null:n { V }
\cs_generate_variant:Nn \tl_map_function:nN  { f }
%    \end{macrocode}
% \end{macro}
%
% \begin{macro}[internal]
% {\zhnum_digits_null_aux:N,\zhnum_digits_zero_aux:N,\zhnum_digits_aux:NN}
% 将输入的数字输出为中文数字串输出。
%    \begin{macrocode}
\cs_new_nopar:Nn \zhnum_digits_null_aux:N { \zhnum_digits_aux:NN \c_true_bool  #1 }
\cs_new_nopar:Nn \zhnum_digits_zero_aux:N { \zhnum_digits_aux:NN \c_false_bool #1 }
\cs_new_nopar:Nn \zhnum_digits_aux:NN
  {
    \str_if_eq:xxTF {#2} . \c_zhnum_dot_tl
      {
        \str_if_eq:xxTF {#2} -
          { \c_zhnum_minus_tl }
          {
            \zhnum_if_number:NT {#2}
              {
                \bool_if:nTF { \int_compare_p:nNn {#2} = \c_zero }
                  { \bool_if:NTF #1 \c_zhnum_null_tl \c_zhnum_zero_tl }
                  { \zhnum_digit_map:n {#2} }
              }
          }
      }
  }
%    \end{macrocode}
% \end{macro}
%
% \begin{macro}{\zhdate}
% 输出中文日期。
%    \begin{macrocode}
\DeclareExpandableDocumentCommand \zhdate { s m }
  {
    \zhnum_date:w #2 \q_stop
    \IfBooleanT {#1} { \zhweekday {#2} }
  }
\cs_new_nopar:Npn \zhnum_date:w #1/#2/#3 \q_stop
  {
    \zhnum_check_time:Nn \zhnum_digits_null:n {#1} \c_zhnum_year_tl
    \zhnum_check_time:Nn \zhnum_integer:f {#2} \c_zhnum_month_tl
    \zhnum_check_time:Nn \zhnum_integer:f {#3} \c_zhnum_day_tl
  }
%    \end{macrocode}
% \end{macro}
%
% \begin{macro}{\zhtoday}
% 输出当天日期。
%    \begin{macrocode}
\cs_new_nopar:Npn \zhtoday
  {
    \zhnum_check_time:Nn \zhnum_digits_null:V \tex_year:D \c_zhnum_year_tl
    \zhnum_check_time:Nn \zhnum_integer:V \tex_month:D \c_zhnum_month_tl
    \zhnum_check_time:Nn \zhnum_integer:V \tex_day:D   \c_zhnum_day_tl
  }
%    \end{macrocode}
% \end{macro}
%
% \begin{macro}[internal]{\zhnum_check_time:Nn}
% 判断是用中文数字还是用阿拉伯数组。
%    \begin{macrocode}
\cs_new_nopar:Nn \zhnum_check_time:Nn
  { \bool_if:NTF \l_zhnum_time_bool {#1} { \int_eval:n } {#2} }
%    \end{macrocode}
% \end{macro}
%
% \begin{macro}{\zhweekday}
% 输出星期
%    \begin{macrocode}
\cs_new_nopar:Npn \zhweekday #1 { \zhnum_week_day:w #1 \q_stop }
%    \end{macrocode}
% \end{macro}
%
% \begin{macro}[internal]{\zhnum_Zeller:nnn,\zhnum_Zeller_aux:Nnnn,\zhnum_two_digits:n}
% 用 Zeller 公式\footnote{\url{http://en.wikipedia.org/wiki/Zeller's_congruence}}
% 计算星期几。
%    \begin{macrocode}
\cs_new_nopar:Nn \zhnum_Zeller:nnn
  {
    \int_compare:nNnTF
      { #1 \zhnum_two_digits:n {#2} \zhnum_two_digits:n {#3} } > { 1582 10 04 }
      { \zhnum_Zeller_aux:Nnnn \zhnum_Zeller_Gregorian:nnn }
      { \zhnum_Zeller_aux:Nnnn \zhnum_Zeller_Julian:nnn }
    {#1} {#2} {#3}
  }
\cs_new_nopar:Nn \zhnum_Zeller_aux:Nnnn
  {
    \int_compare:nNnTF {#3} < \c_three
      { #1 { #2 - \c_one } { #3 + \c_twelve } {#4} }
      { #1 {#2} {#3} {#4} }
  }
\cs_new_nopar:Nn \zhnum_two_digits:n
  {
    \int_compare:nNnT {#1} < \c_ten { 0 }
    \int_eval:n {#1}
  }
%    \end{macrocode}
% \end{macro}
%
% \begin{macro}[internal]{\zhnum_Zeller_Gregorian:nnn}
% 格里历(\zhdate{1582/10/15}及以后)的计算公式
% \[
%   h = \biggl(q + \biggl\lfloor\frac{26(m+1)}{10}\biggr\rfloor + Y +
%   \biggl\lfloor\frac Y4\biggr\rfloor + 6\biggl\lfloor\frac Y{100}\biggr\rfloor
%   + \biggl\lfloor\frac Y{400}\biggr\rfloor\biggr) \pmod 7
% \]
% 其中 $Y$ 为年,$m$ 为月,$q$ 为日;若 $m=1,2$,则令 $m:=m+12$,同时 $Y:=Y-1$。
%    \begin{macrocode}
\cs_new_nopar:Nn \zhnum_Zeller_Gregorian:nnn
  {
    \int_mod:nn
      {
          \int_eval:n {#3}
        + \int_div_truncate:nn { 26 * ( \int_eval:n {#2} + \c_one ) } \c_ten
        + \int_eval:n {#1}
        + \int_div_truncate:nn {#1} \c_four
        + \c_six * \int_div_truncate:nn {#1} \c_one_hundred
        + \int_div_truncate:nn {#1} { 400 }
      }
      { \c_seven }
  }
%    \end{macrocode}
% \end{macro}
%
% \begin{macro}[internal]{\zhnum_Zeller_Julian:nnn}
% 儒略历(\zhdate{1582/10/04}及以前)的计算公式
% \[
%   h = \biggl(q + \biggl\lfloor\frac{26(m+1)}{10}\biggr\rfloor + Y +
%   \biggl\lfloor\frac Y4\biggr\rfloor + 5\biggr) \pmod 7
% \]
%    \begin{macrocode}
\cs_new_nopar:Nn \zhnum_Zeller_Julian:nnn
  {
    \int_mod:nn
      {
          \int_eval:n {#3}
        + \int_div_truncate:nn { 26 * ( \int_eval:n {#2} + \c_one ) } \c_ten
        + \int_eval:n {#1}
        + \int_div_truncate:nn {#1} \c_four
        + \c_five
      }
      { \c_seven }
  }
%    \end{macrocode}
% \end{macro}
%
% \begin{macro}[internal]{\zhnum_week_day:w}
% 用 Zeller 公式计算的结果 $h$ 与实际星期的映射。
%    \begin{macrocode}
\cs_new_nopar:Npn \zhnum_week_day:w #1/#2/#3 \q_stop
  {
    \if_case:w \int_eval:w \zhnum_Zeller:nnn {#1} {#2} {#3} \int_eval_end:
           \c_zhnum_sat_tl
      \or: \c_zhnum_sun_tl
      \or: \c_zhnum_mon_tl
      \or: \c_zhnum_tue_tl
      \or: \c_zhnum_wed_tl
      \or: \c_zhnum_thu_tl
      \or: \c_zhnum_fri_tl
    \fi:
  }
%    \end{macrocode}
% \end{macro}
%
% \begin{macro}{\zhtime}
% 输出时间。
%    \begin{macrocode}
\cs_new_nopar:Npn \zhtime #1 { \zhnum_time:w #1 \q_stop }
\group_begin:
\char_set_lccode:nn { `\; } { `\: }
\tl_to_lowercase:n
  {
    \group_end:
    \cs_new_nopar:Npn \zhnum_time:w #1 ; #2 \q_stop
      {
        \zhnum_check_time:Nn \zhnum_integer:f {#1} \c_zhnum_hour_tl
        \zhnum_check_time:Nn \zhnum_integer:f {#2} \c_zhnum_minute_tl
      }
  }
%    \end{macrocode}
% \end{macro}
%
% \begin{macro}{\zhcurrtime}
% 输出当前时间。
%    \begin{macrocode}
\cs_new_nopar:Npn \zhcurrtime
  {
    \zhnum_check_time:Nn \zhnum_integer:f
      { \int_div_truncate:nn \tex_time:D { 60 } } \c_zhnum_hour_tl
    \zhnum_check_time:Nn \zhnum_integer:f
      { \int_mod:nn \tex_time:D { 60 } } \c_zhnum_minute_tl
  }
%    \end{macrocode}
% \end{macro}
%
% \begin{macro}[internal]{\zhnum_digit_map:n}
% 阿拉伯数字与中文数字的映射。
%    \begin{macrocode}
\cs_new_nopar:Nn \zhnum_digit_map:n
  { \tl_use:c { c_zhnum_ \zhnum_int_to_word:n {#1} _tl } }
%    \end{macrocode}
% \end{macro}
%
% \begin{macro}[internal]{\zhnum_scale_map:n,\zhnum_scale_map_loop:n}
% 大数系统的映射。
%    \begin{macrocode}
\cs_new_nopar:Nn \zhnum_scale_map:n
  {
    \cs_if_exist_use:cF { c_zhnum_scale_ \zhnum_int_to_word:n {#1} _tl }
      { \zhnum_scale_map_hook:n {#1} }
  }
\cs_new_nopar:Nn \zhnum_scale_map_loop:n
  { \zhnum_scale_map:n { \int_mod:nn {#1} \g_zhnum_scale_int } }
\int_new:N \g_zhnum_scale_int
\int_set_eq:NN \g_zhnum_scale_int \c_eleven
\cs_new_eq:NN \zhnum_scale_map_hook:n \zhnum_scale_map_loop:n
%    \end{macrocode}
% \end{macro}
%
% \begin{macro}{\zhnumExtendScaleMap}
%    \begin{macrocode}
\NewDocumentCommand \zhnumExtendScaleMap { > { \TrimSpaces } o m }
  {
   \int_zero:N \l_tmpa_int
    \clist_map_inline:nn {#2}
      {
        \int_incr:N \l_tmpa_int
        \tl_set:Nx \l_tmpa_tl
          { c_zhnum_scale_ \zhnum_int_to_word:n {  \l_tmpa_int + \c_eleven } _tl }
        \tl_if_exist:cF \l_tmpa_tl { \int_incr:N \g_zhnum_scale_int }
        \tl_set:cn { \l_tmpa_tl } {##1}
      }
    \IfValueT {#1} { \cs_set:Nn \zhnum_scale_map_hook:n {#1} }
  }
%    \end{macrocode}
% \end{macro}
%
% \begin{macro}[internal]{\zhnum_int_to_word:n}
% 将整数转换成英文单词。
%    \begin{macrocode}
\cs_new_nopar:Nn \zhnum_int_to_word:n
  {
    \if_case:w \int_eval:w #1 \int_eval_end:
           zero
      \or: one
      \or: two
      \or: three
      \or: four
      \or: five
      \or: six
      \or: seven
      \or: eight
      \or: nine
      \or: ten
      \or: eleven
    \else:
      \prg_case_int:nnn {#1}
        {
          { 20 } { twenty }  { 30  } { thirty }
          { 40 } { forty }   { 200 } { two_hundred }
        }
        { \int_to_roman:n {#1} }
    \fi:
  }
%    \end{macrocode}
% \end{macro}

% 根据需要设置中文阿拉伯数字。
%    \begin{macrocode}
\keys_define:nn { zhnum / options }
  {
    -   .tl_set:N = \l_zhnum_minus_tl        ,
    -0  .tl_set:N = \l_zhnum_null_tl         ,
    E2  .tl_set:N = \l_zhnum_hundred_tl      ,
    E3  .tl_set:N = \l_zhnum_thousand_tl     ,
    FE2 .tl_set:N = \l_zhnum_financial_hundred_tl ,
    FE3 .tl_set:N = \l_zhnum_financial_thousand_tl ,
  }
\clist_map_inline:nn
  { 0 , 1 , 2 , 3 , 4 , 5 , 6 , 7 , 8 , 9 , 10 , 20 , 30 , 40 , 200 }
  {
    \keys_define:nn { zhnum / options }
      { #1 .tl_set:c = { l_zhnum_ \zhnum_int_to_word:n {#1} _tl } }
    \int_compare:nNnF {#1} > \c_ten
      {
        \keys_define:nn { zhnum / options }
          { F#1 .tl_set:c = { l_zhnum_financial_ \zhnum_int_to_word:n {#1} _tl } }
      }
  }
\clist_map_inline:nn
  { 4 , 8 , 12 , 16 , 20 , 24 , 28 , 32 , 36 , 40 , 44 }
  {
    \keys_define:nn { zhnum / options }
      { E#1 .tl_set:c = { l_zhnum_scale_ \zhnum_int_to_word:n { #1 / 4 } _tl } }
  }
\clist_map_inline:nn
  {
    dot , and , parts , year , month , day , weekday , hour , minute
    mon , tue , wed , thu , fri , sat , sun
  }
  { \keys_define:nn { zhnum / options } { #1 .tl_set:c = { l_zhnum_ #1 _tl } } }
%    \end{macrocode}
%
% \begin{macro}[internal]
% {\zhnum_parse_config:,\zhnum_check_simp:nn,\zhnum_check_financial:nn,
%  \zhnum_set_zero:,\zhnum_set_week_day:}
% 将配置文件中的中文数字保存起来。
%    \begin{macrocode}
\cs_new_nopar:Nn \zhnum_parse_config:
  {
    \prop_map_function:NN \g_zhnum_cfg_map_prop \zhnum_check_simp:nn
    \prop_map_function:NN \g_zhnum_cfg_map_prop \zhnum_check_financial:nn
    \zhnum_set_zero:
    \zhnum_set_week_day:
  }
\cs_new_nopar:Nn \zhnum_check_simp:nn
  {
    \zhnum_check_simp_aux:nn {#1} {#2}
    \prop_get:NnNT \g_zhnum_cfg_map_finan_prop {#1} \l_tmpa_tl
      { \exp_args:NnV \zhnum_check_simp_aux:nn { financial_ #1 } \l_tmpa_tl }
  }
\cs_new_nopar:Nn \zhnum_check_simp_aux:nn
  {
    \prop_get:NnNTF \g_zhnum_cfg_map_var_prop { #1 _trad } \l_tmpa_tl
      {
        \prop_get:NnNTF \g_zhnum_cfg_map_var_prop { #1 _simp } \l_tmpb_tl
          {
            \tl_set:cx { l_zhnum_ #1 _tl }
              {
                \exp_not:n { \bool_if:NTF \l_zhnum_simp_bool }
                  { \exp_not:V \l_tmpb_tl } { \exp_not:V \l_tmpa_tl }
              }
          }
          {
            \tl_set:cx { l_zhnum_ #1 _tl }
              {
                \exp_not:n { \bool_if:NTF \l_zhnum_simp_bool }
                  { \exp_not:n {#2} } { \exp_not:V \l_tmpa_tl }
              }
          }
      }
      { \tl_set:cn { l_zhnum_ #1 _tl } {#2} }
  }
\cs_new_nopar:Nn \zhnum_check_financial:nn
  {
    \prop_get:NnNTF \g_zhnum_cfg_map_finan_prop {#1} \l_tmpa_tl
      {
        \tl_set:cx { c_zhnum_ #1 _tl }
          {
            \exp_not:n { \bool_if:NTF \l_zhnum_normal_bool }
              { \exp_not:c { l_zhnum_ #1 _tl } }
              { \exp_not:c { l_zhnum_financial_ #1 _tl } }
          }
      }
      { \tl_set:cx { c_zhnum_ #1 _tl } { \exp_not:c { l_zhnum_ #1 _tl } } }
  }
\cs_new_nopar:Nn \zhnum_set_zero:
  {
    \tl_set:Nx \l_zhnum_zero_tl
      {
        \exp_not:n { \bool_if:nTF \l_zhnum_null_bool }
          { \exp_not:V \l_zhnum_null_tl } { \exp_not:V \l_zhnum_zero_tl }
      }
  }
\cs_new_nopar:Nn \zhnum_set_week_day:
  {
    \tl_set:Nx \l_zhnum_mon_tl { \exp_not:N \c_zhnum_weekday_tl \exp_not:V \l_zhnum_one_tl   }
    \tl_set:Nx \l_zhnum_tue_tl { \exp_not:N \c_zhnum_weekday_tl \exp_not:V \l_zhnum_two_tl   }
    \tl_set:Nx \l_zhnum_wed_tl { \exp_not:N \c_zhnum_weekday_tl \exp_not:V \l_zhnum_three_tl }
    \tl_set:Nx \l_zhnum_thu_tl { \exp_not:N \c_zhnum_weekday_tl \exp_not:V \l_zhnum_four_tl  }
    \tl_set:Nx \l_zhnum_fri_tl { \exp_not:N \c_zhnum_weekday_tl \exp_not:V \l_zhnum_five_tl  }
    \tl_set:Nx \l_zhnum_sat_tl { \exp_not:N \c_zhnum_weekday_tl \exp_not:V \l_zhnum_six_tl   }
    \tl_set:Nx \l_zhnum_sun_tl { \exp_not:N \c_zhnum_weekday_tl \exp_not:V \l_zhnum_day_tl   }
    \clist_map_inline:nn { mon , tue , wed , thu , fri , sat , sun }
      { \tl_set:cx { c_zhnum_ ##1 _tl } { \exp_not:c { l_zhnum_ ##1 _tl } } }
  }
%    \end{macrocode}
% \end{macro}
%
% \begin{macro}[internal]{\zhnum_load_cfg:}
% 根据选定编码载入配置文件。
%    \begin{macrocode}
\cs_new:Nn \zhnum_load_cfg:n
  {
    \cs_if_exist:NT \CJK@makeActive
      {
        \int_compare:nNnTF { \char_value_catcode:n {"080} } = \active
          { \bool_set_false:N \l_zhnum_set_CJK_active_bool }
          { \bool_set_true:N \l_zhnum_set_CJK_active_bool \CJK@makeActive }
      }
    \prop_clear:N \g_zhnum_cfg_map_prop
    \prop_clear:N \g_zhnum_cfg_map_var_prop
    \prop_clear:N \g_zhnum_cfg_map_finan_prop
    \file_input:n { zhnumber - #1 .cfg }
    \zhnum_parse_config:
    \bool_if:nT { \cs_if_exist_p:N \CJK@makeInactive && \l_zhnum_set_CJK_active_bool }
      { \CJK@makeInactive }
  }
%    \end{macrocode}
% \end{macro}
%
% \begin{macro}[internal]
% {\zhnum_set_digits_map:nn,\zhnum_set_digits_map:nnn,
%  \zhnum_set_financial_map:nn,\zhnum_set_financial_map:nnn}
%    \begin{macrocode}
\cs_new_protected_nopar:Nn \zhnum_set_digits_map:nn
  { \prop_put:Nnn \g_zhnum_cfg_map_prop {#1} {#2} }
\cs_new_protected_nopar:Nn \zhnum_set_digits_map:nnn
  {
    \prop_put_if_new:Nnn \g_zhnum_cfg_map_prop {#1} {#3}
    \prop_put:Nnn \g_zhnum_cfg_map_var_prop {#1_#2} {#3}
  }
\cs_new_protected_nopar:Nn \zhnum_set_financial_map:nn
  { \prop_put:Nnn \g_zhnum_cfg_map_finan_prop {#1} {#2} }
\cs_new_protected_nopar:Nn \zhnum_set_financial_map:nnn
  {
    \prop_put_if_new:Nnn \g_zhnum_cfg_map_finan_prop {#1} {#3}
    \prop_put:Nnn \g_zhnum_cfg_map_var_prop { financial_#1_#2 } {#3}
  }
%    \end{macrocode}
% \end{macro}
%
% \begin{macro}[internal]
% {\g_zhnum_cfg_map_prop,\g_zhnum_cfg_map_var_prop,\g_zhnum_cfg_map_finan_prop}
%    \begin{macrocode}
\prop_new:N \g_zhnum_cfg_map_prop
\prop_new:N \g_zhnum_cfg_map_var_prop
\prop_new:N \g_zhnum_cfg_map_finan_prop
%    \end{macrocode}
% \end{macro}
%
% \begin{macro}{encoding,style,null,reset}
% 宏包设置选项。
%    \begin{macrocode}
\keys_define:nn { zhnum / options }
  {
    encoding      .choice_code:n =
      {
        \tl_gset:Nx \c_zhnum_encoding_tl
          { \exp_args:NV \tl_expandable_lowercase:n \l_keys_choice_tl }
        \zhnum_load_cfg:n { \c_zhnum_encoding_tl }
      } ,
    encoding .generate_choices:n = { UTF8 , GBK , Big5 } ,
    encoding          .default:n = { GBK } ,
    encoding / Bg5       .meta:n = { encoding = Big5 } ,
    encoding / unknown   .code:n =
      { \msg_error:nnx { zhnumber } { encoding-invalid } { \l_keys_value_tl } } ,
    style .multichoice: ,
    style / Normal       .code:n =
      {
        \bool_set_false:N \l_zhnum_ancient_bool
        \bool_set_true:N  \l_zhnum_normal_bool
      } ,
    style / Financial    .code:n =
      {
        \bool_set_false:N \l_zhnum_ancient_bool
        \bool_set_false:N \l_zhnum_normal_bool
      } ,
    style / Ancient      .code:n =
      {
        \bool_set_true:N \l_zhnum_ancient_bool
        \bool_set_true:N \l_zhnum_normal_bool
      } ,
    style / Simplified   .code:n = { \bool_set_true:N  \l_zhnum_simp_bool } ,
    style / Traditional  .code:n = { \bool_set_false:N \l_zhnum_simp_bool } ,
    style             .default:n = { Normal , Simplified } ,
    null             .bool_set:N = \l_zhnum_null_bool ,
    time .choice: ,
    time / Chinese       .code:n = { \bool_set_true:N \l_zhnum_time_bool } ,
    time / Arabic        .code:n = { \bool_set_false:N  \l_zhnum_time_bool } ,
    time              .default:n = { Arabic } ,
    reset                .code:n = { \zhnum_load_cfg:n { \c_zhnum_encoding_tl } } ,
  }
\msg_new:nnn { zhnumber } { encoding-invalid }
  {
    The~encoding~'#1'~is~invalid,~please~check~it~over.\\\\
    Available~encoding~are~'UTF8',~'GBK'~and~'Big5'.\\
  }
%    \end{macrocode}
% \end{macro}
%
% \begin{macro}{\zhnumsetup}
% 在文档中设置 \pkg{zhnumber} 的接口。
%    \begin{macrocode}
\NewDocumentCommand \zhnumsetup { m }
  {
    \keys_set:nn { zhnum / options } {#1}
    \tex_ignorespaces:D
  }
%    \end{macrocode}
% \end{macro}
%
% 初始化设置和执行宏包选项。
%    \begin{macrocode}
\keys_set:nn { zhnum / options } { style , time }
\ProcessKeysOptions { zhnum / options }
%    \end{macrocode}
%
% 如果没有选定编码,则根据引擎自动设置编码。
%    \begin{macrocode}
\ExplSyntaxOn
\tl_if_exist:NF \c_zhnum_encoding_tl
  {
    \exp_args:Nnx \keys_set:nn { zhnum / options }
      {
        encoding =
          {
            \bool_if:nTF { \xetex_if_engine_p: || \luatex_if_engine_p: }
              { UTF8 } { GBK }
          }
      }
  }
\ExplSyntaxOff
%    \end{macrocode}
%
%    \begin{macrocode}
%</package>
%    \end{macrocode}
%
% \section{中文数字配置文件}
% \label{sec:zhnum-map}
%
%    \begin{macrocode}
%<*config>
%    \end{macrocode}
%
%    \begin{macrocode}
%<*!big5>
\zhnum_set_digits_map:nnn { minus } { simp }  { 负 }
\zhnum_set_digits_map:nnn { minus } { trad }  { 負 }
%</!big5>
%<*big5>
\zhnum_set_digits_map:nn { minus }       { 負 }
%</big5>
\zhnum_set_digits_map:nn { zero }        { 零 }
%<*!big5>
\zhnum_set_digits_map:nn { null }        { 〇 }
%</!big5>
%<*big5>
\zhnum_set_digits_map:nn { null }        { ○ }
%</big5>
\zhnum_set_digits_map:nn { one }         { 一 }
\zhnum_set_digits_map:nn { two }         { 二 }
\zhnum_set_digits_map:nn { three }       { 三 }
\zhnum_set_digits_map:nn { four }        { 四 }
\zhnum_set_digits_map:nn { five }        { 五 }
\zhnum_set_digits_map:nn { six }         { 六 }
\zhnum_set_digits_map:nn { seven }       { 七 }
\zhnum_set_digits_map:nn { eight }       { 八 }
\zhnum_set_digits_map:nn { nine }        { 九 }
\zhnum_set_digits_map:nn { ten }         { 十 }
\zhnum_set_digits_map:nn { hundred }     { 百 }
\zhnum_set_digits_map:nn { thousand }    { 千 }
\zhnum_set_digits_map:nn { twenty }      { 廿 }
\zhnum_set_digits_map:nn { thirty }      { 卅 }
\zhnum_set_digits_map:nn { forty }       { 卌 }
\zhnum_set_digits_map:nn { two_hundred } { 皕 }
%<*!big5>
\zhnum_set_digits_map:nnn { dot } { simp } { 点 }
\zhnum_set_digits_map:nnn { dot } { trad } { 點 }
%</!big5>
%<*big5>
\zhnum_set_digits_map:nn { dot }   { 點 }
%</big5>
\zhnum_set_digits_map:nn { and }   { 又 }
\zhnum_set_digits_map:nn { parts } { 分之 }
\zhnum_set_digits_map:nn { scale_zero }   { }
%<*!big5>
\zhnum_set_digits_map:nnn { scale_one } { simp } { 万 }
\zhnum_set_digits_map:nnn { scale_one } { trad } { 萬 }
\zhnum_set_digits_map:nnn { scale_two } { simp } { 亿 }
\zhnum_set_digits_map:nnn { scale_two } { trad } { 億 }
%</!big5>
%<*big5>
\zhnum_set_digits_map:nn { scale_one }    { 萬 }
\zhnum_set_digits_map:nn { scale_two }    { 億 }
%</big5>
\zhnum_set_digits_map:nn { scale_three }  { 兆 }
\zhnum_set_digits_map:nn { scale_four }   { 京 }
\zhnum_set_digits_map:nn { scale_five }   { 垓 }
\zhnum_set_digits_map:nn { scale_six }    { 秭 }
\zhnum_set_digits_map:nn { scale_seven }  { 穰 }
%<*!big5>
\zhnum_set_digits_map:nnn { scale_eight } { simp }  { 沟 }
\zhnum_set_digits_map:nnn { scale_eight } { trad }  { 溝 }
\zhnum_set_digits_map:nnn { scale_nine  } { simp }  { 涧 }
\zhnum_set_digits_map:nnn { scale_nine  } { trad }  { 澗 }
%</!big5>
%<*big5>
\zhnum_set_digits_map:nn { scale_eight }  { 澗 }
%</big5>
\zhnum_set_digits_map:nn { scale_ten }    { 正 }
%<*!big5>
\zhnum_set_digits_map:nnn { scale_eleven } { simp } { 载 }
\zhnum_set_digits_map:nnn { scale_eleven } { trad } { 載 }
%</!big5>
%<*big5>
\zhnum_set_digits_map:nn { scale_eleven } { 載 }
%</big5>
\zhnum_set_digits_map:nn { year }    { 年 }
\zhnum_set_digits_map:nn { month }   { 月 }
\zhnum_set_digits_map:nn { day }     { 日 }
%<*!big5>
\zhnum_set_digits_map:nnn { hour } { simp } { 时 }
\zhnum_set_digits_map:nnn { hour } { trad } { 時 }
%</!big5>
%<*big5>
\zhnum_set_digits_map:nn { hour }    { 時 }
%</big5>
\zhnum_set_digits_map:nn { minute }  { 分 }
\zhnum_set_digits_map:nn { weekday } { 星期 }
\zhnum_set_financial_map:nn { null }     { 零 }
\zhnum_set_financial_map:nn { zero }     { 零 }
\zhnum_set_financial_map:nn { one }      { 壹 }
\zhnum_set_financial_map:nn { two }      { 貳 }
%<*!big5>
\zhnum_set_financial_map:nnn { three } { simp } { 叁 }
\zhnum_set_financial_map:nnn { three } { trad } { 叄 }
%</!big5>
%<*big5>
\zhnum_set_financial_map:nn { three }    { 參 }
%</big5>
\zhnum_set_financial_map:nn { four }     { 肆 }
\zhnum_set_financial_map:nn { five }     { 伍 }
%<*!big5>
\zhnum_set_financial_map:nnn { six }   { simp } { 陆 }
\zhnum_set_financial_map:nnn { six }   { trad } { 陸 }
%</!big5>
%<*big5>
\zhnum_set_financial_map:nn { six }      { 陸 }
%</big5>
\zhnum_set_financial_map:nn { seven }    { 柒 }
\zhnum_set_financial_map:nn { eight }    { 捌 }
\zhnum_set_financial_map:nn { nine }     { 玖 }
\zhnum_set_financial_map:nn { ten }      { 拾 }
\zhnum_set_financial_map:nn { hundred }  { 佰 }
\zhnum_set_financial_map:nn { thousand } { 仟 }
%    \end{macrocode}
%
%    \begin{macrocode}
\ExplSyntaxOff
%    \end{macrocode}
%
%    \begin{macrocode}
%</config>
%    \end{macrocode}
%
% \end{implementation}
%
% \PrintIndex
% \Finale
%
\endinput
