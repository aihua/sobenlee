%# -*- coding:utf-8 -*-
%%%%%%%%%%%%%%%%%%%%%%%%%%%%%%%%%%%%%%%%%%%%%%%%%%%%%%%%%%%%%%%%%%%%%%%%%%%%%%%%%%%%%
%%  MAChapter4.tex'


\chapter{导数与微分}\label{ch:4}

从微积分发展的历史来看,先有积分概念,后有导数概念,最后才有极限理论。但从教学顺序来说,我们正好把它倒过来,先讲极限,然后
讲导数、积分。

导数概念最早是法国数学家~Fermat,为了解决极大、极小问题,引入了导数的思想。但与导数概念密切相联的是以下两个问题:已知运动规
律求运动的速度和已知曲线求它的切线。Newton~和~Leibniz~就是在研究这些问题中,发现了微积分这一重要工具。

例如,行星绕太阳作椭圆轨道运行,运行的速度是在不断变化的。天文学家~Kepler~经二十年的观察,总结出行星运行的三大定律,Newton~就
是根据三大定律发现了万有引力定律,并由万有引力定律出发,利用他总结得到的微积分这一工具,从理论上证明了~Kepler~三大定律,这是
微积分发展史上第一次显示了它的威力。若除太阳的引力外,再考虑木星、土星、天王星之间的相互引力,算出的椭圆轨道的偏差,对木星和
土星与观察所得是一致的,而对天王星存在不可忽视的差别。Hadamard~与~Le Verrier~提出解决这一矛盾的设想:存在一颗未被发现的行星干
扰着天王星的运行。他们经过大量的计算,独立地推算出新行星的方位,果然在他们所指示的方位上,找到了这颗行星——海王星。这是微积
分发展史上一个重大的胜利,也使微积分理论真正站住了脚。又例如,天文观察中离不开望远镜,Galileo~发明了第一架天文望远镜,为了改
进望远镜,就需要研究曲线的法线,而求曲线的法线相当于求曲线的切线。所以求一般曲线的切线问题,也导致了导数概念的建立.

Newton~时期的导数概念,建立在“无穷小”的基础之上。“无穷小”是个似零非零的量,在逻辑上并未说清楚。所以,一方面由于微积分在
应用中显示的威力,获得了迅猛的发展;一方面对它的基础——“无穷小”展开了激烈的争论。直到十八世纪有了极限理论之后,“无穷小”是
作为极限为零的变量,微积分才在逻辑上有了坚实的基础。


\section{导数概念}
\begin{exercise}
\item
\end{exercise}
\section{导数的几何意义与极限}
\begin{exercise}
\item
\end{exercise}
\section{导数的四则运算}
\begin{exercise}
\item
\end{exercise}
\section{复合函数求导}
\subsection{复合函数求导}
\subsection{隐函数微分法}
\subsection{对数微分法}
\begin{exercise}
\item
\end{exercise}
\section{反函数与参数式求导}
\subsection{反函数求导}
\subsection{参数式求导}
\subsection{极坐标式求导}
\begin{exercise}
\item
\end{exercise}
\section{微\emspace 分}
\subsection{微分定义}
\subsection{微分与导数}
\subsection{微分的几何意义}
\subsection{一阶微分形式的不变形}
\subsection{微分的应用}
\begin{exercise}
\item
\end{exercise}
\section{高阶导数与高阶微分}
\subsection{高阶导数}
\subsection{Leibniz~公式}
\subsection{其它函数关系的高阶导数}
\subsection{高阶微分}
\begin{exercise}
\item
\end{exercise}
\begin{exercise*}
\item
\end{exercise*}


\endinput
%%
%% End of file `MAChapter4.tex'.