%# -*- coding:utf-8 -*-
%%%%%%%%%%%%%%%%%%%%%%%%%%%%%%%%%%%%%%%%%%%%%%%%%%%%%%%%%%%%%%%%%%%%%%%%%%%%%%%%%%%%%
%%  MAChapter0.tex'


\chapter{预备知识}\label{ch:0}

\section{逻辑符号}

为了书写方便,我们常采用以下一些逻辑符号。

设~$S_1,S_2$~是两个陈述句,它们可以指命题也可以指条件。符号
\[
  S_1\implies S_2
\]
表示命题~$S_1$~成立,则命题~$S_2$~成立;或条件~$S_1$~成立,则条件~$S_2$~也成立。符号
\[
  S_1\iff S_2
\]
表示命题(或条件)~$S_1$~与命题(或条件)~$S_2$~等价。即表示由命题(或条件)~$S_1$~可以推出命题(或条
件)~$S_2$,反过来由命题(或条件)~$S_2$~也可以推出命题(或条件)~$S_1$。

符号
\[
  \forall
\]
表示任意取定,写法是将英文字母~A~倒过来。符号
\[
  \exists
\]
表示存在,写法是将英文字母~E~反过来。

孤立地看这些符号没有什么意思,但组合起来可以表示一句话,这句话可以是正确的,也可以是错误的。如
\begin{alignat*}{4}
&\exists x,\text{使得~}\dfrac 12<x<\dfrac 52 &\enspace&\text{(正确的);}
&\qquad&\forall x,\exists y,\text{使得~}x+y=1 &\enspace&\text{(正确的);}\\
&\forall x,\exists y,\text{使得~}x>y & &\text{(正确的);}
&\qquad&\exists x,\forall y,\text{使得~}x>y & &\text{(错误的)。}
\end{alignat*}
后一句话是说,可以找到一实数,它比任何实数都大,这显然是错误的。

\section{集合初步}

自~Cantor~在十九世纪末创建集合论以来,集合论的概念和方法已渗入到数学的各个分支,成为数学的一种语言。集合论本身也发展
成数学的一个分支,内容十分丰富。

集合不能给予严格的定义,因为定义是用已知的概念去定义未知的概念。如用有理数去定义无理数,这里我们认为有理数是已知的,若
有人喜欢刨根问底,觉得有理数是什么也不清楚,我们可以用整数来定义有理数,进一步用自然数来定义整数,用集合来定义自然数。%
这个过程不可能无穷无尽下去,总有一个概念不能定义,在数学里集合概念就到头了,不能再用其它的数学概念来定义。虽然如此,我
们可以给集合一个描述。先看几个集合的例子:

\begin{enumerate}
\item 所有自然数的全体为一集合,记作~$\mathbb N$;\label{enum:set1}
\item 所有小于~$10$~并且是偶数的自然数全体为一集合;\label{enum:set2}
\item 方程~$x^2+5x+4=0$~的根全体为一集合;\label{enum:set3}
\item 具有北京市户口的人全体为一集合。\label{enum:set4}
\end{enumerate}

尽管集合没有定义,但我们能理解到它是什么意思。一般来说,把具有某种共同特征的事物的全体叫\emph{集合},属于集合的每个个
体叫作该集合的\emph{元素}。

如例~\ref{enum:set1}~中集合的特性是正整数,例~\ref{enum:set4}~中集合的特性是具有北京市户口。根据给定的特性,我们可以判
断每一个元素是属于这个集合,还是不属于这个集合。前面三个例子是数集,例~\ref{enum:set4}~是非数集。以后我们只讨论数集。

集合用大写字母~$A,C,C,X,Y,Z$~表示,元素用小写字母~$a,b,c,x,y,z$~表示。

设~$A$~是一个集合,$a$~是~$A$~的元素,记作
\[
  a\in A;
\]
反之,$a$~不是~$A$~的元素,记作
\[
  a\notin A\quad\text{(或~$a\notinslant A$)。}
\]

\subsection{集合表示法}

集合有两种表示法:一是列举法,如集合
\[
  A=\Brace{2,4,6,8},
\]
这种表示法是将集合的元素在花括弧内一一列举出来;另一是描述法,如集合
\[
  A=\Setb x{x^2-5x+4=0},
\]
这种表示法将花括弧分两部分,用记号~$\mid$~隔开,前面为元素的代表符号,用~$x$~或其它符号,后面为元素具有的性质。

第一种表示法在数学分析中用处不大,因为我们常用的集合为无穷个元素组成,无法一一列出,如所有实数的集合就不可能
写出来。

集合~$\Set x{a\leq x\leq b}$~称为\emph{闭区间},记作~$\Brack{a,b}$;集合~$\Set x{a<x<b}$~称为\emph{开区间},记
作~$\Paren{a,b}$;记号~$\Paren{a,+\infty}$,$[a,+\infty)$,$\Paren{-\infty,+\infty}$~可作类似理解。

\subsection{集合的子集、包含、相等}

两个集合~$A,B$,若对任意~$a\in A$,都有~$a\in B$,这时集合~$A$~包含于集合~$B$,称~$A$~是~$B$~的\emph{子集},记作
\[
  A\subset B\quad\text{或}\quad B\supseteq A\.
\]

例如~$A=\Brace{1,2,3,4,5}$,则集合
\[
  \Brace 1,\quad\Brace{1,2},\quad\Brace{1,3,5}
\]
是~$A$~的子集。集合~$\Brace 1$~表示由~$1$~这一元素组成的集合,概念上不同于元素~$1$~本身,我们可以记
\[
  \Brace 1\subset A,\quad 1\in A\.
\]

为了运算方便,我们把不含任何元素的集合称为\emph{空集},记作~$\eset$。例如
\[
  \Setb x{x^2+1=0,~x~\text{是实数}}=\eset\.
\]
空集包含于任一集合:
\[
  \eset\subset A,
\]
因为,如果不成立,则至少有一元素属于~$\eset$~而不属于~$A$,显然这是不可能的。

根据集合包含关系~$\subset$~的定义,显然有:$A\subset A$;若~$A\subset B$,$B\subset C$,则~$A\subset C$。

若两集合~$A,B$,满足~$A\subset B$,且~$B\subset A$,则称~$A$~与~$B$~相等,记作
\[
  A=B\.
\]
如
\[
  \Brace{1,2,3,4}=\Brace{4,3,2,1}=\Brace{1,1,2,3,4}\.
\]

若~$A\subset B$,且~$A\neq B$,则称~$A$~是~$B$~的真子集。

\subsection{集合的运算}

集合除包含关系外,还可以考虑集合之间的并、交、差等运算。

给定集合~$A,B$,集合~$A,B$~的\emph{并}记为~$A\cup B$,它是~$A,B$~全部元素组成的集合,定义为
\[
  A\cup B:=\Setb x{x\in A~\text{或}~x\in B}\.
\]

用平面图形表示集合,图形的点表示集合的元素,则~$A,B$~图形和在一起就是并集~$A\cup B$~的图形。如\ref{fig:sec0-1}~所示。

由并集的定义易见:$A\cup\eset=A$;$A\cup A=A$;
\begin{align*}
A\cup B &=B\cup A;                     \tag*{(交换律)}\\*
(A\cup B)\cup C & = A\cup(B\cup C)\.   \tag*{(结合律)}
\end{align*}

给定集合~$A,B$,两集合的\emph{交}记为~$A\cap B$,它由~$A,B$~的公共元素组成,定义为:
\[
  A\cap B:=\Setb x{x\in A~\text{与}~x\in B}\.
\]
集合~$A,B$~图形的公共部分(见\ref{fig:sec0-2})就是交集的图形。

显然有:$A\cap\eset=\eset$;$A\cap A=A$;
\begin{align*}
A\cap B &=B\cap A;                     \tag*{(交换律)}\\*
(A\cap B)\cup C & = A\cap(B\cap C)\.   \tag*{(结合律)}
\end{align*}
并且,若~$A\subset B$,则~$A\cap B=A$。

\begin{figure}
\begin{floatrow}[3]
\figurebox{\caption{并集}\label{fig:sec0-1}}
          {\somefigure}
\figurebox{\caption{交集}\label{fig:sec0-2}}
          {\somefigure}
\figurebox{\caption{补集}\label{fig:sec0-3}}
          {\somefigure}
\end{floatrow}
\end{figure}

给定集合~$A,B$,两集合的\emph{差}记为~$A-B$~或~$A\backslash B$,它是在~$A$~内而不在~$B$~内的元素组成的集
合(见\ref{fig:sec0-3}),定义为:
\[
  A-B=\Setb x{x\in A,x\notin B}\.
\]
显然有
\[
  A-A=\eset;\quad A-\eset=A;\quad\eset-A=\eset;\quad A-B=A-(A\cap B)\.
\]

假设我们所考察的集合都是更大集合~$X$~(如实数集)的子集,这时我们把~$A$~对~$X$~的差集称为~$A$~的\emph{补集},记
为~$\cset A$,即
\[
  \cset A=X-A\.
\]

\begin{quiz}
证明:
\[
  \cset A\cap\cset B=\cset{(A\cup B)};\quad
  \cset A\cup\cset B=\cset{(A\cap B)}\.
\]
\end{quiz}

\section{绝对值与不等式}

设~$x$~是实数,$x$~的\emph{绝对值}为一非负实数,记为~$\abs x$,定义为:
\[
  \abs x=\biggl\{\begin{array}{@{}r@{\quad}l}
    x, & x\geq 0;\\
   -x, & x<0\.
  \end{array}
\]
例如~$\abs{3.5}=3.5$,而~$\abs{-3.5}=3.5$。

要说明绝对值的几何意义,我们作一直线,在直线上取定方向,原点~$O$~和单位长度以后,就称此直线为一\emph{数轴},实数
可以与数轴上的点建立起一一对应,每一实数可用数轴上一点来表示,不同的实数用数轴上不同的点表示,因此数与点可以不加
区分。这时,$\abs x$~就表示点~$x$~到原点~$O$~的距离。

\begin{wrapfigure}[7]{O}{0mm}
\somefigure
\caption{数轴}\label{fig:sec0-4}
\end{wrapfigure}

由\ref{fig:sec0-4}~可见:若~$r>0$,点~$x$~位于区间~$(-r,r)$~上时,则点~$x$~到原点的距离小于~$r$;反之,若点~$x$~到原点的距离
小于~$r$,则点~$x$~位于区间~$(-r,r)$~上,即得

\begin{property}
若~$r>0$,则
\[
  \abs x<r\iff -r<x<r\.
\]
\end{property}

容易证明

\begin{property}
给定实数~$x,y$,有
\[
  \abs{x+y}\leq\abs x+\abs y,
\]
当且仅当~$x$~与~$y$~同号时,上式等号成立。
\end{property}

用数学归纳法容易得到
\begin{corollary}
对任意~$a_1,a_2,\dotsc,a_n\in\FR$,有
\[
  \abs[\Big]{\sum_{k=1}^na_k}\leq\sum_{k=1}^n\abs{a_k}\.
\]
\end{corollary}
\begin{corollary}
\[
  \absb{\abs x+\abs y}\leq\abs{x-y}\.
\]
\end{corollary}
\begin{proof}
由
\[
  \abs x=\abs{x-y+y}\leq\abs{x-y}+\abs y,
\]
可得
\[
  \abs x-\abs y\leq\abs{x-y}\.
\]
同理
\[
  \abs y-\abs x\leq\abs{y-x}=\abs{x-y},
\]
所以
\[
  \absb{\abs x+\abs y}\leq\abs{x-y}\.\qedhere
\]
\end{proof}

\begin{property}
\[
  \abs{x\cdot y}=\abs x\cdot\abs y\.
\]
\end{property}

分情况讨论易知上式成立。

需要注意的是
\[
  \sqrt{x^2}=\abs x,
\]
因为左边取算术平方根是一个非负实数。

\begin{exercise*}
\item 用数学归纳法证明下列各题。
  \begin{exlist}
  \item $\dps\frac 1{1\cdot 2}+\frac 1{2\cdot3}+\dotsb+\frac1{n(n+1)}=1-\frac1{n+1}$;
  \item $1^2+2^2+\dotsb+n^2=\dfrac 1n(n+1)(2n+1)$;
  \item $\abs{\sin nx}\leq n\abs{\sin x}$;
  \item $\cos\alpha\cdot\cos2\alpha\cdot\dotsm\cdot\cos 2^n\alpha=\dfrac{\sin 2^{n+1}\alpha}{2^{n+1}\sin\alpha}$,
        其中~$\alpha\neq k\pi$,~$k$~为整数。
\end{exlist}
\item 证明,如果论断
\[
  1+2+\dotsb+n=\frac 12\Paren[\Big]{n+\frac12}^2
\]
对~$n=k$~是成立的,则这个论断对~$n=k+1$~也是成立的。解释这个论断不是对任意~$n$~成立。
\item 设~$x>-1$,且~$x\neq0$,$n\geq 2$,证明,
\[
  (1+x)^n>1+nx\.
\]
\item 证明,
  \begin{exlist}
    \item $\dps\Paren[\Big]{1+\frac 1{n-1}}^n>\Paren[\Big]{1+\frac1n}^{n+1}$,~$n\geq 2$;
    \item $\dps\Paren[\Big]{1+\frac 1{n-1}}^{n-1}<\Paren[\Big]{1+\frac1n}^n$,~$n\geq 2$;
    \item $\dps\Paren[\Big]{1+\frac 1n}^n<4$,~$n\geq 1$。
  \end{exlist}
\item 设~$n\geq 1$,证明,
\[
  \sum_{k=n+1}^{2n}\frac1k=\sum_{k=1}^{2n}\frac{(-1)^{k+1}}k\.
\]
\item 证明~$\abs{a-b}\leq\abs a+\abs b$。问下面证法是否正确?
\[
  \abs{a-b}\leq\abs{a+b}\leq\abs a+\abs b\.
\]
\item 证明,
\[
  \abs[\Big]{\sum_{k=1}^na_k}\geq\abs{a_1}-\sum_{k=2}^n\abs{a_k}\.
\]
\item 解下列不等式。
\begin{exlistcols}[3]
  \item $\abs{x-1}<3$;
  \item $\abs{3-2x}<1$;
  \item $\abs{1+2x}\leq 1$;
  \item $\abs[\Big]{5-\dfrac 1x}<1$;
  \item $\abs{x-1}>2$;
  \item $\abs{x+2}>5$;
  \item $\abs{x^2-2}\leq 1$;
  \item $\abs{x-5}<\abs{x+1}$。
\end{exlistcols}
\item 设~$a<c<b$,证明,
\[
  \abs c\leq\max\Braceb{\abs a,\abs b}\.
\]
\item 证明,
\[
  \frac{\abs{a+b}}{1+\abs{a+b}}\leq\frac{\abs a}{1+\abs a}+\frac{\abs b}{1+\abs b}\.
\]
\item 证明,
\[
  \max\Brace{a,b}=\frac{a+b}2+\frac{\abs{a-b}}2,\quad
  \min\Brace{a,b}=\frac{a+b}2-\frac{\abs{a-b}}2\.
\]
并解释其几何意义。
\item 证明,
\[
  \abs{a+b}^p\leq2^p\max\Braceb{\abs a^p,\abs b^p},\quad p>0\.
\]
\item 设~$a,b>0$,证明,
\begin{exlistcols}
  \item $(a+b)^p\geq a^p+b^p$,~$p>1$;
  \item $(a+b)^p\leq a^p+b^p$,~$0<p<1$。
\end{exlistcols}
\item 证明,对任意实数~$a,b$,有
\[
  \max\Braceb{\abs{a+b},\abs{a-b},\abs{1-b}}\geq\frac12\.
\]
\item 证明,
\[
  \frac1{4n}<\Paren[\bigg]{\frac{1\cdot 3\cdot 5\cdot\dotsm\cdot(2n-1)}{2\cdot 4\cdot 6\cdot\dotsm\cdot(2n)}}^2<\frac1{2n}\.
\]
\item 令~$\dps A_n=\prod_{k=1}^n(1+a_k)$,$\dps B_n=\prod_{k=1}^n\Parenb{1+\abs{a_k}}$,证明,
\[
  \abs{A_n-1}\leq B_n-1\.
\]
\end{exercise*}


\endinput
%%
%% End of file `MAChapter0.tex'.