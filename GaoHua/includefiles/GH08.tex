%# -*- coding:utf-8 -*-
%%%%%%%%%%%%%%%%%%%%%%%%%%%%%%%%%%%%%%%%%%%%%%%%%%%%%%%%%%%%%%%%%%%%%%%%%%%%%%%%%%%%%

\chapter{革命转入中下层:全面整风的发韧}

\section{动员“思想革命”:毛泽东究竟要做什么?}

毛泽东在~1941~年~9~月政治局扩大会议取得了对王明的决定性胜利后,在中央层陆续推出一些重要措施以扩大自己的胜利,同时开始调整战略,积极布置将反对王明、博古的革命引入到党的中下层。经过数月的准备,1942~年~2~月~1~日,毛泽东正式登场,在延安中央党校开学典礼上作动员全党整风的报告——《整顿党风、学风、文风》(收入《毛选》时易名为《整顿党的作风》),2~月~8~日和~9~日,毛又在中央宣传部干部会议上两次发表《反对党八股》的演说。在此之后,毛亲自主持制定了几个有关整风学习的中央决议,至~1942~年~4~月~3~日,复以中宣部的名义,发出《关于在延安讨论中央决定及毛泽东同志整顿三风报告的决定》向全党正式提出开展,“思想革命”的号召。

毛泽东在~2~月所作的报告和他主持制定的有关动员整风的决定,都没有直接点出王明、博古的名字,只是号召反对“主观主义”和“教条主义”。一年前的~1941~年~5~月,毛泽东当着王明等人的面,在《改造我们的学习》的演讲中,怒斥留苏派只是“言必称希腊”的留声机,控诉他们用教条主义来毒害青年,“十七八岁的娃娃,教他们啃《资本论》,啃《反杜林论》。”\footnote{毛泽东在~1941~年~5~月作的《改造我们的学习》报告中把教“十七、八岁的娃娃啃《资本论》、《反杜林论》”列为教条主义最恶劣的表现之一。毛的这句名言在延安不胫而走,再也没人敢钻研《资本论》一类的经典著作。1949~年后,毛的这段话在收入《毛泽东选集》时被删除。原文见边区总学委编:《整顿三风二十二个文件》(延安:1942~年),页~4~—~5;另见王惠德:《忆昔日》,《延安马列学院回忆录》,页~79~—~81。}

现在毛在《整顿党风、学风、文风》和《反对党八股》的演说中又对留苏派及党内知识分子冷嘲热讽,他历数中共党内的苏俄崇拜情结在文宣形式的八大罪恶,甚至将留苏派等贬之为“连猪都不如的蠢货”。毛说:

\begin{quoting}
他们一不会耕田,二不会做工,三不会打仗,四不会办事……只要你认得了三五千字,学会了翻字典,手中又有一个什么书,公家又给了你小米吃,你就可以摇头晃脑的读起来。书是不会走路的,也可以随便把它打开或者关起。这是世界上最容易的事,这比大师傅煮饭容易得多,比他杀猪更容易。你要捉猪,猪会跑,杀它,它会叫,一本书摆在桌子上,既不会跑,又不会叫,随你怎样摆布都可以。……那些将马列主义当宗教教条看待的人,就是这种蒙昧无知的人。对于这种人,应该老实对他说,你的教条没有什么用处,说句不客气的话,实在比屎还没有用。我们看,狗屎可以肥田,人屎可以喂狗。教条呢,既不能肥田,又不能肥狗,有什么用处呢?\footnote{上述内容在收入《毛选》时已被删去,题目也被改为《整顿党的作风》,原文见《整顿三风二十二个文件》。}
\end{quoting}
毛的这番话充分展现了毛个性中尖酸刻薄、狂傲不羁的一面。毛泽东亲自登台作报告,矛头直指“比狗屎还不如的教条”,表明他已决心全力发动整风运动,并以此诏示全党,毛本人是整风运动的最高领导者。此时的毛早已被公认为全党的领袖,他也完全习惯了这个角色。毛已不愿像~1936~—~1938~年那样直接与普通党员见面,现在到了“定朝仪”的时候了。毛需要选择一个代言人,由这个人向延安各机关学校传达毛认为有必要公诸于众的指示,以显示毛的威严和神秘。

毛泽东所选择的代言人正是康生。1942~年~2~月~21~日,康生在延安八路军大礼堂向延安二千二百余干部传达毛泽东整顿三风的报告。康生尤其对“什么是理论家,什么是知识分子”作了“深刻生动的说明”\footnote{《解放日报》,1942~年~2~月~22~日。},他说,“目前所谓知识分子,实际上最无知识,工农分子反而有一点知识”。3~月~7~日,康生又在同一地点向三千多干部传达毛的《反对党八股》。4~月~18~日,康生再在八路军大礼堂向中直机关、军委直属机关干部作长篇动员报告。在这几次大型报告会上,康生极力发挥毛泽东对教条主义的批判以及对知识分子的嘲讽,将毛泽东有关知识分子“其实是比较最无知识”的新概念,在延安三万干部中迅速传播开来。问题是毛泽东为何舍任弼时等不用,单单挑选王明昔日的副手康生作自己的代言人呢?

毛泽东看中康生的是他对自己的绝对顺从和康生这个“魔鬼”形象所具有的威慑作用。康生敢于最早向王明反戈一击,继而不为自己留半点退路,早已深获毛的信任。与毛的其他盟友相比,康生之于毛,有更多的亲和性。另外,毛也不对全党干部转变思想抱十分乐观的态度,毛要利用康生的专业特长,促成全党思想转变。康生这个名字所象征的强制性,将有助于延安的干部在思想改造中加深对毛泽东权威的感性认知。

整风运动既是“一场马克思主义的学习和教育运动”,那么理所应当,负责全党文宣工作的中宣部将处于领导这场运动的中心地位。然而,作为“教条主义大本营”的中宣部自身就是运动的重点整肃对象。1941~年~9~月政治局会议的后期阶段曾经决定成立以中宣部部长张闻天为首的教育委员会,负责改革全党的干部教育工作。但是张闻天十分知趣,他在这次政治局会议后,实际上就离开了核心上层,这个所谓的“教育委员会”还没开展工作就无疾而终。张闻天为了“不妨碍毛主席整风运动方针”的贯彻,决定自我放逐,1942~年~1~月~26~日他率领一个“延安农村工作调查团”去了陕北和晋西北农村,中宣部部长由另一个国际派人物、政治局候补委员凯丰代理。凯丰似乎忘记了他本人就是整风运动所要打击的目标,竟然煞有其事地在《解放日报》发表《如何打破教条主义的学习》的文章。凯丰在这篇不长的文章里,开口“毛泽东同志”闭口“毛主席”,他以指导者的身分,告诉延安的读者,“过去我们的学习方法,受教条主义熏染太深,形式逻辑的思想方法习惯太多”,凯丰试图以自己的“努力”和“紧跟”来换取毛泽东的信任,从而使自己继续留在革命的指导者的行列。\footnote{《解放日报》,1942~年~6~月~11~日。}尽管凯丰对宣传毛的有关整风指示竭尽了全力,他为了“赎罪”,还自鞭自责,检讨中宣部没有把贯彻毛的整风报告作为目前宣传教育工作的中心任务。\footnote{《中共中央关于延安整风的一组文件》,《文献和研究》1984~年第~9~期。}但为时不久,毛还是派了自己的秘书,既非中央委员,更非政治局委员的胡乔木前去“暂代”凯丰,\footnote{凯丰在~1942~年上半年整风运动的初期还十分活跃,他甚至是五月延安文艺座谈会的主要组织者,但是毛泽东不放手让凯丰负责文艺界整风,而是亲自挂帅,同时指派胡乔木协助自己主持文艺界的整风运动。6~月~2~日,毛在中央总学委成立会议上宣布,“因凯丰同志工作很忙,改由康生负责中央总学委机关刊物《学习报》的编委工作”。不久,胡乔木即奉毛泽东之命正式代理了凯丰的中宣部代部长一职。参见胡乔木:《我所知道的田家英》,载董边、镡德山、曾自编:《毛泽东和他的秘书田家英》,页~121。}用这种方式改组了中宣部,将其完全控制在自己的手中。

1942~年冬春之交的延安笼罩着一层厚重的浓雾,广大中下层干部并不知道眼下正在开展的整风运动的真正意图是什么,他们还以为这是类似~1939~—~1940~年学习运动的新一轮学习运动。延安各机关、学校在听了康生作的传达和动员报告后,纷纷成立了整风领导机构,基本停止了日常的业务工作。早已习惯遵从上级指示的干部们正兴趣浓厚的按照上级的布置,制定各人的学习计划,日夜精读指定的文件材料,一时间延安似乎又再现了前几年的景像,成了一所研究马列主义理论的大学校。

毛泽东如此兴师动众,将革命引入中下层,究竟是为了什么?如果仅仅着眼于夺取中共最高领袖的地位,毛在~1938~年六届六中全会上已经基本达到这个目标,经过~1941~年~9~月政治局会议,毛已完全控制了政治局。把原本对党的上层斗争毫不知晓的普通党员强行拉入到这场已见分晓的上层角逐中,果真有这个必要吗?在目前抗战最艰苦的阶段,把延安的日常工作停顿下来,全部转人政治学习,在道理上能站住脚吗?

毛泽东葫芦里究竟装的是什么药,除了中央核心层领导干部,上至各根据地多数领导人,下至广大中下层干部都不甚明白。不过,毛并不担心他们一时跟不上自己的步调。毛作为革命的策略大师对运动如何进行自有安排,各级干部只需顺着毛的思路,照着毛的部署一步步去做就行。毛泽东执意将上层革命引入中下层的根本目的就是要摧毁王明等的党内基础,在全党肃清王明等俄式马列主义的影响,从而确立自己“新解释”的至尊地位。多年来,王明作为俄式马列主义在中国的代表,拉斯大林的大旗作虎皮,在中共党内已建立起广泛的影响。它的一个重要特征就是中共党内普遍存在的照搬马列原典和盲目崇拜苏联的气氛。王明等在这种气氛下,八面来风,如鱼得水,不仅争取到了以周恩来为代表的老干部派的合作和支持,王明本人也嬴得了全党的尊敬。眼下王明虽然已退出中央核心层,但还未正式缴械投降,若不乘势打碎党内根深蒂固的苏联崇拜情结,斩断俄式马列主义伸向中国的须根,在全党搞臭王明及其同伙,一遇风吹草动,很难排除王明有挟苏联支持、在最高层卷土重来而获全党普遍拥戴之巨大危险。

与此同时,毛泽东还要用自己的“新解释”填补王明之后中共在思想上的真空状态,然后用自己的思想彻底改造中共,将党内原先对王明的崇拜引向对自己及其思想的崇拜。但是这种转换过程并非短期就能完成,必须使全党集中一个相对长的时间,用于进行这项除旧布新的思想改造工程。

然而,王明等在党内的实际影响力可能被毛泽东有意夸大了。事实上,王明的影响主要集中在党的上层和党内知识分子之中,在军队和党的中下层影响甚微。无论在红军时代,还是在抗战阶段,中共的主体都是军队,红军将士只知有朱毛,不知有王明和博古。\footnote{罗瑞卿在瑞金时代担任红一军团政治保卫局局长,属于军中高级干部,当时连王明的名字都不知道。参见点点:《非凡的年代》(上海:上海文艺出版社,1987~年),页~85。}在三十年代至四十年代初的中共军中确实存有崇拜苏联的气氛,有留苏经历的军政干部在一段时期受到推崇,也是事实,但军中文盲比率极高,众多官兵文字尚不识,一些党员干部“听也未听说过马列主义”。\footnote{李维汉:《回忆中央党校》,载《回忆与研究》,上(北京:中共党史资料出版社,1986~年),页~391。}中共虽对少数高级军政干部进行过短期的马列基础教育,但这种教育以“少而精”为原则,受训干部只能略知马列皮毛,在这种情况下,何以谈得上“教条主义”!显而易见,在中共党内、军内占主导地位的并非是教条主义传统,而是经验——实用主义传统。毛执意要在基本由农民组成的中共党和军队内开展反教条主义的整风运动,用心殊深,其结果可想而知。将污水泼在有留苏经历的军政干部身上,虽可解一时之愤,但其严重的后果却是从此也将鄙视理论、轻视知识分子的风气发扬光大,并深深地扎根于党的理论与实践中。所以毛此举决非是无事生非,小题大作,而是经过深思熟虑的重大战略行动。

1941~—~1942~年毛泽东的全部兴奋中心都围绕着一件事,这就是如何构筑以自己思想为核心的中共新传统,并将此注入到党的肌体。从这个意义上说,整风运动确实是一场对马列原典的革命,它以教化和强制为双翼,以对俄式马列主义作简化性解释为基本方法,将斯大林主义的核心内容与毛的理论创新,以及中国儒家传统中的道德修养部分互相融合,从而形成了毛的思想革命的基本原则。

毛泽东的思想革命包含了四个重要原则:

一、树立“实用第一”的观点,坚决抛弃一切对现实革命目标无直接功用的理论,把一切无助于中共夺取政权的马列原典一概斥之为“教条”。全力破除对马列原典的迷信,集中打击中共党内崇尚马列原典的老传统及其载体——党内有留苏经历的知识分子和受过西方或国内“正规”教育的知识分子。在毛泽东的精心引导下,中共党内最终形成了熟悉原典有错、少读原典光荣的新风尚。

二、全力肃清“五四”自由、民主、个性解放思想在党内知识分子中的影响,确立“领袖至上”、“集体至上”、“个人渺小”的新观念。为集中打击俄式马列主义,毛泽东在短时期内借助党内自由派知识分子,围剿留苏派,一经利用完毕,毛迅即起用已缴械投降的原留苏派,联合围剿党内残存的“五四”影响。

三、将“农民是中国革命主力军”的观念系统化、理论化,并将其贯穿于中共一切思想活动。

四、把宋明新儒家“向内里用力”的观念融入共产党党内斗争的理论,交替使用思想感化和暴力震慑的手段,大力培养集忠顺与战斗精神为一体的共产主义“新人”的理想人格,并在此基础上构筑党的思想和组织建设的基本范式。

毛泽东的思想革命既有因袭的成份,也有独创的方面,它是列宁、斯大林主义和中国部分传统的混合物。1942~年后,毛得心应手,交替使用教化与强制两种手段,将思想革命的四项原则贯穿于整风运动的全过程。随着整风运动的深人展开,毛泽东的“新解释”迅速取代了俄式“老话”,成为中共的新传统。

\section{冻结政治局,中央总学委的成立}

1941~年~9~月政治局会议后,由毛泽东倡议而成立的中央高级学习组逐渐成为延安最有权势的机构,中央政治局和书记处的许多功能已在静悄悄中被中央高级学习组所取代。在延安经常出头露面的政治局委员和书记处书记只剩下毛泽东、康生、任弼时、陈云等少数人。其他几位,或因病住进了医院(王明),或主动请求离职,悄然前往了农村(张闻天)。王稼祥、凯丰在整风之初活跃了一阵后,很快也销声匿迹了。博古、邓发则成了“问题人物”。至于远在重庆的周恩来和坚守太行山八路军总部的彭德怀虽然暂时还未被触及,但他们几乎对延安上层近几个月所发生的变化毫无所知,直到~1942~年~2~月~21~日,毛才致电周恩来,通报有关~1941~年~9~月政治局会议讨论过去路线的结论精神。\footnote{《毛泽东关于延安整风的一组函电》,《文献和研究》1984~年第~8~期。}在延安之外的政治局委员,也许只有刘少奇对延安发生的一切了如指掌。他几乎与延安同步,在苏北的盐城也开始了整风的部署。

面对政治局和书记处名存实亡的既成事实,毛泽东一不做、二不休,干脆成立中央总学习委员会,作为其个人领导整风运动和全党一切工作的临时最高权力机构。

1942~年~6~月~2~日,中央总学习委员会(以下称中央总学委)宣告成立,由毛泽东亲自任主任,康生任副主任。总学委设秘书处,康生兼任秘书长,李富春任副秘书长。总学委下辖五个分区学习委员会:

中央直属系统(中直机关)分区学习委员会:康生、李富春负责;

中央军委直属系统(军直机关)分区学习委员会:陈云、王稼祥负责;

陕甘宁边区系统分区学习委员会:任弼时、高岗负责;

中央文委系统分区学习委员会:周扬负责;

中央党校分区学习委员会:彭真负责。

中央总学委的设立标志着中央高级学习组成立后开始的权力转移过程得到了进一步深化。中央总学委具有最高权力机关的各种特征,总学委的秘书处作为毛泽东的办事机构,犹如清廷中的军机处,政治局和书记处虽未明文予以撤销,但其许多职能已被中央总学委取代。

毛泽东主控下的中央总学委具有广泛的权力。总学委有创设机构权,1942~年后,中央总学委在延安各分区学委会普遍设立了党的常委会组织。总学委还有权向延安及各根据地发号施令,决定停止一切政治及业务工作,转入整风文件的学习。总学委有权决定参加各地高级学习组成员的名单,判别哪些干部可以进入,哪些干部不能进入。总学委派出的巡视团和巡视员享有至高的权威,可以任意前往各重要机关、学校检查运动进行情况,听取各单位负责人的汇报。

1942~年~6~月~2~至~15~日,在两个星期的时间里总学委举行了三次会议,仅从这三次会议所决定的几项重要措施,就可以清楚看出总学委在整风运动中所起的关键作用和毛泽东领导整风的策略。

一、布置抽阅高级干部的整风笔记。

1942~年~3~至~4~月,毛泽东亲自选编了供干部必读的“二十二个文件”,下令所有参加高级学习组的干部,必须对照文件精神,联系个人的思想和历史经历,写出整风笔记。

6~月~2~日,中央总学委第一次会议决定,总学委有权抽阅参加中央学习组全体高级干部的学习笔记,并由康生拟出第一批抽阅笔记的人员名单。\footnote{《中共中央关于延安整风的一组文件》,《文献和研究》1984~年第~9~期。}6~月~7~日,总学委第二次会议宣布了负责检查笔记的人选名单和分工范围:

\begin{quoting}
一、四学校及文抗——凯丰、乔木、蒋南翔。\footnote{“四学校”指鲁迅艺术学院等党校以外的学校,“文抗”指全国文艺界抗敌协会延安分会。}

二、党校——毛主席、彭真。

三、军委——王家祥\footnote{即王稼祥。}、陈云、陈子健。

四、中央直属——康生、富春、尚昆、曾固、曹轶欧。

五、边区系统——弼时、康生、师哲、廖鲁言。\footnote{《中共中央关于延安整风的一组文件》,《文献和研究》1984~年第~9~期。}
\end{quoting}

从这份名单可以看出,毛泽东是多么急于了解党的高级干部对他所发动的整风运动的真实态度。毛尤其关心中央党校学员的思想反应,这是因为~1941~年后大批延安及外地返延的高级干部被送入中央党校受训,中央党校已成为高干集中地,毛迫切希望了解他们的“活思想”。

二、运用“掺沙子”、混合编组的方法,将国际派孤立在中央学习组下的十个小组。

国际派分子大多为重要干部,一些人甚至是政治局委员或中央委员,无论是论其资格或是运动本身的目的,他们都应被列入中央学习组。为了防范这批“问题人物”在中央学习组“扎堆结伙”,互为呼应,毛泽东在编组上也作了精心安排。中央总学委将国际派成员分散编人各个小组,在不少小组都安排了属于毛营垒、但职位较低的干部作副组长,以收监督、牵制之效。1942~年~6~月~7~日,中央总学委宣布中央学习组混合编成十个小组,\footnote{《中共中央关于延安整风的一组文件》,《文献和研究》1984~年第~9~期。}组长与副组长名单分别如下:

\begin{quoting}
\centering
\begin{tabular}{ccc@{}c}
       & 正组长 & \multicolumn2c{副组长} \\
第一组 & 毛泽东 & 高岗 & 谢老(谢觉哉) \\
第二组 & 朱德   & 彭真 & 吴老(吴玉章) \\
第三组 & 弼时   & 尚昆 & 徐老(徐特立) \\
第四组 & 家祥   & 贺龙 & 陶铸 \\
第五组 & 凯丰   & 林老(林伯渠) & 方强 \\
第六组 & 陈云   & 罗迈   & 蔡畅 \\
第七组 & 博古   & 徐向前 & 定一 \\
第八组 & 邓发   & 陈正人 & 师哲 \\
第九组 & 富春   & 叶剑英 & 乔木 \\
第十组 & 康生   & 谭政   & 萧劲光
\end{tabular}
\end{quoting}

以上十个小组的正组长除了李富春之外,均是政治局委员,但有几个小组,非政治局委员的副组长却比正组长有更大的权威。例如,博古虽是第七组组长,但小组的实际主持人却是毛的盟友陆定一。第八组组长邓发的境况与博古近似,副组长陈正人是毛在江西时代的老部下,为了加强陈正人的力量,毛还把中央社会部工作人员师哲安排为副组长。对于德高望重的朱德,毛甚至也未完全放心,他虽无意猛烈打击朱德,但却颇担心朱德“立场不稳”,于是特派出此时备受他信任的彭真担任朱德的副组长。

三、“引蛇出洞”,摸底排队。

自毛泽东整风报告传达后,延安各机关、学校陆续出现针砭“三风”(党风、学风、文风)的墙报、壁报,其中少数内容尖锐的壁报已引起延安社会的震动。为进一步“收集材料”,1942~年~6~月~7~日召开的中央总学委第二次会议部署进一步放:

\begin{quoting}
要使领导者善于启发,使大家有话敢说,展开争论,暴露思想,然后从容结论。\footnote{《中共中央关于延安整风的一组文件》,《文献和研究》1984~年第~9~期。}
\end{quoting}6~月~15~日,中央总学委第三次会议更具体要求:

\begin{quoting}
各机关、学校出有墙报者,为着展开思想上、工作上的论争,对于投稿之选择,不论其正面的与反面的,正确的与不正确的,均应登载,不得抑制\footnote{《中共中央关于延安整风的一组文件》,《文献和研究》1984~年第~9~期。}。
\end{quoting}中央总学委密切注视着延安各单位的整风动态,全力引导整风朝着毛泽东规划的方向发展。总学委要求各系统分区学委对所属单位一一“摸底排队”。“择其必要者加紧整顿之”,至于“哪个机关、学校应特别注意,由总学委同志与各系统学委商定之”\footnote{《中共中央关于延安整风的一组文件》,《文献和研究》1984~年第~9~期。}。

中央总学委作为毛泽东领导整风运动最重要的机构,其地位与作用类似于二十几年后文化大革命期间的中央文革小组。毛泽东是中央总学委的挂帅人物,是一切重大决策的设计者。

中央总学委在毛的直接领导下行动,也只对毛个人负责。在中央总学委内,康生是第二个最有权势的人物,康生作为总学委副主任兼秘书长,具体秉承毛的旨意办事,其角色类似于文革期间的江青,但远比江青的地位显赫。1942~年底刘少奇返回延安,一段时间后也担任了中央总学委副主任,排名在康生之前,但刘少奇似乎不愿多在幕前活动,因而人们一般仅注意到康生而忽略了刘少奇。在总学委的权力金字塔上,还有其他几个重要角色,他们分别是李富春、彭真和陆定一。

康生在~1941~年~9~月政治局会议后权势得到进一步扩大,一跃成为集部分党权、情报权、干部审查权于一身的延安第二号人物。全面整风发动后,康生又成为毛泽东“圣旨”的传达人和毛与总学委之间的联络员。在党内,康生是中央社会部部长、情报部部长、中央敌工工作委员会主任;在领导整风方面,康生是总学委副主任兼秘书长、党与非党干部审查委员会主任。康生还取代了凯丰,担任了指导整风的总学委机关刊物《学习报》的主编。康生大权在握,权势熏天,1942~年~4~月~5~日,延安《解放日报》在社论中,甚至将康生与毛泽东的名字并列,号召全党“细心研究一下毛泽东和康生同志的报告”,“了解什么叫三风”。

毛泽东有意倚重康生,在经他亲自选定的整风必读“二十二个文件”中,其中有两份是康生作的报告。毛将康生和刘少奇并列,封为“马列主义正确路线”的代表,从而大大抬高了康生的身价。康生自恃有毛泽东撑腰,更加有恃无恐,竟将其妻曹轶欧也拉进了总学委,让她参与抽查高干笔记。康生的奸诈险恶使延安大多数政治局委员都失去了安全感,神憎鬼厌,人见人怕,许多高级领导干部都避康生如避鬼神,惟恐躲之不及。1942~年~6~月,康生向毛泽东提议,让担任各政治局委员的政治秘书协助他做整风的调查研究工作。康生的“调查研究”之真正含义是什么,毛当然一目了然,正可谓求之不得,只是彼此心照不宣罢了。经“党中央同意”,康生正式要求胡乔木(毛泽东的政治秘书)、黄华(朱德的政治秘书)、廖鲁言(王明的政治秘书)、师哲(任弼时的政治秘书)、王鹤寿(陈云的政治秘书)、陶铸(王稼祥的政治秘书)、匡亚明(康生的政治秘书)等\footnote{师哲:《在历史的巨人身边——师哲回忆录》,页~245~—~26、246。}作他的助手,向他及总学委汇报“整风学习的进展情况”。只是此时中共上层正处于急剧变动之际,各政治局委员与毛的关系深浅不一,毛对彼等态度也大有差别,因此,康生此项工作的成效也各不相同:凡与毛关系比较密切的政治局委员,他们的政治秘书多以工作繁忙为由,对康生交办的任务束之高阁,最后则不了了之;只有任弼时的秘书师哲和王明的秘书廖鲁言经常向康生汇报工作。\footnote{师哲:《在历史的巨人身边——师哲回忆录》,页~245~—~26、246。}师哲当时任毛泽东的俄文译员和任弼时的秘书,又是会部的工作人员,与康生接触较多主要是出于工作原因,但廖鲁言与康生联络频繁就显得颇为蹊跷了。因为王明自~1941~年~10~月生病住院后已脱离工作,廖鲁言的“工作汇报”又从何谈起?无非是将王明夫妇日常生活的动态和言论及时向康生汇报。1943~年~1~月,王明对廖鲁言作过一次谈话,王明讲述了在中共历史上三次反对“莫斯科集团”的问题,廖鲁言对王明的谈话作了记录,事后迅速向毛泽东作了汇报。\footnote{王明:《中共五十年》,页~138。}廖鲁言对毛的忠诚自然得到了回报,事实是廖鲁言在中共建国后长期担任国务院农业部部长和中央农村工作部党组副书记,副部长,并于~1956~年在中共八大上当选为中央候补委员,他并没有因长期担任王明的秘书而受到任何打击。

中央总学委的另几个关键人物李富春、彭真、陆定一等也各手握重权。整风运动开始后,身为中央办公厅副主任,中组部副部长的李富春很快被任命为中央总学委副秘书长兼中直机关分区学委组长,主管中枢机构的清洗、整顿工作。在整风期间,李富春几乎与康生并驾齐驱,是毛泽东最倚重的干部之一。1942~年~6~月~7~日,很可能是得到毛的授意,李富春在总学委第二次会议上提议成立中央总学委秘书处,由康生负责领导。李富春又提名吸收杨尚昆(中共北方局书记,刚从太行返回延安)、柯庆施(中央统战部副部长)参加中直系统学委会常委会。\footnote{《中共中央关于延安整风的一组文件》,《文献和研究》1984~年第~9~期。}陈云除参与领导军委系统分区学委,还负责对选派到中央党校受训干部的资格审查工作。彭真则全面掌管全党高级干部培训兼审查中心的中央党校。陆定一在历史上曾受过博古等国际派的严重打击,1941~年从太行八路军总部政治部副主任、宣传部长的任上调回延安,很快受到毛的重用。毛知人善任,授予陆定一的新职责也不平常,他成了凌驾于博古之上、监督新华社和《解放日报》的舆论总管。

1942~年~6~月,中央总学委成立后的一段时间,任弼时的地位发生了微妙的变化,他被委派领导中共西北局的整风工作,没能如康生那样,成为中央总学委副主任。任弼时政治地位的变化,反映了中共核心层斗争的复杂性。任弼时为人较为方正,在整风初始的关键时刻,难免显得碍手碍脚,将其暂时打发一边,不让他与闻某些机密,以待其“觉悟”,可能是毛泽东暂时贬谪任弼时的原因。

1942~年春夏之际,毛泽东祈求多时的整风暴风雨终于向延安各机关、学校袭来。毛的得力助手康生、李富春等已全部披挂上阵——中共重建工程已正式展开。