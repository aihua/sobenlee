%# -*- coding:utf-8 -*-
%%%%%%%%%%%%%%%%%%%%%%%%%%%%%%%%%%%%%%%%%%%%%%%%%%%%%%%%%%%%%%%%%%%%%%%%%%%%%%%%%%%%%
%%  MAChapter11.tex'


\chapter{数值级数}\label{ch:11}
\section{数值级数的基本概念与简单性质}
\subsection{数值级数的收敛与发散}
\subsection{级数的基本性质}
\begin{exercise}
\item
\end{exercise}
\section{正项级数}
\subsection{正项级数收敛的充要条件}
\subsection{比较判别法}
\subsection{d'Alembert~判别法及~Cauchy~判别法}
\subsection{Raabe~判别法}
\subsection{Cauchy~积分判别法}
\begin{exercise}
\item
\end{exercise}
\section{任意项级数}
\subsection{交错级数收敛判别法}
\subsection{绝对收敛与条件收敛}
\subsection{Dirichlet~判别法}
\begin{exercise}
\item
\end{exercise}
\section{收敛级数的性质}
\subsection{无穷级数的可结合性}
\subsection{无穷级数的可交换性问题}
\subsection{级数的乘法}
\begin{exercise}
\item
\end{exercise}
\section{广义积分与级数的联系}
\begin{exercise*}
\item
\end{exercise*}




\endinput
%%
%% End of file `MAChapter11.tex'.