%# -*- coding:utf-8 -*-
%%%%%%%%%%%%%%%%%%%%%%%%%%%%%%%%%%%%%%%%%%%%%%%%%%%%%%%%%%%%%%%%%%%%%%%%%%%%%%%%%%%%%


\chapter{韩道国拐财远遁\KG 汤来保欺主背恩}


诗曰:

\[
燕入非傍舍,鸥归只故池。断桥无复板,卧柳自生枝。
遂有山阳作,多惭鲍叔知。素交零落尽,白首泪双垂。
\]

话说韩道国与来保,自从拿着西门庆四千两银子,江南买货物,到于扬州,抓寻苗青家内宿歇。苗青见了西门庆手札,想他活命之恩,尽力趋奉。又讨了一个女子,名唤楚云,养在家里,要送与西门庆,以报其恩。韩道国与来保两个且不置货,成日寻花问柳,饮酒宿妇。只到初冬天气,景物萧瑟,不胜旅思。方才将银往各处买布匹,装在扬州苗青家安下,待货物买完起身。先是韩道国请个表子,是扬州旧院王玉枝儿,来保便请了林彩虹妹子小红。一日,请扬州盐客王海峰和苗青游宝应湖,游了一日,归到院中。又值玉枝儿鸨子生日,这韩道国又邀请众人,摆酒与鸨子王一妈做生日。使后生胡秀,请客商汪东桥与钱晴川两个,白不见到。不一时,汪东桥与钱晴川就同王海峰来了。至日落时分,胡秀才来,被韩道国带酒骂了两句,说:“这厮不知在那里噇酒,噇到这咱才来,口里喷出来的酒气。客人到先来了这半日,你不知那里来,我到明日定和你算帐。”那胡秀把眼斜瞅着他,走到下边,口里喃喃呐呐,说:“你骂我,你家老婆在家里仰扇着挣,你在这里合蓬着丢!宅里老爹包着你家老婆,\textuni{34B2}的不值了,才交你领本钱出来做买卖。你在这里快活,你老婆不知怎么受苦哩!得人不化白出你来,你落得为人就勾了。”对玉枝儿鸨子只顾说。鸨子便拉出他院子里,说:“胡官人,你醉了,你往房里睡去罢。”那胡秀大吆大喝,白不肯进房。不料韩道国正陪众客商在席上吃酒,听见胡秀口内放屁辣臊,心中大怒,走出来踢了他两脚,骂道:“贼野囚奴,我有了五分银子,雇你一日,怕寻不出人来!”即时赶他去。那胡秀那里肯出门,在院子内声叫起来,说道:“你如何赶我?我没坏了管帐事!你倒养老婆,倒赶我,看我到家说不说!”被来保劝住韩道国,一手扯他过一边,说道:“你这狗骨头,原来这等酒硬!”那胡秀道:“叔叔,你老人家休管他。我吃甚么酒来,我和他做一做。”被来保推他往屋里挺觉去了。正是:

\[
酒不醉人人自醉,色不迷人人自迷。
\]

来保打发胡秀房里睡去不题。韩道国恐怕众客商耻笑,和来保席上觥筹交错,递酒哄笑。林彩虹、小红姊妹二人并王玉枝儿三个唱的,弹唱歌舞,花攒锦簇,行令猜枚,吃至三更方散。次日,韩道国要打胡秀,胡秀说:“小的通不晓一字。”道国被苗青做好做歹劝住了。

话休饶舌。有日货物置完,打包装载上船。不想苗青讨了送西门庆的那女子楚云,忽生起病来,动身不得。苗青说:“等他病好了,我再差人送了来罢。”只打点了些人事礼物,抄写书帐,打发二人并胡秀起身。王玉枝并林彩虹姊妹,少不的置酒马头,作别饯行。从正月初十日起身,一路无词。一日到临清闸上,这韩道国正在船头站立,忽见街坊严四郎,从上流坐船而来,往临清接官去。看见韩道国,举手说:“韩西桥,你家老爹从正月间没了。”说毕,船行得快,就过去了。这韩道国听了此言,遂安心在怀,瞒着来保不说。不想那时河南、山东大旱,赤地千里,田蚕荒芜不收,棉花布价一时踊贵,每匹布帛加三利息,各处乡贩都打着银两远接,在临清一带马头迎着客货而买。韩道国便与来保商议:“船上布货约四千余两,见今加三利息,不如且卖一半,又便宜钞关纳税,就到家发卖也不过如此。遇行市不卖,诚为可惜。”来保道:“伙计所言虽是,诚恐卖了,一时到家,惹当家的见怪,如之奈何?”韩道国便说:“老爹见怪,都在我身上。”来保强不过他,就在马头上,发卖了一千两布货。韩道国说:“双桥,你和胡秀在船上等着纳税,我打旱路同小郎王汉,打着这一千两银子,先去报老爹知道。”来保道:“你到家,好歹讨老爹一封书来,下与钞关钱老爹,少纳税钱,先放船行。”韩道国应诺。同小郎王汉装成驮垛,往清河县家中来。

有日进城,在瓮城南门里,日色渐落,忽撞遇着坟的张安,推着车辆酒米食盐,正出南门。看见韩道国,便叫:“韩大叔,你来家了。”韩道国看见他带着孝,问其故,张安说:“老爹死了,明日三月初九日断七。大娘交我拿此酒米食盒往坟上去,明日与老爹烧纸。”这韩道国听了,说:“可伤,可伤!果然路上行人口似碑,话不虚传。”打头口径进城中。到了十字街上,心中算计:“且住。有心要往西门庆家去,况今他已死了,天色又晚,不如且归家停宿一宵,和浑家商议了,明日再去不迟。”于是和王汉打着头口,径到狮子街家中。二人下了头口,打发赶脚人回去,叫开门,王汉搬行李驮垛进入堂中,径到狮子街家中。二人下了头口,打发赶脚人回去,叫开门,王汉搬行李驮垛进入堂中。老婆一面迎接入门,拜了佛祖。王六儿替他脱衣坐下,丫头点茶吃。韩道国先告诉往回一路之事,道:“我在路上撞遇严四哥与张安,才知老爹死了。好好的,怎的就死了?”王六儿道:“天有不测风云,人有暂时祸福。谁人保得无常!”韩道国一面把驮垛打开,取出他江南置的许多衣裳细软等物,并那一千两银子,一封一封都放在炕上。老婆打开看,都是白光光雪花银两,便问:“这是那里的?”韩道国说:“我在路上闻了信,就先卖了这一千两银子来了。”又取出两包梯己银子一百两,因问老婆:“我去后,家中他也看顾你不曾?”王六儿道:“他在时倒也罢了,如今你这银子还送与他家去?”韩道国道:“正是要和你商议,咱留下些,把一半与他如何?”老婆道:“呸,你这傻奴才料,这遭再休要傻了。如今他已是死了,这里无人,咱和他有甚瓜葛?不急你送与他一半,交他招暗道儿,问你下落。到不如一狠二狠,把他这一千两,咱雇了头口,拐了上东京,投奔咱孩儿那里。愁咱亲家太师爷府中,安放不下你我!”韩道国道:“丢下这房子,急切打发不出去,怎了?”老婆道:“你看没才料!何不叫将第二个来,留几两银子与他,就叫他看守便了。等西门庆家人来寻你,保说东京咱孩儿叫了两口去了。莫不他七个头八个胆,敢往太师府中寻咱们去?就寻去,你我也不怕他。”韩道国道:“争奈我受大官人好处,怎好变心的?没天理了!”老婆道:“自古有天理到没饭吃哩。他占用着老娘,使他这几两银子,不差甚么。想着他孝堂里,我到好意备了一张插桌三牲,往他家烧纸。他家大老婆那不贤良的淫妇,半日不出来,在屋里骂的我好讪的。我出又出不来,坐又坐不住,落后他第三个老婆出来陪我坐,我不去坐,就坐轿子来家了,想着他这个情儿,我也该使他这几两银子。”一席话,说得韩道国不言语了。夫妻二人,晚夕计议已定。到次日五更,叫将他兄弟韩二来,如此这般,叫他看守房子,又把与他一二十两银子盘缠。那二捣鬼千肯万肯,说:“哥嫂只顾去,等我打发他。”这韩道国就把王汉小郎并两个丫头,也跟他带上东京去。雇了二十辆车,把箱笼细软之物都装在车上。投天明出西门,径上东京去了。正是:

\[
撞碎玉笼飞彩凤,顿开金锁走蛟龙。
\]

这里韩道国夫妇东京去了不题。单表吴月娘次日带孝哥儿,同孟玉楼、潘金莲、西门大姐、奶子如意儿、女婿陈敬济,往坟上与西门庆烧纸。张安就告诉月娘,昨日撞见韩大叔来家一节,月娘道:“他来了,怎的不到我家来?只怕他今日来。”在坟上刚烧了纸,坐了没多回,老早就起身来家。使陈敬济往他家,“叫韩伙计去,问他船到那里了?”初时叫着不闻人言,次则韩二出来,说:“俺侄女儿东京叫了哥嫂去了,船不知在那里。”让陈敬济回月娘。月娘不放心,使敬济骑头口往河下寻船。去了一日,到临清马头船上,寻着来保船只。来保问:“韩伙计先打了一千两银子家去了。”敬济道:“谁见他来?张安看见他进城,次日坟上来家,大娘使我问他去,他两口子夺家连银子都拐的上东京去了。如今爹死了,断七过了,大娘不放心,使我来找寻船只。”这来保口中不言,心内暗道:“这天杀,原来连我也瞒了,嗔道路上定要卖这一千两银子,干净要起毛心。正是人面咫尺,心隔千里。”这来保见西门庆已死,也安心要和他一路。把敬济小伙儿引诱在马头上各唱店中、歌楼上饮酒,请表子顽耍。暗暗船上搬了八百两货物,卸在店家房内,封记了。一日钞关上纳了税,放船过来,在新河口起脚装车,往清河县城里来,家中东厢房卸下。

自从西门庆死了,狮子街丝绵铺已关了。对门段铺,甘伙计、崔本卖了银两都交付明白,各辞归房去了。房子也卖了,止有门首解当、生药铺,敬济与傅伙坟开着。原来这来保妻惠祥,有个五岁儿子,名僧宝儿。韩道国老婆王六儿有个侄女儿四岁,二人割衿做了亲家。家中月娘通不知道。这来保交卸了货物,就一口把事情都推在韩道国身上,说他先卖了二千两银子来家。那月娘再三使他上东京,问韩道国银子下落。被他一顿话说:“咱早休去!一个太师老爷府中,谁人敢到?没的招事惹非。得他不来寻你,咱家念佛。到没的招惹虱子头上挠!”月娘道:“翟亲家也亏咱家替他保亲,莫不看些分上儿。”来保道:“他家女儿见在他家得时,他敢只护他娘老子,莫不护咱不成?此话只好在家对我说罢了,外人知道,传出去到不好了。只当丢这几两银子罢,更休题了。”月娘听了无法,也只得罢了。又交他会买头,发卖布货。他会了主儿来,月娘交陈敬济兑银讲价钱,主儿都不服,拿银出去了。来保硬说:“姐夫,你不知买卖甘苦。俺在江湖上走的多,晓得行情,宁可卖了悔,休要悔了卖。这货来家得此价钱就勾了。你十分把弓儿拽满,迸了主儿,显的不会做生意。我不是托大说话,你年少不知事体。我莫不胳膊儿往外撇?不如卖吊了,是一场事。”那敬济听了,使性儿不管了。他也不等月娘来分付,匹手夺过算盘,邀回主儿来。把银子兑了二千余两,一件件交付与敬济经手,交进月娘收了,推货出门。月娘与了他二三十两银子房中盘缠,他便故意儿昂昂大意不收,说道:“你老人家还收了。死了爹,你老人家死水儿,自家盘缠,又与俺们做甚?你收了去,我决不要。”一日晚夕,外边吃的醉醉儿,走进月娘房中,搭伏着护炕,说念月娘:“你老人家青春少小,没了爹,你自家守着这点孩子儿,不害孤另么?”月娘一声儿没言语。

一日,东京翟管家寄书来,知道西门庆死了,听见韩道国说,他家中有四个弹唱出色女子,该多少价钱,说了去,兑银子来,要载到京中答应老太太。月娘见书,慌了手脚,叫将来保来计议,与他去好,不与他去好。来保进入房中,也不叫娘,只说:“你娘子人家不知事,不与他去,就惹下祸了。这个都是过世老头儿惹的,恰似卖富一般,但摆酒请人,就叫家乐出去,有个不传出去的?何况韩伙计女儿又在府中答应老太太,有个不说的?我前日怎么说来,今果然有此勾当钻出来。你不与他,他裁派府县,差人坐名儿来要,不怕你不双手儿奉与他,还是迟了。难说四个都与他,不如今日胡乱打发两个与他,还做面皮。”这月娘沉吟半晌。孟玉楼房中兰香,与金莲房中春梅,都不好打发。绣春又要看哥儿,不出门。因问他房中玉箫与迎春,情愿要去。以此就差来保,雇车辆装载两个女子,往东京太师府中来。不料来保这厮,在路上把这两个女子都奸了。有日到东京,会见韩道国夫妇,把前后事都说了。韩道国谢来保道:“若不是亲戚看顾我,在家阻住,我虽然不怕他,也未免多一番唇舌。”翟谦看见迎春、玉箫两个都生的好模样儿,一个会筝,一个会弦子,都不上十七八岁,进入府中伏侍老太太,赏出两锭元宝来。这来保还克了一锭,到家只拿出一锭元宝来与月娘,还将言语恐吓月娘说:“若不是我去,还不得他这锭元宝拿家来。你还不知,韩伙计两口儿在那府中好不受用富贵,独自住着一所宅子,呼奴使婢,坐五行三。翟管家以老爹呼之,他家女儿韩爱姐,日逐上去答应老太太,寸步不离,要一奉十,拣口儿吃用,换套穿衣。如今又会写,又会算,福至心灵,出落得好长大身材,姿容美貌。前日出来见我,打扮得如琼林玉树一般,百伶百俐,一口一声叫我保叔。如今咱家这两个家乐到那里,还在他手里坟针线哩。”说毕,月娘还甚是知感他不尽。打发他酒馔吃了,与他银子又不受,拿了一匹段子与他妻惠祥做衣服穿,不在话下。

这来保一日同他妻弟刘仓,往临清马头上,将封寄店内布货,尽行卖了八百两银子,暗卖下一所房子,就在刘仓右边门首,就开杂货铺儿。他便日逐随倚祀会茶。他老婆惠祥,要便对月娘说,假推往娘家去。到房子里,从新换了头面衣服,珠子箍儿,插金戴银,往王六儿娘家王母猪家扳亲家,行人情,坐轿看他家女儿去来。到房子里,依旧换了惨淡衣裳,才往西门庆家中来,只瞒过月娘一人不知。来保这厮,常时吃醉了,来月娘房中,嘲话调戏,两番三次。不是月娘为人正大,也被他说念的心邪,上了道儿。又有一般小厮媳妇,在月娘根前,说他媳妇子在外与王母猪作亲家,插金戴银,行三坐五。潘金莲也对月娘说了几次,月娘不信。

惠祥听了此言,在厨房中骂大骂小。来保便装胖字蠢,自己夸奖,说众人:“你每只好在家里说炕头子上嘴罢了!相我水皮子上,顾瞻将家中这许多银子货物来家。若不是我,都吃韩伙计老年箝嘴,拐了往东京去。只呀的一声,干丢在水里也不响。如今还不道俺每一个‘是’,说俺转了主子的钱了,架俺一篇是非。正是割股的也不知,烯香的也不知。自古信人调,丢了瓢。”媳妇子惠祥便骂:“贼嚼舌根的淫妇!说俺两口子转的钱大了,在外行三坐五扳亲。老道出门,问我姊那里借的几件子首饰衣裳,就说是俺落的主子银子治的!要挤撮俺两口子出门,也不打紧。等俺每出去,料莫天也不着饿水鸦儿吃草。我洗净着眼儿,看你这些淫妇奴才,在西门庆家里住牢着!”月娘见他骂大骂小,寻由头儿和人嚷,闹上吊;汉子又两番三次,无人处在根前无礼,心里也气得没入脚处,只得交他两口子搬离了家门。这来保就大剌剌和他舅子开起个布铺来,发卖各色细布,日逐会亲友,行人情,不在话下。正是:

\[
势败奴欺主,时衰鬼弄人。
\]
