%# -*- coding:utf-8 -*-
%%%%%%%%%%%%%%%%%%%%%%%%%%%%%%%%%%%%%%%%%%%%%%%%%%%%%%%%%%%%%%%%%%%%%%%%%%%%%%%%%%%%%
%%  MAChapter0.tex'


\chapter{预备知识}\label{ch:0}

\section{逻辑符号}

为了书写方便,我们常采用以下一些逻辑符号。

设~$S_1,S_2$~是两个陈述句,它们可以指命题也可以指条件。符号
\[
  S_1\implies S_2
\]
表示命题~$S_1$~成立,则命题~$S_2$~成立;或条件~$S_1$~成立,则条件~$S_2$~也成立。符号
\[
  S_1\iff S_2
\]
表示命题(或条件)~$S_1$~与命题(或条件)~$S_2$~等价。即表示由命题(或条件)~$S_1$~可以推出命题(或条
件)~$S_2$,反过来由命题(或条件)~$S_2$~也可以推出命题(或条件)~$S_1$。

符号
\[
  \forall
\]
表示任意取定,写法是将英文字母~A~倒过来。符号
\[
  \exists
\]
表示存在,写法是将英文字母~E~反过来。

孤立地看这些符号没有什么意思,但组合起来可以表示一句话,这句话可以是正确的,也可以是错误的。如
\begin{alignat*}{4}
&\exists x,\text{使得~}\dfrac 12<x<\dfrac 52 &\enspace&\text{(正确的);}
&\qquad&\forall x,\exists y,\text{使得~}x+y=1 &\enspace&\text{(正确的);}\\
&\forall x,\exists y,\text{使得~}x>y & &\text{(正确的);}
&\qquad&\exists x,\forall y,\text{使得~}x>y & &\text{(错误的)。}
\end{alignat*}
后一句话是说,可以找到一实数,它比任何实数都大,这显然是错误的。

\section{集合初步}

自~Cantor~在十九世纪末创建集合论以来,集合论的概念和方法已渗入到数学的各个分支,成为数学的一种语言。集合论本身也发展
成数学的一个分支,内容十分丰富。

集合不能给予严格的定义,因为定义是用已知的概念去定义未知的概念。如用有理数去定义无理数,这里我们认为有理数是已知的,若
有人喜欢刨根问底,觉得有理数是什么也不清楚,我们可以用整数来定义有理数,进一步用自然数来定义整数,用集合来定义自然数。%
这个过程不可能无穷无尽下去,总有一个概念不能定义,在数学里集合概念就到头了,不能再用其它的数学概念来定义。虽然如此,我
们可以给集合一个描述。先看几个集合的例子:

\begin{enumerate}
\item 所有自然数的全体为一集合,记作~$\mathbb N$;\label{enum:set1}
\item 所有小于~$10$~并且是偶数的自然数全体为一集合;\label{enum:set2}
\item 方程~$x^2+5x+4=0$~的根全体为一集合;\label{enum:set3}
\item 具有北京市户口的人全体为一集合。\label{enum:set4}
\end{enumerate}

尽管集合没有定义,但我们能理解到它是什么意思。一般来说,把具有某种共同特征的事物的全体叫\emph{集合},属于集合的每个个
体叫作该集合的\emph{元素}。

如例~\ref{enum:set1}~中集合的特性是正整数,例~\ref{enum:set4}~中集合的特性是具有北京市户口。根据给定的特性,我们可以判
断每一个元素是属于这个集合,还是不属于这个集合。前面三个例子是数集,例~\ref{enum:set4}~是非数集。以后我们只讨论数集。

集合用大写字母~$A,C,C,X,Y,Z$~表示,元素用小写字母~$a,b,c,x,y,z$~表示。

设~$A$~是一个集合,$a$~是~$A$~的元素,记作
\[
  a\in A;
\]
反之,$a$~不是~$A$~的元素,记作
\[
  a\mathrel{\overline\in} A\quad\text{(或~$a\notin A$)。}
\]

\subsection{集合表示法}

集合有两种表示法:一是列举法,如集合
\[
  A=\Brace{2,4,6,8},
\]
这种表示法是将集合的元素在花括弧内一一列举出来;另一是描述法,如集合
\[
  A=\Setb x{x^2-5x+4=0},
\]
这种表示法将花括弧分两部分,用记号~$\mid$~隔开,前面为元素的代表符号,用~$x$~或其它符号,后面为元素具有的性质。

第一种表示法在数学分析中用处不大,因为我们常用的集合为无穷个元素组成,无法一一列出,如所有实数的集合就不可能
写出来。

集合~$\Set x{a\le x\le b}$~称为\emph{闭区间},记作~$\Brack{a,b}$;集合~$\Set x{a<x<b}$~称为\emph{开区间},记
作~$\Paren{a,b}$;记号~$\Paren{a,+\infty}$,$[a,+\infty)$,$\Paren{-\infty,+\infty}$~可作类似理解。

\subsection{集合的子集、包含、相等}

两个集合~$A,B$,若对任意~$a\in A$,都有~$a\in B$,这时集合~$A$~包含于集合~$B$,称~$A$~是~$B$~的\emph{子集},记作
\[
  A\subset B\quad\text{或}\quad B\supseteq A\pid
\]

例如~$A=\Brace{1,2,3,4,5}$,则集合
\[
  \Brace 1,\quad\Brace{1,2},\quad\Brace{1,3,5}
\]
是~$A$~的子集。集合~$\Brace 1$~表示由~$1$~这一元素组成的集合,概念上不同于元素~$1$~本书,我们可以记
\[
  \Brace 1\subset A,\quad 1\in A\pid
\]

为了运算方便,我们把不含任何元素的集合称为\emph{空集},记作~$\eset$。例如
\[
  \Setb x{x^2+1=0,~x~\text{是实数}}=\eset\pid
\]
空集包含于任一集合:
\[
  \eset\subset A,
\]
因为,如果不成立,则至少有一元素属于~$\eset$~而不属于~$A$,显然这是不可能的。

根据集合包含关系~$\subset$~的定义,显然有:$A\subset A$;若~$A\subset B$,$B\subset C$,则~$A\subset C$。

若两集合~$A,B$,满足~$A\subset B$,且~$B\subset A$,则称~$A$~与~$B$~相等,记作
\[
  A=B\pid
\]
如
\[
  \Brace{1,2,3,4}=\Brace{4,3,2,1}=\Brace{1,1,2,3,4}\pid
\]

若~$A\subset B$,且~$A\neq B$,则称~$A$~是~$B$~的真子集。

\subsection{集合的运算}

集合除包含关系外,还可以考虑集合之间的并、交、差等运算。

给定集合~$A,B$,集合~$A,B$~的\emph{并}记为~$A\cup B$,它是~$A,B$~全部元素组成的集合,定义为
\[
  A\cup B:=\Setb x{x\in A~\text{或}~x\in B}\pid
\]

用平面图形表示集合,图形的点表示集合的元素,则~$A,B$~图形和在一起就是并集~$A\cup B$~的图形。如图所示。

由并集的定义易见:$A\cup\eset=A$,$A\cup A=A$,
\begin{alignat*}{2}
A\cup B &=B\cup A & \qquad &\text{(交换律);}\\
(A\cup B)\cup C & = A\cup(B\cup C)&\qquad&\text{(结合律)。}
\end{alignat*}
给定集合~$A,B$,两集合的\emph{交}记为~$A\cap B$,它由~$A,B$~的公共元素组成,定义为:
\[
  A\cap B:=\Setb x{x\in A~\text{与}~x\in B}\pid
\]
集合~$A,B$~图形的公共部分(见图)就是交集的图形。

显然有:$A\cap\eset=\eset$,$A\cap A=A$,
\begin{alignat*}{2}
A\cap B &=B\cap A&\qquad &\text{(交换律),}\\
(A\cap B)\cup C & = A\cap(B\cap C)&\qquad&\text{(结合律)。}
\end{alignat*}
并且,若~$A\subset B$,则~$A\cap B=A$。


\section{绝对值与不等式}
\begin{exercise*}
\end{exercise*}


\endinput
%%
%% End of file `MAChapter0.tex'.