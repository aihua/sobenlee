%# -*- coding:utf-8 -*-
%%%%%%%%%%%%%%%%%%%%%%%%%%%%%%%%%%%%%%%%%%%%%%%%%%%%%%%%%%%%%%%%%%%%%%%%%%%%%%%%%%%%%


\chapter{陈敬济临清逢旧识\KG 韩爱姐翠馆遇情郎}


诗曰:

\[
教坊脂粉洗铅华,一片闲心对落花。
旧曲听来犹有恨,故园归去已无家。
云鬟半挽临妆镜,两泪空流湿绛纱。
今日相逢白司马,樽前重与诉琵琶。
\]

话说一日,周守备与济南府知府张叔夜,领人马剿梁山泊贼王宋江三十六人,万余草寇,都受了招安。地方平复,表奏朝廷,大喜。加升张叔夜为都御史、山东安抚大使、升备周秀为济南兵马制置,管理分巡河道,提察盗贼。部下从征有功人员,各升一级。军门带得敬济名字,升为参谋之职,月给米二石,冠带荣身。守备至十月中旬,领了敕书,率领人马来家。先使人来报与春梅家中知道。春梅满心欢喜,使陈敬济与张胜、李安出城迎接。家中厅上排设酒筵,庆官贺喜。官员人等来拜贺送礼者不计其数。守备下马,进入后堂,春梅、孙二娘接着。参贺已毕,陈敬济就穿大红员领,头戴冠帽,脚穿皂靴,束着角带,和新妇葛氏两口儿拜见。守备见好个女子,赏了一套衣服、十两银子打头面,不在话下。

晚夕,春梅和守备在房中饮酒,未免叙些家常事务。春梅道:“为娶我兄弟媳妇,又费许多东西。”守备道:“阿呀,你止这个兄弟,投奔你来,无个妻室,不成个前程道理。就是费了几两银子,不曾为了别人。”春梅道:“你今又替他挣了这个前程,足以荣身勾了。”守备道:“朝廷旨意下来,不日我往济南府到任。你在家看家,打点些本钱,教他搭个主管,做些大小买卖。三五日教他下去,查算帐目一遭,转得些利钱来,也勾他搅计。”春梅道:“你说的也是。”两个晚夕,夫妻同欢,不可细述。在家中住了十个日子,到十一月初旬时分,守备收拾起身。带领张胜、李安,前去济南到任,留周仁、周义看家。陈敬济送到城南永福寺方回。

一日,春梅向敬济商议:“守备教你如此这般,河下寻些买卖,搭个主管,觅得些利息,也勾家中费用。”这敬济听言,满心欢喜。一日,正打街前走,寻觅主管伙计。也是合当有事,不料撞遇旧时朋友陆二哥陆秉义,作揖说:“哥怎的一向不见?”敬济道:“我因亡妻为事,又被杨光彦那厮拐了我半船货物,坑陷的我一贫如洗。我如今又好了,幸得我姐姐嫁在守备府中,又娶了亲事,升做参谋,冠带荣身。如今要寻个伙计作些买卖,一地里没寻处。”陆秉义道:“杨光彦那厮拐了你货物,如今搭了个姓谢的做伙计,在临清马头上开了一座大酒店,又放债与四方趁熟窠子娼门人使,好不获大利息。他每日穿好衣,吃好肉,骑着一匹驴儿,三五日下去走一遭,算帐收钱,把旧朋友都不理。他兄弟在家开赌场,斗鸡养狗,人不敢惹他。”敬济道:“我去年曾见他一遍,他反面无情,打我一顿,被一朋友救了。我恨他入于骨髓。”因拉陆二郎入路旁一酒店内吃酒。两人计议:“如何处置他,出我这口气?”陆秉义道:“常言说得好:恨小非君子,无毒不丈夫。咱如今将理和他说,不见棺材不下泪,他必然不肯。小弟有一计策,哥也不消做别的买卖,只写一张状子,把他告到那里,追出你货物银子来。就夺了这座酒店,再添上些本钱,等我在马头上和谢三哥掌柜发卖。哥哥你三五日下去走一遭,查算帐目,管情见一月,你稳拍拍的有四十两银子利息,强如做别的生意。”看官听说,当时只因这陆秉义说出这桩事,有分数,数个人死于非命。陈敬济一种死,死之太苦;一种亡,亡之太屈。正是:

\[
非干前定数,半点不由人。
\]
敬济听了,道:“贤弟,你说的是。我到家就对我姐夫和姐姐说。这买卖成了,就安贤弟同谢三郎做主管。”当下两个吃了回酒,各下楼来,还了酒钱。敬济分付陆二哥:“兄弟,千万谨言。”陆二郎道:“我知道。”各散回家。

这敬济就一五一十对春梅说:“争奈他爷不在,如何理会?”有老家人周忠在旁,便道:“不要紧,等舅写了一张状子,该拐了多少银子货物,拿爷个拜贴儿,都封在里面。等小的送与提刑所两位官府案下,把这姓杨的拿去衙门中,一顿夹打追问,不怕那厮不拿出银子来。”敬济大喜,一面写就一纸状子,拿守备拜贴,弥封停当,就使老家人周忠送到提刑院。两位官府正升厅问事,门上人禀道:“帅府周爷差人下书。”何千户与张二官府唤周忠进见,问周爷上任之事,说了一遍。拆开封套观看,见了拜贴、状子。自恁要做分上,即便批行,差委缉捕番捉,往河下拿杨光彦去。回了个拜贴,付与周忠:“到家多上覆你爷、奶奶,待我这里追出银两,伺候来领。”周忠拿回贴到府中,回覆了春梅说话:“即时准行拿人去了。待追出银子,使人领去。”敬济看见两个折贴上面写着:“侍生何永寿、张懋德顿首拜”。敬济心中大喜。

迟不上两日光景,提刑缉捕观察番捉,往河下把杨光彦并兄弟杨二风都拿到衙门中。两位官府,据着陈敬济状子审问。一顿夹打,监禁数日,追出三百五十两银子,一百桶生眼布。其余酒店中家活,共算了五十两,陈敬济状上告着九百两,还差三百五十两银子。把房儿卖了五十两,家产尽绝。这敬济就把谢家大酒楼夺过来,和谢胖子合伙。春梅又打点出五百两本钱,共凑了一千两之数。委付陆秉义做主管,重新把酒楼装修、油漆彩画,阑干灼耀,栋宇光新,桌案鲜明,酒肴齐整。真个是:

\[
启瓮三家醉,开樽十里香。
神仙留玉佩,卿相解金貂。
\]

从正月半头,陈敬济在临清马头上大酒楼开张,见一日也发卖三五十两银子。都是谢胖子和陆秉义眼同经手,在柜上掌柜。敬济三五日骑头口,伴当小姜儿跟随,往河下算帐一遭。若来,陆秉义和谢胖子两个伙计,在楼上收拾一间干净阁儿,铺陈床帐,安放卓椅,糊的雪洞般齐整。摆设酒席,交四个好出色粉头相陪。陈三儿那里往来做量酒。

一日,三月佳节,春光明媚,景物芬芳,翠依依槐柳盈堤,红馥馥杏桃灿锦。陈敬济在楼上,搭伏定绿阑干,看那楼下景致,好生热闹。有诗为证:

\[
风拂烟笼锦绣妆,太平时节日初长。
能添壮士英雄胆,善解佳人愁闷肠。
三尺晓垂杨柳岸,一竿斜插杏花旁。
男儿未遂平生志,且乐高歌入醉乡。
\]

一日,敬济在楼窗后瞧看,正临着河边,泊着两只剥船。船上载着许多箱笼,卓凳家活,四五个人,尽搬入楼下空屋里来。船上有两个妇人,一个中年妇人,长挑身材,紫膛色;一个年小妇人,搽脂抹粉,生的白净标致,约有二十多岁。尽走入屋里来。敬济问谢主管:“是甚么人?也不问一声,擅自搬入我屋里来。”谢主管道:“此两个是东京来的妇人,投亲不着,一时间无处寻房住,央此间邻居范老来说,暂住两三日便去。正欲报知官人,不想官人来问。”这敬济正欲发怒,只见那年小妇人敛衽向前,望敬济深深的道了个万福,告说:“官人息怒,非干主管之事,是奴家大胆,一时出于无奈,不及先来宅上禀报,望乞恕罪。容略住得三五日,拜纳房金,就便搬去。”这敬济见小妇人会说话儿,只顾上上下下把眼看他。那妇人一双星眼斜盼敬济,两情四目,不能定情。敬济口中不言,心内暗想:“倒相那里会过,这般眼熟。”那长挑身材中年妇人,也定睛看着敬济,说道:“官人,你莫非是西门老爷家陈姑爷么?”这敬济吃了一惊,便道:“你怎的认得我?”那妇人道:“不瞒姑爷说,奴是旧伙计韩道国浑家,这个就是我女孩儿爱姐。”敬济道:“你两口儿在东京,如何来在这里?你老公在那里?”那妇人道:“在船上看家活。”敬济急令量酒请来相见。

不一时,韩道国走来作揖,已是掺白须鬓,因说起:“韩中蔡太师、童太尉、李右相、朱太尉、高太尉、李太监六人,都被太学国子生陈东上本参劾,后被科道交章弹奏倒了。圣旨下来,拿送三法司问罪,发烟瘴地面,永远充军。太师儿子礼部尚书蔡攸处斩,家产抄没入官。我等三口儿各自逃生,投到清河县寻我兄弟第二的。不想第二的把房儿卖了,流落不知去向。三口儿雇船,从河道中来,不料撞遇姑夫在此,三生有幸。”因问:“姑夫今还在西门老爷家里?”敬济把头项摇了一摇,说:“我也不在他家了。我在姐夫守备周爷府中,做了参谋官,冠带荣身。近日合了两个伙计,在此马头上开这个酒店,胡乱过日子。你每三口儿既遇着我,也不消搬去,便在此间住也不妨,请自稳便。”妇人与韩道国一齐下礼。说罢,就搬运船上家活箱笼上来。敬济看得心痒,也使伴当小姜儿和陈三儿替他搬运了几件家活。王六儿道:“不劳姑夫费心用力。”彼此俱各欢喜。敬济道:“你我原是一家,何消计较?”敬济见天色将晚,有申牌时分,要回家。分付主管:“咱蚤送些茶盒与他。”上马,伴当跟随来家,一夜心心念念,只是放韩爱姐不下。

过了一日,到第三日早起身,打扮衣服齐整,伴当小姜跟随来河下大酒楼店中,看着做了回买卖。韩道国那边使的八老来请吃茶。敬济心下正要瞧去,恰好八老来请,便起身进去。只见韩爱姐见了,笑容可掬,接将出来,道了万福:“官人请里面坐。”敬济到阁子内会下,王六儿和韩道国都来陪坐。少顷茶罢,彼此叙此旧时的闲话,敬济不住把眼只睃那韩爱姐,爱姐一双一双涎澄澄秋波只看敬济,彼此都有意了。有诗为证:

\[
弓鞋窄窄剪春罗,香体酥胸玉一窝。
丽质不胜袅娜态,一腔幽恨蹙秋波。
\]

少顷,韩道国走出去了。爱姐因问:“官人青春多少?”敬济道:“虚度二十六岁。”敬济问:“姐姐青春几何?”爱姐笑道:“奴与官人一缘一会,也是二十六岁。旧日又是大老爹府上相会过面,如何又幸遇在一处,正是有缘千里来相会。”那王六儿见他两个说得入港,看见关目,推个故事,也走出去了。止有他两人对坐。爱姐把些风月话儿来勾敬济,敬济自幼干惯的道儿,怎不省得!便涎着脸儿,调戏答话。原来这韩爱姐从东京来,一路儿和他娘已做些道路。今见了敬济,也是夙世有缘,三生一笑,不由的情投意合,见无人处,就走向前,挨在他身边坐下,作娇作痴,说道:“官人,你将头上金簪子借我看一看。”敬济正欲拔时,早被爱姐一手按住敬济头髻,一手拔下簪子来。便笑吟吟起身,说:“我和你去楼上说句话儿。”一头说,一头走。敬济得不的这一声,连忙跟上楼来。正是:

\[
风来花自舞,春入鸟能言。
\]
敬济跟他上楼,便道:“姐姐有甚话说?”爱姐道:“奴与你是宿世姻缘,今朝相遇,愿偕枕席之欢,共效于飞之乐。”敬济道:“难得姐姐见怜,只怕此间有人知觉。”韩爱姐做出许多妖娆来,搂敬济在怀,将尖尖玉手扯下他裤子来。两个情兴如火,按纳不住,爱姐不免解衣仰卧,在床上交媾在一处。正是:

\[
色胆如天怕甚事,鸳帏云雨百年情。
\]
敬济问:“你叫几姐?”那韩爱姐道:“奴是端午所生,就叫五姐,又名爱姐。”霎时云收雨散,偎倚共坐。韩爱姐将金簪子原插在他头上,又告敬济说:“自从三口儿东京来,投亲不着,盘缠缺欠。你有银子,见借与我父亲五两,奴按利纳还,不可推阻。”敬济应允,说:“不打紧,姐姐开口,就兑五两来。”两个又坐了半日,恐怕人谈论,吃了一杯茶,爱姐留吃午饭,敬济道:“我那边有事,不吃饭了,少间就送盘缠来与你。”爱姐道:“午后奴略备一杯水酒,官人不要见却,好歹来坐坐。”

敬济在店内吃了午饭,又在街上闲散走了一回。撞见昔日晏公庙师兄金宗明作揖,把前事诉说了一遍。金宗明道:“不知贤弟在守备老爷府中认了亲,在大楼开店,有失拜望。明日就使徒弟送茶来,闲中请去庙中坐一坐。”说罢,宗明归去了。敬济走到店中,陆主管道:“里边住的老韩请官人吃酒,没处寻。”正说着,恰好八老又来请。就请二位主管相陪,再无他客。敬济就同二主管,走到里边房内,蚤已安排酒席齐整。敬济上坐,韩道国主位,陆秉义、谢胖子打横,王六儿与爱姐旁边佥坐,八老往来筛酒下菜。吃过数杯,两个主管会意,说道:“官人慢坐,小人柜上看去。”起身去了。敬济平昔酒量,不十分洪饮,又见主管去了,开怀与韩道国三口儿吃了数杯,便觉有些醉将上来。爱姐便问:“今日官人不回家去罢了?”敬济道:“这咱晚了,回去不得,明日起身去罢。”王六儿、韩道国吃了一回,下楼去了。敬济向袖中取出五两银子,递与爱姐。爱姐到下边交与王六儿,复上来。两个交杯换盏,倚翠偎红,吃至天晚。爱姐卸下浓妆,留敬济就在楼上阁儿里歇了。当下枕畔山盟,衾中海誓,莺声燕语,曲尽绸缪,不能悉记。爱姐在东京蔡太师府中,与翟管家做妾,曾扶持过老太太,也学会些弹唱,又能识字会写,种种可人。敬济欢喜不胜,就同六姐一般,正可在心上。以此与他盘桓一夜,停眠罢宿,免不的第二日起来得迟,约饭时才起来。王六儿安排些鸡子肉圆子,做了个头脑与他扶头。两个吃了几杯暖酒。少顷主管来,请敬济那边摆饭。敬济梳洗毕,吃了饭,又来辞爱姐,要回去。那爱姐不舍,只顾抛泪。敬济道:“我到家三、五日,就来看你,你休烦恼。”说毕,伴当跟随,骑马往城中去了。一路上分付小姜儿:“到家休要说出韩家之事。”小姜儿道:“小的知道,不必分付。”

敬济到府中,只推店中买卖忙,算了帐目不觉天晚,归来不得,歇了一夜。交割与春梅利息银两,见一遭儿也有三十两银子之数。回到家中,又被葛翠屏噪聒:“官人怎的外边歇了一夜?想必在柳陌花街行踏,把我丢在家中,独自空房,就不思想来家。”一连留住陈敬济七八日,不放他往河下来。店中只使小姜儿,来问主管讨算利息。主管一一封了银子去。

韩道国免不得又交老婆王六儿又招惹别的熟人儿,或是商客来屋里走动,吃茶吃酒。这韩道国先前尝着这个甜头,靠老婆衣饭肥家。况王六儿年纪虽老,风韵犹存,恰好又得他女儿来接代,也不断绝这样行业,如今索性大做了。当下见敬济不来,量酒陈三儿替他勾了一个湖州贩丝绵客人何官人来,请他女儿爱姐。那何官人年约五十余岁,手中有千两丝绵绸绢货物,要请爱姐。爱姐一心想着敬济,推心中不快,三回五次不肯下楼来,急的韩道国要不的。那何官人又见王六儿长挑身材,紫膛色,瓜子面皮,描的大大小鬓,涎邓邓一双星眼,眼光如醉,抹的鲜红嘴唇,料此妇人一定好风情,就留下一两银子,在屋里吃酒,和王六儿歇了一夜。韩道国便躲避在外边歇了,他女儿见做娘的留下客,只在楼上不下楼来,自此以后,那何官人被王六儿搬弄得快活,两个打得一似火炭般热,没三两日不来与他过夜。韩道国也禁过他许多钱使。

这韩爱姐见敬济一去十数日不来,心中思想,挨一日似三秋,盼一夜如半夏,未免害木边之目,田下之心。使八老往城中守备府中探听。看见小姜儿,悄悄问他:“官人如何不去?”小姜儿说:“官人这两日有些身子不快,不曾出门。”回来诉与爱姐。爱姐与王六儿商议,买了一副猪蹄,两只烧鸭,两尾鲜鱼,一盒酥饼,在楼上磨墨挥笔,写封柬帖,使八老送到城中与敬济去,叮咛嘱付:“你到城中,须索见陈官人亲收,讨回贴来。”八老怀内揣着柬帖,挑着礼物,一路无词。来到城内守备府前,坐在沿街石台基上。只见伴当小姜儿出来,看见八老:“你又来做甚么?”八老与他声喏,拉在僻净处说:“我特来见你官人,送礼来了。还有话说,我只有此等你。你可通报官人知道。”小姜随即转身进去。不多时,只见敬济摇将出来。那时约五月,天气暑热。敬济穿着纱衣服,头戴着瓦楞帽,凉鞋净袜。八老慌忙声喏,说道:“官人贵体好些?韩爱姐使我稍一柬帖,送礼来了。”敬济接了柬帖,说:“五姐好么?”八老道:“五姐见官人一向不去,心中也不快在那里。多上覆官人,几时下去走走?”敬济拆开柬帖观看上面写着甚言词:

\[
贱妾韩爱姐敛衽拜,谨启情郎陈大官人台下:自别尊颜,思慕之心未尝少怠。向蒙期约,妾倚门凝望,不见降临。昨遣八老探问起居,不遇而回。闻知贵恙欠安,令妾空怀账望,坐卧闷恹,不能顿生两翼而傍君之左右也。君在家,自有娇妻美爱,又岂肯动念于妾,犹吐去之果核也。兹具腥味、茶盒数事,少伸问安诚意,幸希笑纳。情照不宣。外具锦绣鸳鸯香囊一个,青丝一缕,少表寸心。仲夏念日贱妾爱姐再拜。
\]

敬济看了柬帖并香囊。香囊里面安放青丝一缕,香囊上扣着“寄与情郎陈君膝下”八字,依先折了,藏在袖中。府旁侧首有个酒店,令小姜儿:“领八老同店内吃钟酒,等我写回帖与你。”小姜不敢怠慢,把四盒礼物收进去了。敬济走到书院房内,悄悄写了回柬,又包了五两银子,到酒店内问八老:“吃了酒不曾?”八老道:“多谢官人好酒,吃不得了,起身去罢。”敬济将银子并回柬付与八老,说:“到家多多拜上五姐,这五两白金与他盘缠,过三两日,我自去看他。”八老收了银、柬,一直去了。敬济回家,走入房中,葛翠屏便问:“是谁家送的礼物?”敬济悉言:“店主人谢胖子,打听我不快,送礼物来问安。”翠屏亦信其实。两口儿计议,交丫鬟金钱儿拿盘子,拿了一只烧鸭,一尾鲜血,半副蹄子,送到后边与春梅吃,说是店主人家送的,也不查问。此事表过不题。

却说八老到河下,天已晚了,入门将银、柬都付与爱姐收了。拆开银、柬,灯下观看,上面写道:

\[
爱弟敬济顿首字覆爱卿韩五姐妆次:向蒙会问,又承厚款,亦且云情雨意,祚席钟爱,无时少怠。所云期望,正欲趋会,偶因贱躯不快,有失卿之盼望。又蒙遣人垂顾,兼惠可口佳肴,锦囊佳制,不胜感激!只在二三日间,容当面布。外具白金五两,绫帕一方,少伸远芹之敬,优乞心鉴,万万。敬济再拜。
\]
爱姐看了,见帕上写着四句诗曰:

\[
吴绫帕儿织回文,洒翰挥毫墨迹新。
寄与多情韩五姐,永谐鸾凤百年情。
\]
看毕,爱姐把银子付与王六儿。母子千欢万喜,等候敬济,不在话下。正是:得意友来情不厌,知心人至话相投。有诗为证:

\[
碧纱窗下启笺封,一纸云鸿香气浓。
知你挥毫经玉手,相思都付不言中。
\]
