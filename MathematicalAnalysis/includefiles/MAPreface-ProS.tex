%# -*- coding:utf-8 -*-
%%%%%%%%%%%%%%%%%%%%%%%%%%%%%%%%%%%%%%%%%%%%%%%%%%%%%%%%%%%%%%%%%%%%%%%%%%%%%%%%%%%%%
%%  `MAPreface-ProS.tex'

\begin{preface}

由于教材内容不断更新,特别地,我们编的《数学分析》教材中的多元微积分是直接在~$m$~维~Euclid~空间中讨论的,这就要求
习题也应增加相应的内容。而原来广泛采用的~Б. П. Демидович~《数学分析习题集》没有这部分内容的题目,加之因
该书题解的出现,在一定程度上失去了它的训练价值。有鉴于此,我们编撰了这本适合数学专业类使用的习题集。

本集中的习题主要是根据我系(北京大学数学系)习题课资料编撰的。如一元函数部分中让读者自己去判断是非的证明题,就是针
对学生经常出现的一些错误而编写的。习题集也吸收了~1983~年以前历届研究生考试中的部分试题,以及专门化课中遇到的数学分
析的问题。习题中有些内容也是对教材内容的进一步补充,如除原点外~Taylor~级数处处发散的反例,~L'H\^opital~法则的反问
题等等。

本习题集是我系编写的《数学分折》一书的配套教材。除书中个别节无习题外,习题集的章节与书的章节对应,两者顺序是一致的。%
为了查找方便,习题的题号用三个数字表示,第一个数字表示书中的章号、第二个数字表示书中的节号、第三个数字表示习题的题
号。每章习题分基本题与难题两类,两者用星号隔开。基本题中计算题与概念题的数量,对初学者来说稍多些,但基本上可以全做;%
证明题的数量较多,对于我们认为较难的题目都给出了提示,这部分题目,初学者不必全做,能做一半也就可以了。

习趣集中第〇章至第十四章习题由林源渠和方企勤两位同志编写,其中定积分与级数的一部分题目是沈燮昌同志编的。第十五章至
第十九章习题由李正元同志编写,第二十章至第二十四章习题由廖可人同志编写。我系担任过数学分析习题课的同志曾使用本习题
集初稿进行教学,并提出宝贵意见,欧阳光中副教授、董延闿教授审阅书稿时对习题集提出了不少宝贵意见,高等教育出版社的文
小西同志在书稿通读加工中也提出不少宝贵意见,在此向他们表示深深地谢意。

\Sign[编\enspace 者]{1985~年~6~月}
\end{preface}

\endinput
%%
%% End of file `MAPreface-ProS.tex'.