%# -*- coding:utf-8 -*-
%%%%%%%%%%%%%%%%%%%%%%%%%%%%%%%%%%%%%%%%%%%%%%%%%%%%%%%%%%%%%%%%%%%%%%%%%%%%%%%%%%%%%


\chapter{潘金莲售色赴东床\KG 李娇儿盗财归丽院}


诗曰:

\[
倚醉无端寻旧约,却因惆怅转难胜。
静中楼阁深春雨,远处帘栊半夜灯。
抱柱立时风细细,绕廊行处思腾腾。
分明窗下闻裁剪,敲遍栏杆唤不应。
\]

话说西门庆死了,首七那日,却是报国寺十六众僧人做水陆。这应伯爵约会了谢希大、花子繇、祝实念、孙天化、常峙节、白赉光七人,坐在一处,伯爵先开口说:“大官人没了,今一七光景。你我相交一场,当时也曾吃过他的,也曾用过他的,也曾使过他的,也曾借过他的。今日他死了,莫非推不知道?洒土也眯眯后人眼睛儿,他就到五阎王跟前,也不饶你我。如今这等计较,你我各出一钱银子,七人共凑上七钱,办一桌祭礼,买一幅轴子,再求水先生作一篇祭文,抬了去,大官人灵前祭奠祭奠,少不的还讨了他七分银子一条孝绢来,这个好不好?”众人都道:“哥说的是。”当下每人凑出银子来,交与伯爵,整备祭物停当,买了轴子,央水秀才做了祭文。这水秀才平昔知道应伯爵这起人,与西门庆乃小人之朋,于是暗含讥刺,作就一篇祭文。伯爵众人把祭祀抬到灵前摆下,陈敬济穿孝在旁还礼。伯爵为首,各人上了香,人人都粗俗,那里晓得其中滋味。浇了奠酒,只顾把祝文宣念。其文略曰:

\[
维重和元年,岁戊戌,二月戊子期,越初三日庚寅,侍教生应伯爵、谢希大、花子繇、祝实念、孙天化、常峙节、白赉光,谨以清酌庶馐之仪,致祭于故锦衣西门大官人之灵曰:维灵生前梗直,秉性坚刚;软的不怕,硬的不降。常济人以点水,恒助人以精光。囊箧颇厚,气概轩昂。逢乐而举,遇阴伏降。锦裆队中居住,齐腰库里收藏。有八角而不用挠掴,逢虱虮而骚痒难当。受恩小子,常在胯下随帮。也曾在章台而宿柳,也曾在谢馆而猖狂。正宜撑头活脑,久战熬场,胡为罹一疾不起之殃?见今你便长伸着脚子去了,丢下小子辈,如班鸠跌脚,倚靠何方?难上他烟花之寨,难靠他八字红墙。再不得同席而儇软玉,再不得并马而傍温香。撇的人垂头落脚,闪的人牢温郎当。今特奠兹白浊,次献寸觞。灵其不昧,来格来歆。尚飨。
\]
众人祭毕,陈敬济下来还礼,请去卷棚内三汤五割,管待出门不题。

且说那日院中李家虔婆,听见西门庆死了,铺谋定计,备了一张祭桌,使了李桂卿、李桂姐坐轿子来上纸吊问。月娘不出来,都是李娇儿、孟玉楼在上房管待。李家桂卿、桂姐悄悄对李娇儿说:“俺妈说,人已是死了,你我院中人,守不的这样贞节!自古千里长棚,没个不散的筵席。教你手里有东西,悄悄教李铭稍了家去防后。你还恁傻!常言道:‘扬州虽好,不是久恋之家。’不拘多少时,也少不的离他家门。”那李娇儿听记在心。

不想那日韩道国妻王六儿,亦备了张祭桌,乔素打扮,坐轿子来与西门庆烧纸。在灵前摆下祭祀,只顾站着。站了半日,白没个人儿出来陪待。原来西门庆死了,首七时分,就把王经打发家去不用了。小厮每见王六儿来,都不敢进去说。那来安儿不知就里,到月娘房里,向月娘说:“韩大婶来与爹上纸,在前边站了一日了,大舅使我来对娘说。”这吴月娘心中还气忿不过,便喝骂道:“怪贼奴才,不与我走,还来甚么韩大婶、\textuni{23B48}大婶,贼狗攮的养汉淫妇,把人家弄的家败人亡,父南子北,夫逃妻散的,还来上甚么\textuni{23B48}纸!”一顿骂的来安儿摸门不着,来到灵前。吴大舅问道:“对后边说了不曾?”来安儿把嘴谷都着不言语。问了半日,才说:“娘稍出四马儿来了。”这吴大舅连忙进去,对月娘说:“姐姐,你怎么这等的?快休要舒口!自古人恶礼不恶。他男子汉领着咱偌多的本钱,你如何这等待人?好名儿难得,快休如此。你就不出去,教二姐姐、三姐姐好好待他出去,也是一般。做甚么恁样的,教人说你不是。”那月娘见他哥这样说,才不言语了。良久,孟玉楼出来,还了礼,陪他在灵前坐的。只吃一钟茶,妇人也有些省口,就坐不住,随即告辞起身去了。正是:

\[
谁人汲得西江水,难免今朝一面羞。
\]

那李桂卿、桂姐、吴银儿都在上房坐着,见月娘骂韩道国老婆淫妇长、淫妇短,砍一株损百枝,两个就有些坐不住,未到日落,就要家去。月娘再三留他姐儿两个:“晚夕伙计每伴宿,你每看了提偶,明日去罢。”留了半日,桂姐、银姐不去了,只打发他姐姐桂卿家去了。到了晚夕,僧人散了,果然有许多街坊、伙计、主管,乔大户、吴大舅、吴二舅、沈姨父、花子繇、应伯爵、谢希大、常峙节,也有二十余人,叫了一起偶戏,在大卷棚内,摆设酒席伴宿。提演的是“孙荣、孙华杀狗劝夫”戏文。堂客都在灵旁厅内,围着帏屏,放下帘来,摆放桌席,朝外观看。李铭、吴惠在这里答应,晚夕也不家去了。不一时,众人都到齐了。祭祀已毕,卷棚内点起烛来,安席坐下,打动鼓乐,戏文上来。直搬演到三更天气,戏文方了。

原来陈敬济自从西门庆死后,无一日不和潘金莲两个嘲戏,或在灵前溜眼,帐子后调笑。于是赶人散一乱,众堂客都往后边去了,小厮每都收家活,这金莲赶眼错,捏了敬济一把,说道:“我儿,你娘今日成就了你罢。趁大姐在后边,咱就往你屋里去罢。”敬济听了,得不的一声,先往屋里开门去了。妇人黑影里,抽身钻入他房内,更不答话,解开裤子,仰卧在炕上,双凫飞首,教陈敬济好耍。正是:色胆如天怕甚事,鸳帏云雨百年情。真个是:

\[
二载相逢,一朝配偶;数年姻眷,一旦和谐。一个柳腰款摆,一个玉茎忙舒。耳边诉雨意云情,枕上说山盟海誓。莺恣蝶采,旖妮搏弄百千般;狂雨羞云,娇媚施逞千万态。一个不住叫亲亲,一个搂抱呼达达。得多少柳色乍翻新样绿,花容不减旧时红。
\]

霎时云雨了毕,妇人恐怕人来,连忙出房,往后边去了。到次日,这小伙儿尝着这个甜头儿,早辰走到金莲房来,金莲还在被窝里未起来。从窗眼里张看,见妇人被拥红云,粉腮印玉,说道:“好管库房的,这咱还不起来!今日乔亲家爹来上祭,大娘分付把昨日摆的李三、黄四家那祭桌收进来罢。你快些起来,且拿钥匙出来与我。”妇人连忙教春梅拿钥匙与敬济,敬济先教春梅楼上开门去了。妇人便从窗眼里递出舌头,两个咂了一回。正是得多少脂香满口涎空咽,甜唾颙心溢肺奸。有词为证:

\[
恨杜鹃声透珠帘。心似针签,情似胶粘。我则见笑脸腮窝愁粉黛,瘦损春纤宝髻乱,云松翠钿。睡颜酡,玉减红添。檀口曾沾。到如今唇上犹香,想起来口内犹甜。
\]

良久,春梅楼上开了门,敬济往前边看搬祭祀去了。不一时,乔大户家祭来摆下。乔大户娘子并乔大户许多亲眷,灵前祭毕。吴大舅、吴二舅、甘伙计陪侍,请至卷棚内管待。李铭、吴惠弹唱。那日郑爱月儿家也来上纸吊孝。月娘俱令玉楼打发了孝裙束腰,后边与堂客一同坐的。郑爱月儿看见李桂姐、吴银姐都在这里,便嗔他两个不对他说:“我若知道爹没了,有个不来的!你每好人儿,就不会我会儿去。”又见月娘生了孩儿,说道:“娘一喜一忧。惜乎爹只是去世太早了些儿,你老人家有了主儿,也不愁。”月娘俱打发了孝,留坐至晚方散。

到二月初三日,西门庆二七,玉皇庙吴道官十六众道士,在家念经做法事。那日衙门中何千户作创,约会了刘、薛二内相,周守备、荆都统、张团练、云指挥等数员武官,合着上了坛祭。月娘这里请了乔大户、吴大舅、应伯爵来陪待,李铭、吴惠两个小优儿弹唱,卷棚管待去了。俱不必细说。到晚夕念经送亡。月娘分付把李瓶儿灵床连影抬出去,一把火烧了。将箱笼都搬到上房内堆放。奶子如意儿并迎春收在后边答应,把绣春与了李娇儿房内使唤。将李瓶儿那边房门,一把锁锁了。可怜正是:画栋雕梁犹未干,堂前不见痴心客。有诗为证:

\[
襄王台下水悠悠,一种相思两样愁。
月色不如人事改,夜深还到粉墙头。
\]

那时李铭日日假以孝堂助忙,暗暗教李娇儿偷转东西与他掖送到家,又来答应,常两三夜不往家去,只瞒过月娘一人眼目。吴二舅又和李娇儿旧有首尾,谁敢道个不字。初九日念了三七经,月娘出了暗房,四七就没曾念经。十二日,陈敬济破了土回来。二十日早发引,也有许多冥器纸札,送殡之人终不似李瓶儿那时稠密。临棺材出门,也请了报恩寺朗僧官起棺,坐在轿上,捧的高高的,念了几句偈文。念毕,陈敬济摔破纸盆,棺材起身,合家大小孝眷放声号哭。吴月娘坐魂轿,后面坐堂客上轿,都围随材走,径出南门外五里原祖茔安厝。陈敬济备了一匹尺头,请云指挥点了神主,阴阳徐先生下了葬。众孝眷掩土毕。山头祭桌,可怜通不上几家,只是吴大舅、乔大户、何千户、沈姨夫、韩姨夫与众伙计五六处而已。吴道官还留下十二众道童回灵,安于上房明间正寝。阴阳洒扫已毕,打发众亲戚出门。吴月娘等不免伴夫灵守孝。一日暖了墓回来,答应班上排军节级,各都告辞回衙门去了。西门庆五七,月娘请了薛姑子、王姑子、大师父、十二众尼僧,在家诵经礼忏,超度夫主生天。吴大妗子并吴舜臣媳妇,都在家中相伴。

原来出殡之时,李桂卿同桂姐在山头,悄悄对李娇儿如此这般:“妈说,你摸量你手中没甚细软东西,不消只顾在他家了。你又没儿女,守甚么?教你一场嚷乱,登开了罢。昨日应二哥来说,如今大街坊张二官府,要破五百两金银,娶你做二房娘子,当家理纪。你那里便图出身,你在这里守到老死,也不怎么。你我院中人家,弃旧迎新为本,趋火附势为强,不可错过了时光。”这李娇儿听记在心,过了西门庆五七之后,因风吹火,用力不多。不想潘金莲对孙雪娥说,出殡那日,在坟上看见李娇儿与吴二舅在花园小房内,两个说话来。春梅孝堂中又亲眼看见李娇儿帐子后递了一包东西与李铭,塞在腰里,转了家去。嚷的月娘知道,把吴二舅骂了一顿,赶去铺子里做买卖,再不许进后边来。分付门上平安,不许李铭来往。这花娘恼羞变成怒,正寻不着这个由头儿哩。一日因月娘在上房和大妗子吃茶,请孟玉楼,不请他,就恼了,与月娘两个大闹大嚷,拍着西门庆灵床子,啼啼哭哭,叫叫嚎嚎,到半夜三更,在房中要行上吊。丫头来报与月娘。月娘慌了,与大妗子计议,请将李家虔婆来,要打发他归院。虔婆生怕留下他衣服头面,说了几句言语:“我家人在你这里做小伏低,顶缸受气,好容易就开交了罢!须得几十两遮羞钱。”吴大舅居着官,又不敢张主,相讲了半日,教月娘把他房中衣服、首饰、箱笼、床帐、家活尽与他,打发出门。只不与他元宵、绣春两个丫头去。李娇儿生死要这两个丫头。月娘生死不与他,说道:“你倒好,买良为娼。”一句慌了鸨子,就不敢开言,变做笑吟吟脸儿,拜辞了月娘,李娇儿坐轿子,抬的往家去了。

看官听说,院中唱的,以卖俏为活计,将脂粉作生涯;早辰张风流,晚夕李浪子;前门进老子,后门接儿子;弃旧怜新,见钱眼开,自然之理。饶君千般贴恋,万种牢笼,还锁不住他心猿意马。不是活时偷食抹嘴,就是死后嚷闹离门。不拘几时,还吃旧锅粥去了。正是:蛇入筒中曲性在,鸟出笼轻便飞腾。有诗为证:

\[
堪笑烟花不久长,洞房夜夜换新郎。
两只玉腕千人枕,一点朱唇万客尝。
造就百般娇艳态,生成一片假心肠。
饶君总有牢笼计,难保临时思故乡。
\]

月娘打发李娇儿出门,大哭了一场。众人都在旁解劝,潘金莲道:“姐姐,罢,休烦恼了。常言道,娶淫妇,养海青,食水不到想海东。这个都是他当初干的营生,今日教大姐姐这等惹气。”

家中正乱着,忽有平安来报:“巡盐蔡老爹来了,在厅上坐着哩,我说家老爹没了。他问没了几时了,我回正月二十一日病故,到今过了五七。他问有灵没灵,我回有灵,在后边供养着哩。他要来灵前拜拜,我来对娘说。”月娘分付:“教你姐夫出去见他。”不一时,陈敬济穿上孝衣出去,拜见了蔡御史。良久,后边收拾停当,请蔡御史进来西门庆灵前参拜了。月娘穿着一身重孝,出来回礼,再不交一言,就让月娘说:“夫人请回房。”又向敬济说道:“我昔时曾在府相扰,今差满回京去,敬来拜谢拜谢,不期作了故人。”便问:“甚么病症?”陈敬济道:“是痰火之疾。”蔡御史道:“可伤,可伤。”即唤家人上来,取出两匹杭州绢,一双绒袜,四尾白鲞,四罐蜜饯,说道:“这些微礼,权作奠仪罢。”又拿出五十两一封银子来,“这个是我向日曾贷过老先生些厚惠,今积了些俸资奉偿,以全终始之交。”分付平安道:“大官,交进房去。”敬济道:“老爹忒多计较了。”月娘说:“请老爹前厅坐。”蔡御史道:“也不消坐了。拿茶来,吃了一钟就是了。”左右须臾拿茶上来。蔡御史吃了,扬长起身上轿去了。月娘得了这五十两银子,心中又是那欢喜,又是那惨戚。想有他在时,似这样官员来到,肯空放去了?又不知吃酒到多咱晚。今日他伸着脚子,空有家私,眼看着就无人陪待。正是:

\[
人得交游是风月,天开图画即江山。
\]

话说李娇儿到家,应伯爵打听得知,报与张二官知,就拿着五两银子来,请他歇了一夜。原来张二官小西门庆一岁,属兔的,三十二岁了。李娇儿三十四岁,虔婆瞒了六岁,只说二十八岁,教伯爵瞒着。使了三百两银子,娶到家中,做了二房娘子。祝实念、孙寡嘴依旧领着王三官儿,还来李家行走,与桂姐打热,不在话下。

伯爵、李三、黄四借了徐内相五千两银子,张二官出了五千两,做了东平府古器这批钱粮,逐日宝鞍大马,在院内摇摆。张二官见西门庆死了,又打点了上千两金银,往东京寻了枢密院郑皇亲人情,对堂上朱太尉说,要讨提刑所西门庆这个缺。家中收拾买花园,盖房子。应伯爵无日不在他那边趋奉,把西门庆家中大小之事,尽告诉与他,说:“他家中还有第五个娘子潘金莲,排行六姐,生的上画儿般标致,诗词歌赋,诸子百家,拆牌道字,双陆象棋,无不通晓。又写的一笔好字,弹的一手好琵琶。今年不上三十岁,比唱的还乔。”说的那张二官心中火动,巴不的就要了他,便问道:“莫非是当初卖炊饼的武大郎那老婆么?”伯爵道:“就是他。占来家中,今也有五六年光景,不知他嫁人不嫁。”张二官道:“累你打听着,待有嫁人的声口,你来对我说,等我娶了罢。”伯爵道:“我身子里有个人,在他家做家人,名来爵儿。等我对他说,若有出嫁声口,就来报你知道。难得你娶过他这个人来家,也强似娶个唱的。当时西门庆大官人在时,为娶他,不知费了许多心。大抵物各有主,也说不的,只好有福的匹配,你如有了这般势耀,不得此女貌,同享荣华,枉自有许多富贵。我只叫来爵儿密密打听,但有嫁人的风缝儿,凭我甜言美语,打动春心,你却用几百两银子,娶到家中,尽你受用便了。”看官听说,但凡世上帮闲子弟,极是势利小人。当初西门庆待应伯爵如胶似漆,赛过同胞弟兄,那一日不吃他的,穿他的,受用他的。身死未几,骨肉尚热,便做出许多不义之事。正是画虎画皮难画骨,知人知面不知心。有诗为证:

\[
昔年音气似金兰,百计趋奉不等闲。
自从西门身死后,纷纷谋妾伴人眠。
\]
