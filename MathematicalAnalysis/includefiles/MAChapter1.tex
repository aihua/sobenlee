%# -*- coding:utf-8 -*-
%%%%%%%%%%%%%%%%%%%%%%%%%%%%%%%%%%%%%%%%%%%%%%%%%%%%%%%%%%%%%%%%%%%%%%%%%%%%%%%%%%%%%
%%  MAChapter1.tex'


\chapter{函\emspace 数}\label{ch:1}

\section{函数概念}\label{sec:1.1}

高等数学与初等数学的区别,在于研究的对象和研究的方法不同。初等数学研究的对象主要是常量,如追赶问题中,已知甲、乙的速度与
出发时间,要求甲追上乙的时间,这里要求的只是一个数量——时间;而高等数学所研究的对象是事物的运动规律和现象的变化规律。如
考察初速度为零的物体,在真空中自由下落时,Galileo~发现物体下落的距离与下落的时间平方成正比。怎么由这一运动规律,求出物体
速度和加速度时间的变化规律。这里我们要求的已不是一个数,而是物体的运动规律。在某种意义上,我们可以说初等数学主要是常量的
数学,高等数学是变量的数学。

\subsection{变量与常量}

在生产与生活中,我们接触到各种各样的量。有些量在考察过程中是变化的,取着不同的数值,我们称之为\emph{变量};有些量在考察过
程中是不变化的,取相同的值,我们称之为\emph{常量}。

如火车出战、进站、过桥、拐弯时,速度是在变的,速度时快时慢,所以从考察全行程来说,火车的速度为一变量。若考察火车行驶在两站
之间某一行程时,火车的速度也可以是常量。又如地面两观察站,观察空中卫星的位置时,观察站与卫星间的距离及联线间的角度是变量,%
但构成三角形的内角和是不变的,为一常量。

需要指出的是,常量与变量是相对的。一是指实践中把一个量究竟作为变量处理,还是作为常量处理是相对的。如重力加速度~$g$,在有的
问题中我们把各地的~$g$~看成常量,但在重力探矿中就必须把~$g$~看成变量。一是指数学上常量与变量的区分也是相对的,常量我们也把
它看作变量,但这变量在变化过程中总是取同一个数值。这样一来,我们可以说两个变量之和仍为一变量。否则我们只能说,两个变量之和,%
一般来说是变量,但在特殊情况下也可以为一常量,显然这对讨论问题带来很大的不便。

习惯上,变量常用字母~$x,y,t$~来表示,常量常用字母~$a,b,c$~等表示。

还需要指出的是,严格的讲法,应该是从集合出发,只有元素属于或不属于集合的区分,没有常量与变量的区分,所谓“变量”不过是集合
元素的代表符号。我们这里采用常量、变量等术语,是为了使问题显得形象、直观,以便于初学者理解,而不拘泥与术语上的严格性。

\subsection{函数定义}

在实际问题中,我们关心的不是孤立的量,而是量与量之间的依赖关系,即每一量如何随着另一量的变化而变化。这里我们暂时只限于讨论
两个变量的情形,并从几个具体的例子来说明量与量之间存在的依赖关系。

\begin{wrapfigure}[7]{O}{0mm}
\somefigure
\caption{}\label{fig:sec1.1-1}
\end{wrapfigure}

\begin{example}\label{ex:sec1.1-1}
在重力作用下,物体从离地面高为~$h$~米处自由下落,不计空气阻力时,下落路程~$s$~与时间~$t$~满足关系式:
\begin{equation}\label{eq:sec1.1-1}
  s=\frac12gt^2,\quad 0\leq t\leq\sqrt{\frac{2h}{\smash[b]{g}}}\.
\end{equation}
\end{example}

这里~$g=\SI{9.8}{m/s^2}$~是重力加速度,它是常量。$t$~与~$s$~是物体下落过程中的两个变量,当~$t=0$~时,$s=0$,%
当~$t=\smsqrt{2h/g}$~时,得~$s=h$,表示物体已经到达地面。所以~$t$~的变化范围是从~$0$~到~$\smsqrt{2h/g}$,在这范围内~$g$~的每
一个值,由公式~\ref{eq:sec1.1-1}~即可得到对应~$s$~的值。公式~\ref{eq:sec1.1-1}~给出了变量~$s$~与变量~$t$~之间的依赖关系,即
函数关系。

\begin{example}\label{ex:sec1.1-2}
气象台用自动记录器画出了当地某一天的气温变化图(见\ref{fig:sec1.1-1}),图中纵轴表示气温~$T(\si{\degreeCelsius})$,横轴表示时
间~$t$(小时)。
\end{example}

从~$0$~到~$24$~小时内的任一时刻~$t$~的值,根据这条曲线,就可以找出气温~$T$~的唯一确定的值与之对应。这条曲线给
出了变量~$T$~与变量~$t$~之间的函数关系。

\begin{example}\label{ex:sec1.1-3}
给定正实数~$x$,考虑所有不超过~$x$~的素数个数~$N$。
\end{example}

对于区间~$(0,+\infty)$~内每一~$x$,根据上述对应规则,总有唯一的非负整数~$N$~与其对应,为了表示~$N$~依赖于~$x$,我们记作
~$N=\pi(x)$。显然~$N$~与~$x$~没有确切的公式,也没有~$N$~与~$x$~的图像曲线,但~$N$~与~$x$~的对应关系是客观存在的,所以我
们说这个对应关系,给出了~$N$~与~$x$~之间的函数关系。

由上面例子可以看出,变量间有没有函数关系,在于有没有对应关系,不在于有没有公式或图像。这样,就有函数的如下定义。

\begin{definition}
给定集合~$X$,若存在某种对应规则~$f$,对于~$X$~中每一元素~$x\in X$,都有实数集~$\FR$~中唯一的元素~$y$~与之对应,则称~$f$~
是从~$X$~到~$\FR$~的一个\emph{函数},记作
\[
  \map fX\FR\.
\]
函数~$f$~在~$x$~点的值记作~$y=f(x)$,而~$X$~称为函数~$f$~的\emph{定义域},~$x$~称为\emph{自变量},~$y$~称为\emph{因变量}。
\end{definition}


\subsection{函数的图形}
\begin{exercise}
\item
\end{exercise}
\section{函数的几种特性}
\subsection{函数的奇偶性}
\subsection{函数的单调性}
\subsection{函数的有界性}
\subsection{函数的周期性}
\begin{exercise}
\item
\end{exercise}

\section{复合函数与反函数}
\subsection{复合函数}
\subsection{反函数}
\begin{exercise}
\item
\end{exercise}
\section{基本初等函数}

\begin{exercise*}
\item
\end{exercise*}




\endinput
%%
%% End of file `MAChapter1.tex'.