%# -*- coding:utf-8 -*-
%%%%%%%%%%%%%%%%%%%%%%%%%%%%%%%%%%%%%%%%%%%%%%%%%%%%%%%%%%%%%%%%%%%%%%%%%%%%%%%%%%%%%


\chapter{假弟妹暗续鸾胶\KG 真夫妇明谐花烛}


词曰:

\[
追悔当初辜深愿,经年价,两成幽怨。任越水吴山,似屏如障堪游玩,奈独自慵抬眼。赏烟花,听弦管,徒欢娱,转加肠断。总时转丹青,强拈书信频频看,又曾似亲眼见。
\]

话说陈敬济,到于守备府中,下了马,张胜先进去禀报春梅。春梅分付,教他在外边班直房内,用香汤沐浴了身体,后边使养娘包出一套新衣服靴帽来,与他更换了。然后禀了春梅。那时守备还未退厅,春梅请敬济到后堂,盛妆打扮,出来相见。这敬济进门就望春梅拜了四双八拜,让姐姐受礼。那春梅受了半礼,对面坐下。叙了寒温离别之情,彼此皆眼中垂泪。春梅恐怕守备退厅进来,见无人在根前,使眼色与敬济,悄悄说:“等住回他若问你,只说是姑表兄弟。我大你一岁,二十五岁了,四月廿五日午时生的。”敬济道:“我知道了。”不一时,丫鬟拿上茶来,两人吃了茶,春梅便问:“你一向怎么出了家做了道士?守备不知是我的亲,错打了你,悔的要不的。若不是那时就留下你,争奈有雪娥那贱人在这里,不好安插你的。所以放你去了。落后打发了那贱人,才使张胜到处寻你不着,谁知你在城外做工,流落至此地位。”敬济道:“不瞒姐姐说,一言难尽。自从与你相别,要娶六姐,我父亲死在东京,来迟了,不曾娶成,被武松杀了。闻得你好心,葬埋了他永福寺,我也到那里烧纸来。落后又把俺娘没了,刚打发丧事出去,被人坑陷了资本。来家又是大姐死了,被俺丈母那淫妇告了一状,床帐妆奁,都搬的去了。打了一场官司,将房儿卖了,弄的我一贫如洗。多亏了俺爹朋友王杏庵周济,把我才送到临清晏公庙那里出家。不料又被光棍打了,拴到咱府中。自从咱府中出去,投亲不理,投友不顾,因此在寺内佣工。多亏姐姐挂心,使张管家寻将我来,得见姐姐一面,犹如再世为人了。”说到伤心处,两个都哭了。

正说话中间,只见守备退厅,左右掀开帘子,守备进来。这陈敬济向前,倒身下拜。慌的守备答礼相还,说:“向日不知是贤弟,被下人隐瞒,误有冲撞,贤弟休怪。”敬济道:“不才有话,一向缺礼,有失亲近,望乞恕罪。”又磕下头去。守备一手扯起,让他上坐。敬济乖觉,那里肯,务要拉下椅儿旁边坐了。守备关席,春梅陪他对坐下。须臾,换茶上来。吃毕,守备便问:“贤弟贵庚?一向怎的不见?如何出家?”敬济使告说:“小弟虚度二十四岁。俺姐姐长我一岁,是四月二十五日午时生。向因父母双亡,家业凋丧,妻又没了,出家在晏公庙。不知家姐嫁在府中,有失探望。”守备道:“自从贤弟那日去后,你令姐昼夜忧心,常时啾啾唧唧,不安直到如今。一向使人找寻贤弟不着,不期今日相会,实乃三生有缘。”

看官听说,若论周守备与西门庆相交,也该认得陈敬济,原来守备为人老成正气,旧时虽然来往,并不留心管他家闲事。就是时常宴会,皆同的是荆都监、夏提刑一班官长,并未与敬济见面。况前日又做了道士一番,那里还想的到西门庆家女婿?所以被他二人瞒过,只认是春梅姑表兄弟。一面分付左右放桌儿,安排酒上来。须臾,摆设许多杯盘肴馔,汤饭点心,堆满桌上,银壶玉盏,酒泛金波。守备相陪叙话,吃至晚来,掌上灯烛方罢。守备分付家人周仁,打扫西书院干净,那里床帐都有。春梅拿出两床铺盖衾枕,与他安歇。又拨了一个小厮喜儿答应他。又包出两套绸绢衣服来,与他更换。每日饭食,春梅请进后边吃。正是:一朝时运至,半点不由人。光阴迅速,日月如梭,但见:

\[
行见梅花腊底,忽逢元旦新正。
不觉艳杏盈枝,又早新荷贴水。
\]
敬济在守备府里,住了个月有余。一日是四月二十五日,春梅的生日。吴月娘那边买了礼来,一盘寿桃,一盘寿面,两只汤鹅,四只鲜鸡,两盘果品,一坛南酒。玳安穿青衣拿贴儿送来。守备正在厅上坐的,门上人禀报,抬进礼来。玳安递上贴儿,扒在地下磕头。守备看了礼贴儿,说道:“多承你奶奶费心,又送礼来。”一面分付家人:“收进礼去,讨茶来与大官儿吃。把礼贴教小伴当送与你舅收了。封了一方手帕、三钱银子与大官儿,抬盒人钱一百文,拿回贴儿,多上覆。”说毕,守备穿了衣服,就起身拜人去了。玳安只顾在厅前伺候,讨回贴儿。只见一个年少的,戴着瓦楞帽儿,穿着青纱道袍,凉鞋净袜,从角门里走出来,手中拿着贴儿赏钱,递与小伴当,一直往后边去了。“可霎作怪,模样倒好相陈姐夫一般。他如何却在这里?”只见小伴当递与玳安手帕银钱,打发出门。

到于家中,回月娘话。见回贴上写着“周门庞氏敛衽拜”。月娘便问:“你没见你姐?”玳安道:“姐姐倒没见,倒见姐夫来。”月娘笑道:“怪囚,你家倒有恁大姐夫!守备好大年纪,你也叫他姐夫。”玳安道:“不是守备,是咱家的陈姐夫。我初进去,周爷正在厅上,我递上贴儿与他磕了头,他说:‘又生受你奶奶送重礼来。’分付伴当拿茶与我吃,‘把贴儿拿与你舅收了,讨一方手帕、三钱银子与大官儿,抬盒人是一百文钱。’说毕,周爷穿衣服出来,上马拜人去了。半日,只见他打角门里出来,递与伴当回贴赏赐,他就进后边去了,我就押着盒担出来。不是他却是谁?”月娘道:“怪小囚儿,休胡说白道的。那羔子知道流落在那里讨吃?不是冻死,就是饿死,他平白在那府里做甚么?守备认的他甚么毛片儿,肯招揽下他?”玳安道:“奶奶敢和我两个赌,我看得千真万真,就烧的成灰骨儿我也认的。”月娘道:“他穿着甚么?”玳安道:“他戴着新瓦楞帽儿,金簪子。身穿着青纱道袍,凉鞋净袜。吃的好了。”月娘道:“我不信,不信。”这里说话不题。

却说陈敬济进入后边,春梅还在房中镜台前搽脸,描画双蛾。敬济拿吴月娘礼贴儿与他看。因问:“他家如何送礼来与你?是那里缘故?”这春梅便把清明郊外,永福寺撞遇月娘相见的话,诉说一遍。后来怎生平安儿偷了解当铺头面,吴巡简怎生夹打平安儿,追问月娘奸情之事,薛嫂又怎生说人情,守备替他处断了事,落后他家买礼来相谢。正月里,我往他家与孝哥儿做生日,勾搭连环到如今。他许下我生日买礼来看我一节,说了一遍。敬济听了,把眼瞅了春梅一眼,说:“姐姐,你好没志气。想着这贼淫妇那咱,把咱姐儿们生生的拆散开了,又把六姐命丧了,永世千年,门里门外不相逢才好,反替他去说人情儿。那怕那吴典恩拷打玳安小厮,供出奸情来,随他那淫妇一条绳子拴去,出丑见官,管咱每大腿事?他没和玳安小厮有奸,怎的把丫头小玉配与他?有我早在这里,我断不教你替他说人情。他是你我仇人,又和他上门往来做甚么?六月连阴——想他好情儿!”几句话,说得春梅闭口无言。这春梅道:“过往勾当,也罢了,还是我心好,不念旧仇。”敬济道:“如今人好心不得这报哩。”春梅道:“他既送了礼,莫不白受他的?他还等着我这里人请他去哩。”敬济道:“今后不消理那淫妇了,又请他怎的?”春梅道:“不请他又不好意思的。丢个贴儿与他,来不来随他就是了。他若来时,你在那边书院内,休出来见他,往后咱不招惹他就是了。”敬济恼的一声儿不言语,走到前边,写了贴儿。春梅使家人周义去请吴月娘。月娘打扮出门,教奶子如意儿抱着孝哥儿,坐着一顶小轿,玳安跟随,来到府中。春梅、孙二娘都打扮出来,迎接至后厅相见,叙礼坐下。如意儿抱着孝哥儿,相见磕头毕。敬济躲在那边书院内,不走出来,由着春梅、孙二娘在后厅摆茶安席递酒。叫了两个妓女韩玉钏、郑娇儿弹唱,俱不必细说。

玳安在前边厢房内管待。只见一个小伴当,打后边拿着一盘汤饭点心下饭,往西角门书院中走。玳安便问他拿与谁吃,小伴当说:“是与舅吃的。”玳安道:“代舅姓甚么?”小伴当道:“姓陈。”这玳安贼,悄悄后边跟着他到西书院。小伴当便掀帘子进去,放卓儿吃。这玳安悄悄走出外来,依旧坐在厢房内。直待天晚,家中灯笼来接,吴月娘轿子起身。到家,一五一十告诉月娘说:“果然陈姐夫在他家居住。”自从春梅这边被敬济把拦,两家都不相往还。正是:

\[
谁知竖子多间阻,一念翻成怨恨媒。
\]

敬济在府中与春梅暗地勾搭,人都不知。或守备不在,春梅就和敬济在房中吃饭吃酒,闲时下棋调笑,无所不至。守备在家,便使丫头小厮拿饭往书院与他吃。或白日里,春梅也常往书院内,和他坐半日,方归后边来。彼此情热,俱不必细说。

一日,守备领人马出巡,正值五月端午佳节。春梅在西书院花亭上置了一卓酒席,和孙二娘、陈敬济吃雄黄酒,解粽欢娱。丫鬟侍妾都两边侍奉。春梅令海棠、月桂两个侍妾在席前弹唱。当下直吃到炎光西坠、微雨生凉的时分。春梅拿起大金荷花杯来相劝。酒过数巡,孙二娘不胜酒力,起身先往后边房中看去了。独落下春梅和敬济在花亭上吃酒,猜枚行令,你一杯,我一杯。不一时,丫鬟掌上纱灯来,养娘金匮、玉堂打发金哥儿睡去了。敬济输了,便走入书房内躲酒不出来。这春梅先使海棠来请,见敬济不去,又使月桂来,分付:“他不来,你好歹与我拉将来。拉不将来,回来把你这贱人打十个嘴巴。”这月桂走至西书房中,推开门,见敬济歪在床上,推打鼾睡,不动。月桂说:“奶奶叫我来请你老人家,请不去,要打我哩。”那敬济口里喃喃呐呐说:“打你不干我事。我醉了,吃不的了。”被月桂用手拉将起来,推着他:“我好歹拉你去,拉不将你去,也不算好汉。”推拉的敬济急了,黑影子里佯装着醉,作耍当真,搂了月桂在怀里就亲个嘴。那月桂亦发上头上脑说:“人好意叫你,你就大不正,倒做这个营生。”敬济道:“我的儿,你若肯了,那个好意做大不成?”又按着亲了个嘴,方走到花亭上。月桂道:“奶奶要打我,还是我把舅拉将来了。”春梅令海棠斟上大钟,两个下盘棋,赌酒为乐。当下你一盘,我一盘,熬的丫鬟都打睡去了。春梅又使月桂、海棠后边取茶去,两个在花亭上,解佩露相如之玉,朱唇点汉署之香。正是:得多少花阴曲槛灯斜照,旁有坠钗双凤翘。有诗为证:

\[
花亭欢洽鬓云斜,粉汗凝香沁绛纱。
深院日长人不到,试看黄鸟啄名花。
\]
两个正干得好,忽然丫鬟海棠送茶来:“请奶奶后边去,金哥睡醒了,哭着寻奶奶哩。”春梅陪敬济又吃了两钟酒,用茶嗽了口,然后抽身往后边来。丫鬟收拾了家活,喜儿扶敬济归书房寝歇,不在话下。

一日,朝廷敕旨下来,命守备领本部人马,会同济州府知府张叔夜,征剿梁山泊贼王宋江,早晚起身。守备对春梅说:“你在家看好哥儿,叫媒人替你兄弟寻上一门亲事。我带他个名字在军门,若早侥幸得功,朝廷恩典,升他一官半职,于你面上,也有光辉。”这春梅应诺了。迟了两三日,守备打点行装,整率人马,留下张胜、李安看家,止带家人周仁跟了去。不题。

一日,春梅叫将薛嫂儿来,如此这般和他说:“他爷临去分付,叫你替我兄弟寻门亲事,你须寻个门当户对好女儿,不拘十六七岁的也罢,只要好模样儿,联明伶俐些的。他性儿也有些厥劣。”薛嫂儿道:“我不知道他也怎的?不消你老人家分付。想着大姐那等的还嫌哩。”春梅道:“若是寻的不好,看我打你耳刮子不打?我要赶着他叫小妗子儿哩,休要当耍子儿。”说毕,春梅令丫鬟摆茶与他吃。只见陈敬济进来吃饭。薛嫂向他道了万福,说:“姑夫,你老人家一向不见,在那里来?且喜呀,刚刚奶奶分付,交我替你老人家寻个好娘子,你怎么谢我?”那陈敬济把脸儿迸着不言语。薛嫂道:“老花子怎的不言语?”春梅道:“你休要叫他姑夫,那个已是揭过去的帐了,你只叫他陈舅就是了。”薛嫂道:“真该打,我这片子狗嘴,只要叫错了,往后赶着你只叫舅爷罢。”那敬济忍不住,扑吃的笑了,说道:“这个才可到我心上。”那薛嫂撒风撒痴,赶着打了他一下,说道:“你看老花子说的好话儿,我又不是你影射的,怎么可在你心上?”连春梅也笑了。

不一时,月桂安排茶食与薛嫂吃了,说道:“我替你老人家用心踏着,有人家相应好女子儿,就来说。”春梅道:“财礼羹果,花红酒礼,头面衣服,不少他的,只要好人家好女孩儿,方可进入我门来。”薛嫂道:“我晓得,管情应的你老人家心便了。”良久,敬济吃了饭,往前边去了。薛嫂儿还坐着,问春梅:“他老人家几时来的?”春梅便把出家做道士一节说了:“我寻得他来,做我个亲人儿。”薛嫂道:“好好,你老人家有后眼。”又道:“前日你老人家好日子,说那头他大娘来做生日来?”春梅道:“他先送礼来,我才使人请他,坐了一日去了。”薛嫂道:“我那日在一个人家铺床,整乱了一日。心内要来,急的我要不的。”又问:“他陈舅,也见他那头大娘来?”春梅道:“他肯下气见他?为请他,好不和我乱成一块。嗔我替他家说人情,说我没志气。那怕吴典恩打着小厮,攀扯他出官才好,管你腿事?你替他寻分上,想着他昔日好情儿?”薛嫂道:“他老人家也说的是,及到其间,也不计旧仇罢了。”春梅道:“咱既受了他礼,不请他来坐坐儿,又使不的。宁可教他不仁,休要咱不义。”薛嫂道:“怪不的你老人家有恁大福,休的心忒好了!”当下薛嫂儿说了半日话,提着花箱儿,拜辞出门。

过了两日,先来说:“城里朱千户家小姐,今年十五岁,也好陪嫁,只是没了娘的儿了。”春梅嫌小不要。又说应伯爵第二个女儿,年二十二岁。春梅又嫌应伯爵死了,在大爷手内聘嫁,没甚陪送,也不成。都回出婚帖儿来。又迟了几日,薛嫂儿送花儿来,袖中取出个婚贴儿,大红段子上写着:“开段铺葛员外家大女儿,年二址岁,属鸡的,十一月十五日子时生,小字翠屏。”“生的上画儿般模样儿,五短身材,瓜子面皮,温柔典雅,联明伶俐,针指女工,自不必说。父母俱在,有万贯钱财。在大街上开段子铺,走苏杭、南京,无比好人家。陪嫁都是南京床帐箱笼。”春梅道:“既是好,成了这家的罢。”就交薛嫂儿先通信去。那薛嫂儿连忙说去了。正是:

\[
欲向绣房求艳质,须凭红叶是良媒。
\]
有诗为证:

\[
天仙机上系香罗,千里姻缘竟足多。
天上牛郎配织女,人间才子伴娇娥。
\]

这里薛嫂通了信来,葛员外家知是守备府里,情愿做亲,又使一个张媒人同说媒。春梅这里备了两抬茶叶、粮饼、羹果,教孙二娘坐轿子,往葛员外家插定女儿。回来对春梅说:“果然好个女子,生的一表人才,如花似朵,人家又相当。”春梅这里择定吉日,纳采行礼。十六盘羹果茶饼,两盘头面,二盘珠翠,四抬酒,两牵羊,一顶鬒髻,全副金银头面簪环之类。两件罗段袍儿,四季衣服。其余绵花布绢,二十两礼银,不必细说。阴阳生择在六月初八日,准娶过门。春梅先问薛嫂儿:“他家那里有陪床使女没有?”薛嫂儿道:“床帐妆奁都有,只没有使女陪床。”春梅道:“咱这里买一个十三四岁丫头子,与他房里使唤,掇桶子倒水方便些。”薛嫂道:“有,我明日带一个来。”

到次日,果然领了一个丫头,说:“是商人黄四家儿子房里使的丫头,今年才十三岁。黄四因用下官钱粮,和李三还有咱家出去的保官儿,都为钱粮捉拿在监里追赃,监了一年多,家产尽绝,房儿也卖了。李三先死,拿儿子李活监着。咱家保官儿那儿僧宝儿,如今流落在外,与人家跟马哩。”春梅道:“是来保?”薛嫂道:“他如今不叫来保,改了名字叫汤保了。”春梅道:“这丫头是黄四家丫头,要多少银子?”薛嫂道:“只要四两半银子。紧等着要交赃去。”春梅道:“甚么四两半,与他三两五钱银子留下罢。”一面就交了三两五钱雪花官银与他,写了文书。改了名字,唤做金钱儿。

话休饶舌,又早到六月初八。春梅打扮珠翠凤冠,穿通袖大红袍儿,束金镶碧玉带。坐四人大轿,鼓乐灯笼,娶葛家女子,奠雁过门。陈敬济骑大白马,拣银鞍辔,青衣军牢喝道。头戴儒巾,穿着青段圆领,脚下粉底皂靴,头上簪着两支金花。正是:久旱逢甘雨,他乡遇故知。洞房花烛夜,金榜挂名时。一番拆洗一番新。到守备府中,新人轿子落下。头盖大红销金盖袱,添妆含饭,抱着宝瓶进入大门。阴阳生引入画堂,先参拜了堂,然后归到洞房。春梅安他两口儿坐帐,然后出来。阴阳生撒帐毕,打发喜钱出门,鼓手都散了。敬济与这葛翠屏小姐坐了回帐,骑马打灯笼,往岳丈家谢亲。吃的大醉而归。晚夕女貌郎才,未免燕尔新婚,交媾云雨。正是:得多少——

\[
春点杏桃红绽蕊,风欺杨柳绿翻腰。
\]

当夜敬济与这葛翠屏小姐倒且是合得着。两个被底鸳鸯,帐中鸾凤,如鱼似水,合卺欢娱。三日完饭,春梅在府厅后堂张筵挂采,鼓乐笙歌,请亲眷吃会亲酒,俱不必细说。每日春梅吃饭,必请他两口儿同在房中一处吃。彼此以姑妗称之,同起同坐。丫头养娘、家人媳妇,谁敢道个不字?原来春梅收拾西厢房三间,与他做房,里面铺着床帐,糊的雪洞般齐整,垂着帘帏。外边西书院,是他书房。里面亦有床榻、几席、古书并守备往来书柬拜贴,并各处递来手本揭贴,都打他手里过。春梅不时出来书院中,和他闲坐说话,两个暗地交情。正是:

\[
朝陪金谷宴,暮伴绮楼娃。
休道欢娱处,流光逐落霞。
\]
