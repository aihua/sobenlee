%# -*- coding:utf-8 -*-
%%%%%%%%%%%%%%%%%%%%%%%%%%%%%%%%%%%%%%%%%%%%%%%%%%%%%%%%%%%%%%%%%%%%%%%%%%%%%%%%%%%%%

\begin{pre-post-text}{重印版后记}
这本书从酝酿到写作经历了一个漫长的过程,八十年代,我产生了写这本书的念头,但促使我对延安整风这一历史事件萌发兴趣则是在更久远的年代。

我是在~1961~年的南京读小学的,那是一个政治意识畸形发展的年代。从~1963~年初开始,我对母亲订阅的《参考消息》发生了兴趣,经常躲着她偷偷阅读。我也从那时起,养成了每天读江苏省党报《新华日报》的习惯。可是我对那时的社会状况并不清楚——应该说,除了雷锋、革命先烈、越南、红军长征的故事,那时我的头脑中并没有任何其它东西,但是到了~1963~年下半年后,情况发生了变化,我愈来愈注意《参考消息》和报纸上刊载的有关中苏两党论战的报道。1964~年春夏之间,我从《人民日报》上看到苏共中央书记苏斯洛夫在苏共二月全会上作的“反华报告”,第一次看到对斯大林、莫洛托夫制造三十年代大恐怖罪行的揭露,以及对毛泽东、对大跃进、人民公社的批评——这对于我是一个极大的震动(这份报告给我留下极为深刻的印象,以后我长期保留这份《人民日报》)。我开始思考苏斯洛夫报告中所论及的一些词汇:毛泽东是“左倾冒险主义”、“新托洛茨基主义”、“唯意志论”等等(七十年代,我从内部读物才知道,苏斯洛夫是一个顽固的教条主义者。近年出版的俄罗斯资料透露,1964~年苏共党内的革新势力利用与中共的论战,削弱了斯大林主义者在苏联的阵地,一度遏制了保守势力复辟的势头,正是在这样的背景下,保守的苏斯洛夫才在苏共中央二月全会上作了这个报告)。对于这些话,当时我似懂非懂。我联想自己的日常生活,几年前那些饥饿的日子,我随母亲去南京郊外的农场去探望因“右派”问题而被下放劳动的父亲,1963~年夏,我已被南京市外语学校录取,却因政审不通过而被拒之门外。不久甚至连小学也讲起“阶级路线”,我因出身问题越来越感到压力。在这种情况下,我迎来了~1966~年,也就是在这个时候,在学校的号召下,我通读了《毛选》一至四卷,我多次阅读收人《毛选》中的《关于若于历史问题的决议》以及毛的《改造我们的学习》、《反对党八股》,于是我知道了“整风运动”这个词。

紧接着文革爆发,我从每天读的《新华日报》上发现,1966~年~5~月初北京召开的欢迎阿尔巴尼亚党政代表团的群众大会上不见了彭真的名字,接下来我就读小学的一些干部子弟(我的小学邻近南京军区后勤部家属大院和《新华日报》社家属区),手拎红白相间的体操棒在操场上殴打一位“成份不好”的三十多岁姓余的美术教师,校长兼支部书记则装着什么也没看见。

南京~1966~年~8~月下旬的红色恐怖给我留下了终身难忘的印象。我的家庭受到冲击,有一天,我无意中听到父母的谈话,父亲说,这一次可能躲不过去了,再不跑,可能会被活活打死。父亲终于离家出逃。躲在山东农村老家那些纯朴的乡亲中避难,不久,我家附近到处贴满了父亲单位捉拿他的“通缉令”。

在文革的血雨腥风中,我看到了多少景象!曾几何时,那些在文革初期指挥揪斗“死老虎”的当权派自己很快也被拉下了马,“周扬”、“彭罗陆杨”、“刘邓陶”像走马灯似地被“扫入历史的垃圾堆”,真是“一顶顶皇冠落地”!从那时起,我就无师自通地学会了报纸上的“排名学”。1967~年初,在南京大学的操场上,我亲眼看见江苏省委第一书记江渭清被批斗,就在半年前,我们小学的校长还是满口“江政委”唤个不停。不久我又去了省委办公大褛,那里正举办所谓“修正主义老爷腐朽生活”的展览,那宽大的带卫生间和休息室的书记办公室,那嵌在舞厅天花板壁槽内的柔和灯光,无一不使我头脑翻江倒海。

我的家庭背景使我不能参加这场革命,我在家庭中受的教育以及我从各种书籍中所获得的精神营养也使我不会去欣赏那些在革命名义下所干的种种凌虐人的暴行。在文革前,我家有一个区文化馆图书室的借书证,因此我读过不少中外文学、历史读物。至今我还记得,在恐怖的~1966~年~8~月,我如何从母亲的手中夺下她正准备烧掉的那套杨绛翻译、勒萨日著的《吉尔·布拉斯》等十几本书籍。在焚书烈火中被抢救下来的《吉尔·布拉斯》、范文澜的《中国通史简编》、普希金诗选、《唐诗三百首》等给了我许多温暖,让我在黑暗的隧道中看到远处一簇光。

在文化大革命的狂风暴雨中,希望之光是黯淡和飘忽的。1967~年初,我在家附近的长江路南北货商店墙上看到一张写有“特大喜讯”的大字报,上面赫然写着叶剑英元帅最近的一次讲话,他说,我们伟大领袖身体非常健康,医生说,毛主席可以活到一百五十岁。

看到这张大字报,我头脑轰地一响,虽然有所怀疑,但当时的直觉是,这一下,我这一辈子都注定要生活在毛泽东时代了。我马上去找我的好友贺军——他目前住在美国的波士顿——告诉他这个消息,我们一致认为,毛主席不可能活到一百五十岁,因为这违反科学常识。

从这时起,我在心里消悄地对毛泽东有了疑问,我知道在中国,一切都凭他一个人说了算,其他人,即使刘少奇,虽然《历史决议》对他评价极高,虽然在文革前到处都能看到毛、刘并列的领袖标准像,虽然刘少奇夫妇访问东南亚是何等地热烈和风光,但是如果毛泽东不喜欢,刘少奇马上就被打倒。我又看到自己身边发生的一些事,离我家不远一个小巷的破矮平房里,住着与我同校但不同班的一对姐弟和他们的父母,他们的父亲是“阶级敌人”,他们的妈妈是一位普通的劳动妇女。因为不能忍受歧视和侮辱,这位母亲竟失去控制,将毛主席的画像撕碎并呼喊“反动口号”,结果在~1970~年南京的“一打三反”运动中被枪毙。召开公判大会那一天,我的中学将所有学生拉到路边,观看行刑车队通过,美其名曰“接受教育”,这姐弟两人也被安排在人群中,亲眼目睹他们的母亲被五花大绑押赴刑场。车队通过后,学校革委会副主任要求全校各班立即分组讨论,于是所有同学都表态拥护“镇压反革命”——所有这一切都让我对毛产生了看法。我知道这些看法绝不能和任何人讲,甚至不能和自己的父母讲,只能深埋在心中。

在那令人窒息的岁月里,没有希望,没有绿色,除了从小在一起长大的贺军,差不多也没有任何可以与之交心的朋友(即使我们之间的谈话也小心翼翼,绝不敢议论毛泽东),但是,我的心中仍存有一线微弱的光。我的家附近有南京某中学留守处,这个中学已被勒令搬至农村,所有被封存的图书都堆放在留守处的大仓库里,由一姜姓老先生看管(老人是山东人,年轻时被国民党拉去当兵,被解放军俘虏后成为“解放”战士)至令我仍感激这位老先生,。是他允许我每周进一次仓库借一旅行袋的书,下周依时交换。正是在那里,我翻检到~1958~年《文艺报》的《再批判》专辑,因而我第一次读了王实味的《野百合花》和丁玲的《三八节有感》。在那几年,我从这个仓库借去大量的中外文学和历史书籍,至今还记得,孟德斯鸠的《一个波斯人的信札》、罗曼·罗兰的《约翰·克利斯朵夫》、惠特曼的《草叶集》、叶圣陶的《倪焕之》、老舍的《骆驼祥子》,就是在那个时候读的。1971~年后南京图书馆局部恢复开放,我又在每个休息日去那里读《鲁迅全集》,将包括鲁迅译著在内的旧版《鲁迅全集》全部通读了一遍。正是这些作品支撑起我的人文主义的信念。

七十年代中期,国内的政治局势更加险恶,我的一位熟人的弟弟,因愤恨江青的专横,在~1975~年从其家中的阁楼上跳下自杀身亡。我家也每天受到居民小组老太太的监视,只要家里来一外人,她就站在门口探头探脑。1976~年夏天的一个晚上,我与好友贺军坐在长江路人行道的路边,我背诵了鲁迅的话:“地下火在运行,岩浆在奔突……”(1995~年~8~月底,我与贺军在纽约第五大道的街心花园坐了半天,我们共同回忆起往昔岁月,我们都谈到~1976~年夏在长江路边的那次谈话)。

在文革期间,我读了许多毛泽东的内部讲话和有关“两条路线斗争”的资料,这些资料真真假假,其中不少充斥着大量的歪曲和谎言,然而它们还是激起了我强烈的兴趣。结合文革中所发生、暴露出的一切,以及自己的生活感受,我愈来愈有一种想探究中共革命历史的愿望,在这个过程中,我注意到了延安整风运动——这虽然是距那时以前几十年的往事,但我还是隐约感到,眼下一切似乎都与它有联系。在大字报和各种文革材料中,我难道不是经常读到毛和其他“中央首长”的讲话吗:“×××什么最坏,在宁都会议上,他想枪毙我”,“刘少奇在抗战期间勾结王明反对毛主席的独立自主方针”,什么“×××在延安审干中查出是自首分子,因此对他控制使用”,还有“王明化名马马维奇在苏联恶毒攻击伟大领袖毛主席”等等。

在那些年里,我虽然是“生在新社会,长在红旗下”,却不知填了多少表格,从小学、中学到工作单位,每一次都要在“政治面貌厂“社会关系”栏内填写老一套的内容。看着周围的人,大家也一样要填表。我工作单位的人事干事是从老解放区来的,她说,这是党的审干的传统,是从延安整风开始推广的,那么延安整风运动又是怎么一回事呢?带着这些疑问,1978~年秋,我以历史专业作为自己的第一选择:考入了南京大学历史系。

1979~年后的中国大学教育开始发生一系列重大变化,我经历了那几年由思想解放运动而带来的震撼并引发了更多的思考。在课堂上,我再次听老师讲延安整风运动,我也陆续看到一些谈论“抢救”运动的材料,然而所有这些都在维持一个基本解释:延安整风运动是一场伟大的马克思主义的教育运动。1979~年我还读到周扬那篇有名的文章《三次伟大的思想解放运动》,周扬将延安整风与五四运动、七十年代末的思想解放运动相提并论,谓之为“思想解放运动”。在大学读书的那几年,我知道,虽然毛泽东晚年的错误已被批评,但毛的极左的一套仍很深蒂固,它已渗透到当代人思想意识的深处,成为某种习惯性思维,表现在中国现代史、中共党史研究领域,就是官学盛行,为圣人避讳或研究为某种权威论述作注脚几乎成为一种流行的风尚。当然我十分理解前辈学者的矛盾和苦衷,他们或被过去的极左搞怕了,或是因为年轻时受到《联共党史》、《中国共产党的三十年》的思想训练太深,以至根本无法跳出官学的窠臼。

然而,我难以忘怀过去岁月留下的精神记忆,刘知几云,治史要具史才、史学、史识,其最重要之处就是秉笔直书,“在齐太史简,在晋董狐笔”。至令我还清楚记得~1979~年在课堂上听老师讲授司马迁《报任安书》时内心所引起的激动,我也时时忆及范文澜先生对史学后进的教诲:板梁甘坐十年冷,文章不写一字空。所有这些都促使我跳出僵硬教条的束缚,努力发挥出自己的主体意识,让思想真正自由起来。从那时起,我萌生一个愿望,将来要写一本真实反映延安整风的史书,为此我开始搜集资料。

由于延安整风在主流话语中是一个特殊的符号,有关史料的开放一直非常有限。这给研究者带来极大的困难。但在八十年代以后。官方也陆续披露了某些与延安整风运动相关的历史资料,除了少量档案、文件集外,也出版了不少回忆资料,这给研究者既带来了便利,同时也带来了新的问题,这就是如何分析、辨别、解释这些材料。应该说,我在中国大陆长期的生活体验以及我对有关史料的广泛涉猎,加强了我阅读资料的敏感性,我逐渐能够判断在那些话语后面所隐蔽的东西。

经过对多年搜集、积累资料的反复研究和体会,我头脑中的延安整风的轮廓逐渐清晰起来,我开始发现散乱在各种零碎资料之间的有机联系。1991~年~8~月中旬我开始动笔,到~1992~年底,我已完成初稿的三分之二。

从~1993~年始,我的写作速度慢了下来,我感到自己需要对所论述的问题作进一步的思考,同时需要更广泛地搜集、阅读各种资料。

1995~年夏至~1996~年秋,我有机会去设于美国首都华盛顿的约翰斯·霍普金斯大学高级国际问题研究院作访问学者。我在美国的研究题目与延安整风无关,但我仍利用在华盛顿的便利,在国会图书馆工作了一个月。然而很遗憾,国会图书馆中文部虽然收藏十分丰富,但是几乎找不到我所需要的有关延安整风的材料。1996~年~10~月我返国后,又重新开始写作,到了~1998~年夏,全书已经完成。我又用半年时间对书稿作了三次修改补充,1999~年初交稿后,在编辑校对阶段,我接触到若干新资料,对书中的个别内容再次做了充实,于~1999~年春夏之交,全书最后定稿。

我写这本书在思想上一直以求真求实为依归,在写作过程中,始终遵循据事言理的治学方法。我以为,重要的是,首先应将延安整风的来龙去脉叙述清楚,这个问题之所以重要,乃是因为数十年意识形态的解释学早将当年那场事件搞得云环雾绕,面目不清。为此我作了大量的工作,对各种重要的和非重要的资料进行点滴归拢,爬梳鉴别,再对之反复研究体会,使之融汇贯通。这方面的工作用去我最多的时间和精力。

我不反对对延安整风这一重大现象进行严谨的理论分析,且认为,这个工作极为重要,但是我又担心过度解释会妨碍读者自己的判断。陈寅恪先生言,“大处着眼,小处着手”,“滴水观沧海”,因此在本书中,我从实证研究的角度,以分析性论述的方式展开,这也与我个人比较重视历史的个案研究有关。

在写作此书的七年里,我一直怀有深深的遗憾,这就是,我无法得到更重要的原始资料。众所周知,有关延安整风期间的中共中央政治局、书记处、中社部、中组部的档案文献,除少量披露外,绝大部分迄今仍未公开。1992~年,我看到一位负责人在中央档案馆的讲话,他说,鉴于苏东巨变深刻的历史教训,应该加强对档案工作重要性的认识。他指出,党的档案资料的保管,关系到中国社会主义的前途和命运。我可以理解这位负责人的观点,但是站在学术研究的角度,却为不能阅读和利用这些珍贵史料而感到遗憾。

由于这是一本私人写作,十多年来我从自己不多的工资里挤出钱购买了大量的书籍资料,我从没有以此选题申请国家、省级或大学的任何社科研究项目的资助,所以我的另一个遗憾是,我无法对一些当年参加过延安整风运动的人士进行口述采访,如果我做了这样的工作,一定会对本书的内容有所充实。我还有一个遗憾是我没有机会去莫斯科搜寻有关资料。九十年代后,莫斯科开禁历史档案,涉及四十年代苏共与中共交往的文献记录也已开放。中国历史学会的沈志华博士近年来为搜集这些史料作了大量工作,他并已将其中某些材料转送北京研究者(沈博士告诉我,苏共与中共在延安整风期间交往的史料很少),因沈博士去美国,一时联系不上,这也使我深感遗憾。

伏案几载,每天神游于当年的历史景像之中,自然会对延安整风运动及其相关的史事与人物产生种种体会,这方面的体会与感受的绝大部分已化为书中的叙述,但是还有几点需在此予以说明:

一、予生也晚,未能躬逢中共草创革命的年代。吾细读历史,站在二十世纪全局观二十年代后中国共产革命之风起云涌,心中自对中共革命抱持一种深切的同情和理解。吾将其看成是二十世纪中国民族解放和社会改造运动的产物,认为在历史上自有其重大正面价值和意义。

二、从中共革命夺权、推翻国民党统治的角度观之,延安整风运动对于中共革命成功助力巨大。但是,延安整风运动中的某些概念、范式以后又对中国的发展和进步产生若于消极作用,极左思想、权谋政治汇溪成流,终至酿成建国后思想领域一系列过左的政治运动直至文革惨祸,真所谓“成也萧何,败也箫何”!所幸中共十一届三中全会后,国家已逐步走出过去那种怀疑一切、无情斗争的极左道路,但旧习惯思维的清理仍需长期努力。吾期盼旧时极左的“以我划线”、权谋政治永不再来,国家从此能步入民主、法治的轨道,如此,则国家幸甚,民族幸甚!

三、本书涵盖面颇宽,涉及中国现代史上许多著名人物,对于本书论及的所有人物,我只将其看成历史人物,不存任何既定的好恶偏见,主观上力求客观公允,“不虚美,不隐恶”。当然任何研究都不可能完全排除作者的价值关怀,陈衡哲先生曾说过,“若仅缕述某人某国于某年征服某地……那有什么意思”,说的也是研究者的价值关怀问题,只是这种价值关怀不应妨害到叙述的中立和客观。如果说本书的叙述中有什么价值倾向的话,那就是我至今还深以为然的五四的新价值:民主、自由、独立、社会正义和人道主义。

在写作此书的几年里,我得到了一些朋友宝贵的支持和鼓励,在本书即将出版之际,我谨向他们表示真挚的感谢。

上海师范大学的许纪霖教授多年来一直关心我的研究的进展,他还为本书的出版提出许多好的建议。在与许教授的交往中,他的深厚的学养和对二十世纪中国历史的卓越见解总是使我深获教益。

我衷心感谢香港中文大学中国文化研究所的金观涛教授和刘青峰教授。他们对本书的出版提供了热情的帮助,在本书定稿过程中,他们提出一些富有启迪性的建议和意见,对于本书臻于学术规范化的要求,有重要的作用。

我也向我的同事和好友,南京大学历史系颜世安教授和现旅居美国的贺军先生表达我的感激,他们的友谊和支持,对于我一直是一个激励。我曾与美国哈佛大学东亚系孔斐力教授(Philip Kuhn)和美国约翰斯·霍普金斯大学高级国际问题研究院“华盛顿——南京办公室”主任甘安哲博士(Anthony Kane)有过多次关于三十至四十年代中共党史问题的愉快讨论,他们的支持和鼓励对于我的写作是一种推动。

在写作此书的几年里,我始终得到我过去的学生甘思德(Scott Kennedy)和唐山(Jcff Zuckerberg)、林志涛(Felex Lin)的关心和帮助,我的研究生郭洵澈对我帮助尤大,他不仅帮我用电脑输入文字,还与我分享了讨论的乐趣,在此我向他们表示深切的感谢。

我也向本书所引用文字的作者、编者表示我的谢意,我虽然在引述文字时都做有详细的注释,但没有他们提供的资料基础,我要完成这本书也是不可能的。我要向本书的责任编辑郑会欣博士表达我深深的谢意,郑博士自己有大量的研究任务,但他还是拨冗为本书做了许多琐细的工作,他的慷慨支持对本书的出版有重要的帮助。

1998~年夏秋之际,我有机会前往香港中文大学中国文化研究所作访问研究,在“大学服务中心”得到熊景明女士的热情接待和帮助,在这个收藏丰富的史料中心,我为本书补充了若干新的资料,在此向熊景明女士和“大学服务中心”表示深切的谢意。

南京大学历史系资料室的老师们多年来在图书资料方面给了我许多帮助,对他们的友好、善意和敬业精神,我深表感激。最后,我要深深地感谢我的妻子刘韶洪和儿子高欣,我的妻子在每天工作之余,承担了大量的家务,使我可以专心致志进行研究,她还为书稿作了一部分的电脑输入工作。为了写作这本书,许多年我不能和妻子、孩子一同出外游玩,也不能与孩子经常讨论他的功课,没有他们的支持、帮助和理解,我要完成此书是完全不可能的。

\Sign{1999~年~6~月于南京大学}
\end{pre-post-text}

\begin{pre-post-text}{初版后记}
这本书从酝酿到写作经历了一个漫长的过程,80~年代初,我产生了写这本书的念头,但促使我对延安整风这一事件萌发兴趣则是在更久远的年代。

我第一次接触到“延安整风”这个词是在文革爆发前夕的~1966~年春。记得有一位前辈学者曾说过,旧中国黑暗的现实,使中国的青少年比欧美国家的青少年,在政治上更趋于早熟。我想说的是不仅在旧中国情况如此,在毛泽东时代的新中国,情况亦是这样。新中国层出不穷的政治斗争及其对社会的广泛影响,使我不幸地过早地关注起自己不应该去关心的事情。

我读书启蒙的年代是六十年代初的南京,那是一个政治意识畸形发展的年代。从~1963~年初开始,我对母亲订阅的《参考消息》发生了兴趣,经常躲着她偷偷阅读。我也从那时起,养成了每天读江苏省党报《新华日报》的习惯。可是我对那时的社会状况并不清楚——应该说,除了雷锋、革命先烈、越南、红军长征的故事,那时我的头脑中并没有任何其他东西,但是到了~1963~年下半年后,情况发生了变化,我愈来愈注意《参考消息》和报纸上刊载的有关中苏两党论战的报道。1964~年春夏之间,我从《人民日报》上看到苏共中央书记苏斯洛夫在苏共二月全会上作的“反华报告”,第一次看到对毛泽东、对大跃进、人民公社的批评——这对于我是一个极大的震动(这份报告给我留下极为深刻的印象,以后我长期保留这份《人民日报》)。我开始思考苏斯洛夫报告中所论及的一些词汇:毛泽东是“左倾冒险主义”、“半托洛茨基主义”、“唯意志论”等等(七十年代,我从内部读物才知道,苏斯洛夫是一个顽固的教条主义者。近年出版的俄罗斯资料透露,1964~年苏共党内的革新势力利用与中共的论战,削弱了斯大林主义者在苏联的阵地,一度遏止了保守势力复辟的势头,正是在这样的背景下,保守的苏斯洛夫才在苏共中央二月全会上作了这个报告)。对于这些话,当时我似懂非懂。我联想自己的日常生活,几年前那些饥饿的日子,我随母亲去南京郊外的劳改农场去探望因“右派”问题而被下放劳动的父亲,1963~年夏,我已被南京市外语学校录取,却因政审不通过而被拒之门外——我对当时的政策居然产生了一些疑惑。不久甚至连小学也讲起“阶级路线”,我因出身问题越来越感到压力。在这种情况下,我迎来了文化大革命,也就是在这个时候,在学校的号召下,我通读了《毛选》一至四卷,我多次阅读了收入《毛选》中的《关于若干历史问题的决议》以及毛的《改造我们的学习》、《反对党八股》,于是我知道了“整风运动”这个词。

紧接着文革爆发,我从每天读的《新华日报》上发现,1966~年~5~月初北京召开的欢迎阿尔巴尼亚党政代表团的群众大会上不见了彭真的名字,接下来我就读小学的一些干部子弟(我的小学邻近南京军区后勤部家属大院和《新华日报》社家属区),手拎红白相间的的体操棒在操场上殴打一位“成份不好”的~30~多岁的余姓美术教师,校长兼支部书记则装着什么也没看见。

南京~1966~年~8~月下旬的红色恐怖给我留下了终身难忘的印象。我的家庭受到冲击,有一天,我无意中父母的谈话,父亲说,这一次可能躲不过去了,再不跑,可能会被活活打死。父亲终于离家出逃,躲在山东农村老家那些纯朴的乡亲中避难,不久,我家附近到处贴满了父亲单位捉拿他的“通缉令”。

在文革的血雨腥风中,我看到了多少景象!曾几何时,那些在文革初期指挥揪斗“死老虎”的当权派自己很快也被拉下了马,“周扬四条汉子”、“彭罗陆杨”、“刘邓陶”像走马灯似地被“扫入历史的垃圾堆”,真是“一顶顶皇冠落地”!从那时起,我就无师自通地学会了报纸上的“排名学”。1967~年初,在南京大学的操场上,我亲眼看见江苏省委第一书记江渭清被批斗,就在半年前,我的小学的校长还是满口“江政委”唤个不停。不久我又去了省委办公大楼,那里正举办所谓“修正主义老爷腐朽生活”的展览,那宽大的带卫生间和休息室的书记办公室,那嵌在舞厅天花板壁槽内的柔和灯光,以及用从缅甸进口的柚木制成的地板,无一不使我头脑翻江倒海。

我的家庭背景使我不能参加这场革命,我在家庭中受的教育以及我从各种书籍中所获得的精神营养也使我不会去欣赏那些在革命名义下所干的种种凌虐人的暴行。在文革前,我家有一个区文化馆图书室的借书证,因此我读过不少中外文学,历史读物。至今我还记得,在恐怖的~1966~年~8~月,我如何从母亲的手中夺下她正准备烧掉的那套杨绛翻译的勒萨日著的《吉尔·布拉斯》等十几本书籍。在焚书烈火中被抢救下来的《吉尔·布拉斯》、范文澜的《中国通史简编》、普希金诗选、《唐诗三百首》等给了我多少温暖,让我在黑暗的隧道中看到远外一簇光!

在文化大革命的狂风暴雨中,希望之光是黯淡和飘忽的。1967~年初,我在家附近的长江路南北货商店墙上看到一张“特大喜讯”的大字报,上面赫然写着叶剑英元帅最近的一次讲话,他说,我们伟大领袖身体非常健康,医生说,毛主席可以活到~150~岁。看到这张大字报,我头脑轰地一响,虽然有所怀疑,但当时的直觉是,这一下,我这一辈子都注定要生活在毛泽东的时代了。我马上去找我的好友贺军——他目前住在美国的波士顿——告诉他这个消息,我们一致认为,毛主席不可能活到~150~岁,因为这违反科学常识。

从这时起,我在心里悄悄地对毛泽东有了疑问。我知道在中国,一切都凭他一个人说了算,其他人,即使刘少奇,虽然《历史决议》对他评价极高,虽然在文革前到处都能看到毛、刘并列的领袖标准像,虽然刘少奇夫妇访问东南亚是何等的热烈和风光,但是如果毛泽东不喜欢,刘少奇马上就被打倒。我又看到自己身边发生的一些事,离我家不远的一个小巷的破矮平房里,住着与我同校但不同班的一对姐弟和他们的父母,他们的父亲是“历史反革命分子”,他们的妈妈是一个普通的劳动妇女,在街道煤球厂砸煤基(蜂窝煤)。因为不能忍受歧视和侮辱,这位母亲竟失去控制,将毛主席的画像撕碎并呼喊“反动口号”,结果在~1970~年南京的“一打三反”运动中被枪毙。召开公判大会那一天,我的中学将所有学生拉到路边,观看行刑车队通过,美其名曰“接受教育”,这姐弟两人也被安排在人群中,亲眼目睹他们的母亲被五花大绑押赴刑场。车队通过后,学校革委会副主任要求全校各班立即分组讨论,于是所有同学都表态拥护“镇压反革命”-~所有这一切都让我对毛产生了看法。我知道这些看法绝不能和任何人讲,甚至不能和自己的父母讲,只能深埋在心中。

在那令人窒息的岁月里,没有希望,没有绿色,除了从小在一起长大的贺军,差不多也没有任何可以与之交心的朋友(即使我们之间的谈话也小心翼翼,绝不敢议论毛泽东),但是,我的心中仍存有一线微弱的光。我的家附近有南京某中学留守处,这个中学已被勒令搬至农村,所有被封存的图书都堆放在留守处的大仓库里,由一姜姓老先生看管(老人是山东人,年青时被国民党拉去当兵,被解放军俘虏后成为“解放”战士)。至今我仍感激这位老先生,是他允许我每周进一次仓库借一旅行袋的书,下周依时交换。正是在那里,我翻检到~1958~年的《文艺报》的《再批判》专辑,因而我第一次读了王实味的《野百合花》和丁玲的《三八节有感》。在那几年,我从这个仓库借去大量的中外文学和历史书籍,至今还记得,孟德斯鸠的《一个波斯人的信札》、罗曼·罗兰的《约翰·克利斯朵夫》、惠特曼的《草叶集》、叶圣陶的《倪焕之》、老舍的《骆驼祥子》,就是在那个时候读的。1971~年后南京图书馆局部恢复开放,我又在每个休息日去那里读《鲁迅全集》,将包括鲁迅译著在内的旧版《鲁迅全集》全部通读了一遍。正是这些作品支撑起我的人文主义的信念。

70~年代中期,国内的政治局势更加险恶,我的一位熟人的弟弟,因愤恨江青的专横,在~1975~年从其位于南京市卫巷家中的阁楼上跳下自杀身亡。我的家也每天受到居民小组老太太的监视,只要家里来一外人,她就站在门口探头探脑,东张西望。1976~年夏天的一个晚上,我与好友贺军坐在长江路人行道的路边,我背诵了鲁迅的话:“地下火在运行,岩浆在奔突……”(1995~年~8~月底,我与贺军在纽约第五大道的街心花园坐了半天,我们共同回忆起往昔岁月,我们都谈到~1976~年夏在长江路边的那次谈话)。

在文革期间,我读了许多毛的内部讲话和有关“两条路线斗争”的资料,这些资料真真假假,其中不少充斥着大量的歪曲和谎言,然而它们还是激起了我强烈的兴趣。结合文革中所发生、暴露出的一切,以及自己的生活感受,我愈来愈有一种想探究中共革命历史的愿望,在这个过程中,我注意到了延安整风运动——这虽然是距那时以前几十年的往事,但我还是隐约感到,眼下一切似乎都与它有联系。在大字报和各种文革材料中,我难道不是经常读到毛和其他“中央首长”的讲话吗:什么“×××最坏,在宁都会议上,他想枪毙我”,“刘少奇在抗战期间勾结王明反对毛主席的独立自主方针”,什么“×××在延安审干中查是自首分子,因此对他控制使用”,还有“王明化名马马维奇在苏联恶毒攻击伟大领袖毛主席”等等。

在那些年里,我虽然是“生在新社会,长在红旗下”,却不知填了多少表格,从小学、中学到工作单位,每一次都要在“政治面貌”、“社会关系”栏内填写老一套的内容。看看周围的人,大家也一样要填表。我工作单位的人事干事是从老解放区来的,她说,这是党的审干的传统,是从延安整风开始执行和推广的,那么延安整风运动又是怎么一回事呢。带着这些疑问,1978~年秋,我以历史系作为自己的第一选择,考入了南京大学历史系。

1979~年后的中国大学教育开始发生一系列重大变化,我经历了那几年由思想解放运动而带来的震撼并引发了更多的思考。在课堂上,我再次听老师讲延安整风运动,我也陆续看到一些谈论“抢救”运动的材料,然而所有这些都在维持一个基本解释:延安整风运动是一场伟大的马克思主义的教育运动。1979~年我还读到周扬那篇有名的文章《三次伟大的思想解放运动》,周扬将延安整风与五四运动、70~年代末的思想解放运动相提并论,谓之为“思想解放运动”。在大学读书的那几年,我知道,虽然毛泽东晚年的错误已被批评,但毛的极左的一套仍根深蒂固,它已渗透到当代人思想意识的深处,成为某种习惯性思维,表现在中国现代史、中共党史研究领域,就是官学甚行,为圣人避讳,或研究为某种权威著述作注脚,几乎成为一种流行的风尚。当然我十分理解前辈学者的矛盾和苦衷,他们或被过去的极左搞怕了,或是因为年轻时受到的《联共党史》、《中国共产党的三十年》的思想训练太深,以至根本无法跳出官学的窠臼。

然而,我难以忘怀过去岁月留下的精神记忆,刘知几云,治史要具史才、史学、史识,其最重要之处就是秉笔直书,“在齐太史简,在晋董狐笔”。我难以忘记~1979~年在课堂上听老师讲授司马迁《报任安书》时内心所引起的激动,我也时时忆及范文澜先生对史学后进的教诲:板凳甘坐十年冷,文章不写一字空。所有这些都促使我跳出僵硬教条的束缚,努力发挥出自己的主体意识,让思想真正自由起来。从那时起,我萌生一个愿望,将来要写一本真实反映延安整风的史书,为此我开始搜集资料。

由于延安整风在主流话语中是一个特殊的符号,有关史料的开放一直非常有限,这给研究者带来极大的困难,但在~80~年代以后,中国也陆续披露了某些与延安整风运动相关的历史资料,除了少量档案、文件集外,也出版了不少回忆资料,这给研究者既带来了便利,同时也带来了新的问题,这就是如何分析、辨别、解释这些材料。应该说,我在中国大陆长期的生活体验以及我对有关史料的广泛涉猎,加强了我读资料的敏感性,我逐渐能够判断在那些话语后面所隐蔽的东西。

经过对多年搜集、积累资料的反复研究和体会,我头脑中的延安整风的轮廓逐渐清晰起来,我开始发现散乱在各种零碎资料之间的有机联系。1991~年~8~月中旬我开始动笔,~到~1992~年下半年,我已完成初稿的三分之二。

从~1992~年下半年始,我的写作速度慢了下来,一则日常教学工作十分繁重,牵扯了我不少精力;二则我需要更广泛地搜集、阅读各种资料。

1995~年夏至~1996~年秋,我有机会去设于美国首都华盛顿的约翰斯·霍普金斯大学高级国际问题研究院作访问学者。我在美国的研究题目与延安整风无关,但我仍利用在华盛顿的机会,在国会图书馆工作了一个月。但是很遗憾,国会图书馆中文部虽然收藏十分丰富,但是几乎找不到有关延安整风的材料。1996~年~10~月我返国后,又重新开始写作,到了~1998~年夏,全书已经完成。我又用半年时间对书稿作了~3~次修改补充,于~1998~年底,全书杀青。1999~年初交稿后,在编辑校对阶段,我接触到若干新材料,对书中的个别内容再次做了充实,于~1999~年春夏之交,全书最后定稿。

我写这本书在思想上一直以求真求实为依归,在写作过程中,始终遵循据事言理的治学方法。我以为,重要的是,首先应将延安整风的来龙去脉叙述清楚,这个问题之所以重要,乃是因为数十年意识形态的解释学早将当年那场事件搞得云环雾绕,面目不清。为此我作了大量的工作,对各种重要和非重要的资料进行点滴归拢,爬梳鉴别,再对之反复研究体会,使之融汇贯通。这方面的工作用去我最多的时间和精力。

我不反对对延安整风这一重大现象进行严谨的理论分析,且认为,这个工作极为重要,但是我又担心过度解释而妨碍读者自己的判断。陈寅恪先生言,“大处着眼,小处着手”,“滴水观沧海”,因此在本书中,我从实证研究的角度,以分析性论述的方式展开,这也与我个人比较重视历史的个案研究有关。

在写作此书的~7~年里,我一直怀有深深的遗憾,这就是,我无法得到更重要的原始资料。众所周知,有关延安整风期间的中共中央政治局、书记处、中社部、中组部的档案文献,除少量披露外,绝大部分迄今仍未公开。1992~年,我看到一位负责人在中央档案馆的讲话,他说,鉴于苏东巨变的深刻的历史教训,应该加强对档案工作重要性的认识。他指出,中共档案资料的保管,关系到中国社会主义的前途和命运。我可以理解这位负责人的观点,但是从学术研究的角度,却为不能阅读利用这些珍贵史料而感到无穷的遗憾。

由于这是一本站在民间立场的个人写作,十多年来我从自己不多的工资里挤出钱购买了大量的书籍资料,我从没有以此选题申请国家、省级或大学的任何社科研究项目的资助,我知道,即使申请也不会成功。所以我的另一个遗憾是,我无法对一些当年参加过延安整风运动的人士进行口述采访,如果我做了这样的工作,一定会对本书的内容有所充实。

最后,我的遗憾是我应该去莫斯科搜寻有关资料。90~年代后,莫斯科开禁历史档案,涉及~40~年代苏共与中共交往的文献记录也已开放。中国历史学会的沈志华博士近年来为搜集这些史料作了大量工作,他并已将其中某些材料转送北京研究者(沈博士告诉我,苏共与中共在延安整风期间交往的史料很少),因沈博士去美国,一时联系不上,这也使我深感遗憾。

伏案几载,每天神游于当年的历史景像之中,自然会对延安整风运动及其相关的史事与人物产生种种体会,这方面的体会与感受的绝大部分已化为书中的叙述,但是还有几点需在此予以说明:

一、予生也晚也,未能躬逢中共草创革命的年代。吾细读历史,站在~20~世纪全局观二十年代后中共革命之风起云涌,心中自对中共革命抱持一种深切的同情和理解。吾将其看成是~20~世纪中国民族解放和社会改造运动的产物,认为在历史上自有其重大正面价值和意义。

二、从中共革命夺权、推翻国民党统治的角度观之,延安整风运动对于中共革命成功助力巨大,但是延安整风运动中的某些概念、范式以后又对中国的发展和进步产生若干消极作用,极左思想、权谋政治汇溪成流,终至酿成建国后思想领域一系列过左的政治运动直至文革惨祸,真所谓“成也萧何,败也萧何”!所幸中共十一届三中全会后,国家已逐步走出过去那种怀疑一切,无情斗争的极左道路,但旧习惯思维的清理仍需长期努力。吾期盼旧时极左的“以我划线”、权谋政治永不再来,国家从此能步入民主、法治的轨道,如此,则国家幸甚,民族幸甚!

三、本书涵盖面颇宽,涉及中国现代史上许多著名人物,对于本书所论及的所有人物,我只将其看成历史人物,不存任何既定的好恶偏见,主观上力求客观公允,“不虚美,不隐恶”。当然任何研究都不可能完全排除作者的价值关怀,陈衡哲先生曾说过,“若仅缕述某人某国于某年征服某地……那有什么意思”,说的也是研究者的价值关怀问题,只是这种价值关怀不应妨害到叙述的中立和客观。如果说本书叙述中有什么价值倾向的话,那就是我至今还深以为然的五四以后的新价值:民主、自由、独立、社会正义和人道主义。

在写作此书的几年里,我得到了一些朋友的宝贵的支持和鼓励,在本书即将出版之际,我谨向他们表示真挚的感谢。

上海师范大学的许纪霖教授多年来一直关心我研究的进展,他还热情的为本书的出版提出许多好的建议。在与许教授的交往中,他的深厚的学养和对二十世纪中国历史的卓越见解总是使我深获教益。

我衷心感谢香港中文大学中国文化研究所的金观涛教授和刘青峰教授。他们对本书的出版提供了热情的帮助,在本书定稿过程中,他们提出一些富有启迪性的建议和意见,对于本书臻于学术规范化的要求,有重要的作用。

我也向我的同事,南京大学历史系颜世安教授和我的好友,现旅居美国的贺军先生表达我的感激,他们的友谊和支持,对于我一直是一个激励。

我曾与美国哈佛大学东亚系孔斐力教授(Philip Kuhn)和美国约翰斯·霍普金斯大学高级问题研究院“华盛顿——南京办公室”主任甘安哲博士(Anthony Kane)有过多次关于三十至四十年代中共党史问题的愉快的讨论,他们的支持和鼓励对于我的写作是一种推动。

在写作此书的几年里,我始终得到我过去的学生甘思德(Scott Kennedy)和唐山(Jeff Zuckerberg)的关心和帮助,我的研究生郭洵澈对我帮助尤大,他不仅帮我用电脑输入文字,还与我分享了讨论的乐趣,在此我向他们表示深切的感谢。

我也向本书所引用文字的作者、编者表示我的谢意,我虽然在引述文字时都做有详细的注释,但没有他们提供的资料基础,我要完成这本书也是不可能的。

我要向本书的编辑郑会欣博士表达我深深的谢意,郑博士自己有大量的研究任务,但是他还是拨冗为本书做了许多琐细的工作,他的慷慨相助对本书的出版有重要的帮助。

1998~年夏秋之际,我有机会前往香港中文大学中国文化研究所作访问研究,在“大学服务中心”得到熊景明女士的热情接待和帮助,在这个收藏丰富的史料中心,我为本书补充了若干新的资料,在此向熊景明女士和“大学服务中心”表示谢意。

南京大学历史系资料室的老师们多年来在图书资料方面给了我许多帮助,对他们的友好、善意和敬业精神,我深表感激。

最后,我要深深地感谢我的妻子刘韶洪和儿子高欣,我的妻子在每天工作之余,承担了大量的家务,使我可以专心致志进行研究,她还为书稿作了一部分的电脑输入工作。为了写作这本书,许多年我不能和妻子、孩子一同出外游玩,也不能与孩子经常讨论他的功课,没有他们的支持、帮助和理解,我要完成此书是完全不可能的。

\Sign{1999~年~6~月于南京大学}
\end{pre-post-text}


