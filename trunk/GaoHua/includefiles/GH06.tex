%# -*- coding:utf-8 -*-
%%%%%%%%%%%%%%%%%%%%%%%%%%%%%%%%%%%%%%%%%%%%%%%%%%%%%%%%%%%%%%%%%%%%%%%%%%%%%%%%%%%%%

\chapter{整风运动前夕中共的内外环境与毛泽东的强势地位}

\section{四十年代初延安的社会生态构成}

自~1935~年~10~月中共中央随红军长征至陕北,经过数年的经营,在四十年代初,中共已在延安建成一个高度组织化的社会。

四十年代初的延安的有人口三万七、八千人,市区居民的七千人,大部分居住在城南,三万多人是中共中央和边区各机关、学校的干部,他们散居在延安及其郊区。\footnote{据《谢觉哉日记》称:1938~年~9~月他询问延安市市长高朗亭,被告知延安市居民六千多,学生、干部、部队有二万多。查~1939~年仍有许多外来青年进入延安,1939~年末后,外来青年来延安人数急剧减少,这样到四十年代初,延安学生、干部人数就达到三万人左右。另据胡乔木称,抗战爆发后来延安的同志共四万人。笔者认为,这四万人包括在边区各县工作的外来干部,在延安的干部一般估计在三万左右。参见《谢觉哉日记》,上,页~273~—~274;《胡乔木回忆毛泽东》,页~279。}

在中共的设计下,依照瑞金时代的经验,延安人口构成中的这两部分都已被充分地动员和组织起来。在边区和延安市,中共建立了垂直的党政机构和群众团体,中共的政令可以自中共中央、边区党委(西北局)、边区政府一直下达到市、区、乡党组织,直至农村中的党支部。边区自卫军在各区、乡、村都有基层组织,他们不仅从事农业生产,还担负起检查路条、捉拿嫌疑分子的治安保卫任务。

延安的三万多干部更是具有高度的组织性。分属于不同机关和学校的这几万名干部,虽然生存在延安和边区的环境中,却自成体系,与延安的百姓基本不相往来;构成了一个十分独特的中共干部群体。中共干部群体的独特性,首先在于他们绝大多数都是中共党员,具有鲜明的意识形态色彩;第二是他们都是“公家人”。每个人都有自已的“伙食单位”,过着一种军事共产主义的生活。

延安生活的意识形态化与中共的性质及抗战后延安环境的变化有密切关系。中共原本就是一个以意识形态为号召的政治和军事集团,意识形态是维系其存在与发展的基本动力,1937~年后,大批知识青年就是冲着中共的意识形态而投奔延安的。为了安置和训练大量来延的知识分子,中共在延安创立了十余所各类学校,其数量与规模远远超过瑞金时代。

四十年代初,延安几乎成了一座学校城,各类学校作为中共意识形态的训练和传播基地,在延安的政治生活中起着重要的作用。延安有马列学院、中央党校、陕北公学、抗大总校、中国女子大学、鲁艺、泽东青年干部学校、中央组织部训练班、中央职工委员会训练班、西北公学(枣园训练班)、自然科学院、民族学院、军事学院、炮兵学校、军委机要学校、西北行政学院、新文字干部学校等,这些学校与国统区的所谓“正规学校”完全不同,不仅学时较短,课程设计也主要以思想训练为主,因此,大量的学校又强化了延安的意识形态氛围。

生活在延安的干部群体在各自的机关、学校过着一种基本生活用品依赖于平均分配的供给制生活。抗战初期,国民政府对中共军队尚有经费拨出。\footnote{在~1938~年,延安各机关、学校的生活条件总的情况还不错,中组部的干部一周能吃一顿大米,两顿白面,其余都是小米,每周还能吃二、三次肉。参见刘家栋:《陈云在延安》,页~71。}1939~年后,国共关系恶化,边区开始被国民党军队封锁,国民政府拨给中共军队的经费也时断时续,1940~年后更完全中断,造成边区和延安的物质供应出现极严重的困难。

边区一百五十万居民要养活近十万的中共部队和干部,负担极重。为了舒缓粮食压力,1939~年抗大总校和陕北公学迁移至华北办学。中共为了彻底解决物资供应困难的局面,在~1939~年后发起生产运动,到~1940~年后更掀起大生产运动,同时各机关、单位纷纷兴办了各类经济实体,以解决单位内部的物资供应。作为战时环境下为解决生存而采取的一项特殊措施,中共还以“特种物资”的名义,“种某物”,“甚至发展到某货内销”。\footnote{《谢觉哉日记》,下,页~734;另参见弗拉基米洛夫:《延安日记》(台北:~联经出版事业公司,1976~年),页~111、155~—~56;另参见周维仁:《贾拓夫传》(北京:~中共党史出版社,1993~年),页~77。}

1939~年后,延安的物质生活已十分艰苦,但精神文化生活却十分丰富,大批青年知识分子来到延安后,马上就发现自己置身于红色理论的海洋。中共在延安创办了一批报刊,计有《新中华报》(1941~年扩大为《解放日报》)、《解放》周刊、《共产党人》、《八路军军政杂志》、《中国青年》、《中国妇女》、《中国工人》、《中国文化》等。延安最大的出版单位——解放社还出版了《马恩丛书》等各种理论和政治宣传读物。1939~年萧三从苏联返回延安,办起了俱乐部。迅速将交际舞传播了开来,交际舞会、京剧晚会、大合唱、《日出》、《雷雨》,使延安的革命斯巴达式的生活又增添了一种活泼、欢快的气氛,除了间或有日军的空袭,延安似乎已远离战时生活环境。

在延安“公家人”的日常生活中,老干部占据了极重要的位置。所谓“老干部”,大多也只是二十八、九岁至三十多岁,他们一般都经历过长征,现在担负着各机关、学校的领导工作。1941~年后,延安将许多高级干部从各根据地调来延安参加整风学习,另外也带有储存、保养高级干部的意图。中共中央为了照顾担负领导职务的老干部,专门成立了中央保健委员会,为他们提供较好的物质与医疗服务。延安的两个主要医院:~边区医院与中央医院的服务对象也有差别,中央医院主要为领导干部及其家属服务,间或也为经组织介绍的一般干部治疗。为了照顾老干部的身体,1940~年延安光华农场建成,里面存养了一群从华北根据地迁来的荷兰奶牛,享有特灶供应的老干部每天都可享有鲜牛奶供应。

随着抗战初期延安的抗战救亡气氛慢慢转变为日常生活的气氛,延安“公家人”内部的关系也逐渐发生变化,不仅老干部与新干部之间存在着许多矛盾,即使老干部间也是飞短流长,相比之下,延安青年知识分子之间的关系还比较亲密,\footnote{参见王惠德:《忆昔日》,载《延安马列学院回忆录》,页~76。}尽管气氛已与~1937~—~1938~年很不一样。

1940~年后,三万多“公家人”已在延安完全安定了下来,党在思想上、行动上和生活上全面地照管着他们,此时延安与大后方的交通已经中断,除非经组织派遣和遣散,延安的干部已不能返回国统区或自行前往任何地区,在这个高度组织化的社会,个人离开组织就意味着失去了一切。作为中共最高领袖的毛泽东,有理由为在延安建成的新社会感到高兴和满意,毛对延安的这一切并不陌生,早在三十年代初,江西瑞金就初步形成了这样一个社会,只是那时地狭人少,又面临紧迫的战争环境,毛且不是那个社会的真正主人,因而那时毛高兴不起来。现在斗转星移,中共已今非昔比,毛更成了党的最高领袖,他焉能不真正喜悦?数年前丁玲在保安曾以“小朝廷”当面向毛描述了她对陕北苏区的观感,毛就对这个说法“很感兴趣”,\footnote{李锐:《忆丁玲》,载李锐:《直言》(北京:今日中国出版社,1998~年),页~368。}现在毛当然不会在乎蒋介石说边区是什么,“国中之国”、“封建割据”,毛就是要在边区创造一个“国中之国”,未来还要夺取全中国!

\section{与蒋介石、斯大林相周旋}

1941~年春夏,毛泽东向国际派进攻的部署已基本就绪,他的关注目标转向了外部,这就是重庆的国民党政权和远方的莫斯科,毛需要判断清楚蒋介石和斯大林可能将作出的反应,再最后决定自己的行动。

在中国历史屡次的朝代更替中,越王勾践“卧薪尝胆”、“十年生聚、十年教训”一类的故事曾多次重演,毛泽东在抗战前期对蒋介石的策略就堪称是这类故事最成功的范例。1937~年秋,毛以中共不足三万人的兵力,暂时向蒋介石作出妥协的姿态,赢得整军修武、巩固内部的时机。即将展开的延安整风,就是毛利用国共合作局面加速整饬内部的重大举动,毛要抓住时机,先行打扫自己的后院。

在八年对日抗战时期,毛泽东与蒋介石的关系可大致划为两个阶段。1937~—~1940~年,可称上是中共向国民党妥协的阶段,1940~年后,则是从妥协走向分庭抗礼的阶段。在~1937~—~1940~年,毛泽东为了发展中共军事实力,不图虚名,与蒋介石虚与委蛇,使中共的力量,尤其是八路军的实力,在两三年内得到迅速的发展。随着中共实力的壮大,毛泽东不愿意再继续扮演“勾践”的角色,他要和蒋介石一比高低,让中国同时出现两个“太阳”。1943~年在延安的窑洞里,毛泽东对国民党驻延安联络代表徐复观(当时名徐佛观)说:“再过五年、八年,看鹿死谁手!”\footnote{参见~1943~年~11~月~1~日唐纵日记,载公安部档案馆编注:《在蒋介石身边八年——侍从室高级幕僚唐纵日记》(北京:群众出版社,1991~年),页~389。}

“天无二日,人无二主”,让蒋介石承认中国有两个“太阳”,绝非易事,至少要先成为中国共产党的“太阳”。然后才有可能与国民党的“太阳”一决雌雄。从遵义会议后,毛泽东就为实现这个目标进行不懈努力。到~1941~年,一切都水到渠成,形势对毛十分有利。

从陕北的外部环境看,中共军队与日军基本处于胶着状态,自~1940~年~8~月八路军发起“百团大战”后,八路军与日军没有再发生那种震动全局的大规模战斗,此种情况正可使毛泽东腾出手来将大批干部调来延安学习。加之日军距中共腹地陕甘宁边区比较遥远,毛完全可以利用这局部和平的时机,加紧整肃内部。

对于国民党,毛泽东始终保持高度警惕。\footnote{1939~年后国共冲突加剧,毛泽东密切注意国民党可能发动的反共进攻,经常提醒中共负责干部作好国共关系破裂的最坏打算。参见《周恩来年谱》,页~472、474。}然而,毛有充分的把握,确信蒋介石不敢在苏、美、英三国的压力下,冒险进攻陕甘宁边区。潜伏在国民党党、政、军、特机构的中共秘密情报人员,随时可将蒋介石的动态提前报知延安,\footnote{抗战时期,中共的情报工作分为三个相对独立又互相联系的系统:一、以康生为核心的中央社会部;二、以周恩来、李克农、吴克坚为核心的中共南方局敌后委员会(吴克坚情报系统);三、以潘汉年为核心的沪、港情报系统(华南情报局)。康生具体协调这三个情报系统,然而周恩来因长期在国统区担负党的最高领导,他也对全党的情报工作负重要责任。1940~年~9~月~18~日,中共中央发出《关于开展敌后大城市工作的通知》,宣布成立中央敌后工作委员会,由周恩来负总责,康生副之。参见《周恩来年谱》,页~467~—~68。周恩来、李克农在国民党各重要机构、各地方实力派中建立和发展了许多极其重要的战略情报关系,周恩来、董必武于~1938~年初派遣熊向晖打人胡宗南部。熊逐步获得胡的信任,担任了胡宗南的机要秘书,将许多绝密情报报送延安。参见熊向晖:《地下十二年与周恩来》(北京:中共中央党校出版社,1991~年),页~22~—~23、25。中共党员王超北(文革初期中央文革小组成员林杰的岳父)及其部下在~1938~年后也一直在西安从事情报工作,向延安提供了大量的政治、军事情报。参见王超北口述、师宁编写:《来自秘密战线的报告》(天津:百花文艺出版社,1997~年)页~19~—~20、47、70~—~77。王超北于~1962~年以“内奸”罪名被捕,被囚于秦城监狱,文革后获释。属于吴克坚系统的沈安娜从~1938~年就打入国民党中央党部机要处,以其速记特长搜集大量情报,国民党在此阶段召开的历届中央全会、中央常务委员会、国防最高委员会的所有秘密都经沈安娜报送延安。参见薛钰:《周恩来与党的隐蔽战线》,载《中共党史研究》,1998~年第~1~期。}毛完全可以依此早作防范。毛泽东判断蒋介石根本无法插手中共内部的斗争,蒋介石虽乐意看到中共内部自相火并,但只会采取隔岸观火的态度。

毛泽东对蒋介石知之甚深,他对蒋介石的估计基本符合实际。四十年代初,国民党虽然对中共内部的矛盾知之甚详,但在整风运动期间,国民党始终处于旁观的立场。

作为中共老对头的蒋介石,自二十年代末以后,一直高度重视刺探中共内部情报。国民党获知中共情报的主要渠道是投奔国民党的原中共自首人员。自顾顺章、向忠发叛变共产党后,从~1932~年开始,国民党逐渐调整镇压中共的政策,将对共产党员的肉体消灭与胁迫反省、自首相结合。据国民党中央组织部调查科(中统)统计,从~1933~年~7~月到~1934~年~7~月,被国民党抓获的中共党员共~4,505~人,其中自首者~4,209~人。三十年代前期,被国民党捕获和向当局自首的共产党员约两万四千余人,办理“自新”手续的赤色群众三万人。\footnote{中国国民党中央组织部调查科编:《中国共产党之透视》(35~年~2~月~21~日),载吴相湘主编:《中国现代史料丛书》,第~3~辑(台北:文星书店,1962~年),页~1;另参见郭华伦:《中共史论》,第~2~册,页~260。}1934~年~6~月~26~日,中共上海中央局书记李竹声被国民党逮捕,旋即自首;同年~10~月,继任书记盛忠亮也被捕,投靠国民党。从李竹声、盛忠亮等被捕、叛变的中共干部那里,蒋介石获知了在中共党内存在着留俄派与老干部派的尖锐斗争。由共产党变节分子李士群、丁默郊等主办的《社会新闻》、《现代史料》也向社会公布了有关这方面的大量材料。\footnote{《社会新闻》的创办者为原中共自首人员李士群、丁默村。李士群在大革命时期曾参加中共,1927~—~1928~年曾在苏联接受“格伯乌”的训练,返国后在中共中央特科工作。1932~年被国民党逮捕,迅速转向,成为国民党中央组织部调查科上海区直属情报员。不久奉陈立夫之命,李士群等在上海公共租界白克路同春坊新兴书局编辑《社会新闻》。《现代史料》~1933~年由上海海天出版社出版,其幕后主持者也是调查科。}

抗战爆发后,蒋介石获知中共内部情况的最重要来源,是~1938~年投奔国民党的原中共高级领导人张国焘。蒋介石从张国焘那里,完全掌握了~1937~年后有关毛泽东与王明、周恩来等意见分歧的情报。

蒋介石获取中共情报的另一条渠道是国民党派驻延安的观察员。1937~年国共合作实现后,蒋介石在中共首府延安派驻了国民党联络机构和联络参谋。在中共情报及反间谍机关的严密控制下,国民党驻延安观察员虽无从获得中共重要机密,但是,仍能因其就近观察之便,捕获到中共若干战略性的信息。五十年代后,以“现代新儒家”享誉台、港及海外的徐复观曾在~1943~年作为国民党联络参谋驻延安近一年。徐复观在延安期间,与毛泽东、朱德、刘少奇、叶剑英等皆有接触,对中共励精图治留有深刻印象\footnote{徐复观:《刘少奇平反与人类的良知良识》~1980~年~3~月~4~日),载《徐复观杂文续集》(台北:时报文化出版有限公司,1986~年),页~218。}。1943~年~10~月下旬,徐复观返渝后,在与侍从室、军统局高级人员交谈时,“历述延安荒谬狂悖之情形”,认为对中共问题“非用武力不足以解决。任何方法,徒枉空言。而用武力,在目前政治现状下,前途并不可乐观”!徐复观并撰有一延安印象意见书,上报蒋介石、何应钦,受到蒋氏的重视。蒋介石曾在徐复观意见书上作了“眉批”,并下令印成小册子,在少数高级情报人员中传阅\footnote{参见~1943~年~11~月~1~日唐纵日记,载公安部档案馆编注:《在蒋介石身边八年——侍从室高级幕僚唐纵日记》(北京:群众出版社,1991~年),页~388、386~注释~1。}。

蒋介石虽然大致了解中共内部的矛盾,但是他除了隔岸观火之外,别无其它选择。蒋介石十分清楚,他对中共政策的制定毫无影响力。从蒋介石的内心而论,他也只会对即将发生的中共内部的整肃斗争兴高采烈,蒋介石实在希望中共因毛泽东与王明等冲突而四分五裂。

毛泽东对蒋介石的心态完全把握,他虽不时电示在重庆的周恩来与蒋介石周旋,但是毛在~1941~年并没有将蒋介石放在太重要的位置。毛更重视的还是远在莫斯科的斯大林的态度。

1940~年~2~月后,随着中共驻共产国际代表任弼时离开莫斯科返回延安,中共在莫斯科就已经没有正式代表,中共驻共产国际代表团实际上已不复存在。从此,莫斯科与中共间的联络,基本上就由延安与莫斯科之间的电台和苏联派驻重庆和延安的机构承担。

在这一时期,苏联获知中共情报的渠道主要有五条:

一、苏联驻重庆大使馆及苏联塔斯社驻上海分社。抗战期间,苏联对华情报工作除了收集侵华日军活动及国民党各项动态外,也兼及收集中共,尤其是中共与美国联系的情报。

1942~年苏联情报机构与重庆军令部第二厅联合创办“中苏情报合作所”,苏方派有几十名技术人员参与其事,主要负责收集沦陷区日军情报。苏联并通过驻华大使、格伯乌中将潘友新及驻华武官罗申,与中共南方局和中共驻上海地下组织保持秘密联系。苏联并在上海创办中文刊物《时代》,由老资格情报人员罗果夫领导,\footnote{罗果夫约在抗战初来华,以上海为中心进行情报工作,1941~年~8~月在沪创办《时代》,次年~11~月创办《苏联文艺》,吸纳部分中共地下党员参加编辑工作,计有姜椿芳等。罗果夫于~1949~年~10~月~1~日以塔斯社驻华记者的身分参加了中华人民共和国开国大典。}主要搜集日本情报,但也通过刘晓、刘长胜、潘汉年间接了解中共活动情况。

二、苏联驻兰州的外交和军事代表处。抗战开始以后,兰州成了苏联援华物资输人中国的交通枢纽,苏联空军在兰州设立了机场,为此苏联在兰州设立了办事机构。该机构与中共驻兰州办事处保持密切联紧。中共通过八路军驻兰州办事处向苏共提供过若于中共内部情况的情报。\footnote{参见伍修权(时任八路军驻兰州办事处处长)《回忆与纪念》:(北京:中共中央党校出版社,1991~年),页~168。}1956~年,苏共中央向中共移交的档案文件中,就包括中共西北局、南方局的会议记录。\footnote{裴桐(前中央档案馆副馆长):《一九五六年赴苏联接收档案追忆》,载《党的文献》,1989~年第~5~期。}以后担任苏联驻延安观察组代表的弗拉基米洛夫(中文化名孙平),在~1938~—~1940~年曾出任苏联驻兰州军事代表处的副代表。

三、苏联驻迪化总领事馆。从~1934~年起,苏联利用盛世才将其势力大规模渗入新疆,在新疆全省境内派驻大批政治、经济、军事顾问,苏联红军“红八团”长期驻守哈密。斯大林为长期控制盛世才,邀其加入苏共,但不允许盛世才参加中共。\footnote{1944~年~12~月~19~日唐纵与盛世才的谈话,参见《在蒋介石身边八年——侍从室高级幕僚唐纵日记》,页~478。}苏联情报机构在新疆有极广泛的活动,其中包括收集中共在新疆活动情况及中共中央内部情报。

四、苏联驻延安联络小组。1939~年后,苏联在延安即设有联络人员,从~1942~年~5~月起,弗拉基米洛夫以共产国际联络员、苏军情报组和塔斯社记者身分任联络小组负责人,该联络组实际上是斯大林派驻延安的情报机构,有电台译电员、医生等五、六人。

五、苏联在东北、华北各大城市建立的秘密情报网。参加情报网的中共党员由延安派出,但一般不与中共直接发生联系,主要为苏联搜集有关日军的情报。

苏联虽然通过在中国派驻的上述机构,积极搜集中国情报,但由于~1940~年欧洲战争形势日益尖锐,苏德随时可能爆发战争,因此,斯大林更多地关注欧洲战场和苏联国内的战争准备事务。

作为国际共运领袖的斯大林,对于中共内部三十年代中期后的变化,即毛泽东在中共党内的崛起及逐渐控制中共,总的说来,是乐观其成的。尤其从~1938~年起,中共的立场和观点基本上一直得到莫斯科的支持。斯大林当然了解王明,但他更知道王明的老师、共产国际前东方部部长米夫已在~1938~年被自己清洗,因而斯大林并不准备对王明施之以援手。只要毛泽东在涉及苏联等问题的立场与莫斯科保持一致,斯大林就不愿对延安说些什么。

但是到了~1941~年后,情况却发生了变化,斯大林发现,已经牢牢控制中共的毛泽东对来自莫斯科的指示阳奉阴违,而这些恰恰又涉及到保卫苏联、支持苏联等重大问题。毛泽东采用拖延、回避、指东道西等种种方法,拒绝出动八路军进攻驻扎在华北和中蒙边境地区的日本军队。

据不完全统计,从~1941~年初至也~43~年,莫斯科曾多次给毛泽东和中共中央发出电报,要求中共在防止日军进攻苏联问题上与苏军协调行动。莫斯科并且具体要求毛派出八路军大部队破坏华北铁路。向日军发起进攻,以减轻日本军队对苏联的压力。\footnote{弗拉基米洛夫:《延安日记》,页~55、72;另参见《在历史的巨人身边——师哲回忆录》,页~213~—~15。}

对于斯大林所有这类要求,毛泽东均未作出肯定答复,而是用种种办法加以搪塞。斯大林实在小看了毛泽东,他怎么会做这种鸡蛋碰石头的傻事。不管莫斯科如何用“国际主义”的大话“套”毛泽东,他就是不上钩。直到这时斯大林才发现,毛泽东原来是一个“民族主义者”,但已为时晚矣,因为毛已牢牢控制了中共。面对毛在延安“另搞一套”,莫斯科虽然强烈不满,却毫无办法!

毛泽东知道自己已经得罪了斯大林,却并没有“破罐子破摔”。毛深知中共离不开苏联的支持,自己绝不能意气用事,虽然在涉及中共军队和其个人领导权等基本问题上,毛坚持己见,绝不让步,但在其它次要问题上,毛却不放过任何一个机会,主动向斯大林示好,争取斯大林个人对毛的好感。

1941~年~4~月~13~日,苏联与日本签订了有损中国利益的《苏日中立条约》,激起中国朝野强烈的不安。三天后,中共通过新华社正式发表意见,称赞该条约是“苏联外交政策的又一次伟大胜利”,并且对苏日声明中有关“互不侵犯满洲与外蒙”的说辞予以充分理解,宣称苏联此项举措“对全中国争取解放也是有利的”。中共此举,使“向日同情中共之青年;莫不痛哭流涕”。\footnote{1941~年~4~月~14~日,国民政府外交部就《苏日中立条的》发表声明,宣称东北三省及外蒙古为中国领土,决不承认第三国之间妨害中国领土及行政完整的任何协定。沈钧儒、王造时等知名人士欲在报刊公开发文,对《苏日中立条约》表示遗憾,后在周恩来的劝说下作罢,但知识界对苏联的怀疑并未真正化解。参见《在蒋介石身边八年——侍从室高级幕僚唐纵日记》,页~203。}

1941~年~6~月,受周恩来领导,从事国际情报工作的中共秘密党员阎宝航(以东北救亡总会负责人身分在重庆活动),探悉希特勒德国即将进攻苏联的绝密消息,周恩来立即将此情报电告延安,毛泽东迅即电告斯大林。\footnote{阎宝航五十年代曾任中华人民共和国外交部办公厅主任,其子阎明复长期担任毛泽东、刘少奇等领导人的俄文翻译,八十年代升任中共中央书记处候补书记兼中央统战部部长,1989~年因“六四事件”解职,现任中华全国慈善总会负责人。文革期间阎宝航父子均以“苏修特务”的罪名被囚于秦城监狱,阎宝航病死于狱中。参见《周恩来年谱》,页~507。}事后斯大林曾致电毛泽东,表示感谢。\footnote{参见薛钰:《周恩来与党的隐蔽战线》,载《中共党史研究》,1998~年第~1~期。据崔可夫回忆,自他于~1940~年底赴重庆履任后,即与周恩来、叶剑英有密切的接触。参见瓦·伊·崔可夫:《在华使命——一个军事顾问的笔记》,页~49~—~50;}

毛泽东除了向斯大林及时通报中国战场有关情况,还十分重视和苏联驻重庆人员保持良好的关系。在毛的关照下,周恩来经常约见潘友新、崔可夫、罗申,交换对国内政局的意见(苏联驻重庆外交官在向莫斯科汇报中国政局时必须坚持“阶级立场”,苏驻重庆使馆曾因一度向莫斯科反映了中国国内社会舆论对中共的批评而受到苏外交部的严厉指责)。\footnote{《周恩来年谱》,页~473、485。另参见安·麦·列多夫斯基:《在中国的(1942~—~1952)和《外交官笔记》,载俄国《近代史和现代史》杂志,1993~年第~6~期《外交生涯》。}毛泽东本人也经常在延安与弗拉基米洛夫进行长时间的晤谈,试图吸引苏联驻延安代表对毛的主张的理解与好感,并通过弗拉基米洛夫向斯大林传送有利于自己的信息。\footnote{参见师哲:《在历史的巨人身边——师哲回忆录》,页~220。}毛泽东对斯大林所作的一切友善姿态皆出于一个目的,利用斯大林的威望加强自己在中共中央内的地位,在即将向留苏派发起进攻时,麻痹斯大林。

1941~年~6~月~22~日,苏德战争爆发,毛泽东行动的时机终于到来了。现在斯大林已陷于战争的泥沼而无暇东顾,莫斯科鞭长莫及,即使斯大林派驻延安的“眼线”虎视耽眈,紧紧盯住毛泽东,那也无济于事。毛泽东将立即采取行动,对不可一世的斯大林实行远交近攻:铲除其在中共党内的门徒,而仍将斯大林奉为门神。

\section{毛泽东最坚定的盟友:刘少奇及其班底}

毛泽东发动党内斗争的基本策略一向是,为了打击主要敌人,团结依靠一部分人,利用他们作为贯彻自己意见的工具。开展延安整风运动,就是毛这类斗争策略的第一次大规模的操练,在这次战役中,毛的最坚定的盟友和帮手就是刘少奇及其班底。

1937~年春夏,在反对张闻天基础上建立起来的毛泽东、刘少奇联盟,在以后的一年中得到进一步的加强。1938~年~3~月,毛泽东为加强自己的力量,特地将在“十二月政治局会议”上受到王明指责的刘少奇调回延安。刘少奇回到延安后,成为毛泽东的得力助手,在~1938~年上半年毛泽东与长江局的争斗中,旗帜鲜明地站在毛的一边。刘少奇抑制不住对王明的强烈不满,经常在中央工作人员面前指名道姓抨击王明。据当时担任中央秘书处负责人的王首道回忆,在六届六中全会之前,刘少奇曾向他们说过:“我们北方局,不仅成为抗日的根据地,也是支持毛泽东同志与王明右倾机会主义作斗争的根据地。”\footnote{王首道对刘少奇的这段描述可能有点言过其实,虽然刘少奇对王明不满,但“王明右倾机会主义”的概念当时还没有发明。参见王首道:《坚定地站在正确路线一边》,载《怀念刘少奇同志》,页~6。}六届六中全会以后,刘少奇对王明的态度更加激烈。他似乎完全忘了六中全会制定的有关“各政治局委员不得有破坏政治局和其他委员的威信之言论行动”\footnote{《中共扩大的六中全会关于中央委员会工作规则与纪律的决定》(1938~年~11~月~6~日通过),载中央档案馆编,《中共中央文件选集》(1936~—~1938),第~11~册,页~763。}的决定,除了不时在部下面前诉说自己昔日受压的经历外,还在许多场合不指名的对王明进行攻击,称其\footnote{参见《王首道回忆录》,页~216。}是“披了马列主义招摇撞骗,是江湖上卖狗皮膏药的”。

刘少奇对王明所持的势不两立的立场,被毛泽东一一看在眼中。1938~年~9~月下旬,刘少奇终于得到回报,他同时被委以北方局和中原局两个中央局书记的要职。在一个短时期内,刘少奇虽兼着北方局书记的职务,但他工作的重心却转移到了华中。1939~年~1~月~28~日,刘少奇到达河南省确山县竹沟镇,迅速组建了中原局领导机关,具体落实毛有关开辟华中的重大战略行动。

毛泽东派遣刘少奇去华中,首先是因为毛信任刘少奇,在对一系列引起党内争议的重大问题的看法上,刘少奇与毛完全一致。刘少奇积极支持毛有关加速发展中共武装的意见。根据档案记载,从~1938~年春至~11~月中旬,刘少奇单独或与毛泽东、张闻天联名发往各战略区,指导创建根据地的电文、信件多达百余件。\footnote{参见马齐彬、陈绍畴:《刘少奇与华北抗日根据地的创立》,载《文献和研究》,1986~年第~5~期。}刘少奇抵达竹沟后,又于~1939~年~11~月下旬,将中原局机关移至皖东,全力“招兵买马”,开辟了皖东、苏北根据地。刘少奇还提议山东八路军派一部南下,新四军派一部北上,一年后,实现了两部在苏北的会师,壮大了中共在华中的实力。刘少奇还经延安批准,积极部署对日伪的策反工作。1941~年~12~月,刘少奇亲自派遣新四军情报人员冯少白秘密前往日军占领下的上海,\footnote{《刘少奇年谱》,上卷,页~385。冯少白(冯龙)当时在新四军参谋处担任科长,因冯的姑父邵式军(盛宣怀外孙)任汪伪政府税务署署长,与汪政权高层人物有密切交往,因此才被刘少奇选派执行此重要战略任务。从~1941~年~12~月至~1945~年~8~月,冯少白多次进入上海、南京,除了争取物资、钱款资助外,还与汪政权高层人物秘密接触,试图争取周佛海、陈公博起义。1943~年~3~月~10~日,陈公博会见冯少白,给了冯一本电报密码,责成邵式军建立与新四军联络的电台。由于周怫海早与戴笠系统挂钩,中共策反周佛海的活动最终失败。1945~年~8~月,邵式军被中共地下人员接到淮阴,其部分财产(黄金数百条)被安全转给中共地下党。参见冯少白:《抗战时期四进敌占区进行侦察和策反等工作的回忆》,载上海《党史资料丛刊》,1980~年第~2~辑,页~54~—~63。另见施岩:《关于冯少白同志策反周佛海的经过及其结果》,载上海《党史资料丛刊》,1981~年第~3~辑,页~133~—~38。}与李士群、周佛海、陈公博等联络。取得一定的收获。不久,在潘汉年具体策划下,开辟了苏北和上海、南京、镇江之间的物资供应和人员往来的地下信道。

毛泽东委派刘少奇负责中原局的另一个目的是利用刘少奇牵制项英,以求进一步削弱项英的权力。毛泽东对项英长期抱有深刻的疑惧,然而项英雄厚的革命历史和因坚持三年江西游击战争在党内所享有的崇高威望,又使毛一时无从对项英下手。1938~年~9~月下旬,在宣布设立中原局的同时,中共中央决定将项英领导的中共东南分局升格为东南中央局,仍由项英担任书记。1939~年~7~月~3~至~8~月~25~日举行的中央政治局扩大会议,特别表彰了项英领导下的东南局,这些都显示了毛对项英的安抚和让步。\footnote{《周恩来年谱》,页~446~—~48。}但是,毛绝不会坐等项英实力的壮大而无动于衷,就在毛不得不顺应政治局多数人的意见,对项英的工作表示满意的同时,针对项英的活动也在公开进行,设立中原局就是毛旨在抑制项英的一项重大举措。中原局的工作范围包括长江以北、陇海路以南的广大地区,原属长江局和东南分局主管,现在将其划归于刘少奇的工作范围,明显地削弱了项英的权力。

刘少奇被派往中原局后,很快在新四军军力调配问题上,与项英产生了意见分歧,在刘少奇与项英的对抗中,毛泽东坚决站在刘少奇一方。毛知道,1939~年以前,刘少奇几乎从未涉足军队,刘少奇的资历与声望,皆不足与拥有丰富军事和党务经验的项英相抗衡。1940~年夏,为了扩大自己在新四军的影响并支持刘少奇,毛利用调解项英与叶挺矛盾之际,任命与刘少奇关系密切的饶漱石和自己的老部下曾山担任东南局副书记(东南局原不设副书记职)。1940~年~11~月,毛又内定撤除项英的职务,将东南局与中原局合并,组成华中局,由刘少奇担任书记,只是惧于项英的反应,这个决定才未急于出台。12~月~31~日,中央书记处决定“山东分局归中原局管辖,中原局统一领导山东与华中”,\footnote{《刘少奇年谱》,上,页~321、324。}这个决定事实上完全架空了东南局,使其名存实亡。几天以后,“皖南事变”爆发,1~月~9~日,刘少奇致电延安,提议中共明令撤项英职,毛泽东~1~月~10~日覆电,表示目前撤项英职一事暂不必提。\footnote{《刘少奇年谱》,上,页~321、324。}随后几个月,刘少奇在华中主持了清算项英等“错误”的斗争,1941~年~1~月~27~日,延安宣布合并东南局与中原局,成立华中局,由刘少奇任书记,饶漱石任副书记。

毛泽东为了自己的政治目的提携刘少奇,刘少奇则以加大对王明的批判回报毛泽东,毛刘两人互相支持,配合默契,刘少奇成为毛泽东棋盘中炮打王明的马前卒。

1938~年~11~月~6~日,毛泽东在六届六中全会闭幕辞中提到刘少奇的名字,指出“少奇同志说得好”,“如果所谓‘一切经过’(统一战线),就是经过蒋介石和阎锡山,那只是片面的服从”。

1939~年~4~月,刘少奇离开河南确山县竹沟镇返回延安参加政治局扩大会议,从~8~月~7~日至~12~日,刘少奇在延安马列学院作《论共产党员的修养》的演讲,该演讲包含对王明等的影射攻击,立时受到毛泽东的赞赏。毛称其演讲稿“提倡正气,反对邪气”,下令在中共中央机关刊物《解放》周刊发表。\footnote{吴黎平:《〈论共产党员的修养〉出版的前前后后》,载《怀念刘少奇同志》,页~291~—~92。}

由于得到毛泽东的鼓励,刘少奇对王明的态度更趋激烈和严厉。1941~年~7~月,身任新四军政委和华中局书记的刘少奇,在盐城配合乇泽东,也向国际派发起攻击。7~月~2~日,刘少奇作《论党内斗争》的演讲。7~月~13~日,刘少奇给宋亮(孙冶方)覆信,不指名地抨击王明等人。刘少奇顺应党内干部要求缓和党内斗争的心理,谴责王明等在党内人为制造斗争,“借用布尔什维克的名义和形式在党内进行投机”,是一伙嗜好斗争的“斗殴家”。刘少奇还公开指责不久前在皖南事变中遇难的项英等人,“依仗他们的部队和枪杆子——向党闹独立性,向党宣告独立。”

刘少奇敢说敢为,得到毛泽东的高度赏识,在毛的默许和支持下,一批主要由原北方局干部组成的刘少奇的部属在中共党内逐渐崛起。刘少奇原先在中共党内并无个人班底,直到~1936~—~1937~年刘少奇担任北方局书记期间,在刘的周围才逐渐形成一个圈子。1938~年后,随着刘少奇地位上升,刘少奇的老部下、原北方局组织部长彭真出任了由杨尚昆任书记的北方局下属晋察冀边区(后升格为中共晋察冀分局)书记一职。原北方局成员刘澜涛、黄敬、林枫、安子文、姚依林、胡锡奎、林铁、刘仁、李葆华、李大章、宋一平、刘锡五等分别担任了晋察冀、太行、晋绥等根据地党的领导职务。这样,在中共党内就出现了一个以刘少奇为核心、由彭真作骨干的刘少奇的班底。

刘少奇在一系列重大问题上坚定地站在毛泽东一边,但是在四十年代初,刘少奇对于毛的支持并不是完全无保留的,刘在个别重要问题上,仍持有自己独特的看法。

刘少奇是中共党内最早承认和宣传毛泽东贡献的人,但是在这一时期,刘少奇只是把毛视为是党的卓越的领导人,还未将其称之为“伟大领袖”。刘当然知道毛写过一系列论著,然而在~1941~年他却宣称,在中共党内,“伟大的著作还没有出来”。\footnote{刘少奇的这句话见于~1942~年~10~月~10~日出版的中共华中局内部刊物《真》第~2~期。六十年代初刘少奇对此文作了修改,将这句话删去。参见《答宋亮同志》,载《刘少奇选集》,上卷(北京:人民出版社,1981~年),页~218~—~20。}刘少奇的这番言论,与他~1943~年后竭力鼓吹对毛个人崇拜的论调,有如天壤之别。对于批判“教条主义”的问题,刘少奇与毛泽东也有微妙的差别。刘少奇没有如毛泽东那般,将教条主义视为头号敌人,而是认为在中共党内存在着“强调在实际斗争锻炼”、“反对专门理论研究的空气”。刘明确表示,将埋头学习马列理论的同志指责为“学院派”是“完全错误的”。刘甚至放言,“一直到现在”,中共“缺乏理论的弱点仍然未能克服”。\footnote{《答宋亮同志》,载《刘少奇选集》,上卷,页~218~—~20。}刘少奇的这些论调与毛泽东将留苏理论家视为“连猪狗都不如”、“蠢人”的的那些论调,显然是不协调的。

在对待“国际主义”的态度与保存中共实力等问题上,刘少奇与毛泽东的看法也存在一些差距。苏德战争爆发后,刘少奇、陈毅于~1941~年~7~月~12~日致电毛泽东,表示“如日本向苏联进攻,我们应号召全国向日寇反攻,即使国民党不积极反攻,我八路、新四亦必须独力反攻,以牵致(制)日本,敌后某些据点可能放弃”。7~月~18~日,毛泽东覆电刘少奇,提出:“反攻口号是对的”,“但八路、新四大规模动作仍不适宜,……大动必伤元气,于我于苏均不利。”\footnote{《刘少奇年谱》,上,页~362。}

尽管刘少奇在~1941~年对毛泽东仍然持有某种保留,两人在批评留苏派的方法上也存在差别,刘少奇在保存中共实力方面的意见也不如毛那般“老道”,但是双方的政治目标却是完全一致的。对于刘少奇某些不入耳的言论,毛泽东并没有予以深究,在毛最需要帮手之际,放眼党内,能和毛有最多共同语言的,惟刘少奇一人。刘批评经验主义的意见,只需毛泽东稍作调配,就可以制成进攻周恩来等的武器。因此,毛刘政治联盟并没有受到任何影响。

这样,当~1942~年~3~月,刘少奇把领导新四军和华中局的责任委托给饶漱石,于同年年底返回延安后,刘少奇就成为毛泽东领导整风运动的最重要的助手,刘少奇以政治局候补委员的身分立即进入中央书记处,成为在地位上仅次于毛的中共第二号领导人。

\section{毛泽东手中“出鞘的利剑”:康生}

如果说在毛泽东与王明的斗争中,刘少奇最早站在毛的一边。是因为刘少奇在一系列重大问题上的观点与毛一致,毛、刘在反对王明的基础上结成了政治联盟的关系,那么,康生与毛泽东的关系则完全是另一种性质。

康生是因其望风转舵,出卖王明,和竭力撮和毛与江青的婚姻,以及其所拥有的苏联“格伯乌”经验,而得到毛的特别信任和重用,继而成为毛手中的一把利剑。

毛泽东与康生素无渊源,1937~年~11~月底康生与王明等飞返延安之前,毛甚至从没见过康生。由于康生与王明曾有过的密切关系。甫抵延安的康生,并没有立即赢得毛的信任,毛、康由疏远至亲密。大约经历了半年的时间。

康生在中共党内最早以赵容闻名,1925~年在上海大学读书时参加中共,1926~—~1927~年,和周恩来、顾顺章等一同领导了上海三次武装起义。1927~年~10~月中共中央机关秘密迁到上海后,康生曾在中央组织部,江苏省委组织部工作。1931~年~1~月,被六届四中全会后的中共中央任命为中央组织部长。

早在三十年代初,康生作为中共“肃反专家”就声名显赫。1931~年~4~月,顾顺章事件发生后,中央特科的工作机构进行了重新改组。9~月,留沪的部分负责干部组成了有康生参加的六人临时中央政治局,中央特科从此就改由陈云、康生、潘汉年组成的中央特别工作委员会直接领导。指定康生参与领导中央特科,与他在上海三次武装起义期间曾经负责过工人“打狗队”有关。1932~年陈云调任全国总工会党团书记后,康生就成了中央特科的最高负责人。

从~1931~年末至~1933~年~7~月康生离沪赴苏,康生在上海的活动主要就是镇压中共变节分子,以及和国民党特务机关“中央组织部调查科”斗法。由康生领导的“红队”(取“红色恐怖”之意)刺杀了国民党“调查科”上海直属情报员史济美(马绍武)和一批共产党转向分子,在上海引起强烈震动。“红队”成员有王世英、项与年(八十年代中共福建省委书记项南之父)、邝惠安、李士英等人。以后,在五十至六十年代,王世英曾任山西省省长,李士英曾先后担任过上海公安局局长、最高人民检察院副检察长和江苏省副省长等职。

1933~年夏,康生因上海环境日趋恶化,无法在沪立足而转道前往莫斯科,担任中共驻共产国际代表团副团长,成为王明的副手。康生在苏联居留四年期间,竭力吹捧王明,同时实地学习苏联镇压机关格伯乌的经验,在中共留苏人员中历行“肃托”,\footnote{参见仲侃:《康生评传》(北京:红旗出版社,1982~年),页~47~—~51。5、57~—~60。}成为中共党内为数不多的,受过苏联“格伯乌”全面熏陶的“专业人士”。在~1935~年~8~月举行的共产国际七大上,康生被任命为共产国际中央执委会候补委员。

康生在~1937~年之前的经历表明,他和毛泽东没有任何工作和历史关系。因此,当康生返回延安后,毛泽东在一段时间里,和康生只是维持着一般的工作关系,毛辩言鉴貌,仔细观察康生的表现。在“十二月政治局会议”上,康生与王明、陈云一同进入了中央书记处,由此参与了核心决策,但是康生长期脱离国内斗争,不可能在书记处内拥有较多的发言权,他的具体工作则在中央职工委员会和“敌区工作委员会”。毛知道康生是党内老牌的情报和肃反专家,有意发挥其专长,但是,毛又让康生兼任中央职工委员会书记一职,在几乎没有产业工人的陕甘宁边区,此职无疑只是一个虚衔。此时,毛泽东对康生还不知虚实,只能用这一实一虚的方法试探、等待康生。

康生经过一段揣摩和悉心的识别,康生发现毛泽东虽然在政治局内暂时处于下风,但毛对延安的一切皆具有极大的支配力。和毛相比,王明在党内、军内并无坚实的基础,康生认定王明决非是毛的对手,中共领袖必属毛无碍。康生一经掂量、估算,马上采取行动,拿出过去的上司王明作投靠毛泽东的见面礼,以换取毛的信任和重用。

在~1938~年~3~月政治局会议上,康生小心翼翼地迈出了投向毛泽东的第一步。康生表态支持毛有关留王明在延安工作的意见。尽管毛泽东、康生、任弼时在会上提出的这项动议因大多数与会者的反对未获通过,但是,康生试探的目的已经达到,在毛泽东暂时处于少数的时刻。助了毛一臂之力,从而获得了毛的初步信任。

康生在“三月政治局会议”上投给毛泽东的一票,立即带来了立竿见影的效果。一个月后,康生被委任为中央党校校长,离开了冷冷清清的中央职工委员会。

从“三月政治局会议”至六届六中全会召开的半年时间里,康生加紧了和毛泽东的联系,康生和毛泽东、张闻天、刘少奇、陈云采取联合行动,以中央书记处的名义或他们五人联名的形式,不断向以王明为首的长江局提出批评性的意见。康生已愈来愈明显地和王明拉开了距离。

康生虽然已经逐渐获得毛泽东的青睐,但一时还未握有实权,不久,康生力排众议,全力支持毛泽东与江青结婚,从而一举获得毛的完全信任,康生的政治前途从此出现了重大的转折。

1939~年~2~月,毛泽东正式任命康生担任中央社会部部长兼情报部部长、敌区工作委员会副主任。康生成了中共情报和政治保卫工作的最高负责人。将此重责委之于康生,一则显示毛对康生的信任,二则分散了周恩来对情报工作的领导权(周恩来此时兼任敌区工作委员会主任)。

在毛泽东的庇荫下,康生迅速成了延安炙手可热的人物。以中共的“捷尔仁茨基”(苏联十月革命后肃反机关“契卡”的首任领导人)自居的康生,很快将延安的保卫机构分门别类建立和完善起来。社会部开始在延安各机关、学校秘密布设情报侦察网,吸收可靠党员担任“网员”。1939~年,中央社会部在延安“工作人员训练班”的基础上又创办了一个培养情报人员和肃反干部的秘密学校,对外称“西北公学”,康生实际上是该校的校长。

显而易见,康生之于毛泽东是极其重要的,以毛之多疑善变,对属下一向猜忌、防范的性格而论,和毛毫无历史渊源,又无战功和长征经历的康生,能长期获毛的信任是极其罕见的,其主要原因乃是毛、康关系的性质,完全不同于毛与其他中共领导人的关系。毛与刘少奇、任弼时等人的关系,从本质上说,主要是一种政治盟友的关系,而康生之于毛,则犹如家臣。康生对毛泽东的绝对效忠和人身依附,使毛可以轻而易举将康生与其他政治局委员区别开来。对于毛,康生曲意奉承,揣摩迎合,善于体会某些难言之隐而主动为主人分忧;康生又能雷厉风行,坚决贯彻主人意志而不畏毛以外的任何中共元老,实在是为人主者手中须臾不可离身的一把利剑。

经过几年的磨练,1941~年的康生早已铁定了心,做毛泽东的贝利亚(叶若夫被处决后的苏联最高特工头目)。此时的康生,威风凛凛,经常身着俄式皮夹克,足蹬长皮靴,手牵洋狗,每次外出,至少有四名保镖亲随,\footnote{司马璐:《斗争十八年》(全本)(香港:亚洲出版社,1952~年),页~69、74。}已成为延安最令人恐惧的人物。他犹如一头猛犬,随时准备听候主人的吩咐,向敌人猛扑过去。

\section{毛泽东的“内管家”:任弼时、陈云、李富春}

毛泽东所构思的整风运动包含了“破”和“立”的双重内容:一方面,毛要在这场党内斗争中剪除其政治上的对手;另一方面,毛还要创立以自己思想为中心的中共新传统,并将这两方面的成效有机地结合起来,最终为巩固自己的领袖地位而服务。如此艰巨的任务,绝非刘少奇、康生、陈伯达、胡乔木等少数人参与就能完成,毛还需要其他一些干部和工作机构来协助自己,他们应该运用某些机关的力量,坚决贯彻毛的意图,创设制度,培养新人,支持和配合毛的行动。在四十年代初,任弼时、陈云、李富春及其领导下的中央办公厅和中央组织部,实际上扮演的就是毛的“内管家”的角色。

中共是依照苏共模式组建起来的高度组织化的政党,在理论上,党的中央机关应是领导全党的最高司令部。1927~年后,中共逐渐转移至农村,长期艰苦的军事斗争,使军队在党内的作用明显加强,但是在留苏派主政的瑞金时期,中央机关及其领导人的权威仍是不容挑战的。长征开始,中共领导体制发生了深刻的变化,党和政府系统全部并入军队,军队与党实际已溶为一体,打破了党机关至高无上的局面。然而,随着长征结束和延安时代的到来,一度因客观战争形势而被取消的党机关又渐次恢复,特别到了抗战爆发,中共力量获得迅速发展,党的各种机构的工作任务极其繁重,为了适应新形势的需要,又新创立了一些党的机构。这样,就使由留苏派掌握的党机关的作用再一次凸现出来。

对于留苏派控制的党机关力量的复活,毛泽东保持着高度的警惕。一方面,毛不愿过分刺激张闻天等人,在形式上对中央机关的格局采取维持现状的态度;但在另一方面,毛又抓紧时机,利用自己在中央核心层已拥有的优势地位,果断地对党的重要机构进行局部调整。中央秘书处和中央组织部就是毛的局部调整战略的主攻目标。

1935~年后,在中共组织系列中,中央秘书处是除军队以外最重要的机构。中央秘书处在瑞金时期只有少量工作人员,由担任中央秘书长的邓颖超主管,主要负责文电收发(包括与共产国际及上海中央局的电台联络),保管、承担会议记录等秘书性质的工作。由于军事斗争在苏区占据头等重要的地位,中央秘书处主要围绕中央军委工作。当时担任中华苏维埃共和国中央执委会主席的毛泽东,仅有一个工作人员兼收发,其他中共领袖也无专门的秘书机构为之服务。

遵义会议后,毛泽东进入中央军委和中央常委会(书记处),中央秘书处局部恢复建制,其作用开始日益突出。1935~年~6~月,毛让自己过去的秘书王首道参加了两河口会议的秘书工作(王首道在瑞金时期曾一度担任毛泽东的秘书)。其后,毛任命王首道负责中央秘书处,并同时领导中央军委机要科。9~月,毛通过调离邓发、李克农的工作,停止了邓发所担任的国家政治保卫局局长和李克农担任的国家政治保卫局执行部部长的职务。

毛将原由政治局直接领导的国家政治保卫局易名为方面军政治保卫局,并派出王首道担任该局领导。此时的中央秘书处,除了掌管机要文电外,还兼管政治保卫工作。毛通过这些措施,将党、军队等全部机要通讯系统和肃反情报大权牢牢地掌握在自己的手中。

1937~至~1939~年,中央秘书处继续由王首道负责。王首道忠实地履行了他作为毛泽东执行秘书的职责,除了掌管文电收发、联络各方、照顾毛和其他在延安政治局委员的生活起居等各项工作外,王首道还亲自担任政治局会议的记录员。1937~年底,王明返国后召开的“十二月政治局会议”,即由王首道担任会议记录。毛泽东在这次会议上虽处下风,但是其权力并未受到真正动摇。会议结束时,毛命王首道收回所有与会者的笔记本,由王首道代为保管。而事实上,由王首道收回的与会者笔记本以后一直未退还本人。\footnote{1950~年~6~月~9~日在北京召开的中共七届三中全会通过《关于王明同志的决定》,责成王明必须对所犯历史错误作出书面检讨,王明在同年~8~月~17~日给毛泽东及中央书记处写信,要求查阅旧时笔记本和有关报刊。王明提出,1937~年~12~月会议散会时,主席叫王首道同志把各同志笔记本都收下,不准带出去。我的笔记本也被收下去了,以后并没有归还。参见王明:《中共五十年》,页~156。}

尽管王首道对毛泽东忠心耿耿,但在纷繁复杂的新形势下,中央秘书处的工作似乎难以适应新的环境和新的需要。抗战爆发后,原由中央秘书处兼管的政治保卫工作的职能又划归了边区保安处和继之成立的中央社会部,中央秘书处的职权范围有所缩小。自此以后,中央秘书处基本处于维持现状的局面。

为了加强中央秘书处作为毛泽东的耳目和中枢机构的作用,1939~年,毛泽东调其老友、时任中央组织部副部长的李富春兼管中央秘书处,由王首道作李富春的助手。这样,在~1940~年~5~月任弼时就任中央秘书处领导之前,李富春实际上成了中央秘书处的主管。

毛泽东对中央机关另一个重要部门——中央组织部,也不失时机,进行了改组。1935~—~1937~年,中组部部长由博古担任,在这段时期,中组部的工作基本处于停顿状态。因为在战时环境下,干部调配基本由军委掌握,红军抵达陕北后,外来人员极少,集中于陕北的都是经历过长征考验的红军指战员,中组部的日常业务中的一项,诸如对干部进行政治审查,已失去了工作对象。因此,作为中组部部长的博古,其主要工作岗位是在中央书记处而非在中组部。“七七事变”爆发后,大批外来人员进入延安,中组部面临大量的干部审查、分配等工作,持续几年的冷清局面立即改观,中组部成了延安最热闹、工作最繁重的部门之一。恰在此时,陈云随同王明、康生返回延安,博古又要前往长江局,毛泽东遂将中组部部长一职交由陈云担任,并派李富春任副部长,配合陈云开展工作。

陈云,原名廖陈云,一度与项英齐名,是中共党内少数出身工人阶级的高级领导人。1933~年初,陈云进入中央苏区,在~1934~年~1~月举行的中共六届五中全会上当选为政治局委员,并兼任白区工作部部长。1935~年~2~月,陈云受张闻天的委托,离开正在长征中的红军,秘密转赴上海,准备恢复上海党组织,并打通与共产国际的联系渠道。陈云抵沪后,与上海中央局负责人浦化人接上关系,从浦化人处了解到中共在沪组织已全部被国民党破坏,共产国际远东情报局已将工作人员撤出上海。陈云由此判断,恢复中共在沪活动的条件还未成熟。恰在此时,浦化人通过特殊渠道,接到中共驻莫斯科代表团要求国内派人组团前往苏联参加共产国际七大的通知,陈云遂于~7~月左右,在宋庆龄的帮助下,和杨之华、陈潭秋等,乘船秘密前往苏联海参崴。1935~—~1937~年,陈云在苏联期间,是中共驻共产国际代表团的主要成员,但与久居莫斯科从未返国的王明、康生不同,陈云受中共代表团指派,提前于~1937~年春返回新疆,组织营救进入新疆的西路军余部。陈云在星星峡亲自迎接了原西路军高级将领李先念等人。陈云~1937~年春的新疆之行,为他赢得了中共军方将领的广泛敬意。

陈云返回延安后,继续担任政治局委员,并且进入了书记处,但是他的主要工作岗位是在中央组织部。陈云在初抵延安的一个短时间内,曾和政治局绝大多数成员一样,支持王明传达的共产国际关于加强国共统一战线的方针,但是陈云回头较早,从~1938~年~3~月就转变了立场。1938~年~3~至~8~月,在毛泽东与长江局的对立中,陈云与在延安的中央书记处其他成员步调一致,坚决站在毛泽东的一边,成为毛的新盟友。

陈云、李富春领导的中组部,其主要工作是对抗战爆发后奔赴延安的各类人员进行严格的政治审查,根据审查结果,分配、安置他们的工作。陈云和李富春担当的是毛泽东在组织人事领域监护人的角色。

“七七事变”后,一批从国民党监狱释放出来的共产党员和大批青年学生纷纷前往延安,这种现象使党的领导人既喜又惧。喜的是,延安道上,人流如潮,足以说明党的事业兴旺发达,而面临大转变的党正急需大批青年干部;惧的是不知在这些来延人员中是否夹有国民党的暗探和奸细。为了保证来延人员的政治可靠性,中央决定对所有来延人员进行严格的政治审查,于是,这就成了中组部的头等重要工作。

1937~年底至~1938~年底,中组部设有干部科、地方工作科和秘书处三个下设机构,由王鹤寿任干部科科长,刘锡五任地方工作科科长,邓洁任秘书长,在这三个机构中,主要由干部科负责对来延安的干部进行政审和分配。

中组部对来延人员的政审分为四个阶段:

一、八路军驻南京办事处对释放出狱人员和自愿申请去延安的青年学生进行初审。1937~年~8~月底至~10~月,周恩来、叶剑英、李克农凭名单向国民党当局交涉,从南京国民党军人监狱和苏州反省院陆续营救出一百余名中共原负责干部。由首批出狱的黄文杰、刘顺元、刘宁一、王鹤寿、方毅、夏之栩等,在南京八路军办事处组成考察去延人员的审干小组,对申请去延人员进行逐个的政审。\footnote{参见刘顺元:《关于八路军驻京办事处的点滴回忆》,载《抗战初期的八路军驻南京办事处》南京:南京大学出版社,1987~年),页~69。}其工作流程是,出狱人员需写出个人狱中表现的书面材料,并向审干小组提供其他出狱人员在狱中表现的资料。审查小组根据本人的书面材料、口头叙述和其他人的旁证,对申请去延安的人员作出不同的处理:狱中表现坚定的人员,送延安或直接留国统区工作;狱中表现有些问题、需进一步审查的人员,也送延安;狱中表现不好,有叛变自首行为的人员,要其留下通讯地址,再动员其“回家去抗战”。\footnote{刘宁一:《1937~年八路军办事处向国民党交涉释放政治犯的情况》,载《抗战初期的八路军驻南京办事处》,页~71、78。}经过审查小组的遴选,从一千余被释放人员中,挑出七百多人介绍去延安。

对申请去延安的青年学生的审查,则比较简单和宽松。凡持有各地中共地下组织或外围组织介绍信的青年学生,在经过八路军南京办事处的逐个政审后,一般都予介绍去延安。

二、西安八路军办事处接到南京办事处转来的审干小组对去延人员的“鉴定表”后,对持介绍信前来的人,再进行一对一的个别谈话,此谓复审。然后将其集中分送西安附近的泾阳县云阳镇进行下一步的审查。也有个别人员是直接经西安转送延安的。

三、中组部在泾阳县云阳镇设有接待站(检查站),负责对去延人员进行严格的三审。领导审查小组的人员有冯文彬、王观澜、刘辑武、胡乔木以及刚从南京八路军办事处调来的王鹤寿等。在云阳镇的审查重点是盘问出狱人员在国民党狱中期间是否写过“悔过书”。1937~年~12~月,原中共驻青年共产国际代表黄药眠,在云阳镇就经历了这样一番严厉的审查:

\begin{quoting}
同我谈话的是刘辑武。他问我:“你写过悔过书没有?”我忠实地回答:“写过。但是我没有出卖组织。只表示以后不再搞政治了,也不拥护国民党。当时许多领导人都叛变投降,释放出去了,而我则被认为实质上是坚持共产党的立场,所以交军法处,叛处一等有期徒刑十年”。谈完以后,刘辑武就说,“你写个书面报告来吧!”一个星期以后,他就根据我的报告,表达了他的意见。并问我还有什么补充,还有什么不老实的地方,并说老实是至关重要的问题,希望我注意。最后,他还要知道一些当时被释放的人现在的情况。再过一星期,他就找我做第三次谈话。他说,他们决定把我送到延安去解决我的问题。当时我有点犹豫,我说:“还是给我遣散费,我回广东去好了。”他说:“不,你还是到延安去吧!”\footnote{黄药眠:《动荡:我所经历的半个世纪》,页~433。}
\end{quoting}在云阳镇接受审查的去延人员,一般也面临三种安排。“没有问题”,或“虽有问题,但不太严重”的释放人员,大多数被派往延安,接受下一阶段的审查,并等待分配工作。有较严重历史问题的人,则发给路费,劝其返乡或去大后方参加抗战。青年学生中的一部分则被留在云阳里的青年干部训练班(不久即迁至安吴堡)接受政治审查和政治训练,而在云阳镇接待站负责审查工作的冯文彬和胡乔木,同时又是云阳青训班的负责人。

通过云阳镇检查站政审的人员,前往延安的方式也有两种:社会知名度高、年龄较大的人员,可以乘大卡车前往;其他则一律步行前往,由云阳镇至延安约八百余华里,徒步需九至十天的时间。

四、去延人员抵达延安后,中组部很快派人前来谈话,再次对来延人员进行一对一的逐个审查,此是政审的第四个阶段。这时中组部已收到云阳镇检查站转来的材料,对来人的政治情况已基本掌握,于是中组部就依据转来的材料和每人的业务特长对来延人员进行工作安排。到了这一步,才算是正式进入了延安。

司马璐和黄药眠在延安中央组织部受到的不同接待,反映了来延人员原先的革命资历不是决定其今后政治命运的关键因素,而能否获得党的信任,完全取决于其个人在国民党监狱中的表现。\footnote{司马璐早年参加中共,1939~年~6~月因“托派”嫌疑被送出延安,第二年被恢复党籍,1942~脱离中共并参加了中国民主同盟,1949~年前往香港创办《展望》半月刊,逐渐成为香港著名的中共党史专家,编著有《中共党史暨文献精萃》。司马璐于八十年代迁往美国,现居纽约。}

在中组部的眼中,司马璐虽是一个参加革命不久的新同志,但他出身好,政治面目清楚。司马璐出身于苏北海安县贫苦农民家庭,其父在~1927~年大革命失败后被地方民团以“通匪”罪名杀害,在社会底层爬滚几年后,司马璐于~1935~年参加了中共在上海的外围组织,并加人了共青团。1937~年初,司马璐被党组织派往位于镇江的由国民党江苏省省主席陈果夫作后台的“私立江苏流通图书馆”开展地下工作。1937~年~4~月下旬,司马璐被捕。由于未暴露身分,在坐牢一个月零七天后被交保释放,旋即在上海加入了中共。“八一三”后,司马璐经西安八路军办事处和云阳镇审查小组的审查,辗转前往延安,因病在边区医院治疗数月后,于~8~年~3~月,在延安中组部受到陈云的亲自接见。

陈云向司马璐提出两个问题。第一、在镇江是否加入了国民党?第二、被捕后是否写过“悔过书”?在得到满意的答复,并仔细审阅过有关司马璐的书面材料后,陈云亲自分配司马璐前往延安最神秘的机构——位于枣园的中共“敌区工作委员会”,向康生的助手曾希圣(时任“敌区工作委员会”主任秘书)报到。\footnote{司马璐:《斗争十八年》(节本)(香港:自联出版社,1967~年),页~69。}

与司马璐相比,黄药眠虽是~1928~年加入中共的老同志,并曾出任过党的高级职务,但因黄药眠被捕后曾在狱中写过“悔过书”,从此就失去了党的信任。接待黄药眠的不是中组部部长陈云,而是中组部的一般工作人员。据黄药眠回忆:中组部工作人员对他谈了下面一番话:

“看了你的材料以后,我们认为你是在敌人面前屈服。这是你历史上的一个政治污点。关于党籍问题,以后再解决。现在先分配你到新华社去工作,作翻译。”

“我问他:‘你所说的党籍问题以后再解决,是恢复党籍,还是重新入党?’他的答复是:‘重新人党’。我心里想,所谓重新人党,就是十年党龄丢了。坐了三、四年监,还要留下政治污点,我从前也搞过党的工作,我是很懂得这句话的具体含义。”

黄药眠申辩道:“你说的留下政治污点,我是不能接受的,从我个人这件事,孤立地看,你这个估计是对的,但从当时的整个形势看,上海党、团中央局,几乎全部垮台,纷纷自首叛变的时候,就只有我一人判处十年徒刑,现在对我的处理,这不正如叛徒李一凡说的……国民党把你们当作共产党来军法处理,而将来共产党又会把你们作为叛徒处理吗?”

中组部工作人员回答说:“我没有说你是叛徒,只是说你在敌人面前屈服。现在是抗日战争时期,我们无法进行进一步的调查研究,所以暂时就这样吧,你先去上班吧!”。\footnote{黄药眠:《动荡:我所经历的半个世纪》,页~435~—~36。}

黄药眠的安置情况属于中组部分配的第一种类型:对有特殊专长的来延人员,虽有历史问题,但已基本审查清楚,可直接分配工作。黄药眠通晓英、俄语,曾在莫斯科作过翻译,因此被分配至新华社。可谓专业对口。

另一种分配类型是,将来延人员先分配到抗大、陕北公学或中央党校,经过一段时间的政治学习和审查后,再分配具体工作。建国后曾长期担任江苏省省长的惠浴宇~1937~年冬经武汉八路军办事处的介绍,到达延安。惠浴宇是~1928~—~1930~年江苏省海州地区党的领导人,1930~年在上海被捕。直至~1937~年才被国民党释放出狱。惠在狱中无任何“自首”、变节行为,他的这段历史已由在延安的当年狱中难友提出证明,因此惠浴宇在经历了一段审查后于~1938~年~4~月被分配至抗大总校,并被告知,他可以重新人党,惠浴宇就在这种情况下于~1938~年重新入党。以后惠浴宇在其他同志的鼓励下,给当年江苏省委负责人陈云写了一封信,请陈云为他~1928~—~1930~年的党籍作出证明。陈云接见了惠浴宇,与他进行了详细的谈话,最后肯定了惠浴宇的党籍,这样,惠的党籍就从~1928~年算起。不久,惠浴宇就从延安抗大总校被分配至皖南新四军总部。\footnote{惠浴宇口述,俞黑子记录整理:《朋友人》(南京:江苏人民出版社;1995~年),页~95~—~99。}

第三种分配类型是,对那些有较严重政治历史问题,一时又不易审查清楚,但在政治上表现左倾的人员,则劝其返回国统区活动。

陈云、李富春创立的干部审查制度,犹如一道坚固的拦河大坝,经过层层的冲刷,已尽可能将“可疑”分子排拒于延安之外。被允许留在延安效力的“有问题”人员,则尽在组织的控制与掌握中。在陈云、李富春的主持下,中组部的各项工作井然有序,有条不紊,与康生领导的肃反机关保持着密切的合作关系,但在~1938~—~1939~年,中组部的干部审查和安置工作,尚有若干灵活性,因而给许多去延人员留下不错的印象。

例如,李富春对因在延安备受冷遇、歧视而决定回到国统区的黄药眠就曾给予诚恳的挽留。1938~年春,分配到新华社工作的黄药眠因心绪不佳,病卧在床,而黄药眠的上级、昔日在莫斯科的旧友徐冰(邢西萍),则因黄药眠有“历史问题”,对黄十分冷淡,不仅对黄的生活和工作情况不闻不问,甚至将黄药眠退回了中组部。走投无路的黄药眠,只得写信向昔日的旧友周扬求助,在周扬的关照下,黄药眠一度“寄食”在边区文艺家协会。不久,心灰意冷的黄药眠向中组部提出申请,要求去国统区作文化工作,很快得到批准。正当黄药眠准备启程时,遇到了中组部副部长李富春。1929~年,黄在中共江苏省委工作期间,李富春是黄的上级。在李富春的热情挽留下,黄药眠改变主意,决定继续留在延安。然而黄药眠在中组部办交涉时仍受到冷遇,又使黄产生疑虑,觉得李富春“怎么会不知道(自己)已到了延安”?“是谁派工作人员和(自己)谈话的呢”?最后,黄药眠抱定“要革命也不一定要在延安”的决心,一气之下离开了延安。\footnote{黄药眠:《动荡:我所经历的半个世纪》,页~441~—~42。}

黄药眠因受不了中组部工作人员的冷落和歧视,负气离开延安,从他个人而言,是黄的自尊心和清高与党严格的审查制度不相合拍,没能经受住党的“考验”。其实,中共对从事白区工作的干部实行严格的审查,是一项广泛应用于任何人的政策,并非仅仅针对黄药眠,即使那些从未被国民党逮捕,地位较高的干部也得经受中组部或其它部门的“过滤”,杨子烈和王世英在延安受审查的情况就充分说明了这一点。

杨子烈是张国焘夫人,也是~1921~年参加中共的党内元老之一。杨子烈从二十年代初起就广泛参加了中共领导的一系列革命活动,她不仅是中共妇女运动领袖之一,还曾经两次赴莫斯科深造。1931~年春张国焘被派往鄂豫皖根据地,杨子烈被中共中央留在上海从事地下工作,1934~年中共在上海的组织遭国民党破坏后,杨子烈失去了组织联系,她先返乡避乱,又重返上海学产科,直到国共合作实现后才经南京八路军办事处介绍,带着儿子辗转来到延安。杨子烈归队后,最急切的要求就是希望中组部恢复她的党籍,尽管杨子烈是中共老干部,她的丈夫张国焘此时仍是中央政治局委员和边区政府代主席,但是,中组部还是要调查她与组织失去联系后的种种表现,同时要在工作中考验她。杨子烈被分配到边区政府做政治教员,她还义务在边区的中央医院做产科医师。尽管杨子烈工作热情积极,受到院长傅连章的高度评价,但她的党籍仍未能恢复。蔡畅当时和她丈夫李富春都在中组部工作,参加了对杨子烈脱党后一段历史的审查,虽然蔡畅与杨子烈是相识十多年的熟人,对杨子烈的过往历史十分了解,但也未能解决她的党籍问题。\footnote{勉之:《革命圣地承教泽》,载《延安马列学院回忆录》,页~146~—~47;另参见杨子烈:《张国焘夫人回忆录》(原名《往事如烟》(香港:自联出版社,1970~年)),页~344~—~45。}直到~1938~年~6~月,毛泽东批准杨子烈携子离开延安,前往武汉投奔张国焘时,她的党籍一直未能恢复。

如果说杨子烈是因其在地下工作时期曾失去党的关系,到延安后不被党信任,那么王世英一度受到党的冷落,则是因为中共党内对从事白区地下斗争同志存在着根深蒂固的歧视、怀疑的传统。

王世英是三十年代中共特科的重要成员,在康生赴苏联之前,长期在康生领导下从事政治保卫和反间谍的特工斗争。1935~年中共上海临时中央局遭国民党两次大破坏后,王世英率临时中央局转移至天津,继续开展秘密工作。1936~年后,王世英任中共北方局联络局(又称“中共华北联络局”)副局长,受北方局书记刘少奇派遣,代表中共与李宗仁和阎锡山等进行秘密联络,又以红军代表身分在太原主持秘密机关——红军驻太原办事处,为中共打开局面、拓展生存空间作出了重大的贡献。1937~年后,王世英代表中共驻太原和晋东南,与阎锡山交涉周旋,同时多方搜集阎方情报,直到~1938~年初返回延安。

王世英是中共高级干部,从未被国民党抓获,1936~年后,一直与延安保持电讯联系,按照常理,他从前方返回,应受到热情的接待和慰问。但事实是,王世英住在招待所里,“好多天没人过问,自己去找组织,也没人管”。\footnote{段建国、贾岷岫著,罗青长审核:《王世英传奇》(太原:山西人民出版社,1992~年),页~147。该书经原中共中央调查部部长罗青长审核,国家安全部情报史研究处原任处长贺若渊、规任处长谢建华等作了大量的审定工作,全书经国家安全部审定。}延安究竟发生了什么事?为什么对王世英这样重要的干部竟不闻不问?

王世英到延安的时候正是延安上层处在微妙变动的时期,1937~年底至~1938~年初,王明从莫斯科带回的意见一时在中央政治局占了上风,毛泽东被迫采取守势,然而王明等的返回并未真正动摇延安的政治格局,过去的一套制度仍在有条不紊地运转,陈云、李富春主持的中组部并没有一天停止工作。毛泽东或许太忙,一时顾不到王世英。但将随同王世英一同返回延安的部下抓起来,就无法依“常理”解释了。1938~年春,王世英的一个部下箫明被定为叛徒遭到逮捕,另一个部下刘雅洁则被驱逐出根据地,王世英的妻子李果毅在延安也被过去的熟人躲而避之。

王世英在延安被冷落归根到底只有一个原因:党组织对他存在怀疑。王世英虽是中共特科重要干部,也从未失去组织关系,但是,他不像前特科干部李克农、陈赓等人那样曾经进入中央苏区,经历过战争和长征的考验,在那个时期延安上层领导的思维中,只有参加过长征的干部才是可以信任的,对其他人的信任都要打个折扣。

王世英在延安被冷落长达四个月,毕竟王世英不是一般的白区干部,中共也需要王世英丰富的情报和统战经验,加之对王的审查也没有发现任何问题,毛泽东终于接见了王世英并听取了他的工作汇报。在毛的过问下,王世英进入了马列学院第一期学习,两个月后又被派往山西,担任八路军驻二战区办事处主任。

近似于王世英情况的还有贾拓夫。虽然他是唯一参加长征的西北地区党的元老,1937~—~1939~年担任中共陕西省委书记,但在他奉调回延安后,却被降为西北工委委员兼秘书。真正的原因乃是中央接到告发,对~1931~年贾拓夫被捕事产生了怀疑,从此开始了对他的秘密调查。1941~年~10~月~9~日,毛泽东致信贾拓夫:“你已知道,对你的怀疑是题中应有之义,这是对的,但我们现在已决定取消对于你的政治上的怀疑,恢复对于你的完全信任。”\footnote{周维仁:《贾拓夫传》,页~72。}随后贾拓夫被任命为中共西北局常委兼秘书长。

四十年代初,中央组织部的工作已全部纳入制度化的轨道。中直机关的干部由中组部及各直属单位干部科管理,军队干部统归军委总政治部管理,边区干部由边区党委组织部及以后的西北局组织部管理。从~1938~年底至整风运动前夕,中央组织部的规模又有了扩大,在原有的干部科、地方工作科和秘书处之外,新增了交通工作科(1940~年并入中央出版发行部)、总务处。中共中央并决定由中组部代管中央党务委员会,挂靠在干部科。中组部的工作人员也从原来的十多人,发展到六十余人。陈云、李富春运用机关力量,甄别、调配干部,力图使在延安的党员干部各得其所。陈云还为延安的干部作了《怎样做一个共产党员》的报告,要求新老党员忠实于党的路线和纪律。中组部成了毛泽东基本可以放心的后方基地。

和中组部情况相类似,中央秘书处在任弼时接任后,面貌也发生了巨大的变化。1940~年~5、6~月间,任弼时被政治局任命为中共七大筹委会秘书长,实际履行中共中央秘书长的职责(1941~年~9~月,任弼时才被正式任命为中共中央秘书长)。任弼时在中央秘书处的基础上,于~1941~年~9~月正式创建了中共中央的中枢机构——中央书记处办公厅,下设秘书、警卫、总务(行政)三个处,由任弼时兼任中央办公厅主任,李富春担任主持工作的副主任,实际履行中办主任的职责。中央办公厅除了负责机要文电、文件草拟、联络各地等幕僚性业务,它的另一功能就是为毛泽东和其他领导人提供生活服务。任弼时亲自制定了大、中、小灶干部待遇制度,将中共党内事实上存在的等级差序,用物资分配的形式具体体现出来,并使之进一步明确和固定化。延安的伙食制度并非始于~1940~年,早在~1937~年就已形成若干规定。1937~—~1938~年,延安的普通战士每人每天伙食标准为五分钱,一般干部为七分钱。枣园“中央敌区工作委员会”(即以后的社会部)的工作人员为一角五分,此是当时延安一般干部最高的伙食标准。军队团级干部和边区厅级干部的伙食标准为一菜一汤,师级和中央党机关部级干部为两菜一汤。政治局委员则为四菜一汤。

确定不同人员享受不同的物质待遇,是一个颇为复杂的问题。任弼时具体筹划,亲自决定享受小灶待遇的人选,明确规定只有中央委员或相当于中央委员的军政负责干部,才有享受吃小灶的资格。王若飞因做过陈独秀时代的中央秘书长,长期在党内遭到排斥,四十年代初他的政治地位并不高,只是中央党务研究室(名义上专管各根据地党的工作,实际上仅是一个政策研究机构)的负责人,因而被列人吃中灶的档次。为了保障高干的小灶食物供应和其它生活物资的需要,还开辟了从各根据地和国统区调配物资进入延安的供应渠道。\footnote{据师哲回忆,为了满足江青要穿用宁夏滩羊皮制作皮衣、皮裤的要求,中央办公厅运用保安处的外勤,冒着危险,去马鸿逵统治的宁夏去采买。江青要吃阿胶,中央办公厅通过关系去山东采购,经香港和中共驻重庆办事处才辗转送到延安。参见师哲:《在历史巨人身边——师哲回忆录》,页~169。}延安高干供应制度的建立,对于正在形成的以毛为中心的体制具有重大意义,此制度的作用,不仅在于它能够在物质匮乏的条件下确保对党的高级干部的物资供应,更在于它可以在敏感的“价值”和“承认”问题上,直接打击党内小资产阶级知识分子(王明最热烈的听众)自视清高的傲慢。从此,中共历史上曾经有过的罗曼蒂克式的平均主义时期已经结束,中共已进入到强调等级差序的新时代。

1941~年,任弼时、陈云、李富春和毛泽东的关系已经完全确定,他们所领导的部门成为在政治上支持毛的重要阵地。此时,任弼时已是在延安地位仅次于毛的中央书记处常务书记;陈云则以政治局委员,书记处书记的身分,领导中央组织部、中央青委和泽东青年干部学校;李富春虽非政治局委员,但他的实际权力却大大高于许多政治局委员。作为毛的故交,李富春所担任的中央副秘书长、中央办公厅副主任和中组部副部长等职,使他成为延安少数几个与毛关系最密切的人物之一。在即将到来的毛泽东与王明的交锋中,任弼时、陈云、李富春将忠实履行他们作为毛泽东盟友的职责,全力拥戴和支持毛。

\section{扶植地方实力派:高岗的崛起}

在毛泽东集合中共党内各方面力量,向以王明为代表的国际派发起进攻时,高岗作为陕北地方红军的代表,是毛急欲争取的另一个重要人物。在整风运动前夕和整风期间,高岗从一地方党和军队领导人,一跃成为党内的显赫人物,他的崛起,完全依赖于毛的悉心栽培和提拔。

高岗之所以被毛泽东看中,是毛的主观需要和高岗所具有的特殊优势互相结合的结果。在陕北干部中,唯有高岗具备毛泽东所需要的各项素质和条件,因而成为毛泽东提拔、重用的对象。

1935~年~10~月中央红军初抵陕北,毛泽东急需陕北党和红军的支持,以帮助中共中央在陕北立下脚跟。高岗是陕北地方红军中较大的一支——刘志丹部的主要骨干,刘志丹在陕北地区具有广泛的影响,当刘志丹于~1936~年“东征”阵亡后,高岗就成为原刘志丹部的主要领导人。为了显示中央红军与陕北地方红军的团结一致,毛泽东必须从当地红军中挑选出代表人物,给予适当的安排,以巩固中央后方,而高岗正符合这个条件。

高岗在党内斗争中的经历,也是毛泽东任用高岗的一个重要因素。中共在陕北的党组织因历史的因素和战争环境造成的彼此分割,长期未能实现统一。这种情况造成陕北干部之间一直存有隔阂和不和。1935~年~2~月,中共在陕北的两个组织:陕北特委和陕甘边特委召开联席会议,成立了中共西北工作委员会和西北革命军事委员会。统一领导陕北、陕甘边两块根据地和两支红军,由陕甘边方面的刘志丹担任西北军委主席,高岗担任副主席。在西北军委下设立前敌总指挥部,也是由刘志丹任总指挥,高岗任政委。\footnote{雷云峰等:《陕甘宁边区大事记述》(西安:三秦出版社,1990~年)。页~50~—~51。}

1935~年~7~月,原左联成员朱理治以上海中央局驻北方代表的身分,由北方局派出,到达陕北,开始依靠原陕北特委以郭洪涛为首的一批干部,8~月,上海中央局代表聂洪钧到达陕北,组成以朱理治为书记的“沪局与北局派驻陕北苏区代表团”,成为中共在陕北的最高领导机构。9~月中旬,徐海东、程子华率红二十五军长征到达陕北,程子华也参加了以朱理治为首的代表团。朱理治还改组了中共西北工作委员会和西北军委,任命聂洪钧为西北军委主席,成立中共陕甘晋省委,由朱理治、郭洪涛任正副书记。此时,刘志丹、高岗虽受到朱理治、郭洪涛的排斥,但仍未被完全剥夺权力,刘志丹担任了由红二十五军和陕北地方红军红二十六军、红二十七军组成的红二十五军团的副军长兼参谋长,高岗任政治部主任,但刘、高很快就被卷人到肃反狂潮中。在~1935~年~9~至~10~月间,朱理治以中央代表的身分,指使西北保卫局局长戴季英等,在红二十五军团发动肃反,逮捕了刘志丹和高岗等人。

1935~年~11~月~3~日,毛泽东、张闻天、博古在听取了前来迎接的程子华、郭洪涛、聂洪钧的报告后,下令对刘志丹、高岗、习仲勋等暂缓处理,并立即派王首道全权处理刘、高案件。经过以博古为首的中央党务委员会审查陕北肃反五人委员会的复查,宣布对刘志丹、高岗、习仲勋、马文瑞等平反,并给陕北肃反的直接责任者聂洪钧、戴季英以党纪处分。受过旧中央“错误路线”的打击,又得到毛泽东解救的高岗,可以十分自然地接受毛对旧中央政治路线的批判,成为毛向王明等进攻的一名主攻手。

高岗受到毛泽东重用的另一个原因是高岗身上的乡村小知识分子气质。三十至四十年代在毛泽东周围的中共领导人,大多数都有留苏或在国内大城市求学受教育的经历,这批人在性格和气质方面,往往和出身农家而从未出过洋的毛泽东多有不合,而高岗的气质则与毛泽东有较多的亲和性。高岗原名高硕卿,陕西榆林县人,初等师范毕业,1927~年大革命失败后,在中共陕西地下省委负责人贾拓夫和北方局代表孔原(陈铁铮)的领导下,在西北地区长期从事兵运和参加领导中共地方武装的工作,在一批文化程度较低的陕北武装同志中间,高岗和刘志丹是其中少数具有一定的理论和政策水平的干部之一。高岗从没出洋留学,也没去过上海、北平、南京等国内大城市,和留苏派毫无瓜葛。在极端艰苦的环境下,高岗协助刘志丹,独当一面,屡败屡起,始终保持住一支数百人的红军队伍和一块根据地,足以说明高岗颇有韬略而非“教条主义者”之流。高岗在个性上,既有小知识分子的能言善道,又兼有农民无产者的狡黠和粗鄙,他尤其鄙薄在非武装部门工作的知识分子干部,和当时许多党的领导干部对知识分子表示尊敬、重视有所不同,高岗对知识分子的态度极为轻慢,这些都使毛泽东感到高岗与自己在气质上有某种相似,而易引起毛的好感。

1937~年抗战爆发前夕,高岗在毛泽东的栽培下,已逐渐在陕北地方干部中脱颖而出。1937~年~5~月~1~日,高岗被指定为中共陕甘宁特委常委(1938~年~1~月,陕甘宁特区政府又恢复了陕甘宁边区政府的名称,特委会相应改为边区党委)。9~月,中共中央指定高岗等七人为陕甘宁边区政府主席团成员。1938~年~10~月,高岗以陕北党组织代表的身分参加了中共六届六中全会,这是高岗第一次参加中共的重要会议,表明高岗政治地位的上升。六中全会刚闭幕,高岗就正式取代了郭洪涛,担任中共陕甘宁边区党委书记,这是一个地位十分重要、类似于中央局书记的职位。中共元老王若飞因在政治上失势,多年屈居于高岗之下,担任边区党委宣传部长。

毛泽东为了支持高岗的工作,特意将多年来与高岗不和的郭洪涛、朱理治调开。1935~年~11~月后,毛为了立足陕北,对原陕北党领导人长期未予触动。郭洪涛除担任陕甘宁省委和边区党委书记,还被中央任命为中组部副部长,一直到~1938~年~10~月。1938~年~11~月后,毛将郭洪涛派往山东,任命郭为中共山东分局书记。一年后,郭洪涛被调回延安。朱理治在~1938~年后被任命为中共中原局副书记,负责开辟鄂豫边根据地,并担任了新四军鄂豫挺进纵队政委,与任司令员的李先念齐名,朱理治从此和李先念、陈少敏、陶铸等建立了密切的工作和个人关系。但是毛不愿看到朱理治在开疆辟土中建功立业,1940~年也把朱理治调回了延安。两年后,朱理治成了毛泽东、高岗开刀祭旗的第一个牺牲品。

在战争年代,中共干部在党内获得地位,最重要是看其在党的武装斗争中所作的贡献。而给不给干部领导武装斗争的机会,则要看毛泽东对这个干部是否信任。毛泽东既可以给予干部机会,也可以不给,甚至给了某个干部这个机会,还可以收回。朱理治的情况就属于最后一种,贾拓夫和孔原的情况则属于另一种。贾拓夫是高岗的老上级,1934~年前往中央苏区瑞金参加第二届全国苏维埃代表大会,一度作过陈云的助手,担任过中央白区工作部秘书(相当于副部长),后随中央红军长征到达陕北。尽管贾拓夫身为中共西北地区的元老,但贾拓夫从未被毛泽东委以军队职务,到延安后,贾拓夫在党内的地位长期在高岗之下。孔原原先也是高岗的上级。孔原在随陈云于~1935~年夏赴苏联前,长期领导中共北方局,朱理治前往陕北,即由孔原派出。然而孔原从没担任过一个战略区独当一面的军政领导职务,其在党内地位也一直在高岗之下。毛泽东对高岗的态度则完全是另一种情况。1938~年,毛泽东任命高岗担任领导陕北地方部队的陕甘宁保安司令部司令员。随着毛对高岗信任的进一步加深,1939~年~6~月,毛又任命高岗担任了八路军留守兵团的政委。由萧劲光任司令员的八路军留守兵团,下辖三个旅和二个警备(保安)司令部(王震的三五九旅即归八路军留守兵团统辖),是守卫边区唯一重要的军事力量,毛泽东将与自己素无渊源的高岗派任如此重要的职务,显示了毛对高岗的特殊信任。

在毛泽东的扶持和关照下,高岗在边区和军队中的地位迅速加强。1940~年~7~月~11~日,中央政治局将陕甘宁边区党委升格为中共陕甘宁边区中央局,任命高岗为书记。1941~年~5~月~13~日,中央书记处又将边区中央局与中央西北工委统一为西北中央局,由高岗任书记,此时高岗在党内的地位已高于党的元老、陕甘宁边区政府主席林伯渠,而和周恩来、刘少奇等大局书记平起平坐了。

高岗在边区党和军队中地位的确立,加速了一个以高岗为中心的西北地方干部系统的形成。因历史和工作关系而与高岗接触密切的干部,例如习仲勋、马文瑞、刘景范(刘志丹之弟)、张秀山、张邦英、王世泰等分别担任了边区党、政机构的负责人。而在历史上曾和高岗有过矛盾的干部,则受到高岗的排斥。中共西北武装斗争最早参加者和领导者之一的阎红彦因多次向延安有关部门揭发高岗在~1932~年~6~月临真镇战斗中临阵脱逃的旧事而遭到高岗的打击。\footnote{临真镇位于延长县西南的八十里处,是一个山区小镇。为执行陕西省委攻打韩城的命令,1932~年~6~月~3~日,刘志丹领导的陕甘游击三支队向临真镇发起进攻,担任三支队二大队政委的高岗在战斗危急中,带领十余人,临阵脱逃,使战斗“前功尽弃,转胜为败”。战后,队党委决定开除高岗的党籍,并下令通缉高岗,后高岗返队,受到刘志丹的批评和“留党察看”的处分。参见毕兴、贺安华:《阎红彦传略》(成都:四川人民出版社,1987~年),页~59~—~60、110;112~—~13。}阎红彦因此于~1938~年和~1940~年两次被调离工作岗位派往马列学院和八路军留守兵团军政研究班带职学习。毛泽东、刘少奇等其他中共领导人清楚知道阎红彦反映的有关高岗历史“污点”的全部事实,但对高岗的信任仍一如往常,这也完全符合毛一贯的用人之道,即“历史问题”可大可小,关键要看跟什么人,站在什么线上,只要大节无亏,在政治上又对自己有大帮助,具体“小疵”可一笔带过。反之,若在政治上不和自己站在一边,即使无任何历史问题,也会被弃之一旁。十余年以后,高岗事发,毛泽东将阎红彦昔日揭发高岗的旧事重提,并让阎红彦在中央会议上揭发高岗的“反党阴谋”,\footnote{参见毕兴、贺安华:《阎红彦传略》(成都:四川人民出版社,1987~年),页~59~—~60、110;112~—~13。}毛之翻手为云,覆手为雨,足见一斑;只不过这已是后话了。高岗对于毛泽东的提拔和重用心存无限感激,他清楚知道,若无毛的鼎力相助,自己绝不会成为边区党的领导人。为了巩固自己“西北王”的地位,进而在党内谋求更大的发展,高岗除了全力支持毛泽东,别无任何选择。头脑灵活的高岗几乎不需要点拨就可发现毛泽东在党内的头号对手是王明,他要在对王明的态度上,向毛表明自己的立场。1941~年夏,中央政治局分工王明指导中共西北局,王明在短时间内曾经过问西北局和边区政府的工作。据《谢觉哉日记》记载,1941~年~8~月~24~日,王明曾在边区政府谈粮食问题\footnote{《谢觉哉日记》,上,页~335。}。

高岗对王明虽然表面客气,但却在毛泽东面前讲王明的坏话,他对毛说:“原来我们以为苏联飞机给我们带来什么好东西,却不知道这是祸从天降。”\footnote{参见师哲:《在历史的巨人身边——师哲回忆录》,页~166。}高岗用这句话,向毛献上了自己的忠心。毛泽东看准了高岗全部的心理活动,用地位、权力、名誉将高岗紧紧拴住。毛将把高岗作为反对王明、博古的一门钢炮来使用。毛深信,在向王明等的进攻中,高岗将随时听从自己的召唤。

\section{重新调整与毛泽东的关系:处境尴尬的军方}

中共军队在毛泽东发动的整风运动中处在一个十分微妙的地位:一方面,军队是毛依靠的最重要的力量;另一方面,军队的某些主要干部又是运动的整肃对象。军队所处的尴尬境地,使军队领导人左右为难,进退失据,在经历一番痛苦的抉择后,才重新调整并适应了与毛的关系。

从二十年代后期中共有了自己的军队开始,在一个相当长的时期内,中共军队内部一直存在两种力量的微妙平衡。第一种力量姑且名之为“红色军事专家派”,这一派的首领是周恩来。属于周恩来系统的军事干部由三方面人员组成:一、二十年代末至三十年代初,从苏联军事院校学成返国,经由当时担任中共中央军事部长的周恩来分配至中央苏区和其他苏区的干部;二、出身黄埔军校,参加南昌暴动被打散后,再由周恩来分配至中央苏区和其他苏区的干部;三、1931~年底周恩来进入中央苏区后,与周恩来密切配合或在周领导下的军事干部。中共军队内部的第二种力量可以称之为“井冈山派”,这一派的首领是毛泽东。属于毛泽东系统的军事干部则由四部分人员组成:一、跟随毛泽东上井冈山,参加秋收暴动的农军和武汉国民政府警卫团的余部;二、1928~至~1931~年聚集在毛周围的赣南、闽西地方红军;三、1928~年~4~月随朱德、陈毅上井冈山的参加南昌暴动的余部;四、1928~年~7~月平江暴动后上井冈山的彭德怀部。

由周恩来和毛泽东分别代表的中共军队中的这两股力量各有其特点。“红色军事专家派”的内部关系较为松散,派系色彩比较淡化,周恩来的个人感召力和周在中共党内的革命历史,是维系这一派军事干部的主要动力。由于“红色军事专家派”的许多干部曾在苏联学习过,他们一般对苏联和共产国际有较深的感情。“井冈山派”的内部关系则比较复杂,毛泽东固然是这一派当之无愧的领袖,但是由于毛的专断性格与朱德、陈毅多有冲突,在一个时期内,毛的权威受到朱、陈的抑制。1929~年后,毛利用各种手段强化了自己在“井冈山派”中的领袖地位,并在自己身边聚集了一批军事干部,但却遗留下大量的个人恩怨。到了周恩来抵达中央苏区后,一批对毛不满的军事干部重又聚拢在周恩来的周围,使得毛一时形单影只,处境颇为窘迫。

中共军队内的两股力量在周恩来抵达瑞金后,经过周的精心调和,在艰苦的战争环境下,逐步融为一体。周恩来、朱德、刘伯承等不放过任何机会,努力争取改善与毛的关系,至于一般高中级军事干部更不存在藩篱之隔。

周恩来极为注意利用党的权威加强军队内部的团结,在周的领导下,军队一直保持高度的统一。遵义会议后,毛泽东从现实需要出发,也深感运用党的权威对维系军队团结的极端重要性,因此军队内部的这种团结局面在遵义会议后仍然维持。1935~至~1936~年,毛与张闻天、周恩来、博古等密切合作,运用中共中央的权威,处理并最终解决了张国焘的分裂问题。1937~年春,随着西路军的最终失败,中共军队的统一基本完成。

具有讽刺意味的是,在中共军队完成统一时,作为党中央军委主席的毛泽东,其个人对军队的控制却相对减弱了。1937~年秋以后的一段时期内,毛泽东对八路军的指挥一度失灵,毛对新四军更是鞭长莫及。由项英指挥的新四军,对王明、周恩来领导的长江局言听计从,而与毛则貌合神离,这一切都引致毛对军队领导人的强烈不满。

毛泽东对军队领导人的不满,并非仅限于他们在抗战后的表现,这种不满还和历史上的矛盾复杂地交织在一起。毛是一个自尊心极强、报复心也极强的人,只是因为眼前的现实需要,他才容忍下他们昔日对自己的冒犯。在中共军队内真正获毛信任的干部屈指可数。

毛泽东最宠信的军队将领首推林彪。自~1929~年林彪在朱、毛纷争中当面指责朱德、积极拥戴毛泽东以后,林彪就获得了毛的特殊信任。毛之对于林彪,犹如父亲,对其过失从未真正计较。遵义会议后,林彪认为毛率红军四渡赤水,使红军过于疲劳,写信给中央要求以彭德怀取代毛作军队指挥。毛不责怪林彪,却将怒火发向与此毫无关联的彭德怀。毛看重林彪英勇善战,战功卓著,对林彪的倚重和爱护超过任何军事将领。1938~年~6~月,林彪遵毛的指示,赴苏联治疗,直至~1942~年~2~月~8~日才返回延安。当林彪返回延安时,毛更亲自迎接,使在场的许多人惊愕不已。因为~1940~年周恩来和朱德从外地返回延安时,毛泽东都不曾亲自迎接。1942~年~2~月~17~日,中共中央在延安为林彪举行盛大欢迎大会,与会者达千余人。林彪在致词中援引季米特洛夫的话说,“季米特洛夫说:苏联的党,由于团结在斯大林同志的周围,而有今天伟大的联共党,中国的党,应该团结在毛泽东同志的周围,以便建设起伟大的中国党,建设起伟大的新民主主义的新中国。”在这次欢迎大会上,林彪还表态坚决支持整风运动,号召全党应效忠毛泽东。他说,“我们在政治思想,应如毛泽东同志最近所号召的,反对主观主义和宗派主义。要来一个坚决彻底的转变,建设无产阶级立场的唯物主义方法的党……我们忠实于我们的民族,忠实于我们的党,忠实于我们的领袖。”\footnote{《解放日报》,1942~年~2~月~18~日。}林彪以青年将领的身分受如此超规格的隆重欢迎,且已会搬出共产国际领导人来鼓吹毛泽东,这只能说明毛、林关系之密切和林彪已获毛的“路线交底”。1942~年底至~1943~年~7~月,毛泽东命林彪代表自己与蒋介石在西安、重庆数度会面,随后又让林彪在延安静养,以备将来与蒋介石逐鹿中原,再分天下时领兵出山。

毛泽东对彭德怀则爱恨交加。彭是为中共夺取政权出力最多的将领之一,但为人耿直,不善逢迎。毛爱其骁勇善战,对中共事业忠心耿耿,但又恨其“不听话”,自尊意识、独立意识强烈,而常视其为头有反骨的魏延。1937~年后,毛让彭出任八路军副总指挥,但彭却在~1937~年~12~月至~1938~年~3~月间,为八路军出兵山西及华北事,多次打电报向王明领导的长江局请示汇报工作。在毛看来,彭德怀此举无疑是另寻党内靠山,而与自己分庭抗礼。其实,毛对彭德怀的意见多属误解。1937~年底,八路军出兵山西,涉及与阎锡山和国民党的各种关系,彭德怀向当时党内实际主管统战和国共关系事宜的长江局请示应对方针,完全是在工作范围之内的正常关系。但是毛却认定彭对自己权威并未真正心悦诚服,执意要让彭德怀在整风运动中洗一个烫水澡。

作为“红军之父”的朱德,在中共军内的地位十分尴尬,在毛泽东的眼中,声望卓著的朱德实际上是无足轻重的。自从~1929~年朱德就军内民主化问题与毛发生争论并遭失败后,朱德就被笼罩在毛泽东的身光之下,在制定重大政策方面很少发挥影响。朱德性格敦厚,为人随和,虽然对毛大权独揽、独断专行不乏意见,但为了维护“朱、毛”团结一致的形象和中共的最高利益,对毛一向忍让,从不与毛公开对立。朱德和周恩来、彭德怀也维持着良好的工作和个人关系,即使对王明、博古等新一代党的领导人,也多持善意的态度。王明返国以后,朱德对加强中央的集体领导抱有希望,但是很快就传来了季米特洛夫支持毛为中共领袖的“口信”。朱德对毛表示支持,同时也委婉地向毛进言,希望毛能够察纳雅言。1938~年~9~月~26~日,朱德在为听取王稼祥传达季米特洛夫口信而举行的政治局会议上发言,他说:“党内同志要实行正确的自我批评,党员要维护对党的领袖的信仰。因此领导同志要有接受批评的精神。领袖要听人家说自己的好话,同时还要听说自己不好的话。”\footnote{参见《朱德年谱》,页~198。}朱德的这番话,实际上将他对毛泽东的微妙态度曲折地表达出来。1940~年~5~月,朱德从太行山八路军总部返回延安,其个人对前方八路军的影响已完全被切断。尽管朱德对毛已无任何妨碍,但毛仍对朱德怀有某种戒心。作为中共的一种象征,朱德虽继续享有八路军总司令的崇高荣誉,但中共所有军政大权均在毛的掌握中。

毛泽东对于刘伯承、聂荣臻、朱瑞的态度是有亲有疏,区别对待的。刘伯承、聂荣臻、朱瑞都是具有留洋经历、且与周恩来有较密切历史关系的高级军事干部,属于以周恩来为代表的“红色军事专家派”。他们自三十年代初陆续进入中央苏区后,在周恩来的领导下,担负着保卫中央苏区沉重的任务,对当时党的路线方针不起重大作用。刘伯承、聂荣臻等在瑞金时期,与毛泽东的关系虽不密切,但也无任何个人冲突。遵义会议后,刘伯承在对待毛的态度上,既高度尊重,又不失个人尊严。抗战开始后,刘伯承、聂荣臻、朱瑞等分别被赋予开辟几个战略区的重任,都取得了重大进展。在上述三人中,毛对聂荣臻较为亲近,而对刘伯承、朱瑞则相对疏远。聂荣臻在抗战后,比较努力执行毛的指示,在召兵买马、开疆辟土中成绩显著,因而获得毛的赞赏。毛对刘伯承的情绪更多由历史因素造成。刘伯承在瑞金时期,积极仿效苏联红军正规化的经验,又长期担任红军总参谋长,在宁都会议上,刘伯承曾赞成苏区中央局的意见,兼之刘伯承严谨的红色军人的气质,都在使毛对刘伯承产生一种疏离感。朱瑞是中共六届五中全会选出的中央候补委员,作为瑞金时期一名出色的“红色指挥员”,在抗战之初的用人之际,被毛泽东委之为中共山东分局书记,但毛又对这位毕业于莫斯科克拉辛炮兵学院的留苏生放不下心,不久即派自己的老部下罗荣桓出师山东。山东中共武装力量长期未能实现统一指挥,开疆拓土不尽顺利,个中原因十分复杂,但毛却认定山东局面未尽理想的责任全在朱瑞这个“教条主义者”的身上。

毛泽东对于早年出身绿林豪杰、与自己毫无渊源的贺龙倒是颇为欣赏和信任的。在毛的眼中,贺龙是属于不喜读书的莽张飞一类人物。加之,贺龙在毛与张国焘的对立中坚定地站在自己的一边,对毛的新权威表示充分的尊重。因此,贺龙不是毛在整风运动中要触及的对象。

毛泽东对原红四方面军总指挥徐向前的态度是颇含深义的。西路军失败后,毛让徐向前留在延安,而将原四方面军的将领划归刘伯承、邓小平领导的一二九师指挥,只给徐向前一个副师长的头衔。1939~年,毛命徐向前以八路军第一挺进纵队司令员的身份,带领百余人的队伍进入山东,一年后旋即将徐向前召回延安。返回延安后,毛任命徐向前为留守兵团副司令员。徐向前除了偶而参加会议,基本处于休养状态。毛泽东仍在继续观察徐向前。

在整风前夕,以“参座”著称于中共党和军队的叶剑英在中共上层的地位是颇为微妙的。叶剑英是中共为数不多的军人政治家,曾在苏联学习。1931~年进入中央苏区后,叶剑英长期在周恩来领导下,从事红军总参谋部的工作,属于“红色军事专家”。红军长征结束后,叶剑英协助周恩来开展对西北军、东北军的统战谋划,以后又是长江局、南方局的主要成员之一,基本上已脱离了军队系统,直至~1942~年返回延安,重新参与军委幕僚。尽管叶剑英既不统兵打仗,也无个人的干部班底,但他有留苏、和王明、周恩来合作的经历。因此,“教条主义”和“经验主义”都和叶剑英沾上边。由于叶剑英并非决策人物,在长征期间毛与张国焘的斗争中,曾助毛一臂之力,因此叶剑英将在整风运动中受到一定的触及,而非急风暴雨式的批判。

毛泽东在中共军中最不信任的对象是项英。1941~年初皖南事变后不久,项英遇难,毛在军中最大的障碍已经排除。可是毛对新四军代军长陈毅也心存芥蒂。毛长久不能忘记~1929~年陈毅与朱德合作反对自己的往事。毛需要看到陈毅对当年这段历史公案的新认识,因此,陈毅也将是延安整风运动的触及对象。

由此可见,在四十年代初毛泽东与中共高级将领错综复杂的矛盾中,历史因素和毛的个人好恶占据很大的比重。这些矛盾又和毛泽东与王明、周恩来的矛盾紧密地交织在一起。毛为了一并解决这些矛盾,“毕其功于一役”,巧妙地运用自己身兼党、军领袖的双重身分,使自己处于任何人也无法反对的地位。

毛泽东在军队高级干部面前,经常以党领袖的面目出现。毛告诫他们必须时刻牢记“我们的原则是党指挥枪,而决不允许枪指挥党”,实质是提醒军方,不得无视毛的个人权威,必须无条件服从毛。抗战爆发后,毛沿用中共治军的传统方法,并赋于新的内容。毛将忠实于自己的党的高级干部派往几个大战略根据地担任政治委员。

由邓小平在太行山协助刘伯承,由刘少奇的老部下彭真在晋察冀协助聂荣臻。由刘少奇、饶漱石在华中协助、监督陈毅。这些皆是毛旨在巩固自己对军队领导的重大战略安排。在更多的场合,毛泽东又以军方代言人自居。他警告党的高级领导人,“有了枪确实又可以造党”,“延安的一切就是枪杆子造出来的,枪杆子里面出一切东西”,\footnote{毛泽东:《战争和战略问题》(1938~年~11~月~6~日),载《毛泽东选集》第~2~卷(北京:人民出版社,1952~年),页~511。}公开羞辱手无一兵一卒的王明、博古等。毛以军队为后盾,逼国际派交权,使王明等步步退却,无任何招架之力。毛泽东的第二副面孔,符合军方的利益,有利于扩大军队在中共党内的影响。因此,尽管军队一些主要领导人本身也是整风运动所要触及的对象,但是毛仍可以获得中共军队对整风运动的支持。

于是,摆在中共军队高级领导人面前的路只有一条,平静地等待即将到来的整风运动的风暴,从速调整与毛泽东的关系,全力支持毛泽东为中共最高领袖。