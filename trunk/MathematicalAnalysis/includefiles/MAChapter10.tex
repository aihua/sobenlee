%# -*- coding:utf-8 -*-
%%%%%%%%%%%%%%%%%%%%%%%%%%%%%%%%%%%%%%%%%%%%%%%%%%%%%%%%%%%%%%%%%%%%%%%%%%%%%%%%%%%%%
%%  MAChapter10.tex'


\chapter{广义积分}\label{ch:10}

\section{无穷积分的概念}
\begin{exercise}
\item 求下列无穷积分。
\begin{exlistcols}
  \item $\int_0^\pinf\dfrac x{(a^2+x^2)^3}\dif x$;
  \item $\int_0^\pinf\dfrac{\dif x}{x(1+x^2)}$;
  \item $\int_0^\pinf\dfrac{\ln x}{(1+x)^2}\dif x$;
  \item $\int_0^\pinf\dfrac{\dif x}{1+x^3}$;
  \item $\int_0^\pinf\me^{-x}\cos\alpha x\dif x$;
  \item $\int_0^\pinf\me^{-x}\sin\alpha x\dif x$;
  \item $\int_0^\pinf\dfrac{1+x^2}{1+x^4}\dif x$;
  \item $\int_0^\pinf\dfrac{\dif x}{1+x^4}$;
  \item $\int_0^\pinf\dfrac{\sqrt x}{1+x^2}\dif x$;
  \item $\int_0^\pinf\dfrac{\dif x}{(2x^2+1)\sqrt{1+x^2}}$;
  \item $\int_0^\pinf\dfrac x{(x^2+a^2)x^{\sfrac32}}\dif x$;
  \item $\int_0^\pinf\dfrac{\dif x}{(1+x^n)\sqrt[n]{1+x^n}}$。
\end{exlistcols}
\item 求下列无穷积分。
\begin{exlistcols}
  \item $\int_0^\pinf x^n\me^{-x}\dif x$;
  \item $\int_0^\pinf\dfrac{\dif x}{(a^2+x^2)^n}\mcond{a>0}$;
  \item $\int_0^\pinf\dfrac{\dif x}{(ax^2+2bx+c)^n}$,其中~$n\in\mN$,且~$ac-b^2>0$。
\end{exlistcols}
\item 对任意~$x\in\mintc a\pinf$,有~$f(x)\leq h(x)\leq g(x)$,并且~$\int_a^\pinf f(x)\dif x$~与~$\int_a^\pinf g(x)\dif x$~收
敛。证明广义积分~$\int_a^\pinf h(x)\dif x$~收敛。
\item 若~$f(x)$~在~$\mintco a\pinf$~上单调下降,且积分~$\int_a^\pinf f(x)\dif x$~收敛。证明~$\lim_{x\to\infty}xf(x)=0$。
\item 若~$f(x)$~在~$\mintco a\pinf$~上单调下降,且~$\lim_{x\to\pinf}f(x)=0$,同时~$f'(x)$~在~$\mintco a\pinf$~上存在。证明广义积分
~$\int_a^\pinf f(x)\dif x$~收敛当且仅当~$\int_a^\pinf xf'(x)\dif x$~收敛。
\item 证明
\[
  \frac12-\frac1{2\me}<\int_0^\pinf\me^{-x^2}\dif x<1+\frac1{2\me}。
\]
\item 若~$f(x)$~为周期函数,并且~$\int_\minf^\pinf f(x)\dif x$~收敛。证明~$\int_0^xf(t)\dif t\eqcolon F(x)\equiv0$,$x\in\mR$。
\end{exercise}

\section{无穷积分的收敛性判别法}
\begin{exercise}
\item 判别下列无穷积分的敛散性。
\begin{exlistcols}
  \item $\int_0^\pinf\dfrac{x^2}{x^4-x^2+1}\dif x$;
  \item $\int_1^\pinf\dfrac{\dif x}{x\sqrt[3]{1+x^2}}$;
  \item $\int_0^\pinf x^p\me^{-x}\dif x\mcond{p\geq0}$;
  \item $\int_1^\pinf\dfrac{\ln x}{x^p}\dif x$;
  \item $\int_1^\pinf\dfrac{\ln^nx}{x^2}\dif x\mcond{n\in\mN}$;
  \item $\int_0^\pinf\dfrac{\sin^2x}x\dif x$;
  \item $\int_0^\pinf\dfrac{x^p}{1+x^q}\dif x\mcond{p,q>0}$;
  \item $\int_0^\pinf\dfrac{\cos ax}{1+x^n}\dif x$;
  \item $\int_0^\pinf\dfrac{\dif x}{1+x\mabs{\cos x}}$;
  \item $\int_1^\pinf\mparenbb{\ln\mparenB{1+\dfrac1x}-\dfrac1{1+x}}\dif x$;
  \item $\int_1^\pinf\ln\mparenB{\cos\dfrac1x+\sin\dfrac1x}\dif x$;
  \item $\int_0^\pinf\dfrac1{x^2}\ln\mparenbb{1-\dfrac{\sin^2x}2}^{\msp -1}\dif x$;
  \item $\int_0^\pinf\dfrac{P_n(x)}{Q_m(x)}\dif x$,其中~$P_n(x)$~与~$Q_m(x)$~分别为~$n$~次及~$m$~次多项式,且~$Q_m(x)\neq0$。
\end{exlistcols}
\item 讨论下列无穷积分的敛散性及绝对收敛。
\begin{exlistcols}
  \item $\int_0^\pinf\dfrac{\cos^2x}x\dif x$;
  \item $\int_1^\pinf\dfrac{\cos x}x\dif x$;
  \item $\int_0^\pinf\dfrac{\sin^2x}x\dif x$;
  \item $\int_1^\pinf\dfrac{\cos x}{x^p}\dif x\mcond{p>0}$。
\end{exlistcols}
\item 设~$f(x)$~单调下降趋于零,且~$f'(x)\in C\mintco 0\pinf$。证明~$\int_0^\pinf f'(x)\sin^2x\dif x$~收敛。
\item 设~$f(x)\in C\mintco0\pinf$,且~$\int_0^\pinf f(x)\dif x$~条件收敛。证明,无穷积分
\[
  \int_0^\pinf\max\mrangeb{f(x)}0\dif x,\quad
  \int_0^\pinf\min\mrangeb{f(x)}0\dif x
\]
发散。
\item 设~$\int_a^\pinf f(x)\dif x$~与~$\int_a^\pinf f'(x)\dif x$~收敛。证明~$\lim_{x\to\pinf}f(x)=0$。
\item 设~$f(x)$~在~$\mintco a\pinf$~上非负且单调下降,又设~$\int_a^\pinf g(x)\dif x$~收敛。证明,
\begin{exlist}
  \item $\int_a^\pinf f(x)g(x)\dif x$~收敛;
  \item 存在~$\xi\in\mintc a\pinf$,使得
  \[
    \int_a^\pinf f(x)g(x)\dif x=f(a)\int_a^\xi g(x)\dif x 。
  \]
\end{exlist}
\end{exercise}

\section{瑕积分的概念}
\begin{exercise}
\item 计算下列瑕积分。
\begin{exlistcols}[3]
  \item $\int_0^a\dfrac{\dif x}{\sqrt{a-x}}$;
  \item $\int_0^ax\sqrt{\dfrac x{a-x}}\dif x$;
  \item $\int_\alpha^\beta\dfrac{\dif x}{\sqrt{(x-\alpha)(\beta-x)}}$;
  \item $\int_0^{\frac\pi2}\cos x\ln\sin x\dif x$;
  \item $\int_0^1x^n\ln^nx\dif x$;
  \item $\int_0^{\frac\pi2}\sqrt{\tan x}\dif x$;
  \item $\int_0^1\ln x\dif x$;
  \item $\int_0^{\frac\pi2}\dfrac{\sin x}{a^2\sin^2+b^2\cos^2x}\dif x\mcond{a,b\neq0}$。
\end{exlistcols}
\item 设函数~$f(x)$~在区间~$\mintoc 01$~上是单调函数,且在~$x=0$~的邻域内无界。证明,若~$\int_0^1f(x)\dif x$~收敛,则
\[
  \lim_\ntoinf\frac1n\sum_{k=1}^nf\mparenB{\frac kn}=\int_0^1f(x)\dif x 。
\]
\item 求极限~$\lim_\ntoinf\dfrac{\sqrt[n]{n!}}n$。
\item 设~$f(x)\in C\mintco ab$,而~$\lim_{x\to b-0}f(x)=\pinf$,同时~$\int_a^bf(x)\dif x$~收敛。又设~$g(x)$~在~$\mintc ab$~上
非负可积。证明,
\begin{exlist}
  \item $\int_a^bf(x)g(x)\dif x$~收敛;
  \item 若~$\int_a^bg(x)\dif x=0$,则~$\int_a^bf(x)g(x)\dif x=0$;
  \item 存在~$\xi\in\mintc ab$,使得
  \[
    \int_a^bf(x)g(x)\dif x=f(\xi)\int_a^bg(x)\dif x 。
  \]
\end{exlist}
\item 证明,
\[
  \frac\pi{2\sqrt2}<\int_0^1\frac{\dif x}{\sqrt{1-x^4}}<\frac\pi2 。
\]
\end{exercise}

\section{瑕积分收敛性判别法}
\begin{exercise}
\item 判别下列瑕积分的敛散性。
\begin{exlistcols}[3]
  \item $\int_0^1\dfrac{\dif x}{\ln x}$;
  \item $\int_0^1\dfrac{\ln x}{1-x}\dif x$;
  \item $\int_0^1\ln x\ln(1-x)\dif x$;
  \item $\int_0^\pi\dfrac{\dif x}{\sqrt{\sin x}}$;
  \item $\int_0^{\frac\pi2}\dfrac{\dif x}{\sin^px\cos^qx}$;
  \item $\int_0^1x^\alpha\ln x\dif x$;
  \item $\int_0^1\dfrac{1-\cos x}{x^m}\dif x$;
  \item $\int_0^1\dfrac{x^{p-1}-x^{q-1}}{\ln x}\dif x$。
\end{exlistcols}
\item 判别下列积分的敛散性。
\begin{exlistcols}[3]
  \item $\int_0^\pinf\dfrac{\dif x}{\sqrt{x(x-1)(x-2)}}$;
  \item $\int_1^\pinf\ln\mparenbb{1-\dfrac1{x^2}}^{-1}\dif x$;
  \item $\int_0^\pinf x^{p-1}\me^{-x}\dif x$;
  \item $\int_0^\pinf\dfrac{\arctan^qx}{x^p}\dif x$;
  \item $\int_0^\pinf\dfrac{\ln(1+x)}{x^p}\dif x$;
  \item $\int_0^\pinf\dfrac{\dif x}{x^p+x^q}$;
  \item $\int_1^\pinf\dfrac{\dif x}{x^p\ln^qx}$;
  \item $\int_0^\pinf\dfrac{x^m}{x^n+x^p}\dif x$。
\end{exlistcols}
\item 讨论下列积分的敛散性与绝对收敛性。
\begin{exlistcols}
  \item $\int_0^\pinf\dfrac{\sin^px}{1+x^q}\dif x$;
  \item $\int_0^\pinf\dfrac{\sin^px}{x^q}\dif x$;
  \item $\int_0^\pinf\sin x^2\dif x$;
  \item $\int_1^\pinf\ln^px\dfrac{\sin x}x\dif x$。
\end{exlistcols}
\item 判别下列积分的敛散性。
\begin{exlistcols}[3]
  \item $\int_0^\pinf\sin x^p\dif x$;
  \item $\int_0^\pinf\dfrac{\cos x}{x^p}\dif x$;
  \item $\int_0^\pinf\dfrac{\sin x}x\me^{-x}\dif x$;
  \item $\int_0^\pinf\dfrac{x-\mfloor x-\dfrac12}{x^p}\dif x$;
  \item $\int_0^\pinf\dfrac{\sin x\cos\dfrac1x}{x^p}\dif x$;
  \item $\int_0^\pinf\dfrac{\cos x\sin\dfrac1x}{x^p}\dif x$。
\end{exlistcols}
\item 设~$f(x)$~在~$\minto ab$~上连续,且~$\int_a^bf^2(x)\dif x$~收敛。证明~$\int_a^b\mabsb{f(x)}\dif x$~收敛。
\end{exercise}

\begin{exercise*}
\item 证明,
\[
  \max_{s\in\mintc01}\int_0^1\mabsb{\ln\mabs{s-t}}\dif t=1=\ln 2 。
\]
\item 证明,
\[
  \int_1^\pinf\mparenbb{\frac1{\mfloor x}-\frac1x}\dif x=\lim_\ntoinf\mparenB{1+\frac12+\dotsb+\frac1n-\ln n}。
\]
\item 设~$f(x)$~当~$x\to\pinf$~时,单调下降地趋于零。证明下列积分同时收敛,
\[
  \int_0^\pinf f(x)\dif x,\quad\int_0^\pinf f(x)\sin^2x\dif x 。
\]
\item 设~$\int_a^\pinf f(x)\dif x$~收敛,且~$xf(x)$~在~$\mintco a\pinf$~上单调下降。证明,
\begin{exlistcols}
  \item $xf(x)\geq0$,$x\in\mintco a\pinf$;
  \item $\lim_{x\to\pinf}xf(x)\cdot\ln x=0$。
\end{exlistcols}
\item 已知对任意的~$x\in\mR$,有
\[
  \lim_{p\to\pinf}\int_{x-p}^{x+p}f(t)\dif t=\phi(x)。
\]
证明,
\[
  \phi\mparenB{\frac{x_1+x_2}2}=\frac12\mparenb{\phi(x_1)+\phi(x_2)}。
\]
\item 设~$f(x)$~在~$\mintco 01$~上有连续的导数,且
\[
  \mabsb{f'(x)}\leq\frac{M}{(1-x)^{1-\alpha}}\mcond*{0<\alpha\leq 1}。
\]
证明,$f(x)$~在~$\mintc 01$~上满足
\[
  \mabsb{f(x_1)-f(x_2)}\leq\frac M\alpha\mabs{x_1-x_2}^\alpha 。
\]
\item 设~$\phi(x)$~连续,且~$\lim_{x\to\pinf}\phi(x)=1$。记
\[
  \psi(x)\coloneq\int_x^\pinf\dfrac{\phi(t)}{t^{1+\alpha}}\dif t\mcond*{\alpha>0}。
\]
证明,$\psi(x)\sim\dfrac C{x^\alpha}\mcond{x\to\pinf}$,并确定~$C$~的值。
\item 设~$\phi(x)=x-\mfloor x-\dfrac12$。证明,
\[
  \lim_{x\to\pinf}x\int_x^\pinf\dfrac{\phi(t)}{t^2}\dif t=0 。
\]
\item 设~$F(x)\coloneq\int_0^x\mparenbb{\dfrac1t-\mfloorbb{\dfrac1t}}\dif t$。证明~$F'(0)=\dfrac12$。
\item 设~$f(x)$~在~$\mintco0\pinf$~上连续,且~$f(\pinf)\coloneq\lim_{x\to\pinf}f(x)$~存在。证明,
\[
  \int_0^\pinf\dfrac{f(ax)-f(bx)}x\dif x=\mparenb{f(0)-f(\pinf)}\ln\frac ba\mcond*{b>a>0}。
\]
\item 设~$f(x)$~在~$\mintco a\pinf$~上一致连续,且~$\int_a^\pinf f(x)\dif x$~收敛。证明~$\lim_{x\to\pinf}f(x)=0$。
\item 设
\[
  \rho(\xi)=\frac1\pi\frac y{(\xi-x)^2+y^2},
\]
其中~$\xi,x$~为任一实数,而~$y$~为正实数。证明,
\[
  \int_\minf^\pinf\txts\smbsqrt{\mabs{\xi-x}}\,\rho(\xi)\dif\xi=\smbsqrt{2y}。
\]
\item 设~$x,y>0$~且~$x\neq y$。证明,
\[
  \smbsqrt{xy}<\frac{x-y}{\ln x-\ln y}<\frac{x+y}2。
\]
\end{exercise*}




\endinput
%%
%% End of file `MAChapter10.tex'.