%# -*- coding:utf-8 -*-
%%%%%%%%%%%%%%%%%%%%%%%%%%%%%%%%%%%%%%%%%%%%%%%%%%%%%%%%%%%%%%%%%%%%%%%%%%%%%%%%%%%%%

\chapter{从“延安之春”到斗争王实味}

\section{利用自由主义打击教条主义:延安“自由化”言论的出笼}

1942~年的初春,春寒料峭,毛泽东整顿三风的报告犹如一股春风,将延安知识分子干部的心吹得暖烘烘的。各机关、学校纷纷办起壁报,人们踊跃著文,响应毛泽东的号召,向主观主义、教条主义、宗派主义发起猛烈的攻击。就在这时,毛泽东开始了一个重大行动:清剿“教条主义的大本营”\footnote{师哲:《在历史的巨人身边——师哲回忆录》,页~246。陈明(丁玲的丈夫)曾在延安中央研究院的前身马列学院学习。他在回忆中提到马列学院曾被一些人指责为“教条主义大本营”。参见陈明:《回忆与怀念》,载《延安马列学院回忆录》,页~314~—~15.}——延安中央研究院。

延安中央研究院(以下简称中研院)是中共为培养高级理论干部而设立的一所“红色教授学院”,它的前身是马列学院,这是国际派的一个世袭领地,自~1938~年~5~月~5~日成立,即长期由张闻天兼任院长,副院长一职则由中共马克思主义经济学开山人之一、日本马克思主义理论大师河上肇的弟子王学文担任。张闻天和王学文是中共党内极少见的掌握数门外语、精通马克思主义理论的宿耆,但是毛泽东对张闻天、王学文领导下的充满浓厚理论学习气氛的马列学院并无好感。1940~年~5~月,王学文因长期在日本留学,熟悉日本情况,被调到总政治部任敌工部部长,王学文虽然继续挂名副院长,但张闻天在工作中已失去一重要助手。1941~年~5~月,马列学院改名为马列研究院,7~月,又易名为中央研究院。毛将马列研究院改名的目的是为了淡化马列作为中共原型的形象,以凸显中共的民族主义色彩,此举也是为最后摧毁斯大林伸向中共的这块精神租界地尽早作好舆论和心理上的准备。

1942~年~1~月,中宣部部长兼中研院院长张闻天,主动下乡进行社会调查,在临行之前,张闻天去中研院讲话,他“以庄重的口吻说毛泽东同志的学习态度和学习方法始终是扎扎实实的,脚踏实地的,理论密切联系实际的,是全党学习的楷模”,号召大家要“老老实实”地向毛学习。张闻天随即话锋一转,开始自我贬损,声称:“我没有什么值得学习的,我不过是一个缺乏实践的梁上君子罢了。\footnote{江围:《难忘的岁月》,载《延安马列学院回忆录》,页~103。}毫无疑问,张闻天这番话会一字不差”地报到毛泽东那里,张闻天用这种方式向毛传递了他彻底认输的信息。张闻天离开延安后,中宣部代部长凯丰以“带罪之身”奉命领导中央文宣部门的整风运动,但他极为知趣,不肯过问中研院的运动。罗迈(李维汉)则在毛的支持下,以中宣部副部长的身份坐镇中研院。罗迈系中共元老,也是毛青年时代的朋友,但曾一度依附国际派得罪过毛。1935~年后,罗迈审时度势,积极靠拢毛,更在~1941~年~9~月政治局会议上对自己所犯的“错误”作了“深刻的检查”,已在一定程度上获得了毛的谅解。1942~年毛泽东派罗迈主持中研院整风运动,就是要在政治斗争的“风口浪尖”上,进一步考察他。对此,罗迈心领神会,1942~年~3~月~16~日,他在《解放日报》发表《要清算干部教育中的教条主义》的署名文章,指责~1938~年后的中共干部教育,造成了“两耳不闻窗外事,一心只读马列书”的风气,将矛头直指领导中央文宣和干部教育工作的张闻天。罗迈十分清楚,毛正不动声色地注意着自己,他也知道毛早已派陈伯达挂职于中研院,陈伯达这个“包打听”将会把中研院的大小事情迅速禀告于毛。

毛泽东派罗迈前往中研院的目的十分明显:揭开中研院的“盖子”,先打掉院内那批留苏、留东、西洋的“理论权威”的傲气,再进而将斗争矛头直指“教条主义的祖师爷”——中研院院长张闻天和王明、博古。

中研院被毛泽东选作整肃国际派的重点单位绝不是偶然的。1938~年后,经张闻天亲自挑选,马列学院集中了一批中共著名的学者和理论家:担任中研院各研究室负责人的张如心、王思华等皆有留苏或留欧美、留日背景,且均为二十年代末和三十年代初入党的老党员;各研究室的一级研究人员也多是一时之选,其中大多数人都是三十年代左翼文化运动的风云人物,有的人还有译著出版。但由于后一类人入党时间较晚,且大多不曾留苏,所以在政治上的地位并不高。

毛泽东为了在心理上彻底打垮中研院内这批马克思主义理论家的意志,交替使用了两种手段:第一,在政治上和人格上公开羞辱他们;第二,挑动青年知识分子的不满情绪,“放火烧荒”,将青年的怒火引至国际派身上。

出身于长沙第一师范的毛泽东有着极其强烈的自尊心,对所谓“大知识分子”一直怀有根深蒂固的不信任感。他从来就怀疑党内那批理论家在内心深处并不承认自己,毛也猜度这批人甚至根本就否认中共党内有“教条主义”一说,他们完全可能以中共绝大多数党员没读过马列著作为由,拒绝接受“教条主义危害论”。毛泽东知道,对付这笔“红色教授”,仅仅用说理辩论那一套纠缠不清的“文明的方式”显然是不够的,最有效的方法就是将他们臭骂一通,使他们在劈头盖脸般的责骂中,斯文扫地,无地自容。

对付“红色教授”,毛泽东自有其法术,其中最厉害的一着就是剥掉他们引以自豪的“理论家”头衔,赐他们一个“坏透了的留声机”的恶名。

秉持毛泽东的旨意,《解放日报》在社论中嘲笑那些以熟读马列为奇货可居的“红色教授”,自以为“谁背的书最熟,谁就是最好的理论家”,其实是在闹“天大的笑话”。社论声色俱厉地警告他们,必须自我脱帽(“理论家”之帽),并以无可质疑的权威口吻教训这些“理论家”:

\begin{quoting}
会引证马列主义之警句的人不能称为理论家,能以马列主义精神方法解决实际问题的人,才能称为理论家。\footnote{参见《解放日报》~1942~年~2~月~2~日社论《整顿学风、党风、文风》,这是配合前一天毛泽东在中央党校所作同名报告而发表的重要社论,文章极有可能经过毛泽东的亲自修改。}
\end{quoting}
为了一举打掉“红色教授”身上的傲慢,毛的得力助手胡乔木在《解放日报》上著文,干脆直接将他们喻为“废物”。当然,这样的骂人话实在太难听了。于是,胡乔木将“废物”从轻发落,改称为“坏透了的留声机”(“因为他们决没有把所见所闻背得一字不差的本领”)并且宣称共产党内这种人“多的是”\footnote{参见《解放日报》~1942~年~3~月~9~日社论《教条与裤子》,此篇社论由胡乔木起草,经毛泽东亲自修改,可视为毛泽东、胡乔木的共同作品,已收人~1992~年~5~月人民出版社出版的《胡乔木文集》第~1~卷。胡乔木在该卷序言中特别说明,他在四十至五十年代写的评论,绝大多数经过毛泽东的亲笔修改。}!

毛泽东的凌厉攻势果然有效,中研院的“红色教授”们个个吓得胆战心惊,忙不迭地表态支持反教条主义。但是,毛却对他们思想的迅速转变十分怀疑,因为这些“理论家”并未将“教条主义”与他们个人联系起来,而是“你也来呀,我也来呀,大家把主观主义、宗派主义、党八股的尾巴割下来呀”,好似“教条主义”与他们无关。对这批“理论家”随风转舵毫不脸红的行径,毛泽东、胡乔木表示出极大的愤怒,他们借《解放日报》之口挖苦道:“他们叫得愈多愈响,就愈成为讽刺。任是什么漂亮的盒子,一触到他们的指头,就都变为顽石了”。\footnote{《教条与裤子》,载《解放日报》~1942~年~3~月~9~日;另参见《胡乔木文集》,第~1~卷(北京:人民出版社,1992~年),页~48。}

其实“红色教授”们已经够可怜了,仅仅在一夜间他们就从凤凰变成了草鸡。他们忍痛宣布自己原先视为安身立命之本的马列著作“比屎还没有用处”,\footnote{参见中央研究院中国经济研究室主任王思华的《二十年来我的教条主义》,载中国社科院新闻研究所中国报刊史研究室编:《延安文萃》,上(北京:北京出版社,1984~年),页~152。}这些昔日的理论家从最良好的愿望出发,尽可能地去理解毛的“新解释”,他们已对毛表现出最恭顺的服从,所未做的只是将自己骂得狗血喷头,而阻碍他们达到这一步的则是个人的尊严感,然而毛泽东所要索取的正是这份个人的独立与尊严!


1942~年~3~月~9~日,《解放日报》发表了一篇经毛泽东修改,由胡乔木撰写,堪称中国言论史之奇文的社论《教条和裤子》。\footnote{台湾中央研究院近代史所研究员陈永发在其《延安的阴影》一书中称,“目前文献中,可以看到,最早使用(‘脱裤子,割尾巴’)这个说法的是~1943~年~4~月的陕甘宁边区整风总结”,此说有误。“脱裤子,割尾巴”的说法,最早出现在~1942~年~3~月~9~日的《解放日报》刊载的《教条与裤子》的社论;随即迅速传播开来;成为整风运动中使用频率最高的一组词汇。这组词汇的发明人,现在也可以确定,他们不是别人,正是胡乔木和毛泽东。参见陈永发:《延安的阴影》(台北:台湾中央研究院近代史研究所,1990~年),页~317.}这篇社论以大胆使用粗俗文字于政治斗争,开创了中共文宣语言的新范式。毛泽东、胡乔木通过《教条与裤子》的社论,责令“红色教授”们“脱裤子”:为什么“脱裤子”呢?因为“问题发生在他们的贵体下”,还因为裤子下“躲着一条尾巴,必须脱掉裤子才看得见”。而各人“尾巴的粗细不等”,割尾巴所需用的“刀的大小不等,血的多少不等”,因而只有先脱掉裤子,才谈得上“用刀割”,“裤子上面出教条,这就是教条和裤子的有机联系,谁要是诚心诚意地想反对教条主义,那么他第一着就得有脱裤子的决心和勇气”\footnote{《教条与裤子》,载《解放日报》~1942~年~3~月~9~日;另参见《胡乔木文集》,第~1~卷(北京:人民出版社,1992~年),页~48。}。

然而毛泽东对“主观主义、教条主义的大师们”能否自觉“脱裤子”并不抱奢望,他决定采取一项重大措施——动员青年知识分子帮助“红色教授”和老干部“脱裤子”。

毛泽东摆出一副青年知识分子保护人的姿态,频频发表同情他们的言论,将延安的青年知识分子一步步诱人早已为他们准备好的进攻堑壕。

在毛泽东言论中最具鼓动性的内容是他对“宗派主义”的解释。1941~年~9~月~10~日,毛在政治局扩大会议上把“宗派主义”主要解释为排斥、歧视知识分子的“首长本位观”,而不是他以后惯所喻指的“王明、博古宗派集团”!

毛泽东说:

\begin{quoting}
宗派主义现在也有。在延安,首长才吃得开,许多科学家,文学家都被人看不起。宗派主义是排挤非党干部的一种风气,即排外主义。同时也排内\footnote{3~毛泽东:《反对主观主义和宗派主义》(1941~年~9~月~10~日),载《文献和研究》,1985~年第~1~期。}。
\end{quoting}1942~年~2~月~2~日,《解放日报》在社论中遵照毛的上述口径,批评宗派主义“对于党内同志则轻视疏远,少团结,少帮助。对于党外干部,则少了解,少关心。对‘三三制’的实行,又是不坚决、不彻底”。\footnote{参见《解放日报》~1942~年~2~月~2~日社论《整顿学风、党风、文风》,这是配合前一天毛泽东在中央党校所作同名报告而发表的重要社论,文章极有可能经过毛泽东的亲自修改。}一时间,毛泽东似乎在鼓动延安青年知识分子“反官僚”!

3~月中下旬,毛泽东的言论向更“开明”的方向发展,《解放日报》连续发表反宗派主义的社论。3~月~14~日,《解放日报》刊载社论《从自己制造的囚笼中跳出来》,抨击某些共产党员的“孤立主义”错误,宣称党和党外人士的关系问题,是党的生死问题,也是革命的成败问题。3~月~19~日,又发表《发扬民主作风》的社论,再度批判一部分党员排外的宗派主义情绪,要求中共党员“虚怀若谷”,“倾听各种不同意见”。毛在为中共中央起草的《关于共产党员与党外人士的关系的决定(草案)》中亲笔写道(这份决定因王实味事件的发生,以后并没有下发和公布——引者注):

\begin{quoting}
任何愿与我党合作的党外人员,对我党和我党党员及干部都有批评的权利。除破坏抗战团结者的恶意攻击以外,一切善意批评,不论是文字的,口头的或其它方式的,党员及党组织都应虚心倾听。正确的批评,应加接受,即使其批评有不确当者,亦只可在其批评完毕,并经过慎重考虑之后,加以公平的与善意的解释。绝对不可文过饰非,拒绝党外人员的批评,或曲解善意批评为攻击,而造成党外人员对党的过失缄口不言的现象。党外人员对于违犯政府法令或党的政策的党员及干部,除得向法庭或行政机关依法控诉外,并有权向各级党委控告,直到党的中央。

《新华日报》、《解放日报》及各抗日根据地的报纸刊物,应吸取广大党外人员发表言论,使一切反法西斯反日本帝国主义的人都有机会在我党党报上说话,并尽可能吸收党外人员参加编辑委员会,使报纸刊物办得更好。党报工作者必须学会善于吸引党外人员在党报上写文章,写通讯的方式和方法。某些党报工作者的主观主义与宗派主义态度,须受到批评。\footnote{毛泽东起草的《关于共产党员与党外人员的关系》的决定,仅是一纸文字,在当时及以后数十年间一直没有发表,求实出版社在~1982~年出版的内部读物《延安整风运动纪事》第~87~页虽提及此决定后来没有发出,但未作任何解释。上述引文引自《毛泽东新闻工作文选》(北京:新华出版社,1983~年),页~94。1993~年中共中央文献研究室编辑出版的《毛泽东文集》第~2~卷也收人了此文。}
\end{quoting}

毛泽东的这些话说得何等好啊!十五年后的~1957~年春,他不也说过类似的话吗?然而,若细心观察,毛还是在文件中预留了伏笔,毛在这份文件中给“批评”作了一个二分法分类——“善意批评”和“破坏抗战团结者的恶意攻击”,至于判断何谓“善意批评”,何谓“恶意攻击”,其解释权自然属于党的领导机关和党机关的领导人。尽管如此,毛泽东在也~42~年~3~月下旬起草的这份以后被束之高阁的文件的主旨精神却是鼓动“放”!

1942~年的~3~月,“自由化”的微风从毛泽东的窑洞里飘拂出来,从毛泽东窑洞里进出的客人有一位就是非党作家萧军。毛泽东在一次深夜长谈中甚至将自己在党内所遭受的排斥和打击的细节向萧军和盘托出。\footnote{王德芬(萧军夫人):《萧军在延安》,载《新文学史料》,1987~年第~4~期;另参见张毓茂:《萧军传》(重庆:重庆出版社,1992~年),页~233~—~34。}从这件事中不难看出毛泽东在~1942~年~3~月鼓动“自由化”的真正意图。然而担任《解放报》社社长的博古并不完全知道毛泽东的打算,他眼见毛泽东“批评”的板斧又挥向自己负责的党报工作,于是迅速跟上毛的调子,在他的默许下,丁玲在《解放日报》文艺栏推出由党员与非党员撰写的一组批评性文字,以示党报工作的改进。

1942~年~3~月~9~日,就在《教条和裤子》社论问世的当天,《解放日报》刊出了丁玲的《三八节有感》,紧接着,由丁玲任主编、陈企霞任副主编的《解放日报》文艺栏又先后发表了王实味的《野百合花》、和经毛泽东亲笔润色、作了修改的萧军的《论同志之“爱”与“耐”》,\footnote{据当时在《解放日报》文艺栏任编辑的黎辛回忆,王实味的《野百合花》是经丁玲看过,签过“可以用”意见才见报的。博古因工作繁忙,没有事先审阅,文章见报后的次日,博古特来到编辑室询问王实味何许人也,并打听此文发表的经过,博古并叮嘱该文的下一部分“以后不要发表了”。3~月~23~日,《野百合花》下半部分又在《解放日报》刊出,博古又到编辑室询问此事,编辑陈企霞说明此文曾送博古审查,博古解释他因事忙,稿子没看,并表示他对此事负责。参见黎辛:《〈野百合花〉·延安整风·〈再批判〉》,载《新文学史料》,1995~年第~4~期。页~69~—~70。另参见王德芬:《安息吧,萧军老伴!》,载《新文学史料》,1989~年第~2~期,页~114。王德芬自~1938~年与萧军结婚后,于~1940~年~6~月与萧军同赴延安,萧军夫妇当时并不知道,毛泽东其实并不真正欣赏萧军。毛泽东虽然亲笔为萧军的《论同志的“爱”与“耐”》作了修改和润色,但萧军文章渗透的人性论并不合毛的口味。1942~年~4~月~8~日,延安已转入对王实味的批判,只是由于箫军的文章具有外人不了解的特殊背景,《解放日报》才破例予以刊载。毛泽东出于其个人的政治目的,本来有意利用萧军的豪爽性格,但毛很快就发现萧军个性倔强,难以驾驭,遂对萧军产生了反感。1958~年,《文艺报》第~2~期将萧军经由毛修改润色的《论同志的“爱”与“耐”》,与王实味的《野百合花》,丁玲的《三八节有感》等文章,汇编成《再批判专辑》,作为供批判右派的“大毒草”重新公布。毛亲自主持此事,并亲笔加了“奇文共欣赏,疑义相与析”的很长一段编者按语。}以及艾青的《了解作家,尊重作家》、罗烽的《还是杂文时代》,这些文章一经刊出,立即轰动了延安。

丁玲、王实味、萧军等人文章的共同特点是:用文学的形式对毛泽东在~1941~年~9~月~10~日政治局扩大会议上批评宗派主义的一段话:“在延安,首长才吃得开,许多科学家,文学家都被人瞧不起”作具体的解释和进一步的发挥,尽管他们当时都不知道毛的这个讲话。

毛泽东的“煽惑”终于将瓶子里的“魔鬼”驱赶了出来!丁玲等尖锐抨击了在延安普遍存在的“首长至上”的现象。这些文章还曲折地表达了广大青年知识分子对延安“新生活”的失望:基层单位的领导毫无政策和文化水平,对上奴颜卑膝,对下则横眉冷对,动辄用政治大帽子压制普通党员的不满。文章的作者纷纷要求扩大党内民主,在“同志爱”的基础上建立充满友爱、平等精神的革命队伍的新型关系。

在《解放日报》文艺栏发表的文章中,最具影响力的是中研院文艺研究室特别研究员王实味的《野百合花》。

\section{呼唤人道的、民主的社会主义:王实味言论中的意义}

王实味是受了五四民主和科学精神的影响,满怀乌托邦社会改造的理想,转而接受了马克思主义,从而投身共产主义运动的那一代左翼知识分子的突出代表。1926~年,时年二十岁的王实味在其就读的北京大学文科预科加入了中国共产党,一年后因与女友恋爱受到了中共支部书记的指责而不再参加支部的组织生活。从~1926~年起,王实味开始在北京、上海的文学刊物上发表文学作品,1929~年后,长期住在上海,有过一本创作小说集和五本文学译著问世。1937~年~10~月王实味来到延安,先入鲁迅艺术学院,后经张闻天亲自挑选,调入马列学院编译室,参与翻译马列经典著作,几年中译述达百万字左右。王实味个性耿直傲介,看不惯马列学院编译室负责人陈伯达等谀上压下的种种表现,与他们的个人关系十分紧张,但却十分尊敬张闻天、王学文和范文澜(原任马列学院中国史研究室主任,1941~年~8~月后任中研院副院长)。马列研究院改名为中央研究院后,王实味转入由欧阳山任主任的中国文艺研究室作特别研究员,享受中灶待遇。

从~1942~年~2~月始,年届三十六岁的王实味受毛泽东整顿三风号召之《解放日报》及中研院《矢与的》壁报上连续鼓舞,陆续在《谷雨》杂志、发表文章,计有《政治家、艺术家》,《野百合花》,《我对罗迈同志在整风检查动员大会上发言的批评》,《零感两则》等。王实味的上述文章,从内容上看,与丁玲、萧军、艾青等人的文章完全一致,只是更具尖锐性和批判性。

王实味大胆地揭露了延安“新生活”的阴影,相当准确地反映了延安青年知识分子理想渐趋破灭后产生的沮丧和失望的情绪,并对在革命口号下逐渐强化的等级制度及其官僚化趋向表示了严重的忧虑。

1937~—~1938~年,成千上万受埃德加·斯诺《西行漫记》、范长江《中国的西北角》和《塞上行》强烈吸引的知识青年,怀着对中共的崇仰和对未来新生活的憧憬,从天南海北奔向延安。他们的到来正好和急欲“招兵买马”、壮大自身力量的中共的现实目标相契合,因此受到中共领导的热烈欢迎,而与外界隔绝多年的老红军也热忱欢迎给他们带来各种信息的知识青年。延安一时到处充满着青年的欢声笑语,似乎成了一座青年乌托邦城邦。

知识青年在延安感受到一种完全迥异于国民党统治区的氛围,最令人振奋的是,在人与人关系上充满着一种同志式的平等精神。尤其从国统区中小城镇前来延安的女同志,更是觉得“卸掉了束缚在身上的枷锁,分外感到自由”。一首流传在延安的歌曲真实反映了当年她们的感受:

\begin{quoting}
冰河,在春天里解冻;万物,在春天里复生;全世界被压迫的妇女,在“三八”发出自由的吼声……从此,我们……我们定要……打碎这锁人的牢笼\footnote{文白:《金色年华——马列学院的八小时之外》,载《延安马列学院回忆录》,页~189。}!
\end{quoting}在这个时期,由于毛泽东的领袖权威还未最后形成,中共政治生活中的礼仪化色彩较为淡薄,毛泽东、王明、张闻天、朱德等党的领袖穿着朴素,言谈随和,经常前往各学校作报告,前呼后拥的现象还不突出。除了“毛主席”这个称呼已被叫习惯而继续沿用以外,其他中共领袖都可以被青年人直呼为“同志”,无论是“王明同志”、“洛甫同志”,还是“恩来同志”、“博古同志”,都未闻有谁将“书记”、“部长”一类头衔与他们的名字相联。集中在延安各学校学习的青年学生经常就马列基础知识和党的领导人的报告,展开热烈的讨论,“他们无限崇仰‘两万五’穿草鞋和会打草鞋的人”,“一到了自己的队伍里,就天真烂漫得很,虔诚到了家,对自己的领袖人物更是从心里往外热爱他们,一想到烈士,就肃然起敬”。为了表示与旧社会一刀两断,许多人甚至改了自己的姓名。艰苦的物质生活非但未减弱知识青年的热情,相反,在这种充满平等精神的新环境里,他们体验到心灵净化的崇高,对中共的政治目标产生出更为强烈的认同感。在这个时期,延安男女青年的交往还比较自由,十月革命后苏俄柯伦泰夫人的“杯水主义”一度流行,一些重要干部率先“与传统作彻底决裂”,上行下效,“打游击”和“革命的恋爱”成为新生活的一项标记,\footnote{宋振庭:《真理是朴素的,历史是无清的——为长诗《于立鹤》再版说几句话》,载严慰冰:《魂归江南》(上海:上海文艺出版社,1987~年),页~3。另“打游击”是喻指恋爱对象的转换犹如战无固定限制的游击战一样,经常处于变动之中,此种风尚一度流行于~1937~—~1938~年的延安。}使得理想主义的氛围更加浓厚。

然而~1938~—~1939~年后,随着国共关系的恶化,延安与外界的联系基本中断,在封闭的环境下,延安的社会气氛和精神生活领域开始出重大的变化:

一、毛泽东有意利用王明、康生从莫斯科带回的斯大林“反托派”精神为己服务,放纵康生在延安营造“肃托”精神恐怖,青年知识分子无端失踪的事件时有所闻。伴随“肃托”阴影的扩大,对毛泽东的个人崇拜也逐渐升温,延安各学校原有的自由讨论的学习活动渐渐转变为对毛报告的歌颂,广大知识青年的主动性逐渐消失,自赎意识与沮丧感日益蔓延。

二、上下尊卑的等级差序制度逐步完善,新老干部的冲突逐步表面化。在任弼时的督导下,1940~年延安的大、中、小三灶制度正式在全党推行,舞会成为延安高级干部生活中的一项重要内容,对领导干部的安全保卫工作已制度化。各单位的领导多由参加过长征的老同志担任,知识分子的思想和生活习惯开始受到严厉的指责,批评知识青年的词汇,诸如“小资产阶级的动摇性”,“小资产阶级的情调”,愈来愈经常出现在报刊和领导干部的口中,成为笼罩在青年知识分子头上的精神低气压。

三、恋爱自由逐渐受到限制。“杯水主义”现象显然与差序等级制度相违背,作为一种“时尚”它在~1939~年就告结束,而代之以干部级别为基础的、由领导介绍批准的婚姻制度。

到了~1941~年,延安的青年知识分子忽然发现,他们已从青年乌托邦理想国的主人,一下子跌落至“等级差序”制度下的最底层!

从王实味给我们提供的延安两个女青年的对话中,不难看出延安青年知识分子的失望和激愤:“……动不动,就说人家小资产阶级平均主义,其实,他自己倒真有点特殊主义。事事都只顾自己特殊化。对下面同志,身体好也罢,坏也罢,病也罢,死也罢,差不多漠不关心!”

“哼,到处乌鸦一般黑,我们底××同志还不也是这样!”

“说得好听!阶级友爱呀,什么呀——屁!好象连人对人的同情心都没有!平常见人装得笑嘻嘻,其实是皮笑肉不笑,肉笑心不笑,稍不如意,就瞪起眼睛,摆出首长架子来训人。”

“大头子是这样,小头子也是这样。我们的科长,对上是毕恭毕敬的,对我们,却是神气活现,好几次同志病了,他连看都不伸头看一下。可是一次老鹰抓了他一只小鸡,你看他多么关心这件大事呀!以后每次看见老鹰飞来,他都嚎嚎的叫,扔土块去打它——自私自利的家伙!”

“我两年来换了三四个工作机关,那些首长以及科长、主任之类,真正关心干部爱护干部的,实在太少了。”

和这一切相对应,王实味又为后世提供了一幅伴随革命“蜕化”,“新阶级”破土而出的逼真画面:

抗日前线的将士在浴血奋战,“每一分钟都有我们亲爱的同志在血泊中倒下”;延安中央大礼堂的舞会在通宵达旦地举行,“歌啭玉堂春,舞回金莲步”;“害病的同志喝不到一口曲汤,青年学生一天只得到两餐稀粥”;“颇”为健康的‘大人物’有着非常不必要不合理的‘享受’,“食分五等,衣着三色”……\footnote{王实味:《野百合花》,载《解放日报》,1942~年~3~月~13~日。}

作为一个狂热的理想主义者,面对这一切,王实味感到不平,有如骨刺在鲠。他似乎十分怀恋~1937~—~1938~年那段充满理想主义色彩的岁月,但他毕竟又信奉马克思主义,知道“共产主义不是平均主义”。于是王实味申明自己也属于“干部服小厨房阶层,葡萄并不酸”,放言直谏绝非为争人利益。他甚至对延安的等级差序制度表现出最善意的理解,认为“对那些健康上需特殊优待的重要负责者予以特殊的优待是合理的而且是必要的。一般负重要责任者,也可略予优待”。只是处在当前“艰难困苦的革命过程中”,“许多人都失去最可宝贵的健康的时期”,为了“产生真正铁一般的团结”,“负责任更大的人,倒更应该表现与下层同甘共苦(这倒是真正应该发扬的民族美德的精神)”。

王实味确实太书呆子气了。他毕竟未亲身经历过~1927~—~1937~年的“现代农民革命战争”,他不知道他所提出的这些要求对于某些出身于农民的高干实在是强人所难。斗转星移,中共虽然还未打下天下,但已有几块相对稳定的地盘,一些人已做不来与群众“同甘共苦”、“吃一锅番瓜汤”那类事了!虽然从总体上讲,在四十年代初的延安,干部物质待遇的差别还不是十分明显:大、中、小三灶伙食供应制度和斜纹布(黑色)、平布(青灰色)、土布三级服装供应制度,如果和~1949~年后的高干特供制度相比,简直不能同日而语,但为何王实味竟感到“大人物”有如“异类”呢?真正使王实味以及延安青年知识分子产生疏离和异己感的是以“食分五等,衣着三色”为特征的、与高干地位相联系的干部特殊待遇制度以及由此产生的“高干至上”的浓厚的社会气氛。

延安的重要高干的家里一般都有组织分配专门照顾首长孩子的保姆,保护首长安全的警卫员,以及照顾首长生活起居的勤务员或公务员(负责为首长打洗脸水和洗脚水,在牙刷上放牙粉等),有的首长的勤务员也由警卫员兼任,另有“伙夫”、“马夫”各一人。\footnote{毛泽东不喜称刘伯承从苏联红军条例中引进的“炊事员”“饲养员”这两个洋名词,而习惯使用具有“中国特色”的“伙夫”、“马夫”的称谓。}在范围不大的延安城,人们经常可以看到警卫员紧紧跟着首长和他们的家属沿着延河漫步,每逢星期六,人们也不难看到首长派来接爱人回家度周末的警卫员已早早守候在各学校、机关的门口。甚至连孩子们都知道:街上奔跑的那辆由海外华侨捐赠给八路军前方将士的救护车,是“毛主席的汽车”。至于毛泽东居住的杨家岭、枣园戒备森严,明哨、暗哨密布,非邀不得靠近,更是公开的秘密。

1942~年春,毛泽东派李卓然登门邀请塞克到其住处谈话,被塞克拒绝,理由是“有拿枪站岗的地方我不去”,直到毛泽东吩咐撤去岗哨,塞克才在邓发的陪同下去见了毛泽东。\footnote{黄樾:《延安四怪》(北京:中国青年出版社,1998~年)。页~124。另据抗战初期代表四川地方实力派杨森前往延安的杜重石回忆,当他在~1938~年初夏前往见毛泽东时,在毛的住处附近不仅见到哨兵,他更受到哨兵的“人身检查”。参见杜重石:《风雨岁月》(香港:天地图书有限公司,1993~年),页~38~—~39。}这一切在王实味和延安青年知识分子的眼中,全都成了背离革命道义原则的有力证据,使他们产生了强烈的义愤和不平等感。

最令人心悸的是作为一个书生的王实味竟敢从抨击“食分五等,衣看三色”的现象进而试图剖析产生等级差序制度的思想和历史根源,间接或直接地向手握兵符和镇压机关大权的毛泽东发起了挑战。王实味看到的“革命圣地”绝非是纤尘不染的共产主义殿堂,在延安的阳光下,不仅存在着浓重的阴影,还有人“间接助长黑暗”,“甚至直接制造黑暗”。证据之一,即是有人以“发扬民族美德”为等级差序制度辩护——而这类“民族化”言辞的始作俑者不是别人,正是毛泽东。王实味是一个十分敏锐的知识分子,他亲自体验了~1938~年六届六中全会后在延安逐步深化的“马克思主义中国化”的过程,感受到在社会气氛方面所发生的深刻变化。他不会忘记~1940~年延安理论界所发生的关于“民族形式”问题的讨论,正是随着这类讨论的展开,在“民族形式”的掩护下,中国传统政治文化的某些价值被堂而皇之地引入到中共的理论及其实践中。终于,王实味得出自己的结论:中国专制主义的旧传统已严重侵蚀了中共的肌体,即使在延安,“旧中国的肮脏污秽也沾染了我们自己,散布细菌,传染疾病”,而旧传统一旦与以“必然性”面目出现的俄式马克思主义相结合,就构成了有中国特色的等级差序制度的理论基础。

王实味质问道,难道我们可以用“了解国情”,“尊重国情”作借口,迁就、迎合中国的落后传统吗?难道因为“黑暗面”存在有其必然性,就欢迎、保护黑暗吗?王实味笔锋一转,忽将批评的矛头直指毛泽东独创的名言“天塌不下来”论:

\begin{quoting}
在“必然性”底“理论”之后,有一种“民族形式”的“理论”叫做“天塌不下来”。是的,天是不会塌下来的。可是,我们的工作和事业,是否因为“天塌下下来”就不受损失呢?这一层,“大师”们底脑子绝少想到甚至从未想到。
\end{quoting}

从四十年代初开始,毛泽东就愈来愈喜欢讲“天塌不下来”,在形势紧张,中共面临困境时,毛爱谈“天塌不下来”;在闻知党内外有不满意见时,毛更爱说这句话:

“有意见,你让人家讲吗,天又不会塌下来!”,“有话就说,有屁就放,天塌不下来!”

“我劝同志们硬着头皮顶住,地球照转,天塌不下来!”

满腔热血的王实味慷慨陈辞:

\begin{quoting}
在认识这必然性以后,我们就须要以战斗的布尔什维克能动性,去防止黑暗的产生,削减黑暗的滋长……要想在今天,把我们的阵营里一切黑暗消灭净尽,这是不可能的;但把黑暗消灭至最小限度,却不但可能,而且必要。
\end{quoting}

王实味继而警告道:

\begin{quoting}
如果让这“必然性”“必然”地发展下去,则天——革命事业的天——是“必然”要塌下来的。别那么安心罢。
\end{quoting}

王实味在~1942~年所发出的这番“危言耸听”的预测,终于事隔四十九年后在“第一个社会主义国家”——苏联和东欧、蒙古各社会主义国家得到应验,苏联甚至都已不复存在。

只是,王实味当时还指望毛泽东警醒,以求防患于未然,但是王实味再一次失望了:

“‘大师’们不惟不曾强调这一点(指防止黑暗面的滋长——引言注),他们只指出‘必然性’就睡觉去了。”

“其实,不仅睡觉而已。在‘必然性’底借口之下,‘大师’们对自己也就很宽容了。”

尽管毛泽东和其他领导人不着急,王实味却忧心如焚。他为防止“革命的天塌下来”,向全党贡献出他的药方:重振共产主义的理想之光,在革命道义的基础上建立新型的人与人之间的平等关系,使共产革命充满永久的动力。

王实味饱含激情,无限感怀地回忆起为实现主义而牺牲了生命的李芬烈士,企图用追忆殉道烈士的方式来唤起人们胸中已沉寂的理想主义,用烈士滚沸的热血来涤荡革命队伍中已经形成的带有专制色彩的人身依附的现象。王实味更期盼借早期共产主义者身上的圣洁之光,重振革命者已渐丧失的自豪感、自尊感和自信感。王实味梦想中共能在国际共产主义运动中首先实践关心人、尊重人、重视人的人道主义准则,让自由、平等、博爱的精神成为“新社会”的自觉原则。\footnote{王实味对斯大林主义的本质已有一定的认识,》他曾在私下对一些人说过,“斯大林人性不可爱”,“斯大林的性情太粗暴了”,“苏联对于季诺维也夫叛国案的审判是可怀疑的”,“苏联在清党时,斯大林不知造就了多少罪恶”,参见温济泽:《斗争日记》,载《王实味冤案平反纪实(北京:群众出版社,1993~年),页~188、192。在公开场合,他借毛抨击“言必称希腊”之话,回击那些援引苏联也有特供制度而为延安特供制度辩护的人,要这些人“闭嘴”。}王实味热情歌颂青年的“纯洁、敏感、热情、勇敢”,呼吁当政者万勿因延安的青年由于得不到“爱和热”而发出“牢骚”就厌恶、嫌弃他们。王实味不无沉痛地说:“延安的青年已经够老成的了”,若再打击他们(例如某个在墙报上写文章的青年,遭该机关“首长”批评打击,“致陷于半疯狂状态”),那么这个世界就太“寂寞了”。王实味甚至“超现实主义”地做起类似武者小路实笃笔下的“一个青年的梦”,以为凡为人者,皆存一颗温暖友爱的同情怜悯之心。他写道:

\begin{quoting}
我底理性和良心叫我永远以最温和的语调称呼他们为“炊事员同志”(尽管在延安称伙夫为“炊事员同志”有些讽刺意味,因为即使不反对平均主义,也不会有“半个伙夫会妄想与‘首长’过同样的生活”)。\footnote{王实味:《野百合花》,载《解放日报》,1942~年~3~月~23~日。}
\end{quoting}从抨击延安“歌舞升平”的景象到剖析等级差序制度的思想和历史根源,进而再开出一付“民主”、“博爱”的疗救药方,王实味的论调在延安领导人听来,的确太刺耳了!不仅如此,甚至听来有如托洛茨基的“工人国家蜕化论”!在这里,他们的判断并不错,王实味的思想确实和毛泽东的“中国化的马克思主义”格格不人。王实味鼓吹的“平等”、“博爱”观念,从根本上说,和建立在政治功利主义哲学基础上的“领袖至上”观念存在着严重的对立。从王实味的人道主义思想出发,必然引发出人的主体精神的高扬;而确立“领袖至上”的观念的首要条件就是削弱乃至消灭人的自主性,使人成为“齿轮和螺丝钉”。1942~年的毛泽东正全力追求实现这个“宏伟”的目标,将全体党员改造成“党的(也即是毛的)驯服工具”,如果听任王实味“蛊惑人心”的言论继续扩散,势必严重抵消毛的努力。因此,王实味及其思想就成了毛泽东不得不予以清除的障碍。

王实味呼唤平等,抨击延安的等级差序制度,也直接侵犯了毛泽东和享受等级制度之惠的老干部的利益,至于王实味言论中渗透的对革命的幻灭感,若从毛的角度看,则有可能“侵蚀”并危及毛“打江山”的伟业。1927~年后,作为“现代农民革命战争”主体的农民,已成了中共革命的主要动力,尽管因受俄式马克思主义的影响,使这场发生在二十世纪中国的“农民革命战争”增添了许多新因素,但是中国历史上的“农民造反”模式仍对中共武装革命发生了巨大的影响。在不少参加这场“现代农民革命战争”的农民看来,“打江山”的意义就在于“排座次,坐江山”,尤其随着毛泽东在三十年代中期控制了中共军队后。这场“农民革命战争”的俄式共产主义因素就逐渐被烙有毛个人印记的民族化共产主义因素所置换。深受农民造反传统浸润的毛泽东极为重视利用农民的感情、行为和愿望为自己“打天下”的伟业服务,毛在马列主义的术语范围内,转换概念,引进大量中国传统农民造反的语汇和行为,使得中共军队既是苏联红军式的军队,又是一支具有浓厚中国传统农民起义军色彩的队伍。作为适应战时环境的一种军事共产主义的分配方法,和体现“论功行赏”传统原则的延安等级差序制度就是这样一种新旧杂揉的混合物。这个制度一经建立就受到农民出身的军队高级将领的普遍欢迎和敬重,至于受过俄式教育的中共高级文职干部,也因它夹杂斯大林等级制的因素,而对其采取“欣然接受”的态度。如今王实味放肆攻击被他们视为是未来革命成功后将要实行的权力与财富再分配制度的先期象征符号,怎能不遭致高干们共同的愤怒?于是当某军队高级将领破口大骂王实味“反党”时,那些具有留苏背景的高级文职干部,也纷纷口诛笔伐,声讨王实味“反列宁主义”。

毛泽东对等级差序制度的态度要比某些“武人”复杂的多。毛在等级制度问题上的看法是灵活的,其变化中轴点是看其是否有利于一统天下的伟业和他对权力的获得与巩固。从本质上说,毛泽东是中共等级差序制度的最大维护者,而当他的权力还未达到顶峰,或自感权力受到威胁时,毛往往又变成一个“反官僚主义”的急先锋,只有在这时,毛才会释放一些,换上一付“小人物”保护者的面孔,然后煽动下层头脑中的“虚无主义”、“反特权”,驱赶群众为自己的政治目的服务;一俟目标实现或情况有变,毛则马上翻脸,挑出几个“替罪羊”,批判起“自由化”和“无政府主义”,将那些被愚弄的群众和知识分子一网打尽。此种计谋毛一生屡试不爽,1942~年则是牛刀初试。王实味与一般左翼人士不同之处在于,他不仅执着于五四自由、民主的理念,他更受到青年马克思人道主义思想,和继承发扬了这种思想的第二国际社会民主主义思想,以及反斯大林主义的托洛茨基部分观点的强烈影响。

作为近代欧美文明一部分的青年马克思人道主义思想和社会民主主义的思想传统,在战乱频仍、国民文化素质低下的二十世纪中国,除了在五四时期及三十至四十年代曾引起少数知识分子的兴趣,犹如惊鸿一瞥,始终不曾在中国落户。它和在第三国际指导下建立的中共几乎毫无思想联系,尤其随着~1927~年国民党屠杀中共的一声枪响。中共历史上的古典共产主义革命阶段的终结,社会民主主义思想对中共的影响已接近于零。王实味作为一个异数,他的社会民主主义观点主要来源于他对马克思和其他社会民主主义思想家著作的直接阅读,以及在阅读基础上的独立思考。

王实味与托派的部分观点存在共鸣也是一个明显的事实。三十年代初期,王实味与托派有所来往,曾翻译过托洛茨基自传的部分章节和被斯大林隐瞒的列宁遗嘱,对苏共内部复杂、残酷的斗争有所了解。1937~年王实味投奔延安,但头脑中的“异端”思想并未完全根除,托洛茨基所描绘的“工人国家官僚化”的景象,连同辗转传至延安的~1937~—~1938~年莫斯科“叛国案大审判”的不祥消息,刺激、震撼了王实味,逼使他自觉或不自觉地拿起“蜕化论”这把标尺,来衡量延安所发生的一切,而他在延安所见所闻正好与他头脑中的“蜕化”影像完全重合!

\section{风向突转:毛泽东拿王实味开刀}

王实味的《野百合花》和刊登在中央研究院《矢与的》壁报上的其它言论一经问世,立即在延安引起轰动,一时间各学校、机关纷纷仿效中研院,也办起了各种类似《矢与的》的壁报,至~1942~年~3~月底,4~月初,各大单位创办或计划创办的壁报有:

西北局的《西北风》;

延安自然科学院的《整风》、《向日葵》、《心里话》;

民族学院的《脱报》(脱者,“脱裤子”也);

中央医院的《显微镜》(拟创办);

延安学生疗养院的《整风》\footnote{“学生疗养院”为干部疗养院的代号,即如中央党校的代号曾为“中山图书馆”,机要通讯局的代号为“中央农委”。}。

这些壁报连同存在时间较久的中央青委《轻骑队》壁报,构成了延安壁报群中的几朵耀眼的浪花。

与此同时,各学校、机关“反官僚,争民主”的浪涛正汹涌澎湃:中央研究院,百分之九十五的工作人员同情、支持王实味的观点,研究院领导罗迈和张如心等成为众矢之的。

延安大学也出现了民主“一边倒”的局势。在~3~月~26~日全体党员大会上,与会者控诉了“个别领导同志,以主观武断的态度处理问题,缺少民主精神,以‘尊重组织’、‘尊重领导人的威信’等为借口变相地压制民主,以致造成一部分同志不敢讲话和不愿讲话的现象”,会议“打破了过去大家畏缩不敢发言的空气,充分发扬了民主”,有人甚至提议,应以清算的方式,明确责任问题,“是非属谁,责任属谁,根究到底,必须得出正确的结论”。各支部代表一致认为:

\begin{quoting}
领导者的威信是建立在自己的工作上,建立在正确的认识和主张上,若建立在别人不敢谈话的基础上,是非常危险的。\footnote{参见《延安整风运动纪事》,页~83。}
\end{quoting}
几个还未完全“脱”尽率真之气的党员老知识分子,竟然也在一时高涨的“鸣放”空气中翩然起舞。

中央研究院副院长范文澜是三十年代初入党的老党员,1934~年被捕,1939~年赴延安后重新加人共产党。他在《矢与的》壁报上撰文,号召“以民主之矢,射邪风之的”;\footnote{李维汉:《中央研究院的研究工作和整风运动》,载李维汉:《回忆与研究》,下,页~485。}

自然科学院院长徐特立经历过中央苏区的党内斗争,讲话就含蓄得多,他提议不要多追究个人责任,但是也按捺不住,鼓励青年“大胆发言,认真检讨”;\footnote{参见《延安整风运动纪事》,页~89。}

已经奉命停课,或即将停课转入整风的各学校,和已转入整风检查阶段的各机关,都不同程度地出现了动荡的局面,青年知识分子普遍要求“揭盖子”和割领导的“尾巴”……

此时的毛泽东又是何种反应呢?他是大畅所欲,抑或是“龙颜震怒”?据胡乔木回忆,当毛泽东看到在《解放日报》上连载的王实味的《野百会花》后,气愤地“猛拍办公桌上的报纸”,厉声问道,“这是王实味挂帅,还是马克思挂帅”?毛当即给《解放日报》打电话,“要求报纸作出深刻检查”\footnote{参见《胡乔木回忆毛泽东》,页~449。}。毛泽东感到情况不妙,担心运动将失去控制,迅速改变原先制定的利用“自由主义”打击教条主义的策略;在亲自前往看过在《矢与的》壁报后,毛召集高干会议,决定抛出王实味作为靶子,先行将“自由化”打压下去。\footnote{这次由毛泽东主持的高干会议召开于~1942~年~4~月初,代表文艺界参加的只有周扬和丁玲两人。会议的议题是批判王实味的《野百合花》,曹轶欧(康生妻子)、贺龙等在发言中都严厉指责了丁玲。毛在会议总结中将丁玲与王实味区分了开来,声称丁玲同王实味不一样,丁玲的文章有建议,虽说也有批评,而王实味则是托派。参见丁玲:《延安文艺座谈会的前前后后》,载艾克恩编:《延安文艺回忆录》(北京:中国社会科学出版社,1992~年),页~62;另据戴晴称,1943~年贺龙曾在大会上“把《三八节有感》的作者骂为‘臭婊子’”,参见戴睛:《梁漱溟、王实味、储安平》,页~102。}

1942~年~3~月~31~日,毛泽东在《解放日报》改版座谈会上放出“反击”的气球,他抓住“立场”、“绝对平均观念”和“冷嘲暗箭”三个问题,向延安青年知识分发出严厉警告:

\begin{quoting}
有些人是从不正确的立场说话的,这就是绝对平均的观念和冷嘲暗箭的办法。近来颇有些求绝对平均,但这是一种幻想,不能实现的。我们工作制度中确有许多缺点,应加改革,但如果要求绝对平均,则不但现在,将来也是办不到的。小资产阶级的空想社会主义思想,我们应该拒绝。...~冷嘲暗箭,则是一种销蚀剂,是对团结不利的。\footnote{毛泽东:《在〈解放日报〉改版座谈会上的讲话》,载《毛泽东新闻工作文选》,页~91。}
\end{quoting}毛泽东的上述警告赫然刊登在~4~月~2~日《解放日报》的头版,但是在~1942~年春天,延安大多数青年干部还未练就从报纸上观察政治风向的本领,竟然将毛的警告置之脑后,继续我行我素,“鼓噪”民主。延安青年知识分子对毛泽东发出的政治信号熟视无睹,这并不是第一次,早在中央研究院~3~月~18~日召开动员整风大会的前夕,《解放日报》就分别刊登罗迈、张如心文章,\footnote{1942~年~3~月~16~日《解放日报》刊登罗迈影射攻击张闻天的文章《要清算干部教育中的教条主义》,3~月~17~日又刊登张如心不指名攻击王明、博古的《清算德波林主义,开展反主观主义的斗争》一文。}已从侧面传达了毛对他们的保护态度,只是在当时,王实味和中研院的绝大多数干部根本没有把这当成一回事。如果说上次的疏忽碰巧与毛泽东“放火烧荒”的意图相吻合,毛也并非不愿看到罗迈这个昔日的政敌被群众“火烧”一下,那么当毛已改变主意,将王实味视为首要敌人之后,中研院再“炮轰”罗迈,就属犯上作乱的行为了,只能增添毛剿灭自由主义的决心。

1942~年~4~月~3~日,也就是《解放日报》刊登毛泽东警告的第二天,中宣部正式发出有名的“四三决定”(即《关于在延安讨论中央决定及毛泽东同志整顿三风报告的决定》),这个决定是“在毛泽东直接领导下”,\footnote{李维汉语,参见李维汉:《回忆与研究》,下,页~486。}针对中研院整风出现的“自由化”倾向,特为“纠偏”而制定的。《决定》明确申明:整风必须在各部门的领导机关负责人领导下进行,不得以群众选举的方式,组织领导整风的检查委员会;在检查工作时,不仅只检查领导方面的,而要检查下面的和各个侧面的;每人都必须反省自己的全部历史。\footnote{中央档案馆编:《中共中央文件选集》(1941~—~1942),第~13~册,页~364~—~66。}“四三决定”的颁布不仅结束了短暂的延安之春,而且标志着自~1941~年~10~月就秘密酝酿的干部审查运动即将拉开帷幕(当时成立了以康生为首的“党与非党干部审查委员会”),整风不久将转人严酷的审干肃反阶段。

4~月~5~日,《解放日报》刊登胡乔木起草的《整顿三风必须正确进行》的社论,胡乔木在社论中指斥整风已出现了“不正确的方法”,再次重复毛泽东~3~月~31~日发出的警告,不指名地抨击王实味是“从不正确的立场来说话”,谴责王实味等的“错误的观念,错误的办法,不但对于整顿三风毫无补益,而且是有害的”。\footnote{《整顿三风必须正确进行》,《解放日报》社论,1942~年~4~月~5~日。载《胡乔木文集》,第~1~卷。页~57。}4~月~13~日,延安壁报的始祖,创刊于~I941~年~4~月的中央青委《轻骑队》壁报编委会在《解放日报》上作出初步检讨,与胡乔木关系密切的中央青委的几个青年已从胡乔木处获知运动将转向的信息。《轻骑队》编委会的这份检讨正好与毛泽东的目标相一致。毛很清楚,延安的青年知识分子普遍同情王实味而他却不能将他们全部打成反革命——毛只求杀一儆百,从此封住他们的嘴,继而改造他们的思想,使其心悦诚服,老实就范即可。因此毛乐意让《轻骑队》壁报在认错后过关,这也算是给中央青委书记陈云的一个“面子”。4~月~23~日。《解放日报》果真发表了《轻骑队》编委会的《我们的自我批评》,他们在此文中申明自己是“一群政治上幼稚的青年同志”,承认自己的言论“助长了同志间的离心倾向”,“产生了涣散组织的恶果”\footnote{《轻骑队》编委会:《我们的自我批评》,载《延安文萃》,上,页~57。},《轻骑队》获得了解脱。\footnote{据当年参加《轻骑队》壁报的李锐称,《轻骑队》没有编委会,在闻知胡乔木对《轻骑队》的意见后(胡责成《轻骑队》成员童大林对编辑方针的错误作出检查),许立群对《轻骑队》的“错误”作了长篇检查,胡乔木让童大林对原文作了压缩,将《轻骑队》的检查送给毛泽东过目,毛给文章加上《我们的自我批评》的标题,发表于~1942~年~4~月~23~日的《解放日报》。参见宋晓梦:《李锐与延安〈轻骑队〉》,载广州《岭南文化时报》,1998~年~9~月~10~日。虽然《轻骑队》检讨文章称,“我们决心把第二年的《轻骑队》来一个彻底的改造”,但事实上《轻骑队》已寿终正寝,再没复刊。}

王实味此时并不知风向已变,仍在中研院振振有词,继续发表情绪激昂的演说,殊不知他已被毛泽东选中,即将被当作活祭推上燃烧着的火台,成为警吓众“猴”的一只待宰之“鸡”!

1942~年~4~月~7~日,在前一阶段因遭到巨大的批评压力,暂时退避一旁的罗迈遵循毛泽东的“反击”部署,从容跃入前台。具有丰富的党内斗争经验且十分熟悉毛泽东个性的罗迈,为了向毛显示自己的忠心,先将中研院出现的“自由化”与王明、张闻天挂起钩来,\footnote{罗迈在~1942~年~4~月~6~日毛泽东主持召开的中央高级组会议上发言,将中央研究院整风中出现的“偏向”归咎于“过去教条主义的教育”,参见李维汉:《回忆与研究》,下,页~486。一年后,张闻天对王实味问题也作了检讨,他在《整风笔记》中写道:由于“放松了对各种错误思想的斗争,以致如王实味一类的反动思想在整风开始后,得以取得全院绝大多数人的同情”,“我曾经想在马列学院内创造一种新的学风,新的党风,而结果却发展了教条主义,自由主义与党八股。这种学风,党风与文风,正是为小资产阶级知识分子以及特务分子所欢迎的”。引自唐天然:《有关延安文艺运动的“党务广播”稿》,载《新文学史料》,1991~年第~2~期,页~187。}再有条不紊、胸有成竹地部署中研院的反王实味斗争。很快,中研院原先支持、同情王实味的干部,被骤然降临的风暴吓得不知所措,随即为求自保,纷纷反戈一击,或痛哭流涕检讨自己立场不稳,上当受骗;或义愤填膺,控诉王实味一贯“反党”、“反领导”。一些人甚至作出与王实味“势不两立”的模样,要求组织上严惩王实味。在这种群体性歇斯底里的疯狂状态中,王实味被控的罪名也不断升级,到了~1942~年~6~月,王实味的头上已有三项“铁帽子”:反党分子(不久又升格为“反党集团头目”)、托匪、国民党特务(又称“国民党探子”)。在持续的精神恐惧中,6~月初,王实味突发书呆子“异想”,宣布退出中共,以为就此可以摆脱一切。但是,王实味又大错特错了,罗迈等绝不会允许王实味退党,而是要将其开除出党。他承认或不承认“错误”,更是无关紧要,他的归宿早已由“上级”作了安排,作为一个难得的“坏人”标本,等待他的将是被捕入狱,即使王实味痛哭流涕,收回退党声明,承认自己的言论犯了弥天大罪,跪在中央组织部磕头求饶,也丝毫无济于事\footnote{参见温济泽:《王实味冤案平反纪实》,页~7。温济泽是当时中央研究院党总支工作人员,写过一篇记述中研院斗争王实味大会的《斗争日记》,发表于《解放日报》~1942~年~6~月~28、29~日。1979~年后,温济泽为平反于实味冤案作了大量的申诉工作,在温济泽和其他人努力下,1991~年~2~月~7~日,王实味冤案得到彻底平反。}。1942~年~11~月后,王实味已处于隔离状态,失去了人身自由,1943~年~4~月~1~日,王实味被康生下令逮捕,次日被关押进中社部监狱,\footnote{凌云(中社部当时的工作人员,曾参与审讯王实味,八十年代曾任国家安全部部长)《王实味的最后五十个月》:(1993~年~5~月),载《王实味冤案平反纪实》,页~74。}从此成了活死人,除了偶而被带出来向来延安的外国或国统区记者发表一番自唾自弃的说辞外,\footnote{《新民报》记者赵超构在~1944~年~5~至~7~月参加了赴延安的中外记者参观团,返回重庆后,写了《延安一月记》,记述了他见到王实味的情景。据凌云称,1942~年~11~月初,康生曾致电在重庆的周恩来,要周提供有关王实味等是“托匪”的资料,“但周的覆电没有提供任何对康生有用的材料”。赵超构写道,王实味对中外记者说,他“在休养中”,王实味“谈话的神情完全像演讲”,他的表情严肃的可怕,“说到他过去的‘错误’,有时竟是声色俱厉的”。参见赵超构:《延安一月记》,载王克之编:《延安内幕》(上海:经纬书店,1946~年),页~52、51;另据凌云称,让王实味见中外记者是党中央的指示。王实味是由中社部的干部陪他去见记者的,当时“王实味还是顾全大局的”,只是返回后,王实味“非常恼火躺在床上,握紧拳头,表示了极大的不满,说当众承认是托派‘是自我牺牲,是被迫的’。”参见凌云:《王实味的最后五十个月》,载《王实味冤案平反纪实》,页~78。}王实味的日常“工作”就是写交待材料,直到~1947~年~7~月~1~日在山西兴县被康生下令砍了头。

从毛泽东的角度看,整肃王实味,并下令在延安展开批王斗争乃是箭在弦上,不得不发。1942~年~6~月初,在批王斗争达到高潮时,一位与王实味友善又与萧军相熟的作家李又然,请求与毛关系较熟的萧军向毛代为说项,然而当萧军向毛陈述后,却遭到了毛的断然拒绝;毛并警告萧军不要插手\footnote{萧军对张毓茂的谈话,参见张毓茂:《我所知道的萧军先生》,载《新文学史料》,1989~年第~2~期,页~139~—~40。}。

1942~年春,王实味这头从魔瓶中跑出的“魔鬼”使毛泽东大为震惊,毛本指望大大小小的王实味们可以把一把大火烧到王明、博古一类的“大尾巴”上,谁知王实味等乱燃野火,横扫一切,竟敢把矛头指向了新秩序的基石——等级差序制度,真可谓犯上作乱,是可忍,孰不可忍!这时,毛方体会到等级差序制度的“甜露”还不很久,毛愈来愈相信,等级差序制度是建立新秩序的重要保障。在唤起农民“觉悟”,组成浩浩荡荡的“打天下”队伍时,一个“阶级斗争”,一个“论功行赏”,无疑是壮大和凝聚革命力量的两个最有效的武器,只不过“阶级斗争”是公开亮出的旗号,另一个则是在内部实行的方法。王实味向等级差序制度挑战,影射、讽刺这是旧中国的“污秽”,涣散人心,居心叵测,只能使青年知识分子顿生革命的“无意义感”,无疑是企图摧毁革命。正是基于这种认识,毛泽东才勃然变脸,全然不顾自己曾抨击过延安“首长至上”的现象,以及大力鼓励青年知识分子“割大尾巴”的事实,厉声谴责起王实味的“绝对平均主义”和“小资产阶级的空想社会主义”。毛也发现,利用自由主义反对教条主义是一种格外需要小心驾驭的、充满巨大风险的政治游戏,不到绝对必要时不能轻易操之,否则极易引火烧身。1942~年春,王明等已显出一副衰相,虽说国际派仍是毛在党内的头号敌人,但毛已再无必要利用王实味之流来围剿王明。当然,事情既已发生,也无需惊慌失惜,正好借此契机,因势利导,同时开展“两条战线的斗争”,将自由主义和教条主义一锅煮,把反王实味的斗争转化为一场整肃“异端”的运动,这也可叫做“将坏事变为好事”。于是“延安之春”的帷幕刚落下,审干肃反的锣鼓就已敲响!最后在审干战役的高潮中,1943~年~5~月,毛泽东干脆关闭中央研究院,把“红色教授”们全部赶进中央党校去受审。

总之,王实味是在劫难逃,延安的广大青年知识分子也命中注定要在这已置放好断头台的革命广场上经历一场红色风暴的洗礼,以实现脱胎换骨的蜕变。当然,最不幸的还是王实味,虽说每一场大革命都有一批冤死鬼,但王实味的毁灭却因是一场精心设计的计谋的牺牲品而格外令人扼腕同情。毛泽东决定拿王实味开刀,挑选在当时社会知名度不甚高的他作斗争靶子,而有意放过了和王实味有相同思想的知名女作家丁玲,是基于一种深谋远虑的考量:若在延安打击~1936~年底就投奔陕北的《莎菲女士的日记》的作者丁玲,势必将在国统区文化教育界和青年学生中引起极大震动;而清洗王实味,则犹如下一场“毛毛雨”,最多地皮湿一层,稍刻就会踪影全无,不致严重削弱国统区左翼文化人和青年学生对延安的向心力。尽管毛泽东政治行为中的功利主义的性质在王实味事件中得到充分的暴露,但是世人既有势利眼,也就不能单单责怪毛泽东的老谋深算,工于心计。毛泽东的策略果真获得了成功:虽然国民党当局围绕王实味事件出版了小册子,王实味的命运也曾引起国统区一些文化人的关注,但是王实味的名气毕竟不很大,而国统区的严重腐败使更多的人把国民党在王实味事件上的反应,视为是老生常谈的“反共八股”而不予置理——王实味终于被历史尘封了。

\section{毛泽东为什么要给延安文化人套上“辔头”?}

1942~年春王实味事件的爆发及其巨大反响引起毛泽东的高度警惕,王实味的拥护者和最忠实的听众基本上是那些前来延安的“文化人”,而在这些人当中,文艺界人士又占了相当大的比重。毛泽东十分清楚,王实味事件绝非是一孤立、偶然的事件,王实味的言论集中反映了延安文艺界人士的不满情绪,代表了他们的观点、态度和立场,毛联想到延安文艺界多年来存在的问题,判断在延安文艺界存在着一股反叛的潜流。

延安文艺界果真有反叛的迹象吗?毛泽东的估计显然过于严重了。从~1937~—~1941~年,延安的文艺界人士想党之所想,急党之所急,自觉服从中共的政治需要,积极宣传党的路线、方针、政策。延安文艺界以街头诗,活报剧,木刻版画,大标语,大合唱等文艺形式和小说、剧本的创作,热情歌颂抗战和八路军、新四军,抒发对延安和敌后根据地的热爱,同时还根据中共的宣传口径,批评、抨击国民党。在党组织的支持下,一些作家还将自己的作品投寄给在国统区创办的左翼文艺刊物,中共在国统区的机关报《新华日报》也不时刊登延安作家的小说和诗作。在国统区刊物上发表的延安作家的作品,吸引了更多的青年向往中共和延安。

延安文艺界对毛泽东也显示出充分的尊重。一度挂名延安鲁迅艺术文学院院长的毛泽东曾多次在鲁艺发表演讲。在和鲁艺师生的接触中,毛表现出虚怀若谷的姿态,赢得文艺界人士的普遍好感。1938~年春,途经武汉转赴江南新四军的一位署名“大漠”的青年,特将他所记录的毛泽东在鲁艺的演讲稿整理成《毛泽东论鲁迅》,投寄给胡风主办的《七月》,此文在武汉的发表使毛泽东在中共领袖的身份之外,还增添了“革命文学理论家”的色彩\footnote{胡风:《一点回忆》,载《胡风晚年作品选》(南宁:漓江出版社,1987~年),页~81~—~82。}。1939~年底,毛泽东在和鲁艺戏剧系主任张庚交谈时提到延安文化活动不够活跃,提议排演曹禺的《日出》以丰富延安的文化生活\footnote{艾克恩:《〈在延安文艺座谈会上的讲话〉与延安文艺运动》,载艾克恩编:《延安文艺回忆录》,页~408~—~409。}。戏剧界人士闻风而动,经过二十多天的突击,于~1940~年元旦正式公演了《日出》;不久又在胡乔木的“关怀下”排演了《雷雨》、《钦差大臣》等一批中外名剧\footnote{艾克恩:《〈在延安文艺座谈会上的讲话〉与延安文艺运动》,载艾克恩编:《延安文艺回忆录》,页~408~—~409。},受到毛和延安广大干部和知识分子的欢迎。文艺界人士不曾想到,两年后,毛泽东会改变态度,将遵照他个人指示开展的所有这一切指责为“脱离群众”、“关门提高”、“顽强表现小资产阶级情感”的行为。

尽管延安文艺界为宣传中共政治路线竭尽了全力,但是作为政治家的毛泽东并不真正满意。他敏锐地感到在文艺界存在着一股与正在形成的新权威相对立的离心倾向,而这种离心倾向,在毛那里可以简化为两句话:文化人不听话,不尊重领导。

如果说“不听话、不尊重领导”是指文艺界人士在服务于党的政治目标的前提下,仍然保留了某些个人独立性和原有的审美习惯、语言特点和生活习性以及文艺家独立不羁的个性风格,那差不多是事实:

在这一时期,文艺界人士虽然尊重毛泽东,但并没有把毛泽东视为中共唯一领袖,尤其没把毛看成是精通文艺问题、可以指导自己创作的理论大师。不少文艺家还未养成尊重党在文艺方面领导人的习惯。一些来延安前就有较高成就的文艺家,“孤芳自赏”,独立特行,根本不把“文艺界领导同志”周扬放在眼中,经常对周扬的领导作风反唇相讥。

在来自全国不同地区的文艺界人士中,许多人还带着原有的小圈子的色彩,在延安就有以周扬为首的“鲁艺派”和以丁玲为首的“文抗派”两个摊子。1939~年,周扬担任“鲁艺”副院长,实际主持鲁艺工作,在周扬的周围逐渐聚合了何其芳、周立波、陈荒煤、沙可夫、沙汀、刘白羽、林默涵、贺敬之等。1939~年~5~月,经张闻天同意,全国文艺界抗敌协会延安分会(“文抗”)成立,主要成员有丁玲、萧军、舒群、艾青、白朗、罗烽等。两个摊子互相排斥,势如水火,经常在延安打笔墨官司。1941~年~6~月~17~至~19~日,周扬在《解放日报》发表《文学与生活漫谈》,以领导者的口吻指责延安一些作家说:“他们写不出东西却把原因归之为没有肉吃”。8~月~1~日,萧军、艾青、舒群、罗烽、白朗在争取到毛泽东的同意后,在延安“文抗”机关刊物《文艺月报》著文(此文寄给《解放日报》,但被退回),指斥周扬既“有自己的小厨房可以经常吃到肉”,却无端贬低其他人只知和首长闹平等争肉吃。萧军愤然写道:“到延安来的都不是为了来吃肉,是为了来革命;正如周扬到延安来不仅仅是为了当院长(鲁艺),吃小厨房和出门有马骑……一样。”\footnote{张毓茂:《萧军传》,页~230~—~31。}在这些纷争的背后,实际上存在着两种创作思想的分歧。周扬称“鲁艺派”主张“歌颂光明”,而“文抗派”主张“暴露黑暗”。\footnote{参见周扬:《与赵浩生谈历史功过》,载《延安文艺回忆录》,页~35、38。}延安文艺界的这些纷争似乎表明延安的文化人还生活在上海的亭子间里,一句话,延安的文艺界仍然沿着过去轨道运行,在逐渐形成的毛的“新秩序”下,文艺界自成天地,俨然是一个世外桃源。

其实,文艺界的“不听话”并非有意与毛泽东的“新秩序”对抗,而是三十年代左翼文艺运动的残阳余韵和正旭日东升的毛主义的不协调而迸发出的火花。

“红色的三十年代”是一个世界性的现象,法西斯主义的崛起和由对西方制度的怀疑、动摇而产生的幻灭感和深刻的精神危机,促使西方一部分知识分子将人类的前途寄托于斯大林进行的苏联共产主义试验上,因此,从三十年代初至~1939~年苏联参与瓜分波兰前,许多著名的知识分子纷纷向左转,而在向左转的知识分子中,情感丰富,且对政治和社会生活变化最具敏感性的文学家又占有最大的比重。

和欧美作家从三十年代初才逐渐左倾有所不同的是,中国作家的向左转则早在二十年代后期就开始了。由于中国知识分子生存的环境比西方作家更加恶劣,左翼作家普遍将文学作为参与社会改造的特殊工具,赋予文学以巨大的社会批判的功用。作为中国险恶的政治和社会生态环境的产物,以鲁迅为代表的中国左翼文学,一方面继承了五四新文学关照社会现实和直面人生的传统,另一方面又吸取了俄罗斯文学深厚的人道主义的精神,表现出强烈的社会批判性和激进化倾向。

中国左翼文学的另一特点是,左翼作家很少像他们的欧美同行那样以独立的个人形式向社会陈述自己的理念,而是结成比较紧密的集团,以集团的形式表达自己的文学和政治主张。受逐渐斯大林化的苏联共产党文艺政策和领导方式的影响,中共在~1928~年后将很大的精力投放在组织以上海为中心的“普罗文学”运动上,并在~1930~年~2~月领导成立了“中国左翼作家联盟”。文艺的集团化固然可以形成风潮,在吸引追随者方面发挥特殊作用,但也易于引发文艺家之间的纠纷和冲突。尽管瞿秋白、鲁迅、胡风、冯雪峰、周扬等以“左联”为依托,集合了大批的知识青年,使左翼文学的队伍越益壮大,但中国文艺界从此也就埋下了长期纷争的种子。

三十年代初、中期,由于日本帝国主义的入侵加剧了国内局势的动荡,加之俄式马克思主义文艺理论的广泛传播,中国左翼文学得到蓬勃的发展,蒋光赤、柔石、叶紫、萧军、萧红等一批作家迅速崛起,在他们作品中所表现出的强烈的社会关怀,引起了生活在社会底层的广大下层知识分子的普遍共鸣。贫穷至极的生活和对国民党统治下社会现实的不满,将一批批青年吸引到左翼文学的行列中,促使他们走上用文学反抗社会的道路。

在三十年代的上海、北平等大都市中,创作具有社会批判内容的诗歌、小说、剧本成为知识青年和大中学生显示“革命精神”和自我价值的一种时尚。三两个朋友,凑一、两百元钱就可以自费创办一份文学刊物,如果再将几篇作品结集印成小册子,一个作家或诗人也就诞生了。尽管国民党当局也曾颁布“图书审查条例”,但是国民党图书审查的重点对象只是鲁迅、胡风、茅盾、丁玲等少数著名作家,对活动在上海租界内和大中城市的一般作者,“图书审查条例”形同空文,几乎不产生作用。正是由于这些原因,三十年代的左翼文学家多如过江之鲫,文学青年则更是不可计数。当然,作品有高下之分野,作家也有影响大小之区分。如果某一作品被鲁迅、巴金等主办的刊物选用,再经胡风、周扬等革命文艺理论家为之点评,那么,该作者从此就名声大震,顿时从一般文学家中脱颖而出,成为有影响的著名作家了。然而有此殊荣的作家毕竟不多,多数人尽管已出版了数本创作集和译著,但知名度并不高,除了作家周围的朋友圈子,一般读者是很少与闻的。

对于所有这些已出名或尚不出名的左翼文艺家,国民党统治下的社会现实是令人窒息和不堪忍受的,与此相映照的则是苏联小说中展现的和在自己头脑中存在的莫斯科的“光明天地”,正是左翼文学强烈的批判意识和对新社会的渴望,最终将大批左翼文艺家吸引至中国的莫斯科——延安。对于理想主义和乌托邦新村生活,赴延安的左翼文艺家并不陌生,除了早在书本上与之神交外,部分左翼作家在三十年代还亲身体验过。以作家碧野为例,出身贫穷而酷爱文学的碧野在三十年代初身无分文只身来到文化古都北平,就曾与十一个境况相近的青年组成了一个共产主义新村寄住在潮州会馆。这些青年或写诗作文,或在北京大学旁听和去图书馆看书,过着极为贫困、但却是“有衣同穿、有饭同吃”的生活。在他们的脑海里,延安正是这样一个和他们心目中景仰的、“克里姆林宫塔尖上的红星光芒”照耀下的苏联完全相似的理想国,在那块土地上,充满着社会正义的原则,\footnote{碧野:《人生的花与果》,载《新文学史料》,1992~年第~2~期,页~54~—~63。}人们个性自由,到处都可以“自由呼吸”。所以不少左翼文艺家“到延安一进边区,就匍匐在地上亲吻土地”。\footnote{参见周扬:《与赵浩生谈历史功过》,载《延安文艺回忆录》,页~36。}诗人柯仲平更直接将延安比作但丁《神曲》中的天堂\footnote{参见王琳:《狂飘诗人柯仲平传》(北京:中国文联出版社,1992~年),页~416。}。

从~1937~年到~1939~年,投奔延安的文化人,除了个别名气较大由延安指名调入的人之外,绝大多数为一般的左翼青年知识分子,只有少数人读过大学。这些名气不大的作家和艺术家,甫抵延安一身轻松,又是“自由呼吸”,又是“自由歌唱”,而中共为了发展实力,对来延安的文化人一度也做到“礼贤下士”,毛泽东更是真真假假,对延安文艺界取得的成绩满口称赞,甚至还和一些文艺家交上了“朋友”,于是文艺家们陶醉了。

这是一批“单纯到透明”的青年人,尽管延安缺吃少穿,物质生活极为艰苦,但是他们毫无怨言,只把延安视为中国的莫斯科。现在他们又传唱着这样一首苏联歌曲:

\begin{quoting}
人们骄傲地称呼是同志,

它比一切尊称都光荣。

有这称呼各处都是家庭,

无非(分)人种黑白棕黄红。\footnote{韦君宜:《思痛录》(北京:十月文艺出版社,1998~年),页~5~—~6}
\end{quoting}

然而,毛泽东与延安文艺家的“蜜月”仅维持了两年。随着毛的政治地位日趋加强,和延安等级差序制度的逐渐形成,以毛泽东和红军老干部为一方,与以赴延安左翼文艺家为另一方的矛盾就逐渐公开化了。

逐渐地,延安的文艺家对一些老干部和老红军的思维习惯和生活方式看不顺眼了,许多青年文艺家竟发现原先心目中的英雄既无文化,更不懂马列主义理论,个别人开口“妈个屄”,闭口“妈个屄”,但却会摆“首长”的架势。于是他们开始拿起笔,批评起延安生活的阴暗面。

毛泽东和红军老干部对文艺家的不满则主要集中在文艺家的“无组织、无纪律”和个性的狂放不羁方面。1940~年,丁玲在延安“文抗”主办的文学刊物《谷雨》上发表批评延安官僚主义的著名小说《在医院中》,毛泽东没有说什么,但已对丁玲严加注意。1942~年~2~月~17~日毛泽东、王稼祥等参观华君武等主办的《讽刺画展》,毛虽对画展有所称赞,但在约华君武谈话时,却要求华君武在漫画创作上“要注意片面性”。毛并举华君武一幅发表在《解放日报》上谈植树的漫画,说“不要笼统说延河旁所有植的树都枯死了,应当说是那一段地点的树,植得不好”。\footnote{参见《延安整风运动纪事》,页~67;华君武:《延安的漫画活动》,载孙新元、尚德全编:《延安岁月——延安时期革命美术回忆录》(西安:陕西人民美术出版社,1985~年),页~137~—~38。}

如果说毛泽东对延安文化人的不满更多的集中在政治方面,那么一些老干部的不满则来自于对知识分子根深蒂固的怀疑和歧视。许多红军老干部从未进过城市,更长期生活在疑惧知识分子的环境中\footnote{仇视和乱杀知识分子最严重的是张国焘领导的红四方面军及鄂豫皖、川北根据地。1937~年进入延安中央党校的原红四方面军的干部中,许多人明明识字,却硬伪装成文盲,惟恐因识字而遭清算,参见《成仿吾传》编写组编:《成仿吾传》(北京:中共中央党校出版社,1988~年),页~111;知识分子出身的干部还有一个轻蔑的称谓,叫作“白脚杆”。参见黄火青:《一个平凡共产党员的经历》,页~126。},抗战之初,一经接触满口新名词、思想和生活习惯与己迥异的知识分子,虽觉新鲜,但为时不久,隔阂即生,生怕这些见过大世面的诗人瞧不起自己。于是“非我族类,其心必异”之类的念头再度复活,重又退回到原先怀疑、恐惧知识分子的蜗牛壳。

作为志在夺取天下的毛泽东,对待知识分子的态度自然有别于某些目光短浅的军中老干部(需要指出的是多数军中高级将领,如朱德、彭德怀、刘伯承、邓小平、聂荣臻、罗荣桓、叶剑英、陈毅、徐向前、关向应等都十分重视、尊重知识分子,工人出身的关向应因嗜读鲁迅作品,有“小鲁迅”之称。对知识分子抱有偏狭态度的只是个别高级将领和军中中下级干部)。毛一方面需要文艺家,另一方面又恐驾驭不住文艺家,因而一段时间内,毛的态度摇摆不定,既投之以饵,间或鼓励抚慰一番;又从内心深处鄙视舞文弄墨、吹拉弹唱的文艺家;更厌恶于彼等日空一切、抗上犯下之狂态,故时不时对文艺家旁敲侧击一下。只是这群文艺家旧习难改,依然故我,因而毛在~1942~年后对文艺家越发讨厌。据师哲回忆,一次他在陪毛泽东从杨家岭出来在延河滩散步时,正巧碰上一位作家,只见他手执拐杖不停飞舞,见到毛泽东打了个招呼,仍继续摆动手中的拐杖,遇到驮盐的毛驴队,仍然如是。毛泽东极为气愤,向师哲道,“这是流氓行径,目中无人。……只是因为他认识几个方块字,就不把老百姓放在眼里,坐在人民的头上显威风……目空一切,摆臭架子!”\footnote{师哲:《在历史巨人身边——师哲回忆录》,页~237。}当年在延安文艺界人士中有用拐杖习惯的只有萧军、塞克等数人,他们又何尝能想到,被自己称为“知己”的毛泽东,对自己的冒失和莽撞会如此愤怒,竟然在哲面前将其骂作“流氓”!

更为严重的是,毛泽东对延安文艺界人士的反感又和对张闻天的不满掺和在一起,使毛和延安文化人的矛盾更趋复杂化。

张闻天是继瞿秋白之后中共党内对文艺问题有专门研究的极少数高级领导人。张闻天青年时代曾直接从事过文学创作,他具有比其他中共领袖丰富得多的文学素养和较高的艺术鉴赏力。抗战以来,张闻天作为中共文宣方面最高负责人,多次代表中共中央就文艺问题发表意见,但是张闻天在有关文艺的几个重要问题上的观点和毛泽东的看法并不一致。

关于新文化的性质问题。张闻天在~1940~年~1~月~5~日至~7~日作的《抗战以来中华民族新文化运动与今后任务》(又称《文化政策》)的报告较毛泽东提前四天,对新文化的性质作了具体阐释\footnote{张闻天:《抗战以来中华民族新文化与今后任务》,载《张闻天选集》(北京:人民出版社,1985~年),页~252~—~54。}。张闻天认为所谓新文化,它应是“民族的、民主的、科学的、大众的”。在这里,张闻天将“民主的”单独列出,表明他对新文化民主特性的重视。然而毛泽东在也年~1~月~9~日作的题为《新民主主义的政治和文化》(以后改名为《新民主主义论》)的报告却对此问题话焉不清。毛泽东指出,“民族的科学的大众的文化,就是人民大众反帝反封建的文化,就是新民主主义的文化,就是中华民族的新文化”,“这种新民主主义的文化是大众的,因而也是民主的”——毛泽东将“大众的”等同于“民主的”,是疏忽抑或是别具用心,人们不得而知。但显而易见,“大众的”决不能取代“民主的”,因为“大众的”既有专制愚昧的内容,也可有民主科学的内容,毛以“大众的”代替“民主的”,实际上是模糊了新文化民主的性质。

关于“大众化”、“中国化”的问题。张闻天早在~1937~年~11~月就初步接触到这个问题。1940~年~1~月,张闻天针对延安文艺界已出现的孤立强调通俗化的倾向,明确指出“通俗化不是曲解新文化,使新文化庸俗化”,张闻天强调,“无论如何,现代文艺的各种形式比较中国旧文艺的形式是进步的”。\footnote{引自程中原:《张闻天论稿》(南京:河海大学出版社,1990~年),页~346~—~47、349。}张闻天的上述言论是对在毛泽东鼓励下正在兴起的复归旧传统倾向的一次回击。

关于对知识分子的态度问题。张闻天在对待文化人的工作方式、个性特点、生活习惯等方面都表现出充分的理解和尊重,其态度远较毛泽东温和。张闻天在~1940~年~10~月~10~日代中共中央宣传部、中共中央文化工作委员会起草的《关于各抗日根据地文化人与文化团体的指示》中,要求“纠正党内一部分同志轻视、厌恶、猜疑文化人的落后心理”。他指出:“爱好写作,要求写作,是文化人的特点”,“文化人的最大要求,及对文化人的最大鼓励,是他们的作品的发表”,因此,“应该用一切方法在精神上,物质上保障文化人写作的必要条件”,“应该在实际上保证他们写作的充分自由”。张闻天进而提出,为了“保证文化人有充分研究的自由与写作时间”,“(文化)团体内容不必要很严格的组织生活与很多的会议”。至于对作家的批评,“应采取严正的、批判的、但又是宽大的立场,力戒以政治口号与偏狭的公式去非难作者,尤其不应出以讥笑怒骂的态度”。\footnote{《张闻天选集》,页~290~—~93。}张闻天的上述较为开明的意见,源于五四新文化运动对他的影响,以及三十年代初在上海领导左翼文化运动时的经验。相比较于毛泽东,张闻天更熟悉五四以后新文化运动发展的历史。尽管张闻天不能彻底摆脱中共党内长期存在的反智主义传统的影响,但是在对待知识分子的态度上,张闻天以及博古等人却比毛泽东多了一分宽容,而少了许多农民式的狭隘。\footnote{据在延安《解放日报》与博古同过事的丁玲、舒群等人的回忆,博古对待下属亲切随和,在《解放日报》文艺副刊遭到来自杨家岭方面的指责时,博古一般都率先承担责任,从不对下属横加指责。1942~年~4~月初,当丁玲在毛泽东主持召开的批判王实味的高干座谈会上,遭到贺龙、曹轶欧等人猛烈攻击时,博古悄悄坐到了丁玲身旁,宽语安慰了丁玲。事隔四十年后,当丁玲回忆起当时情景时,还禁不住写下她对博古深深的感激。博古也保护过《解放日报》副刊编辑陈企霞。参见丁玲:《延安文艺座谈会的前前后后》,载艾克思编:《延安文艺回忆录》,页~62。另参见陈恭怀:《陈企霞传略》,载《新文学史料》,1989~年第~3~期,页~181。}1943~年,正是由于张闻天和博古的保护,才使~1941~年~11~月投奔延安的五四时期的著名诗人高长虹免遭中央社会部的逮捕。\footnote{言行:《高长虹传略》,载《新文学史料》,1990~年第~4~期,页~198。}而张闻天在有关党的文艺政策及对知识分子政策方面的所有意见,在整风运动期间,都被毛泽东斥之为“自由主义”而遭到严厉的指责\footnote{《关于延安对文化人的工作的经验介绍》(1943~年~4~月~22~日党务广播),载《新文学史料》,1991~年第~2~期,页~188、138。}。

如果说~1941~年前,毛泽东因忙于应付内外形势和巩自己的政治地位尚无暇顾及文艺问题,对张闻天的一些与己不同的看法还能容忍的话,那么到了~1942~年当毛已腾出手时,他就再也不能允许张闻天就文艺问题说三道四了。

于是,王实味事件的爆发,就成了毛泽东整肃延安文艺界的最佳突破口,毛泽东决心利用这一“反面典型”,扩大战线,一并收拾延安所有的文化人,以求一劳永逸地解决文化界的所有问题,最终确立自己作为文艺界大法师的至高无上的地位。

\section{延安文艺座谈会与毛泽东“党文化”观的形成}

在中共党内,毛泽东是得到全党公认的首屈一指的学问大家。毛具有党内无人企及的极其丰富的中国传统文化的知识,他不仅极其熟悉并爱好唐诗宋词、《昭明文选》、红楼、水浒、三国、野史稗记一类古典文学,同时也嗜读鲁迅杂文,然而毛对鲁迅之外的五四以来的新文学作品却很少涉猎,一是兴趣不大,二是长年深居军中无机会阅读。毛对外国文学作品就知之更少。

毛泽东的“无产阶级文艺理论”有两个来源,一是他的助手为他准备的列宁、斯大林有关文艺问题的部分论述,第二是他在周扬、胡乔木等向他提供的三十年代中共领导上海左翼文艺活动的背景资料基础上所作的总结。

毛泽东在文艺方面的主要顾问是周扬和胡乔木这两个新朋友,而不是昔日的熟人冯雪峰。和张闻天、博古、杨尚昆等人完全不同,毛在~1937~年以前与上海左翼文艺界几乎毫无联系。1933~年底,与鲁迅关系密切、曾任中共中央宣传部文化工作委员会书记的冯雪峰从上海到达江西瑞金,毛泽东曾约见冯雪峰,两人进行过一番有关鲁迅的著名谈话。\footnote{冯夏熊(冯雪峰之子),《冯雪峰——一位坚韧不拔的作家》:载《回忆雪峰》(北京:中国文史出版社,1986~年),页~12~—~13。}但是,1933~年底到~1934~年~10~月,正是毛泽东在政治上最为失意的时候,毛除了向冯雪峰打听鲁迅的情况,对上海左翼作家的活动并无很大兴趣。毛与担任马克思共产主义大学副校长的冯雪峰也很少接触,更谈不上彼此间已建立了良好的个人关系。在这个时期,中共文艺工作的元老、担任中华苏维埃和国教育部长的瞿秋白虽和毛泽东偶有接触,但彼此都是被冷落的人物,心境不佳,加之双方性格并不投合,因而也无雅兴讨论文艺问题。遵义会议后,面对繁重的军务与急剧变化的形势,毛泽东更是无暇顾及文艺问题。

长征结束后,毛泽东、张闻天等决定启用冯雪峰,让其秘密赴沪执行一项特殊任务,此时中共生存乃是压倒一切的头等大事,文艺工作一时还排不上主要议事日程。1936~年~4~月上旬,毛泽东、张闻天、周恩来指派冯雪峰携电台和活动经费秘密返回上海,临行前,周恩来、张闻天向冯雪峰交待的任务是,在上海建立电台,与沈钧儒等上海救国会领袖取得联系,重新恢复中共在上海的组织和情报系统。“附带管一管”左翼文化活动。\footnote{有关这段历史事实有两项已得到互相证实的资料来源,一是冯雪峰在文革期间所写的“交待材料”:《有关~1936~年周扬等人的行动以及鲁迅提出“民族革命战争的大众文学”口号的经过》,另一为张闻天的《1943~年延安整风笔记》,参见《雪峰文集》,第~4~卷(北京:人民文学出版社,1985~年),页~506~—~507;另参见程中原:《张闻天论稿》,页~492~—~93。}1936~年~1~月~25~日,冯雪峰抵沪,遵照张闻天的吩咐,第二天即移居鲁迅家中。出于特殊的谨慎,冯雪峰返沪后,没有立即与以周扬、胡乔木为首的中共文委系统取得联系(冯雪峰相信陕北有关中共在沪地下组织已全部被国民党破坏的说法),4~月~27~日冯雪峰与鲁迅、胡风商议提出了“民族革命战争的大众文学”的口号,随即以鲁迅为一方,和以周扬为另一方的“两个口号”的论争正式爆发。

冯雪峰乃是一文化人。他虽衔重大使命来沪,但是冯的兴趣仍在他过去领导过的左翼文化方面。在他于~1936~年~12~月领导、组织了中共上海临时工作委员会后,冯雪峰就将有关中共组织与情报工作交由潘汉年等承担,他自己则将主要精力集中于文化界。1937~年~1~月冯雪峰返陕北向毛泽东等汇报后又折回上海,遵延安命将中共上海临时委员会全盘工作向刘晓作了移交。至此,冯雪峰的“中央代表”的身份即告结束。然而由于冯雪峰曾疏远周扬而与鲁迅、胡风关系密切,已触犯了周扬等人,激起了周扬等的极度不满,周扬等产生了被抛弃、冷落的深深不平感,并拒绝与冯雪峰见面\footnote{对于~1936~年~4~月下旬,冯雪峰衔中共中央命来沪,没有先找周扬接头,而是住进鲁迅家一事,周扬在~1979~年还为此耿耿于怀。参见《周扬关于现代文学的一次谈话》载《新文学史料》,1990~年第~1~期,页~125。上海中共临时文委另一领导人夏衍在事隔五十年后也仍对冯雪峰当年的这一行动表示了强烈的不满,参见夏衍:《懒寻旧梦录》,页~313~—~15。}。

七七事变爆发后,周扬、艾思奇、何干之、王学文被指名调往延安,不久周扬被任命为边区教育厅长,旋又被任命为鲁迅艺术文学院副院长,实际主持鲁艺的工作(鲁艺院长为吴玉章,但他并不到院主事),这样就和毛泽东建立起工作上的联系。恰在这时,冯雪峰因与中共驻南京代表团负责人博古发生严重争执,一气之下,竟向潘汉年请长假,于~1937~年~12~月返回家乡浙江义乌,脱离中共组织关系长达两年之久,至~1939~年下半年才由中共中央东南局恢复了组织关系。冯雪峰此举带来严重后果,毛泽东从此埋下了对其反感、厌恶的种子,而周扬则在延安与毛的关系日益接近,周扬的才干逐渐引起毛的注意。

周扬原名周起应,最早是以俄苏文学翻译家于三十年代中期在上海左翼文化界崭露头角的。自~1933~年上海左翼文化运动的领导人瞿秋白、冯雪峰相继进入江西中央根据地后,周扬就成了左联和中共文委的领导人。1935~年~2~月,周扬躲过了国民党对中共上海中央局的毁灭性大逮捕,与夏衍、胡乔木等组织了中共临时文委,团结了一百余名文化界的中共党员。但此时周扬领导的中共临时文委实际上已和长征中的中共中央毫无组织联系。在“左联”前期,周扬曾译过几本介绍苏联文学、音乐的读物,编译了《高尔基创作四十周年纪念论文集》,和周立波合译过一本介绍苏联大学生生活的长篇小说,但基本上没有自已创作的文学作品问世,因而曾被鲁迅讽刺为“空头文学家”。但是,周扬却因政治倾向的因素和对俄苏文学的爱好,对俄国十九世纪别林斯基、杜勃罗留波夫和车尔尼雪夫斯基的文学理论,以及苏联共产党文艺政策及日共文艺理论十分熟悉。1937~年周扬在上海生活书店出版了他最著名的译著《安娜·卡列尼娜》,加上他曾撰写过一些介绍苏联社会主义现实主义文艺理论的文章,因此当周扬赴延安时,除了翻译家的头衔外,他已获有“文艺理论家”的声誉。

周扬的“文艺理论家”的身份,在延安得到重视。在~1937~年后前往延安的文化人中间,尽管作家、诗人、艺术家比比皆是,但是“文艺理论家”却寥如晨星,尤其周扬还有前中共上海临时文委领导人的政治身份,所以周扬很快被委以重任。1937~—~1940~年,周扬紧密配合毛泽东对文艺工作的有关指示,经常在《解放》周刊、《新中华报》发表阐释文章。\footnote{例如周扬在~1938~年~6~月~8~日《解放》周刊上,发表的《新的现实与文学上的新的任务》一文,较早提出作家的创作必须随着生活环境的变化而改变,应重视文艺的通俗化和大众化的问题,周扬的上述看法与毛泽东的看法完全合拍。参见《周扬文集》,第~1~卷(北京:人民文学出版社,1984~年),页~246~—~47、251。}由于周扬善于引述列宁、车尔尼雪夫斯基的言论为毛的论点作注脚,周扬逐渐赢得毛的信任,在~1942~年以前,就成了毛在文艺方面的首席代言人。

在周扬为确立自己在毛泽东那里的地位而努力奋斗时,周扬昔日的朋友胡乔木发挥了重要的作用。1935~—~1937~年,胡乔木在沪活动期间,是躲在“左联”和“左翼社联”幕后活动的一个不十分引人注目的角色,虽然胡乔木并没有写出有较大影响的作品,但他却是~1935~年以后以周扬为首的中共上海临时文委的主要成员与周扬有着密切的关系。1936~—~1937~年,胡乔木经历了周扬与冯雪峰等的对立与冲突。是属于周扬派的主要成员,但胡乔木在“两个口号”论争中较少出头露面,因而没有引起外界的注意。1937~年~7~月,胡乔木虽较周扬早几个月进入陕北,但长期被留置在安吴堡青训班,和在延安正日渐走红的周扬形成了强烈的反差,但是到了~1941~年~2~月胡乔木调入毛泽东身边做政治秘书,情况就发生了戏剧性的变化,胡乔木的地位迅速超过了周扬。由于胡乔木对三十年代上海左翼文化活动的历史十分熟悉,他成了毛了解文艺问题的主要顾问之一(江青也起类似的作用),而胡对“左联”内部矛盾等问题的看法,十分自然地对毛泽东产生了影响。

1939~年胡乔木调回延安,胡乔木与周扬在延安又有了经常接触的机会。此时胡乔木已较少以自己的名字公开在报刊上发表文章,周扬则在精心研究如何运用列宁、斯大林有关文艺问题的论述为毛泽东构筑文艺理论。周扬集中精力翻译了车尔尼雪夫斯基的《生活与美学》,试图从这位俄国革命文艺理论家那里发掘出“文艺为政治服务”的信条。然而周扬对毛作出的最大贡献无疑是他精心编选的马克思、列宁、斯大林、鲁迅等论文艺的语录。\footnote{周扬翻译的车尔尼雪夫斯基论著《生活与美学》于~1942~年由延安了新华书店出版,周扬选编的《马克思主义与文艺》的部分内容自~1942~年始陆续刊载,1945~年由延安解放社正式出版。}

周扬、胡乔木对毛泽东的另一个重大帮助则是向他介绍有关“两个口号之争”。对于“两个口号”的争论,早在~1937~年~1~月,冯雪峰返回延安向中共中央汇报时,毛泽东就有所了解。1937~年~9~月后,周扬、艾思奇、王学文等抵达延安后,毛泽东又从另一方(“国防文学派”)那里,详细解了双方的观点及其有关背景。1938~年~5~月,毛泽东还与“国防文学派”的重要人物徐懋庸作了详细交谈。从陈伯达处,毛更了解到既非周扬派,又非鲁迅、冯雪峰派的第三种看法。在周扬与鲁迅两种对立的观点间,尽管毛泽东较倾向于鲁迅提出的“民族革命战争的大众文学”的口号,而不大欣赏“国防文学”这个烙有王明印记的文学主张,但这只是问题的一个方面。毛泽东毕竟不是文艺家,他根本无意拘泥于这种文人间的无谓的争论,毛所关心的是“国防文学”的口号确实有助于实现文艺家的抗日统一战线,而眼下帮助自己构筑文艺体系的正是这帮“国防文学派”。现在,提出“民族革命战争大众文学”口号的鲁迅已经作古,冯雪峰又目无组织私自离队,胡风虽然在国统区十分活跃,然此人既非中共党员,政治面目且十分可疑,毛泽东毫无必要为了已死的鲁迅而抛弃周扬。于是毛泽东一锤定音:两个口号无优劣之分,皆是革命口号,\footnote{参见徐懋庸:《我和毛主席的一些接触》,载《徐懋庸回忆录》(北京:人民文学出版社,1982~年),页~103~—~104。}毛泽东并在私下对周扬说,鲁迅也有“党八股”,\footnote{《周扬关于现代文学的一次谈话》,载《新文学史料》,1990~年第~1~期,页~124。}对周扬表示了充分的理解。

对于鲁迅,毛泽东所持的是政治功利主义的态度。从读者的角度,毛固然惊叹、钦佩鲁迅对中国历史、社会和中国国民性的深刻认识和尖锐的剖析,毛也激赏鲁迅对三十年代左翼文艺家种种浅薄所作的无情的揭露和抨击。但是,毛泽东在阅读鲁迅作品时更多的是站在中共领袖的角度,从这个角度出发,鲁迅对于毛泽东有时就显得并不可爱。鲁迅蔑视一切权贵的个性和独立不羁的精神虽然可以与毛的精神世界形成沟通,但毛决不认为,在延安也需发扬这种精神。至于鲁迅倡导的抨击时弊、揭露丑类的杂文形式,毛明确无误地表示,在共产党区域,它已基本失去作用。有鉴于鲁迅对毛的价值及其局限性,毛迫切需要将鲁迅和鲁迅的遗产修改成适合于自己政治需要的样式:一方面,毛要利用鲁迅打击左翼文化人的“虚狂”;另一方面又要创作出鲁迅如何“服从共产党领导”的故事,以防止文化人援引鲁迅来反抗延安的新秩序。正是基于这样的需要,毛泽东才看中了被鲁迅鄙视但对自己言听计从的周扬。与鲁迅有隙、生恐有人援鲁迅反对自己的周扬,对塑造毛氏版本的鲁迅新形象最为积极,周扬又因在上海左翼文化界党同伐异,树敌甚多,只会在延安的新环境下更猛烈地打击文艺界的“小资产阶级的作风”。于是被毛慧眼看中的周扬就在胡乔木的暗助下,摇身一变,成为毛氏鲁迅学最权威的解释者和统领延安文艺界的“领导同志”。

1942~年春,毛泽东在胡乔木的协助下,分别召集了一批文艺界人士前来住处个别交谈。在被毛召见的文艺家中,既有属于周扬派的陈荒煤、何其芳、刘白羽;也有当时处于受压制状态的萧军、艾青等。经过一番“调查研究”,毛泽东认为将自己已成熟的文艺主张公之于众的时刻到来了。

1942~年~5~月~2~日,有一百余人参加的延安文艺座谈会正式开始,毛泽东发表讲话,5~月~23~日,毛又在座谈会上作总结性发言,是为有名的《在延安文艺座谈会上的讲话》,此文几经修改,发表于次年~10~月~19~日的《解放日报》,毛泽东的这篇报告,标志着毛氏“党文化”观的正式形成。毛氏“党文化”观直接师承斯大林,与具有极其强烈的政治功利性和反艺术美学的日丹诺夫主义一脉相承,作为有中国特征的“党文化”观,毛的文艺思想则较俄式的“党文化”观更加政治化,表现出更浓厚的反智色彩。

毛氏“党文化”观包含下列五个核心概念:

一、文艺是政治斗争的工具,革命文艺的最高目标和最重要的任务就是利用文艺的各种形式为党的政治目标服务。具体而言,中共领导的文艺的基本方向是“工农兵方向”,“创作自由”是资产阶级的虚伪口号,文艺家只能,也必须以此方向作为自己的创作原则和创作内容。革命的文艺家应心甘情愿地做革命的“齿轮和螺丝钉”。

二、和工农兵相比,知识分子是最无知和最肮脏的,文艺家的主体意识是资产阶级个人主义的无稽之谈,因此知识分子必须永远接受“无产阶级”的改造。

三、人道主义、人性论是资产阶级文艺观的集中体现,革命文艺家必须与之坚决斗争和彻底决裂。绝不允许描写工农兵在反抗、斗争之外的任何属于非阶级意识的表现。

四、鲁迅的杂文时代已经过去,严禁暴露革命队伍中的阴暗面。

五、反对从五四新文化运动遗留下的文艺表现形式上的欧化倾向,文艺家是否利用“民族形式”并不仅仅是文艺表现的个别问题,而是属于政治立场和世界观的重大问题。

上述五个方面的内容包括了从创作主体、文艺功能,到创作题材和创作形式等文艺学的所有领域,构成了一个严密的党文化体系。从表面上看,毛氏党文化观竭力强调文艺的社会和政治功用,与中国文学中的“文以载道”的传统有某些相似之处,但“文以载道”并不意味着可以取消文艺的审美功能,将文艺等同于试帖课和八股文。毛泽东在这里将“文以载道”夸大到极端的地步,同时又吸取了明清颜习斋等鄙薄读书人的反智思想,结合斯大林主义,最终建立起具有强烈民族主义色彩的中共官方文艺路线。一言以蔽之,毛氏文艺思想的实质是将文艺视为图解政治的宣传工具,将文艺家看成是以赎罪之身(身为知识分子的“原罪”)为党的中心工作服务的“战士”。

毛泽东极为看重文艺界对其《讲话》的反应。延安文艺座谈会一结束,周扬就立即担负起宣传、解释毛泽东党文艺观的新使命,从此延安文艺界的气氛开始发生根本性的变化。伴随这种变化,以周扬为核心的中共文艺统制体系也迅速得到确立,周扬终于成了名副其实、手执响鞭的“奴隶总管”。

1942~年后,中共对文艺家的行政和思想控制进一步严密化,延安及各根据地文艺刊物、文艺团体被完全置于各级党委宣传部门和由周扬领导的官办文艺领导机构的领导之下,自发的文艺刊物已不复存在,所有的文艺家均被纳入各类行政组织之内。在延安的文艺家只有萧军不堪被日渐官僚化的体制所束缚,于~1943~年~12~月自我放逐,前往延安县川口区第六乡的刘庄,与妻子儿女劳动垦荒,过着几近穴居人的原始生活,半年后萧军全家返回延安,仍旧被纳入到行政组织之内\footnote{张毓茂:《萧军传》,页~241~—~46。}。

和苏联作家协会一样,由周扬领导的鲁艺和边区文协承担着许多与文艺无关的政治功能。延安鲁艺在~1942~年后与康生领导的中央社会部和各单位的审干肃反领导小组密切合作,在延安文艺界大挖“特务”,结果文艺家大多成为特务嫌疑和被“抢救”的对象。1943~年延安几乎全部的文艺界人士都被集中到中央党校第三部,在那里逐个接受严格的政治审查。

对于周扬来说,配合中央社会部在文艺界肃反固然是压倒一切的政治任务,但是周扬要提高自己在毛泽东心目中的地位,主要还取决于他能否组织创作出一批体现毛氏党文化观的文艺作品。于是周扬将少数作家暂时调出中央党校第三部和鲁艺,住进设于桥儿沟的“创作之家”,\footnote{延安“创作之家”名义上由中共中央西北局领导,西北局宣传部具体主管,但周扬作为西北局成员,实际上负责“创作之家”。}希望他们能够创造出体现毛泽东文艺思想的文艺作品。

被周扬挑选,获准进入“创作之家”这个“世外桃源”的作家大致分四类:

最有希望和最有可能在近期内创作出符合毛泽东党文化观的作品,思想较为“纯正”的作家,如周而复、杨朔;

思想虽未完全改造好,但具有特殊才能,可能创作出为党急需的文艺作品的非党文艺家,如塞克夫妇、艾青夫妇;

少数资历较深,一时无法安置的重要作家,如从苏联返回延安的萧三;

个别来自国统区且较具影响力,有可能返回重庆的非党作家,如高长虹。

住进“创作之家”的大多数作家,除了高长虹一人外,\footnote{在“抢救运动”中,高长虹因直接向中共中央提意见,甚至要向斯大林提意见,被康生诬为青年党,幸而得到张闻天、博古的保护,才未入狱,但高长虹很快从公众生活中消失了。参见言行:《高长虹传略》,载《新文学史料》,1990~年第~4~期,页~198.}都对能在急风暴雨的审干抢救高潮中被保护,心存感激,他们中的一些人果然不负周扬的厚望,创作出最早一批体现毛氏党文化观的文艺作品,使周扬在毛泽东面前挣足了脸面,巩固了自己的地位。

延安文艺座谈会召开之后,毛泽东、胡乔木还加强了对周恩来领导下的重庆左翼文化界的干预。自抗战爆发以来,周恩来在武汉、重庆团结了大批文艺界人士,在这些人中既有原属周扬系统的夏衍、田汉,又有与鲁迅关系密切的胡风,还有中立作家老舍、巴金等。对于老舍等非中共作家,周恩来十分尊重,即便对于思想一贯左倾的胡风,周恩来一般也不干预他的创作活动,相反,在胡风创办刊物遭遇困难时,还曾给予经济上的资助。\footnote{1943~年,胡风从桂林回到重庆,为续办《七月》重新登记事,向周恩来求援,周一口答应,开给胡风一张三万元保证金的支票,1945~年春胡风主编的《希望》正式出版。参见胡风:《再返重庆》(之二),载《新文学史料》,1989~年第~1~期,页~34;另参见胡风:《关于〈七月〉和〈希望〉的答问》,载《胡风晚年作品选》,页~122。}重庆《新华日报》的副刊上,不时刊载胡风和与胡风关系密切的文艺家有关文艺理论的文章,这一切在~1942~年之前均未受到延安的非议。

但是到了~1943~年,情况发生了很大的变化。毛泽东、胡乔木在已完全掌握了延安的文宣机构后,将手掌伸向了周恩来领导的《新华日报》。延安对《新华日报》和重庆几家有中共背景的刊物上发表的乔冠华、舒芜等几篇烙有胡风“主观战斗精神”的文章十分不满,指责《新华日报》“发表了许多自作聪明错误百出的东西,如××论民族形式,×××论生命力,×××论深刻等”。\footnote{参见中国社会科学院新闻研究所编:《中国共产党新闻工作文件汇编》;上卷(北京:新华出版社,1980~年),页~89。1943~年~12~月~16~日,董必武在《关于检查〈新华日报〉、〈群众〉、〈中原〉刊物错误的问题致周恩来中宣部电》中表示已“依据中宣部指示对于怀(即乔冠华)观点作进一步检查”,见《中华儿女》,1992~年第~2~期;另参见《中国共产党新闻工作文件汇编》,上卷,页~139~—~40。}显然,在毛泽东、胡乔木的眼中,《新华日报》等以工作环境特殊为借口,提倡感性生活,强调作家的主体意识,实际上是在宣传与毛氏“党文化”观完全对立的“资产阶级”文艺主张。

为了彻底纠正《新华日报》的自由主义倾向,1944~—~1945~年,毛泽东、胡乔木特派几位已经“改造好”了的周扬手下的重要干部林默涵、何其芳、刘白羽、周而复、袁水拍等从延安前往重庆,宣传毛泽东的《在延安文艺座谈会上的讲话》,对重庆的左翼文化界人士进行知识分子必须加强思想改造的现身说法式的训导。然而对于何其芳等人的训导,除了郭沫若等极少数人之外,重庆大多数左翼文艺家都以沉默相对,只有胡风一人公开表示了少许的疑问和保留,而冯雪峰则十分不以为然。\footnote{1944~年~3~月~18~—~19~日,胡风参加了由冯乃超主持召开的重庆左翼文艺界人士学习毛泽东(讲话)的座谈会,胡风在发言中强调左翼作家在国统区的任务是与国民党的“反动政策”和“反动文艺”作斗争,而不是“培养工农兵作家”,胡风也未谈思想改造的问题。不久,何其芳又向重庆左翼作家介绍延安思想改造运动,引起与会者的强烈反感。“会后就有人说:好快,他已经改造好了,就跑来改造我们!连冯雪峰后来都气愤地说:‘他妈的!我们革命的时候他在哪里?’”参见胡风:《再返重庆》(之二),载《新文学史料》,1989~年第~1~期,页~35;另参见李辉:《胡风集团冤案始末》(北京:人民日报出版社,1989~年),页~50~—~52。}然而胡风、冯雪峰并不知道,延安方面早已在密切注意着他们对毛氏党文化观的任何反应,为他们以及其他国统区的作家,建立了一本“功过簿”,以待日后进行总清算。\footnote{何其芳等回延安后,~向毛泽东汇报了去重庆传达《讲话》的情况。胡风写于~1945~年初的论文《置身在为民主的斗争里面》,长期以来被认为是胡风反对毛泽东《讲话》的“罪证”,胡风在这篇文章里不指名地批评了将“思想改造”庸俗化为“善男信女式的忏悔”的倾向。1948~年中共果然在香港组织了对胡风的第一次大规模批判。冯雪峰则在~1945~—~1946~年发表了《论艺术力及其它》、《论民主革命的文艺活动》、《题外的话》等一系列文章,系统地抨击了正甚嚣尘上的文艺机械论和公式主义,冯雪峰尖锐指出:“研究或评价具体作品,用什么抽象的‘政治性’、‘艺术性’的代数学式的说法,也说是甚么都弄糟了。如果这样地去指导创作,则更坏”。冯雪峰的上述看法,在当时就被认为是“反对毛泽东的”。参见陈涌:《关于雪峰文艺思想的几件事》,载《回忆雪峰》,页~216。1946~年~6~月~10~日至~11~日,何其芳在《解放日报》连载《关于现实主义》一文,抨击画室(即冯雪峰)对毛《讲话》的态度。冯雪峰对毛泽东的《讲话》的态度直接影响到~1949~年后他的政治前途,1954~年~12~月~31~日。毛泽东将冯雪峰的诗和政治寓言《火》和《火狱》批转给刘少奇、周恩来、朱德、陈云、邓小平、彭真、陈伯达、胡乔木等人传阅,明显表明对冯雪峰的严重不满。参见《建国以来毛泽东文稿》,第~4~册(北京:中央文献出版社,1990~年),页~644。1957~年~8~月~27~日,冯雪峰被正式宣布是“三十年一贯的反党分子”,并被冠以“右派”帽子。}

如果说,在重庆和大后方的左翼作家,在~1942~年后为自己所受到的毛氏“党文化”的压力而感到困惑,那么他们所承受的压力也就是这么一些,他们毕竟还未失去创作上的自由,而在延安的文艺家则面临着迫在眉睫的选择:或者走王实味自我毁灭的道路,或者走何其芳抛弃旧我、追求“新生”的自新之道。延安的文艺家在经历了最初的震撼后,纷纷开始“脱胎换骨”。丁玲也许是延安文艺家中最早“脱出”的一个。在~1942~年~4~月初毛泽东亲自主持的关于王实味、丁玲问题的高干学习会议后,丁玲来了一个一百八十度的大转弯。丁玲在~6~月~11~日中央研究院召开的王实味思想批判座谈会上,一方面斥骂王实味“卑劣、小气、反复无常、复杂而阴暗”,号召“反对一切对王实味还可能有的小资产阶级温情,人道主义,失去原则的,抽象的,自以为是的‘正义感’”;另一方面,又自恼自责,称自己的《三八节有感》“是篇坏文章”。要求那些同情她遭遇的读者“读文件去吧”。\footnote{丁玲:《文艺界对王实味应有的态度及反省》(1942~年~6~月~11~日)载《丁玲集外文选》(北京:人民文学出版社,1983~年),页~134~—~37。}为了摆脱与王实味的牵连,丁玲在~1942~年~10~月~19~日延安纪念鲁迅的大会上,还极不礼貌地对待在王实味问题上仗义直言的萧军,声称中共的朋友遍天下,丢掉箫军,不过是九牛一毛。\footnote{参见王德芬:《安息吧,萧军老伴!》,载《新文学史料》,1989~年第~2~期,页~108。}其实在丁玲“布尔什维克的战斗性”的背后掩藏着很深的痛苦,几十年后,她自称《三八节有感》使她“受几十年的苦楚”,给她带来一生的灾难,因而不敢随意为文,生恐“又自找麻烦,遗祸后代”!\footnote{《丁玲日记》,1978~年~10~月~8~日,载《新文学史料》,1990~年第~3~期,页~15。}但是在~1942~年丁玲却心甘情愿地接受胡乔木、周扬的指导,按照他们解释的毛泽东的“党文化”观,创作出一篇又一篇的符合党的路线的作品\footnote{例如丁玲写作《太阳照在桑干河上》就得到胡乔木的直接“指导”。1948~年~6~月,丁玲将刚刚完稿的《太阳照在桑干河上》的誊抄复写件面交已迁居于河北省平山县西柏坡村的胡乔木,请胡审读。1948~年~7~月,胡乔木、萧三、艾思奇等在传阅小说稿后一致认定,“这是一本最早的最好的表现了中国农村阶级斗争的书”。胡乔木立即向毛泽东作了汇报,毛认为写得好,个别地方修改一下,就可以发表。胡乔木随即打电报给已在大连准备赴苏联、匈牙利访问的丁玲,传达修改意见。丁玲随即奉命对原稿作了修改。9~月,《太阳照在桑干河上》在“党中央的直接关心下”赶印出来。丁玲终于带上这部小说于~1948~年~11~月~9~日离开哈尔滨出国。参见龚明德:《〈太阳照在桑干河上〉版本变迁》,载《新文学史料》,1991~年第~1~期,页~121~—~22。}。于是丁玲暂时成了一名“毛泽东的文艺战士”,然而懦弱乖巧的丁玲何尝料到,十多年后她又跌入几乎和王实味一样悲惨的深渊。所不同的是,这一次丁玲没被处死,而是被送到了北大荒,文革爆发后又被关押进秦城监狱。
