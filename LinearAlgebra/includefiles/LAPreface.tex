%# -*- coding:utf-8 -*-
%%%%%%%%%%%%%%%%%%%%%%%%%%%%%%%%%%%%%%%%%%%%%%%%%%%%%%%%%%%%%%%%%%%%%%%%%%%%%%%%%%%%%


\chapter{序\hspace{\ccwd}言}

\begin{fzliukai}
\zihao{-4}\TeXGyreBonum\boldmath
\setlength\parskip{\baselineskip-\ccwd}
本书初稿完成于~1983~年.当时中国科学技术大学数学系领导冯克勤教授委托编著者编写一本供数学系用的线性代数讲义.接受这项任务后,我们
专程到北京,拜访了中国科学院系统科学研究所万哲先研究员、中国科学院数学研究所许以超研究员、北京大学数学系聂灵沼教授和中国科学院研
究生院曾肯成教授,请教他们对数学系线性代数教学的设想.他们都热情地给予指导,从而为编写讲义提供了坚实的基础.1984~年春天,讲义便开
始在数学系~83~级使用,并作为数学系线性代数教材一直使用到现在.1985~年,讲义曾获得中国科学技术大学首次颁发的优秀教材一等奖.此后,
在使用过程中对讲义又作了进一步的修改.出版前编著者又作了全面的加工和充实.

线性代数研究的是线性空间以及线性空间的线性变换.在线性空间取定一组基下,线性变换便和矩阵建立了一一对应关系.这样,线性变换就和矩
阵紧密联系起来.于是,研究线性空间以及线性空间关于线性变换的分解即构成了线性代数的几何理论,而研究矩阵在各种关系下的分类问题则是
线性代数的代数理论.本书编写的一个着眼点是,着力于建立线性代数的这两大理论之间的联系,并从这种联系去阐述线性代数的理论.

当然,线性代数内容非常丰富,本书尽可能地按照~1980~年教育部颁发的综合性大学理科数学专业高等代数教学大纲进行选择,并在体系安排与叙
述方式上尽量吸收中国科学技术大学数学系长期从事线性代数教学的老师与同事们,特别是曾肯成教授、许以超研究员的教学经验.在处理行列式
理论时,采用了曾肯成教授~1965~年首先在中国科学技术大学数学系使用的将~$n$~阶行列式视为数域~$\mF$~上的~$n$~维向量空间
的规范反对称~$n$~重线性函数的讲法;在处理线性方程组理论时,利用了矩阵在相抵下的标准形理论;在处理~Jordan~标准形时,先考虑了
线性空间关于线性变换的分解,然后再用纯矩阵方法处理了~Jordan~标准形.同时也着重于阐述已故著名数学家华罗庚教授的独具特色的矩阵方法.

为了便于读者掌握解题技巧,提高分析问题、解决问题的能力,本书几乎每一章都专门用一节讲述各种典型例题.这部分内容是为习题课安排的.
每一节后面都附有大量习题,教师与读者可以根据具体情况选择使用.这些习题除了在众多的线性代数、矩阵论教材以及习题集中选取之外,有一
些是取自我国历届研究生试题,还有一些是自己编撰的.在教学过程中,有些同学对线性代数的某些课题产生了兴趣,进行了一些研究.有些结果
即成为本书的习题.这里应该提到的有中国科学技术大学数学系~81~级同学陈贵忠、黄瑜、窦昌柱,82~级同学陈秀雄等.

冯克勤教授对本书的编写自始至终都给予了热情的关心与帮助.在编写过程中,得到万哲先、许以超、聂灵沼、曾肯成等研究员和教授的热心指导,
编者谨致衷心感谢.中国科学技术大学数学系陆洪文教授,杜锡录、李尚志副教授曾经使用本书的前身{\fzwkai{}——《线性代数讲义》——}作为
教材,他们对讲义的修改提出许多有益的意见.中国科学技术大学数学系讲师屈善坤、徐俊明协助编者仔细地审核了原稿,安徽大学数学系夏恩虎同
志、中国科学技术大学~86~级硕士研究生黄道德审核了习题,在此一并致谢.

由于编著者水平所限,错误与缺点在所难免.热忱欢迎同行们和广大读者批评指正.

\begin{flushright}
\begin{minipage}[t][10mm][b]{5cm}\normalsize\normalfont\centering
{\fzqiti\zihao{4}李炯生\hspace{.6\ccwd}查建国}\par
一九八八年元月于合肥
\end{minipage}
\end{flushright}
\end{fzliukai}



%%%%%%%%%%%%%%%%%%%%%%%%%%%%%%%%%%%
%%%%%%%%%%%%%%%%%%%%%%%%%%%%%%%%%%%

\PrintChEndLogo

%%%%%%%%%%%%%%%%%%%%%%%%%%%%%%%%%%%
%%%%%%%%%%%%%%%%%%%%%%%%%%%%%%%%%%%
