%# -*- coding:utf-8 -*-
%%%%%%%%%%%%%%%%%%%%%%%%%%%%%%%%%%%%%%%%%%%%%%%%%%%%%%%%%%%%%%%%%%%%%%%%%%%%%%%%%%%%%
%%  MAChapter3.tex'


\chapter{连\emspace 续}\label{ch:3}

事物的运动和变化有两种形式:一种是渐变,一种是突变。比如公共汽车行驶时,车速是连续变化的,或称渐变。遇到急刹车时,车速
有一突变。反映自然界的渐变与突变,人们提出了连续与间断的概念。这些概念虽然不能用来描述自然界中一切渐变与突变现象(如脉
冲现象等),但能很好地描述自然界中一类渐变与突变现象。

\section{连续与间断}

怎么去刻画连续性呢?能否说函数在每一点极限存在,它就表示一条连续曲线呢?不能。比如在~$\mR$~上定义的函数
\[
  f(x)=\begin{Bdcases}
    2, & x=1;\\
    1, & x\neq1,
  \end{Bdcases}
\]
它在每一点~$x_0$~的极限都存在,且
\[
  \lim_{x\to x_0}f(x)=1。
\]
但函数的图形不是一条连续的直线,而是在~$(1,1)$~点有一点窟窿的直线加上一点~$(1,2)$,显然不能认为函数是连续的。原因是在~$x=1$~处
极限值不等于函数值:
\[
  \lim_{x\to1}f(x)=1\neq 2=f(1)。
\]

于是我们有如下定义。

\begin{definition}\label{def:sec3.1-1}
在~$(a,b)$~上给定函数~$f(x)$,$x_0\in(a,b)$,若
\begin{equation}\label{eq:sec3.1-1}
\lim_{x\to x_0}f(x)=f(x_0),
\end{equation}
则称函数~$f(x)$~在~$x_0$~点连续,$x_0$~称为\emph{连续点},否则就称~$x_0$~为\emph{间断点}。
\end{definition}

直观地说,就是当动点~$x$~趋于定点~$x_0$~时,若动点函数值趋于定点的函数值,则函数在~$x_0$~点连续。若~$x_0$~是连续点,则
当自变量在~$x_0$~点有无限小的变化,引起因变量的变化也无限的小。

$(a,b)$~上给定的函数~$f(x)$~在~$x_0\in(a,b)$~连续的定义改用“$\e-\delta$”说法,就是对任意~$\e>0$,都存在~$\delta>0$,当
~$\mabs{x-x_0}<\delta$~时,有
\[
  \mabsb{f(x)-f(x_0)}<\e 。
\]

\begin{example}
设函数
\[
  f(x)=\begin{Bdcases}
    x, & x\in\mR\difset\mQ;\\
    0, & x\in\mQ 。
  \end{Bdcases}
\]
证明~$x=0$~是函数~$f(x)$~的连续点。
\end{example}
\begin{proof}
对任意~$\e>0$,取~$\delta=\e$,当~$\mabs{x-0}=\mabs x<\delta$~时,
\[
  \mabsb{f(x)-f(0)}=\mabsb{f(x)}\leq\mabs x<\e,
\]
由\ref{def:sec3.1-1}~可知~$f(x)$~在~$x=0$~点连续。
\end{proof}

\begin{remark}
容易看出,当~$x_0\neq0$~时,极限~$\lim_{x\to x_0}f(x)$~不存在,即~$x_0$~不是~$f(x)$~的连续点。所以函数在一点连续,不能保证
在该点的一个邻域内也连续。
\end{remark}

\ref{eq:sec3.1-1}~可以拆成下面三个式子
\begin{equation}\label{eq:sec3.1-2}
\begin{gathered}
\lim_{x\to x_0-0}f(x)=f(x_0-0);\\
\lim_{x\to x_0+0}f(x)=f(x_0+0);\\
f(x_0-0)=f(x_0)=f(x_0+0) 。
\end{gathered}
\end{equation}
这三个式子也可以合并成~\ref{eq:sec3.1-1}~一个式子。即函数在~$x_0$~点连续,等价于函数在~$x_0$~点的左、右极限存在,且左、右极限
等于该点的函数值。\ref{eq:sec3.1-2}~中只要有一个式子不成立,\ref{eq:sec3.1-1}~就不成立。所以,对于一点,只要~\ref{eq:sec3.1-2}~中
有一个不成立,该点就是间断点。根据~\ref{eq:sec3.1-2}~不成立的情形,我们可以把间断点加以分类。

\begin{definition}\label{def:sec3.1-2}
\begin{thmenumlist}
\item 若函数在~$x_0$~点左、右极限存在并相等,但不等于该点的函数值,则称~$x_0$~为\emph{可去间断点};
\item 若函数在~$x_0$~点左、右极限存在但不相等,则称~$x_0$~为\emph{第一类间断点};
\item 若函数在~$x_0$~点左、右极限至少有一个不存在,则称~$x_0$~为\emph{第二类间断点}。
\end{thmenumlist}
\end{definition}

例如,函数
\[
  f(x)=\begin{cBdcases}
    \arctan\frac1x, & x\neq0;\\
    0, & x=0,
  \end{cBdcases}
\]
在~$x_0=0$~点的左、右极限分别为~$-\dfrac\pi2$~和~$\dfrac\pi2$,所以~$x=0$~是函数的第一类间断点(见\ref{fig:sec3.1-1})。

\begin{figure}
\begin{floatrow}[3]
\figurebox{\caption{}\label{fig:sec3.1-1}}{\somefigure}
\figurebox{\caption{}\label{fig:sec3.1-2}}{\somefigure}
\figurebox{\caption{}\label{fig:sec3.1-3}}{\somefigure}
\end{floatrow}
\end{figure}

又例如,函数
\[
  f(x)=\begin{cBdcases}
    \frac1x, & x\neq0;\\
    0, & x=0,
  \end{cBdcases}
\]
在~$x_0=0$~点的左、右极限不存在,所以~$x=0$~是函数的第二类间断点(见\ref{fig:sec3.1-2})。

再例如,函数
\[
  f(x)=\begin{cBdcases}
    \sin\frac1x, & x\neq0;\\
    0, & x=0,
  \end{cBdcases}
\]
在~$x_0=0$~点的左、右极限不存在,所以~$x=0$~也是函数的第二类间断点(见\ref{fig:sec3.1-3})。

关于间断点有下面定理。

\begin{theorem}\label{thm:sec3.1-1}
设~$f(x)$~在区间~$(a,b)$~上单调,则~$f(x)$~只有第一类间断点。
\end{theorem}
\begin{proof}
不妨设~$f(x)$~单调上升。那么对任意~$x_0\in(a,b)$,当~$x\to x_0-0$~时,函数值~$f(x)$~上升,并有上界~$f(x_0)$,所以极限存在,且
\[
  \lim_{x\to x_0-0}f(x)=f(x_0-0)\leq f(x_0) 。
\]
同理,当~$x\to x_0+0$~时,函数值~$f(x)$~下降,并有下界~$f(x_0)$,所以极限存在,且
\[
  \lim_{x\to x_0+0}f(x)=f(x_0+0)\geq f(x_0) 。
\]

若~$f(x_0-0)=f(x_0+0)$,则~$x_0$~是函数的连续点;若~$f(x_0-0)\neq f(x_0+0)$,则~$x_0$~是函数的第一类间断点。由于~$x_0$~的
任意性,所以区间上每一点不是连续点就是第一类间断点。
\end{proof}

\begin{remark}
\ref{thm:sec3.1-1}~中条件改为闭区间~$[a,b]$~时,结论仍成立。比如在~$x=a$~点,有
\[
  \lim_{x\to a+0}f(x)=f(a+0)\geq f(a)。
\]
若等号成立,则称~$a$~为连续点;否则为第一类间断点。
\end{remark}

由函数一点连续的定义可得函数在区间上连续的定义。

\begin{definition}
若函数~$f(x)$~在区间上每一点连续(闭区间情形,区间端点指单侧连续),则称函数在\emph{区间上连续}。
\end{definition}

记号~$C(a,b)$~表示区间~$(a,b)$~上所有连续函数的集合,记号~$f(x)\in C(a,b)$~表示~$f(x)$~是~$(a,b)$~上的连续函数。记
号~$f(x)\in C[a,b]$~作类似理解。

\begin{quiz}
\begin{thmenumlist}
\item 讨论单调函数能否有无穷多个间断点;
\item 试举出每一点都是第二类间断点的函数。
\end{thmenumlist}
\end{quiz}

\begin{exercise}
\item 研究下列函数的连续性,并指出间断点的类型。
\begin{exlistcols}
  \item $f(x)=\sgn x$;
  \item $g(x)=x-\mfloor x$;
  \item $f\mparenb{g(x)}$,~$g\mparenb{f(x)}$;
  \item $\mparenB{\dfrac1x-\dfrac1{x+1}}\Big/\mparenB{\dfrac1{x-1}-\dfrac1x}$。
\end{exlistcols}
\item 指出下列函数的间断点,并说明间断点的类型。
\begin{exlistcols}
  \item $y=\dfrac x{(1+x)^2}$;
  \item $y=\cos^2\dfrac1x$;
  \item $y=\sgn(\sin x)$;
  \item $y=1\Big/\mfloorbb{\dfrac1x}\mcond{0<x\leq 1}$。
\end{exlistcols}
\item 适当选取常数~$a$,使得下列函数~$f(x)$~连续。
\[
  f(x)=\begin{cBdcases}
    \me^x, & x<0; \\
    a+x,   & x\geq0
  \end{cBdcases}
\]
\item 设~$f(x)$~在~$x=x_0$~连续,$g(x)$~在~$x=x_0$~间断。判断~$f(x)+g(x)$~和~$f(x)\cdot g(x)$~在~$x=x_0$~的连续性。
\item 设~$f(x),g(x)$~在~$x=x_0$~间断。判断~$f(x)+g(x)$~和~$f(x)\cdot g(x)$~在~$x=x_0$~的连续性。
\item 举出处处都不连续,但取绝对值后却是处处连续的函数的例子。
\item 设~$f(x)$~在~$x=x_0$~点连续,且~$f(x_0)>0$。证明,存在~$\delta>0$,当~$x\in U(x_0;\delta)$~时,$f(x)>0$。并判断下面证法的
正确性。
\begin{exproof}
对任意~$\e>0$,不妨设~$0<\e<f(x_0)$。由函数~$f(x)$~在~$x_0$~点连续,存在~$\delta>0$,当~$\mabs{x-x_0}<\delta$~时,有
\[
  \mabsb{f(x)-f(x_0)}<\e,
\]
即有
\[
  f(x)>f(x_0)-\e>0\mcond*{x\in U(x_0;\delta)}。\qedhere
\]
\end{exproof}
\end{exercise}


\section{连续函数的运算}

连续是极限的一种特殊情形,由极限运算即可得到连续函数的运算。

\begin{theorem}\label{thm:sec3.2-1}
设~$f(x),g(x)$~在~$x_0$~点连续,则
\begin{enumlistcols}
\item $f(x)\pm g(x)$~在~$x_0$~点连续;
\item $f(x)\cdot g(x)$~在~$x_0$~点连续;
\item 若~$g(x_0)\neq0$,$\dfrac{f(x)}{g(x)}$~在~$x_0$~点连续。
\end{enumlistcols}
\end{theorem}

由此可得,若~$f(x),g(x)\in C(a,b)$,则
\[
  f(x)\pm g(x)\in C(a,b);\quad
  f(x)\cdot g(x)\in C(a,b)。
\]
若对任意~$x\in(a,b)$,均有~$g(x)\neq0$,则
\[
  \dfrac{f(x)}{g(x)}\in C(a,b)。
\]

\begin{theorem}\label{thm:sec3.2-2}
设~$y=f(t)$~在~$t=t_0$~点连续,$t=g(x)$~在~$x=x_0$~点连续,且~$t_0=g(x_0)$,则~$y=f\mparenb{g(x)}$~在~$x=x_0$~点连续。
\end{theorem}
\begin{proof}
对任意~$\e>0$,由~$f(t)$~在~$t_0$~点连续,存在~$\eta>0$,当~$\mabs{t-t_0}<\eta$~时,有
\[
  \mabsb{f(t)-f(t_0)}<\e;
\]
对于~$\eta>0$,由~$g(x)$~在~$x_0$~点连续,存在~$\delta>0$,当~$\mabs{x-x_0}<\delta$~时,有
\[
  \mabsb{g(x)-g(x_0)}=\mabs{t-t_0}<\eta 。
\]
所以,当~$\mabs{x-x_0}<\delta$~时,有
\[
  \mabsb{f(t)-f(t_0)}=\mabsb{f\mparenb{g(x)}-f\mparenb{g(x_0)}}<\e,
\]
即函数~$f\mparenb{g(x)}$~在~$x_0$~点连续。
\end{proof}

\begin{remark}
有\ref{thm:sec3.2-2}~可得,若~$g(x)\in C(a,b)$,值域属于~$(\alpha,\beta)$,$f(t)\in C(\alpha,\beta)$,%
则~$f\mparenb{g(x)}\in C(a,b)$。
\end{remark}

下面讨论一些初等函数的连续性。

\begin{enumlist}
\item 多项式
\[
  P_n(x)=a_0+a_1x+\dotsb+a_nx^n\in C(\mR);
\]
\item 有理函数
\[
  R(x)=\frac{P_n(x)}{Q_m(x)},
\]
其中分子、分母分别为~$n$~次和~$m$~次多项式。定义域可能要除去分母为零的点,记作~$D$,则~$R(x)\in C(D)$。

上面两条结论由常数与~$x$~在~$\mR$~上连续及连续函数的四则运算即可看出。
\item 三角函数~$y=\sin x\in C(\mR)$。
\begin{proof}
对任意~$x_0\in\mR$~和任意~$\e>0$,要使
\[
  \mabsb{\sin x-\sin x_0}<\e,
\]
只要
\[
  \mabsbb{2\cos\frac{x+x_0}2\sin\frac{x-x_0}2}\leq2\mabsbb{\sin\frac{x-x_0}2}
  \leq2\mabsbb{\frac{x-x_0}2}=\mabs{x-x_0}<\e 。
\]

取~$\delta=\e$,则当~$\mabs{x-x_0}<\delta$~时,就有~$\mabs{\sin x-\sin x_0}<\e$。所以~$\sin x$~在~$x_0$~点连续。%
由于~$x_0$~的任意性,得~$\sin x\in C(\mR)$。
\end{proof}
\end{enumlist}

由正弦函数~$\sin x$~的连续性可知
\[
y=\cos x=\sin\mparenB{\frac\pi2-x}\in C(\mR);\quad
y=\tan x=\frac{\sin x}{\cos x}\in C(D),
\]
这里~$D$~为~$\mR$~除去点
\[
  x=\mparenB{k+\frac12}\pi\mcond*{0,\pm1,\pm2,\dotsc}
\]
的集合。

\begin{exercise}
\item 设~$f(x),g(x)$~在~$[a,b]$~上连续。证明
\begin{exlistcols}
  \item $\mabsb{f(x)}\in C[a,b]$;
  \item $\max\mrange{f(x)}{g(x)}\in C[a,b]$;
  \item $\min\mrange{f(x)}{g(x)}\in C[a,b]$。
\end{exlistcols}
\item 设~$f(x)\in C[a,b]$,证明~$f_t(x)\in C[a,b]$,其中
\[
  f_t(x)=\begin{cBdcases}
    f(x),& x\in\mathset x{f(x)>t};\\
    t,   & x\in\mathset x{f(x)\leq t}。
  \end{cBdcases}
\]
\item 设~$f(x)$~在~$(0,+\infty)$~上连续,且满足
\[
  f(\cramped{x^2})=f(x)\mcond*{x>0}。
\]
证明~$f(x)$~为一常数。
\item 设~$f(x)$~在~$x=0$~处连续,且对任意~$x,y\in\mR$,有
\[
  f(x+y)=f(x)+f(y)。
\]
证明,$f(x)$~在~$\mR$~上连续,且~$f(x)=f(1)\cdot x$。
\item 设~$f(x)$~在~$(a,b)$~上只有第一类间断点,且对任意~$x,y\in(a,b)$,有
\[
  f\mparenB{\frac{x+y}2}\leq\frac{f(x)+f(y)}2。
\]
证明~$f(x)$~在~$(a,b)$~上连续。
\item 设~$f(x)$~在~$\mintco 0\pinf$~上连续,且~$0\leq f(x)\leq x\mcond{x\geq0}$。设~$a_1\geq0$,%
而~$a_{n+1}=f(a_n)\mcond{n=1,2,\dotsc}$。证明,
\begin{exlistcols}
  \item $\lim_\ntoinf a_n$~存在;
  \item 设~$\lim_\ntoinf a_n=\ell$,则~$f(\ell)=\ell$;
  \item 如果将条件改为~$0\leq f(x)<x\mcond{x>0}$,则~$\ell=0$。
\end{exlistcols}
\end{exercise}


\section{连续函数的中间值性质}

我们叙述连续函数的零点性质,它的证明以后再给出。

\begin{theorem}\label{thm:sec3.3-1}
设~$f(x)\in C\mintc ab$且~$f(a)f(b)<0$,则存在~$\xi\in\minto ab$,使得
\[
  f(\xi)=0 。
\]
\end{theorem}

\ref{thm:sec3.3-1}~的几何意义是,若连续曲线由~$x$~轴之下跑到~$x$~轴之上,则中间至少要经过~$x$~轴一次(见\ref{fig:sec3.3-1})。

\begin{figure}
\begin{floatrow}
\figurebox{\caption{}\label{fig:sec3.3-1}}{\somefigure}
\figurebox{\caption{}\label{fig:sec3.3-2}}{\somefigure}
\end{floatrow}
\end{figure}

\begin{theorem}[中间值定理]\label{thm:sec3.3-2}
设~$f(x)\in C\mintc ab$,且~$\eta$~介于~$f(a)$~与~$f(b)$~之间,则存在~$\xi\in\mintc ab$,使得
\[
  f(\xi)=\eta 。
\]
\end{theorem}
\begin{proof}
若~$\eta$~等于~$f(a)$~或~$f(b)$,取~$\xi=a$~或~$b$~即可。

若~$\eta$~严格介于~$f(a)$~与~$f(b)$~之间,即
\[
  f(a)<\eta<f(b)\quad\text{或}\quad f(a)>\eta>f(b) 。
\]
作函数~$F(x)=f(x)-\eta\in C\mintc ab$。显然
\[
  F(a)\cdot F(b)=\mparenb{f(a)-\eta}\cdot\mparenb{f(b)-\eta}<0,
\]
根据\ref{thm:sec3.3-1},存在~$\xi\in(a,b)$,使得
\[
  F(\xi)=0\implies f(\xi)=\eta 。\qedhere
\]
\end{proof}

\ref{thm:sec3.3-2}~说明,连续函数可以取得~$f(a)$~与~$f(b)$~之间的一切值。反之,若一个函数能取到~$f(a)$~与~$f(b)$~之间
的一切值,它是否一定连续呢?例如函数
\[
  f(x)=\begin{cdcases} %\LEFTRIGHT
    x, & x\in\mintco 01;\\
    3-x, & x\in\mintc 12;\\
    x, & x\in\mintoc 23,
  \end{cdcases}
\]
它取到~$f(0)=0$~与~$f(3)=3$~之间的一切值,但显然函数~$f(x)$~不连续(见\ref{fig:sec3.3-2})。

\begin{example}
证明方程
\[
  x^3+4x^2-3x-1=0
\]
在~$\mR$~上有三个根。
\end{example}
\begin{proof}
令
\[
  f(x)=x^3-x^2-3x-1\in C(\mR) 。
\]
\begin{enumlist}
\item 因为~$f(0)=-1<0$,而~$f(1)=1>0$,所以多项式~$f(x)$~在~$\minto 01$~有一个零点,即方程~$f(x)=0$~有一个实根;
\item 因为~$f(-1)=5>0$,所以方程在~$\minto {-1}0$~内有一个实根;
\item 因为~$\lim_{x\to\minf}f(x)=\minf$,故存在~$x_2$,使得~$f(x_2)<0$,所以方程在~$\minto{-x_2}{-1}$~内有一个实根。\qedhere
\end{enumlist}
\end{proof}

\begin{theorem}\label{thm:sec3.3-3}
设~$y=f(x)\in C\minto ab$,且严格上升,记
\[
  \alpha\coloneq\inf_{x\in\minto ab}\mbraceb{f(x)},\quad\beta\coloneq\sup_{x\in\minto ab}\mbraceb{f(x)},
\]
其中~$\alpha$~可以使~$\minf$,而~$\beta$~可以为~$\pinf$,则
\begin{enumlistcols}
\item 在~$\minto\alpha\beta$~上存在反函数~$x=\phi(y)$;
\item $x=\phi(y)$~在~$\minto\alpha\beta$~上严格单调上升;
\item $\phi(y)\in C\minto\alpha\beta$。
\end{enumlistcols}
\end{theorem}
\begin{proof}
\begin{thmenumlist}
\item 因为~$y=f(x)$~严格上升,所以反函数一定存在,需要证明的是反函数的定义域恰好为~$\minto\alpha\beta$,也就是要证明
区间上的严格单调函数,如果函数连续,则其值域一定是区间。

事实上,对任意~$y_0\in\minto\alpha\beta$,由上、下确界的定义,存在~$x',x''\in\minto\alpha\beta$,使得
\[
  f(x')<y_0<f(x'')。
\]
在~$\mintc{x'}{x''}$~或~$\mintc{x''}{x'}$~上应用\ref{thm:sec3.3-2},则存在~$x_0\in\minto{x'}{x''}\subset\minto ab$,使得
~$f(x_0)=y_0$,即~$y_0$~是~$f(x)$~的函数值。由~$y_0$~的任意性,得到~$\minto\alpha\beta$~为~$f(x)$~的值域($\alpha,\beta$~不
可能是~$f(x)$~的函数值),即~$\minto\alpha\beta$~为~$x=\phi(y)$~的定义域。
\item 设~$y_1,y_2\in\minto\alpha\beta$,且~$y_1<y_2$,往证
\[
  x_1=\phi(y_1)<\phi(y_2)=x_2 。
\]
事实上,假若~$x_1\geq x_2$,由反函数定义及~$f(x)$~的严格上说,得到
\[
  y_1=f(x_1)\geq f(x_2)=y_2,
\]
这与~$y_1<y_2$~矛盾,所以~$\phi(y)$~严格上升。
\item 已知~$\phi(y)$~在~$\minto\alpha\beta$~上严格上升,值域为区间~$\minto ab$,要证明~$\phi(y)\in C\minto\alpha\beta$,也就是要
证明区间上的严格单调函数,如果函数的值域为一个区间,则函数一定连续。

事实上,设~$\phi(y)$~在~$y_0\in\minto\alpha\beta$~点不连续。由\ref{thm:sec3.1-1}~可知
\[
  \lim_{y\to y_0-0}\phi(y)=\phi(y_0-0)\leq\phi(y_0),\quad
  \lim_{y\to y_0+0}\phi(y)=\phi(y_0+0)\geq\phi(y_0)。
\]
以上两式必有一式不等号成立,不妨设~$\phi(y_0-0)<\phi(y_0)$,则函数的值域包含在两个不相交的区间
~$\mintob{\phi(\alpha+0)}{\phi(y_0-0)}$~与~$\mintcb{\phi(y_0)}{\phi(\beta-0)}$内,这与值域为一区间~$\minto ab$~矛盾,这
矛盾说明~$\phi(y)\in C\minto\alpha\beta$。\qedhere
\end{thmenumlist}
\end{proof}
\begin{remark}\label{rem:sec3.3-1}
\ref{thm:sec3.3-3}~中条件改为~$f(x)\in\minto ab$,且严格上升,令~$f(a)=\alpha$,而~$f(b)=\beta$,则结论中的
开区间~$\minto\alpha\beta$~相应地改为闭区间~$\mintc\alpha\beta$~即可。对严格下降函数,可以同样讨论。
\end{remark}

利用\ref{thm:sec3.3-3},我们可以得出反三角函数的连续性。
\begin{enumlist}
\item $y=\arcsin x\in C\mintc{-1}1$。

因为~$x=\sin y$~在~$\mintc{-\sfrac\pi2}{\sfrac\pi2}$~上严格上升、连续,由\ref{thm:sec3.3-3}~的证明及\ref{rem:sec3.3-1}~可知
~$\sin y$~的值域为~$\mintc{-1}1$,且~$y=\arcsin x$~在~$\mintc{-1}1$~上严格上升、连续。
\item $y=\arccos x\in C\mintc{-1}1$。

因为~$x=\cos y$~在~$\mintc 0\pi$~上严格下降、连续,故其值域为~$\mintc{-1}1$,且~$y=\arccos x$~在~$\mintc{-1}1$~上严格下降、连续。
\item $y=\arctan x\in C(\mR)$。

因为~$x=\tan y$~在~$\minto{-\sfrac\pi2}{\sfrac\pi2}$~上严格上升、连续,故其值域为~$\mR$,且~$x=\arctan x$~在~$\mR$~上
严格上升、连续。
\end{enumlist}

\begin{exercise}
\item 证明方程~$x^3+x+q=0\mcond{p>0}$~有且只有一个实根。
\item 设~$f(x)$~在~$\minto ab$~上连续,而~$\mvec x{1,2,:,n}\in\minto ab$。证明,存在~$\xi\in\minto ab$,使得
\[
  f(\xi)=\frac1n\sum_{i=1}^nf(x_i) 。
\]
\item 设~$f(x)$~在~$\minto ab$~上连续,若~$x_n,y_n\in\minto ab\mcond{n\geq1}$~且~$\lim_\ntoinf x_n=\lim_\ntoinf y_n=b$,
\[
  \lim_\ntoinf f(x_n)=A,\quad \lim_\ntoinf f(y_n)=B\mcond{B>A}。
\]
证明,对任意~$\eta\in\minto AB$,存在~$\mbrace{z_n}\subset\minto ab$,使得~$\lim_\ntoinf z_n=b$,并且~$\lim_\ntoinf f(z_n)=\eta$。
\item 设~$f(x)\in C\mintc ab$,而~$\mabsb{f(x)}$~在~$\mintc ab$~上单调。证明~$f(x)$~也在~$\mintc ab$~上单调。
\item 设~$f(x)\in C\mintc ab$,且~对任意$x_1,x_2\in\mintc ab$,当~$x_1\neq x_2$~时,有~$f(x_1)\neq f(x_2)$。证
明,$f(x)$~在~$\mintc ab$~上严格单调上升或严格单调下降。
\item 设~$f(x)\in C(\mR)$,并且对任意~$x,y\in\mR$,函数~$f(x)$~满足
\[
  \mabsb{f(x)-f(y)}\leq k\mabs{x-y}\mcond*{0<k<1}。
\]
证明,
\begin{exlistcols}
  \item 函数~$kx-f(x)$~单调上升;
  \item 存在~$\xi\in\mR$,使得~$f(\xi)=\xi$。
\end{exlistcols}
\end{exercise}


\section{初等函数的连续性}

前面我们已知多项式、有理函数、三角函数与反三角函数在其定义域上是连续的。这节我们讨论指数函数、对数函数和幂函数的连续性。

由\ref{thm:sec3.3-3}~的证明,我们可得出如下结论:区间上的单调函数,它连续当且仅当其值域为一区间。当讨论指数函数~$a^x\mcond{a>0}$~
的连续性时,如果用到正实数的对数存在,实质上是用到单调函数~$a^x$~的值域为一区间,而~$a^x$~的值域为一区间时用到~$a^x$~的连续性,这
无异于用~$a^x$~的连续性来证其连续性。为了避免这个逻辑上的毛病,我们从指数定义出发来证明其连续性。

在给出指数函数定义之前,我们需要下面的引理。同时,为了叙述简单起见,只考虑~$a>1$~情形。

\begin{lemma}\label{lem:sec3.4-1}
设~$a>1$,$n$~为正整数,则存在实数~$b>1$,使得~$a=b^n$~或者~$\sqrt[n]a=b$。
\end{lemma}
\begin{proof}
在区间~$\mintc 1a$~上考虑函数
\[
  f(x)=x^n 。
\]
显然它是连续的,且
\[
  f(1)=1<a<f(a)=a^n 。
\]
由连续函数中间值定理(见\ref{fig:sec3.4-1}),存在~$b\in\minto 1a$,使得
\[
  a=b^n\quad\text{或}\quad\sqrt[n]a=b 。\qedhere
\]
\end{proof}

\begin{wrapfigure}[8]{O}{0mm}
\somefigure
\caption{}\label{fig:sec3.4-1}
\end{wrapfigure}

\ref{lem:sec3.4-1}~保证大于~$1$~的实数的~$n$~次算术根总是存在的,且算术根也大于~$1$。这样就有下面定义。

\begin{definition}\label{def:sec3.4-1}
若正有理数~$q=\dfrac mn$,其中~$n,m$~为正整数且互素,定义
\[
  a^q\coloneq\mparenb{\sqrt[n]a}^m;
\]
若~$q$~为负有理数,定义
\[
  a^q\coloneq\frac1{a^{-q}};
\]
而定义~$a^0\coloneq1$。

若~$\lambda$~无理数,定义
\[
  a^\lambda\coloneq\sup_{q<\lambda}\mbrace{a^q}\mcond*{q\in\mQ}。
\]
\end{definition}

还需说明\ref{def:sec3.4-1}~中上确界的存在性。不难证明,当~$q$~为有理数时,$a^q$~仍满足熟知的指数的三条性质,所以当
有理数~$q_1<q_2$~时,
\[
  \frac{a^{q_2}}{a^{q_1}}=a^{q_2-q_1}>1\implies a^{q_2}>a^{q_1}。
\]
由此即可看出定义中上确界存在,且不管~$x$~是无理数还是有理数,总有
\[
  a^x\coloneq\sup_{q<x}\mbrace{a^q}\mcond*{q\in\mQ}。
\]
这样我们就得到在~$\mR$~上定义的指数函数~$a^x$。

首先证明~$a^x$~严格上升。

设给定实数~$x_1,x_2$,且~$x_1<x_2$,总可以找到有理数~$q_1,q_2$,使得
\[
  x_1<q_1<q_2<x_2,
\]
因此
\[
  a^{x_1}=\sup_{q\leq x_1}\mbrace{a^q}\leq a^{q_1}<a^{q_2}\leq\sup_{q\leq x_2}\mbrace{a^q}=a^{x_2},
\]
即~$a^x$~在~$\mR$~上严格上升。

其次证明~$f(x)=a^x$~在~$\mR$~上连续。

对任意~$\e>0$,存在自然数~$N$,使得
\[
  a<(1+\e)^N 。
\]
对任意~$x_0\in\mR$,总可以找到有理数~$q_1,q_2$,使得
\[
  q_1<x_0<q_2,\quad q_2-q_1=\frac 1N 。
\]
事实上,用~$\sfrac kN$~的点把实轴分成无穷个小区间,这里~$k=0,\pm1,\pm2,\dotsc$,若~$x_0$~落在某一小区间内部,则把该
小区间的左、右端点即为~$q_1$~和~$q_2$;若~$x_0$~正好是小区间端点,只要把小区间平移~$\sfrac1{(2N)}$,即化为内点情形。由%
\ref{thm:sec3.1-1}~可知
\[
  f(x_0)\leq f(x_0+0)\leq a^{q_2},\quad
  f(x_0)\geq f(x_0-0)\geq a^{q_1}。
\]
所以
\[
  1\leq\frac{f(x_0+0)}{f(x_0-0)}\leq\frac{a^{q_2}}{a^{q_1}}=a^{q_2-q_1}=a^{\frac1N}<1+\e,
\]
由~$\e$~的任意性,即得
\[
  f(x_0+0)=f(x_0-0)=f(x_0)=a^{x_0},
\]
这说明函数~$a^x$~在~$x_0$~点连续。由~$x_0$~的任意性可知~$a^x$~在~$\mR$~上连续。

利用函数~$a^x$~的连续性,已知性质~$a^{x_1}a^{x_2}=a^{x_1+x_2}$~对有理数~$x_1,x_2$~成立,通过取极限,即得上述性质对任意
实数~$x_1,x_2$~成立。

进一步可得出对数函数与幂函数的连续性。

\begin{enumlist}
\item $y=\log_ax\in C\minto 0{+\infty}$。

因为~$x=a^y$~在~$\mR$~上严格上升、连续,故其值域为~$\minto 0{+\infty}$,且~$y=\log_ax$~在~$\minto 0{+\infty}$~上严格上升、连续。
\item ~$y=x^\alpha\in C\minto 0{+\infty}$。

因为~$y=x^\alpha=\me^{\alpha\ln x}$,由复合函数连续性即可看出。
\end{enumlist}

最后可得结论:\emph{一切初等函数在其定义域上是连续的}。

\begin{example}\label{ex:sec3.4-1}
若~$u(x),v(x)$~在~$x_0$~点连续,且~$u(x_0)>0$,则~$u(x)^{v(x)}$~也在~$x_0$~点连续(在~$x_0$~附近考虑)。
\end{example}
\begin{proof}
注意到,$u(x)^{v(x)}=\me^{v(x)\ln u(x)}$。因为~$u(x)$~在~$x_0$~点连续,$\ln t$~在~$t=u(x_0)$~点连续,由复合函数的连续性,%
可知~$\ln u(x)$~在~$x_0$~点连续。由乘积的连续性,可得~$v(x)\ln u(x)$~在~$x_0$~点连续。又~$\me^y$~在~$y=v(x_0)\ln u(x_0)$~点
连续,再用复合函数的连续性,得
\[
  u(x)^{v(x)}=\me^{v(x)\ln u(x)}
\]
在~$x_0$~点连续。
\end{proof}

\begin{example}
若~$\lim_{x\to x_0}u(x)=a>0$,$\lim_{x\to x_0}v(x)=b$,则
\[
  \lim_{x\to x_0}u(x)^{v(x)}=a^b 。
\]
\end{example}
\begin{proof}
补充定义
\[
  u(x_0)=a,\quad v(x_0)=b,
\]
则函数~$u(x),v(x)$~在~$x_0$~点连续。由\ref{ex:sec3.4-1}~可得
\[
  \lim_{x\to x_0}u(x)^{v(x)}=u(x_0)^{v(x_0)}=a^b 。
\]

当~$x_0=+\infty$~或~$-\infty$~时,利用变换~$x=\sfrac1t$,就可以化为~$t=0$~情形,所以这时候结论仍成立。
\end{proof}

求极限我们讨论过加、减、乘、除四条运算法则,现在又多了一条幂指运算法则,这样共有五条求极限的法则。

\begin{example}
\[
  \lim_{x\to+\infty}\mparenbb{1+\sin\frac1x}^x 。
\]
\end{example}
\begin{proof}
注意到
\[
  \mparenbb{1+\sin\frac1x}^x=\mparenbb{1+\sin\frac1x}^{\frac1{\sin(\sfrac1x)}\cdot\frac{\sin(\sfrac1x)}{\sfrac1x}},
\]
令
\[
  u(x)=\mparenbb{1+\sin\frac1x}^{\frac1{\sin(\sfrac1x)}},\quad
  v(x)=\frac{\sin(\sfrac1x)}{\sfrac1x} 。
\]
由
\[
  \lim_{x\to+\infty}u(x)=\me,\quad \lim_{x\to+\infty}v(x)=1,
\]
所以
\[
  \lim_{x\to+\infty}\mparenbb{1+\sin\frac1x}^x=\me^1=\me 。\qedhere
\]
\end{proof}


\begin{exercise}
\item 求下列极限。
\begin{exlistcols}
  \item $\lim_{x\to1}\mparenbb{\dfrac{1+x}{2+x}}^{\frac{1-\sqrt x}{1+x}}$;
  \item $\lim_\ntoinf\mparen{\arctan x}^{\cos\frac1x}$。
\end{exlistcols}
\item 求下列极限。
\begin{exlistcols}
  \item $\lim_{x\to0}\mparen{\cos x}^{\frac1{x^2}}$;
  \item $\lim_{x\to0}\mparen{\cos x+\sin x}^{\frac1x}$。
\end{exlistcols}
\item 证明,方程~$x^\alpha=\ln x\mcond{\alpha<0}$~在正实轴上有且仅有一根。
\end{exercise}



\section{有界闭区间上连续函数的性质}

这节我们讨论连续函数的最大最小值与一致连续性,它们的证明以后再给出。

\begin{theorem}[最大最小值定理]\label{thm:sec3.5-1}
设~$f(x)\in C\mintc ab$,则~$f(x)$~一定有最大值与最小值。即存在~$x_1,x_2\in\mintc ab$,使得对任意~$x\in\mintc ab$,有
\[
  f(x)\leq f(x_1)=\max_{x\in\mintc ab}\mbraceb{f(x)},\quad
  f(x)\geq f(x_2)=\min_{x\in\mintc ab}\mbraceb{f(x)}。
\]
此时,$x_1,x_2$~分别称为函数的\emph{最大值点}与\emph{最小值点}。
\end{theorem}

从几何上看,若连续曲线有最左、最右的点,则必有最高、最低的点。

注意\ref{thm:sec3.5-1}~中闭区间条件不能减弱。例如,函数~$f(x)=\sfrac1x$~在~$\minto 01$~上连续,但函数在开区间上没有最大值和
最小值。

下面介绍一致连续的概念。

设~$f(x)$~在~$\minto ab$~上连续,即指每一点连续。例如,它在~$x_1$~点连续,则对任意~$\e>0$,存在~$\delta_1=\delta(\e,x_1)>0$,当
~$\mabs{x-x_1}<\delta_1$~时,有
\[
  \mabsb{f(x)-f(x_1)}<\e 。
\]
由它在~$x_2$~点连续,则对上述~$\e>0$,存在~$\delta_2=\delta(\e,x_2)>0$,当~$\mabs{x-x_2}<\delta_2$~时,有
~$\mabs{x-x_1}<\delta_1$~时,有
\[
  \mabsb{f(x)-f(x_2)}<\e 。
\]

我们发现,对同一个~$\e>0$,由于函数在每一点连续,多以每一点都存在~$\delta(\e,x)>0$,但对不同的点~$x$,找到的~$\delta$~的大小
是不同的,$\delta$~不仅依赖于~$\e$,也依赖于~$x$。在曲线平坦部分(比如~$x_1$~点附近)找到的~$\delta$~可以大一些,而在曲线
陡的部分(比如~$x_2$~点附近)找到的~$\delta$~必须小一些。从图形上看,在陡处的点适用的~$\delta$,对平坦处的点也是适用的,如
~$x$~落在~$x_1$~的~$\delta_1$~邻域时,$f(x)$~落在~$f(x_1)$~的~$\e$~邻域,则~$x$~落在~$x_1$~的~$\delta_2$~时,$f(x)$~更落
在~$f(x_1)$~的~$\e$~邻域。但在平坦处的点适用的~$\delta$,对陡处的点不一定适用(见\ref{fig:sec3.5-1})。这样,就提出一个问
题,对任意~$\e>0$,能否找到对每一个~$x$~都适用的公共~$\delta$~呢?

\fixwrapfloatsep
\begin{wrapfigure}{O}{0mm}
\somefigure
\caption{}\label{fig:sec3.5-1}
\end{wrapfigure}

定义域上有无穷个~$x$,也就有无穷个~$\delta$,若只有有限个~$\delta$,则最小的~$\delta$~就是公共的~$\delta$,现在无穷个~$\delta$~里
面,不一定有大于零的最小~$\delta$,也就不一定存在公共的~$\delta$。从几何直观来看,如果曲线有一个地方最陡,则最陡处的~$\delta$,对
其它点也适用,这时就找到了公共的~$\delta$;若曲线无限地变陡,没有一处坡度最陡的地方,这时就找不到公共的~$\delta$。

对于用公共~$\delta$~的函数,我们给出如下定义。

\begin{definition}
设~$f(x)$~在区间~$I$~上定义,若对任意~$\e>0$,存在~$\delta=\delta(\e)>0$,使得对任意~$x_1,x_2\in I$,%
当~$\mabs{x_1-x_2}<\delta$~时,有
\[
  \mabsb{f(x_1)-f(x_2)}<\e,
\]
则称~$f(x)$~在~$I$~上\emph{一致连续}。
\end{definition}

显然,函数在区间~$I$~上一致连续,必在区间~$I$~上连续。

\begin{example}
证明~$y=x^2$~在~$\mintc 0a$~上一致连续。
\end{example}
\begin{proof}
对任意~$\e>0$,当~$x_1,x_2\in\mintc 0a$~时,有
\[
  \mabsb{x_1^2-x_2^2}<\e\impliedby
  \mabs{x_1+x_2}\cdot\mabs{x_1-x_2}\leq 2a\mabs{x_1-x_2}<\e 。
\]
去~$\delta=\dfrac\e{2a}$,则当~$\mabs{x_1-x_2}<\delta$~时,就有
\[
  \mabsb{x_1^2-x_2^2}<\e 。
\]
由定义可知函数在~$\mintc 0a$~上一致连续。
\end{proof}

\begin{example}
证明~$y=\sqrt x$~在~$\mintco 0{+\infty}$~上一致连续。
\end{example}

由\ref{fig:sec3.5-2}~看出,函数在~$x=0$~处坡度最陡,$x=0$~处的~$\delta$~即可最为公共的~$\delta$。

%\fixwrapfloatsep
\begin{wrapfigure}{O}{0mm}
\somefigure
\caption{}\label{fig:sec3.5-2}
\end{wrapfigure}

\begin{proof}
对任意~$\e>0$,取~$\delta=\e^2>0$,当~$x_1,x_2\in\mintco 0\pinf$,$\mabs{x_1-x_2}<\delta$,我们分两种情形讨论。
\begin{enumlist}
\item $\max\mrange{x_1}{x_2}\geq\e^2$~时,
\[
  \mabsb{\sqrt{x_1}-\sqrt{x_2}}=\mabsbb{\frac{x_1-x_2}{\sqrt{x_1}+\sqrt{x_2}}}\leq\frac{\mabs{x_1-x_2}}{\sqrt{\e^2}}
  <\frac\delta\e=\e;
\]
\item $\max\mrange{x_1}{x_2}<\e^2$~时,无妨设~$x_1\geq x_2$~时
\[
  \mabsb{\sqrt{x_1}-\sqrt{x_2}}\leq\sqrt{x_1}<\sqrt{\e^2}=\e 。
\]
所以函数~$\sqrt x$~在~$\mintco 0\pinf$~上一致连续。\qedhere
\end{enumlist}
\end{proof}

从图形可以看出,闭区间上的连续函数,总有一处坡度最陡,在这一处坡度最陡,在这一处使用的~$\delta$~即为公共的~$\delta$,于是
有如下定理。

\begin{theorem}\label{thm:sec3.5-3}
设函数~$f(x)\in C\mintc ab$,则~$f(x)$~在~$\mintc ab$~上一致连续。
\end{theorem}

有没有不一致连续的连续函数呢?有的。例如,函数~$y=\sfrac1x$,当~$x\to0$~是,曲线无限地变陡,且没有最陡的点,因此没有
公共的~$\delta$,也就在~$\minto 01$~上不一致连续。又例如~$y=\sin(\sfrac1x)$,当~$x\to 0$~时,也无限地变陡,且没有最陡的点,因此
不可能找出公共的~$\delta$,所以在~$\minto 01$~上也不一致连续。上面只是直观叙述,怎么来严格论证呢?为此我们还是从一致连续开始。

一致连续是说,对区间上任意两点,不管这两点位置在哪儿,只要这两点距离小于~$\delta$,相应的曲线高度之差总小于~$\e$。根据这一点,要
检查曲线是否一致连续,可以作一形象的说法,把曲线比作工厂的产品,一致连续表示合格品。我们是用如下办法检查产品是否是合格品。用一系列
直径为~$\e$~的空心圆管,若对每一直径为~$\e$~的空心圆管,总能锯下长为~$\delta$~的一小段,把这小段圆管套进曲线一端,允许小段管子上下
评语,而不允许倾斜或侧转,如果圆管能从曲线一端进去,平行移动从另一端出来,而不碰到曲线,就认为检查通过。对每一直径为~$\e$~的圆
管,检查都能通过,产品就是合格品,即曲线一致连续。

%\fixwrapfloatsep
\begin{wrapfigure}{O}{0mm}
\somefigure
\caption{}\label{fig:sec3.5-3}
\end{wrapfigure}

所以,不一致连续,即产品是不合格品,那么总有一道检查手续通不过,即存在~$\e_0>0$~的圆管,不可能锯下一小段使得上面的检查通过。也就
是是,对任意~$\delta>0$,总可以锯下长为~$\delta'$~的一小段($\delta>\delta'$),不可能从曲线一端套进去,平行移动从另一端出来,而
不碰到曲线,这意味着小圆管总要在曲线的某个地方被卡住(见\ref{fig:sec3.5-3}),即存在~$x',x''$,$\mabs{x'-x''}=\delta'<\delta$,但
\[
  \mabsb{f(x')-f(x'')}\geq\e_0 。
\]

综上所述,就可以得到不一致连续的“$\e-\delta$”说法:

存在~$\e_0>0$,使得对任意~$\delta>0$,都存在~$x',x''\in I$,即使~$\mabs{x'-x''}<\delta$,但是
\begin{equation}\label{eq:sec3.5-1}
  \mabsb{f(x')-f(x'')}\geq\e_0 。
\end{equation}

形式上把“任意”与“存在”交换一下,由一致连续的说法就变为不一致连续的说法。

在~\ref{eq:sec3.5-1}~中取~$\delta_n=\sfrac1n\mcond{n=1,2,\dotsc}$,就可以把不一致连续改述如下:

存在~$\e_0>0$,存在~$I$~内的序列~$\mbrace{x_n'},\mbrace{x_n''}$,$\lim_\ntoinf\mparen{x_n'-x_n''}=0$,但
\begin{equation}\label{eq:sec3.5-2}
  \mabsb{f(x_n')-f(x_n'')}\geq\e_0 。
\end{equation}

反之,若~\ref{eq:sec3.5-2}~成立,则~\ref{eq:sec3.5-1}~也成立,所以不一致连续的两种说法是等价的。

把序列看成动点所取得值,不一致连续就是说,存在两个动点,动点变动时距离趋于零,而动点的函数值总保持一定距离~$\e_0$。

利用~\ref{eq:sec3.5-2}~判别不一致连续,我们还嫌不便,事实上可以改得更简便些:

存在~$I$~内的序列~$\mbrace{x_n'}$~和~$\mbrace{x_n''}$,$\lim_\ntoinf\mparen{x_n'-x_n''}=0$,而
\begin{equation}\label{eq:sec3.5-3}
\lim_\ntoinf\mparenb{f(x_n')-f(x_n'')}=C\neq 0,
\end{equation}
则函数在~$I$~内不一致连续。

\ref{eq:sec3.5-3}~蕴涵~\ref{eq:sec3.5-2}。事实上,令~$\e_0=\sfrac{\mabs C}2>0$,当~$n>N$~时,
\[
  \mabsB{\mabsb{f(x_n')-f(x_n'')}-\mabs C}\leq\frac{\mabs C}2,
\]
所以
\[
  \mabsb{f(x_n')-f(x_n'')}\geq\mabs C-\frac{\mabs C}2=\frac{\mabs C}2 。
\]
再令~$\mtilde x_n'=x_{N+n}'$~和~$\mtilde x_n''=x_{N+n}''$,则
\[
  \lim_\ntoinf\mparenb{\mtilde x_n'-\mtilde x_n''}=\lim_\ntoinf\mparenb{x_{n+N}'-x_{n+N}''}=0,
\]
但是
\[
  \mabsb{\mtilde x_n'-\mtilde x_n''}\geq\frac{\mabs C}2=\e_0 。
\]
这就是说找到了~\ref{eq:sec3.5-2}~所要求的~$\e_0$~与序列~$\mbrace{\mtilde x_n'}$~和~$\mbrace{\mtilde x_n'}$。

这样,判断不一致连续,就归结为寻找符合~\ref{eq:sec3.5-3}~的两个序列即可。

\begin{example}
证明~$f(x)=\dfrac1x$~在~$\minto 01$~上不一致连续。
\end{example}
\begin{proof}
对于~$n=1,2,\dotsc$,取
\[
  x_n'=\dfrac1{n+1},\quad x_n''=\dfrac1{n+2},
\]
则
\[
  x_n'-x_n''=\frac1{n+1}-\frac1{n+2}\to 0,
\]
但
\[
  f(x_n')-f(x_n'')=(n+1)-(n+2)\to -1\neq 0,
\]
所以~$\sfrac1x$~在~$\minto01$~上不一致连续。
\end{proof}

\begin{example}
证明~$f(x)=x^2$~在~$\minto 0\pinf$~上不一致连续。
\end{example}
\begin{proof}
对于~$n=1,2,\dotsc$,取~$x_n'=\sqrt n$,$x_n''=\sqrt{n+1}$,则
\[
  x_n'-x_n''=\sqrt n-\sqrt{n+1}=\frac{-1}{\sqrt n+\sqrt{n+1}}\to 0,
\]
但是
\[
  f(x_n')-f(x_n'')=n-(n+1)=-1\to -1\neq0,
\]
所以~$x^2$~在~$\minto 0\pinf$~上不一致连续。
\end{proof}

\begin{exercise}
\item 设~$f(x)\in C\mintco a\pinf$,且~$\lim_{x\to\pinf}f(x)$~存在。证明~$f(x)$~在~$\mintco a\pinf$~上有界。
\item 设~$f(x)\in C(\mR)$,且~$\lim_{x\to\pm\infty}f(x)=+\infty$。证明~$f(x)$~在~$\mR$~上取到它的最小值。
\item 设~$f(x)\in C\mintco ab$,且~$\lim_{x\to b-0}f(x)=B$。
\begin{exlist}
  \item 若存在~$x_1\in\mintco ab$,使得~$f(x_1)>B$,证明~$f(x)$~在~$\mintco ab$~上达到最大值;
  \item 若存在~$x_1\in\mintco ab$,使得~$f(x_1)=B$,讨论~$f(x)$~在~$\mintco ab$~上能否达到最大值。
\end{exlist}
\item 设~$f(x),g(x)\in C\mintc ab$。证明
\[
  \max_{x\in\mintc ab}\mbraceb{\mabsb{f(x)+g(x)}}\leq
  \max_{x\in\mintc ab}\mbraceb{\mabsb{f(x)}}+\max_{x\in\mintc ab}\mbraceb{\mabsb{g(x)}}。
\]
\item 设~$f(x)\in C(\mR)$,且~$\lim_{x\to\infty}f\mparenb{f(x)}=\infty$。证明~$\lim_{x\to\infty}f(x)=\infty$。
\item 设~$f(x)\in C\mintc 01$,且~$f(x)>0$。令
\[
  M(x)\coloneq\max_{t\in\mintc0x}\mbrace{f(t)}\mcond*{x\in\mintc 01}。
\]
证明函数
\[
  Q(x)\coloneq\lim_\ntoinf\mparenbb{\frac{f(x)}{M(x)}}^{\msp n}
\]
连续当且仅当~$f(x)$~在~$\mintc 01$~上时单调上升的。
\item 设~$f(x)\in C\mintc ab$,记
\[
  g(x)\coloneq\max_{t\in\mintc ax}\mbraceb{f(t)}\mcond*{x\in\mintc ab}。
\]
证明~$g(x)\in C\mintc ab$。
\item 设~$f(x)\in C\mintc ab$,并且对任意~$x\in\mintc ab$,存在~$y\in\mintc ab$,使得~$\mabsb{f(y)}\leq\dfrac12\mabsb{f(x)}$。证
明,存在~$\xi\in\mintc ab$,使得~$f(\xi)=0$。
\item 设~$f(x)\in C(\mR)$,若~$\lim_{x\to+\infty}f(x)=+\infty$,且~$f(x)$~的最小值~$f(a)<a$。证明~$f\mparenb{f(x)}$~至少在
两点取到最小值。
\item 设~$f(x)\in\mintc{-1}1$,$f(0)>0$,$f(\pm1)=0$。证明,存在常数~$a>0$~和~$b$,使得
\[
g(x)=-a\mabs x+b\geq f(x)\mcond*{x\in\mintc{-1}1};
\]
且存在~$c\mcond{-1<c<1}$,使得~$g(c)=-a\mabs c+b=f(c)$。
\begin{exlistcols*}
\item 证明~$y=x$~和~$y=\sin x$~在~$\mR$~上一致连续。
\item 证明~$y=\sqrt[3]x$~在~$\mintco 0\pinf$~上一致连续。
\end{exlistcols*}
\item 证明下列函数在~$\mR$~上不一致连续。
\begin{exlistcols}
  \item $y=x^3$;
  \item $y=x\cdot\sin x$。
\end{exlistcols}
\item 证明~$f(x)=\dfrac{\mabs{\sin x}}x$~在~$\minto{-1}0$~和~$\minto 01$~上一致连续,但是在~$0<\mabs x<1$~上并非一致连续。
\item 设~$f(x)$~在~$\minto ab$~上一致连续。证明,
\begin{exlistcols}
  \item $f(x)$~在~$\minto ab$~上有界;
  \item $y=\ln x$~在~$\minto 01$~上不一致连续。
\end{exlistcols}
\item 设~$f(x),g(x)$~在~$\minto ab$~上一致连续。证明~$f(x)+g(x)$~和~$f(x)\cdot g(x)$~也在~$\minto ab$~上一致连续。
\item 若周期函数~$f(x)\in C(\mR)$,证明
\begin{exlistcols}
  \item $f(x)$~在~$\mR$~上一致连续;
  \item $f(x)=\sin^2x+\sin x^2$~不是周期函数。
\end{exlistcols}
\item 设~$f(x)$~在~$\mintc ab$~上满足~\emph{Lipschitz~条件},即对任意~$x,y\in\mintc ab$,有
\[
  \mabsb{f(x)-f(y)}\leq k\mabs{x-y},
\]
这里~$k$~为常数。证明~$f(x)$~在~$\mintc ab$~上一致连续。
\item 设~$f(x)$~在~$\mintco a\pinf\mcond{a>0}$~上满足~Lipschitz~条件,证明~$\dfrac{f(x)}x$~在~$\mintco a\pinf$~上一致连续。
\item 设~$f(x)\in C(\mR)$,且~$\lim_{x\to-\infty}f(x)$~与~$\lim_{x\to+\infty}f(x)$~存在。证明~$f(x)$~在~$\mR$~上一致连续,并
判断下述证明的正确性。
\begin{exproof}
对任意~$\e>0$,有~$\lim_{x\to+\infty}f(x)=A$,存在~$X_1>0$,使得当~$x>X_1$~时,有
\[
  \mabsb{f(x)-A}<\frac\e2 。
\]
所以当~$x_1,x_2>X_1$~时,有
\[
  \mabsb{f(x_1)-f(x_2)}\leq\mabsb{f(x_1)-A}+\mabsb{f(x_2)-A}<\e 。
\]
由此即可得到~$f(x)$~在~$\mintco {X_1}\pinf$~上一致连续。

同理,由~$\lim_{x\to-\infty}f(x)=B$,存在~$X_2<0$,当~$x_1,x_2<X_2$~时,有
\[
  \mabsb{f(x_1)-f(x_2)}<\e 。
\]
由此即可得到~$f(x)$~在~$\mintoc\minf{X_2}$~上一致连续。

又因为~$f(x)$~在闭区间~$\mintc{X_2}{X_1}$~上连续,故在闭区间~$\mintc{X_2}{X_1}$~上一致连续。

综上所述就得到~$f(x)$~在~$\mR$~上一致连续。
\end{exproof}
\item 证明~$f(x)$~在闭区间~$\mintc ab$~上连续必一致连续,并判断下述证明的正确性。
\begin{exproof}
用反证法。若~$f(x)$~在~$\mintc ab$~上不一致连续,则存在~$\e_0>0$,使得对任意~$\delta_1>0$,存在~$x_1,x_2\in\mintc ab$,%
即使~$\mabs{x_1}{x_2}<\delta_1$,但是
\begin{equation}\label{eq:sec3.5-ex-1}
\mabsb{f(x_1)-f(x_2)}\geq\e_0 。
\end{equation}

由~$f(x)$~在~$x=x_2$~点连续,对上述~$\e_0$,存在~$\delta>0$,使得当~$\mabs{x-x_2}<\delta$~时,有
\begin{equation}\label{eq:sec3.5-ex-2}
\mabsb{f(x_1)-f(x_2)}<\e_0 。
\end{equation}

由于~\ref{eq:sec3.5-ex-1}~对任意~$\delta_1>0$~成立,取~$\delta_1=\delta$,则当~$\mabs{x_1-x_2}<\delta$~时,有
\begin{equation}\label{eq:sec3.5-ex-3}
\mabsb{f(x_1)-f(x_2)}\geq\e_0 。
\end{equation}

又因为~\ref{eq:sec3.5-ex-2}~对任意满足~$\mabs{x-x_2}<\delta$~的~$x$~成立。取~$x=x_1$,则当~$\mabs{x_1-x_2}<\delta$~时,有
\begin{equation}\label{eq:sec3.5-ex-4}
\mabsb{f(x_1)-f(x_2)}<\e_0 。
\end{equation}
\ref{eq:sec3.5-ex-3}~与~\ref{eq:sec3.5-ex-4}~矛盾,这说明反证法假设不成立,从而~$f(x)$~在~$\mintc ab$~上一致连续。
\end{exproof}
\end{exercise}

\begin{exercise*}
\item 设~$f(x)$~在~$\mR$~上单调,且对任意~$x,y\in\mR$,都满足函数方程
\[
  f(x+y)=f(x)+f(y) 。
\]
证明,
\begin{exlistcols}
  \item $f(nx)=nf(x)\mcond{n\in\mZ}$;
  \item $f\mparenB{\dfrac xn}=\dfrac1nf(x)\mcond{n\in\mZ\difset\mbrace 0}$;
  \item $f(\alpha x)=\alpha f(x)\mcond{\alpha\in\mQ}$;
  \item $f(x)$~在~$x=0$~处连续;
  \item $f(x)$~在~$\mR$~上连续;
  \item $f(x)=f(1)\cdot x$。
\end{exlistcols}
\item 设~$f(x)$~在~$\mR$~上单调,且对任意~$x,y\in\mR$,都满足函数方程
\[
  f\mparenB{\frac{x+y}2}=\frac{f(x)+f(y)}2 。
\]
证明~$f(x)=f(0)+f(1)\cdot x$。
\item 设~$f(x)$~在~$\minto ab$~上连续,且对任意~$x,y\in\minto ab$,有
\[
  f\mparenB{\frac{x+y}2}\leq\frac{f(x)+f(y)}2 。
\]
%证明,
\begin{exlist}\FixExHead
  \item 对任意~$\mvec x{1,2,:,n}\in\minto ab$,有
  \[
    f\mparenB{\frac{x_1+x_2+\dotsb+x_n}n}\leq\frac1n\mparenb{f(x_1)+f(x_2)+\dotsb+f(x_n)};
  \]
  \item 对任意~$t\in\minto 01$~和任意~$x_1,x_2\in\minto ab$,有
  \[
    f\mparenb{tx_1+(1-t)x_2}\leq tf(x_1)+(1-t)f(x_2)。
  \]
\end{exlist}
\item 证明~Riemann~函数
\[
  R(x)=\begin{Bdcases}
    \frac1n, & x=\frac mn~\text{且}~(m,n)=1;\\
    0,       & x\in\mR\difset\mQ,
  \end{Bdcases}
\]
在任一有理点间断,而在任一无理点连续。
\item 设~$f(x)\in C\mintc 01$,$f(0)=f(1)=0$,并且~$f(x)$~有正的最大值。证明,存在二次多项式
\[
  Q(x)=ax^2+bx+c\mcond{a<0},
\]
使得~$Q(x)\geq f(x)\mcond{0\leq x\leq 1}$~且~$Q(\alpha)=f(\alpha)\mcond{0<\alpha<1}$。
\item 设~$f(x)\in C(\mR)$,且~$\lim_{x\to\pm\infty}f\mparenb{f(x)}=+\infty$。证明~$\lim_{x\to\pm\infty}f(x)=+\infty$。
\item 证明,方程~$x^2+3x^3+7x^4-5x^6-8x^7=0$~有且仅有一个正实数根。
\item 设~$f(x),g(x)\in C\mintc ab$。证明,
\begin{exlist}
  \item 若~$f(x_0)=\max_{x\in\mintc ab}\mbraceb{\mabsb{f(x)}}>\max_{x\in\mintc ab}\mbraceb{\mabsb{g(x)}}$,则~$f(x_0)>g(x_0)$;
  \item 若~$\max_{x\in\mintc ab}\mbraceb{\mabsb{f(x)}}>\max_{x\in\mintc ab}\mbraceb{\mabsb{g(x)}}$,则~$f(x)$~与~$f(x)-g(x)$~在
  ~$\mabsb{f(x)}$~的最大值点处有相同的符号。
\end{exlist}
\item 设在~$\mintc{-1}1$~上,对于~$n=1,2,\dotsc$,有
\[
  T_n(x)=\frac1{2^{n-1}}\cos(n\arccos x) 。
\]
%证明,
\begin{exlist}\FixExHead
  \item $\mabs{T_n(x)}$~在~$n+1$~个点~$x_k=\cos\dfrac kn\pi\mcond{k=0,1,\dotsc,n}$~上达到最大值;
  \item 对任一首项系数为~$1$~的~$n$~次多项式~$P(x)=x^n+a_1x^{n-1}+\dotsc+a_n$,必有
  \[
    \max_{x\in\mintc{-1}1}\mbraceb{\mabsb{P(x)}}\geq\max_{x\in\mintc{-1}1}\mbraceb{\mabsb{T_n(x)}}=\frac1{2^{n-1}} 。
  \]
\end{exlist}
\item 设~$f(x)$~在~$\mintc ab$~上定义。对~$\mintc ab$~上任一闭区间~$\mintc{x_1}{x_2}\subset\mintc ab$~和介于~$f(x_1)$~与~$f(x_2)$~
之间的任一常数~$\ell$,方程~$f(x)=\ell$~在~$\mintc{x_1}{x_2}$~上有且仅有有限个解。证明~$f(x)\in C\mintc ab$。
\item 证明,在~$x\geq0$~上有非零基本初等连续函数~$f(x)$~满足
\[
  f(2^2x)=f(2x)+f(x) 。
\]
\end{exercise*}



\endinput
%%
%% End of file `MAChapter3.tex'.