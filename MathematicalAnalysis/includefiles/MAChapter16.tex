%# -*- coding:utf-8 -*-
%%%%%%%%%%%%%%%%%%%%%%%%%%%%%%%%%%%%%%%%%%%%%%%%%%%%%%%%%%%%%%%%%%%%%%%%%%%%%%%%%%%%%
%%  MAChapter16.tex'


\chapter{多元数值函数的微分学}\label{ch:16}
\section{偏导数}
\subsection{偏导数的概念}
\subsection{偏导数的求法}
\subsection{微分中值定理}
\subsection{偏导数的存在性与函数的连续性}
\begin{exercise}
\item
\end{exercise}
\section{全微分与可微性}
\subsection{一次逼近与全微分的概念}
\subsection{连续性与可微性,偏导数与可微性}
\subsection{全微分的四则运算法则}
\subsection{全微分的几何意义}
\begin{exercise}
\item
\end{exercise}
\section{复合函数的偏导数与可微性}
\subsection{复合函数的求导法则——链锁法则}
\subsection{复合函数的可微性与一阶全微分形式的不变性}
\begin{exercise}
\item
\end{exercise}
\section{方向导数}
\subsection{方向导数的概念与计算}
\subsection{梯度向量}
\begin{exercise}
\item
\end{exercise}
\section{高阶偏导数和高阶全微分}
\subsection{高阶偏导数的概念}
\subsection{混合偏导数与求导顺序}
\subsection{复合函数的高阶偏导数}
\subsection{高阶全微分}
\section{Taylor~公式}
\begin{exercise}
\item
\end{exercise}
\section{由一个方程式确定的隐函数及其微分法}
\subsection{函数的存在唯一性与连续性}
\subsection{隐函数的可微性}
\begin{exercise}
\item
\end{exercise}
\begin{exercise*}
\item
\end{exercise*}




\endinput
%%
%% End of file `MAChapter16.tex'.