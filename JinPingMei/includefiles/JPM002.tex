%# -*- coding:utf-8 -*-
%%%%%%%%%%%%%%%%%%%%%%%%%%%%%%%%%%%%%%%%%%%%%%%%%%%%%%%%%%%%%%%%%%%%%%%%%%%%%%%%%%%%%


\chapter{俏潘娘帘下勾情\KG 老王婆茶坊说技}


词曰:

\[
芙蓉面,冰雪肌,生来娉婷年已笄。袅袅倚门余。梅花半含蕊,似开还闭。初见帘边,羞涩还留住;再过楼头,款接多欢喜。行也宜,立也宜,坐也宜,偎傍更相宜。
\]

话说当日武松来到县前客店内,收拾行李铺盖,交土兵挑了,引到哥家。那妇人见了,强如拾得金宝一般欢喜,旋打扫一间房与武松安顿停当。武松分付土兵回去,当晚就在哥家歇宿。次日早起,妇人也慌忙起来,与他烧汤净面。武松梳洗裹帻,出门去县里画卯。妇人道:“叔叔画了卯,早些来家吃早饭,休去别处吃了。”武松应的去了。到县里画卯已毕,伺候了一早晨,回到家,那妇人又早齐齐整整安排下饭。三口儿同吃了饭,妇人双手便捧一杯茶来,递与武松。武松道:“交嫂嫂生受,武松寝食不安,明日拨个土兵来使唤。”那妇人连声叫道:“叔叔却怎生这般计较!自家骨肉,又不服事了别人。虽然有这小丫头迎儿,奴家见他拿东拿西,蹀里蹀斜,也不靠他。就是拨了土兵来,那厮上锅上灶不乾净,奴眼里也看不上这等人。”武松道:“恁的却生受嫂嫂了。”有诗为证:

\[
武松仪表岂风流,嫂嫂淫心不可收。
笼络归来家里住,相思常自看衾稠。
\]

话休絮烦。自从武松搬来哥家里住,取些银子出来与武大,买饼馓茶果,请那两边邻舍。都斗分子来与武松人情。武大又安排了回席,不在话下。过了数日,武松取出一匹彩色段子与嫂嫂做衣服。那妇人堆下笑来,便道:“叔叔如何使得!既然赐与奴家,不敢推辞。”只得接了,道个万福。自此武松只在哥家宿歇。武大依前上街挑卖炊饼。武松每日自去县里承差应事,不论归迟归早,妇人顿茶顿饭,欢天喜地伏侍武松,武松倒觉过意不去。那妇人时常把些言语来拨他,武松是个硬心的直汉。

有话即长,无话即短,不觉过了一月有余,看看十一月天气,连日朔风紧起,只见四下彤云密布,又早纷纷扬扬飞下一天瑞雪来。好大雪!怎见得?但见:

\[
万里彤雪密布,空中瑞祥飘帘。琼花片片舞前檐。剡溪当此际,濡滞子猷船。顷刻楼台都压倒,江山银色相连。飞盐撒粉漫连天。当时吕蒙正,窑内叹无钱。
\]

当日这雪下到一更时分,却早银妆世界,玉碾乾坤。次日武松去县里画卯,直到日中未归。武大被妇人早赶出去做买卖,央及间壁王婆买了些酒肉,去武松房里簇了一盆炭火。心里自想道:“我今日着实撩斗他他一撩斗,不怕他不动情。”那妇人独自冷冷清清立在帘儿下,望见武松正在雪里,踏着那乱琼碎玉归来。妇人推起帘子,迎着笑道:“叔叔寒冷?”武松道:“感谢嫂嫂挂心。”入得门来,便把毡笠儿除将下来。那妇人将手去接,武松道:“不劳嫂嫂生受。”自把雪来拂了,挂在壁子上。随即解了缠带,脱了身上鹦哥绿纻丝衲袄,入房内。那妇人便道:“奴等了一早晨,叔叔怎的不归来吃早饭?”武松道:“早间有一相识请我吃饭,却才又有作杯,我不耐烦,一直走到家来。”妇人道:“既恁的,请叔叔向火。”武松道:“正好。”便脱了油靴,换了一双袜子,穿了暖鞋,掇条凳子,自近火盆边坐地。那妇人早令迎儿把前门上了闩,后门也关了。却搬些煮熟菜蔬入房里来,摆在桌子上。武松问道:“哥哥那里去了?”妇人道:“你哥哥出去买卖未回,我和叔叔自吃三杯。”武松道:“一发等哥来家吃也不迟。”妇人道:“那里等的他!”说犹未了,只见迎儿小女早暖了一注酒来。武松道:“又教嫂嫂费心。”妇人也掇一条凳子,近火边坐了。桌上摆着杯盘,妇人拿盏酒擎在手里,看着武松道:“叔叔满饮此杯。”武松接过酒去,一饮而尽。那妇人又筛一杯酒来,说道:“天气寒冷,叔叔饮过成双的盏儿。”武松道:“嫂嫂自请。”接来又一饮而尽。武松却筛一杯酒,递与妇人。妇人接过酒来呷了,却拿注子再斟酒放在武松面前。那妇人一径将酥胸微露,云鬟半軃,脸上堆下笑来,说道:“我听得人说,叔叔在县前街上养着个唱的,有这话么?”武松道:“嫂嫂休听别人胡说,我武二从来不是这等人。”妇人道:“我不信!只怕叔叔口头不似心头。”武松道:“嫂嫂不信时,只问哥哥就是了。”妇人道:“啊呀,你休说他,那里晓得甚么?如在醉生梦死一般!他若知道时,不卖炊饼了。叔叔且请杯。”连筛了三四杯饮过。那妇人也有三杯酒落肚,哄动春心,那里按纳得住。欲心如火,只把闲话来说。武松也知了八九分,自己只把头来低了,却不来兜揽。妇人起身去烫酒。武松自在房内却拿火箸簇火。妇人良久暖了一注子酒来,到房里,一只手拿着注子,一只手便去武松肩上只一捏,说道:“叔叔只穿这些衣裳,不寒冷么?”武松已有五七分不自在,也不理他。妇人见他不应,匹手就来夺火箸,口里道:“叔叔你不会簇火,我与你拨火。只要一似火盆来热便好。”武松有八九分焦燥,只不做声。这妇人也不看武松焦燥,便丢下火箸,却筛一杯酒来,自呷了一口,剩下半盏酒,看着武松道:“你若有心,吃我这半盏儿残酒。”武松匹手夺过来,泼在地下说道:“嫂嫂不要恁的不识羞耻!”把手只一推,争些儿把妇人推了一交。武松睁起眼来说道:“武二是个顶天立地噙齿戴发的男子汉,不是那等败坏风俗伤人伦的猪狗!嫂嫂休要这般不识羞耻,为此等的勾当,倘有风吹草动,我武二眼里认的是嫂嫂,拳头却不认的是嫂嫂!”妇人吃他几句抢得通红了面皮,便叫迎儿收拾了碟盏家伙,口里说道:“我自作耍子,不直得便当真起来。好不识人敬!”收了家伙,自往厨下去了。正是:

\[
落花有意随流水,流水无情恋落花。
\]

这妇人见勾搭武松不动,反被他抢白了一场。武松自在房中气忿忿,自己寻思。天色却是申牌时分,武大挑着担儿,大雪里归来。推门进来,放下担儿,进的里间,见妇人一双眼哭的红红的,便问道:“你和谁闹来?”妇人道:“都是你这不不争气的,交外人来欺负我。”武大道:“谁敢来欺负你?”妇人道:“情知是谁?争奈武二那厮。我见他大雪里归来,好意安排些酒饭与他吃,他见前后没人,便把言语来调戏我。便是迎儿眼见,我不赖他。”武大道:“我兄弟不是这等人,从来老实。休要高声,乞邻舍听见笑话。”武大撇了妇人,便来武二房里叫道:“二哥,你不曾吃点心?我和你吃些个。”武松只不做声,寻思了半晌,一面出大门。武大叫道:“二哥,你那里去?”也不答应,一直只顾去了。武大回到房内,问妇人道:“我叫他又不应,只顾望县里那条路去了。正不知怎的了?”妇人骂道:“贼馄饨虫!有甚难见处?那厮羞了,没脸儿见你,走了出去。我猜他一定叫人来搬行李,不要在这里住。却不道你留他?”武大道:“他搬了去,须乞别人笑话。”妇人骂道:“混沌魍魉,他来调戏我,到不乞别人笑话!你要便自和他过去,我却做不的这样人!你与了我一纸休书,你自留他便了。”武大那里敢再开口。被这妇人倒数骂了一顿。正在家两口儿絮聒,只见武松引了个土兵,拿着条扁担,迳来房内收拾行李,便出门。武大走出来,叫道:“二哥,做甚么便搬了去?”武松道:“哥哥不要问,说起来装你的幌子,只由我自去便了。”武大那里再敢问备细,由武松搬了出去。那妇人在里面喃喃呐呐骂道:“却也好,只道是亲难转债,人不知道一个兄弟做了都头,怎的养活了哥嫂,却不知反来咬嚼人!正是花木瓜空好看。搬了去,倒谢天地,且得冤家离眼睛。”武大见老婆这般言语,不知怎的了,心中反是放不下。自从武松搬去县前客店宿歇,武大自依前上街卖炊饼。本待要去县前寻兄弟说话,却被这妇人千叮万嘱,分付交不要去兜揽他,因此武大不敢去寻武松。

说这武松自从搬离哥家,捻指不觉雪晴,过了十数日光景。却说本县知县自从到任以来,却得二年有余,转得许多金银,要使一心腹人送上东京亲眷处收寄,三年任满朝觐,打点上司。一来却怕路上小人,须得一个有力量的人去方好,猛可想起都头武松,须得此人方了得此事。当日就唤武松到衙内商议道:“我有个亲戚在东京城内做官,姓朱名靦,见做殿前太尉之职,要送一担礼物,捎封书去问安。只恐途中不好行,若得你去方可。你休推辞辛苦,回来我自重赏。”武松应道:“小人得蒙恩相抬举,安敢推辞!既蒙差遣,只此便去。”知县大喜,赏了武松三杯酒,十两路费。不在话下。

且说武松领了知县的言语,出的县门来,到下处,叫了土兵,却来街上买了一瓶酒并菜蔬之类,迳到武大家。武大却街上回来,见武松在门前坐地,交土兵去厨下安排。那妇人余情不断,见武松把将酒食来,心中自思:“莫不这厮思想我了?不然却又回来怎的?到日后我且慢慢问他。”妇人便上楼去重匀粉面,再整云鬟,换了些颜色衣服,来到门前迎接武松。妇人拜道:“叔叔,不知怎的错见了,好几日并不上门,叫奴心里没理会处。今日再喜得叔叔来家。没事坏钞做甚么?”武松道:“武二有句话,特来要与哥哥说知。”妇人道:“既如此,请楼上坐。”三个人来到楼上,武松让哥嫂上首坐了,他便掇杌子打横。土兵摆上酒,并嗄饭一齐拿上来。武松劝哥嫂吃。妇人便把眼来睃武松,武松只顾吃酒。酒至数巡,武松问迎儿讨副劝杯,叫土兵筛一杯酒拿在手里,看着武大道:“大哥在上,武二今日蒙知县相公差往东京干事,明日便要起程,多是两三个月,少是一月便回,有句话特来和你说。你从来为人懦弱,我不在家,恐怕外人来欺负。假如你每日卖十扇笼炊饼,你从明日为始,只做五扇笼炊饼出去,每日迟出早归,不要和人吃酒。归家便下了帘子,早闭门,省了多少是非口舌。若是有人欺负你,不要和他争执,待我回来,自和他理论。大哥你依我时,满饮此杯!”武大接了酒道:“兄弟见得是,我都依你说。”吃过了一杯,武松再斟第二盏酒,对那妇人说道:“嫂嫂是个精细的人,不必要武松多说。我的哥哥为人质朴,全靠嫂嫂做主。常言表壮不如里壮,嫂嫂把得家定,我哥哥烦恼做甚么!岂不闻古人云:篱牢犬不入。”那妇人听了这句话,一点红从耳边起,须臾紫涨了面皮,指着武大骂道:“你这个混沌东西。有甚言语在别处说,来欺负老娘!我是个不带头巾的男子汉,叮叮当当响的婆娘!拳头上也立得人,胳膊上走得马,不是那腲脓血搠不出来鳖!老娘自从嫁了武大,真个蚂蚁不敢入屋里来,甚么篱笆不牢犬儿钻得入来?你休胡言乱语,一句句都要下落!丢下一块瓦砖儿,一个个也要着地!”武松笑道:“若得嫂嫂做主,最好。只要心口相应。既然如此,我武松都记得嫂嫂说的话了,请过此杯。”那妇人一手推开酒盏,一直跑下楼来,走到在胡梯上发话道:“既是你聪明伶俐,恰不道长嫂为母。我初嫁武大时,不曾听得有甚小叔,那里走得来?是亲不是亲,便要做乔家公。自是老娘晦气了,偏撞着这许多鸟事!”一面哭下楼去了。正是:

\[
苦口良言谏劝多,金莲怀恨起风波。
自家惶愧难存坐,气杀英雄小二哥。
\]

那妇人做出许多乔张致来。武大、武松吃了几杯酒,坐不住,都下的楼来,弟兄洒泪而别。武大道:“兄弟去了,早早回来,和你相见。”武松道:“哥哥,你便不做买卖也罢,只在家里坐的。盘缠,兄弟自差人送与你。”临行,武松又分付道:“哥哥,我的言语休要忘了,在家仔细门户。”武大道:“理会得了。”武松辞了武大,回到县前下处,收拾行装并防身器械。次日领了知县礼物,金银驼垛,讨了脚程,起身上路,往东京去了,不题。

只说武大自从兄弟武松说了去,整整吃那婆娘骂了三四日。武大忍声吞气,由他自骂,只依兄弟言语,每日只做一半炊饼出去,未晚便回来。歇了担儿,便先去除了帘子,关上大门,却来屋里坐的。那妇人看了这般,心内焦燥,骂道:“不识时浊物!我倒不曾见,日头在半天里便把牢门关了,也吃邻舍家笑话,说我家怎生禁鬼。听信你兄弟说,空生着卵鸟嘴,也不怕别人笑耻!”武大道:“由他笑也罢,我兄弟说的是好话,省了多少是非。”被妇人啐在脸上道:“呸!浊东西!你是个男子汉,自不做主,却听别人调遣!”武大摇手道:“由他,我兄弟说的是金石之语。”原来武松去后,武大每日只是晏出早归,到家便关门。那妇人气生气死,和他合了几场气。落后闹惯了,自此妇人约莫武大归来时分,先自去收帘子,关上大门。武大见了,心里自也暗喜,寻思道:“恁的却不好?”有诗为证:

\[
慎事关门并早归,眼前恩爱隔崔嵬。
春心一点如丝乱,任锁牢笼总是虚。
\]

白驹过隙,日月如梭,才见梅开腊底,又早天气回阳。一日,三月春光明媚时分,金莲打扮光鲜,单等武大出门,就在门前帘下站立。约莫将及他归来时分,便下了帘子,自去房内坐的。一日也是合当有事,却有一个人从帘子下走过来。自古没巧不成话,姻缘合当凑着。妇人正手里拿着叉竿放帘子,忽被一阵风将叉竿刮倒,妇人手擎不牢,不端不正却打在那人头上。妇人便慌忙陪笑,把眼看那人,也有二十五六年纪,生得十分浮浪。头上戴着缨子帽儿,金铃珑簪儿,金井玉栏杆圈儿;长腰才,身穿绿罗褶儿;脚下细结底陈桥鞋儿,清水布袜儿;手里摇着洒金川扇儿,越显出张生般庞儿,潘安的貌儿。可意的人儿,风风流流从帘子下丢与个眼色儿。这个人被叉竿打在头上,便立住了脚,待要发作时,回过脸来看,却不想是个美貌妖娆的妇人。但见他黑鬒鬒赛鸦鸰的鬓儿,翠弯弯的新月的眉儿,香喷喷樱桃口儿,直隆隆琼瑶鼻儿,粉浓浓红艳腮儿,娇滴滴银盆脸儿,轻袅袅花朵身儿,玉纤纤葱枝手儿,一捻捻杨柳腰儿,软浓浓粉白肚儿,窄星星尖翘脚儿,肉奶奶胸儿,白生生腿儿,更有一件紧揪揪、白鲜鲜、黑茵茵,正不知是甚么东西。观不尽这妇人容貌。且看他怎生打扮?但见:

\[
头上戴着黑油油头发髢髻,一迳里踅出香云,周围小簪儿齐插。斜戴一朵并头花,排草梳儿后押。难描画,柳叶眉衬着两朵桃花。玲珑坠儿最堪夸,露来酥玉胸无价。毛青布大袖衫儿,又短衬湘裙碾绢纱。通花汗巾儿袖口儿边搭剌。香袋儿身边低挂。抹胸儿重重纽扣香喉下。往下看尖翘翘金莲小脚,云头巧缉山鸦。鞋儿白绫高底,步香尘偏衬登踏。红纱膝裤扣莺花,行坐处风吹裙裤。口儿里常喷出异香兰麝,樱桃口笑脸生花。人见了魂飞魄丧,卖弄杀俏冤家。
\]
那人一见,先自酥了半边,那怒气早已钻入爪洼国去了,变做笑吟吟脸儿。这妇人情知不是,叉手望他深深拜了一拜,说道:“奴家一时被风失手,误中官人,休怪!”那人一面把手整头巾,一面把腰曲着地还喏道:“不妨,娘子请方便。”却被这间壁住的卖茶王婆子看见。那婆子笑道:“兀的谁家大官人打这屋檐下过?打的正好!”那人笑道:“倒是我的不是,一时冲撞,娘子休怪。”妇人答道:“官人不要见责。”那人又笑着大大地唱个喏,回应道:“小人不敢。”那一双积年招花惹草,惯觑风情的贼眼,不离这妇人身上,临去也回头了七八回,方一直摇摇摆摆遮着扇儿去了。

风日晴和漫出游,偶从帘下识娇羞。只因临去秋波转,惹起春心不自由。当时妇人见了那人生的风流浮浪,语言甜净,更加几分留恋:“倒不知此人姓甚名谁,何处居住。他若没我情意时,临去也不回头七八遍了。”却在帘子下眼巴巴的看不见那人,方才收了帘子,关上大门,归房去了。

看官听说,这人你道是谁?却原来正是那嘲风弄月的班头,拾翠寻香的元帅,开生药铺复姓西门单讳一个庆字的西门大官人便是。只因他第三房妾卓二姐死了,发送了当,心中不乐,出来街上行走,要寻应伯爵到那里去散心耍子。却从这武大门前经过,不想撞了这一下子在头上。却说这西门大官人自从帘子下见了那妇人一面,到家寻思道:“好一个雌儿,怎能勾得手?”猛然想起那间壁卖茶王婆子来,堪可如此如此,这般这般:“撮合得此事成,我破费几两银子谢他,也不值甚的。”于是连饭也不吃,走出街上闲游,一直迳踅入王婆茶坊里来,便去里边水帘下坐了。王婆笑道:“大官人却才唱得好个大肥喏!”西门庆道:“干娘,你且来,我问你,间壁这个雌儿是谁的娘子?”王婆道:“他是阎罗大王的妹子,五道将军的女儿,问他怎的?”西门庆道:“我和你说正话,休要取笑。”王婆道:“大官人怎的不认得?他老公便是县前卖熟食的。”西门庆道:“莫不是卖枣糕徐三的老婆?”王婆摇手道:“不是,若是他,也是一对儿。大官人再猜。”西门庆道:“敢是卖馉饳的李三娘子儿?”王婆摇手道:“不是,若是他,倒是一双。”西门庆道:“莫不是花胳膊刘小二的婆儿?”王婆大笑道:“不是,若是他时,又是一对儿。大官人再猜。”西门庆道:“干娘,我其实猜不着了。”王婆哈哈笑道:“我好交大官人得知了罢,他的盖老便是街上卖炊饼的武大郎。”西门庆听,跌脚笑道:“莫不是人叫他三寸丁谷树皮的武大么?”王婆道:“正是他。”西门庆听了,叫起苦来,说是:“好一块羊肉,怎生落在狗口里!”王婆道:“便是这般故事,自古骏马却驮痴汉走,美妻常伴拙夫眠。月下老偏这等配合。”西门庆道:“干娘,我少你多少茶果钱?”王婆道:“不多,由他,歇些时却算不妨。”西门庆又道:“你儿子王潮跟谁出去了?”王婆道:“说不的,跟了一个淮上客人,至今不归,又不知死活。”西门庆道:“却不交他跟我,那孩子倒乖觉伶俐。”王婆道:“若得大官人抬举他时,十分之好。”西门庆道:“待他归来,却再计较。”说毕,作谢起身去了。

约莫未及两个时辰,又踅将来王婆门首,帘边坐的,朝着武大门前半歇。王婆出来道:“大官人,吃个梅汤?”西门庆道:“最好多加些酸味儿。”王婆做了个梅汤,双手递与西门庆吃了。将盏子放下,西门庆道:“干娘,你这梅汤做得好,有多少在屋里?”王婆笑道:“老身做了一世媒,那讨不在屋里!”西门庆笑道:“我问你这梅汤,你却说做媒,差了多少!”王婆道:“老身只听得大官人问这媒做得好。”西门庆道:“干娘,你既是撮合山,也与我做头媒,说头好亲事,我自重重谢你。”王婆道:“看这大官人作戏!你宅上大娘子得知,老婆子这脸上怎吃得那耳刮子!”西门庆道:“我家大娘子最好性格。见今也有几个身边人在家,只是没一个中得我意的。你有这般好的,与我主张一个,便来说也不妨。若是回头人儿也好,只是要中得我意。”王婆道:“前日有一个倒好,只怕大官人不要。”西门庆道:“若是好时,与我说成了,我自重谢你。”王婆道:“生的十二分人才,只是年纪大些。”西门庆道:“自古半老佳人可共,便差一两岁也不打紧。真个多少年纪?”王婆道:“那娘子是丁亥生,属猪的,交新年却九十三岁了。”西门庆笑道:“你看这风婆子,只是扯着风脸取笑。”说毕,西门庆笑着起身去。

看看天色晚了,王婆恰才点上灯来,正要关门,只见西门庆又踅将来,迳去帘子底下凳子上坐下,朝着武大门前只顾将眼睃望。王婆道:“大官人吃个和合汤?”西门庆道:“最好!干娘放甜些。”王婆连忙取一钟来与西门庆吃了。坐到晚夕,起身道:“干娘,记了帐目,明日一发还钱。”王婆道:“由他,伏惟安置,来日再请过论。”西门庆笑了去。到家甚是寝食不安,一片心只在妇人身上。就是他大娘子月娘,见他这等失张失致的,只道为死了卓二姐的缘故,倒没做理会处。当晚无话。

次日清晨,王婆恰才开门,把眼看外时,只见西门庆又早在街前来回踅走。王婆道:“这刷子踅得紧!你看我着些甜糖抹在这厮鼻子上,交他抵不着。那厮全讨县里人便宜,且交他来老娘手里纳些贩钞,嫌他几个风流钱使。”原来这开茶坊的王婆,也不是守本分的,便是积年通殷勤,做媒婆,做卖婆,做牙婆,又会收小的,也会抱腰,又善放刁,端的看不出这婆子的本事来。但见:

\[
开言欺陆贾,出口胜隋何。只凭说六国唇枪,全仗话三齐舌剑。只鸾孤凤,霎时间交仗成双;寡妇鳏男,一席话搬说摆对。解使三里门内女,遮莫九皈殿中仙。玉皇殿上侍香金童,把臂拖来;王母宫中传言玉女,拦腰抱住。略施奸计,使阿罗汉抱住比丘尼;才用机关,交李天王搂定鬼子母。甜言说诱,男如封涉也生心;软语调合,女似麻姑须乱性。藏头露尾,撺掇淑女害相思;送暖偷寒,调弄嫦娥偷汉子。
\]


这婆子正开门,在茶局子里整理茶锅,张见西门庆踅过几遍,奔入茶局子水帘下,对着武大门首,不住把眼只望帘子里瞧。王婆只推不看见,只顾在茶局子内煽火,不出来问茶。西门庆叫道:“干娘,点两杯茶来我吃。”王婆应道:“大官人来了?连日少见,且请坐。”不多时,便浓浓点两盏稠茶,放在桌子上。西门庆道:“干娘,相陪我吃了茶。”王婆哈哈笑道:“我又不是你影射的,如何陪你吃茶?”西门庆也笑了,一会便问:“干娘,间壁卖的是甚么?”王婆道:“他家卖的拖煎阿满子,干巴子肉翻包着菜肉匾食饺,窝窝蛤蜊面,热烫温和大辣酥。”西门庆笑道:“你看这风婆子,只是风。”王婆笑道:“我不风,他家自有亲老公。”西门庆道:“我和你说正话。他家如法做得好炊饼,我要问他买四五十个拿的家去。”王婆道:“若要买炊饼,少间等他街上回来买,何消上门上户!”西门庆道:“干娘说的是。”吃了茶,坐了一回,起身去了。

良久,王婆在茶局里冷眼张着,他在门前踅过东,看一看,又转西去,又复一复,一连走了七八遍。少顷,迳入茶房里来。王婆道:“大官人侥幸,好几日不见面了。”西门庆便笑将起来,去身边摸出一两一块银子,递与王婆,说道:“干娘,权且收了做茶钱。”王婆笑道:“何消得许多!”西门庆道:“多者干娘只顾收着。”婆子暗道:“来了,这刷子当败。且把银子收了,到明日与老娘做房钱。”便道:“老身看大官人象有些心事的一般。”西门庆道:“如何干娘便猜得着?”婆子道:“有甚难猜处!自古入门休问荣枯事,观着容颜便得知。老身异样跷蹊古怪的事,不知猜勾多少。”西门庆道:“我这一件心上的事,干娘若猜得着时,便输与你五两银子。”王婆笑道:“老身也不消三智五猜,只一智便猜个中节。大官人你将耳朵来:你这两日脚步儿勤,赶趁得频,一定是记挂着间壁那个人。我这猜如何?”西门庆笑将起来道:“干娘端的智赛隋何,机强陆贾。不瞒干娘说,不知怎的,吃他那日叉帘子时见了一面,恰似收了我三魂六魄的一般,日夜只是放他不下。到家茶饭懒吃,做事没入脚处。不知你会弄手段么?”王婆哈哈笑道:“老身不瞒大官人说,我家卖茶叫做鬼打更。三年前六月初三日下大雪,那一日卖了个泡茶,直到如今不发市,只靠些杂趁养口。”西门庆道:“干娘,如何叫做杂趁?”王婆笑道:“老身自从三十六岁没了老公,丢下这个小厮,没得过日子。迎头儿跟着人说媒,次后揽人家些衣服卖,又与人家抱腰收小的,闲常也会作牵头,做马百六,也会针灸看病。”西门庆听了,笑将起来:“我并不知干娘有如此手段!端的与我说这件事,我便送十两银子与你做棺材本。你好交这雌儿会我一面。”王婆便呵呵笑道:“我自说耍,官人怎便认真起来。你也!”且看下回分解。有诗为证:

\[
西门浪子意猖狂,死下功夫戏女娘。
亏杀卖茶王老母,生交巫女会襄王。 
\]