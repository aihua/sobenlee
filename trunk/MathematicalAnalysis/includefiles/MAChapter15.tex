%# -*- coding:utf-8 -*-
%%%%%%%%%%%%%%%%%%%%%%%%%%%%%%%%%%%%%%%%%%%%%%%%%%%%%%%%%%%%%%%%%%%%%%%%%%%%%%%%%%%%%
%%  MAChapter15.tex'


\chapter{Euclid~空间与多元函数}\label{ch:15}

\section{$m$~维~Euclid~空间}
\subsection{$m$~维向量空间}
\subsection{$m$~维~Euclid~空间}
\subsection{$\mR{m}$~中点列的收敛性}
\begin{exercise}
\item 设~$\mbrace{\mvec x_{n}}$~是~$\mR{m}$~中的点列,若它有极限,证明~$\mbrace{\mvec x_n}$~有界。
\item 设~$\mvec x_{n},\mvec y_{n}\in\mR{m}$,且有~$\lim_\ntoinf\mvec x_{n}=\mvec a$~与~$\lim_\ntoinf\mvec y_{n}=\mvec b$。%
证明~$\lim_\ntoinf \mvec x_{n}\cdot\mvec y_{n}=\mvec a\cdot\mvec b$。
\item 设~$\mvec\ell\in\mR{m}$,且~$\mabs{\mvec\ell}=1$,记~$\mangle{\mvec\ell}{\mvec e_{i}}=\theta_i$~为~$\mvec\ell$~与坐标向量
~$\mvec e_{i}\mcond{i=1,2,\dotsc,m}$~的夹角。证明
\[
  \mvec\ell=\mparenb{\cos\theta_1,\cos\theta_2,\dotsc,\cos\theta_n}。
\]
\item 对任意~$\mvec x,\mvec y\in\mR{m}$,记~$\theta=\mangle{\mvec x}{\mvec y}$~为~$\mvec x$~与~$\mvec y$~的夹角。证明,
\begin{exlistcols}
  \item $\mabs{\mvec x-\mvec y}^2=\mabs{\mvec x}^2+\mabs{\mvec y}^2-2\mabs{\mvec x}\cdot\mabs{\mvec y}\cos\theta$;
  \label{exer-15.1.5-1}
  \item 若~$\mvec x,\mvec y$~正交,则~$\mabs{\mvec x+\mvec y}^2=\mabs{\mvec x}^2+\mabs{\mvec y}^2$;\label{exer-15.1.5-2}
  \item 说明~\ref{exer-15.1.5-1}~与~\ref{exer-15.1.5-2}~的几何意义。
\end{exlistcols}
\item 对任意~$\mvec x,\mvec y\in\mR{m}$,证明
\[
  \mabs{\mvec x+\mvec y}^2+\mabs{\mvec x-\mvec y}^2=2\mabs{\mvec x}^2+2\mabs{\mvec y}^2。
\]
\item 对任意~$\mvec x,\mvec y\in\mR{m}$,讨论
\begin{exlist}
  \item Cauchy~不等式~$\mabs{\minp{\mvec x}{\mvec y}}\leq\mabs{\mvec x}\cdot\mabs{\mvec y}$~中等号何时成立;
  \item 三角不等式~$\mabs{\mvec x+\mvec y}\leq\mabs{\mvec x}+\mabs{\mvec y}$~中等号何时成立。
\end{exlist}
\item 设~$\mvec x=\mparen{\mvec x{1,2,:,m}}\in\mR{m}$。证明,
\begin{exlist}
  \item 存在常数~$K_1,K_2>0$,对任意~$\mvec x\in\mR{m}$~有
  \[
    K_2\sum_{i=1}^m\mabs{x_i}\leq\mabs{\mvec x}\leq K_1\sum_{i=1}^m\mabs{x_i}
    \mcond*{\mabs{\mvec x}\coloneq\mparenbb{\sum_{i=1}^nx_i^2}^{\msp\frac12}};
  \]
  \item 存在常数~$M_1,M_2>0$,对任意~$\mvec x\in\mR{m}$~有
  \[
    M_2\max_{1\leq i\leq m}\mbrace{\mabs{x_i}}\leq\mabs{\mvec x}\leq M_1\max_{1\leq i\leq m}\mbrace{\mabs{x_i}}。
  \]
\end{exlist}
\end{exercise}

\section{Euclid~空间中的点集}
\subsection{点与集合的关系}
\subsection{开集与闭集}
\subsection{$\mR{m}$~中的某些特殊点集与若干术语}
\begin{exercise}
\item 设~$\Omega\subset\mR{2}$,画出下列区域~$\Omega$~的图形,并说明其是开区域还是闭区域,是有界域还是无界域。
\begin{exlistcols}
  \item $\Omega=\mathsetb{\minto xy}{y>0,~x>y,~x<1}$;
  \item $\Omega=\mathsetB{\minto xy}{\dfrac14x^2-1\leq y\leq 2-x}$;
  \item $\Omega=\mathsetb{\minto xy}{x^2+y^2\neq1}$;
  \item $\Omega=\mathsetb{\minto xy}{xy=1}$;
  \item $\Omega=\mathsetb{\minto xy}{\minto xy\neq\minto 00}$;
  \item $\Omega=\mathsetb{\minto xy}{y\in\mintco02,~x\in\mintc{2y}{2y+2}}$。
\end{exlistcols}
\item 设~$\mvec x\in\mR{m}$,说明下列点集在~$\mR{m}$~上是开区域还是闭区域,是有界域还是无界域。
\begin{exlistcols}
  \item $a<\mabs{\mvec x}<b\mcond{b>a>0}$;
  \item $\mparen{a_1\mvec x\cdot\mvec e_{1}}^2+\dotsb+\mparen{a_m\mvec x\cdot\mvec e_{m}}^2\leq1$;
  \item $x_m=5$,而~$x_1^2+x_2^2+\dotsb+x_{m-1}^2<1$。
\end{exlistcols}
\item 设~$\mvec z\in\mR{m}$~是常向量,且~$C$~为常数。证明,
\begin{exlistcols}
  \item 半空间~$\mathset{\mvec x}{\mvec x\in\mR{m},~\mvec x\cdot\mvec z<C}$~是开集;
  \item $\mathset{\mvec x}{\mvec x\in\mR{m},~\mvec x\cdot\mvec z\geq C}$~是闭集。
\end{exlistcols}
\item 求下列集合~$\Omega$~的边界~$\bound\Omega$,内部~$\interior\Omega$~和闭包~$\closure\Omega$。
\begin{exlist}
  \item $\Omega\subset\mR{2}$,~$\Omega=\mathsetb{\minto xy}{x>-1,~y\in\minto0{x+1}}$;
  \item $\Omega\subset\mR{2}$,~$\Omega=\mathsetb{\minto{r\cos\theta}{r\sin\theta}}{r\in\minto01,~\theta\in\minto0{2\pi}}$;
  \item $\Omega\subset\mR{2}$,~$\Omega=\mathsetb{\minto xy}{x~\text{或}~y~\text{是无理数}}$;
  \item $\Omega\subset\mR{m}$,
  ~$\Omega=\mathsetb{\mvec x}{\mvec x,\mvec x_{0}\in\mR{m},~0<\mabs{\mvec x-\mvec x_{0}}\leq\delta}$,~$\delta>0$;
  \item $\Omega\subset\mR{m}$,~$\Omega=\mathsetb{\mvec x}{\mvec x\in\mR{m},~\mabs x=1}$。
\end{exlist}
\item 设~$A,B\subset\mR{m}$。证明,
\begin{exlistcols}
  \item $\interior{\mparen{A\cap B}}=\interior A\cap\interior B$;
  \item $\closure{A\cup B}=\closure A\cup\closure B$。
\end{exlistcols}
\item 设~$A,B\subset\mR{m}$。证明有下述集合包含关系成立,并举出使等号不成立的例子。
\begin{exlistcols}
  \item $\interior{\mparen{A\cup B}}\supset\interior A\cap\interior B$;
  \item $\closure{A\cap B}\subset\closure A\cap\closure B$;
\end{exlistcols}
\item 设~$G_1,G_2$~是~$\mR{m}$~中两个不相交的开集。证明~$G_1\cap\closure G_2=\eset$。
\item 设~$E\subset\mR{m}$~是闭集,且~$x\nin E$。证明~$\rho\minto{\mvec x}E>0$。若~$E$~不是闭集,讨论结论是否依然成立。
\begin{exlistcols*}
\item 设~$E\subset\mR{m}$。证明~$\closure E=\mathset{\mvec x}{\rho\minto{\mvec x}E=0}$。
\item 对任意~$E\subset\mR{m}$,证明~$\bound E$~是闭集。
\end{exlistcols*}
\item 设~$E\subset\mR{m}$,且~$\mvec x,\mvec y\in\mR{m}$。证明
~$\rho\minto{\mvec x}E\leq\rho\minto{\mvec x}{\mvec y}+\rho\minto{\mvec y}E$。
\item 设~$A,B$~是~$\mR{m}$~中任意两个点集。证明,对任意~$\mvec x\in\mR{m}$~有
~$\rho\minto AB\leq\rho\minto{\mvec x}A+\rho\minto{\mvec x}B$。
\item 设~$E\subset\mR{m}$,令
\[
  A\coloneq\mathsetb{\mvec x}{\mvec x\in\mR{m},~\rho\minto{\mvec x}E<r},\quad
  B\coloneq\mathsetb{\mvec x}{\mvec x\in\mR{m},~\rho\minto{\mvec x}E\leq r}。
\]
\begin{exlistcols}
  \item 证明~$A$~是开集;
  \item 证明~$B$~是闭集。
\end{exlistcols}
\item 设~$F$~是闭集,而~$G$~是开集。证明,
\begin{exlistcols}
  \item $F\difset G$~是闭集;
  \item $G\difset F$~是开集。
\end{exlistcols}
\item 设~$E_1,E_2$~是~$\mR{m}$~中的开集,且~$E_1\cap E_2=\eset$。证明~$\bound{\mparen{E_1\cup E_2}}=\bound E_1\cup\bound E_2$。
\item 对任意~$E\subset\mR{m}$。
\begin{exlistcols}
  \item 证明~$\bound{\closure E}\subset\bound E$;
  \item 证明或反驳~$\bound{\closure E}=\bound E$。
\end{exlistcols}
\item 设~$1\leq s\leq m-1\mcond{m>1}$,且~$A\subset\mR{s}$,而~$B\subset\mR{m-s}$,则~$A\times B\subset\mR{m}$。证明,
\begin{exlistcols}
  \item 若~$A,B$~是开集,则~$A\times B$~是开集;
  \item 若~$A,B$~是闭集,则~$A\times B$~是闭集。
\end{exlistcols}
\end{exercise}

\section{$m$~维~Euclid~空间的性质}
\subsection{$\mR{m}$~空间的完备性}
\subsection{聚点原理}
\subsection{有限覆盖定理}
\begin{exercise}
\item 设~$\mR{m}$~中的点列~$\mbrace{\mvec x_{n}}$~满足~$\sum_{n=1}^\sinf\mabs{\mvec x_{n+1}-\mvec x_{n}}<\pinf$。证明
~$\mbrace{\mvec x_{n}}$~是收敛列。
\item 设~$\mR{2}$~中有一列闭长方形
\[
  A_n=\mathsetb{\minto xy}{x\in\mintc{a_n}{b_n},~y\in\mintc{c_n}{d_n}},\quad A_n\supset A_{n+1}\mcond*{n=1,2,\dotsc}。
\]
又设~$A_n$~的对角线长~$\ell_n$~收敛于~$0$。用~$\mR$~上的闭区间套定理证明,存在唯一的~$\minto{x_0}{y_0}\in\mR{2}$,使得
~$\minto{x_0}{y_0}\in A_n\mcond{n=1,2,\dotsc}$。
\item\label{exer-closeSetThm} 设~$\mbrace{F_n}$~是~$\mR{m}$~上的有界闭集列,并且~$F_n\supset F_{n+1}\mcond{n=1,2,\dotsc}$。%
又设~$F_n$~的直径~$d_n$~收敛于~$0$。证明,存在唯一的~$\mvec x_{0}\in\mR{m}$,使得~$\mvec x_{0}\in F_n\mcond{n=1,2,\dotsc}$。
\item 设~$A\subset\mR{m}$~为有界闭集,而~$\setfam A$~是~$E$~的一个开覆盖,它由一列开集~$\mbrace{A_n}$~组成。试用闭集套定理(参
看\ref{exer-closeSetThm})证明,在~$\setfam A$~中存在~$E$~的有限子覆盖,即存在~$n_0$,使得
\[
  \bigcup_{i=1}^{n_0}A_{n_i}\supset E 。
\]
\item 设~$E\subset\mR{m}$~为有界闭集,而~$d(E)$~为~$E$~的直径。证明,存在~$\mvec x_{0},\mvec y_{0}\in E$,使得
~$\rho\minto{\mvec x_{0}}{\mvec y_{0}}=d(E)$。
\item 设~$E_1,E_2$~是~$\mR{m}$~中的闭集,并且~$E_1,E_2$~中至少有一个为有界集。证明,
\begin{exlist}
  \item 存在~$\mvec x_{0}\in E_1$~与~$\mvec y_{0}\in E_2$,使得
  ~$\rho\minto{\mvec x_{0}}{\mvec y_{0}}=\rho\minto{E_1}{E_2}$;
  \item 若~$E_1\cap E_2=\eset$,则~$\rho\minto{E_1}{E_2}\neq0$。
\end{exlist}
\item 设~$A,B$~是~$\mR{m}$~中互不相交的有界闭集。证明,存在开集~$W,V\subset\mR{m}$,满足~$W\supset A$~而~$V\supset B$,并且
~$W\cap V=\eset$。
\item 设~$\Omega$~是有界开区域,而~$G$~是闭区域,并且~$G\subset\Omega$。证明,存在开区域~$V$,使得~$G\subset B$~且
~$\closure V\subset\Omega$。
\end{exercise}

\section{多元向量函数}
\subsection{映\emspace 射}
\subsection{向量函数}
\subsection{多元函数的几何表示}
\begin{exercise}
\item 试用函数定义说明函数~$u=f(x_1)\mcond{a\leq x_1\leq b}$~能否看作以~$\mvec x{1,2,:,m}$~为自变量,而以~$u$~为因变量的~$m$~
元函数。若可以,请指出它的定义域。
\item 确定并画出下列二元函数的定义域。
\begin{exlistcols}[3]
  \item $u=x+\smbsqrt y$;
  \item $u=\sqrt{1-x^2}+\smbsqrt{y^2-1}$;
  \item $u=\smbsqrt{x^2+y^2-1}$;
  \item $u=\smbsqrt{1-x^2-y^2}$;
  \item $u=\ln\mparen{x+y}$;
  \item $u=f(x,y)=C\mcond{\text{常数}}$;
  \item $u=f(x,y)=\smbsqrt y$;
  \item $u=\mbinom mn$;
  \item[]
  \item $u=\arcsin\dfrac yx$;
  \item $u=\arccos\dfrac x{x+y}$;
  \item $u=\sqrt{\dfrac{2x-x^2-y^2}{\smash[b]{x^2+y^2-x}}}$;
  \item $u=\arcsin\dfrac x{y^2}+\arcsin\mparen{1-y}$。
\end{exlistcols}
\item 确定下列三元函数的定义域,并描述定义域的图形。
\begin{exlistcols}
  \item $u=\ln\mparen{xyz}$;
  \item $u=\ln\mparenb{-1-x^2-y^2+z^2}$;
  \item $u=\dfrac1{x^2+y^2}$;
  \item $u=\arccos\dfrac z{\smbsqrt{x^2+y^2}}$。
\end{exlistcols}
\item 画出下列二元函数的等高线,其中~$a>0$。
\begin{exlistcols}[3]
\item $z=\mparen{x+y}^2$;
\item $z=\dfrac1{x^2+2y^2}$;
\item $z=\smbsqrt{xy}$;
\item $z=\mabs x+y$;
\item $z=\me^{\frac{2x}{x^2+y^2}}$;
\item $z=\ln\sqrt{\dfrac{(x-a)^2+y^2}{\smash[b]{(x+a)^2+y^2}}}$。
\end{exlistcols}
\item 描述下列三元函数~$f(x,y,z)$~的等量面(即~$f(x,y,z)=C$~的曲面,这里~$C$~为常数)。
\begin{exlistcols}[3]
  \item $u=x^2+y^2-z^2$;
  \item $u=(x+y)^2+z^2$;
  \item $u=\dfrac z{x^2+y^2+1}$。
\end{exlistcols}
\item 作下列二元函数的图象。
\begin{exlistcols}[3]
  \item $u=1-\sqrt{x^2+z^2}$;
  \item $u=x^2+4y^2$;
  \item $u=x^2-y^2$;
  \item $u=\dfrac1{2x^2+3y^2}$;
  \item $u=\me^{-\mparen{x^2+y^2}}$;
  \item $u=\smbsqrt{xy}$。
\end{exlistcols}
\item 说明下列方程表示什么曲面。任选其中一个变量为因变量,其余的为自变量,说明其否是二元函数;若是,指出其定义域。
\begin{exlistcols}
  \item $\dfrac{x^2}{a^2}+\dfrac{y^2}{b^2}+\dfrac{z^2}{c^2}-1=0$;
  \item $4x^2+4x+\dfrac{y^2}4-y-3z=0$。
\end{exlistcols}
\item 说明下列函数的图象具有什么性质。
\begin{exlistcols}[4]
  \item $z=f\mparenb{x^2+z^2}$;
  \item $z=f\mparenB{\dfrac yx}$;
  \item $z=f\mparen{y-ax}$;
  \item $z=zf\mparenB{\dfrac yx}$。
\end{exlistcols}
\end{exercise}

\section{多元函数的极限}
\subsection{多元数值函数的极限}
\subsection{向量函数的极限}
\subsection{累次极限}
\begin{exercise}
\item 设~$\mvec x=\mparen{\mvec x{1,:,m}}$,而~$\mvec a=\mparen{\mvec a{1,:,m}}$。证明~$\lim_{\mvec x\to\mvec a}\vecfunc f{x}=A$~
当且仅当对任意~$\e>0$,存在~$\delta>0$,只要~$\mvec x\neq\mvec a$,且~$\mabs{x_i-a_i}<\delta\mcond{i=1,2,\dotsc,m}$,就有
~$\mabsb{\vecfunc f{x}-A}<\e$。
\item 叙述下列定义。
\begin{exlistcols}
  \item $\lim_{\substack{x\to a\\ y\to\pinf}}f(x,y)=A$;
  \item $\lim_{\minto xy\to\minto{x_0}{y_0}}f(x,y)=\pinf$;
  \item $\lim_{\substack{x\to\pinf\\ y\to\minf}}f(x,y)=A$;
  \item $\lim_{\mvec x\to\pinf}\vecfunc f{x}=A\mcond{\mvec x\in\mR{m}}$。
\end{exlistcols}
\item 按定义证明下列极限等式。
\begin{exlistcols}
  \item $\lim_{\minto xy\to\minto 32}\mparen{3x-4y}=1$;
  \item $\lim_{\minto xy\to\minto 11}\mparen{x^2+y^2}=2$;
  \item $\lim_{\minto xy\to\minto aa}\mabsbb{\dfrac1{x-y}}=\pinf$;
  \item $\lim_{\substack{x=3\\ y\to\pinf}}\dfrac{xy-2}{y+1}=3$。
\end{exlistcols}
\item 求下列极限。
\begin{exlistcols}
  \item $\lim_{\minto xy\to\minto 00}\dfrac{\me^x+\me^y}{\cos x-\sin y}$;
  \item $\lim_{\minto xy\to\minto 00}\dfrac{x^2y^{\frac32}}{x^4+y^2}$;
  \item $\lim_{\minto xy\to\minto 00}\dfrac{\sin\mparen{xy}} x$;
  \item $\lim_{\minto xy\to\minto 00}x^2y^2\ln\mparenb{x^2+y^2}$;
  \item $\lim_{\substack{x\to\pinf\\ y\to\pinf}}\mparenb{x^2+y^2}\me^{-(x+y)}$;
  \item $\lim_{\minto xy\to\minto 00}\dfrac{\sin\mparenb{x^3+y^3}}{x^2+y^2}$;
  \item $\lim_{\substack{x\to\pinf\\ y\to\pinf}}\mparenbb{\dfrac{xy}{x^2+y^2}}^{\msp x^2}$;
  \item $\lim_{\substack{x\to\infty\\ y\to\infty}}\mparenbb{1+\dfrac1x}^{\msp\frac{x^2}{x+y}}$。
\end{exlistcols}
\item 对下列函数~$f(x,y)$,证明~$\lim_{\minto xy\to\minto 00}f(x,y)$~不存在。
\begin{exlistcols}[3]
  \item $f(x,y)=\dfrac{x^2}{x^2+y^2}$;
  \item $f(x,y)=\dfrac{x^4+3x^2y^2+2xy^3}{\mparen{x^2+y^2}^2}$;
  \item $f(x,y)=\dfrac{x^3+y^3}{x^2+y}$;
  \item $f(x,y)=\dfrac{x^4y^4}{\mparen{x^2+y^4}^3}$;
  \item $f(x,y)=\dfrac{x^2y^2}{x^3+y^3}$。
\end{exlistcols}
\item 设~$f(x,y)=\dfrac1{xy}$,而~$r=\smbsqrt{x^2+y^2}$。并且有区域
\[
  \Omega_1=\mathsetB{\minto xy}{\frac1k x\leq y\leq kx,~k>1~\text{为常数}},\quad
  \Omega_2=\mathsetb{\minto xy}{x>0,~y>0}。
\]
\begin{exlistcols}
  \item 判断~$\lim_{\substack{r\to\pinf\\ \minto xy\in\Omega_1}}f(x,y)$~是否存在;
  \item 判断~$\lim_{\substack{r\to\pinf\\ \minto xy\in\Omega_2}}f(x,y)$~是否存在。
\end{exlistcols}
\item 设~$\alpha,\beta>0$,且记~$r=\smbsqrt{x^2+y^2}$。
\begin{exlistcols}
  \item 证明~$\cramped{x^\alpha y^\beta=o\mparen{r^\ell}}\mcond{r\to0}$,其中~$0<\ell<\alpha+\beta$;
  \item 判断~$\lim_{\minto xy\to\minto 00}\dfrac{\cramped{x^\alpha y^\beta}}{\cramped{r^{\alpha+\beta}}}$~是否存在。
\end{exlistcols}
\item 设~$\mvec x=\mparen{\mvec x{1,2,:,m}}$,而~$r=\mabs{\mvec x}$。证明,对任意
~$1\leq i,j\leq m$,有
\[
  x_io\mparen{x_j}=o\mparen{r^2}\mcond*{r\to0}。
\]
\item 叙述并证明~$\lim_{\substack{x\to\pinf\\ y\to\minf}}f(x,y)$~存在的~Cauchy~收敛准则。
\item 证明~$\lim_{\substack{x\to\pinf\\ y\to\minf}}\mparenb{f(x)+g(y)}$~存在当且仅当~$\lim_{x\to\pinf}f(x)$~与
~$\lim_{y\to\minf}g(y)$~同时存在。
\item 设~$f(x)$~定义在~$\mR$~上,并且在任意有限区间上可积。证明无穷积分~$\int_\minf^\pinf f(x)\dif x$~收敛当且仅当
极限~$\lim_{\substack{A\to\pinf\\ B\to\minf}}\int_A^Bf(x)\dif x$~存在。
\end{exercise}

\section{多元函数的连续性}
\subsection{多元连续函数的定义与运算}
\subsection{连续函数的基本性质}
\begin{exercise}
\item 证明下列函数是全平面上的连续函数。
\begin{exlistcols}
  \item $f(x,y)=\begin{cbbdcases}
    \frac{\sin xy}{\smbsqrt{x^2+y^2}}, & x^2+y^2\neq0;\\
    0, & x^2+y^2=0;
  \end{cbbdcases}$
  \item $f(x,y)=\begin{cBdcases}
    y^2\ln\mparenb{x^2+y^2}, & x^2+y^2\neq0;\\
    0, & x^2+y^2=0 。
  \end{cBdcases}$
\end{exlistcols}
\item 指出下列函数的本性不连续点,并说明理由。
\begin{exlistcols}[3]
  \item $\dfrac{x^2-y^2}{x^2+y^2}$;
  \item $\dfrac x{x+y}$;
  \item $\dfrac{x+y}{x^3+y^3}$;
  \item $\dfrac1{\sin x\sin y}$;
  \item $\me^{-\frac xy}$;
  \item $\mabs x^{\frac1{\mabs y}}$。
\end{exlistcols}
\item 讨论由以下式子定义的函数~$\map{f(x,y)}{\mR{2}}{\mR{2}}$~的连续性。
\[%\LEFTRIGHT
  f(x,y)=\begin{cBBdcases}
    \mintobb{\frac x{\mparenb{x^2+y^2}^\alpha}}{~\frac y{\mparenb{x^2+y^2}^\alpha}}, & \minto xy\neq\minto 00;\\
    \minto 00, & \minto xy=\minto 00 。
  \end{cBBdcases}
\]
\item 设~$u=f(x,u)$~在~$M_0\minto{x_0}{y_0}$~连续,且~$f(x_0,y_0)>0$。证明,存在~$M_0$~的一个邻域~$U(M_0)$,使得~$f(x,y)$~
在~$U(M_0)$~上取正值。
\item 设向量函数~$\mvec f(x)=\mparenb{f_1(x),\dotsc,f_n(x)}$~在~$x_0$~点连续,且~$\mvec f(x_0)\neq0$。证明,存在~$x_0$~
的一个邻域~$U(x_0)$,使得~$\mvec f(x)$~在~$U(x_0)$~上均不为零向量。
\item 设函数~$\map f{\mR{m}}{\mR{1}}$~连续,对任意~$C\in\mR$,作集合
\[
  E\coloneq\mathsetb{\mvec x}{\vecfunc f{x}>C},\quad
  F\coloneq\mathsetb{\mvec x}{\vecfunc f{x}\geq C}。
\]
证明,集合~$E$~是~$\mR{m}$~上的开集,而~$F$~是~$\mR{m}$~上的闭集。
\item 设~$f(x,y)$~在~$\minto{x_0}{y_0}$~点连续。问作为~$x$~的一元函数~$f(x,y_0)$~在~$x_0$~处是否连续?作为~$y$~的一元函数
~$f(x_0,y)$~在~$y_0$~处是否连续?反之,若~$f(x,y_0)$~在~$x=x_0$~处连续,且~$f(x_0,y)$~在~$y=y_0$~处连续,问~$f(x,y)$~在
~$\minto{x_0}{y_0}$~处是否一定连续?请说明理由。
\item 证明,函数
\[
  f(x,y)=\begin{cbbdcases}
    \frac{x^2y}{x^4+y^3}, & x^2+y^2\neq0;\\
    0, & x^2+y^2=0
  \end{cbbdcases}
\]
在~$\minto 00$~点沿过此点的每一射线
\[
  \Biggl\lbrace\begin{aligned}
  x&=t\cos\alpha;\\
  y&=t\sin\alpha,\end{aligned}\quad 0\leq t<\pinf
\]
连续,即~$\lim_{t\to0}f\minto{t\cos\alpha}{t\sin\alpha}=f(0,0)$,但~$f(x,y)$~在~$\minto 00$~点不连续。
\item 设~$f\minto xy$~定义在~$\Omega$~上,若~$f\minto xy$~在~$x$~处连续,对~$y$~满足~Lipschitz~条件,即对~$\Omega$~上任意两点
~$\minto x{y'}$~与~$\minto x{y''}$,有
\[
  \mabsb{f\minto x{y'}-f\minto x{y''}}\leq L\mabs{y'-y''},
\]
其中~$L$~为常数。证明~$f\minto xy$~在~$\Omega$~上连续。
\item 设~$E$~是~$\mR{m}$~上的任意点集。证明~$\rho\minto{\mvec x}E$~在~$\mR{m}$~上一致连续。
\item 设~$E$~是~$\mR{m}$~上的有界闭集。证明,
\begin{exlist}
  \item 对给定的~$\mvec a\in\mR{m}$,存在~$\mvec x_{0}\in E$,使得~$\rho\minto{\mvec a}E=\rho\minto{\mvec a}{\mvec x_{0}}$;
  \item 存在~$\mvec x_{0},\mvec y_{0}\in E$,使得~$d(E)=\rho\minto{\mvec x_{0}}{\mvec y_{0}}$。
\end{exlist}
\item 设二元数值函数~$f\minto xy$~在全平面连续,且~$\lim_{r\to\pinf}f\minto xy=A$,其中~$r=\smbsqrt{x^2+y^2}$。证明,
\begin{exlistcols}
  \item $f\minto xy$~在全平面有界;
  \item $f\minto xy$~在全平面一致连续。
\end{exlistcols}
\item 证明,
\begin{exlistcols}
  \item 二元函数~$\cos\mparen{xy}$~在~$\mR{2}$~上不一致连续;
  \item 三元函数~$\sin\mparen{xyz}$~在~$\mR{3}$~上不一致连续。
\end{exlistcols}
\item 设二元矩形函数~$f\minto xy$~在矩形区域~$\mintc ab\times\mintc cd$~上连续,函数序列~$\mbrace{\phi_n(x)}$~在~$\mintc ab$~
上一致收敛且~$c\leq\phi_n(x)\leq d$。证明
\[
  F_n(x)=f\mintob x{\phi_n(x)}\mcond*{n=1,2,\dotsc}
\]
也在~$\mintc ab$~上一致收敛。
\item 设~$\vecfunc f{x}$~在~$\mR{m}$~上连续,对~$\mvec x\neq0$,有~$\vecfunc f{x}>0$。且对任意~$\mvec x\in\mR{m}$~和~$c>0$,有
~$f(c\mvec x)=c\vecfunc f{x}$。证明,存在~$a,b>0$,使得
\[
  a\mabs{\mvec x}\leq \vecfunc f{x}\leq b\mabs{\mvec x}\mcond*{\mvec x\in\mR{m}}。
\]
\item 设~$A$~是~$m$~阶方阵,且~$\det A\neq0$。证明,存在常数~$\alpha>0$,使得对任意~$\mvec x\in\mR{m}$,有
\[
  \mabs{A\mvec x}\geq\alpha\mabs{\mvec x} 。
\]
\item 设有二元数值函数~$f\minto xy$~在圆周
\[
  C\colon\mparen{x-x_0}^2+\mparen{y-y_0}^2=R^2
\]
上连续。证明,$f\minto xy$~在~$C$~上达到上确界~$M$~和下确界~$m$,并且它取~$\minto mM$~上的所有值至少两次。
\end{exercise}

\begin{exercise*}
\item 设~$f(x,y)$~在~$\minto 00$~邻域上有定义,函数
\[
  F\minto r\theta=f\minto{r\cos\theta}{r\sin\theta}\mcond*{r\geq0,\enspace\theta\in\mintc0{2\pi}}
\]
对~$r$~($\theta$~固定)连续,对~$\theta$~(关于~$r$)一致连续。证明~$f\minto xy$~在~$\minto 00$~点连续。
\item 证明,若~$f\minto xy$~分别对每一变量~$x$~和~$y$~是连续的,并且对其中的一个是单调的,则~$f\minto xy$~是二元连续函数。
\item 设有~$m$~元数值函数
\[
  g\mparen{\mvec a{1,2,:,m}}=\max_{x\in\mintc{-1}1}\mbraceb{\mabsb{a_1x^{m-1}+a_2x^{m-2}+\dotsb+a_{m-1}x+a_m}}。
\]
%%证明,
\begin{exlist}\FixExHead
  \item $g\mparen{\mvec a{1,:,m}}$~是~$\mR{m}$~上的连续函数;
  \item $g\mparen{\mvec a{1,:,m}}$~在单位球面~$a_1^2+\dotsb+a_m^2=1$~取正的最小值;
  \item 当~$a_1^2+\dotsb+a_m^2\to\pinf$~时,有~$g\mparen{\mvec a{1,:,m}}\to\pinf$;
  \item 存在~$\mintc{-1}1$~上的多项式~$T_m(x)$,使得
  \[
    \max_{x\in\mintc{-1}1}\mbraceb{\mabsb{T_m(x)}}=
    \inf_{\mvec a{1,:,m}}\mbraceB{\max_{x\in\mintc{-1}1}\mbraceb{\mabsb{a_1x^{m-1}+\dotsb+a_{m-1}x+a_m}}}。
  \]
\end{exlist}
\item 设集合~$E\subset\mR{m}$~不是闭集,函数~$\vecfunc f{x}$~在~$E$~上一致连续。证明,可以唯一地把函数~$\vecfunc f{x}$~延拓到
~$\closure E\difset E$~上,使得延拓后的函数在~$\closure E$~上一致连续。即存在唯一的函数~$\hat f(\mvec x)$,它在~$\closure E$~上
一致连续,且~$x\in E$~时~$\hat f(\mvec x)=\vecfunc f{x}$。
\item 我们称~$\vecfunc f{x}$~在~$\mR{m}$~上某区域~$\Omega$~是~Lipschitz~的,若存在一个常数~$L$,使得当~$\mvec x,\mvec y\in\Omega$~
时,有
\[
  \mabsb{\vecfunc f{x}-\vecfunc f{y}}\leq L\mabs{\mvec x-\mvec y}。
\]
我们称~$\vecfunc f{x}$~在~$\Omega$~是\emph{局部~Lipschitz}~的,若对~$\Omega$~上每一点,存在该点的一个邻域
~$U_0\mcond{U_0\subset\Omega}$,使得~$\vecfunc f{x}$~在~$U_0$~上时~Lipschitz~的。试证明,若~$\vecfunc f{x}$~在~$\Omega$~上是局部
~Lipschitz~的,而~$D$~是~$\Omega$~上的有界闭区域,则~$\vecfunc f{x}$~在~$D$~上是~Lipschitz~的。
\item 设函数~$\map f{\mR{m}}\mR$,满足以下条件,
\begin{exlist}
  \item 对任意~$\mvec x\in\mR{m}$,有~$\vecfunc f{x}\geq0$,且仅当~$\mvec x=0$~时~$\vecfunc f{x}=0$;
  \item 对任意~$\lambda\in\mR$,有~$f(\lambda\mvec x)=\mabs\lambda \vecfunc f{x}$;
  \item 对任意~$\mvec x,\mvec y\in\mR{m}$,有~$f(\mvec x+\mvec y)\leq \vecfunc f{x}+\vecfunc f{y}$。
\end{exlist}
%%证明,
\begin{exlist}\FixExHead
  \item 存在常数~$M>0$,使得对任意~$\mvec x\in\mR{m}$,有~$\vecfunc f{x}\leq M\mabs{\mvec x}$;
  \item $\vecfunc f{x}$~在~$\mR{m}$~上是~Lipschitz~的,因而一致连续;
  \item 存在常数~$m>0$,使得对任意~$\mvec x\in\mR{m}$,有~$m\mabs{\mvec x}\leq \vecfunc f{x}$。
\end{exlist}
\item 设~$\vecfunc f{x}$~在~$\mvec x_{0}$~的邻域上有界,令
\begin{align*}
  M_f\minto{\mvec x_{0}}\delta&\coloneq\sup\mathsetb{\vecfunc f{x}}{\mabs{\mvec x-\mvec x_{0}}<\delta},\\
  m_f\minto{\mvec x_{0}}\delta&\coloneq\inf\mathsetb{\vecfunc f{x}}{\mabs{\mvec x-\mvec x_{0}}<\delta}。
\end{align*}
于是存在极限
\[
  \omega_f(\mvec x_{0})=\lim_{\delta\to0+0}\mparenb{M_f\mparen{\mvec x_{0}}-m_f\mparen{\mvec x_{0}}},
\]
称它为~$\vecfunc f{x}$~在~$\mvec x_{0}$~处的振幅。证明~$\vecfunc f{x}$~在~$\mvec x_{0}$~处连续当且仅当
~$\omega_f(\mvec x_{0})=0$。
\item 设函数~$\map f{\mR{m}}{\mR{\ell}}$~对任意~$E\subset\mR{m}$,有~$f\mparen{\closure E}\subset\closure{f(E)}$。证明,若
~$\mvec x_{n}\to \mvec x_{0}\in\mR{m}$,则必有收敛子序列~$f\mparen{\mvec x_{n_k}}$,且
~$f\mparen{\mvec x_{n_k}}\to f(\mvec x_{0})$。
\item 证明函数~$\map f{\mR{m}}{\mR{\ell}}$~连续当且仅当对任意~$E\subset\mR{m}$,都有~$f\mparen{\closure E}\subset\closure{f(E)}$。
\item 证明函数~$\map f{\mR{m}}{\mR{\ell}}$~连续当且仅当对~$\mR{\ell}$~上任一开集~$E$,都有~$f^{-1}(E)$~是~$\mR{m}$~上的开集。
\item 设~$\phi(\mvec x)$~是~$\mR{m}$~上的实值函数,满足
\begin{exlist}
  \item 对任意~$\mvec x\in\mR{m}$,有~$\phi(\mvec x)\geq0$,且~$\phi(\mvec x)=0$~当且仅当~$\mvec x=0$;
  \item 对任意~$\mvec x\in\mR{m}$~和任意~$\alpha\in\mR$,有~$\phi(\alpha\mvec x)=\mabs\alpha\phi(\mvec x)$;
  \item 对任意~$\mvec x,\mvec y\in\mR{m}$,有~$\phi(\mvec x+\mvec y)\leq\phi(\mvec x)+\phi(\mvec y)$。
\end{exlist}
这时,称~$\phi(\mvec x)$~为~$\mR{m}$~中~$\mvec x$~的\emph{范数},又记为~$\mnorm x\coloneq\phi(\mvec x)$。试证明,
\begin{exlist}
  \item 对任意~$\mvec x=\mparen{\mvec x{1,:,m}}\in\mR{m}$,定义
  \[
    \mnorm{\mvec x}_1=\sum_{i=1}^m\mabs{x_i},\quad
    \mnorm{\mvec x}_2=\max_{1\leq i\leq m}\mbraceb{\mabs{x_i}},
  \]
  则~$\mnorm{\mvec x}_1$~和$\mnorm{\mvec x}_2$~均是~$\mvec x$~的范数;
  \item $\phi(\mvec x)=\mnorm{\mvec x}$~是~$\mR{m}$~上的一致连续函数;
  \item 设~$\mnorm{{}\cdot{}}$~是~$\mR{m}$~上的任意一种范数,则存在~$M_1,M_2>0$,对任意~$\mvec x\in\mR{m}$,有
  \[
    M_1\mnorm{\mvec x}_1\leq\mnorm{\mvec x}\leq M_2\mnorm{\mvec x}_2 。
  \]
\end{exlist}
\end{exercise*}




\endinput
%%
%% End of file `MAChapter15.tex'.