%# -*- coding:utf-8 -*-
%%%%%%%%%%%%%%%%%%%%%%%%%%%%%%%%%%%%%%%%%%%%%%%%%%%%%%%%%%%%%%%%%%%%%%%%%%%%%%%%%%%%%
%%  MAChapter13.tex'


\chapter{幂级数}\label{ch:13}

\section{幂级数的收敛半径与收敛区间}
\begin{exercise}
\item 确定下列幂级数的收敛半径,并讨论收敛区间端点的敛散性。
\begin{exlistcols}
  \item $\sum_{n=0}^\sinf\dfrac{x^n}{3^{\sqrt n}}$;
  \item $\sum_{n=1}^\sinf\dfrac{(2n)!!}{(2n+1)!!}x^n$;
  \item $\sum_{n=1}^\sinf\dfrac{(-1)^n}{n\sqrt[n]n}x^n$;
  \item $\sum_{n=1}^\sinf\mparenbb{1+\dfrac1n}^{\msp -n^2}x^n$;
  \item $\sum_{n=1}^\sinf\mparenbb{1+\dfrac1n}^{\msp n^2}x^{2n}$;
  \item $\sum_{n=1}^\sinf\mparenbb{\dfrac{a^n}n+\dfrac{b^n}n}x^n\mcond{a,b>0}$;
  \item $\sum_{n=1}^\sinf\mparenbb{1+2\cos\dfrac n4\pi}^{\msp n}x^n$;
  \item $\sum_{n=1}^\sinf\mparenbb{1+2\cos\dfrac n4\pi}^{\msp n}\dfrac{(x+1)^n}{\ln n}$;
  \item $\sum_{n=1}^\sinf\dfrac{x^n}{5^n+7^n}$;
  \item $\sum_{n=1}^\sinf\dfrac{x^n}{n^p}$;
  \item $\sum_{n=1}^\sinf nx^n$;
  \item $\sum_{n=1}^\sinf\dfrac{x^n}{n(n+1)}$;
  \item $\sum_{n=1}^\sinf\dfrac{4^n+(-3)^n}n(x+1)^n$;
  \item $\sum_{n=1}^\sinf\dfrac{(n!)^n}{(2n)!}x^n$;
  \item $\sum_{n=1}^\sinf a^{n^2}x^n\mcond{0<a<1}$;
  \item $\sum_{n=1}^\sinf\mparenB{1+\dfrac12+\dotsb+\dfrac1n}x^n$;
  \item $\sum_{n=1}^\sinf\dfrac{3^{-\sqrt n}}{\sqrt{n^2+1}}x^n$;
  \item $\sum_{n=1}^\sinf\dfrac{x^{n^2}}{2^n}$。
\end{exlistcols}
\item 设幂级数~$\sum_{n=0}^\sinf a_nx^n$~的收敛半径为~$R$,且在~$\minto{-R}R$~上一致收敛。证明~$\sum_{n=0}^\sinf a_nx^n$~在
~$\mintc{-R}R$~上一致收敛。
\item 设~$f(x)=\sum_{n=0}^\sinf a_nx^n$~的收敛半径为~$R>0$。求~$F(x)=\dfrac{f(x)}{1-x}$~的幂级数展开式,并证明新幂级数的收敛半径
不会比~$R$~大。
\item 给定数列~$\mbrace{a_n}$,已知~$\limsup_\ntoinf\sqrt[n]{\mabs{a_n}}=1$。令~$S_n=\sum_{k=1}^na_k$,证明
~$\limsup_\ntoinf\sqrt[n]{\mabs{S_n}}=1$。
\item 设~$\mabsbb{\sum_{k=0}^na_kx_0^k}\leq M\mcond{n=0,1,\dotsc;x_0>0}$。证明,当~$x\in\minto 0{x_0}$~时,有
\begin{exlistcols}
  \item $\sum_{n=0}^\sinf a_nx^n$~收敛;
  \item $\mabsbb{\sum_{n=0}^\sinf a_nx^n}\leq M$。
\end{exlistcols}
\item 设~$a_n\geq0$,而~$\sum_{n=0}^\sinf a_nx^n$~的收敛半径为~$1$,和函数为~$f(x)$。若~$\sum_{n=0}^\sinf a_n$~发散,%
证明~$\lim_{x\to1}f(x)=\pinf$。
\item 两个收敛半径相等的幂级数,讨论它们经过相加和相乘后得到的幂级数的收敛半径。
\end{exercise}

\section{幂级数的性质}
\begin{exercise}
\item 设~$f(x)=\sum_{n=1}^\sinf\dfrac{x^n}{n^2\ln(1+n)}$。证明,
\begin{exlistcols}
  \item $f(x)\in C\mintc{-1}1$,而~$f'(x)\in C\minto{-1}1$;
  \item $f(x)$~在~$x=-1$~点可微;
  \item $\lim_{x\to1-0}f'(x)=\pinf$;
  \item $f(x)$~在~$x=1$~点不可微。
\end{exlistcols}
\item 给定零阶~Bessel~函数\label{exer-Bessel0}
\[
  y=J_0(x)=1+\sum_{n=1}^\sinf(-1)^n\frac{x^{2n}}{(n!)^22^{2n}}。
\]
证明,它在~$\mR$~上满足方程
\[
  xy''+y'+xy=0。
\]
\item 用逐项微分或逐项积分求下列级数的和。
\begin{exlistcols}[3]
  \item $\sum_{n=1}^\sinf\dfrac{x^n}n$;
  \item $\sum_{n=1}^\sinf nx^n$;
  \item $\sum_{n=1}^\sinf n(n+1)x^n$;
  \item $\sum_{n=1}^\sinf\dfrac{(-1)^{n-1}}{n(2n-1)}x^{2n}$;
  \item $\sum_{n=1}^\sinf\dfrac{n^2+1}{n!\,2^n}x^n$;
  \item $\sum_{n=1}^\sinf\dfrac{(-1)^nn^3}{(n+1)!}x^n$;
  \item $\sum_{n=1}^\sinf\dfrac{x^{4n+1}}{4n+1}$;
  \item $\sum_{n=1}^\sinf(2^{n+1}-1)x^n$;
  \item $\sum_{n=1}^\sinf n^2x^{n-1}$;
  \item $\sum_{n=1}^\sinf\dfrac{(2n+1)^2}{n!}x^{2n+1}$;
  \item $\sum_{n=1}^\sinf\dfrac{2k-1}{2^k}$;
  \item $\sum_{n=1}^\sinf\dfrac1{n(2n+1)}$;
  \item $\sum_{n=1}^\sinf\dfrac{(-1)^n(2n^2+1)}{(2n+1)!}x^{2n+1}$。
\end{exlistcols}
\item 证明,
\begin{exlistcols}
  \item $\dfrac\pi4=\sum_{n=1}^\sinf\dfrac{(-1)^{n-1}}{2n-1}$;
  \item $\dfrac{\pi^2}{16}=\sum_{n=0}^\sinf\dfrac{(-1)^n}{n+1}\mparenB{1+\dfrac13+\dotsb+\dfrac1{2n+1}}$。
\end{exlistcols}
\item 设~$f(x)=\sum_{n=0}^\sinf a_nx^n$~的收敛半径为~$R=\pinf$。令~$f_n(x)\coloneq\sum_{k=0}^na_kx^k$。%
证明~$f\mparenb{f_n(x)}$~在~$\mintc ab$~上一致收敛于~$f\mparenb{f(x)}$。
\end{exercise}

\section{初等函数的~Taylor~级数展开}
\begin{exercise}
\item 将下列函数展开为~Maclaurin~级数。
\begin{exlistcols}[4]
  \item $\dfrac1{a-x}\mcond{a\neq0}$;
  \item $\dfrac1{(1+x)^2}$;
  \item $\dfrac1{(1+x)^3}$;
  \item $\dfrac{x^2}{x^2-3x+1}$;
  \item $\sinh x$;
  \item $\cosh x$;
  \item $\cos^2x$;
  \item $\sin^2x$;
  \item $\dfrac x{(1-x)(1-x^2)}$;
  \item $\dfrac x{\sqrt{1-3x}}$;
  \item $\ln\mparenb{1+x+x^2}$;
  \item $\arcsin x$。
\end{exlistcols}
\item 将下列函数在指定点附近展开为幂级数。
\begin{exlistcols}[3]
  \item $\dfrac1{a-x}$,~$x=b\neq a$;
  \item $\dfrac{x^2}{x^2-3x+1}$,~$x=2$;
  \item $\dfrac1{1-x-x^2}$,~$x=2$;
  \item $(1+x)\me^{-x}$,~$x=0$;
  \item $\ln\dfrac1{2+2x+x^2}$,~$x=-1$;
  \item $\ln^2(1-x)$,~$x=0$;
  \item $\arctan\dfrac{2x}{1-x^2}$,~$x=0$。
\end{exlistcols}
\item 设~$x>0$。证明,
\[
  \ln x=2\mparenBB{\frac{x-1}{x+1}+\frac13\mparenbb{\dfrac{x-1}{x+1}}^{\msp 3}
   +\frac15\mparenbb{\dfrac{x-1}{x+1}}^{\msp 5}+\dotsb} 。
\]
\item 利用基本初等函数的展式,求下列函数的幂级数,并指明收敛区间。
\begin{exlistcols}
  \item $(1+x^2)^{-\frac32}$;
  \item $\ln\mparenb{1+x+x^2+x^3}$;
  \item $\dfrac1{1-3x+2x^2}$;
  \item $x\arctan x-\ln\sqrt{1+x^2}$。
\end{exlistcols}
\item 利用逐项求导或逐项积分证明下列函数展式成立。
\begin{exlist}
  \item $\ln\mparenb{x+\sqrt{1+x^2}}=x+\sum_{n=1}^\sinf(-1)^n\dfrac{(2n-1)!!}{(2n)!!}\dfrac{x^{2n+1}}{2n+1}$,~$\mabs x\leq 1$;
  \item $\arctan\dfrac{2x}{2-x^2}=\sum_{n=0}^\sinf(-1)^{\mfloor{\frac n2}}\dfrac{x^{2n+1}}{2^n(2^n+1)}$,~$\mabs x\leq\sqrt 2$。
\end{exlist}
\item 通过求幂级数的系数证明下列展式成立。
\begin{exlist}
  \item $\dfrac{\arcsin x}{\sqrt{1-x^2}}=\sum_{n=0}^\sinf\dfrac{(2n)!!}{(2n+1)!!}x^{2n+1}$,~$x\in\mintc{-1}1$;
  \item $\dfrac{\ln\mparenb{x+\sqrt{1+x^2}}}{\sqrt{1+x^2}}=\sum_{n=0}^\sinf(-1)^n\dfrac{(2n)!!}{(2n+1)!!}x^{2n+1}$,
  ~$x\in\mintoc{-1}1$。
\end{exlist}
\item 证明下列展式成立。
\begin{exlist}
  \item $\arcsin^2x=\sum_{n=0}^\sinf\dfrac{(2n)!!}{(2n+1)!!}\dfrac{x^{2n+2}}{n+1}$,~$x\in\mintc{-1}1$;
  \item $\ln^2\mparenb{x+\sqrt{1+x^2}}=\sum_{n=0}^\sinf(-1)^n\dfrac{(2n)!!}{(2n+1)!!}\dfrac{x^{2n+2}}{n+1}$,~$x\in\mintc{-1}1$。
\end{exlist}
\item 通过幂级数相乘求下列函数的展开式。
\begin{exlistcols}
  \item $\dfrac{\ln(1+x)}{1+x}$;
  \item $\arctan^2x$。
\end{exlistcols}
\item\begin{exlist}\FixExHead
  \item 下述函数项级数在~$\mintc 0{\sfrac\pi2}$~上一致收敛于~$x$,
  \[
    \sin x+\sum_{n=1}^\sinf\frac{(2n-1)!!}{(2n)!!}\frac{\sin^{2n+1}x}{2n+1}=x;
  \]
  \item $\sum_{n=1}^\sinf\dfrac1{(2n-1)^2}=\dfrac{\pi^2}8$。
\end{exlist}
\item\begin{exlist}\FixExHead
  \item 下述函数项级数在~$\mintc 0{\sfrac\pi2}$~上一致收敛于~$x^2$,
  \[
    \sum_{n=0}^\sinf\frac{(2n)!!}{(2n+1)!!}\frac{\sin^{2n+1}x}{n+1}=x^2;
  \]
  \item $\sum_{n=1}^\sinf\dfrac1{n^2}=\dfrac{\pi^2}6$。
\end{exlist}
\item 证明,
\[
  \frac2\pi\int_0^{\frac\pi2}\cos(x\sin\theta)\dif\theta=J_0(x),
\]
其中~$J_0$~为零阶~Bessel~函数(参看\ref{exer-Bessel0})。
\item 设在~$x\in\mintc{-a}a$~上有~$\mabsb{f^{(k)}(x)}\leq M\mcond{k=0,1,\dotsc}$。证明,
\begin{exlist}
  \item $f(x)$~可以在~$\minto{-a}a$~上展成幂级数;
  \item $f(x)$~可以延拓到~$\mR$,其中~$\mR$~上无穷次可微。
\end{exlist}
\item 用待定系数法,求下列关于~$x$~的函数的幂级数展开式。
\begin{exlistcols}
  \item $\dfrac{x\cos\theta-x^2}{1-2x\cos\theta+x^2}$,~$x\in\minto{-1}1$;
  \item $\dfrac{\sin\theta}{1-2x\cos\theta+x^2}$,~$x\in\minto{-1}1$。
\end{exlistcols}
\item 求下列关于~$x$~的函数的幂级数展开式。
\begin{exlistcols}
  \item $\arctan\dfrac{x\sin\theta}{1-x\cos\theta}$,~$x\in\minto{-1}1$;
  \item $\ln\mparenb{1-2x\cos\theta+x^2}$,~$x\in\minto{-1}1$。
\end{exlistcols}
\item 利用~Abel~定理,求下列级数的和。
\begin{exlistcols}
  \item $\sum_{n=1}^\sinf\dfrac{\cos n\theta}n$,~$\theta\in\minto0{2\pi}$;
  \item $\sum_{n=1}^\sinf\dfrac{\sin n\theta}n$,~$\theta\in\minto0{2\pi}$。
\end{exlistcols}
\item 求下列定积分。
\begin{exlist}
  \item $\int_0^{2\pi}\ln\mparenb{1-2r\cos\theta+r^2}\dif\theta$,~$0<r<1$~或~$r>1$;
  \item $\int_0^{2\pi}\dfrac{1-r^2}{1-2r\cos\theta+r^2}\dif\theta$。
\end{exlist}
\end{exercise}

\pushstar
\section{Stirling~公式}
\popstar

\begin{exercise}
\item 用~Stirling~公式判别下列级数的敛散性。
\begin{exlistcols}
  \item $\sum_{n=1}^\sinf(n!)^{-\frac1n}$;
  \item $\sum_{n=1}^\sinf\dfrac{n!\,\me^n}{n^{n+p}}$。
\end{exlistcols}
\end{exercise}

\section{幂级数的应用}
\subsection{近似计算}
\subsection{定积分的计算}
\subsection{微分方程的幂级数解法}

\section{用多项式一致逼近闭区间上的连续函数}

\begin{exercise*}
\item 求幂级数
\[
  \sum_{n=1}^\sinf\mparenbb{1+2\cos\dfrac n4\pi}^{\msp n}\dfrac{x^n}n
\]
的收敛半径,并讨论端点的敛散性。
\item 其函数项级数~$\sum_{n=0}^\sinf\mparenb{x(1+x)}^{3n}$~的收敛域;若把每项展开为~$3^n+1$~项,从而使级数成为幂级数时,求此
幂级数的收敛半径。
\item 给定数列~$\mbrace{a_n}$,对应有级数~$\sum_{n=1}^\sinf\dfrac{a_n}{n^x}$。证明,
\begin{exlist}
  \item 若级数在~$x_1$~点收敛,则~$x>x_1$~时也收敛;
  \item 若级数在~$x_1$~点发散,则~$x<x_1$~时也发散。
  \item 存在~$C$~($C$~可以为无穷),使得级数当~$x>C$~时收敛,而当~$x<C$~时发散,这时~$C$~称为收敛指标;
  \item 级数在~$x\in\mintco{C+\e}\pinf\mcond{\e>0}$~上一致收敛;
  \item 级数在~$x\in\mintco{C+1+\e}\pinf\mcond{\e>0}$~上绝对一致收敛。
\end{exlist}
\item 给定数列~$\mbrace{a_n}$,对应有级数~$\sum_{n=1}^\sinf a_n\me^{-nx}$。证明,
\begin{exlist}
  \item 若级数在~$x_1$~点收敛,则~$x>x_1$~时也收敛;
  \item 若级数在~$x_1$~点发散,则~$x<x_1$~时也发散。
  \item 存在收敛指标~$C$~($C$~可以为无穷),使得级数当~$x>C$~时收敛,而当~$x<C$~时发散;
  \item 级数在~$x\in\mintco{C+\e}\pinf\mcond{\e>0}$~上绝对一致收敛。
\end{exlist}
\item 证明,
\[
  f(x)=\sum_{n=1}^\sinf(-1)^{n-1}\frac1{n^x}\begin{Bdcases}
    \geq\ln^x2, & x\in\mintco1\pinf;\\
    \leq\ln^x2, & x\in\mintoc01 。
  \end{Bdcases}
\]
\item\begin{exlist}
  \item\label{exer-13-6-1}设
  \[
    f(x)=\sum_{n=0}^\sinf\frac1{n!}\frac{a^n}{1+x^2a^{2n}}\mcond*{a>1}。
  \]
  证明~$f(x)$~在~$\mR$~上任意次可微;
  \item 设~$g(x),h(x)$~在~$\mR$~上任意次可微,且~$g(x)=h(x^2)$。证明
  \[
    \frac{g^{(2n)}(0)}{(2n)!}=\frac{h^{(n)}(0)}{n!};
  \]
  \item 证明~\ref{exer-13-6-1}~中函数~$f(x)$~的~Taylor~级数,除~$x=0$~外处处发散(即收敛半径为~$0$)。
\end{exlist}
\item 设~$a\in\minto{-1}1$,且
\[
  \frac{v_n}{v_{n-1}}=a\txts\sqrt{\dfrac{n-1}{n+1}}\dps\mcond*{n=2,3,\dotsc}。
\]
对于~$c>0$,令~$x_{n+1}=x_n+cv_n^2\mcond{n=1,2,\dotsc}$。求极限~$\lim_\ntoinf x_n$。
\item 利用~Taylor~公式证明,
\[
  \sum_{k=1}^\sinf\frac1{k(k+1)\dotsm(k+m)}=\frac1{m!\,m}。
\]
\end{exercise*}




\endinput
%%
%% End of file `MAChapter13.tex'.