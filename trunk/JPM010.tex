%# -*- coding:utf-8 -*-
%%%%%%%%%%%%%%%%%%%%%%%%%%%%%%%%%%%%%%%%%%%%%%%%%%%%%%%%%%%%%%%%%%%%%%%%%%%%%%%%%%%%%


\chapter{义士充配孟州道\KG 妻妾玩赏芙蓉亭}


词曰:

\[
八月中秋,凉飙微逗,芙蓉却是花时候。谁家姊妹斗新妆,园林散步携手。折得花枝,宝瓶随后,归来玩赏全凭酒。三杯酩酊破愁城,醒时愁绪应还又。
\]

话说武二被地方保甲拿去县里见知县,不题。且表西门庆跳下楼窗,扒伏在人家院里藏了。原来是行医的胡老人家。只见他家使的一个大胖丫头,走来毛厕里净手,蹶着大屁股,猛可见一个汉子扒伏在院墙下,往前走不迭,大叫:“有贼了!”慌的胡老人急进来。看见,认得是西门庆,便道:“大官人,且喜武二寻你不着,把那人打死了。地方拿他县中见官去了。这一去定是死罪。大官人归家去,料无事矣。”西门庆拜谢了胡老人,摇摆来家,一五一十对潘金莲说,二人拍手喜笑,以为除了患害。妇人叫西门庆上下多使些钱,务要结果了他,休要放他出来。西门庆一面差心腹家人来旺儿,馈送了知县一副金银酒器、五十两银子,上下吏典也使了许多钱,只要休轻勘了武二。

知县受了贿赂,到次日升厅。地方押着武松并酒保、唱的一班人,当厅跪下。县主翻了脸,便叫:“武松!你这厮昨日诬告平人,我已再三宽你,如何不遵法度,今又平白打死人?”武松道:“小人本与西门庆有仇,寻他厮打,不料撞遇此人。他隐匿西门庆不说,小人一时怒起,误将他打死。只望相公与小人做主,拿西门庆正法,与小人哥哥报这一段冤仇。小人情愿偿此人误伤之罪。”知县道:“这厮胡说,你岂不认得他是县中皂隶!今打杀他,定别有缘故,为何又缠到西门庆身上?不打如何肯招!”喝令左右加刑。两边内三四个皂隶,把武松拖翻,雨点般打了二十。打得武二口口声冤道:“小人也有与相公效劳用力之处,相公岂不怜悯?相公休要苦刑小人!”知县听了此言,越发恼了,道:“你这厮亲手打死了人,尚还口强,抵赖那个?”喝令:“好生与我拶起来!”当下又拶了武松一拶,敲了五十杖子,教取面长枷带了,收在监内。一干人寄监在门房里。内中县丞、佐二官也有和武二好的,念他是个义烈汉子,有心要周旋他,争奈都受了西门庆贿赂,粘住了口,做不的主张。又见武松只是声冤,延挨了几日,只得朦胧取了供招,唤当该吏典并仵作、邻里人等,押到狮子街,检验李外传身尸,填写尸单格目。委的被武松寻问他索讨分钱不均,酒醉怒起,一时斗殴,拳打脚踢,撞跌身死。左肋、面门、心坎、肾囊,俱有青赤伤痕不等。检验明白,回到县中。一日,做了文书申详,解送东平府来,详允发落。

这东平府尹,姓陈双名文昭,乃河南人氏,极是个清廉的官,听的报来,随即升厅。但见他:

\[
平生正直,秉性贤明。幼年向雪案攻书,长大在金銮对策。常怀忠孝之心,每发仁慈之政。户口登,钱粮办,黎民称颂满街衢;词颂减,盗贼休,父老赞歌喧市井。正是:名标青史播千年,声振黄堂传万古。贤良方正号青天,正直清廉民父母。
\]
这府尹陈文昭升了厅,便教押过这干犯人,就当厅先把清河县申文看了,又把各人供状招拟看过,端的上面怎生写着?文曰:

\[
东平府清河县,为人命事呈称:犯人武松,年二十八岁,系阳谷县人氏。因有膂力,本县参做都头。因公差回还,祭奠亡兄,见嫂潘氏不守孝满,擅自嫁人。是日,松在巷口缉听,不合在狮子街上王鸾酒楼上撞遇李外传。因酒醉,索讨前借钱三百文,外传不与;又不合因而斗殴,相互不服,揪打踢撞伤重,当时身死。比有唱妇牛氏、包氏见证,致被地方保甲捉获。委官前至尸所,拘集仵作、里甲人等,检验明白,取供具结,填图解缴前来,覆审无异。拟武松合依斗殴杀人,不问手足、他物、金两,律绞。酒保王鸾并牛氏、包氏,俱供明无罪。今合行申到案发落,请允施行。政和三年八月日知县李达天、县丞乐和安、主簿华荷禄、典史夏恭基、司吏钱劳。
\]

府尹看了一遍,将武松叫过面前,问道:“你如何打死这李外传?”那武松只是朝上磕头告道:“青天老爷!小的到案下,得见天日。容小的说,小的敢说。”府尹道:“你只顾说来。”武松遂将西门庆奸娶潘氏,并哥哥捉奸,踢中心窝,后来县中告状不准,前后情节细说一遍,道:“小的本为哥哥报仇,因寻西门庆厮打,不料误打死此人。委是小的负屈含冤,奈西门庆钱大,禁他不得。小人死不足惜,但只是小人哥哥武大含冤地下,枉了性命。”府尹道:“你不消多言,我已尽知了。”因把司吏钱劳叫来,痛责二十板,说道:“你那知县也不待做官,何故这等任情卖法?”于是将一干人众,一一审录过,用笔将武松供招都改了,因向佐二官说道:“此人为兄报仇,误打死这李外传,也是个有义的烈汉,比故杀平人不同。”一面打开他长枷,换了一面轻罪枷枷了,下在牢里。一干人等都发回本县听候。一面行文书着落清河县,添提豪恶西门庆,并嫂潘氏、王婆、小厮郓哥、仵作何九,一同从公根勘明白,奏请施行。武松在东平府监中,人都知道他是条好汉,因此押牢禁子都不要他一文钱,到把酒食与他吃。

早有人把这件事报到清河县。西门庆知道了,慌了手脚。陈文昭是个清廉官,不敢来打点他。只得走去央求亲家陈宅心腹,并使家人来旺星夜往东京下书与杨提督。提督转央内阁蔡太师。太师又恐怕伤了李知县名节,连忙赍了一封密书,特来东平府下与陈文昭,免提西门庆、潘氏。这陈文昭原系大理寺寺正,升东平府府尹,又系蔡太师门生,又见杨提督乃是朝廷面前说得话的官,以此人情两尽,只把武松免死,问了个脊杖四十,刺配二千里充军。况武大已死,尸伤无存,事涉疑似,勿论。其余一干人犯释放宁家。申详过省院,文书到日,即便施行。陈文昭从牢中取出武松来,当堂读了朝廷明降,开了长枷,免不得脊杖四十,取一具七斤半铁叶团头枷钉了,脸上刺了两行金字,迭配孟州牢城。其余发落已完,当堂府尹押行公文,差两个防送公人,领了武松解赴孟州交割。

当日武松与两个公人出离东平府,来到本县家中,将家活多变卖了,打发那两个公人路上盘费,央托左邻姚二郎看管迎儿:“倘遇朝廷恩典,赦放还家,恩有重报,不敢有忘。”街坊邻舍,上户人家,见武二是个有义的汉子,不幸遭此,都资助他银两,也有送酒食钱米的。武二到下处,问土兵要出行李包裹来,即日离了清河县上路,迤逦往孟州大道而行。有诗为证:

\[
府尹推详秉至公,武松垂死又疏通。
今朝刺配牢城去,病草萋萋遇暖风。
\]

这里武二往孟州充配去了,不题。且说西门庆打听他上路去了,一块石头方落地,心中如去了痞一般,十分自在。于是家中分付家人来旺、来保、来兴儿,收拾打扫后花园芙蓉亭干净,铺设围屏,挂起锦障,安排酒席齐整,叫了一起乐人,吹弹歌舞。请大娘子吴月娘、第二李娇儿、第三孟玉楼、第四孙雪娥、第五潘金莲,合家欢喜饮酒。家人媳妇、丫鬟使女两边侍奉。但见:

\[
香焚宝鼎,花插金瓶。器列象州之古玩,帘开合浦之明珠。水晶盘内,高堆火枣交梨;碧玉杯中,满泛琼浆玉液。烹龙肝,炮凤腑,果然下箸了万钱;黑熊掌,紫驼蹄,酒后献来香满座。碾破凤团,白玉瓯中分白浪;斟来琼液,紫金壶内喷清香。毕竟压赛孟尝君,只此敢欺石崇富。
\]

当下西门庆与吴月娘居上,其余多两傍列坐,传杯弄盏,花簇锦攒。饮酒间,只见小厮玳安领下一个小厮、一个小女儿,才头发齐眉,生得乖觉,拿着两个盒儿,说道:“隔壁花家,送花儿来与娘们戴。”走到西门庆、月娘众人跟前,都磕了头,立在傍边,说:“俺娘使我送这盒儿点心并花儿与西门大娘戴。”揭开盒儿看,一盒是朝廷上用的果馅椒盐金饼,一盒是新摘下来鲜玉簪花。月娘满心欢喜,说道:“又叫你娘费心。”一面看菜儿,打发两个吃了点心。月娘与了那小丫头一方汗巾儿,与了小厮一百文钱,说道:“多上覆你娘,多谢了。”因问小丫头儿:“你叫什么名字?”他回言道:“我叫绣春。小厮便是天福儿。”打发去了。月娘便向西门庆道:“咱这花家娘子儿,倒且是好,常时使小厮丫头送东西与我们。我并不曾回些礼儿与他。”西门庆道:“花二哥娶了这娘子儿,今不上二年光景。他自说娘子好个性儿。不然房里怎生得这两个好丫头。”月娘道:“前者他家老公公死了出殡时,我在山头会他一面。生得五短身材,团面皮,细湾湾两道眉儿,且是白净,好个温克性儿。年纪还小哩,不上二十四五。”西门庆道:“你不知,他原是大名府梁中书妾,晚嫁花家子虚,带一分好钱来。”月娘道:“他送盒儿来,咱休差了礼数,到明日也送些礼物回答他。”

看官听说:原来花子虚浑家姓李,因正月十五所生,那日人家送了一对鱼瓶儿来,就小字唤做瓶姐。先与大名府梁中书为妾。梁中书乃东京蔡太师女婿,夫人性甚嫉妒,婢妾打死者多埋在后花园中。这李氏只在外边书房内住,有养娘伏侍。只因政和三年正月上元之夜,梁中书同夫人在翠云楼上,李逵杀了全家老小,梁中书与夫人各自逃生。这李氏带了一百颗西洋大珠,二两重一对鸦青宝石,与养娘走上东京投亲。那时花太监由御前班直升广南镇守,因侄男花子虚没妻室,就使媒婆说亲,娶为正室。太监到广南去,也带他到广南,住了半年有余。不幸花太监有病,告老在家,因是清河县人,在本县住了。如今花太监死了,一分钱多在子虚手里。每日同朋友在院中行走,与西门庆都是前日结拜的弟兄。终日与应伯爵、谢希大一班十数个,每月会在一处,叫些唱的,花攒锦簇顽耍。众人又见花子虚乃是内臣家勤儿,手里使钱撒漫,哄着他在院中请婊子,整三五夜不归。正是:

\[
紫陌春光好,红楼醉管弦。
人生能有几?不乐是徒然。
\]

此事表过不题。且说当日西门庆率同妻妾,合家欢乐,在芙蓉亭上饮酒,至晚方散。归来潘金莲房中,已有半酣,乘着酒兴,要和妇人云雨。妇人连忙熏香打铺,和他解衣上床。西门庆且不与他云雨,明知妇人第一好品箫,于是坐在青纱帐内,令妇人马爬在身边,双手轻笼金钏,捧定那话,往口里吞放。西门庆垂首玩其出入之妙,鸣咂良久,淫情倍增,因呼春梅进来递茶。妇人恐怕丫头看见,连忙放下帐子来。西门庆道:“怕怎么的?”因说起:“隔壁花二哥房里到有两个好丫头,今日送花来的是小丫头。还有一个也有春梅年纪,也是花二哥收用过了。但见他娘在门首站立,他跟出来,却是生得好模样儿。谁知这花二哥年纪小小的,房里恁般用人!”妇人听了,瞅了他一眼,说道:“怪行货子,我不好骂你,你心里要收这个丫头,收他便了,如何远打周折,指山说磨,拿人家来比奴。奴不是那样人,他又不是我的丫头!既然如此,明日我往后边坐一回,腾个空儿,你自在房中叫他来,收他便了。”西门庆听了,欢喜道:“我的儿,你会这般解趣,怎教我不爱你!”二人说得情投意洽,更觉美爱无加,慢慢的品箫过了,方才抱头交股而寝。正是:

\[
自有内事迎郎意,殷勤快把紫箫吹。
\]
有《西江月》为证:

\[
纱帐香飘兰麝,娥眉惯把箫吹。雪莹玉体透房帏,禁不住魂飞魄碎。玉腕款笼金钏,两情如醉如痴。才郎情动嘱奴知,慢慢多咂一会。
\]

到次日,果然妇人往孟玉楼房中坐了。西门庆叫春梅到房中,收用了这妮子。正是:

\[
春点杏桃红绽蕊,风欺杨柳绿翻腰。
\]

潘金莲自此一力抬举他起来,不令他上锅抹灶,只叫他在房中铺床叠被,递茶水,衣服首饰拣心爱的与他,缠得两只脚小小的。原来春梅比秋菊不同,性聪慧,喜谑浪,善应对,生的有几分颜色,西门庆甚是宠他。秋菊为人浊蠢,不谙事体,妇人常常打的是他。正是:

\[
燕雀池塘语话喧,蜂柔蝶嫩总堪怜。
虽然异数同飞鸟,贵贱高低不一般。
\]
