%# -*- coding:utf-8 -*-
%%%%%%%%%%%%%%%%%%%%%%%%%%%%%%%%%%%%%%%%%%%%%%%%%%%%%%%%%%%%%%%%%%%%%%%%%%%%%%%%%%%%%
%%  MAChapter7.tex'


\chapter{定积分}\label{ch:7}
\section{定积分的概念}
\subsection{实际问题中的例}
\subsection{定积分的概念及几何意义}
\begin{exercise}
\item
\end{exercise}
\section{Newton-Leibniz~公式}
\begin{exercise}
\item
\end{exercise}
\section{可积函数}
\subsection{函数可积的充分必要条件}
\subsection{可积函数类}
\begin{exercise}
\item
\end{exercise}
\section{定积分的性质}
\subsection{定积分的基本性质}
\subsection{积分第一中值定理}
\begin{exercise}
\item
\end{exercise}
\section{变限的定积分与原函数的存在性}
\begin{exercise}
\item
\end{exercise}
\section{定积分的换元法与分部积分法}
\subsection{定积分的换元法}
\subsection{定积分的分部积分法}
\subsection{积分第二中值定理}
\begin{exercise}
\item
\end{exercise}
\section{定积分的近似计算}
\subsection{矩形法}
\subsection{梯形法}
\subsection{Simpson~公式}
\begin{exercise}
\item
\end{exercise}
\begin{exercise*}
\item
\end{exercise*}




\endinput
%%
%% End of file `MAChapter7.tex'.