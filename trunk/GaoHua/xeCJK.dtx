% \iffalse meta-comment
%<*internal>
\iffalse
%</internal>
%<*readme>

xeCJK is a package written for XeLaTeX which allows users to typeset
CJK scripts easily.


 - different default fonts for CJK and other characters;
 - spaces automatically ignored between CJK characters;
 - special effects on full-width CJK punctuation; and
 - automatic adjustment of the space between CJK and other characters.


This package is licensed in LPPL.

If you are interested in the process of development you
may observe

    http://code.google.com/p/ctex-kit/updates/list

- Sun Wenchang <sunwch@hotmail.com>

%</readme>
%<*internal>
\fi
%</internal>
%
%<*internal>
\begingroup
%</internal>
%<*batchfile>
\input docstrip.tex
\keepsilent
\preamble

 Version 2.4.5 (31-Jan-2012)

 Copyright (C) Wenchang Sun <sunwch@hotmail.com>

 This file may be distributed and/or modified under the
 conditions of the LaTeX Project Public License, either version 1.3
 of this license or (at your option) any later version.
 The latest version of this license is in
   http://www.latex-project.org/lppl.txt
 and version 1.3 or later is part of all distributions of LaTeX
 version 2005/12/01 or later.

\endpreamble
\askforoverwritefalse
\generate{\file{\jobname.sty}{\from{\jobname.dtx}{package}}}
%</batchfile>
%<batchfile>\endbatchfile
%<*internal>
\generate{\file{\jobname.ins}{\from{\jobname.dtx}{batchfile}}}
\nopreamble\nopostamble
\generate{\file{README.txt}{\from{\jobname.dtx}{readme}}}
\endgroup
%</internal>
%<*driver>
\ProvidesFile{xeCJK.dtx}
%</driver>
%
%<*driver>
\documentclass[a4paper,full]{l3doc}
\usepackage{xeCJK}
\usepackage{metalogo}
\usepackage{enumitem}
\setlist{nosep}
\setmainfont[Ligatures=TeX]{TeX Gyre Pagella}
\setmonofont{Inconsolata}
\setCJKmainfont[BoldFont=SimHei]{SimSun}
\setCJKmonofont{FangSong}
\def\MacroFont{\small\normalfont\ttfamily}
\parindent=2em
\AtBeginDocument{
  \DeleteShortVerb{\"}
  \MakeShortVerb{\|}}
\begin{document}
  \DocInput{\jobname.dtx}
\end{document}
%</driver>
% \fi
%
% \GetFileInfo{\jobname.sty}
%
% \title{\pkg{xeCJK} 宏包}
% \author{孙文昌}
% \date{\filedate\qquad\fileversion}
% \maketitle
%
% \begin{documentation}
%
% \section{简介}
%
% \pkg{xeCJK} 是一个 \XeLaTeX 宏包,用于排版 CJK 文字,包括字体选择和标点控制
% 等。主要特点:
% \begin{enumerate}
% \item 分别设置 CJK 和英文字体;
% \item 自动忽略 CJK 文字间的空格而保留其它空格,允许在非标点汉字和英文
% 字母 (a--z, A--Z) 间断行;
% \item 提供多种标点处理方式: 全角式、半角式、开明式、行末半角式和 CCT 式;
% \item 自动调整中英文间空白。
% \end{enumerate}
%
%\end{documentation}
%
%
%
%\begin{implementation}
%
%\section{\pkg{xeCJK} 代码实现}
%
%    \begin{macrocode}
%<*package>
%    \end{macrocode}
%
%    \begin{macrocode}
\RequirePackage{l3keys2e}
\ProvidesExplPackage {xeCJK} {2012/01/31} {2.4.5}
  {package for typesetting CJK scripts with XeLaTeX}
%    \end{macrocode}
%
% \pkg{xeCJK} 必须使用 \XeTeX 引擎的支持。
%    \begin{macrocode}
\xetex_if_engine:F
  {
    \PackageError { xeCJK }
      {
        ^^J The~xeCJK~package~requires~XeTeX~to~function.^^J
        You~must~change~your~typesetting~engine~to~"xelatex"^^J
       instead~of~plain~"latex"~or~"pdflatex".
      }
    \tex_endinput:D
  }
%    \end{macrocode}
%
% |\XeTeXglyphbouds| 可以得到一个字符的上下左右边距,用于标点压缩。
%    \begin{macrocode}
\token_if_primitive:NF \XeTeXglyphbounds
  {
    \PackageError { xeCJK }
      {
        ^^J \token_to_str:N \XeTeXglyphbounds\c_space_tl not~defined.^^J
        You~have~to~update~XeTeX~to~the~version~0.9995.0~or~later.
      }
  }
%    \end{macrocode}
%
% 抑制换行符产生的空格。
%    \begin{macrocode}
\tex_endlinechar:D \c_minus_one
%    \end{macrocode}
%
% \begin{macro}{xeCJKactive}
% \pkg{xeCJK} 功能开关。
%    \begin{macrocode}
\keys_define:nn { xeCJK / options }
  {
    xeCJKactive .choice:,
    xeCJKactive / true  .code:n = { \XeTeXinterchartokenstate = \c_one  } ,
    xeCJKactive / false .code:n = { \XeTeXinterchartokenstate = \c_zero } ,
    xeCJKactive      .default:n = { true },
  }
%    \end{macrocode}
% \end{macro}
%
% \subsection{字符类别设定}
%
% \pkg{xeCJK} 需要以下字符类别用于字符输出。其中 |Default|、|CJK|、|FullLeft|、
% |FullRight|、|Boundary| 为 \XeTeX\ 中预定义的类别,\pkg{xeCJK} 新增加了\
% |HalfLeft|、|HalfRight|、|NormalSpace|。
% \begin{center}
% \begin{tabular}{cll}
% \hline
%   类别        & 说明                & 例子 \\ \hline
% |Default|     & 西文一般符号       & abc123 \\
% |CJK|         & CJK 表意符号       & 汉字ぁぃぅ \\
% |FullLeft|    & 全角左标点         & (《:“ \\
% |FullRight|   & 全角右标点         & ,。)》” \\
% |HalfLeft|    & 半角左标点         & ( [ \{ \\
% |HalfRight|   & 半角右标点         & , . ? ) ] \} \\
% |NormalSpace| & 前后原始间距的符号 & / \\
% |Boundary|    & 边界              & 空格 \\
% \hline
% \end{tabular}
% \end{center}
%
% \begin{macro}[int]
%   { \xeCJK_Default_class, \xeCJK_CJK_class, \xeCJK_FullLeft_class,
%     \xeCJK_FullRight_class, \xeCJK_Boundary_class }
% 这四类是 \XeTeX\ 预定义的类别。
%    \begin{macrocode}
\int_const:Nn \xeCJK_Default_class   {   0 }
\int_const:Nn \xeCJK_CJK_class       {   1 }
\int_const:Nn \xeCJK_FullLeft_class  {   2 }
\int_const:Nn \xeCJK_FullRight_class {   3 }
\int_const:Nn \xeCJK_Boundary_class  { 255 }
%    \end{macrocode}
% \end{macro}
%
% \begin{macro}{\xeCJK_new_class:n}
% 新建一个字符类别。
%    \begin{macrocode}
\cs_set:Nn \xeCJK_new_class:n
  {
    \exp_after:wN \newXeTeXintercharclass \cs:w xeCJK_#1_class\cs_end:
  }
%    \end{macrocode}
% \end{macro}
%
% \begin{macro}[int]
%   { \xeCJK_HalfLeft_class, \xeCJK_HalfRight_class, \xeCJK_NormalSpace_class }
% 新增西文半角左/右标点和前后原始间距的符号类。
%    \begin{macrocode}
\xeCJK_new_class:n { HalfLeft }
\xeCJK_new_class:n { HalfRight }
\xeCJK_new_class:n { NormalSpace }
%    \end{macrocode}
% \end{macro}
%
% \begin{macro}{\g_xeCJK_base_class_clist, \g_xeCJK_CJK_class_clist}
%    \begin{macrocode}
\clist_set:Nn \g_xeCJK_base_class_clist
  {
    Default, CJK, FullLeft, FullRight, Boundary,
    HalfLeft, HalfRight, NormalSpace
  }
\clist_set:Nn \g_xeCJK_CJK_class_clist { CJK, FullLeft, FullRight }
%    \end{macrocode}
% \end{macro}
%
% \begin{macro}{\xeCJK_class_num:n}
% |#1| 为字符类别名称,用于取得字符类别对应的编号。
%    \begin{macrocode}
\cs_new_nopar:Nn \xeCJK_class_num:n { \use:c { xeCJK_#1_class } }
%    \end{macrocode}
% \end{macro}
%
% \begin{macro}{\xeCJKDeclareCharClass}
% \begin{macro}[aux]{\xeCJK_setcharclass_aux:nn}
% 用于设置字符所属的类别,|#1| 为类别名称,|#2| 为字符的 |Unicode|,相邻字符用
% 半角逗号隔开,支持类似 |"1100 -> "11FF| 起止范围的使用方式。
%    \begin{macrocode}
\NewDocumentCommand \xeCJKDeclareCharClass { > { \TrimSpaces } m m }
  {
    \clist_set:Nx \l_tmpa_clist {#2}
    \clist_map_inline:Nn \l_tmpa_clist
      { \xeCJK_setcharclass_aux:nn {##1} { \xeCJK_class_num:n {#1} } }
  }
\NewDocumentCommand \xeCJK_setcharclass_aux:nn
  { > { \SplitArgument { 1 } { -> } } m m }
  { \xeCJKsetcharclass #1 {#2} }
%    \end{macrocode}
% \end{macro}
% \end{macro}
%
% \begin{macro}{\xeCJKsetcharclass}
% |#1| 和 |#2| 为字符类别起止的 |Unicode|,|#3| 为类别名称对应编号。
%     \begin{macrocode}
\NewDocumentCommand \xeCJKsetcharclass { m m m }
  {
    \int_set:Nn \l_tmpa_int {#1}
    \IfNoValueTF {#2}
      { \int_set:Nn \l_tmpb_int \l_tmpa_int }
      { \int_set:Nn \l_tmpb_int {#2} }
    \int_set:Nn \l_tmpc_int {#3}
    \loop
      \XeTeXcharclass \l_tmpa_int = \l_tmpc_int
      \if_int_compare:w \l_tmpa_int < \l_tmpb_int
        \int_incr:N \l_tmpa_int
    \repeat
  }
%    \end{macrocode}
% \end{macro}
%
% \begin{macro}{\xeCJKResetPunctClass}
% 用于重置标点符号所属的字符类。
%    \begin{macrocode}
\NewDocumentCommand \xeCJKResetPunctClass { }
  {
    \xeCJKDeclareCharClass { CJK } { "B7 }
    \xeCJKDeclareCharClass { HalfLeft }
      { "28 , "2D , "5B , "60 , "7B }
    \xeCJKDeclareCharClass { HalfRight }
      {
        "21 , "22 , "25 , "27 , "29 , "2C ,
        "2E , "3A , "3B , "3F , "5D , "7D ,
      }
    \xeCJKDeclareCharClass { FullLeft }
      {
        "2018 , "201C , "3008 , "300A , "300C , "300E , "3010 , "3012 ,
        "3014 , "3016 , "3018 , "301A , "301D , "301F , "3036 , "FE59 ,
        "FE5B , "FE5D , "FE5F , "FE60 , "FE69 , "FE6B , "FF03 , "FF04 ,
        "FF08 , "FF20 , "FF3B , "FF5B , "FFE0 , "FFE1 , "FFE5 , "FFE6 ,
      }
    \xeCJKDeclareCharClass { FullRight }
      {
        "2019 , "201D , "2014 , "2026 , "2500 , "3001 , "3002 , "3005 ,
        "3006 , "3009 , "300B , "300D , "300F , "3011 , "3015 , "3017 ,
        "3019 , "301B , "301E , "3041 , "3043 , "3045 , "3047 , "3049 ,
        "3063 , "3083 , "3085 , "3087 , "308E , "309B , "309C , "309D ,
        "309E , "30A1 , "30A3 , "30A5 , "30A7 , "30A9 , "30C3 , "30E3 ,
        "30E5 , "30E7 , "30EE , "30F5 , "30F6 , "30FB , "30FC , "30FD ,
        "30FE , "FE50 , "FE51 , "FE52 , "FE54 , "FE55 , "FE56 , "FE57 ,
        "FE5A , "FE5C , "FE5E , "FE6A , "FF01 , "FF05 , "FF09 , "FF0C ,
        "FF0E , "FF1A , "FF1B , "FF1F , "FF3D , "FF5D , "FF61 , "FF63 ,
        "FF64 , "FF65 , "FF67 , "FF68 , "FF69 , "FF6A , "FF6B , "FF6C ,
        "FF6D , "FF6E , "FF6F , "FF70 , "FF9E , "FF9F ,
      }
  }
%    \end{macrocode}
% \end{macro}
%
% 设置 CJK 文字的范围。
%     \begin{macrocode}
\xeCJKDeclareCharClass { CJK }
  {
%    \end{macrocode}
% \begin{itemize}
% \item Hangul Jamo (谚文字母)
%    \begin{macrocode}
    "1100 -> "11FF ,
%    \end{macrocode}
% \item CJK Radicals Supplement (中日韩部首补充)
%    \begin{macrocode}
    "2E80 -> "2EFF ,
%    \end{macrocode}
% \item Kangxi Radicals (康熙部首)
%    \begin{macrocode}
    "2F00 -> "2FDF ,
%    \end{macrocode}
% \item Ideographic Description Characters (表意文字描述符)
%    \begin{macrocode}
    "2FF0 -> "2FFF ,
%    \end{macrocode}
% \item CJK Symbols and Punctuation (中日韩符号和标点)
%    \begin{macrocode}
    "3000 -> "303F ,
%    \end{macrocode}
% \item Hiragana (日文平假名)
%    \begin{macrocode}
    "3040 -> "309F ,
%    \end{macrocode}
% \item Katakana (日文片假名)
%    \begin{macrocode}
    "30A0 -> "30FF ,
%    \end{macrocode}
% \item Bopomofo (注音字母)
%    \begin{macrocode}
    "3100 -> "312F ,
%    \end{macrocode}
% \item Hangul Compatibility Jamo (谚文兼容字母)
%    \begin{macrocode}
    "3130 -> "318F ,
%    \end{macrocode}
% \item Kanbun (象形字注释标志)
%    \begin{macrocode}
    "3190 -> "319F ,
%    \end{macrocode}
% \item Bopomofo Extended (注音字母扩展)
%    \begin{macrocode}
    "31A0 -> "31BF ,
%    \end{macrocode}
% \item CJK Strokes (中日韩笔画)
%    \begin{macrocode}
    "31C0 -> "31EF ,
%    \end{macrocode}
% \item Katakana Phonetic Extensions (日文片假名语音扩展)
%    \begin{macrocode}
    "31F0 -> "31FF ,
%    \end{macrocode}
% \item Enclosed CJK Letters and Months (带圈中日韩字母和月份)
%    \begin{macrocode}
    "3200 -> "32FF ,
%    \end{macrocode}
% \item CJK Compatibility (中日韩兼容)
%    \begin{macrocode}
    "3300 -> "33FF ,
%    \end{macrocode}
% \item CJK Unified Ideographs Extension-A (中日韩统一表意文字扩展 A)
%    \begin{macrocode}
    "3400 -> "4DBF ,
%    \end{macrocode}
% \item Yijing Hexagrams Symbols (易经六十四卦符号)
%    \begin{macrocode}
    "4DC0 -> "4DFF,
%    \end{macrocode}
% \item CJK Unified Ideographs (中日韩统一表意文字)
%    \begin{macrocode}
    "4E00 -> "9FFF ,
%    \end{macrocode}
% \item Yi Syllables (彝文音节)
%    \begin{macrocode}
    "A000 -> "A48F ,
%    \end{macrocode}
% \item Yi Radicals (彝文字根)
%    \begin{macrocode}
    "A490 -> "A4CF ,
%    \end{macrocode}
% \item Hangul Jamo Extended-A (谚文扩展 A)
%    \begin{macrocode}
    "A960 -> "A97F ,
%    \end{macrocode}
% \item Hangul Syllables (谚文音节)
%    \begin{macrocode}
    "AC00 -> "D7AF ,
%    \end{macrocode}
% \item Kana Supplement (日文假名增补)
%    \begin{macrocode}
    "B000 -> "B0FF ,
%    \end{macrocode}
% \item Hangul Jamo Extended-B (谚文扩展 B)
%    \begin{macrocode}
    "D7B0 -> "D7FF ,
%    \end{macrocode}
% \item CJK Compatibility Ideographs (中日韩兼容表意文字)
%    \begin{macrocode}
    "F900 -> "FAFF ,
%    \end{macrocode}
% \item CJK Compatibility Forms (中日韩兼容形式)
%    \begin{macrocode}
    "FE30 -> "FE4F ,
%    \end{macrocode}
% \item Halfwidth and Fullwidth Forms (半角及全角形式)
%    \begin{macrocode}
    "FF00 -> "FFEF ,
%    \end{macrocode}
% \item CJK Unified Ideographs Extension-B (中日韩统一表意文字扩展 B)
%    \begin{macrocode}
    "20000 -> "2A6DF ,
%    \end{macrocode}
% \item CJK Unified Ideographs Extension-C (中日韩统一表意文字扩展 C)
%    \begin{macrocode}
    "2A700 -> "2B73F ,
%    \end{macrocode}
% \item CJK Unified Ideographs Extension-D (中日韩统一表意文字扩展 D)
%    \begin{macrocode}
    "2B740 -> "2B81F ,
%    \end{macrocode}
% \item CJK Compatibility Ideographs Supplement (中日韩兼容表意文字增补)
%    \begin{macrocode}
    "2F800 -> "2FA1F ,
%    \end{macrocode}
% \end{itemize}
%    \begin{macrocode}
  }
%    \end{macrocode}
%
% 重置标点符号的字符类。
%    \begin{macrocode}
\xeCJKResetPunctClass
%    \end{macrocode}
%
% \begin{macro}{\normalspacedchars}
% 声明前后不加间距的字符。
%    \begin{macrocode}
\NewDocumentCommand \normalspacedchars { m }
  {
    \tl_map_inline:nn {#1}
      { \XeTeXcharclass `##1 = \xeCJK_class_num:n { NormalSpace } }
  }
\normalspacedchars{/}
%    \end{macrocode}
% \end{macro}
%
% \begin{macro}{\xeCJK_inter_class_toks:nnn,\xeCJK_inter_class_toks:nnc}
% 在相邻类别之间插入内容。
%    \begin{macrocode}
\cs_set:Nn \xeCJK_inter_class_toks:nnn
  {
    \XeTeXinterchartoks \xeCJK_class_num:n {#1} \xeCJK_class_num:n {#2} = {#3}
  }
\cs_generate_variant:Nn \xeCJK_inter_class_toks:nnn { nnc }
%    \end{macrocode}
% \end{macro}
%
% \begin{macro}{\xeCJK_clear_inter_class_toks:nn}
% 清除相邻类别之间的内容。
%    \begin{macrocode}
\cs_set:Nn \xeCJK_clear_inter_class_toks:nn
  {
    \XeTeXinterchartoks
    \xeCJK_class_num:n {#1} \xeCJK_class_num:n {#2} = { \c_empty_tl }
  }
%    \end{macrocode}
% \end{macro}
%
% \begin{macro}{\xeCJK_pre_inter_class_toks:nnn}
% 在相邻类别之间已有的内容前增加内容。
%    \begin{macrocode}
\cs_set:Nn \xeCJK_pre_inter_class_toks:nnn
  {
    \use:x
      {
        \XeTeXinterchartoks \xeCJK_class_num:n {#1} \xeCJK_class_num:n {#2} =
          {
            \exp_not:n {#3} \tex_the:D \XeTeXinterchartoks
              \xeCJK_class_num:n {#1} \xeCJK_class_num:n {#2}
          }
      }
  }
%    \end{macrocode}
% \end{macro}
%
% \begin{macro}{\xeCJK_app_inter_class_toks:nnn,\xeCJK_app_inter_class_toks:nnc}
% 在相邻类别之间已有的内容后追加内容。
%    \begin{macrocode}
\cs_set:Nn \xeCJK_app_inter_class_toks:nnn
  {
    \use:x
      {
        \XeTeXinterchartoks \xeCJK_class_num:n {#1} \xeCJK_class_num:n {#2} =
          {
            \tex_the:D \XeTeXinterchartoks \xeCJK_class_num:n {#1}
            \xeCJK_class_num:n {#2} \exp_not:n {#3}
          }
      }
  }
\cs_generate_variant:Nn \xeCJK_app_inter_class_toks:nnn { nnc }
%    \end{macrocode}
% \end{macro}
%
% \begin{macro}{\xeCJK_get_inter_class_toks:nn}
% 取出相邻类别之间的内容。
%    \begin{macrocode}
\cs_set:Nn \xeCJK_get_inter_class_toks:nn
  {
    \tex_the:D \XeTeXinterchartoks \xeCJK_class_num:n {#1} \xeCJK_class_num:n {#2}
  }
%    \end{macrocode}
% \end{macro}
%
% \begin{macro}{\xeCJK_copy_inter_class_toks:nnnn}
% 将 |#3| 和 |#4| 之间的内容复制到 |#1| 和 |#2| 之间。
%    \begin{macrocode}
\cs_set:Nn \xeCJK_copy_inter_class_toks:nnnn
  {
    \use:x
      {
        \xeCJK_inter_class_toks:nnn {#1} {#2}
          { \xeCJK_get_inter_class_toks:nn {#3} {#4} }
      }
  }
%    \end{macrocode}
% \end{macro}
%
% \begin{macro}{\xeCJK_empty_CJK_toks:}
% 清除边界与 CJK 文字、全角左右标点之间的内容。
%    \begin{macrocode}
\cs_set_nopar:Nn \xeCJK_empty_CJK_toks:
  {
    \clist_map_inline:Nn \g_xeCJK_CJK_class_clist
      { \xeCJK_clear_inter_class_toks:nn { Boundary } {##1} }
  }
%    \end{macrocode}
% \end{macro}
%
% \subsection{字符输出规则}
%
%    \begin{macrocode}
\clist_map_inline:nn { Default, HalfLeft, HalfRight }
  {
    \clist_map_inline:nn { CJK, FullLeft, FullRight }
      {
        \xeCJK_inter_class_toks:nnn {#1} {##1}
          {
            \bgroup
            \xeCJK_get_font:
            \xeCJK_clear_inter_class_toks:nn {#1} {##1}
            \xeCJK_empty_CJK_toks:
          }
      }
    \xeCJK_app_inter_class_toks:nnn {#1} { CJK } { \CJKsymbol }
    \clist_map_inline:nn { FullLeft, FullRight }
      {
        \xeCJK_app_inter_class_toks:nnc {#1} {##1} { xeCJK_CJK_and_##1:n }
      }
    \xeCJK_inter_class_toks:nnn { CJK } {#1} { \egroup }
  }
%    \end{macrocode}
%
%    \begin{macrocode}
\xeCJK_pre_inter_class_toks:nnn { Default } { CJK } { \CJKecglue }
%    \end{macrocode}
%
%    \begin{macrocode}
\clist_map_inline:nn { Default, HalfLeft }
  {
    \xeCJK_inter_class_toks:nnn { Boundary } {#1}
      { \int_compare:nNnT \tex_lastkern:D = \c_one { \c_space_token } }
    \xeCJK_app_inter_class_toks:nnn { CJK } {#1} { \CJKecglue }
  }
%    \end{macrocode}
%
%    \begin{macrocode}
\clist_map_inline:nn { Default, HalfRight }
  {
    \xeCJK_inter_class_toks:nnn {#1} { Boundary }
      { \peek_meaning:NF \c_space_token  { { \xeCJK_zero_kern: } } }
  }
%    \end{macrocode}
%
%    \begin{macrocode}
\xeCJK_inter_class_toks:nnn { Boundary } { CJK } { \xeCJK_Boundary_and_CJK: }
%    \end{macrocode}
%
% \begin{macro}{\xeCJK_Boundary_and_CJK:}
%    \begin{macrocode}
\int_const:Nn \c_etex_math_node \c_ten
\cs_set_nopar:Nn \xeCJK_Boundary_and_CJK:
  {
    {
      \int_compare:nNnTF \tex_lastkern:D = \c_one
        { \CJKglue }
        {
          \int_compare:nNnTF \tex_lastkern:D = \c_two
            { \c_space_token }
            {
              \int_compare:nNnT \etex_lastnodetype:D = \c_etex_math_node
                { \CJKecglue }
            }
        }
    }
    \bgroup
    \xeCJK_empty_CJK_toks:
    \xeCJK_get_font:
    \CJKsymbol
  }
%    \end{macrocode}
% \end{macro}
%
%    \begin{macrocode}
\xeCJK_inter_class_toks:nnn { CJK } { Boundary }
  {
    \egroup
    { \xeCJK_CJK_kern: }
    \xeCJK_ignorespaces:
  }
\xeCJK_inter_class_toks:nnn { CJK } { CJK } { \xeCJK_CJK_and_CJK:n }
%    \end{macrocode}
%
% \begin{macro}{\xeCJK_CJK_and_CJK:n}
%    \begin{macrocode}
\cs_set_nopar:Nn \xeCJK_CJK_and_CJK:n
  {
    \bool_if:NTF \l_xeCJK_checksingle_bool
      { \xeCJK_checksingle:n {#1} }
      { \CJKglue \CJKsymbol {#1} }
  }
%    \end{macrocode}
% \end{macro}
%
%    \begin{macrocode}
\xeCJK_inter_class_toks:nnn { FullLeft } { CJK } { \nobreak \CJKsymbol }
%    \end{macrocode}
%
%    \begin{macrocode}
\xeCJK_inter_class_toks:nnn { FullRight } { CJK }
  { \xeCJK_after_FullRight: \CJKsymbol }
%    \end{macrocode}
%
%    \begin{macrocode}
\clist_map_inline:nn { Default, HalfLeft, HalfRight, Boundary }
  {
    \xeCJK_inter_class_toks:nnn { FullLeft } {#1} { \nobreak \egroup }
    \xeCJK_inter_class_toks:nnn { FullRight } {#1}
      { \xeCJK_after_FullRight: \egroup }
  }
%    \end{macrocode}
%
%    \begin{macrocode}
\clist_map_inline:nn { FullLeft, FullRight }
  {
    \xeCJK_app_inter_class_toks:nnn {#1} { Boundary } { \tex_ignorespaces:D }
  }
%    \end{macrocode}
%
%    \begin{macrocode}
\clist_map_inline:nn { CJK, FullLeft, FullRight }
  {
    \xeCJK_inter_class_toks:nnn { NormalSpace } {#1}
      {
        \bgroup
        \xeCJK_empty_CJK_toks:
        \xeCJK_get_font:
        \CJKsymbol
      }
    \xeCJK_inter_class_toks:nnn {#1} { NormalSpace } { \egroup }
  }
%    \end{macrocode}
%
%    \begin{macrocode}
\clist_map_inline:nn { FullLeft, FullRight }
  {
    \xeCJK_inter_class_toks:nnn { Boundary } {#1}
      {
        \bgroup
        \xeCJK_empty_CJK_toks:
        \xeCJK_get_font:
        \use:c { xeCJK_CJK_and_#1:n }
      }
  }
%    \end{macrocode}
%
% \begin{macro}{\xeCJK_punct_rule:n,\xeCJK_punct_rule:c}
% |#1| 为标点符号字体中的的实际左/右空白的负值,用于抹去标点符号的左/右空白。
%    \begin{macrocode}
\cs_set_nopar:Nn \xeCJK_punct_rule:n
  {
    \tex_vrule:D         \@width #1
    \@depth \c_zero_dim  \@height \c_zero_dim \scan_stop:
  }
\cs_generate_variant:Nn \xeCJK_punct_rule:n { c }
%    \end{macrocode}
% \end{macro}
%
% \begin{macro}{\xeCJK_punct_skip:n,\xeCJK_punct_skip:c}
% |#1| 为根据所选的标点处理方式在标点符号左/右增加的空白。
%    \begin{macrocode}
\cs_set_nopar:Nn \xeCJK_punct_skip:n
  { \skip_horizontal:n { #1 \@plus 0.1 em \@minus 0.1 em } }
\cs_generate_variant:Nn \xeCJK_punct_skip:n { c }
%    \end{macrocode}
% \end{macro}
%
% \begin{macro}{\xeCJK_punct_kern:n,\xeCJK_punct_kern:c}
% |#1| 为相邻两个标点之间的间距。
%    \begin{macrocode}
\cs_set_nopar:Nn \xeCJK_punct_kern:n { \tex_kern:D #1 \scan_stop: }
\cs_generate_variant:Nn \xeCJK_punct_kern:n { c }
%    \end{macrocode}
% \end{macro}
%
% \begin{macro}{\xeCJK_before_FullLeft:n}
%    \begin{macrocode}
\cs_set_nopar:Nn \xeCJK_before_FullLeft:n
  {
    \tl_set:Nx \l_xeCJK_lastpunct_tl {#1}
    \xeCJK_punct_rule:c { \l_xeCJK_current_punct_tl/rule/l/#1 }
    \CJKpunctsymbol {#1}
  }
%    \end{macrocode}
% \end{macro}
%
% \begin{macro}{\xeCJK_after_FullRight:}
%    \begin{macrocode}
\cs_set_nopar:Nn \xeCJK_after_FullRight:
  {
    \xeCJK_punct_rule:c { \l_xeCJK_current_punct_tl/rule/r/\l_xeCJK_lastpunct_tl }
    \xeCJK_punct_skip:c { \l_xeCJK_current_punct_tl/glue/r/\l_xeCJK_lastpunct_tl }
  }
%    \end{macrocode}
% \end{macro}
%
%    \begin{macrocode}
\clist_map_inline:nn { CJK, FullLeft, FullRight }
  {
    \clist_map_inline:nn { FullLeft, FullRight }
      { \xeCJK_inter_class_toks:nnc {#1} {##1} { xeCJK_#1_and_##1:n } }
  }
%    \end{macrocode}
%
% \begin{macro}{\xeCJK_CJK_and_FullLeft:n}
%    \begin{macrocode}
\int_const:Nn \c_etex_hlist_node \c_one
\cs_set_nopar:Nn \xeCJK_CJK_and_FullLeft:n
  {
    \xeCJK_get_punct_bounds:nn { l } {#1}
    \int_compare:nNnF \etex_lastnodetype:D = \c_etex_hlist_node
      { \xeCJK_punct_skip:c { \l_xeCJK_current_punct_tl/glue/l/#1 } }
    \xeCJK_before_FullLeft:n {#1}
  }
%    \end{macrocode}
% \end{macro}
%
% \begin{macro}{\xeCJK_CJK_and_FullRight:n}
%    \begin{macrocode}
\cs_set_nopar:Nn \xeCJK_CJK_and_FullRight:n
  {
    \xeCJK_get_punct_bounds:nn { r } {#1}
    \tl_if_in:NnTF \l_xeCJK_special_punct_tl {#1} { \CJKglue } { \nobreak }
    \tl_gset:Nx \l_xeCJK_lastpunct_tl {#1}
    \CJKpunctsymbol {#1}
  }
%    \end{macrocode}
% \end{macro}
%
% \begin{macro}{\xeCJK_FullLeft_and_FullLeft:n}
%    \begin{macrocode}
\cs_set_nopar:Nn \xeCJK_FullLeft_and_FullLeft:n
  {
    \nobreak
    \xeCJK_get_punct_bounds:nn { l } {#1}
    \xeCJK_get_kern:nn { \l_xeCJK_lastpunct_tl } {#1}
    \xeCJK_punct_kern:c
      { \l_xeCJK_current_punct_tl/kern/\l_xeCJK_lastpunct_tl-#1 }
    \xeCJK_before_FullLeft:n {#1}
  }
%    \end{macrocode}
% \end{macro}
%
% \begin{macro}{\xeCJK_FullLeft_and_FullRight:n}
%    \begin{macrocode}
\cs_set_nopar:Nn \xeCJK_FullLeft_and_FullRight:n
  {
    \nobreak
    \xeCJK_get_punct_bounds:nn { r } {#1}
    \xeCJK_get_kern:nn { \l_xeCJK_lastpunct_tl } {#1}
    \xeCJK_punct_kern:c
      { \l_xeCJK_current_punct_tl/kern/\l_xeCJK_lastpunct_tl-#1 }
    \nobreak
    \tl_set:Nx \l_xeCJK_lastpunct_tl {#1}
    \CJKpunctsymbol {#1}
  }
%    \end{macrocode}
% \end{macro}
%
% \begin{macro}{\xeCJK_FullRight_and_FullLeft:n}
%    \begin{macrocode}
\cs_set_nopar:Nn \xeCJK_FullRight_and_FullLeft:n
  {
    \xeCJK_punct_rule:c
      { \l_xeCJK_current_punct_tl/rule/r/\l_xeCJK_lastpunct_tl }
    \xeCJK_get_punct_bounds:nn { l } {#1}
    \xeCJK_get_kern:nn { \l_xeCJK_lastpunct_tl } {#1}
    \xeCJK_punct_kern:c
      { \l_xeCJK_current_punct_tl/kern/\l_xeCJK_lastpunct_tl-#1 }
    \xeCJK_punct_nobreak:
    \xeCJK_before_FullLeft:n {#1}
 }
%    \end{macrocode}
% \end{macro}
%
% \begin{macro}{\xeCJK_FullRight_and_FullRight:n}
%    \begin{macrocode}
\cs_set_nopar:Nn \xeCJK_FullRight_and_FullRight:n
  {
    \xeCJK_punct_rule:c
      { \l_xeCJK_current_punct_tl/rule/r/\l_xeCJK_lastpunct_tl }
    \xeCJK_get_punct_bounds:nn { r } {#1}
    \xeCJK_get_kern:nn { \l_xeCJK_lastpunct_tl } {#1}
    \xeCJK_punct_kern:c
      { \l_xeCJK_current_punct_tl/kern/\l_xeCJK_lastpunct_tl-#1 }
    \nobreak
    \tl_set:Nx \l_xeCJK_lastpunct_tl {#1}
    \CJKpunctsymbol {#1}
  }
%    \end{macrocode}
% \end{macro}
%
% \begin{macro}{\CJKglue}
%     \begin{macrocode}
\cs_new_nopar:Npn \CJKglue
  { \skip_horizontal:n { \c_zero_skip \@plus .08\baselineskip } }
%    \end{macrocode}
% \end{macro}
%
% \begin{macro}{\xeCJK_CJK_kern:, \xeCJK_zero_kern:}
%     \begin{macrocode}
\cs_set_nopar:Nn \xeCJK_CJK_kern:  { \tex_kern:D -1 sp  \tex_kern:D 1 sp }
\cs_set_nopar:Nn \xeCJK_zero_kern: { \tex_kern:D -2 sp  \tex_kern:D 2 sp }
%    \end{macrocode}
% \end{macro}
%
% \begin{macro}{CJKecglue}
% CJK 与西文和数学行内数学公式之间自动添加的空白。
%    \begin{macrocode}
\keys_define:nn { xeCJK / options }
  {
    CJKecglue .tl_set_x:N = \CJKecglue,
  }
%    \end{macrocode}
% \end{macro}
%
% \begin{macro}{CJKspace}
% 是否保留 CJK 文字间的空白,默认不保留。
%    \begin{macrocode}
\keys_define:nn { xeCJK / options }
  {
    CJKspace  .bool_set:N = \l_xeCJK_space_bool ,
    CJKspace   .default:n = { true },
  }
%    \end{macrocode}
% \end{macro}
%
% \begin{macro}{\xeCJK_ignorespaces:}
%     \begin{macrocode}
\cs_set_nopar:Nn \xeCJK_ignorespaces:
  {
    \peek_meaning:NTF \c_space_token
      {
        \bool_if:NF \l_xeCJK_space_bool
          {
            \peek_meaning_ignore_spaces:NTF $
              { \c_space_token }
              {
                \peek_charcode_ignore_spaces:NTF \scan_stop:
                  { \c_space_token }
                  {
                    \peek_catcode_ignore_spaces:NT \c_catcode_active_tl
                      { \c_space_token }
                  }
              }
          }
    }{
      \peek_meaning:NT $ { \CJKecglue }
    }
  }
%    \end{macrocode}
% \end{macro}
%
% \begin{macro}{CJKchecksingle}
% 避免单个汉字单独占一行。
%    \begin{macrocode}
\keys_define:nn { xeCJK / options }
  {
    CJKchecksingle .bool_set:N = \l_xeCJK_checksingle_bool,
    CJKchecksingle .default:n  = { true },
  }
%    \end{macrocode}
% \end{macro}
%
% \begin{macro}{\xeCJK_checksingle:n}
% \begin{macro}[aux]{\xeCJK_checksingle_aux:n}
%    \begin{macrocode}
\cs_set_nopar:Nn \xeCJK_checksingle:n
  {
    \peek_catcode:NTF \c_catcode_other_token
      {
        \tl_set:Nn \l_xeCJK_last_char_tl { \CJKsymbol {#1} }
        \xeCJK_checksingle_aux:n
      }
      { \CJKglue \CJKsymbol {#1} }
  }
\cs_set_nopar:Nn \xeCJK_checksingle_aux:n
  {
    \tl_set:Nn \l_xeCJK_current_char_tl { \CJKglue \l_xeCJK_last_char_tl #1 }
    \tl_set:Nn \l_xeCJK_current_nobreak_char_tl { \l_xeCJK_last_char_tl #1 }
    \peek_meaning:NTF \c_space_token
      {
        \peek_meaning_ignore_spaces:NTF \tex_par:D
          { \l_xeCJK_current_nobreak_char_tl }
          { \l_xeCJK_current_char_tl \c_space_token }
      }
      { \l_xeCJK_current_char_tl }
  }
%    \end{macrocode}
% \end{macro}
% \end{macro}
%
% \subsection{增加 CJK 子分区}
%
% \begin{macro}{\xeCJK_UL_subclass_patch_tl,\g_xeCJK_CJK_subclass_clist}
%    \begin{macrocode}
\tl_clear_new:N \xeCJK_UL_subclass_patch_tl
\clist_clear_new:N \g_xeCJK_CJK_subclass_clist
%    \end{macrocode}
% \end{macro}
%
% \begin{macro}{\xeCJK_erase_CJKsymbol:, \xeCJK_restore_CJKsymbol:}
%    \begin{macrocode}
\cs_set:Nn \xeCJK_erase_CJKsymbol:
  {
    \cs_gset_eq:NN \CJKsymbol_Block \CJKsymbol
    \cs_gset_eq:NN \CJKsymbol \prg_do_nothing:
  }
\cs_set:Nn \xeCJK_restore_CJKsymbol:
  {
    \cs_gset_eq:NN \CJKsymbol \CJKsymbol_Block
    \CJKsymbol
  }
%    \end{macrocode}
% \end{macro}
%
% \begin{macro}{\xeCJKDeclareSubCJKBlock}
% 声明 CJK 子区范围,|#1| 为自定义名称,|#2| 为子区的 |Unicode| 范围。
%    \begin{macrocode}
\NewDocumentCommand \xeCJKDeclareSubCJKBlock { m m }
  { \xeCJKDeclareSubCharClass { CJK } {#1} {#2} }
%    \end{macrocode}
% \end{macro}
%
% \begin{macro}{\xeCJKCancelSubCJKBlock}
% 取消对 CJK 子区的声明。
%    \begin{macrocode}
\NewDocumentCommand \xeCJKCancelSubCJKBlock { m }
  { \xeCJKCancelSubCharClass { CJK } {#1} }
%    \end{macrocode}
% \end{macro}
%
% \begin{macro}{\xeCJKDeclareSubCharClass}
%    \begin{macrocode}
\NewDocumentCommand \xeCJKDeclareSubCharClass
  { > { \TrimSpaces } m > { \TrimSpaces } m m }
  {
    \tl_set:Nx \l_tmpa_tl { #1/#2 }
    \cs_if_free:cTF { xeCJK_\l_tmpa_tl _class }
      {
        \xeCJK_new_class:n { \l_tmpa_tl }
        \clist_gset:cx { g_xeCJK_\l_tmpa_tl _range_clist } {#3}
      }
      { \clist_gput_right:cx { g_xeCJK_\l_tmpa_tl _range_clist } {#3} }
    \xeCJKDeclareCharClass { \l_tmpa_tl } {#3}
    \clist_map_inline:Nn \g_xeCJK_base_class_clist
      {
        \xeCJK_copy_inter_class_toks:nnnn { \l_tmpa_tl } {##1} {#1}  {##1}
        \xeCJK_copy_inter_class_toks:nnnn {##1} { \l_tmpa_tl } {##1} {#1}
        \str_if_eq:xxTF {##1} { CJK }
          {
            \xeCJK_pre_inter_class_toks:nnn {##1} { \l_tmpa_tl }
              { \xeCJK_push_font:nn {#1} {#2} }
          }
          {
            \xeCJK_pre_inter_class_toks:nnn {##1} { \l_tmpa_tl }
              { \xeCJK_erase_CJKsymbol: }
            \xeCJK_app_inter_class_toks:nnn {##1} { \l_tmpa_tl }
              { \xeCJK_push_font:nn {#1} {#2} \xeCJK_restore_CJKsymbol: }
          }
    }
    \xeCJK_copy_inter_class_toks:nnnn { \l_tmpa_tl } { \l_tmpa_tl } {#1} {#1}
    \clist_if_empty:NF \g_xeCJK_CJK_subclass_clist
      {
        \clist_map_inline:Nn \g_xeCJK_CJK_subclass_clist
          {
            \xeCJK_copy_inter_class_toks:nnnn { \l_tmpa_tl } { #1/##1 } {#1} {#1}
            \xeCJK_copy_inter_class_toks:nnnn { #1/##1 } { \l_tmpa_tl } {#1} {#1}
            \xeCJK_pre_inter_class_toks:nnn { \l_tmpa_tl } { #1/##1 }
              { \xeCJK_push_font:nn {#2} {##1} }
            \xeCJK_pre_inter_class_toks:nnn { #1/##1 } { \l_tmpa_tl }
              { \xeCJK_push_font:nn {##1} {#2} }
          }
      }
    \clist_gput_right:Nn \g_xeCJK_CJK_subclass_clist {#2}
    \exp_args:NNx \clist_gput_right:Nn \g_xeCJK_CJK_class_clist \l_tmpa_tl
    \clist_map_inline:nn { CJK, FullLeft, FullRight }
      {
        \xeCJK_pre_inter_class_toks:nnn { \l_tmpa_tl } {##1}
          { \xeCJK_push_font:nn {#2} {#1} }
      }
    \xeCJK_new_features_key:n {#2}
    \tl_put_right:Nn \xeCJK_UL_subclass_patch_tl
      {
        \clist_map_inline:nn { Default, HalfLeft, HalfRight }
          {
            \str_if_eq:xxTF {##1} { HalfLeft }
              {
                \xeCJK_inter_class_toks:nnn {#1} { #1/#2 }
                  { \CJKecglue \CJKsymbol }
              }
              {
                \xeCJK_inter_class_toks:nnn {#1} { #1/#2 } { \CJKsymbol }
              }
            \xeCJK_inter_class_toks:nnn { Boundary } { #1/#2 }
              {
                \xeCJK_erase_CJKsymbol:
                \xeCJK_Boundary_and_CJK:
                \xeCJK_push_font:nn {#1} {#2}
                \xeCJK_restore_CJKsymbol:
              }
          }
      }
  }
%    \end{macrocode}
% \end{macro}
%
% \begin{macro}{\xeCJKCancelSubCharClass}
%    \begin{macrocode}
\NewDocumentCommand \xeCJKCancelSubCharClass
  { > { \TrimSpaces } m > { \TrimSpaces } m }
  {
    \exp_args:Nx \clist_map_inline:nn {#2}
      {
        \xeCJKDeclareCharClass {#1} { \clist_use:c { g_xeCJK_#1/##1_range_clist } }
        \clist_gclear:c { g_xeCJK_#1/##1_range_clist }
      }
  }
%    \end{macrocode}
% \end{macro}
%
% \subsection{字体切换}
%
% \begin{macro}{\l_xeCJK_current_coor_tl}
% 记录当前的字体信息。
%    \begin{macrocode}
\tl_set:Nn \l_xeCJK_current_coor_tl
  { xeCJK/\l_xeCJK_family_tl/\f@series/\f@shape/\f@size }
%    \end{macrocode}
% \end{macro}
%
% \begin{macro}{\xeCJK_font_gset:Nn,\xeCJK_font_gset:cn}
%    \begin{macrocode}
\cs_new:Nn \xeCJK_font_gset:Nn { \tex_global:D \tex_font:D #1 = #2 \scan_stop: }
\cs_generate_variant:Nn \xeCJK_font_gset:Nn { c }
%    \end{macrocode}
% \end{macro}
%
% \begin{macro}{\xeCJK_push_font:nn,\xeCJK_get_font:}
% 缓存当前字体。
%    \begin{macrocode}
\bool_new:N \l_xeCJK_rmdefault_bool
\cs_new_nopar:Nn \xeCJK_get_font: { \xeCJK_push_font:nn aa }
\cs_new_nopar:Nn \xeCJK_push_font:nn
  {
    \str_if_eq:xxF {#1} {#2}
      {
        \tl_remove_once:Nn \l_xeCJK_family_tl { /#1 }
        \str_if_eq:xxF {#2} { CJK }
          {
            \tl_put_right:Nn \l_xeCJK_family_tl { /#2 }
            \cs_if_free:cT { xeCJK/font/\l_xeCJK_family_tl }
              { \bool_set_true:N \l_xeCJK_rmdefault_bool }
          }
      }
    \cs_if_exist_use:cF { \l_xeCJK_current_coor_tl }
      {
        \cs_if_exist_use:cF { xeCJK/font/\l_xeCJK_family_tl }
          {
            \bool_if:NT \l_xeCJK_rmdefault_bool
              {
                \bool_gset_false:N \l_xeCJK_rmdefault_bool
                \cs_if_exist_use:cF { xeCJK/font/\CJKrmdefault/#2 }
                  {
                    \tl_set:Nx \l_tmpa_tl \l_xeCJK_family_tl
                    \tl_remove_once:Nn \l_tmpa_tl { /#2 }
                    \cs_if_exist_use:c { xeCJK/font/\l_tmpa_tl }
                  }
              }
          }
        \group_begin:
          \get@external@font
          \xeCJK_font_gset:cn \l_xeCJK_current_coor_tl \external@font
        \group_end:
      }
  }
%    \end{macrocode}
% \end{macro}
%
% \subsection{标点处理}
%
% \begin{macro}{\g_xeCJK_punctstyle_clist}
% 标点处理方式。
%    \begin{macrocode}
\clist_set:Nx \g_xeCJK_punctstyle_clist
  { CCT , halfwidth , fullwidth , marginkerning , mixedwidth , plain }
\clist_map_inline:Nn \g_xeCJK_punctstyle_clist
  { \tl_const:cn { c_xeCJK_ps_#1_tl } {#1} }
%    \end{macrocode}
% \end{macro}

% \begin{macro}{\keys_choices_generate:V, .generate_choices:V}
%    \begin{macrocode}
\cs_if_free:NT \keys_choices_generate:V
  { \cs_generate_variant:Nn \keys_choices_generate:n { V } }
\cs_new_protected:cpn { \c_keys_props_root_tl .generate_choices:V } #1
  { \keys_choices_generate:V {#1} }
%    \end{macrocode}
% \end{macro}
%
% \begin{macro}
%    {CJKallowbreakbetweenpuncts, PunctStyle, KaiMingPunct, SpecialPunct}
% 相关选项声明。
%    \begin{macrocode}
\keys_define:nn { xeCJK / options }
  {
    CJKallowbreakbetweenpuncts .choice:,
    CJKallowbreakbetweenpuncts / true  .code:n =
      {
        \cs_set_nopar:Nn \xeCJK_punct_nobreak: { \skip_horizontal:N \c_zero_skip }
      },
    CJKallowbreakbetweenpuncts / false .code:n =
      { \cs_set_eq:NN \xeCJK_punct_nobreak: \nobreak },
    CJKallowbreakbetweenpuncts .default:n  = { true },
    KaiMingPunct  .tl_set_x:N = \l_xeCJK_mixedwidth_punct_tl ,
    KaiMingPunct+ .code:n =
      { \tl_put_right:Nx \l_xeCJK_mixedwidth_punct_tl {#1} },
    KaiMingPunct- .code:n =
      {
        \tl_set:Nx \l_tmpa_tl {#1}
        \tl_map_inline:Nn \l_tmpa_tl
          { \tl_remove_all:Nn \l_xeCJK_mixedwidth_punct_tl {##1} }
      },
    SpecialPunct  .tl_set_x:N = \l_xeCJK_special_punct_tl ,
    SpecialPunct+ .code:n =
      { \tl_put_right:Nx \l_xeCJK_special_punct_tl {#1} },
    SpecialPunct- .code:n =
      {
        \tl_set:Nx \l_tmpa_tl {#1}
        \tl_map_inline:Nn \l_tmpa_tl
          { \tl_remove_all:Nn \l_xeCJK_special_punct_tl {##1} }
      },
    PunctStyle .choice_code:n =
      {
        \tl_set:Nx \l_xeCJK_punctstyle_tl { \l_keys_choice_tl }
        \tl_if_eq:NNT \l_keys_choice_tl \c_xeCJK_ps_plain_tl
          { \keys_set:nn { xeCJK / options } { CJKallowbreakbetweenpuncts = true } }
      },
    PunctStyle .generate_choices:V = \g_xeCJK_punctstyle_clist,
    PunctStyle / banjiao       .meta:n = { PunctStyle = halfwidth },
    PunctStyle / quanjiao      .meta:n = { PunctStyle = fullwidth },
    PunctStyle / kaiming       .meta:n = { PunctStyle = mixedwidth },
    PunctStyle / hangmobanjiao .meta:n = { PunctStyle = marginkerning },
    PunctStyle .default:n  = { fullwidth },
    PunctStyle / unknown .code:n =
      {
        \PackageError { xeCJK } { Punctstyle~"\l_keys_value_tl"~is~not~defined. }
      },
  }
%    \end{macrocode}
% \end{macro}
%
% \begin{macro}{\dim_set:Nv,\dim_add:Nv,\dim_set_min:Nv}
%    \begin{macrocode}
\cs_if_free:NT \dim_set:Nv { \cs_generate_variant:Nn \dim_set:Nn { Nv } }
\cs_if_free:NT \dim_add:Nv { \cs_generate_variant:Nn \dim_add:Nn { Nv } }
\cs_if_free:NT \dim_set_min:Nv { \cs_generate_variant:Nn \dim_set_min:Nn { Nv } }
%    \end{macrocode}
% \end{macro}
%
% \begin{macro}{\l_xeCJK_current_punct_tl}
% 记录当前字体状态下的标点处理方式。
%    \begin{macrocode}
\tl_set:Nn \l_xeCJK_current_punct_tl
  { \l_xeCJK_current_coor_tl/\l_xeCJK_punctstyle_tl }
%    \end{macrocode}
% \end{macro}
%
% \begin{macro}{\xeCJK_get_punct_bounds:nn}
% |#1| 为 |l/r|,|#2| 为标点字符,返回标点的实际左/右空白的负值和根据标点处理方式
% 决定添加的左/右空白。全角方式直接添加标点的实际左/右空白;半角方式添加标点左右空白
% 中的最小值;开明方式空半角,只在开明标点再增加实际左/右空白的一半。
%     \begin{macrocode}
\bool_new:N \l_xeCJK_punct_dokerning_bool
\cs_set_nopar:Nn \xeCJK_get_punct_bounds:nn
  {
    \cs_if_free:cT { \l_xeCJK_current_punct_tl/rule/#1/#2 }
      {
        \cs_if_free:cT { \l_xeCJK_current_punct_tl/space/#1/#2 }
          { { \xeCJK_get_font: \xeCJK_get_punct_dimen:n {#2} } }
        \bool_set_true:N \l_xeCJK_punct_dokerning_bool
        \tl_if_eq:NNTF \l_xeCJK_punctstyle_tl \c_xeCJK_ps_plain_tl
          { \bool_set_false:N \l_xeCJK_punct_dokerning_bool }
          {
            \tl_if_in:NnT \l_xeCJK_special_punct_tl {#2}
              { \bool_set_false:N \l_xeCJK_punct_dokerning_bool }
          }
        \bool_if:NTF \l_xeCJK_punct_dokerning_bool
          {
            \dim_set:Nv \l_tmpa_dim { \l_xeCJK_current_punct_tl/space/#1/#2 }
            \tl_set:Nx \l_tmpa_tl { \str_if_eq:xxTF {#1} { l } { r } { l } }
            \dim_set:Nv \l_tmpb_dim
              { \l_xeCJK_current_punct_tl/space/\l_tmpa_tl/#2 }
            \dim_set_min:Nn \l_tmpb_dim \l_tmpa_dim
            \prg_case_tl:Nnn \l_xeCJK_punctstyle_tl
              {
                \c_xeCJK_ps_halfwidth_tl { \prg_do_nothing: }
                \c_xeCJK_ps_mixedwidth_tl
                  {
                    \tl_if_in:NnT \l_xeCJK_mixedwidth_punct_tl {#2}
                      { \dim_add:Nn \l_tmpb_dim { .5\l_tmpa_dim } }
                  }
                \c_xeCJK_ps_CCT_tl
                  {
                    \tl_if_in:NnTF \l_xeCJK_mixedwidth_punct_tl {#2}
                      { \dim_add:Nn \l_tmpb_dim { .5\l_tmpa_dim } }
                      { \dim_add:Nn \l_tmpb_dim { .2\l_tmpa_dim } }
                  }
              }
              { \dim_set_eq:NN \l_tmpb_dim \l_tmpa_dim }
            \dim_set_max:Nn \l_tmpb_dim \c_zero_dim
          }
          {
            \dim_zero:N \l_tmpa_dim
            \dim_zero:N \l_tmpb_dim
          }
          \tl_gset:cx { \l_xeCJK_current_punct_tl/rule/#1/#2 }
            { - \dim_use:N \l_tmpa_dim }
          \tl_gset:cV { \l_xeCJK_current_punct_tl/glue/#1/#2 } \l_tmpb_dim
      }
  }
%    \end{macrocode}
% \end{macro}
%
% \begin{macro}[pTF]{\xeCJK_punct_if_right:n}
%    \begin{macrocode}
\prg_new_conditional:Nnn \xeCJK_punct_if_right:n { p , T , F , TF }
  {
    \cs_if_free:cTF { \l_xeCJK_current_punct_tl/glue/r/#1 }
      { \prg_return_false: } { \prg_return_true: }
  }
%    \end{macrocode}
% \end{macro}
%
% \begin{macro}{\xeCJK_get_kern:nn}
% 标点挤压。行末半角方式,相邻标点不挤压;全角方式,相邻两个标点占 1.5 汉字宽度,
% 半角、开明和 CCT 方式相邻标点占一个汉字宽度。
%    \begin{macrocode}
\cs_set_nopar:Nn \xeCJK_get_kern:nn
  {
    \cs_if_free:cT { \l_xeCJK_current_punct_tl/kern/#1-#2 }
      {
        \dim_zero:N \l_tmpa_dim
        \prg_case_tl:Nnn \l_xeCJK_punctstyle_tl
          {
            \c_xeCJK_ps_marginkerning_tl
              {
                \xeCJK_punct_if_right:nT {#1}
                  {
                    \dim_add:Nv \l_tmpa_dim { \l_xeCJK_current_punct_tl/glue/r/#1 }
                  }
                \xeCJK_punct_if_right:nF {#2}
                  {
                    \dim_add:Nv \l_tmpa_dim { \l_xeCJK_current_punct_tl/glue/l/#2 }
                  }
              }
            \c_xeCJK_ps_fullwidth_tl
              { \xeCJK_get_kern_aux:Nnnn \l_tmpa_dim { 1.5 em } {#1} {#2} }
          }
          {
            \bool_if:nTF
              {
                  ( \xeCJK_punct_if_right_p:n {#1} &&
                    \xeCJK_punct_if_right_p:n {#2} )
                ||
                  ( ! ( \xeCJK_punct_if_right_p:n {#1} ) &&
                    ! ( \xeCJK_punct_if_right_p:n {#2} ) )
              }
              { \dim_zero:N \l_tmpa_dim }
              {
                \xeCJK_get_kern_aux:Nnnn \l_tmpa_dim { 1 em } {#1} {#2}
                \dim_compare:nNnT \l_tmpa_dim < { .1 em }
                  {
                    \xeCJK_punct_if_right:nF {#2}
                      {
                        \dim_set:Nv \l_tmpa_dim
                          { \l_xeCJK_current_punct_tl/space/l/#1 }
                        \dim_set_min:Nv \l_tmpa_dim
                          { \l_xeCJK_current_punct_tl/space/r/#1 }
                      }
                  }
                \xeCJK_punct_if_right:nT {#2}
                  {
                    \dim_set_min:Nv \l_tmpa_dim
                      { \l_xeCJK_current_punct_tl/glue/r/#2 }
                  }
              }
          }
        \dim_set_max:Nn \l_tmpa_dim \c_zero_dim
        \tl_gset:cV { \l_xeCJK_current_punct_tl/kern/#1-#2 } \l_tmpa_dim
    }
  }
%    \end{macrocode}
% \end{macro}
%
% \begin{macro}{\xeCJK_get_kern_aux:Nnnn}
% |#3| 和 |#4| 为相邻的两个标点,|#2| 为要确定的相邻两个标点总共占的宽度,|#1| 是
% 尺寸寄存器,用于记录根据所占宽度计算得出的相邻两个标点间距。
%    \begin{macrocode}
\cs_set_nopar:Nn \xeCJK_get_kern_aux:Nnnn
  {
    \dim_zero:N \l_tmpb_dim
    \xeCJK_punct_if_right:nTF {#3}
      { \dim_add:Nv \l_tmpb_dim { \l_xeCJK_current_punct_tl/space/l/#3 } }
      { \dim_add:Nv \l_tmpb_dim { \l_xeCJK_current_punct_tl/glue/l/#3 } }
    \xeCJK_punct_if_right:nTF {#4}
      { \dim_add:Nv \l_tmpb_dim { \l_xeCJK_current_punct_tl/glue/r/#4 } }
      { \dim_add:Nv \l_tmpb_dim { \l_xeCJK_current_punct_tl/space/r/#4 } }
    \dim_set:Nn #1
      {
        #2 - \l_tmpb_dim
           - \tl_use:c { \l_xeCJK_current_punct_tl/dimen/#3 }
           - \tl_use:c { \l_xeCJK_current_punct_tl/dimen/#4 }
      }
  }
%    \end{macrocode}
% \end{macro}
%
% \begin{macro}{\xeCJK_get_punct_dimen:n}
% 返回标点的左右空白和实际尺寸。对于破折号,计算两标点之间的空白,保证它中间不被断开。
%    \begin{macrocode}
\cs_set:Nn \xeCJK_get_punct_dimen:n
  {
    \xeCJK_get_glyph_bounds:nn { left } {#1}
    \tl_gset:cV { \l_xeCJK_current_punct_tl/space/l/#1 } \l_xeCJK_glyph_bounds_dim
    \dim_set_eq:NN \l_tmpa_dim \l_xeCJK_glyph_bounds_dim
    \xeCJK_get_glyph_bounds:nn { right } {#1}
    \tl_gset:cV { \l_xeCJK_current_punct_tl/space/r/#1 } \l_xeCJK_glyph_bounds_dim
    \dim_add:Nn \l_tmpa_dim \l_xeCJK_glyph_bounds_dim
    \tl_gset:cx { \l_xeCJK_current_punct_tl/dimen/#1 }
      { \dim_eval:n { 1 em - \l_tmpa_dim } }
    \tl_if_in:NnT \l_xeCJK_special_punct_tl {#1}
      {
        \str_if_eq:xxF {#1} { … }
          {
            \exp_args:NNo \clist_remove_all:Nn
              \g_xeCJK_punctstyle_clist \c_xeCJK_ps_plain_tl
            \clist_map_inline:Nn \g_xeCJK_punctstyle_clist
              {
                \tl_gset:cx { \l_xeCJK_current_coor_tl/##1/kern/#1-#1 }
                  { - \dim_use:N \l_tmpa_dim }
              }
          }
      }
  }
%    \end{macrocode}
% \end{macro}
%
% \begin{macro}{\xeCJK_get_glyph_bounds:nn}
% 用 |\XeTeXglyphbounds| 计算标点的上下左右空白。
%    \begin{macrocode}
\dim_zero_new:N \l_xeCJK_glyph_bounds_dim
\cs_set_nopar:Nn \xeCJK_get_glyph_bounds:nn
  {
    \prg_case_str:xxn {#1}
      {
        { left   } { \int_set:Nn \l_tmpa_int \c_one   }
        { top    } { \int_set:Nn \l_tmpa_int \c_two   }
        { right  } { \int_set:Nn \l_tmpa_int \c_three }
        { bottom } { \int_set:Nn \l_tmpa_int \c_four  }
      }
      { \prg_do_nothing: }
    \dim_set:Nn \l_xeCJK_glyph_bounds_dim
      { \XeTeXglyphbounds \l_tmpa_int \int_eval:n { \XeTeXcharglyph `#2 } }
  }
%    \end{macrocode}
% \end{macro}
%
% \begin{macro}{\xeCJKsetkern}
% 手动设置相邻标点的距离,仅在当前字体状态下有效。
%    \begin{macrocode}
\NewDocumentCommand \xeCJKsetkern  { m m m }
  { \tl_gset:cx { \l_xeCJK_current_punct_tl/kern/#1-#2 } {#3} }
%    \end{macrocode}
% \end{macro}
%
% \subsection{后备字体}
%
% \begin{macro}{AutoFallBack, FallBack}
% 后备字体的宏包选项声明。
%    \begin{macrocode}
\keys_define:nn { xeCJK / options }
  {
    AutoFallBack .choice:,
    AutoFallBack / true  .code:n = { \xeCJKenablefallback },
    AutoFallBack / false .code:n = { \xeCJKdisablefallback },
    AutoFallBack      .default:n = { true },
    FallBack    .meta:n = { AutoFallBack = #1 },
    fallback    .meta:n = { AutoFallBack = #1 },
    FallBack .default:n = { true },
    fallback .default:n = { true },
  }
%    \end{macrocode}
% \end{macro}
%
% \begin{macro}{\xeCJKenablefallback, \xeCJKdisablefallback}
% 后备字体的启用开关,默认关闭。
%     \begin{macrocode}
\NewDocumentCommand \xeCJKenablefallback { }
  {
    \cs_set_eq:NN \xeCJK_fallback_CJKsymbol \CJKsymbol
    \cs_set_nopar:Npn \CJKsymbol ##1
      { \xeCJK_fallback_CJKsymbol { \xeCJK_fallback_testsymbol:n {##1} } }
  }
\NewDocumentCommand \xeCJKdisablefallback { }
  { \cs_set_eq:NN \CJKsymbol \xeCJK_fallback_CJKsymbol }
\NewDocumentCommand \setCJKfallbackfamilyfont { m O{} m }
  { \xeCJK_setCJKfamilyfont:nnn { #1/FallBack } {#2} {#3} }
%    \end{macrocode}
% \end{macro}
%
% \begin{macro}{\xeCJK_fallback_testsymbol:n}
% 测试当前字体中是否存在当前字符,如存在则直接输出,否则启用后备字体。
%     \begin{macrocode}
\cs_new_nopar:Nn \xeCJK_fallback_testsymbol:n
  {
    \font_glyph_if_exist:NnTF \tex_font:D {`#1} {#1}
      {
        {
          \cs_if_exist_use:c { xeCJK/font/\l_xeCJK_family_tl/FallBack }
          \font_glyph_if_exist:NnF \tex_font:D {`#1}
            {
              \cs_if_exist_use:c
                { xeCJK/font/\l_xeCJK_family_tl/FallBack/FallBack }
            }
          #1
        }
      }
  }
%    \end{macrocode}
% \end{macro}
%
%
% \subsection{CJK 字体族声明方式}
%
%    \begin{macrocode}
\bool_new:N \g_xeCJK_AutoFakeBold_bool
\bool_new:N \l_xeCJK_AutoFakeBold_bool
\bool_new:N \g_xeCJK_AutoFakeSlant_bool
\bool_new:N \l_xeCJK_AutoFakeSlant_bool
\fp_new:N \g_xeCJK_EmboldenFactor_fp
\fp_new:N \l_xeCJK_EmboldenFactor_fp
\fp_new:N \g_xeCJK_SlantFactor_fp
\fp_new:N \l_xeCJK_SlantFactor_fp
%    \end{macrocode}
%
% \begin{macro}
%    {AutoFakeBold, AutoFakeSlant, EmboldenFactor, SlantFactor, BoldFont, SlantFont}
% 伪粗体和伪斜体的宏包选项声明。
%    \begin{macrocode}
\keys_define:nn { xeCJK / options }
  {
    AutoFakeBold .choice:,
    AutoFakeBold / true    .code:n =
      { \bool_set_true:N  \g_xeCJK_AutoFakeBold_bool },
    AutoFakeBold / false   .code:n =
      { \bool_set_false:N \g_xeCJK_AutoFakeBold_bool },
    AutoFakeBold / unknown .code:n =
      {
        \bool_set_true:N  \g_xeCJK_AutoFakeBold_bool
        \fp_set:Nn \g_xeCJK_EmboldenFactor_fp { \l_keys_value_tl }
      },
    AutoFakeBold .default:n  = { true },
    AutoFakeSlant .choice:,
    AutoFakeSlant / true     .code:n =
      { \bool_set_true:N  \g_xeCJK_AutoFakeSlant_bool },
    AutoFakeSlant / false    .code:n =
      { \bool_set_false:N \g_xeCJK_AutoFakeSlant_bool },
    AutoFakeSlant / unknown  .code:n =
      {
        \bool_set_true:N  \g_xeCJK_AutoFakeSlant_bool
        \fp_set:Nn \g_xeCJK_SlantFactor_fp { \l_keys_value_tl }
      },
    AutoFakeSlant .default:n = { true },
    EmboldenFactor .fp_set:N = \g_xeCJK_EmboldenFactor_fp,
    SlantFactor    .fp_set:N = \g_xeCJK_SlantFactor_fp,
    BoldFont  .meta:n = { AutoFakeBold  = #1 },
    boldfont  .meta:n = { AutoFakeBold  = #1 },
    SlantFont .meta:n = { AutoFakeSlant = #1 },
    slantfont .meta:n = { AutoFakeSlant = #1 },
    BoldFont  .default:n = { true },
    boldfont  .default:n = { true },
    SlantFont .default:n = { true },
    slantfont .default:n = { true },
  }
%    \end{macrocode}
% \end{macro}
%
% \begin{macro}{\keys_set_known:nxN}
%    \begin{macrocode}
\cs_if_free:NT \keys_set_known:nxN
  { \cs_generate_variant:Nn \keys_set_known:nnN { nx } }
%    \end{macrocode}
% \end{macro}
%
% \begin{macro}{\xeCJK_new_features_key:n}
% \begin{macro}[aux]{\xeCJK_get_features:n}
% \begin{macro}[aux]{\xeCJK_get_features_aux:nn}
% 用于定义 CJK 子区字体和备用字体的选项。
%     \begin{macrocode}
\cs_new_nopar:Nn \xeCJK_new_features_key:n
  {
    \keys_define:nn { xeCJK / features }
      {
        #1 .code:n =
          {
            \tl_if_empty:xTF {##1}
              { \tl_put_right:Nn \l_xeCJK_familyname_tl { /#1 } }
              {
                \xeCJK_get_features:n {##1}
                \xeCJK_setCJKfamilyfont:nnn { \l_xeCJK_familyname_tl/#1 }
                  { \l_xeCJK_get_features_clist }
                  { \l_xeCJK_get_features_tl }
              }
          },
        #1 .default:n = \c_empty_tl,
      }
  }
\cs_if_free:NT \tl_if_head_eq_charcode:oNTF
  { \cs_generate_variant:Nn \tl_if_head_eq_charcode:nNTF { o } }
\cs_if_free:NT \tl_if_head_group:oTF
  { \cs_generate_variant:Nn \tl_if_head_group:nTF { o } }
\cs_new_nopar:Nn \xeCJK_get_features:n
  {
    \tl_set:Nx \l_tmpa_tl {#1}
    \clist_clear:N \l_xeCJK_get_features_clist
    \tl_if_head_eq_charcode:oNTF \l_tmpa_tl [
      { \exp_after:wN \xeCJK_get_features_aux:nn \l_tmpa_tl \c_empty_tl }
      {
        \tl_if_head_group:oTF \l_tmpa_tl
        {
          \exp_after:wN \tl_set:Nx \exp_after:wN
          \l_xeCJK_get_features_tl \l_tmpa_tl
        }
        { \tl_set:Nx \l_xeCJK_get_features_tl \l_tmpa_tl }
      }
    \tl_if_empty:NTF \l_xeCJK_get_features_tl
      { \tl_set:Nx \l_xeCJK_get_features_tl \l_xeCJK_fontname_tl }
      { \tl_replace_all:Nnx \l_xeCJK_get_features_tl * \l_xeCJK_fontname_tl }
  }
\cs_new_nopar:Npn \xeCJK_get_features_aux:nn [#1]#2
  {
    \clist_set:Nx \l_xeCJK_get_features_clist {#1}
    \tl_set:Nx \l_xeCJK_get_features_tl {#2}
  }
%    \end{macrocode}
% \end{macro}
% \end{macro}
% \end{macro}
%
% \begin{macro}{FallBack}
%    \begin{macrocode}
\xeCJK_new_features_key:n { FallBack }
%    \end{macrocode}
% \end{macro}
%
% \begin{macro}{.clist_set_x:N,.clist_set_x:c}
%    \begin{macrocode}
\cs_set_protected:cpn { \c_keys_props_root_tl .clist_set_x:N } #1
  { \keys_variable_set:NnN #1 { clist } x }
\cs_set_protected:cpn { \c_keys_props_root_tl .clist_set_x:c } #1
  { \keys_variable_set:cnN {#1} { clist } x }
%    \end{macrocode}
% \end{macro}
%
% \begin{macro}{\g_xeCJK_features_id_prop, BoldFont, ItalicFont, BoldItalicFont,
% SlantedFont, BoldSlantedFont,BoldFeatures, ItalicFeatures, BoldItalicFeatures,
% SlantedFeatures, BoldSlantedFeatures}
% 调用字体的属性声明,同 \pkg{fontspec} 宏包。
%    \begin{macrocode}
\cs_new_nopar:Nn \xeCJK_map_features_id:n
  { \prop_get:Nn \g_xeCJK_features_id_prop { #1 } }
\prop_new:N \g_xeCJK_features_id_prop
\prop_put:Nnn \g_xeCJK_features_id_prop { bf   } { Bold        }
\prop_put:Nnn \g_xeCJK_features_id_prop { it   } { Italic      }
\prop_put:Nnn \g_xeCJK_features_id_prop { bfit } { BoldItalic  }
\prop_put:Nnn \g_xeCJK_features_id_prop { sl   } { Slanted     }
\prop_put:Nnn \g_xeCJK_features_id_prop { bfsl } { BoldSlanted }
\prop_map_inline:Nn \g_xeCJK_features_id_prop
  {
    \keys_define:nn { xeCJK / features }
      {
        #2Font        .tl_set_x:c = { l_xeCJK_fontname_#1_tl },
        #2Features .clist_set_x:c = { l_xeCJK_features_#1_clist } ,
      }
  }
%    \end{macrocode}
% \end{macro}
%
% \begin{macro}{AutoFakeBold, AutoFakeSlant, Mono}
%    \begin{macrocode}
\keys_define:nn { xeCJK / features }
  {
    AutoFakeBold  .choice:,
    AutoFakeBold / false   .code:n =
      { \bool_set_false:N \l_xeCJK_AutoFakeBold_bool },
    AutoFakeBold / unknown .code:n =
      {
        \bool_set_true:N \l_xeCJK_AutoFakeBold_bool
        \fp_set:Nn \l_xeCJK_EmboldenFactor_fp { \l_keys_value_tl }
      },
    AutoFakeBold .default:n  = { \g_xeCJK_EmboldenFactor_fp },
    AutoFakeSlant  .choice:,
    AutoFakeSlant / false   .code:n =
      { \bool_set_false:N \l_xeCJK_AutoFakeSlant_bool },
    AutoFakeSlant / unknown .code:n =
      {
        \bool_set_true:N \l_xeCJK_AutoFakeSlant_bool
        \fp_set:Nn \l_xeCJK_SlantFactor_fp { \l_keys_value_tl }
      },
    AutoFakeSlant .default:n  = { \g_xeCJK_SlantFactor_fp },
    Mono .choice:,
    Mono / Exspace .code:n = { \xeCJK_set_monoexspace: },
    Mono / Scale   .code:n =
      {
        \xeCJK_setmonoscale:
        \tl_set:Nx \l_xeCJK_monoscale_tl
          { Scale = { \fp_use:N \g_xeCJK_monoscale_fp } }
      }
  }
%    \end{macrocode}
% \end{macro}
%
% \begin{macro}{\xeCJK_set_init:}
%    \begin{macrocode}
\cs_new:Nn \xeCJK_set_init:
  {
    \tl_clear:N \l_xeCJK_monoscale_tl
    \prop_clear:N \l_xeCJK_add_font_prop
    \prop_map_inline:Nn \g_xeCJK_features_id_prop
      {
        \tl_clear:c     { l_xeCJK_fontname_##1_tl    }
        \clist_clear:c  { l_xeCJK_features_##1_clist }
      }
    \clist_clear:N \l_xeCJK_pass_features_clist
    \bool_set_eq:NN \l_xeCJK_AutoFakeBold_bool   \g_xeCJK_AutoFakeBold_bool
    \bool_set_eq:NN \l_xeCJK_AutoFakeSlant_bool  \g_xeCJK_AutoFakeSlant_bool
    \fp_set_eq:NN \l_xeCJK_EmboldenFactor_fp \g_xeCJK_EmboldenFactor_fp
    \fp_set_eq:NN \l_xeCJK_SlantFactor_fp    \g_xeCJK_SlantFactor_fp
  }
%    \end{macrocode}
% \end{macro}
%
% \begin{macro}{\xeCJK_setCJKfamilyfont:nnn}
% 设置一个 CJK 新字体族,与 |\newfontfamily| 类似,增加 |FallBack| 选项。
%    \begin{macrocode}
\cs_new_protected_nopar:Nn \xeCJK_setCJKfamilyfont:nnn
  {
    \group_begin:
    \xeCJK_set_init:
    \tl_set:Nx \l_xeCJK_familyname_tl {#1}
    \tl_set:Nx \l_xeCJK_fontname_tl   {#3}
    \keys_set_known:nxN { xeCJK / features }
      { \g_xeCJK_default_features_clist, #2 } \l_xeCJK_pass_features_clist
    \xeCJK_parse_features:
    \xeCJK_pass_features:
    \cs_gset_protected_nopar:cpx { xeCJK/font/\l_xeCJK_familyname_tl }
      {
        \exp_not:N \fontspec [ \l_xeCJK_pass_features_clist ] { \l_xeCJK_fontname_tl }
      }
    \clist_gset:cx { xeCJK_fontoptions_\l_xeCJK_familyname_tl } {#2}
    \tl_gset:cx { xeCJK_fontname_\l_xeCJK_familyname_tl } { \l_xeCJK_fontname_tl }
    \group_end:
  }
%    \end{macrocode}
% \end{macro}
%
% \begin{macro}
%  {\xeCJK_add_font:nn,\xeCJK_add_font_if_new:nn}
%    \begin{macrocode}
\cs_new:Nn \xeCJK_add_font:nn
  {
    \prop_put:Nxx \l_xeCJK_add_font_prop
      { \xeCJK_map_features_id:n {#1} Font } {#2}
  }
\cs_new:Nn \xeCJK_add_font_if_new:nn
  {
    \prop_put_if_new:Nxx \l_xeCJK_add_font_prop
      { \xeCJK_map_features_id:n {#1} Font } {#2}
  }
\cs_generate_variant:Nn \xeCJK_add_font:nn { nx, nc, nv }
\cs_generate_variant:Nn \xeCJK_add_font_if_new:nn { nx, nc, nv }
\cs_generate_variant:Nn \prop_put:Nnn { Nxx }
\cs_generate_variant:Nn \prop_put_if_new:Nnn { Nxx }
%    \end{macrocode}
% \end{macro}
%
% \begin{macro}{\xeCJK_add_fake_bold:n,\xeCJK_add_fake_slant:n}
%    \begin{macrocode}
\cs_new:Nn \xeCJK_add_fake_bold:n
  {
    \clist_put_left:cx { l_xeCJK_features_#1_clist }
      { FakeBold = { \fp_use:N \l_xeCJK_EmboldenFactor_fp } }
  }
\cs_new:Nn \xeCJK_add_fake_slant:n
  {
    \clist_put_left:cx { l_xeCJK_features_#1_clist }
      { FakeSlant = { \fp_use:N \l_xeCJK_SlantFactor_fp } }
  }
%    \end{macrocode}
% \end{macro}
%
% \begin{macro}{\xeCJK_parse_features:,\xeCJK_parse_features:n}
%    \begin{macrocode}
\cs_new:Nn \xeCJK_parse_features:
  {
    \prop_map_inline:Nn \g_xeCJK_features_id_prop { \xeCJK_parse_features:n {##1} }
  }
\cs_new:Nn \xeCJK_parse_features:n
  {
    \xeCJK_if_font_exist:nTF {#1}
      {
        \xeCJK_add_font:nv {#1} { l_xeCJK_fontname_#1_tl }
        \xeCJK_if_it_or_sl:nT {#1}
          {
            \xeCJK_if_font_exist:nF {bf#1}
              { \xeCJK_add_font:nv {bf#1} { l_xeCJK_fontname_#1_tl } }
          }
      }
      { \xeCJK_set_fake:n {#1} }
  }
%    \end{macrocode}
% \end{macro}
%
% \begin{macro}[pTF]{\xeCJK_if_font_exist:n}
% \begin{macro}[pTF]{\xeCJK_if_it_or_sl:n}
%    \begin{macrocode}
\prg_set_conditional:Nnn \xeCJK_if_font_exist:n { p, T, F, TF }
  {
    \exp_args:Nc \tl_if_empty:xTF { l_xeCJK_fontname_#1_tl }
      { \prg_return_false: } { \prg_return_true:  }
  }
\prg_set_conditional:Nnn \xeCJK_if_it_or_sl:n { p, T, F, TF }
  {
    \bool_if:nTF
      { \str_if_eq_p:xx { it } {#1} || \str_if_eq_p:xx { sl } {#1} }
      { \prg_return_true:  } { \prg_return_false: }
  }
%    \end{macrocode}
% \end{macro}
% \end{macro}
%
% \begin{macro}{\xeCJK_set_fake:n}
%    \begin{macrocode}
\cs_new:Nn \xeCJK_set_fake:n
  {
    \str_if_eq:xxTF {#1} { bf }
      {
        \bool_if:NT \l_xeCJK_AutoFakeBold_bool { \xeCJK_add_fake_bold:n {#1} }
      }{
        \bool_if:NTF \l_xeCJK_AutoFakeSlant_bool
          { \xeCJK_add_fake_slant:n {#1} }
          { \xeCJK_if_it_or_sl:nT {#1} { \xeCJK_map_it_sl:n {#1} } }
        \xeCJK_if_it_or_sl:nF {#1}
          {
            \bool_if:NT \l_xeCJK_AutoFakeBold_bool { \xeCJK_add_fake_bold:n {#1} }
          }
      }
    \xeCJK_add_font_if_new:nn {#1} { * }
  }
%    \end{macrocode}
% \end{macro}
%
% \begin{macro}{\xeCJK_map_it_sl:n}
%    \begin{macrocode}
\cs_new:Nn \xeCJK_map_it_sl:n
  {
    \xeCJK_if_map_font_exist:nT {#1}
      {
        \xeCJK_add_font:nx {#1} { \xeCJK_get_map_font:n {#1} }
        \xeCJK_if_font_exist:nF {bf#1}
          { \xeCJK_add_font:nx {bf#1} { \xeCJK_get_map_font:n {#1} } }
      }
  }
%    \end{macrocode}
% \end{macro}
%
% \begin{macro}{\xeCJK_get_map_font:n}
%    \begin{macrocode}
\cs_new:Nn \xeCJK_get_map_font:n
  { \tl_use:c { l_xeCJK_fontname_\str_if_eq:xxTF {#1} {it} {sl} {it} _tl } }
%    \end{macrocode}
% \end{macro}
%
% \begin{macro}[pTF]{\xeCJK_if_map_font_exist:n}
%    \begin{macrocode}
\prg_set_conditional:Nnn \xeCJK_if_map_font_exist:n { p, T, F, TF }
  {
    \tl_if_empty:xTF { \xeCJK_get_map_font:n {#1} }
      { \prg_return_false: } { \prg_return_true:  }
  }
%    \end{macrocode}
% \end{macro}
%
% \begin{macro}{\xeCJK_pass_features:}
%    \begin{macrocode}
\cs_new:Nn \xeCJK_pass_features:
  {
    \prop_map_inline:Nn \g_xeCJK_features_id_prop
      {
        \clist_if_empty:cF { l_xeCJK_features_##1_clist }
          {
            \clist_put_right:Nx \l_xeCJK_pass_features_clist
              { ##2Features = { \clist_use:c { l_xeCJK_features_##1_clist } } }
          }
      }
    \prop_map_inline:Nn \l_xeCJK_add_font_prop
      { \clist_put_right:Nx \l_xeCJK_pass_features_clist { ##1 = { ##2 } } }
    \tl_if_empty:NF \l_xeCJK_monoscale_tl
      { \clist_put_right:Nx \l_xeCJK_pass_features_clist \l_xeCJK_monoscale_tl }
  }
%    \end{macrocode}
% \end{macro}
%
% \begin{macro}{\CJKfamily}
% 用于切换 CJK 字体族。
%     \begin{macrocode}
\tl_new:N \l_xeCJK_family_tl
\NewDocumentCommand \CJKfamily { m }
  {
    \cs_if_free:cTF { xeCJK/font/#1 }
      {
        \cs_if_free:cT { xeCJK_warned_aux_#1 }
          {
            \PackageWarning { xeCJK }
              {
                Unknown~CJK~family~`#1'~is~ignored.^^J
                Use~\token_to_str:N \setCJKfamilyfont\c_space_tl to~
                define~a~CJK~family.
              }
            \tl_new:c { xeCJK_warned_aux_#1 }
          }
    }
    { \tl_set:Nx \l_xeCJK_family_tl {#1} }
  }
%    \end{macrocode}
% \end{macro}
%
% \begin{macro}{\setCJKfamilyfont, \newCJKfontfamily, \CJKfontspec}
% 分别用于预声明 CJK 字体和随机调用 CJK 字体。
%    \begin{macrocode}
\NewDocumentCommand \setCJKfamilyfont { m O{} m }
  { \xeCJK_setCJKfamilyfont:nnn {#1} {#2} {#3} }
\NewDocumentCommand \newCJKfontfamily { o m O{} m }
  {
    \IfNoValueTF {#1}
      { \tl_set:Nx \l_xeCJK_familyname_tl { \cs_to_str:N #2 } }
      { \tl_set:Nx \l_xeCJK_familyname_tl {#1} }
    \xeCJK_setCJKfamilyfont:nnn { \l_xeCJK_familyname_tl } {#3} {#4}
    \cs_new_protected_nopar:Npx #2
      { \exp_not:N \CJKfamily { \l_xeCJK_familyname_tl } }
  }
\int_new:N \g_xeCJK_fontspec_int
\NewDocumentCommand \CJKfontspec { O{} m }
  {
    \int_gincr:N \g_xeCJK_fontspec_int
    \tl_set:Nx \l_xeCJK_family_fontspec_tl
      { xeCJK_family_fontspec_\int_use:N \g_xeCJK_fontspec_int }
    \xeCJK_setCJKfamilyfont:nnn { \l_xeCJK_family_fontspec_tl } {#1} {#2}
    \exp_args:Nx \CJKfamily { \l_xeCJK_family_fontspec_tl }
  }
%    \end{macrocode}
% \end{macro}
%
% \begin{macro}{\defaultCJKfontfeatures, \addCJKfontfeatures}
% 分别用于设置 CJK 字体的默认属性和增加当前 CJK 字体的属性。
%    \begin{macrocode}
\clist_clear_new:N \g_xeCJK_default_features_clist
\NewDocumentCommand \defaultCJKfontfeatures { m }
  { \clist_gset:Nn \g_xeCJK_default_features_clist {#1} }
\NewDocumentCommand \addCJKfontfeatures { m }
  {
    \cs_if_free:NF \l_xeCJK_family_tl
      {
        \clist_put_right:cx { xeCJK_fontoptions_\l_xeCJK_family_tl } {#1}
        \CJKfontspec
          [ \clist_use:c { xeCJK_fontoptions_\l_xeCJK_family_tl } ]
          { \tl_use:c { xeCJK_fontname_\l_xeCJK_family_tl } }
      }
  }
\cs_set_eq:NN \addCJKfontfeature \addCJKfontfeatures
%    \end{macrocode}
% \end{macro}
%
% \begin{macro}
%   { \CJKrmdefault, \setCJKmainfont, \CJKsfdefault, \setCJKsansfont,
%     \CJKttdefault, \CJKfamilydefault }
% 设置文档的 CJK 普通字体、无衬线字体。
%    \begin{macrocode}
\cs_if_free:NT \CJKrmdefault { \tl_set:Nn \CJKrmdefault { rm } }
\NewDocumentCommand \setCJKmainfont { O{} m }
  { \xeCJK_setCJKfamilyfont:nnn { \CJKrmdefault } {#1} {#2} }
\cs_new_eq:NN \setCJKromanfont \setCJKmainfont
\cs_if_free:NT \CJKsfdefault { \tl_set:Nn \CJKsfdefault { sf } }
\NewDocumentCommand \setCJKsansfont { O{} m }
  { \xeCJK_setCJKfamilyfont:nnn { \CJKsfdefault } {#1} {#2} }
\cs_if_free:NT \CJKttdefault { \tl_set:Nn \CJKttdefault { tt } }
%    \end{macrocode}
% \end{macro}
%
% \begin{macro}{\CJKfamilydefault}
% CJK 默认字体族,作用于 |\normalfont|。
%    \begin{macrocode}
\cs_if_free:NT \CJKfamilydefault { \tl_set:Nn \CJKfamilydefault { \CJKrmdefault } }
\AtBeginDocument { \CJKfamily \CJKfamilydefault }
\tl_new:c { xeCJK/font/\CJKfamilydefault }
%    \end{macrocode}
% \end{macro}
%
% 补丁 |\normalfont|、|\rmfamily|,|\sffamily| 和 |\ttfamily|,使其同时对
% CJK 字体族有效。
% \begin{macrocode}
\etex_protected:D \tl_put_right:Nn \normalfont { \CJKfamily \CJKfamilydefault }
\etex_protected:D \tl_put_right:Nn \rmfamily   { \CJKfamily \CJKrmdefault }
\etex_protected:D \tl_put_right:Nn \sffamily   { \CJKfamily \CJKsfdefault }
\etex_protected:D \tl_put_right:Nn \ttfamily   { \CJKfamily \CJKttdefault }
\cs_set_eq:NN \reset@font \normalfont
%    \end{macrocode}
%
% \subsection{处理等宽字体和抄录环境}
%
% \begin{macro}{\g_xeCJK_monoscale_fp,\g_xeCJK_exspace_dim}
%    \begin{macrocode}
\fp_new:N \g_xeCJK_monoscale_fp
\fp_gset_eq:NN \g_xeCJK_monoscale_fp \c_one_fp
\dim_gzero_new:N \g_xeCJK_exspace_dim
%    \end{macrocode}
% \end{macro}
%
% \begin{macro}{\setCJKmonoscale}
% \begin{macro}[aux]{\xeCJK_setmonoscale:}
%     \begin{macrocode}
\cs_new:Nn \xeCJK_setmonoscale:
  {
    \dim_gzero:N \g_xeCJK_exspace_dim
    \group_begin:
      \fontfamily \ttdefault \selectfont
      \fp_gset_from_dim:Nn \g_xeCJK_monoscale_fp
        { \c_two \tex_fontdimen:D \c_two \tex_font:D }
      \fp_gdiv:Nn \g_xeCJK_monoscale_fp \f@size
    \group_end:
  }
\NewDocumentCommand \setCJKmonoscale { }
  {
    \xeCJK_setmonoscale:
    \addCJKfontfeatures { Scale = \fp_use:N \g_xeCJK_monoscale_fp }
  }
%    \end{macrocode}
% \end{macro}
% \end{macro}
%
% \begin{macro}{\setCJKmonoscalefont}
%     \begin{macrocode}
\NewDocumentCommand \setCJKmonoscalefont { O{} m }
  {
    \xeCJK_setCJKfamilyfont:nnn { \CJKttdefault } { Mono = Scale, #1 } {#2}
  }
%    \end{macrocode}
% \end{macro}
%
% \begin{macro}{\setCJKmonoexspace}
% \begin{macro}[aux]{\xeCJK_set_monoexspace:}
%     \begin{macrocode}
\cs_new:Nn \xeCJK_set_monoexspace:
  {
    \fp_gset_eq:NN \g_xeCJK_monoscale_fp \c_one_fp
    \group_begin:
      \fontfamily \ttdefault \selectfont
      \dim_gset:Nn \g_xeCJK_exspace_dim
        { \c_two \tex_fontdimen:D \c_two \tex_font:D - \f@size \p@ }
    \group_end:
  }
\NewDocumentCommand \setCJKmonoexspace { } { \xeCJK_set_monoexspace: }
%    \end{macrocode}
% \end{macro}
% \end{macro}
%
% \begin{macro}[aux]
%  { \xeCJK_fixed_ecglue:, \xeCJK_fixed_cjkglue:, \xeCJK_flexible_ecglue:,
%    \xeCJK_flexible_ecglue:, \xeCJK_flexible_cjkglue: }
%    \begin{macrocode}
\cs_new:Nn \xeCJK_fixed_ecglue:  { \skip_horizontal:N .5\g_xeCJK_exspace_dim }
\cs_new:Nn \xeCJK_fixed_cjkglue: { \skip_horizontal:N \g_xeCJK_exspace_dim }
\tl_set_eq:NN \l_xeCJK_flexible_punctstyle_tl \l_xeCJK_punctstyle_tl
\cs_set_eq:NN \xeCJK_flexible_ecglue:  \CJKecglue
\cs_set_eq:NN \xeCJK_flexible_cjkglue: \CJKglue
%    \end{macrocode}
% \end{macro}
%
% \begin{macro}{\CJKfixedspacing}
%     \begin{macrocode}
\NewDocumentCommand \CJKfixedspacing { m }
  {
    \tl_if_eq:NNF \l_xeCJK_punctstyle_tl \c_xeCJK_ps_plain_tl
      {
        \tl_set_eq:NN \l_xeCJK_flexible_punctstyle_tl \l_xeCJK_punctstyle_tl
        \punctstyle { plain }
      }
    \cs_if_eq:NNF \CJKecglue \xeCJK_fixed_ecglue:
      {
        \cs_set_eq:NN \xeCJK_flexible_ecglue: \CJKecglue
        \cs_set_eq:NN \CJKecglue \xeCJK_fixed_ecglue:
      }
    \cs_if_eq:NNF \CJKglue \xeCJK_fixed_cjkglue:
      {
        \cs_set_eq:NN \xeCJK_flexible_cjkglue: \CJKglue
        \cs_set_eq:NN \CJKglue \xeCJK_fixed_cjkglue:
      }
  }
\AtBeginDocument { \tl_put_right:Nn \verbatim@font { \CJKfixedspacing } }
%    \end{macrocode}
% \end{macro}
%
% \begin{macro}{\CJKflexiblespacing}
%     \begin{macrocode}
\NewDocumentCommand \CJKflexiblespacing { m }
  {
    \cs_set_eq:NN \l_xeCJK_punctstyle_tl \l_xeCJK_flexible_punctstyle_tl
    \cs_set_eq:NN \CJKecglue \xeCJK_flexible_ecglue:
    \cs_set_eq:NN \CJKglue   \xeCJK_flexible_cjkglue:
  }
%    \end{macrocode}
% \end{macro}
%
% \begin{macro}{\setCJKmonoexspacefont}
%     \begin{macrocode}
\NewDocumentCommand \setCJKmonoexspacefont { O{} m }
  {
    \xeCJK_setCJKfamilyfont:nnn { \CJKttdefault } { Mono = Exspace, #1 } {#2}
  }
%    \end{macrocode}
% \end{macro}
%
% \begin{macro}{\setCJKmonofont}
% 设置文档的 CJK 等宽字体族。
%    \begin{macrocode}
\NewDocumentCommand \setCJKmonofont { s t+ O{} m }
  {
    \IfBooleanTF {#1}
      { \setCJKmonoscalefont [#3] {#4} }
      { \IfBooleanTF {#2}
          { \setCJKmonoexspacefont [#3] {#4} }
          { \xeCJK_setCJKfamilyfont:nnn { \CJKttdefault } {#3} {#4} }
     }
  }
%    \end{macrocode}
% \end{macro}
%
% \subsection{\pkg{xeCJK} 其它选项}
%
% \begin{macro}{CJKnumber, indentfirst, normalindentfirst}
% 是否启用 \pkg{CJKnumber} 宏包和首行是否缩进,其中 |CJKnumber| 选项仅
% 在 \pkg{xeCJK} 宏包调用的时候有效。并将 \pkg{xeCJK} 中未知的选项传递给
% \pkg{fontspec} 宏包。
%     \begin{macrocode}
\keys_define:nn { xeCJK / options }
  {
    CJKnumber .bool_set:N = \l_xeCJK_number_bool ,
    CJKnumber  .default:n = { true },
    indentfirst .choice:,
    indentfirst / true  .code:n =
      { \cs_set_eq:NN \@afterindentfalse \prg_do_nothing: },
    indentfirst / false .code:n =
      { \cs_set_eq:NN \if@afterindent \if_false: },
    indentfirst      .default:n = { true },
    normalindentfirst .meta:n = { indentfirst = false },
    unknown .code:n =
      { \PassOptionsToPackage { \l_keys_key_tl } { fontspec } },
  }
%    \end{macrocode}
% \end{macro}
%
% \subsection{\pkg{xeCJK} 初始化设置}
%
% \pkg{xeCJK} 宏包的初始化设置。
%     \begin{macrocode}
\keys_set:nn { xeCJK / options }
  {
    xeCJKactive = true , CJKallowbreakbetweenpuncts = false ,
    indentfirst = true , CJKspace = false , CJKecglue = { \c_space_token } ,
    CJKchecksingle = false , CJKnumber = false , PunctStyle = quanjiao ,
    EmboldenFactor = 4 , SlantFactor = .167 ,
    KaiMingPunct = { . 。? ! } , SpecialPunct = { — … ─ } ,
  }
%    \end{macrocode}
%
% \begin{macro}{\CJKsymbol, \CJKpunctsymbol}
%    \begin{macrocode}
\cs_new_nopar:Npn \CJKsymbol      #1 {#1}
\cs_new_nopar:Npn \CJKpunctsymbol #1 {#1}
%    \end{macrocode}
% \end{macro}
%
% 执行宏包选项,并载入 \pkg{fontspec} 宏包。
%    \begin{macrocode}
\ProcessKeysOptions { xeCJK / options }
\RequirePackage { fontspec }
%    \end{macrocode}
%
% \begin{macro}{\xeCJKsetup}
% 在导言区或文档中设置 \pkg{xeCJK} 的接口。
%     \begin{macrocode}
\NewDocumentCommand \xeCJKsetup { m }
  {
    \keys_set:nn { xeCJK / options } {#1}
    \tex_ignorespaces:D
  }
%    \end{macrocode}
% \end{macro}
%
% \begin{macro}{\CJKspace, \CJKnospace}
%     \begin{macrocode}
\NewDocumentCommand \CJKspace { }
  { \keys_set:nn { xeCJK / options } { CJKspace = true } }
\NewDocumentCommand \CJKnospace { }
%    \end{macrocode}
% \end{macro}
%
% \begin{macro}{\makexeCJKactive, \makexeCJKinactive}
%     \begin{macrocode}
  { \keys_set:nn { xeCJK / options } { CJKspace = false } }
\NewDocumentCommand \makexeCJKactive { }
  { \keys_set:nn { xeCJK / options } { xeCJKactive = true } }
\NewDocumentCommand \makexeCJKinactive { }
  { \keys_set:nn { xeCJK / options } { xeCJKactive = false } }
%    \end{macrocode}
% \end{macro}
%
% \begin{macro}{\xeCJKallowbreakbetweenpuncts, \xeCJKnobreakbetweenpuncts}
%     \begin{macrocode}
\NewDocumentCommand \xeCJKallowbreakbetweenpuncts { }
  { \keys_set:nn { xeCJK / options } { CJKallowbreakbetweenpuncts = true } }
\NewDocumentCommand \xeCJKnobreakbetweenpuncts { }
  { \keys_set:nn { xeCJK / options } { CJKallowbreakbetweenpuncts = false } }
%    \end{macrocode}
% \end{macro}
%
% \begin{macro}{\DeclareKaiMingPunct, \DeclareSpecialPunct}
%     \begin{macrocode}
\NewDocumentCommand \DeclareKaiMingPunct { t+ t- m }
  { \IfBooleanTF {#1}
      { \keys_set:nn { xeCJK / options } { KaiMingPunct+ = {#3} } }
      { \IfBooleanTF {#2}
          { \keys_set:nn { xeCJK / options } { KaiMingPunct- = {#3} } }
          { \keys_set:nn { xeCJK / options } { KaiMingPunct  = {#3} } }
      }
  }
\NewDocumentCommand \DeclareSpecialPunct { t+ t- m }
  { \IfBooleanTF {#1}
      { \keys_set:nn { xeCJK / options } { SpecialPunct+ = {#3} } }
      { \IfBooleanTF {#2}
          { \keys_set:nn { xeCJK / options } { SpecialPunct- = {#3} } }
          { \keys_set:nn { xeCJK / options } { SpecialPunct  = {#3} } }
      }
  }
%    \end{macrocode}
% \end{macro}
%
% \begin{macro}
%   {\xeCJKsetEmboldenFactor, \xeCJKsetSlantFactor}
%    \begin{macrocode}
\NewDocumentCommand \xeCJKsetEmboldenFactor { m }
  { \keys_set:nn { xeCJK / options } { EmboldenFactor = {#1} } }
\NewDocumentCommand \xeCJKsetSlantFactor { m }
  { \keys_set:nn { xeCJK / options } { SlantFactor = {#1} } }
%    \end{macrocode}
% \end{macro}
%
% \begin{macro}{\punctstyle, \xeCJKplainchr}
%    \begin{macrocode}
\NewDocumentCommand \punctstyle { m }
  { \keys_set:nn { xeCJK / options } { PunctStyle = {#1} } }
\cs_new_nopar:Npn \xeCJKplainchr { \punctstyle { plain } }
%    \end{macrocode}
% \end{macro}
%
% \begin{macro}{\CJKsetecglue}
%    \begin{macrocode}
\NewDocumentCommand \CJKsetecglue { m }
  { \keys_set:nn { xeCJK / options } { CJKecglue = {#1} } }
\cs_set_eq:NN \xeCJKsetecglue \CJKsetecglue
%    \end{macrocode}
% \end{macro}
%
% \subsection{兼容性修补}
%
% \begin{macro}{\/}
% \begin{macro}[aux]{\xeCJK_itcorr_aux}
% 修复倾斜校正。
%    \begin{macrocode}
\cs_set_eq:NN \xeCJK_itcorr_aux \/
\cs_set_nopar:Npn \/
  {
    \scan_stop:
    \int_compare:nNnT \tex_lastkern:D = \c_two
      { \tex_unkern:D \tex_unkern:D }
    \xeCJK_itcorr_aux
  }
\cs_set_eq:NN \@@italiccorr \/
%    \end{macrocode}
% \end{macro}
% \end{macro}
%
% \begin{macro}{\xeCJK_patch:Nnn}
% 给已有宏内容前后附加补丁。
%    \begin{macrocode}
\cs_new_nopar:Nn \xeCJK_patch:Nnn
  {
    \tl_put_left:Nn  #1 {#2}
    \tl_put_right:Nn #1 {#3}
  }
%    \end{macrocode}
% \end{macro}
%
% 单独处理宽度有分歧的几个标点:包括省略号、破折号、间隔号、引号等中西文混用的
% 符号, 保证其命令形式输出的是西文字体。并对一些编码的符号宏包做特殊处理。
% \begin{macrocode}
\AtBeginDocument
  {
    \clist_map_inline:nn
      {
        \textellipsis , \textemdash , \textperiodcentered ,
        \textcentereddot , \textquoteleft , \textquoteright ,
        \textquotedblleft , \textquotedblright
      }
      {
        \xeCJK_patch:Nnn #1 { \group_begin: \makexeCJKinactive } { \group_end: }
      }
    \xeCJK_patch:Nnn \tipaencoding { \makexeCJKinactive } { }
    \cs_set_eq:NN \xeCJK_aux_r \r
    \cs_set_nopar:Npn \r #1
      {
        \group_begin:
          \makexeCJKinactive \xeCJK_aux_r {#1}
        \group_end:
      }
    \@ifpackageloaded { pifont }
      {
        \RenewDocumentCommand \Pifont { m }
          {
            \fontfamily {#1}   \fontencoding { U }
            \fontseries { m }    \fontshape { n }
            \selectfont
            \makexeCJKinactive
          }
      } { }
  }
%    \end{macrocode}
%
% 禁止在 \pkg{xeCJK} 宏包后再载入 \pkg{CJK} 宏包。
%    \begin{macrocode}
\tl_set:cn { ver@CJK.sty } { 2020/01/01 }
%    \end{macrocode}
%
% \begin{macro}[aux]
%   {\xeCJK_ULprepunctchar:n, \xeCJK_ULpostpunctchar:n, \xeCJK_ULroutines:}
%    \begin{macrocode}
\cs_set_protected_nopar:Nn \xeCJK_ULprepunctchar:n
  {
    { \makexeCJKinactive \CJKpunctsymbol {#1} \nobreak }
    \tex_ignorespaces:D
  }
\cs_set_protected_nopar:Nn \xeCJK_ULpostpunctchar:n
  {
    { \makexeCJKinactive \CJKpunctsymbol {#1} }
    \xeCJK_ignorespaces:
  }
\cs_set_protected_nopar:Nn \xeCJK_ULroutines:
  {
    \xeCJK_inter_class_toks:nnn { Default }   { CJK } { \CJKecglue \CJKsymbol }
    \xeCJK_inter_class_toks:nnn { HalfLeft }  { CJK } { \CJKsymbol }
    \xeCJK_inter_class_toks:nnn { HalfRight } { CJK } { \CJKecglue \CJKsymbol }
    \xeCJK_inter_class_toks:nnn { Boundary }  { CJK } { \xeCJK_Boundary_and_CJK: }
    \clist_map_inline:nn { Default, HalfLeft, HalfRight, Boundary }
      {
        \xeCJK_inter_class_toks:nnn {##1} { FullLeft }  { \xeCJK_ULprepunctchar:n }
        \xeCJK_inter_class_toks:nnn {##1} { FullRight } { \xeCJK_ULpostpunctchar:n }
      }
    \xeCJK_UL_subclass_patch_tl
  }
%    \end{macrocode}
% \end{macro}
%
% 对 \pkg{ulem} 宏包打补丁,以支持 \pkg{CJKfntef} 宏包。
% \begin{macrocode}
\AtBeginDocument
  {
    \cs_if_exist:NT \UL@hook
      {
        \addto@hook \UL@hook
          {
            \cs_set_eq:NN \xeCJK_UL_CJKsymbol \CJKsymbol
            \cs_set_eq:NN \xeCJK_UL_CJKpunctsymbol \CJKpunctsymbol
            \cs_set_nopar:Npn \CJKsymbol #1
              {
                { \xeCJK_get_font: \xeCJK_UL_CJKsymbol {#1} }
                \xeCJK_CJK_kern: \xeCJK_ignorespaces:
              }
            \cs_set_nopar:Npn \CJKpunctsymbol #1
              { { \xeCJK_get_font: \xeCJK_UL_CJKpunctsymbol {#1} } }
            \xeCJK_ULroutines:
          }
      }
  }
%    \end{macrocode}
%
% 使用 \pkg{CJKnumb} 宏包。
%    \begin{macrocode}
\bool_if:NT \l_xeCJK_number_bool
  {
    \tl_set:Nn \CJK@UnicodeEnc { UTF8 }
    \cs_set_nopar:Npn \CJKaddEncHook #1#2
      { \cs_set_nopar:cpn { xeCJK_enc_#1 } {#2} }
    \cs_set_nopar:Npn \Unicode #1#2
      { \tex_char:D \int_eval:n { #1 * \c_two_hundred_fifty_six + #2 } }
    \RequirePackage { CJKnumb }
    \use:c { xeCJK_enc_\CJK@UnicodeEnc }
    \tl_set:Nn \CJK@tenthousand { 万 }
    \tl_set:Nn \CJK@hundredmillion { 亿 }
  }
%    \end{macrocode}
%
% \begin{macro}{\xeCJKcaption}
% 可以使用 \pkg{CJK} 宏包中的 |.cpx| 文件。
%    \begin{macrocode}
\cs_if_free:NT \CJK@ifundefined
  { \cs_set_eq:NN \CJK@ifundefined \cs_if_free:NTF }
\NewDocumentCommand \xeCJKcaption { o m }
  {
    \IfValueT {#1} { \XeTeXdefaultencoding "#1" }
    \char_set_catcode_letter:N \@
    \file_input:n { #2.cpx }
    \char_set_catcode_other:N \@
    \XeTeXdefaultencoding "UTF-8"
  }
%    \end{macrocode}
% \end{macro}
%
%    \begin{macrocode}
\tex_endlinechar:D `\^^M
\char_set_catcode_ignore:n { "FEFF }
%    \end{macrocode}
%
%    \begin{macrocode}
%</package>
%    \end{macrocode}
%
% \end{implementation}
%
