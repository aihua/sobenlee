%# -*- coding:utf-8 -*-
%%%%%%%%%%%%%%%%%%%%%%%%%%%%%%%%%%%%%%%%%%%%%%%%%%%%%%%%%%%%%%%%%%%%%%%%%%%%%%%%%%%%%
%%  MAChapter1.tex'


\chapter{函\emspace 数}\label{ch:1}

\section{函数概念}\label{sec:1.1}

高等数学与初等数学的区别,在于研究的对象和研究的方法不同。初等数学研究的对象主要是常量,如追赶问题中,已知甲、乙的速度与
出发时间,要求甲追上乙的时间,这里要求的只是一个数量——时间;而高等数学所研究的对象是事物的运动规律和现象的变化规律。如
考察初速度为零的物体,在真空中自由下落时,Galileo~发现物体下落的距离与下落的时间平方成正比。怎么由这一运动规律,求出物体
速度和加速度时间的变化规律。这里我们要求的已不是一个数,而是物体的运动规律。在某种意义上,我们可以说初等数学主要是常量的
数学,高等数学是变量的数学。

\subsection{变量与常量}

在生产与生活中,我们接触到各种各样的量。有些量在考察过程中是变化的,取着不同的数值,我们称之为\emph{变量};有些量在考察过
程中是不变化的,取相同的值,我们称之为\emph{常量}。

如火车出战、进站、过桥、拐弯时,速度是在变的,速度时快时慢,所以从考察全行程来说,火车的速度为一变量。若考察火车行驶在两站
之间某一行程时,火车的速度也可以是常量。又如地面两观察站,观察空中卫星的位置时,观察站与卫星间的距离及联线间的角度是变量,%
但构成三角形的内角和是不变的,为一常量。

需要指出的是,常量与变量是相对的。一是指实践中把一个量究竟作为变量处理,还是作为常量处理是相对的。如重力加速度~$g$,在有的
问题中我们把各地的~$g$~看成常量,但在重力探矿中就必须把~$g$~看成变量。一是指数学上常量与变量的区分也是相对的,常量我们也把
它看作变量,但这变量在变化过程中总是取同一个数值。这样一来,我们可以说两个变量之和仍为一变量。否则我们只能说,两个变量之和,%
一般来说是变量,但在特殊情况下也可以为一常量,显然这对讨论问题带来很大的不便。

习惯上,变量常用字母~$x,y,t$~来表示,常量常用字母~$a,b,c$~等表示。

还需要指出的是,严格的讲法,应该是从集合出发,只有元素属于或不属于集合的区分,没有常量与变量的区分,所谓“变量”不过是集合
元素的代表符号。我们这里采用常量、变量等术语,是为了使问题显得形象、直观,以便于初学者理解,而不拘泥与术语上的严格性。

\subsection{函数定义}

在实际问题中,我们关心的不是孤立的量,而是量与量之间的依赖关系,即每一量如何随着另一量的变化而变化。这里我们暂时只限于讨论
两个变量的情形,并从几个具体的例子来说明量与量之间存在的依赖关系。

\begin{wrapfigure}{O}{0mm}
\somefigure
\caption{}\label{fig:sec1.1-1}
\end{wrapfigure}

\begin{example}\label{ex:sec1.1-1}
在重力作用下,物体从离地面高为~$h$~米处自由下落,不计空气阻力时,下落路程~$s$~与时间~$t$~满足关系式:
\begin{equation}\label{eq:sec1.1-1}
  s=\frac12gt^2,\quad 0\leq t\leq\sqrt{\frac{2h}{\smash[b]{g}}}。
\end{equation}
\end{example}

这里~$g=\SI{9.8}{m/s^2}$~是重力加速度,它是常量。$t$~与~$s$~是物体下落过程中的两个变量,当~$t=0$~时,$s=0$,%
当~$t=\smsqrt{2h/g}$~时,得~$s=h$,表示物体已经到达地面。所以~$t$~的变化范围是从~$0$~到~$\smsqrt{2h/g}$,在这范围内~$g$~的每
一个值,由公式~\ref{eq:sec1.1-1}~即可得到对应~$s$~的值。公式~\ref{eq:sec1.1-1}~给出了变量~$s$~与变量~$t$~之间的依赖关系,即
函数关系。

\begin{example}\label{ex:sec1.1-2}
气象台用自动记录器画出了当地某一天的气温变化图(见\ref{fig:sec1.1-1}),图中纵轴表示气温~$T(\si{\degreeCelsius})$,横轴表示时
间~$t$(小时)。
\end{example}

从~$0$~到~$24$~小时内的任一时刻~$t$~的值,根据这条曲线,就可以找出气温~$T$~的唯一确定的值与之对应。这条曲线给
出了变量~$T$~与变量~$t$~之间的函数关系。

\begin{example}\label{ex:sec1.1-3}
给定正实数~$x$,考虑所有不超过~$x$~的素数个数~$N$。
\end{example}

对于区间~$(0,+\infty)$~内每一~$x$,根据上述对应规则,总有唯一的非负整数~$N$~与其对应,为了表示~$N$~依赖于~$x$,我们记作
~$N=\pi(x)$。显然~$N$~与~$x$~没有确切的公式,也没有~$N$~与~$x$~的图象曲线,但~$N$~与~$x$~的对应关系是客观存在的,所以我
们说这个对应关系,给出了~$N$~与~$x$~之间的函数关系。

由上面例子可以看出,变量间有没有函数关系,在于有没有对应关系,不在于有没有公式或图象。这样,就有函数的如下定义。

\begin{definition}
给定集合~$X$,若存在某种对应规则~$f$,对于~$X$~中每一元素~$x\in X$,都有实数集~$\FR$~中唯一的元素~$y$~与之对应,则称~$f$~
是从~$X$~到~$\FR$~的一个\emph{函数},记作
\[
  \map fX\FR。
\]
函数~$f$~在~$x$~点的值记作~$y=f(x)$,而~$X$~称为函数~$f$~的\emph{定义域},~$x$~称为\emph{自变量},~$y$~称为\emph{因变量}。
\end{definition}

\ref{ex:sec1.1-1}~中对应规则~$f$~为:
\[
  \frac12g(\enspace)^2,
\]
即先对自变量作平方运算,然后再乘常数~$g/2$,函数的定义域为闭区间~$\Brack{0,\smsqrt{2h/g}}$;

\ref{ex:sec1.1-2}~中对应规则~$f$~为给定的曲线,函数的定义域为~$[0,24]$;

\ref{ex:sec1.1-3}~中对应规则~$f$~为不超过自变数的素数个数,函数的定义域为开区间~$(0,+\infty)$。

函数定义中包含两个要素,对应规则与定义域。对应规则比能求值的公式更一般,有了能求值的公式当然就有对应规则,当有对应规则
不一定存在能求值的公式,所以在函数定义中我们只说:“有实数集~$\FR$~中唯一的元素~$y$~与之对应”,而不说“能求出实数~$y$~与
之对应”。从函数定义来说,没有求定义域的问题,但在习惯上函数往往是通过公式给出的,这样,使式子有意义的自变量取值范围就
称为函数的\emph{定义域}。

在定义中,我们把对应规则~$f$~称为函数,而把~$f(x)$~称为函数值。严格来说,对应规则不是数,是某种规则,而函数值是数,这两者
是不同的。如~$y=\sin x$,无论~$y$~或~$\sin x$~都不能说是~$x$~的函数,而是函数值,函数是~$\sin$。知道函数我们可以确定各点
的函数值;反之,知道各点的函数值,也就确定了一个函数。如给定~$y=f(x)$,~$x\in X$,则由它所确定的函数的平方记为~$f^2$,
\[
  f^2\colon x\longmapsto\Parenb{f(x)}^2,\quad x\in X。
\]

在分析范围内,除去今后要讲得微分概念及外微分概念之外,我们常把函数与函数值不加区分,函数可以用~$f$,也可以用~$f(x)$~或
~$y=f(x)$~来表示,函数值用~$f(x)$~表示。

设~$A$~是~$X$~的子集,函数~$f$~在~$X$~上定义,我们称函数
\[
  \phi\colon\phi(x)=f(x),\quad x\in A,
\]
为函数~$f$~在~$A$~上的\emph{限制函数},记作~$f\rest A$;相反,函数~$f$~就称为函数~$\phi$~在~$X$~上的\emph{扩充}或\emph{延拓}。

如果两个对应关系有相同的定义域和相同的对应规则,不管变量采用什么记号,都认为是同一个函数。如
\[
  y=\pi x^2,\enspace x\in[0,+\infty),\qquad
  A=\pi R^2,\enspace R\in[0,+\infty),
\]
应认为是同一个函数。

对不同的函数,可以用不同的字母如~$g,h$~或~$\phi,\psi$~等表示。如果~$f(x)$~表示某一函数,而~$g(x)$~为另一函数,若它们的
定义域~$X$~相同,且对任意~$x\in X$,都有
\[
  f(x)=g(x),
\]
则称这两个函数\emph{相等}。如在~$\FR$~上,函数~$f(x)=2\cos^2x-1$~与~$g(x)=1-2\sin^2x$~是相等的。

下面再举几个函数的例子。

\begin{example}\label{ex:sec1.1-4}
绝对值函数
\[
  y=\abs x=\begin{rbdcases}
    x, & x\geq 0;\\
   -x, & x<0。
  \end{rbdcases}
\]
\end{example}

这个函数的对应规则是:在~$x\geq0$~时按公式~$y=x$~计算函数值;在~$x<0$~时按公式~$y=-x$~计算函数值。从函数定义来看,它是定义
在~$\FR$~上的一个函数,称为分段定义的函数,而不是两个函数。

\begin{example}\label{ex:sec1.1-5}
取整函数~$y=\Floor x$,记号~$\Floor x$~表示不超过~$x$~的最大整数。
\end{example}

当~$x=3.5$~时,$\Floor{3.5}=3$;当~$x=3$~时,$\Floor{3}=3$;当~$x=-3.5$~时,$\Floor{3.5}=-4$。一般有
\[
  \Floor x\leq x<\Floor x+1。
\]

这个例子中,事实上我们先给出了对应规则,然后用记号~$\Floor{\enspace}$~表示这一对应规则,记号用熟了以后,也可以说是公式。所
以,一些最基本的公式,实质上是对应规则的符号表示。如~$y=\sin x$,是先有角边对应规则,然后引入符号~$\sin$~表示这一对应规则。

\begin{example}\label{ex:sec1.1-6}
常值函数~$y=C$。
\end{example}
它的对应规则是:对于自变量~$x$~的每一个值,都用常数~$C$~与之对应。这个例子说明函数定义中“有唯一的~$y$~值与之对应”,指的
是单值的意思,即只有一个~$y$~值与之对应,不是指一一对应,即不要求对不同的~$x$,要不同的~$y$~与之对应。

\subsection{函数的图象}

函数的公式表示法,有它的优缺点。优点是便于计算、便于推理,但缺点是不直观、不形象。而我们考虑的问题,寻找方法时,常常凭借
函数的直观形象,所以有必要讨论函数的图象表示。

给定~$x\in X$,求出函数值~$f(x)$,以~$x$~为横坐标,以~$f(x)$~为纵坐标,可以画出平面上一点~$\Parenb{x,f(x)}$。然后让点~$x$~
在定义域~$X$~上变化,相应的点~$\Parenb{x,f(x)}$~就画出平面上一条曲线~$y=f(x)$,这条曲线就称为函数~$f(x)$~的图象。

\begin{definition}
称平面上点集
\[
  D=\Setb{\Parenb{x,f(x)}}{x\in X}
\]
为函数~$f(x)$~的图象。
\end{definition}

一般来说,函数的图象为一曲线,函数不同,所画出的曲线也不同。给定一曲线(设曲线与垂直线最多只有一个交点),也就是
给定一个函数,所以函数与上述曲线可以不加区分,把函数所成曲线,也可以把曲线说成函数。

\ref{fig:sec1.1-2},\ref{fig:sec1.1-3}~和~\ref{fig:sec1.1-4}~分别给出了\ref{ex:sec1.1-4},\ref{ex:sec1.1-5}~和%
\ref{ex:sec1.1-6}~的图象。

\begin{figure}
\begin{floatrow}[3]
\figurebox{\caption{}\label{fig:sec1.1-2}}
          {\somefigure}
\figurebox{\caption{}\label{fig:sec1.1-3}}
          {\somefigure}
\figurebox{\caption{}\label{fig:sec1.1-4}}
          {\somefigure}
\end{floatrow}
\end{figure}

并不是每一个函数的图象都能画出来。如数学上有名的~Dirichlet~函数,它对人们提高函数概念的认识是有意义的,其定义为:
\[
  D(x)=\chi_{_\FQ}(x)\coloneq\begin{cbdcases}
  1, & x\in\FQ;\\
  0, & x\in\FR-\FQ.
  \end{cbdcases}
\]

函数也可以解释成变换、映照或映射,它把集合~$X$~中的点,变为集合
\[
  \Setb{f(x)}{x\in X}
\]
中的点。上述集合称为函数的\emph{值域},记作~$f(X)$。设点~$x\in X$,则称~$y=f(x)$~是~$x$~的\emph{象},这
时\ref{ex:sec1.1-4},\ref{ex:sec1.1-5}~和\ref{ex:sec1.1-6}~的函数可分别用\ref{fig:sec1.1-5},%
\ref{fig:sec1.1-6}~和~\ref{fig:sec1.1-7}~来表示。

\begin{figure}
\begin{floatrow}[3]
\figurebox{\caption{}\label{fig:sec1.1-5}}
          {\somefigure}
\figurebox{\caption{}\label{fig:sec1.1-6}}
          {\somefigure}
\figurebox{\caption{}\label{fig:sec1.1-7}}
          {\somefigure}
\end{floatrow}
\end{figure}

\begin{exercise}
\item 判断并说明下列函数是否相等。
\begin{exlistcols}[4]
  \item $f(x)=\dfrac{x-1}{x^2-1}$,
  \item[]$g(x)=\dfrac1{x+1}$;
  \item $f(x)=x$,\quad $g(x)=(\sqrt x)^2$;
  \item $f(x)=\log_ax^2$,
  \item[]$g(x)=2\log_ax$;
  \item $f(x)=\sqrt{x+1}\cdot\sqrt{x-1}$,~$g(x)=\sqrt{x^2-1}$。
\end{exlistcols}
\item 设
\[
  f(x)=\begin{cbdcases}
  x+1, & x\leq 1;\\
  x^2,&x>1。\end{cbdcases}
\]
求~$f(x)$,~$f\Parenb{f(1)}$。
\item 设~$f(x)=2x^2+2x-4$。求~$f(1)$,~$f\Parenb{f(1)}$,~$f(x^2)$,~$f^2(x)$,~$f(-x^2)$,~$f(a+b)$,~$f(a)+f(b)$。
\item 设~$f(x)=\dfrac{x+2}{x+1}$。
\begin{exlistcols}
  \item 求~$f(1)$,~$f\Parenb{f(1)}$,~$f\Parenb{f\Parenb{f(1)}}$;
  \item 求~$f(\sqrt2)$;
  \item 证明,对任意~$x>0$,且~$x\neq\sqrt2$,有~$\abs{f^2(x)-2}<\abs{x^2-2}$。
\end{exlistcols}
\item 确定下列函数的定义域。
\begin{exlistcols}
  \item $y=\sqrt{\ln\dfrac1{\smash[b]{4}}(5x-x^2)}$;
  \item $y=\dfrac1x-\sqrt{2x^2+5x+3}$;
  \item $y=\sqrt{\cos x^2}$;
  \item $y=\ln\Paren[\Big]{\sin\dfrac\pi x}$。
\end{exlistcols}
\item 确定函数~$y=\sqrt{x-x^2}$~的定义域和值域。
\item 作出下列函数的图象。
\begin{exlistcols}[3]
  \item $y=\abs{x-1}$;
  \item $y=x-\Floor x$;
  \item $y=\ln(1+x)$;
  \item $y=\ln ax$,~$a=2,-2$;
  \item $y=3\sin2\Paren[\Big]{x+\dfrac\pi8}$;
  \item $y=3\sin\Paren[\Big]{2x+\dfrac\pi8}$。
\end{exlistcols}
\item 作函数~$y=\abs{x-a}+\dfrac12\abs{x-b}$,$a<b$~的图象。
\item 设~$f(x)$~如\ref{fig:sec1.1-ex1}~所示,试写出其表达式,并作出下列函数的图象。
\begin{exlistcols}[3]
  \item $y=f(-x)$;
  \item $y=-f(x)$;
  \item $y=\abs{f(x)}$;
  \item $y=f(\abs x)$;
  \item $y=f\Paren[\Big]{\dfrac x2}$;
  \item $y=f(2x)$。
\end{exlistcols}
\item 某水渠的横断面是一个等腰梯形(见\ref{fig:sec1.1-ex2}),底宽~$2$~米,坡度为~$1:1$~(即倾角
为~\ang{45}),~\textit{ABCD}~叫过水断面,求过水断面的面积~$S$~与水深~$h$~的函数关系。
\begin{figure}
\begin{floatrow}[2]
\figurebox{\caption{}\label{fig:sec1.1-ex1}}
          {\somefigure}
\figurebox{\caption{}\label{fig:sec1.1-ex2}}
          {\somefigure}
\end{floatrow}
\end{figure}
\item 一窗户(见\ref{fig:sec1.1-ex3})下面为矩形,上面为半圆形,周长为~$\ell$,试将窗户的面积表示成底边~$x$~的函数。
\begin{figure}
\begin{floatrow}[2]
\figurebox{\caption{}\label{fig:sec1.1-ex3}}
          {\somefigure}
\figurebox{\caption{}\label{fig:sec1.1-ex4}}
          {\somefigure}
\end{floatrow}
\end{figure}
\item 梯形如\ref{fig:sec1.1-ex4}~所示,当以垂直于~$x$~轴的直线扫过该梯形时,若直线的垂足为~$x\,(x\in\FR)$,试将扫过面积表示为~$x$~的
函数。
\end{exercise}



\section{函数的几种特性}

\subsection{函数的奇偶性}

设函数~$y=f(x)$~的定义域~$X$~关于原点对称,即~$x\in X$~时,有~$-x\in X$。若函数满足,对任意~$x\in X$,都有
\begin{equation}\label{eq:sec2.1-1}
f(-x)=-f(x),
\end{equation}
则称~$f(x)$~是\emph{奇函数};若函数满足,对任意~$x\in X$,都有
\begin{equation}\label{eq:sec2.1-2}
f(-x)=f(x),
\end{equation}
则称~$f(x)$~是\emph{偶函数}。

奇函数的图象是关于原点对称的。因为由~\ref{eq:sec2.1-1},若~$\Parenb{x,f(x)}$~在图象上,则它的关于原点对称的
点~$\Parenb{-x,-f(x)}=\Parenb{-x,f(-x)}$~也在图象上(见\ref{fig:sec1.2-8})。

偶函数的图象是关于~$y$~轴对称的。因为由~\ref{eq:sec2.1-2},若~$\Parenb{x,f(x)}$~在图象上,则它的关于~$y$~轴对称的
点~$\Parenb{-x,f(x)}=\Parenb{-x,f(-x)}$~也在图象上(见\ref{fig:sec1.2-9})。

\begin{figure}
\begin{floatrow}[2]
\figurebox{\caption{奇函数}\label{fig:sec1.2-8}}
          {\somefigure}
\figurebox{\caption{偶函数}\label{fig:sec1.2-9}}
          {\somefigure}
\end{floatrow}
\end{figure}

例如函数~$y=\cos x$,~$y=x^2$~是偶函数,函数~$y=\sin x$,~$y=x^3$~是奇函数,而函数~$y=\sin x+\cos x$~既非奇函数也非偶函数。

\subsection{函数的单调性}

设函数~$f(x)$~在区间~$X$~上有定义,若对任意~$x_1,x_2\in X$,且~$x_1<x_2$~时,有
\[
  f(x_1)\leq f(x_2)\quad\Parenb{f(x_1)\geq f(x_2)},
\]
则称~$f(x)$~在此区间上\emph{单调上升}或\emph{单调递增}(\emph{单调下降}或\emph{单调递减})。

又若~$x_1<x_2$~时,有
\[
  f(x_1)<f(x_2)\quad\Parenb{f(x_1)>f(x_2)},
\]
则称~$f(x)$~在此区间上\emph{严格单调上升}或\emph{严格单调递增}(\emph{严格单调下降}或\emph{严格单调递减})。

上述函数统称为\emph{单调函数}。

例如~$y=x^2$~在~$(0,+\infty)$~上严格单调上升,在~$(-\infty,0)$~上严格单调下降;$y=x^3$~在~$\FR$~上严格单调上升;%
$y=\Floor x$~在~$\FR$~上单调上升。

\subsection{函数的有界性}

\begin{wrapfigure}[10]{O}{0mm}
\somefigure
\caption{有界函数}\label{fig:sec1.2-10}
\end{wrapfigure}

设函数~$f(x)$~在~$X$~上有定义,若存在~$M>0$,使得对任意~$x\in X$,都有
\[
  \abs{f(x)}\leq M,
\]
则称函数~$f(x)$~在~$X$~上是\emph{有界}的。

例如~$y=\sin x$~在~$\FR$~上是有界的,因为取~$M=1$,对任一~$x\in\FR$,都有~$\abs{\sin x}\leq1$。

函数在~$X$~上有界,从几何上看,即它的图象(见\ref{fig:sec1.2-10})位于直线~$y=-M$~与~$y=M$~之间。那么函数~$f(x)$~在~$X$~上
无界,即找不到直线~$y=M$~和~$y=-M$,使曲线全部落在两直线之间。这么说不好检验,为了便于检验,我们不妨换一种方式来说:函
数~$f(x)$~在~$X$~上无界,即任意取直线~$y=M$~和~$y=-M$,曲线不全落在两直线之间(若曲线全落在两直线之间,不就成为有界了
吗!),而曲线不全落在两直线之间,意味着总有一点乱在两直线之外,用符号表示为,存在~$x'\in X$,使得~$\abs{f(x')}>M$。

于是,我们得到函数~$f(x)$~在~$X$~上无界的定义:对于任意~$M>0$,存在~$x'\in X$,使得
\[
  \abs{f(x')}>M,
\]
则~$f(x)$~在~$X$~上无界。

例如函数~$f(x)=\dfrac1x$~在~$(0,+\infty)$~上无界。因为对任意~$M>0$,取~$x'=\dfrac1{2M}\in(0,+\infty)$,
\[
  f(x')=2M>M,
\]
所以函数在~$(0,+\infty)$~上无界。

我们可以发现一条规律,只要吧有界定义中的“存在”换成“任意”,“任意”换成“存在”,“$\leq$”换成“$>$”,就是函数无界的定义。

\subsection{函数的周期性}

设函数~$f(x)$~在~$\FR$~上有定义,若存在~$\ell>0$,使得对任意~$x\in\FR$,都有
\[
  f(x+\ell)=f(x),
\]
则称~$f(x)$~是周期函数,而~$\ell$~成为函数~$f(x)$~的\emph{周期}。

显然周期函数有无穷多个周期,如~$\ell$~是函数的周期,则有
\[
  f(x+\ell)=f\Parenb{(x+\ell)+\ell}=f(x+\ell)=f(x),
\]
所以~$2\ell$~也是~$f(x)$~的周期。一般地,有
\[
  f(x+k\cdot\ell)=f(x),
\]
其中~$k$~为正整数。

若无穷多个周期~$\ell$~中,有一个最小的正数~$T$,则称~$T$~为周期函数~$f(x)$~的\emph{最小正周期},简称\emph{周期}。

例如,正弦函数~$y=\sin x$~是周期为~$2\pi$~的函数,因为
\[
  \sin(x+2\pi)=\sin x。
\]

不在整个实轴上定义的函数,也可以讨论它的周期性。如正切函数~$t=\tan x$~的定义域为实轴除去点
\[
  x=\Paren[\Big]{k+\frac12}\pi,\quad k=0,\pm1,\pm2,\dotsc,
\]
同样可以讨论它的周期性。因为
\[
  \tan(x+\pi)=\tan x,
\]
所以~$\tan x$~是周期为~$\pi$~的周期函数。

既然周期函数的值每隔一个周期都是相同的,所以给周期函数作图时,只要作出一个周期的图象,然后周而复始的画这图象,即得
整个周期函数的图象。

对~Dirichlet~函数~$D(x)$~而言,任何有理数~$\ell$~都是它的周期。因为有理数之和为有理数,无理数与有理数之和为无理数,所以
\[
  D(x+\ell)=D(x),
\]
但它没有最小的正周期。

\begin{exercise}
\item 对下列函数
\begin{exlistcols}[3]
  \item $y=\abs{\sin x}$;
  \item $y=x-\Floor x$;
  \item $y=\tan\abs x$;
  \item $y=\sec 2x$;
  \item $y=\cos x+\sin x$;
  \item $y=\sqrt{x(2-x)}$,
\end{exlistcols}
分别讨论
\begin{exlistcols}
  \item 函数的定义域和值域;
  \item 函数的奇偶性;
  \item 函数的周期性;
  \item 作出函数的图象。
\end{exlistcols}
\item 证明~$y=\ln\Parenb{x+\sqrt{1+x^2}}$~在~$\FR$~上是奇函数,且严格单调上升。
\item 设~$f(x)$~在~$\FR$~上为奇函数,且在~$[0,+\infty)$~上严格单调上升,证明~$f(x)$~在~$\FR$~上严格单调上升。
\item 设~$f(x)=\smsqrt x\,(0\leq x<1)$。
\begin{exlist}
  \item 将~$f(x)$~延拓到~$(-1,1)$~上,使其成为偶函数;
  \item 将~$f(x)$~延拓到~$\FR$~上,使其成为周期为~$1$~的周期函数。
\end{exlist}
\item 设~$f(x)$~在~$[0,a)$~上有定义,这里~$a>0$。
\begin{exlist}
  \item 将~$f(x)$~延拓到~$(-a,a)$,使其成为偶函数;
  \item 将~$f(x)$~延拓到~$\FR$~上,使其成为周期为~$a$~的周期函数。
\end{exlist}
\item 证明,两个奇函数之积为偶函数,而奇函数与偶函数之积为奇函数。
\item 证明,任一在~$\FR$~上定义的函数都可分解为奇函数与偶函数之和。
\item 设~$f(x)$~是周期为~$T\,(T>0)$~的周期函数,证明,~$f(-x)$~也是周期为~$T$~的周期函数。
\item 设~$f(x),g(x)$~为~$\FR$~上的单调函数,证明,~$f\Parenb{g(x)}$~也是~$\FR$~上的单调函数。
\item 设~$f(x)$~在~$(0,+\infty)$~上定义,$x_1,x_2>0$。证明,
\begin{exlist}
  \item 若~$\dfrac{f(x)}x$~单调下降,则~$f(x_1+x_2)\leq f(x_1)+f(x_2)$;
  \item 若~$\dfrac{f(x)}x$~单调上升,则~$f(x_1+x_2)\geq f(x_1)+f(x_2)$。
\end{exlist}
\item 设~$x_1,x_2>0$。证明,
\begin{exlistcols}
  \item $(x_1+x_2)^p\leq x_1^p+x_2^p$,~$0<p\leq1$;
  \item $(x_1+x_2)^p\geq x_1^p+x_2^p$,~$p>1$。
\end{exlistcols}
\item 设~$f(x),g(x)$~在~$(a,b)$~上单调上升。证明,~$\max\Braceb{f(x),g(x)}$~和~$\min\Braceb{f(x),g(x)}$~也在~$(a,b)$~上单调上升。
\item 用肯定语气叙述:在~$\FR$~上,
\begin{exlistcols}
  \item $f(x)$~不是奇函数;
  \item $f(x)$~不是周期函数;
  \item $f(x)$~不是单调上升函数;
  \item $f(x)$~不是单调函数。
\end{exlistcols}
\item 用肯定语气叙述:
\begin{exlistcols}
  \item $f(x)$~在~$(a,b)$~上无上界;
  \item $f(x)$~在~$(a,b)$~上没有零点;
  \item $f(x)$~在~$(a,b)$~上没有比中点函数值更大的点;
  \item $f(x)$~在~$(a,b)$~上没有左边函数值比右边函数值都小的点。
\end{exlistcols}
\end{exercise}

\section{复合函数与反函数}

这节我们将函数的运算。函数除加、减、乘、除四则运算外,还有复合函数与反函数的运算。有了函数的运算,才能使我们从几个已知
函数出发,构造除许许多多新的函数。

设~$f(x),g(x)$~在~$X$~上定义,则
\[
  f(x)\pm g(x),\quad f(x)\cdot g(x),\quad\frac{f(x)}{g(x)}\,\Parenb{g(x)\neq0}
\]
也是~$X$~上的函数。四则运算没有什么新的内容,需要讨论的是复合函数与反函数。

\subsection{复合函数}

例如函数
\[
  y=\sin x^2,
\]
可以看成是函数~$y=\sin u$~和~$u=x^2$~的复合,函数~$\sin x^2$~就称为函数~$y=\sin u$~和~$u=x^2$~的复合函数。一般地,有
如下的定义。

\begin{definition}
设函数~$y=f(u)$~的定义域包含函数~$u=g(x)$~的值域,则在函数~$g(x)$~的定义域~$X$~上可以确定一个函数
\[
  y=f\Parenb{g(x)},
\]
称为~$g$~与~$f$~的\emph{复合函数},有时也记作~$f\comp g$。
\end{definition}

变量~$u$~称为中间变量,给定自变量~$x\in X$,通过对应规则~$g$,确定中间变量~$u$,再通过对应规则~$f$,确定出因变量~$y$,这样
就建立起自变量~$x$~与因变量~$y$~之间的对应规则~$f\comp g$。

我们知道,函数概念中主要是对应规则,定义域和值域。至于自变量和因变量采用什么记号是无关紧要的。所以只要~$f(x)$~的定义域包
含~$g(x)$~的值域,就可以讨论复合函数~$f\Parenb{g(x)}$。

若函数~$f(x)$~的定义域包含~$g(x)$~的值域,并且函数~$g(x)$~的定义域包含~$f(x)$~的值域,那么复合函数~$f\comp g$~与~$g\comp f$~存
在。一般来说,
\[
  f\comp g\neq g\comp f。
\]
例如~$f(x)=\sin x$,~$g(x)=x^2$,则
\[
  \sin x^2\neq(\sin x)^2。
\]
这说明复合函数与四则运算不同,它没有交换律。容易证明结合律是成立的,即
\[
  f\comp(g\comp h)=(f\comp g)\comp h。
\]

以后用时,既要会把几个简单函数复合成一个函数,也要会把一个函数才成几个简单函数的复合。


\subsection{反函数}

在圆面积公式
\[
  S=\pi R^2
\]
中,$R$~是自变量,而~$S$~为因变量,表示圆的面积随半径的变化而变化。事实上,半径~$R$~与面积~$S$~同时发生变化,很难说哪个先变,%
哪个后变,因此没有理由一定要把~$R$~取作自变量,也可以把面积~$S$~取作自变量,这时半径~$R$~就是面积~$S$~的函数
\[
  R=\sqrt{\frac S\pi},
\]
这个函数就称为原来面积函数的反函数。

作为复合函数运算时,我们对函数的定义域和值域加以限制,而对对应规则没有什么限制。求反函数时,我们需要对对应规则加以限制,只有
具有一一对应的函数才能求反函数。

一一对应其实不是新的概念。如所有实数与实轴上的点是一一对应的;班上每个同学与其学号是一一对应的,确切说有下面的定义。

\begin{definition}
设~$f(x)$~在~$X$~上有定义。对任意~$x_1,x_2\in X$,若
\[
  x_1\neq x_2\implies f(x_1)\neq f(x_2),
\]
或
\[
  f(x_1)=f(x_2)\implies x_1=x_2,
\]
则称函数~$f$~在~$X$~上是一一对应的。
\end{definition}

所以,一一对应的函数就是把不同的~$x$~变成不同的~$y$,具体证明时,常采用等号形式比较方便。

一一对应的函数必有反函数存在。

\begin{definition}
设~$y=f(x)$~在~$X$~上一一对应,值域为~$Y$,对于任意~$y\in Y$,用满足~$f(x)=y$~的唯一确定的~$x\in X$~与之对应,由
这样的对应关系所确定的函数~$x=f^{-1}(y)$,就称为原来函数~$y=f(x)$~的\emph{反函数}。
\end{definition}

函数与反函数的对应规则、定义域和值域是不同的,反函数的定义域和值域,恰好是原来函数的值域和定义域,即
\[
  \map fXY\implies\map{f^{-1}}YX。
\]
显然有
\[
  f^{-1}\comp f=\map {\mathbb 1}XX,\quad
  f\comp f^{-1}=\map {\mathbb 1}YY,\quad
  (f^{-1})^{-1}=\map fXX,
\]
其中~$\mathbb 1$~表示恒等变换。

若~$f(x)$~在~$X$~上严格单调,由于严格单调函数是一一对应的,所以严格单调的函数,必有反函数存在。反之,一一对应的函数,不一定是
严格单调的。如下列函数(见\ref{fig:sec1.2-11})
\[
  f(x)=\begin{cbdcases}
  x,   & 0\leq x<1;\\
  3-x, & 1\leq x\leq 2。
  \end{cbdcases}
\]
在~$[0,2]$~上时一一对应的,但不是单调函数。这个函数的反函数仍是它自己。

\begin{figure}
\begin{floatrow}[2]
\figurebox{\caption{}\label{fig:sec1.2-11}}
          {\somefigure}
\figurebox{\caption{}\label{fig:sec1.2-12}}
          {\somefigure}
\end{floatrow}
\end{figure}

下面讨论反函数的图象。因为从方程的观点来看,函数与反函数没有什么区别,点~$(x,y)$~满足方程~$y=f(x)$,也一定满足
方程~$x=f^{-1}(y)$。所以,取~$x$~为自变量画出的函数曲线~$y=f(x)$,若改取~$y$~为自变量,它就是反函数~$x=f^{-1}(y)$~的
曲线(见\ref{fig:sec1.2-12})。这样观察反函数曲线时,就要沿着~$y$~轴去看,很不方便。习惯上我们把自变量轴放在水平位置,%
为此,只要把~$xy$~平面绕直线~$y=x$~选择~\ang{180},$y$~轴就转到水平位置,$x$~轴转到垂直位置,旋转后的曲线就是反函数
~$x=f^{-1}(y)$~的图象(见\ref{fig:sec1.2-13})。又一个习惯,当单独讨论反函数时,总是把自变量用~$x$~来记,因变量用~$y$~
来记,所以旋转后,再把记号改过来,把~$x$~改成~$y$,$y$~改成~$x$,即为反函数~$y=f^{-1}(x)$~的图象(见\ref{fig:sec1.2-14})。%
若反函数与原函数放在一起讨论时,反函数仍记为~$x=f^{-1}(y)$。

\begin{figure}
\begin{floatrow}[2]
\figurebox{\caption{}\label{fig:sec1.2-13}}
          {\somefigure}
\figurebox{\caption{}\label{fig:sec1.2-14}}
          {\somefigure}
\end{floatrow}
\end{figure}

需要指出的是,变量记号不是本质的,我们可以把~$x=f^{-1}(y)$,$y\in Y$,称为~$y=f(x)$~的反函数,同样,我们也可以把
~$y=f^{-1}(x)$,$x\in Y$,称为函数~$y=f(x)$~的反函数。

反函数还有另一种定义,正如实数~$a$~的倒数有两种定义(一种是:$a\neq0$,称~$1/a$~为~$a$~的倒数;令一种是:存在~$b\in\FR$,%
使得~$a\cdot b=1$,则称~$b$~为~$a$~的倒数)一样,反函数也可以用复合函数来定义。我们把它写成定理。

\begin{theorem}
给定函数~$y=f(x)$,其定义域和值域分别记作~$X$~和~$Y$,若在集合~$Y$~上存在函数~$g(y)$,满足对任意~$x\in X$,都有
\[
  g\Parenb{f(x)}=x,
\]
则对任意~$y\in Y$,有
\[
  g(y)=f^{-1}(y)。
\]
\end{theorem}
\begin{proof}
问题是要证~$y=f(x)$~的反函数~$x=f^{-1}(y)$~存在,且等于~$g(y)$。

先证明反函数存在,即要证~$f(x)$~在~$X$~上是一一对应的。事实上,对于任意~$x_1,x_2\in X$,若~$f(x_1)=f(x_2)$,由定理条件
\[
  g\Parenb{f(x_1)}=x_1,\quad g\Parenb{f(x_2)}=x_2,
\]
因为函数~$g(y)$~在同一点值应相同,得到~$x_1=x_2$,这就确保了函数~$f(x)$~在~$X$~上是一一对应的。

其次证明~$g(y)=f^{-1}(y)$。事实上,对于任意~$y\in Y$,因~$y$~属于~$f(x)$~的值域,所以存在~$x\in X$,使得
\[
  y=f(x)。
\]
由反函数定义
\[
  x=f^{-1}(y),
\]
又由定理条件可知
\[
  g(y)=g\Parenb{f(x)}=x,
\]
结合以上两式,即可得到
\[
  g(y)=f^{-1}(y)。\qedhere
\]
\end{proof}

\begin{exercise}
\item 设~$f(x)=\dfrac{1-x}{1+x}$,证明,$f\Parenb{f(x)}=x$。
\item 设~$f(x)=\dfrac{ax+b}{cx+d}$,确定~$f\Parenb{f(x)}=x$~的条件。
\item 确定下列函数的反函数及其定义域。
\begin{exlistcols}
  \item $\dps y=\frac12\Paren[\Big]{x-\frac 1x}$,\enspace$x\in(0,+\infty)$;
  \item $y=\dfrac12\Parenb{\me^x-\me^{-x}}$,\enspace$x\in\FR$。
\end{exlistcols}
\item 在下列指定定义域上确定函数~$\dps y=\frac12\Paren[\Big]{x+\frac1x}$~的反函数。
\begin{exlistcols}
  \item $0<\abs x<1$;
  \item $\abs x>1$。
\end{exlistcols}
\item 若~$f(x)$~是一一对应的奇函数,证明其反函数也是奇函数。
\item 讨论函数~$y=x-\e\sin x\,(0<\e<1)$~的反函数的存在性。
\item 设~$f(x)=\arccos x$,而~$g(x)=\sin x$。确定复合函数~$f\Parenb{g(x)}$~和~$g\Parenb{f(x)}$~的定义域和值域,并作出它们的图象。
\item 设
\[
  f(x)=\begin{cbdcases}
    -x-1, & x\leq0;\\
    x,& x>0,
  \end{cbdcases}\qquad
  g(x)=\begin{cbdcases}
    x,    & x\leq0;\\
    -x^2, & x>0。
  \end{cbdcases}
\]
确定复合函数~$f\Parenb{g(x)}$~和~$g\Parenb{f(x)}$。
\item 设~$f(x)=\dfrac x{\sqrt{1+x^2}}$,确定~$n$~次复合函数~$(f\comp f\comp\dotsb\comp f)(x)$。
\item 设~$f(x)=\abs{1+x}-\abs{1-x}$,确定~$n$~次复合函数~$(f\comp f\comp\dotsb\comp f)(x)$。
\item 若~$f(x)$~在~$\FR$~上定义,且~$f\Parenb{f(x)}\equiv x$。
\begin{exlistcols}
  \item 确定这种函数的数量;
  \item 在要求~$f(x)$~在~$\FR$~上严格上升的条件下,重新确定这种函数的数量。
\end{exlistcols}
\item 若~$f(x),g(x)$~可以按两种顺序复合,且~$f\comp g=g\comp f$,判断并说明~$f$~与~$g$~是否互为反函数。
\item 设函数~$y=f(x)$~的定义域为~$X$,值域为~$Y$,函数~$z=g(y)$~的定义域为~$Y$,值域为~$Z$。证明,函数~$z=g\Parenb{f(x)}$~有
反函数当且仅当~$f(x)$~和~$g(y)$~都有反函数存在,且~$(g\comp f)^{-1}=f^{-1}\comp g^{-1}$。
\end{exercise}


\section{基本初等函数}

常数函数、幂函数、指数函数、对数函数、三角函数和反三角函数,这六种函数叫作\emph{基本初等函数}。

基本初等函数经过有限次加,减,乘,除,复合运算所得到的函数,称为\emph{初等函数}。要研究初等函数,首先就要熟悉基本初等函数的
性质。基本初等函数的简单性质,在初等数学中已经讲过,这里只是结合图象把它们的性质复习一下。

\subsubsection{常数函数}

常数函数~$y=C\,(x\in\FR)$~的图象是通过~$(0,C)$~点,且平行于~$x$~轴的直线(见\ref{fig:sec1.2-15})。

\subsubsection{幂函数}

\ref{fig:sec1.2-16}~是幂函数~$y=x^\alpha\,(0<x<+\infty,\alpha\neq0)$~的图象。当~$\alpha>0$~时,函
数~$y=x^\alpha$~在~$(0,+\infty)$~上严格上升;而当~$\alpha<0$~时,函数~$y=x^\alpha$~在~$(0,+\infty)$~上严格下降。

函数~$y=x^\alpha$~与~$y=x^{\frac1\alpha}$~互为反函数。

\begin{figure}
\begin{floatrow}
\begin{minipage}{.4\hsize}
\figurebox{\caption{常数函数}\label{fig:sec1.2-15}}
          {\somefigure}
\figurebox{\caption{幂函数}\label{fig:sec1.2-16}}
          {\somefigure}
\end{minipage}\qquad
\begin{minipage}{.4\hsize}
\figurebox{\caption{指数函数}\label{fig:sec1.2-17}}
          {\somefigure}
\figurebox{\caption{对数函数}\label{fig:sec1.2-18}}
          {\somefigure}
\end{minipage}
\end{floatrow}
\end{figure}

\subsubsection{指数函数}

\ref{fig:sec1.2-17}~是指数函数~$y=a^x\,(a>0,a\neq1)$~的图象。当~$a>1$~时,函数~$y=a^x$~在~$\FR$~上严格上升;而当~$a<1$~时,函
数~$y=a^x$~在~$\FR$~上严格下降。

\subsubsection{对数函数}

\ref{fig:sec1.2-18}~是对数函数~$y=\log_ax\,(a>0,a\neq1,0<x<+\infty)$~的图象。当~$a>1$~时~$y=\log_ax$~在~$(0,+\infty)$~上严格
上升;而当~$a<1$~时,函数~$y=\log_ax$~在~$(0,+\infty)$~上严格下降。

函数~$y=a^x$~与~$y=\log_ax$~互为反函数。

\subsubsection{三角函数}

\ref{fig:sec1.2-19}~是正弦函数~$y=\sin x\,(x\in\FR)$~的图象,余弦函数~$y=\cos x$~见\ref{fig:sec1.2-20}。

正切函数~$y=\tan x\Parenb{x\neq (k+1/2)\pi,k=0,\pm1,\pm2,\dotsc}$~见\ref{fig:sec1.2-21},余切函
数~$y=\cot x\,(x\neq k\pi,k=0,\pm1,\pm2)$~见\ref{fig:sec1.2-22}。

\begin{figure}
\begin{floatrow}
\begin{minipage}{.4\hsize}
\figurebox{\caption{正弦函数}\label{fig:sec1.2-19}}
          {\somefigure}
\figurebox{\caption{余弦函数}\label{fig:sec1.2-20}}
          {\somefigure}
\end{minipage}\qquad
\begin{minipage}{.4\hsize}
\figurebox{\caption{正切函数}\label{fig:sec1.2-21}}
          {\somefigure}
\figurebox{\caption{余切函数}\label{fig:sec1.2-22}}
          {\somefigure}
\end{minipage}
\end{floatrow}
\end{figure}

\subsubsection{反三角函数}

反三角函数不是一一对应的,为了讨论反函数,我们必需取定一个严格单调分支。对于正弦函数,我们
取~$\Brack[\Big]{-\dfrac\pi2,\dfrac\pi2}$~上这一严格上升分支;对于余弦函数,我们取~$[0,\pi]$~上这一严格下降分支;对正切函数,%
取~$\Paren[\Big]{-\dfrac\pi2,\dfrac\pi2}$~上这一严格上升分支;对于余切函数,取~$(0,\pi)$~上这一严格下降分支。每个分支都是
一一对应的函数,所以反函数存在。

反正弦函数~$y=\arcsin x\Paren[\Big]{-1\leq x\leq1,-\dfrac\pi2\leq y\leq\dfrac\pi2}$~见\ref{fig:sec1.2-23};

反余弦函数~$y=\arccos x\Paren{-1\leq x\leq1,0\leq y\leq\pi}$~见\ref{fig:sec1.2-24};

反正切函数~$y=\arctan x\Paren[\Big]{x\in\FR,-\dfrac\pi2<y<\dfrac\pi2}$~见\ref{fig:sec1.2-25};

反正弦函数~$y=\arccot x\Paren[\Big]{x\in\FR,0<y<\pi}$~见\ref{fig:sec1.2-26}。

\begin{figure}
\begin{floatrow}
\begin{minipage}{.4\hsize}
\figurebox[1.2\FBwidth]{\caption{反正弦函数}\label{fig:sec1.2-23}}
          {\somefigure}
\figurebox[1.2\FBwidth]{\caption{反余弦函数}\label{fig:sec1.2-24}}
          {\somefigure}
\end{minipage}\qquad
\begin{minipage}{.4\hsize}
\figurebox[1.2\FBwidth]{\caption{反正切函数}\label{fig:sec1.2-25}}
          {\somefigure}
\figurebox[1.2\FBwidth]{\caption{反余切函数}\label{fig:sec1.2-26}}
          {\somefigure}
\end{minipage}
\end{floatrow}
\end{figure}

\begin{example}
作函数~$y=1+2\sin\pi x$~的图象。
\end{example}
\begin{solution}
注意到
\[
  2\sin\pi(x+2)=2\sin\pi x,
\]
可知函数~$2\sin\pi x$~是以~$2$~为周期的周期函数,振幅为~$2$,因此可以画出~$2\sin\pi x$~一个周期的图形(见\ref{fig:sec1.2-27}),%
然后将次图象再往上平移~$1$~个单位,即为所有函数的图象(见\ref{fig:sec1.2-28})。
\end{solution}

\begin{figure}
\begin{floatrow}[2]
\figurebox{\caption{}\label{fig:sec1.2-27}}
          {\somefigure}
\figurebox{\caption{}\label{fig:sec1.2-28}}
          {\somefigure}
\end{floatrow}
\end{figure}

\begin{example}
  作函数~$y=\sin\Paren[\Big]{x+\dfrac\pi4}$~的图象。
\end{example}

把这个函数与~$y=\sin x$~比较,这两个函数振幅相同,周期相同,函数~$y=\sin\Paren[\Big]{x+\dfrac\pi4}$~在~$x=-\dfrac\pi4$~处
的值正好等于函数~$y=\sin x$~在~$x=0$~处的值,所以函数~$y=\sin\Paren[\Big]{x+\dfrac\pi4}$~的图象,可以由函数~$y=\sin x$~的
图象往左平移~$\dfrac\pi4$~而得到(见\ref{fig:sec1.2-29})。

\begin{figure}
\begin{floatrow}[3]
\figurebox{\caption{}\label{fig:sec1.2-29}}
          {\somefigure}
\figurebox{\caption{}\label{fig:sec1.2-30}}
          {\somefigure}
\figurebox{\caption{}\label{fig:sec1.2-31}}
          {\somefigure}
\end{floatrow}
\end{figure}

\begin{example}\label{ex:sec1.3-4}
证明,$y_n=\cos(n\arccos x)$~是~$[-1,1]$~上的~$n$~次代数多项式。
\end{example}
\begin{proof}
容易看出~$y_0=1$,
\[
  y_1=\cos(\arccos x)=x。
\]

若结论对~$k=0,1,\dotsc,n-1$~已成立,要证~$k=n$~时也成立。由恒等式
\begin{align*}
y_n+y_{n+2}
&=\cos(n\arccos x)+\cos\Parenb{(n-2)\arccos x}\\
&=2\cos\Parenb{(n-1)\arccos x}\cos(\arccos x)\\
&=2x\cdot y_{n-1}
\end{align*}
可得
\[
  y_n=2x\cdot y_{n-1}-y_{n-2},\quad n\geq2,
\]
由归纳假设,$y_{n-1}$~和~$y_{n-2}$~分别为~$n-1$~次和~$n-2$~次代数多项式,所以~$y_n$~是~$n$~次代数多项式。
\end{proof}

\ref{ex:sec1.3-4}~中多项式~$y_n$~称为~\emph{Чебышёв~多项式}。

\begin{example}
设~$f(x)=\arctan x$,~$g(x)=\tan x$,确定复合函数~$f\Paren{g(x)}$~与~$g\Paren{f(x)}$,并作出它们的图象。
\end{example}
\begin{proof}
当~$g(x)$~在~$f$~的值域~$\Paren[\Big]{-\dfrac\pi2,\dfrac\pi2}$~上考虑时,$g$~与~$f$~互为反函数,所以
\[
  g\Parenb{f(x)}\equiv x,\quad x\in\FR,
\]
图象为\ref{fig:sec1.2-30}。

为了确定~$f\Parenb{g(x)}$,注意在~$g$~的定义域上,$f$~不是~$g$~的反函数,但由~$g$~的周期性可证~$f\comp g$~的周期性。%
因为~$g(x+\pi)=g(x)$,所以
\[
  f\Parenb{g(x+\pi)}=f\Parenb{g(x)},
\]
即~$f\Parenb{g(x)}$~也是周期为~$\pi$~的周期函数,只要作出函数在一个周期上的图象即可。当限制自变量
在~$\Paren[\Big]{-\dfrac\pi2,\dfrac\pi2}$~区间时,$f$~是~$g$~的反函数,由反函数定义
\[
  f\Parenb{g(x)}=x,\quad x\in\Paren[\Big]{-\dfrac\pi2,\dfrac\pi2},
\]
所以函数~$f\Parenb{g(x)}$~的图象(见\ref{fig:sec1.2-31})为无穷多条平行的直线段。利用取整函数的记号,可以
统一表达式为
\[
f\Parenb{g(x)}=x-\Floor[\Big]{\frac1\pi\Paren[\Big]{x+\frac\pi2}}\pi,\quad x\neq\Paren[\Big]{k+\frac12}\pi,
\]
上式中,$k=0,\pm1,\pm2,\dotsc$。
\end{proof}


\begin{exercise*}
\item\relax
\begin{exlist}
  \item 有一个身高为~$a$~的人,在离路灯杆~$b$~处沿一直线以匀速~$c$~在路灯下行走,设路灯高为~$h\,(h>a)$,确定他的头影的轨迹;
  \item 如果这个人沿着曲线~$y=f(x)$~行进,重新确定他的头影的轨迹。
\end{exlist}
\item\relax
\begin{exlist}
  \item 在同一坐标系上画出~$y_1=\sin x$~和~$y_2=\dfrac2\pi x$~在~$x\in\Brack[\Big]{0,\dfrac\pi2}$~部分~的图象,并
        观察~$y_1,y_2$~的图象位置关系;
  \item 通过以上观察证明,在非钝角~$\triangle\mathit{ABC}$~中,有~$\sin A+\sin B+\sin C>2$。
\end{exlist}
\item 证明~$f(x)=\sin x+\cos\sqrt 2x$~不是一个周期函数。
\item 设~$R(x)$~为一有理函数。证明,
\begin{exlist}
  \item 若~$R(-x)=R(x)$,则~$R(x)=R_1(x^2)$,这里~$R_1$~为某一有理函数;
  \item 若~$R(-x)=-R(x)$,则~$R(x)=x R_2(x^2)$,这里~$R_2$~为某一有理函数。
\end{exlist}
\item 设~$f(x)$~在~$\FR$~上有定义,若~$f\Parenb{f(x)}$~存在唯一的不动点,证明~$f(x)$~也存在唯一的不动点。
\item 设~$f(x)$~在~$\FR$~上有定义,若~$f\Parenb{f(x)}$~有且仅有两个相异的不动点~$a,b$,证明只存在下列两种情况:
\begin{exlistcols}
  \item $a$~和~$b$~都是~$f$~的不动点;
  \item $f(a)=b$,~$f(b)=a$。
\end{exlistcols}
\end{exercise*}




\endinput
%%
%% End of file `MAChapter1.tex'.