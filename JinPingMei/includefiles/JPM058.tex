%# -*- coding:utf-8 -*-
%%%%%%%%%%%%%%%%%%%%%%%%%%%%%%%%%%%%%%%%%%%%%%%%%%%%%%%%%%%%%%%%%%%%%%%%%%%%%%%%%%%%%


\chapter{潘金莲打狗伤人\KG 孟玉楼周贫磨镜}


词曰:

\[
愁旋释,还似织;泪暗拭,又偷滴。嗔怒着丫头,强开怀,也只是恨怀千叠。拚则而今已拚了,忘只怎生便忘得!又还倚栏杆,试重听消息。
\]

话说当日西门庆陪亲朋饮酒,吃的酩酊大醉,走入后边孙雪娥房里来。雪娥正顾灶上,看收拾家火,听见西门庆往房里去,慌的两步做一步走。先是郁大姐在他炕上坐的,一面撺掇他往月娘房里和玉箫、小玉一处睡去了。原来孙雪娥也住着一明两暗三间房——一间床房,一间炕房。西门庆也有一年多没进他房中来。听见今日进来,连忙向前替西门庆接衣服,安顿中间椅子上坐的。一面揩抹凉席,收拾铺床,薰香澡牝,走来递茶与西门庆吃了,搀扶上床,脱靴解带,打发安歇。一宿无话。

到次日廿八,乃西门庆正生日。刚烧毕纸,只见韩道国后生胡秀到了门首,下头口。左右禀知西门庆,就叫胡秀到厅上,磕头见了。问他货船在那里,胡秀递上书帐,说道:“韩大叔在杭州置了一万两银子缎绢货物,见今直抵临清钞关,缺少税钞银两,未曾装载进城。”西门庆看了书帐,心内大喜,吩咐棋童看饭与胡秀吃了,教他往乔亲家爹那里见见去。就进来对吴月娘说:“韩伙计货船到了临清,使后生胡秀送书帐上来,如今少不的把对门房子打扫,卸到那里,寻伙计收拾,开铺子发卖。”月娘听了,就说:“你上紧寻着,也不早了。”西门庆道:“如今等应二哥来,我就对他说。”不一时,应伯爵来了。西门庆陪着他在厅上坐,就对他说:“韩伙计杭州货船到了,缺少个伙计发卖。”伯爵就说:“哥,恭喜!今日华诞的日子,货船到,决增十倍之利,喜上加喜。哥若寻卖手,不打紧,我有一相识,却是父交子往的朋友,原是缎子行卖手,连年运拙,闲在家中,今年才四十多岁,眼力看银水是不消说,写算皆精,又会做买卖。此人姓甘,名润,字出身,现在石桥儿巷住,倒是自己房儿。”西门庆道:“若好,你明日叫他见我。”

正说着,只见李铭、吴惠、郑奉三个先来磕头。不一时,杂耍乐工都到了。厢房中打发吃饭。只见答应的节级拿票来回话说:“小的叫唱的,止有郑爱月儿不到。他家鸨子说,收拾了才待来,被王皇亲家人拦往宅里唱去了。小的只叫了齐香儿、董娇儿、洪四儿三个,收拾了便来也。”西门庆听见他不来,便道:“胡说!怎的不来?”便叫过郑奉问:“怎的你妹子我这里叫他不来?果系是被王皇亲家拦了去?”那郑奉跪下便道:“小的另住,不知道。”西门庆道:“他说往王皇亲家唱就罢了?敢量我拿不得来!”便叫玳安儿近前吩咐:“你多带两个排军,就拿我个侍生帖儿,到王皇亲家宅内见你王二老爹,就说我这里请几位客吃酒,郑爱月儿答应下两三日了,好歹放了他来。倘若推辞,连那鸨子都与我锁了,墩在门房儿里。这等可恶!”一面叫郑奉:“你也跟了去。”那郑奉又不敢不去,走出外边来,央及玳安儿说道:“安哥,你进去,我在外边等着罢。一定是王二老爹府里叫,怕不还没去哩。有累安哥,若是没动身,看怎的将就叫他好好的来罢。”玳安道:“若果然往王家去了,等我拿帖儿讨去;若是在家藏着,你进去对他妈说,教他快收拾一答儿来,俺就替他回护两句言语儿,爹就罢了。你每不知道他性格,他从夏老爹宅里定下,你不来,他可知恼了哩。”这郑奉一面先往家中说去,玳安同两个排军、一名节级也随后走来。

且说西门庆打发玳安去了,因向伯爵道:“这个小淫妇儿,这等可恶!在别人家唱,我这里叫他不来。”伯爵道:“小行货子,他晓的甚么?他还不知你的手段哩!”西门庆道:“我倒见他酒席上说话儿伶俐,叫他来唱两日试他,倒这等可恶!”伯爵道:“哥今日拣这四个粉头,都是出类拔萃的尖儿了。”李铭道:“二爹,你还没见爱月儿哩!”伯爵道:“我同你爹在他家吃酒,他还小哩,这几年倒没曾见,不知出落的怎样的了。”李铭道:“这小粉头子,虽故好个身段儿,光是一味妆饰,唱曲也会,怎生赶的上桂姐一半儿。爹这里是那里?叫着敢不来!就是来了,亏了你?还是不知轻重。”正说着,只见胡秀来回话道:“小的到乔爹那边见了来了,伺候老爹示下。”西门庆教陈敬济:“后边讨五十两银子,令书童写一封书,使了印色,差一名节级,明日早起身,一同下去,与你钞关上钱老爹,教他过税之时青目一二。”须臾,陈敬济取了一封银子来交与胡秀,胡秀领了文书并税帖,次日早同起身,不在话下。

忽听喝的道子响,平安来报:“刘公公与薛公公来了。”西门庆忙冠带迎接至大厅,见毕礼数,请至卷棚内,宽去上盖蟒衣,上面设两张交椅坐下。应伯爵在下,与西门庆关席陪坐。薛内相便问:“此位是何人?”西门庆道:“去年老太监会过来,乃是学生故友应二哥。”薛内相道:“却是那快耍笑的应先儿么?”应伯爵欠身道:“老公公还记的,就是在下。”须臾,拿茶上来吃了。只见平安走来禀道:“府里周爷差人拿帖儿来说,今日还有一席,来迟些,叫老爹这里先坐,不须等罢。”西门庆看了帖儿,便说:“我知道了。”薛内相因问:“西门大人,今日谁来迟?”西门庆道:“周南轩那边还有一席,使人来说休要等他,只怕来迟些。”薛内相道:“既来说,咱虚着他席面就是。”

正说话间,王经拿了两个帖儿进来:“两位秀才来了。”西门庆见帖儿上,一个是倪鹏,一个是温必古,就知倪秀才举荐了同窗朋友来了,连忙出来迎接。见都穿着衣巾进来,且不看倪秀才,只见那温必古,年纪不上四旬,生的端庄质朴,落腮胡,仪容谦仰,举止温恭。未知行藏如何,先观动静若是。有几句单道他好:

\[
虽抱不羁之才,惯游非礼之地。功名蹭蹬,豪杰之志已灰;家业凋零,浩然之气先丧。把文章道学,一并送还了孔夫子;将致君泽民的事业及荣身显亲的心念,都撇在东洋大海。和光混俗,惟其利欲是前;随方逐圆,不以廉耻为重。峨其冠,博其带,而眼底旁若无人;阔其论,高其谈,而胸中实无一物。三年叫案,而小考尚难,岂望月桂之高攀;广坐衔杯,遁世无闷,且作岩穴之隐相。
\]
西门庆让至厅上叙礼,每人递书帕二事与西门庆祝寿。交拜毕,分宾主而坐。西门庆道:“久仰温老先生大才,敢问尊号?”温秀才道:“学生贱字日新,号葵轩。”西门庆道:“葵轩老先生。”又问:“贵庠?何经?”温秀才道:“学生不才,府学备数。初学《易经》。一向久仰大名,未敢进拜。昨因我这敝同窗倪桂岩道及老先生盛德,敢来登堂恭谒。”西门庆道:“承老先生先施,学生容日奉拜。只因学生一个武官,粗俗不知文理,往来书柬无人代笔。前者因在敝同僚府上会遇桂岩老先生,甚是称道老先生大才盛德。正欲趋拜请教,不意老先生下降,兼承厚贶,感激不尽。”温秀才道:“学生匪才薄德,谬承过誉。”茶罢,西门庆让至卷棚内,有薛、刘二老太监在座。薛内相道:“请二位老先生宽衣进来。”西门庆一面请宽了青衣,请进里面,各逊让再四,方才一边一位,垂首坐下。

正叙谈间,吴大舅、范千户到了,叙礼坐定。不一时,玳安与同答应的和郑奉都来回话道:“四个唱的都叫来了。”西门庆问:“可是王皇亲那里?”玳安道:“是王皇亲宅内叫,还没起身,小的要拿他鸨子墩锁,他慌了,才上轿,都一答儿来了。”西门庆即出到厅台基上站立。只见四个唱的一齐进来,向西门庆磕下头去。那郑爱月儿穿着紫纱衫儿,白纱挑线裙子。腰肢袅娜,犹如杨柳轻盈;花貌娉婷,好似芙蓉艳丽。正是:

\[
万种风流无处买,千金良夜实难消。
\]
西门庆便向郑爱月儿道:“我叫你,如何不来?这等可恶!敢量我拿不得你来!”那郑爱月儿磕了头起来,一声儿也不言语,笑着同众人一直往后边去了。到后边,与月娘众人都磕了头。看见李桂姐、吴银儿都在跟前,各道了万福,说道:“你二位来的早。”李桂姐道:“我每两日没家去了。”因说:“你四个怎的这咱才来?”董娇儿道:“都是月姐带累的俺们来迟了。收拾下,只顾等着他,白不起身。”郑爱月儿用扇儿遮着脸,只是笑,不做声。月娘便问:“这位大姐是谁家的?”董娇儿道:“娘不知道,他是郑爱香儿的妹子郑爱月儿。才成人,还不上半年光景。”月娘道:“可倒好个身段儿。”说毕,看茶吃了,一面放桌儿,摆茶与众人吃。潘金莲且揭起他裙子,撮弄他的脚看,说道:“你每这里边的样子,只是恁直尖了,不象俺外边的样子趫。俺外边尖底停匀,你里边的后跟子大。”月娘向大妗子道:“偏他恁好胜,问他怎的!”一回又取下他头上金鱼撇杖儿来瞧,因问:“你这样儿是那里打的?”郑爱月儿道:“是俺里边银匠打的。”须臾,摆下茶,月娘便叫:“桂姐、银姐,你陪他四个吃茶。”不一时,六个唱的做一处同吃了茶。李桂姐、吴银儿便向董娇儿四个说:“你每来花园里走走。”董娇儿道:“等我每到后边走走就来。”李桂姐和吴银儿就跟着潘金莲、孟玉楼,出仪门往花园中来。因有人在大卷棚内,就不曾过那边去。只在这边看了回花草,就往李瓶儿房里看官哥儿。官儿心中又有些不自在,睡梦中惊哭,吃不下奶去。李瓶儿在屋里守着不出来。看见李桂姐、吴银儿和孟王楼、潘金莲进来,连忙让坐。桂姐问道:“哥儿睡哩?”李瓶儿道:“他哭了这一日,才睡下了。”玉楼道:“大娘说,请刘婆子来看他看,你怎的不使小厮请去?”李瓶儿道:“今日他爹好日子,明日请他去罢。”

正说话中间,只见四个唱的和西门大姐、小玉走来。大姐道:“原来你每都在这里,却教俺花园内寻你。”玉楼道:“花园内有人,咱们不好去的,瞧了瞧儿就来了。”李桂姐问洪四儿:“你每四个在后边做甚么,这半日才来?”洪四儿道:“俺每在后边四娘房里吃茶来。”潘金莲听了,望着玉楼、李瓶儿笑,问洪四儿:“谁对你说是四娘来?”董娇儿道:“他留俺每在房里吃茶,他每问来:‘还不曾与你老人家磕头,不知娘是几娘?’他便说:‘我是你四娘哩。’”金莲道:“没廉耻的小妇奴才,别人称你便好,谁家自己称是四娘来。这一家大小,谁兴你、谁数你、谁叫你是四娘?汉子在屋里睡了一夜儿,得了些颜色儿,就开起染房来了。若不是大娘房里有他大妗子,他二娘房里有桂姐,你房里有杨姑奶奶,李大姐有银姐在这里,我那屋里有他潘姥姥,且轮不到往你那屋里去哩!”玉楼道:“你还没曾见哩——今日早晨起来,打发他爹往前边去了,在院子里呼张唤李的,便那等花哨起来。”金莲道:“常言道:奴才不可逞,小孩儿不宜哄。”又问小玉:“我听见你爹对你奶奶说,要替他寻丫头。说你爹昨日在他屋里,见他只顾收拾不了,因问他。那小淫妇就趁势儿对你爹说:‘我终日不得个闲收拾屋里,只好晚夕来这屋里睡罢了。’你爹说:‘不打紧,到明日对你娘说,寻一个丫头与你使便了。’——真个有此话?”小玉道:“我不晓的,敢是玉箫听见来?”金莲向桂姐道:“你爹不是俺各房里有人,等闲不往他后边去。莫不俺每背地说他,本等他嘴头子不达时务,惯伤犯人,俺每急切不和他说话。”正说着,绣春拿了茶上来。正吃间,忽听前边鼓乐响动,荆都监众人都到齐了,递酒上座,玳安儿来叫四个唱的,就往前边去了。

那日,乔大户没来。先是杂耍百戏,吹打弹唱。队舞才罢,做了个笑乐院本。割切上来,献头一道汤饭。只见任医官到了,冠带着进来。西门庆迎接至厅上叙礼。任医官令左右,毡包内取出一方寿帕、二星白金来,与西门庆拜寿。说道:“昨日韩明川说,才知老先生华诞。恕学生来迟!”西门庆道:“岂敢动劳车驾,又兼谢盛仪。外日多谢妙药。”彼此拜毕,任医官还要把盏,西门庆辞道:“不消了。”一面脱了大衣,与众人见过,就安在左首第四席,与吴大舅相近而坐。献上汤饭并手下攒盒,任医官谢了,令仆从领下去。四个唱的弹着乐器,在旁唱了一套寿词。西门庆令上席分头递酒。下边乐工呈上揭帖,刘、薛二内相拣了韩湘子度陈半街《升仙会》杂剧。才唱得一折,只见喝道之声渐近。平安进来禀道:“守备府周爷来了。”西门庆慌忙迎接。未曾相见,就先请宽盛服。周守备道:“我来要与四泉把一盏。”薛内相说道:“周大人不消把盏,只见礼儿罢。”于是二人交拜毕,才与众人作揖,左首第三席安下钟箸。下边就是汤饭割切上来,又是马上人两盘点心、两盘熟肉、两瓶酒。周守备谢了,令左右领下去,然后坐下。一面觥筹交错,歌舞吹弹,花攒锦簇饮酒。正是:

\[
舞低杨柳楼头月,歌罢桃花扇底风。
\]

吃至日暮,先是任医官隔门去的早。西门庆送出来,任医官因问:“老夫人贵恙觉好了?”西门庆道:“拙室服了良剂,已觉好些。这两日不知怎的,又有些不自在。明日还望老先生过来看看。”说毕,任医官作辞上马而去。落后又是倪秀才、温秀才起身。西门庆再三款留不住,送出大门,说道:“容日奉拜请教。寒家就在对门收拾一所书院,与老先生居住。连宝眷都搬来,一处方便。学生每月奉上束修,以备菽水之需。”温秀才道:“多承厚爱,感激不尽。”倪秀才道:“此是老先生崇尚斯文之雅意矣。”打发二秀才去了。

西门庆陪客饮酒,吃至更阑方散。四个唱的都归在月娘房内,唱与月娘、大妗子、杨姑娘众人听。西门庆还在前边留下吴大舅、应伯爵,复坐饮酒。看着打发乐工酒饭吃了,先去了。其余席上家火都收了,又吩咐从新后边拿果碟儿上来,教李铭、吴惠、郑奉上来弹唱,拿大杯赏酒与他吃。应伯爵道:“哥今日华诞设席,列位都是喜欢。”李铭道:“今日薛爷和刘爷也费了许多赏赐,落后见桂姐、银姐又出来,每人又递了一包与他。只是薛爷比刘爷年小,快顽些。”不一时,画童儿拿上果碟儿来,应伯爵看见酥油鲍螺,就先拣了一个放在口内,如甘露洒心,入口而化。说道:“倒好吃。”西门庆道:“我的儿,你倒会吃!此是你六娘亲手拣的。”伯爵笑道:“也是我女儿孝顺之心。”说道:“老舅,你也请个儿。”于是拣了一个,放在吴大舅口内。又叫李铭、吴惠、郑奉近前,每人拣了一个赏他。

正饮酒间,伯爵向玳安道:“你去后边,叫那四个小淫妇出来。我便罢了,也叫他唱个儿与老舅听,再迟一回儿,便好去。今日连递酒,他只唱了两套,休要便宜了他。”那玳安不动身,说道:“小的叫了他了,在后边唱与妗子和娘每听哩,便来也。”伯爵道:“贼小油嘴,你几时去来?还哄我。”因叫王经:“你去。”那王经又不动。伯爵道:“我使着你每都不去,等我自去罢。”正说着,只闻一阵香风过,觉有笑声,四个粉头都用汗巾儿答着头出来。伯爵看见道:“我的儿,谁养的你恁乖!搭上头儿,心里要去的情,好自在性儿。不唱个曲儿与俺每听,就指望去?好容易!连轿子钱就是四钱银子,买红梭儿米买一石七八斗,够你家鸨子和你一家大小吃一个月。”董娇儿道:“哥儿,恁便宜衣饭儿,你也入了籍罢了。”洪四儿道:“这咱晚,七八有二更,放了俺每去罢了。”齐香儿道:“俺每明日还要起早,往门外送殡去哩。”伯爵道:“谁家?”齐香儿道:“是房檐底下开门的那家子。”伯爵道:“莫不又是王三官儿家?前日被他连累你那场事,多亏你大爹这里人情,替李桂儿说,连你也饶了。这一遭,雀儿不在那窠儿罢了。”齐香儿笑骂道:“怪老油嘴,汗邪了你,恁胡说。”伯爵道:“你笑话我老?我半边俏!把你这四个小淫妇儿还不够摆布哩。”洪四儿笑道:“哥儿,我看你行头不怎么好,光一味好撇。”伯爵道:“我那儿,到跟前看手段还钱。”又道:“郑家那贼小淫妇儿,吃了糖五老座子儿,白不言语,有些出神的模样,敢记挂着那孤老儿在家里?”董娇儿道:“他刚才听见你说,在这里有些怯床。”伯爵道:“怯床不怯床,拿乐器来,每人唱一套,你每去罢,我也不留你了。”西门庆道:“也罢,你们两个递酒,两个唱一套与他听罢。”齐香儿道:“等我和月姐唱。”当下,郑月儿琵琶,齐香儿弹筝,坐在交床上,歌美韵,放娇声,唱了一套《越调·斗鹌鹑》“夜去明来”。董娇儿递吴大舅酒,洪四儿递应伯爵酒,在席上交杯换盏,倚翠偎红。正是:

\[
舞回明月坠秦楼,歌遏行云迷楚馆。
\]

当下,酒进数巡,歌吟两套,打发四个唱的去了。西门庆还留吴大舅坐,又叫春鸿上来唱了一套南曲,才吩咐棋童备马,拿灯笼送大舅。大舅道:“姐夫不消备马,我同应二哥一路走罢。”西门庆道:“既如此,教棋童打灯笼送到家。”吴大舅与伯爵起身作别。西门庆送至大门首,因和伯爵说:“你明日好歹上心,约会了那甘伙计来见我,批合同。我会了乔亲家,好收拾那边房子卸货。”伯爵道:“哥不消吩咐,我知道。”一面作辞,与吴大舅同行,棋童打着灯笼。吴大舅便问:“刚才姐夫说收拾那里房子?”伯爵道:“韩伙计货船到,他新开个缎子铺,收拾对门房子,叫我替他寻个伙计。”大舅道:“几时开张?咱每亲朋少不的作贺作贺。”须臾,出大街,到了伯爵小胡同口上,吴大舅要棋童:“打灯笼送你应二爹到家。”伯爵不肯,说道:“棋童,你送大舅,我不消灯笼,进巷内就是了。”一面作辞,分路回家。棋童便送大舅去了。

西门庆打发李铭等唱钱去了,回后边月娘房中歇了一夜。到次日,果然伯爵领了甘出身,穿青衣走来拜见,讲说买卖之事。西门庆叫将崔本来会乔大户,那边收拾房子,开张举事。乔大户对崔本说:“将来凡一应大小事,随你亲家爹这边只顾处,不消计较。”当下就和甘伙计批了合同。就立伯爵作保,得利十分为率:西门庆五分,乔大户三分,其余韩道国、甘出身与崔本三分均分。一面修盖土库,装画牌面,待货车到日,堆卸开张。后边又独自收拾一所书院,请将温秀才来作西宾,专修书柬,回答往来士夫。每月三两束修,四时礼物不缺,又拨了画童儿小厮伏侍他。西门庆家中宴客,常请过来陪侍饮酒,俱不必细说。

不觉过了西门庆生辰。第二日早晨,就请了任医官来看李瓶儿,又在对门看着收拾。杨姑娘先家去了,李桂姐、吴银儿还没家去。吴月娘买了三钱银子螃蟹,午间煮了,请大妗子、李桂姐、吴银儿众人围着吃了一回。只见月娘请的刘婆子来看官哥儿,吃了茶,李瓶儿就陪他往前边房里去了。刘婆子说:“哥儿惊了,要住了奶。”又留下几服药。月娘与了他三钱银子,打发去了。孟玉楼、潘金莲和李桂姐、吴银儿、大姐都在花架底下,放小桌儿,铺毡条,同抹骨牌赌酒顽耍。孙雪娥吃众人赢了七八钟酒,不敢久坐,就去了。众人就拿李瓶儿顶缺。金莲又教吴银儿、桂姐唱了一套。当日众姊妹饮酒至晚,月娘装了盒子,相送李桂姐、吴银儿家去了。

潘金莲吃的大醉归房,因见西门庆夜间在李瓶儿房里歇了一夜,早晨又请任医官来看他,恼在心里。知道他孩子不好,进门不想天假其便——黑影中躧了一脚狗屎,到房中叫春梅点灯来看,一双大红缎子鞋,满帮子都展污了。登时柳眉剔竖,星眼圆睁,叫春梅打着灯把角门关了,拿大棍把那狗没高低只顾打,打的怪叫起来。李瓶儿使过迎春来说:“俺娘说,哥儿才吃了老刘的药,睡着了,教五娘这边休打狗罢。”潘金莲坐着,半日不言语。一面把那狗打了一回,开了门放出去,又寻起秋菊的不是来。看着那鞋,左也恼,右也恼,因把秋菊唤至跟前说:“这咱晚,这狗也该打发去了,只顾还放在这屋里做甚么?是你这奴才的野汉子?你不发他出去,教他恁遍地撒屎,把我恁双新鞋儿——连今日才三四日儿——躧了恁一鞋帮子屎。知道我来,你也该点个灯儿出来,你如何恁推聋妆哑装憨儿的?”春梅道:“我头里就对他说,你趁娘不来,早喂他些饭,关到后边院子里去罢。他佯打耳睁的不理我,还拿眼儿瞅着我。”妇人道:“可又来,贼胆大万杀的奴才,我知道你在这屋里成了把头,把这打来不作准。”因叫他到跟前:“瞧,躧的我这鞋上的龌龊!”哄得他低头瞧,提着鞋拽巴,兜脸就是几鞋底子。打的秋菊嘴唇都破了,只顾揾着抹血,忙走开一边。妇人骂道:“好贼奴才,你走了!”教春梅:“与我采过来跪着,取马鞭子来,把他身上衣服与我扯去。好好教我打三十马鞭子便罢,但扭一扭儿,我乱打了不算。”春梅于是扯了他衣裳,妇人教春梅把他手扯住,雨点般鞭子打下来,打的这丫头杀猪也似叫。那边官哥才合上眼儿,又惊醒了。又使了绣春来说:“俺娘上覆五娘,饶了秋菊罢,只怕唬醒了哥哥。”那潘姥姥正\textuni{22C49}在里间炕上,听见打的秋菊叫,一骨碌子爬起来,在旁边劝解。见金莲不依,落后又见李瓶儿使过绣春来说,又走向前夺他女儿手中鞭子,说道:“姐姐少打他两下儿罢,惹得他那边姐姐说,只怕唬了哥哥。为驴扭棍不打紧,倒没的伤了紫荆树。”金莲紧自心里恼,又听见他娘说了这一句,越发心中撺上把火一般。须臾,紫漒了面皮,把手只一推,险些儿不把潘姥姥推了一交。便道:“怪老货,你与我过一边坐着去!不干你事,来劝甚么?甚么紫荆树、驴扭棍,单管外合里应。”潘姥姥道:“贼作死的短寿命,我怎的外合里应?我来你家讨冷饭吃,教你恁顿摔我?”金莲道:“你明日夹着那老\textuni{23B48}走,怕他家拿长锅煮吃了我!”潘姥姥听见女儿这等擦他,走到里边屋里呜呜咽咽哭去了,随着妇人打秋菊。打够二三十马鞭子,然后又盖了十栏杆,打的皮开肉绽,才放出来。又把他脸和腮颊都用尖指甲掐的稀烂。李瓶儿在那边,只是双手握着孩子耳朵,腮边堕泪,敢怒而下敢言。

西门庆在对门房子里,与伯爵、崔本、甘伙计吃了一日酒散了,迳往玉楼房中歇息。到次日,周守备家请吃补生日酒,不在家。李瓶儿见官哥儿吃了刘婆子药不见动静,夜间又着惊唬,一双眼只是往上吊吊的。因那日薛姑子、王姑子家去,走来对月娘说:“我向房中拿出他压被的一对银狮子来,要教薛姑子印造《佛顶心陀罗经》,赶八月十五日岳庙里去舍。”那薛姑子就要拿着走,被孟玉楼在旁说道:“师父你且住,大娘,你还使小厮叫将贲四来,替他兑兑多少分两,就同他往经铺里讲定个数儿来,每一部经多少银子,到几时有,才好。你教薛师父去,他独自一个,怎弄的来?”月娘道:“你也说的是。”一面使来安儿叫了贲四来,向月娘众人作了揖,把那一对银狮子上天平兑了,重四十一两五钱。月娘吩咐,同薛师父往经铺印造经数去了。

潘金莲随即叫孟玉楼:“咱送送两位师父去,就前边看看大姐,他在屋里做鞋哩。”两个携着手儿往前边来。贲四同薛姑子、王姑子去了。金莲与玉楼走出大厅东厢房门首,见大姐正在檐下纳鞋,金莲拿起来看,却是沙绿潞绸鞋面。玉楼道:“大姐,你不要这红锁线子,爽利着蓝头线儿,好不老作些!你明日还要大红提跟子?”大姐道:“我有一双是大红提跟子的。这个,我心里要蓝提跟子,所以使大红线锁口。”金莲瞧了一回,三个都在厅台基上坐的。玉楼问大姐:“你女婿在屋里不在?”大姐道:“他不知那里吃了两盅酒,在屋里睡哩。”孟玉楼便向金莲道:“刚才若不是我在旁边说着,李大姐恁哈帐行货,就要把银子交姑子拿了印经去。经也印不成,没脚蟹行货子藏在那大人家,你那里寻他去?早是我说,叫将贲四来,同他去了。”金莲道:“恁有钱的姐姐,不赚他些儿是傻子,只象牛身上拔一根毛儿。你孩儿若没命,休说舍经,随你把万里江山舍了也成不的。如今这屋里,只许人放火,不许俺每点灯。——大姐听着,也不是别人。偏染的白儿不上色,偏他会那等轻狂使势,大清早晨,刁蹬着汉子请太医看。他乱他的,俺每又不管。每常在人前会那等撇清儿说话:‘我心里不耐烦,他爹要便进我屋里推看孩子,雌着和我睡,谁耐烦!教我就撺掇往别人屋里去了。俺每自恁好罢了,背地还嚼说俺们。’那大姐姐偏听他一面词儿。不是俺每争这个事,怎么昨日汉子不进你屋里去,你使丫头在角门子首叫进屋里?推看孩子,你便吃药,一径把汉子作成和吴银儿睡了一夜,一迳显你那乖觉,叫汉子喜欢你,那大姐姐就没的话说了。昨日晚夕,人进屋里躧了一脚狗屎,打丫头赶狗,也嗔起来,使丫头过来说,唬了他孩子了。俺娘那老货,又不知道,走来劝甚么的驴扭棍伤了紫荆树。我恼他那等轻声浪气,叫我墩了他两句,他今日使性子家去了。——去了罢!教我说,他家有你这样穷亲戚也不多,没你也不少。”玉楼笑道:“你这个没训教的子孙,你一个亲娘母儿,你这等讧他!”金莲道:“不是这等说。——恼人的肠子,单管黄猫黑尾,外合里应,只替人说话。吃人家碗半,被人家使唤。得不的人家一个甜头儿,千也说好,万也说好。——想着迎头儿养了这个孩子,把汉子调唆的生根也似的,把他便扶的正正儿的,把人恨不的躧到泥里头还躧。今日恁的天也有眼,你的孩儿也生出病来了。”

正说着,只见贲四往经铺里交回银子,来回月娘话,看见玉楼、金莲和大姐都在厅台基上坐的,只顾在仪门外立着,不敢进来。来安走来说道:“娘每闪闪儿,贲四来了。”金莲道:“怪囚根子,你叫他进去,不是才乍见他来?”来安儿说了,贲四低着头,一直后边见月娘、李瓶儿,说道:“银子四十一两五钱,眼同两个师父交付与翟经儿家收了。讲定印造绫壳《陀罗》五百部,每部五分;绢壳经一千部,每部三分。共该五十五两银子。除收过四十一两五钱,还找与他十三两五钱。准在十四日早抬经来。”李瓶儿连忙向房里取出一个银香球来,叫贲四上天平兑了,十五两。李瓶儿道:“你拿了去,除找与他,别的你收着,换下些钱,到十五日庙上舍经,与你们做盘缠就是了,省的又来问我要。”贲四于是拿了香球出来,李瓶儿道:“四哥,多累你。”贲四躬着身说道:“小人不敢。”走到前边,金莲、玉楼又叫住问他:“银子交付与经铺了?”贲四道:“已交付明白。共一千五百部经,共该五十五两银子,除收过四十一两五钱,刚才六娘又与了这件银香球。”玉楼、金莲瞧了瞧,没言语,贲四便回家去了。玉楼向金莲说道:“李大姐象这等都枉费了钱。他若是你的儿女,就是榔头也桩不死;他若不是你儿女,莫说舍经造像,随你怎的也留不住他。信着姑子,甚么茧儿干不出来!”

两个说了一回,都立起来。金莲道:“咱每往前边大门首走走去。”因问大姐:“你去不去?”大姐道:“我不去。”潘金莲便拉着玉楼手儿,两个同来到大门里首站立。因问平安儿:“对门房子都收拾了?”平安道:“这咱哩?昨日爹看着就都打扫干净了。后边楼上堆货,昨日教阴阳来破土,楼底下还要装厢房三间,土库搁缎子,门面打开,一溜三间,都教漆匠装新油漆,在出月开张。”玉楼又问:“那写书的温秀才,家小搬过来了不曾?”平安道,“从昨日就过来了。今早爹吩咐,把后边那一张凉床拆了与他,又搬了两张桌子、四张椅子与他坐。”金莲道:“你没见他老婆怎的模样儿?”平安道:“黑影子坐着轿子来,谁看见他来!”

正说着,只见远远一个老头儿,斯琅琅摇着惊闺叶过来。潘金莲便道:“磨镜子的过来了。”教平安儿:“你叫住他,与俺每磨磨镜子。我的镜子这两日都使的昏了,吩咐你这囚根子,看着过来再不叫!俺每出来站了多大回,怎的就有磨镜子的过来了?”那平安一面叫住磨镜老儿,放下担儿,金莲便问玉楼道:“你要磨,都教小厮带出来,一答儿里磨了罢。”于是使来安儿:“你去我屋里,问你春梅姐讨我的照脸大镜子、两面小镜子儿,就把那大四方穿衣镜也带出来,教他好生磨磨。”玉楼吩咐来安:“你到我屋里,教兰香也把我的镜子拿出来。”那来安儿去不多时,两只手提着大小八面镜于,怀里又抱着四方穿衣镜出来。金莲道:“臭小囚儿,你拿不了,做两遭儿拿,如何恁拿出来?一时叮当了我这镜子怎了?”玉楼道:“我没见你这面大镜子,是那里的?”金莲道:“是人家当的,我爱他且是亮,安在屋里,早晚照照。”因问:“我的镜子只三面?”玉楼道:“我大小只两面。”金莲道:“这两面是谁的?”来安道:“这两面是春梅姐的,捎出来也叫磨磨。”金莲道:“贼小肉儿,他放着他的镜子不使,成日只挝着我的镜子照,弄的恁昏昏的。”共大小八面镜于,交付与磨镜老叟,教他磨。当下绊在坐架上,使了水银,那消顿饭之间,都净磨的耀眼争光。妇人拿在手内,对照花容,犹如一汪秋水相似。有诗为证:

\[
莲萼菱花共照临,风吹影动碧沉沉。
一池秋水芙蓉现,好似姮娥傍月阴。
\]

妇人看了,就付与来安儿收进去。玉楼便令平安,问铺子里傅伙计柜上要五十文钱与磨镜的。那老子一手接了钱,只顾立着不去。玉楼教平安问那老子:“你怎的不去?敢嫌钱少?”那老子不觉眼中扑簌簌流下泪来,哭了。平安道:“俺当家的奶奶问你怎的烦恼。”老子道:“不瞒哥哥说,老汉今年痴长六十一岁,在前丢下个儿子,二十二岁尚未娶妻,专一浪游,不干生理。老汉日逐出来挣钱养活他。他又不守本分,常与街上捣子耍钱。昨日惹了祸,同拴到守备府中,当土贼打回二十大棍。归来把妈妈的裙袄都去当了。妈妈便气了一场病,打了寒,睡在炕上半个月。老汉说他两句,他便走出来不往家去,教老汉逐日抓寻他,不着个下落。待要赌气不寻他,老汉恁大年纪,止生他一个儿子,往后无人送老;有他在家,见他不成人,又要惹气。似这等,乃老汉的业障。有这等负屈衔冤,各处告诉,所以泪出痛肠。”玉楼叫平安儿:“你问他,你这后娶婆儿今年多大年纪了?”老子道:“他今年五十五岁了,男女花儿没有,如今打了寒才好些,只是没将养的,心中想块腊肉儿吃。老汉在街上恁问了两三日,白讨不出块腊肉儿来。甚可嗟叹人子。”玉楼道:“不打紧处,我屋里抽屉内有块腊肉儿哩。”即令来安儿:“你去对兰香说,还有两个饼锭,教他拿与你来。”金莲叫:“那老头子,问你家妈妈儿吃小米儿粥不吃?”老汉子道:“怎的不吃!那里有?可知好哩。”金莲也叫过来安儿来:“你对春梅说,把昨日你姥姥捎来的新小米儿量二升,就拿两根酱瓜儿出来,与他妈妈儿吃。”那来安去不多时,拿出半腿腊肉、两个饼锭、二升小米、两个酱瓜儿,叫道:“老头子过来,造化了你!你家妈妈子不是害病想吃,只怕害孩子坐月子,想定心汤吃。”那老子连忙双手接了,安放在担内,望着玉楼、金莲唱了个喏,扬长挑着担儿,摇着惊闺叶去了。平安道:“二位娘不该与他这许多东西,被这老油嘴设智诓的去了。他妈妈子是个媒人,昨日打这街上走过去不是,几时在家不好来?”金莲道:“贼囚,你早不说做甚么来?”平安道:“罢了,也是他造化。可可二位娘出来看见叫住他,照顾了他这些东西去了。”正是:

\[
闲来无事倚门楣,恰见惊闺一老来。
不独纤微能济物,无缘滴水也难为。
\]
