%# -*- coding:utf-8 -*-
%%%%%%%%%%%%%%%%%%%%%%%%%%%%%%%%%%%%%%%%%%%%%%%%%%%%%%%%%%%%%%%%%%%%%%%%%%%%%%%%%%%%%


\chapter{招宣府初调林太太\KG 丽春院惊走王三官}


词曰:

\[
香烟袅,罗帏锦帐风光好。风光好,金钗斜軃,凤颠鸾倒。
恍疑身在蓬莱岛,邂逅相逢缘不小。缘不小,最开怀处,蛾眉淡扫。
\]

话说玳安同文嫂儿到家,平安说:“爹在对门房子里。”进去禀报。西门庆正在书房中和温秀才坐的,见玳安,随即出来,小客位内坐下。玳安道:“文嫂儿叫了来,在外边伺候。”西门庆即令:“叫他进来。”那文嫂悄悄掀开暖帘,进入里面,向西门庆磕头。西门庆道:“文嫂,许久不见你。”文嫂道:“小媳妇有。”西门庆道:“你如今搬在那里住了?”文嫂道:“小媳妇因不幸为了场官司,把旧时那房儿弃了,如今搬在大南首王家巷住哩。”西门庆吩咐道:“起来说话。”那文嫂一面站立在旁边。西门庆令左右都出去,那平安和画童都躲在角门外伺候,只玳安儿影在帘儿外边听。西门庆因问:“你常在那几家大人家走跳?”文嫂道:“就是大街皇亲家,守备府周爷家,乔皇亲、张二老爹、夏老爹家,都相熟。”西门庆道:“你认的王招宣府里不认的?”文嫂道:“是小媳妇定门主顾,太太和三娘常照顾我的花翠。”西门庆道:“你既相熟,我有桩事儿央及你,休要阻了我。”向袖中取出五两一锭银子与他,悄悄和他说:“如此这般,你怎的寻个路儿把他太太吊在你那里,我会他会儿,我还谢你。”那文嫂听了,哈哈笑道:“是谁对爹说来?你老人家怎的晓得来?”西门庆道:“常言:人的名儿,树的影儿。我怎得不知道!”文嫂道:“若说起我这太太来,今年属猪,三十五岁,端的上等妇人,百伶百俐,只好象三十岁的。他虽是干这营生,好不干的细密!就是往那里去,许多伴当跟随,径路儿来,迳路儿去。三老爹在外为人做人,他怎在人家落脚?——这个人传的讹了。倒是他家里深宅大院,一时三老爹不在,藏掖个儿去,人不知鬼不觉,倒还许。若是小媳妇那里,窄门窄户,敢招惹这个事?就是爹赏的这银子,小媳妇也不敢领去。宁可领了爹言语,对太太说就是了。”西门庆道:“你不收,便是推托,我就恼了。事成,我还另外赏几个绸缎你穿。”文嫂道:“愁你老人家没有也怎的?上人着眼觑,就是福星临。”磕了个头,把银子接了,说道:“待小媳妇悄悄对太太说,来回你老人家。”西门庆道:“你当件事干,我这里等着。你来时,只在这里来就是了,我不使小厮去了。”文嫂道:“我知道。不在明日,只在后日,随早随晚,讨了示下就来了。”一面走出来。玳安道:“文嫂,随你罢了,我只要你一两银子,也是我叫你一场。你休要独吃。”文嫂道:“猢狲儿隔墙掠筛箕,还不知仰着合着哩。”于是出门骑上驴子,他儿子笼着,一直去了。西门庆和温秀才坐了一回,良久,夏提刑来,就冠冕着同往府里罗同知——名唤罗万象那里吃酒去了。直到掌灯以后才来家。

且说文嫂儿拿着西门庆五两银子,到家欢喜无尽,打发会茶人散了。至后晌时分,走到王招宣府宅里,见了林太太,道了万福。林氏便道:“你怎的这两日不来看看我?”文嫂便把家中会茶,赶腊月要往顶上进香一节告诉林氏。林氏道:“你儿子去,你不去罢了。”文嫂儿道:“我如何得去?只教文(纟堂)代进香去罢了。”林氏道:“等临期,我送些盘缠与你。”文嫂便道:“多谢太太布施。”说毕,林氏叫他近前烤火,丫鬟拿茶来吃了。这文嫂一面吃了茶,问道:“三爹不在家了?”林氏道:“他又有两夜没回家,只在里边歇哩。逐日搭着这伙乔人,只眠花卧柳,把花枝般媳妇儿丢在房里,通不顾,如何是好?”文嫂又问:“三娘怎的不见?”林氏道:“他还在房里未出来哩。”这文嫂见无人,便说道:“不打紧,太太宽心。小媳妇有个门路儿,管就打散了这伙人,三爹收心,也再不进院去了。太太容小媳妇,便敢说;不容便不敢说。”林氏道:“你说的话儿,那遭儿我不依你来?你有话只顾说不妨。”这文嫂方说道:“县门前西门大老爹,如今见在提刑院做掌刑千户,家中放官吏债,开四五处铺面:缎子铺、生药铺、绸绢铺、绒线铺,外边江湖又走标船,扬州兴贩盐引,东平府上纳香蜡,伙计主管约有数十。东京蔡太师是他干爷,朱太尉是他卫主,翟管家是他亲家,巡抚巡按都与他相交,知府知县是不消说。家中田连阡陌,米烂成仓,身边除了大娘子——乃是清河左卫吴千户之女,填房与他为继室——只成房头、穿袍儿的,也有五六个。以下歌儿舞女,得宠侍妾,不下数十。端的朝朝寒食,夜夜元宵。今老爹不上三十一二年纪,正是当年汉子,大身材,一表人物。也曾吃药养龟,惯调风情;双陆象棋,无所不通;蹴踘打毬,无所不晓;诸子百家,拆白道字,眼见就会。端的击玉敲金,百怜百俐。闻知咱家乃世代簪缨人家,根基非浅,又见三爹在武学肄业,也要来相交,只是不曾会过,不好来的。昨日闻知太太贵诞在迩,又四海纳贤,也一心要来与太太拜寿。小媳妇便道:‘初会,怎好骤然请见的。待小的达知老太太,讨个示下,来请老爹相见。’今老太太不但结识他来往相交,只央浼他把这干人断开了,须玷辱不了咱家门户。”林氏被文嫂这篇话说的心中迷留摸乱,情窦已开,便向文嫂儿较计道:“人生面不熟,怎好遽然相见?”文嫂道:“不打紧,等我对老爹说。只说太太先央浼他要到提刑院递状,告引诱三爹这起人,预先请老爹来私下先会一会,此计有何不可?”说得林氏心中大喜,约定后日晚夕等候。

这文嫂讨了妇人示下归家,到次日饭时,走来西门庆宅内。西门庆正在对门书院内坐的,忽玳安报:“文嫂来了。”西门庆听了,即出小客位,令左右放下帘儿。良久,文嫂进入里面,磕了头,玳安知局,就走出来了。文嫂便把怎的说念林氏:“夸奖老爹人品家道,怎样结识官府,又怎的仗义疏财,风流博浪,说得他千肯万肯,约定明日晚间,三爹不在家,家中设席等候。假以说人情为由,暗中相会。”西门庆听了,满心欢喜。又令玳安拿了两匹绸缎赏他。文嫂道,“爹明日要去,休要早了。直到掌灯,街上人静时,打他后门首扁食巷中——他后门旁有个住房的段妈妈,我在他家等着。爹只使大官儿弹门,我就出来引爹入港,休令左近人知道。”西门庆道:“我知道。你明日先去,不可离寸地,我也依期而至。”说毕,文嫂拜辞出门,又回林氏话去了。

西门庆那日,归李娇儿房中宿歇,一宿无话。巴不到次日,培养着精神。午间,戴着白忠靖巾,便同应伯爵骑马往谢希大家吃生日酒。席上两个唱的。西门庆吃了几杯酒,约掌灯上来,就逃席走出来了。骑上马,玳安、琴童两个小厮跟随。那时约十九日,月色朦胧,带着眼纱由大街抹过,迳穿到扁食巷王招宣府后门来。那时才上灯一回,街上人初静之后。西门庆离他后门半舍,把马勒住,令玳安先弹段妈妈家门。原来这妈妈就住着王招宣家后房,也是文嫂举荐,早晚看守后门,开门闭户。但有入港,在他家落脚做窝。文嫂在他屋里听见弹门,连忙开门。见西门庆来了,一面在后门里等的西门庆下了马,除去眼纱儿,引进来,吩咐琴童牵了马,往对门人家西首房檐下那里等候,玳安便在段妈妈屋里存身。这文嫂一面请西门庆入来,便把后门关了,上了栓,由夹道进内。转过一层群房,就是太太住的五间正房,旁边一座便门闭着。这文嫂轻敲敲门环儿,原来有个听头。少顷,见一丫鬟出来,开了双扉。文嫂导引西门庆到后堂,掀开帘拢,只见里面灯烛荧煌,正面供养着他祖爷太原节度颁阳郡王王景崇的影身图:穿着大红团袖,蟒衣玉带,虎皮交椅坐着观看兵书。有若关王之像,只是髯须短些。迎门朱红匾上写着“节义堂”三字,两壁隶书一联:“传家节操同松竹,报国勋功并斗山。”西门庆正观看之间,只听得门帘上铃儿响,文嫂从里拿出一盏茶来与西门庆吃。西门庆便道:“请老太太出来拜见。”文嫂道:“请老爹且吃过茶着,刚才禀过太太知道了。”不想林氏悄悄从房门帘里望外边观看,见西门庆身材凛凛,一表人物,头戴白缎忠靖冠,貂鼠暖耳,身穿紫羊绒鹤氅,脚下粉底皂靴,就是个——

\[
富而多诈奸邪辈,压善欺良酒色徒。
\]

林氏一见满心欢喜,因悄悄叫过文嫂来,问他戴的孝是谁的。文嫂道:“是他第六个娘子的孝,新近九月间没了不多些时。饶少杀,家中如今还有一巴掌人儿。他老人家,你看不出来?出笼儿的鹌鹑——也是个快斗的。”这婆娘听了,越发欢喜无尽。文嫂催逼他出去,妇人道:“我羞答答怎好出去?请他进来见罢。”文嫂一面走出来,向西门庆说:“太太请老爹房内拜见哩。”于是忙掀门帘,西门庆进入房中,但见帘幙垂红,毡毺铺地,麝兰香霭,气暖如春。绣榻则斗帐云横,锦屏则轩辕月映。妇人头上戴着金丝翠叶冠儿,身穿白绫宽绸袄儿,沉香色遍地金妆花缎子鹤氅,大红宫锦宽襕裙子,老鹳白绫高底鞋儿。就是个绮阁中好色的娇娘,深闺内施\textuni{23B48}的菩萨。有诗为证:

\[
云浓脂腻黛痕长,莲步轻移兰麝香。
醉后情深归绣帐,始知太太不寻常。
\]

西门庆一见便躬身施礼,说道:“请太太转上,学生拜见。”林氏道:“大人免礼罢。”西门庆不肯,就侧身磕下头去拜两拜。妇人亦叙礼相还。拜毕,西门庆正面椅子上坐了,林氏就在下边梳背炕沿斜佥相陪。文嫂又早把前边仪门闭上了,再无一个仆人在后边。三公子那边角门也关了。一个小丫鬟名唤芙蓉,拿茶上来,林氏陪西门庆吃了茶,文嫂就在旁说道:“太太久闻老爹执掌刑名,敢使小媳妇请老爹来央烦桩事儿,未知老爹可依允不依?”西门庆道:“不知老太太有甚事吩咐?”林氏道:“不瞒大人说,寒家虽世代做了这招宣,不幸夫主去世年久,家中无甚积蓄。小儿年幼优养,未曾考袭,如今虽入武学肄业,年幼失学。外边有几个奸诈不良的人,日逐引诱他在外飘酒,把家事都失了。几次欲待要往公门诉状,诚恐抛头露面,有失先夫名节。今日敢请大人至寒家诉其衷曲,就如同递状一般。望乞大人千万留情把这干人怎生处断开了,使小儿改过自新,专习功名,以承先业,实出大人再造之恩,妾身感激不浅,自当重谢。”西门庆道:“老太太怎生这般说。尊家乃世代簪缨,先朝将相。令郎既入武学,正当努力功名,承其祖武,不意听信游食所哄,留连花酒,实出少年所为。太太既吩咐,学生到衙门里,即时把这干人处分惩治,庶可杜绝将来。”这妇人听了,连忙起身,向西门庆道了万福,说道:“容日妾身致谢大人。”西门庆道:“你我一家,何出此言。”

说话之间,彼此眉目顾盼留情。不一时,文嫂放桌儿摆上酒来,西门庆故意辞道:“学生初来进谒,倒不曾送礼来,如何反承老太太盛情留坐!”林氏道:“不知大人下降,没作整备。寒天聊具一杯水酒,表意面已。”丫鬟筛上酒来,端的金壶斟美酿,玉盏贮佳肴。林氏起身捧酒,西门庆亦下席道:“我当先奉老太太一杯。”文嫂儿在旁插口说道:“老爹且不消递太太酒。这十一月十五日是太太生日,那日送礼来与太太祝寿就是了。”西门庆道:“阿呀!早时你说。今日是初九,差六日。我在下一定来与太太登堂拜寿。”林氏笑道:“岂敢动劳大人!”须臾,大盘大碗,就是十六碗美味佳肴,旁边绛烛高烧,下边金炉添火,交杯一盏,行令猜枚,笑雨嘲云。

酒为色胆。看看饮至莲漏已沉、窗月倒影之际,一双竹叶穿心,两个芳情已动。文嫂已过一边,连次呼酒不至。西门庆见左右无人,渐渐促席而坐,言颇涉邪,把手捏腕之际,挨肩擦膀之间。初时戏搂粉项,妇人则笑而不言;次后款启朱唇,西门庆则舌吐其口,鸣咂有声,笑语密切。妇人于是自掩房门,解衣松佩,微开锦帐,轻展绣衾,鸳枕横床,凤香薰被,相挨玉体,抱搂酥胸。原来西门庆知妇人好风月,家中带了淫器包在身边,又服了胡僧药。妇人摸见他阳物甚大,西门庆亦摸其牝户,彼此欢欣,情兴如火。展猿臂,不觉蝶浪蜂狂;跷玉腿,那个羞云怯雨!正是:

\[
纵横惯使风流阵,那管床头堕玉钗。
\]

西门庆当下竭平生本事,将妇人尽力盘桓了一场。缠至更深天气,方才精泄。妇人则发乱钗横,花憔柳困。两个并头交股,搂抱片时,起来穿衣。妇人款剔银灯,开了房门,照镜整容,呼丫鬟捧水净手。复饮香醪,再劝美酌。三杯之后,西门庆告辞起身,妇人挽留不已,叮咛频嘱。西门庆躬身领诺,谢扰不尽,相别出门。妇人送到角门首回去了。文嫂先开后门,呼唤玳安、琴童牵马过来,骑上回家。街上已喝号提铃,更深夜静,但见一天霜气,万籁无声。西门庆回家,一宿无话。

到次日,西门庆到衙门中发放已毕,在后厅叫过该地方节级缉捕,吩咐如此这般:“王招宣府里三公子,看有甚么人勾引他,院中在何人家行走,即查访出名字来,报我知道。”因向夏提刑说:“王三公子甚不学好,昨日他母亲再三央人来对我说,倒不关他儿子事,只被这干光棍勾引他。今若不痛加惩治,将来引诱坏了人家子弟。”夏提刑道:“长官所见不错,必该治他。”节级缉捕领了西门庆钧语,当日即查访出各人名姓来,打了事件,到后晌时分来西门庆宅内呈递揭帖。西门庆见上面有孙寡嘴、祝实念、小张闲、聂钺儿、向三、于宽、白回子,乐妇是李桂姐、秦玉芝儿。西门庆取过笔来,把李桂姐、秦玉芝儿并老孙、祝实念名字都抹了,吩咐:“这小张闲等五个光棍,即与我拿了,明日早带到衙门里来。”众公人应诺下去。至晚,打听王三官众人都在李桂姐家吃酒踢行头,都埋伏在房门首。深更时分,刚散出来,众公人把小张闲、聂钺、于宽、白回子、向三五人都拿了。孙寡嘴与祝实念扒李桂姐后房去了,王三官藏在李桂姐床底下,不敢出来。桂姐一家唬的捏两把汗,更不知是那里的人,乱央人打听实信。王三官躲了一夜不敢出来。李家鸨子又恐怕东京下来拿人,到五更时分,撺掇李铭换了衣服,送王三官来家。

节级缉捕把小张闲等拿在听事房吊了一夜。到次日早晨,西门庆进衙门与夏提刑升厅,两边刑杖罗列,带人上去。每人一夹二十大棍,打得皮开肉绽,鲜血迸流,响声震天,哀号恸地。西门庆嘱咐道:“我把你这起光棍,专一引诱人家子弟在院飘风,不守本分,本当重处,今姑从轻责你这几下儿。再若犯在我手里,定然枷号,在院门首示众!”喝令左右:“叉下去!”众人望外,金命水命,走投无命。

两位官府发放事毕,退厅吃茶。夏提刑因说起:“昨日京中舍亲崔中书那里书来,说衙门中考察本上去了,还未下来哩。今日会了长官,咱倒好差人往怀庆府同僚林苍峰那里,打听打听消息去。他那里临京近。”西门庆道:“长官所见甚明。”即唤走差的上来吩咐:“与你五钱银子盘缠,即拿俺两个拜帖,到怀庆府提刑林千户老爹那里,打听京中考察本示下,看经历司行下照会来不曾。务要打听的实,来回报。”那人领了银子、拜帖,又到司房结束行装,讨了匹马,长行去了。两位官府才起身回家。

却说小张闲等从提刑院打出来,走在路上各人思想,更不料今日受这场亏是那里药线,互相埋怨。小张闲道:“莫不还是东京那里的消息?”白回子道:“不是。若是那里消息,怎肯轻饶素放?”常言说得好:乖不过唱的,贼不过银匠,能不过架儿。聂钺儿一口就说道:“你每都不知道,只我猜得着。此一定是西门官府和三官儿上气,嗔请他表子,故拿俺每煞气。正是:龙斗虎伤,苦了小獐。”小张闲道:“列位倒罢了,只是苦了我在下了。孙寡嘴、祝麻子都跟着,只把俺每顶缸。”于宽道:“你怎的说浑话?他两个是他的朋友,若拿来跪在地下,他在上面坐着,怎生相处?”小张闲道:“怎的不拿老婆?”聂钺道:“两个老婆,都是他心上人。李家桂姐是他的表子,他肯拿来!也休怪人,是俺每的晦气,偏撞在这网里。才夏老爹怎生不言语,只是他说话?这个就见出情弊来了。如今往李桂姐家寻王三官去!白为他打了这一屁股疮来不成?便罢了,就问他要几两银子盘缠,也不吃家中老婆笑话。”于是迳入勾栏,见李桂姐家门关的铁桶相似。叫了半日,丫头隔门问是谁,小张闲道:“是俺每,寻三官儿说话。”丫头回说:“他从那日半夜就回家去了,不在这里。无人在家中,不敢开门。”这众人只得回来,到王招宣府内,迳入他客位里坐下。王三官听见众人来寻他,唬得躲在房里不敢出来。半日,使出小厮永定儿来说:“俺爹不在家了。”众人道:“好自在性儿!不在家了,往那里去了?叫不将来!”于宽道:“实和你说了罢,休推睡里梦里。刚才提刑院打了俺每,押将出来。如今还要他正身见官去哩!”搂起腿来与永定瞧,教他进里面去说:“为你打俺每,有甚要紧!”一个个都躺在凳上声疼叫喊。

那王三官儿越发不敢出来,只叫:“娘,怎么样儿?如何救我则可。”林氏道:“我女妇人家,如何寻人情去救得?”求了半日,见外边众人等得急了,要请老太太说话。那林氏又不出去,只隔着屏风说道:“你每略等他等,委的在庄上,不在家了。我这里使小厮叫他去。”小张闲道:“老太太,快使人情他来!这个疖子终要出脓,只顾脓着不是事。俺每为他连累打了这一顿。刚才老爹吩咐押出俺每来要他。他若不出来,大家都不得清净,就弄的不好了。”

林氏听言,连忙使小厮拿出茶来与众人吃。王三官唬的鬼也似,逼他娘寻人情。直到至急之处,林氏方才说道:“文嫂他只认的提刑西门官府家,昔年曾与他女儿说媒来,在他宅中走的熟。”王三官道:“就认的西门提刑也罢。快使小厮请他来。”林氏道:“他自从你前番说了他,使性儿一向不来走动,怎好又请他?他也不肯来。”王三官道:“好娘,如今事在至急,请他来,等我与他陪个礼儿便了。”林氏便使永定儿悄悄打后门出去,请了文嫂来。王三官再三央及他,一口一声只叫:“文妈,你认的提刑西门大官府,好歹说个人情救我。”这文嫂故意做出许多乔张致来,说道:“旧时虽故与他宅内大姑娘说媒,这几年谁往他门上走!大人家深宅大院,不去缠他。”王三官连忙跪下说道:“文妈,你救我,恩有重报,不敢有忘。那几个人在前边只要出官,我怎去得?”文嫂只把眼看他娘,他娘道:“也罢,你便替他说说罢了。”文嫂道:“我独自个去不得。三叔,你衣巾着,等我领你亲自到西门老爹宅上,你自拜见央浼他,等我在旁再说,管情一天事就了了。”王三官道:“见今他众人在前边催逼甚急,只怕一时被他看见怎了?”文嫂道:“有甚难处勾当?等我出去安抚他,再安排些酒肉点心茶水哄他吃着,我悄悄领你从后门出去,干事回来,他就便也不知道。”

这文嫂一面走出前厅,向众人拜了两拜,说道:“太太教我出来,多上覆列位哥每:本等三叔往庄上去了,不在家,使人请去了,便来也。你每略坐坐儿。吃打受骂,连累了列位。谁人不吃盐米,等三叔来,教他知遇你们。你们千差万差来人不差,恒属大家只要图了事。上司差派,不由自己。有了三叔出来,一天大事都了了。”众人听了,一齐道:“还是文妈见的多,你老人家早出来说恁句有南北的话儿,俺每也不急的要不的。执杀法儿只回不在家,莫不俺每自做出来的事?你恁带累俺每吃官棒,上司要你,假推不在家。吃酒吃肉,教人替你不成?文妈,你是晓道理的,你出来,俺每还透个路儿与你——破些东西儿,寻个分上儿说说,大家了事。你不出来见俺每,这事情也要消缴,一个缉捕问刑衙门,平不答的就罢了?”文嫂儿道:“哥每说的是。你每略坐坐儿,我对太太说,安排些酒饭儿管待你每。你每来了这半日也饿了。”众人都道:“还是我的文妈知人苦辣。不瞒文妈说,俺每从衙门里打出来,黄汤儿也没曾尝着哩!”这文嫂走到后边,一力窜掇,打了二钱银子酒,买了一钱银子点心,猪羊牛肉各切几大盘,拿将出去,一壁哄他众人在前边大酒大肉吃着。

这王三官儒巾青衣,写了揭帖,文嫂领着,带上眼纱,悄悄从后门出来,步行径往西门庆家来。到了大门首,平安儿认的文嫂,说道:“爹才在厅上,进去了。文妈有甚话说?”文嫂递与他拜帖,说道:“哥哥,累你替他禀禀去。”连忙问王三官要了二钱银子递与他,那平安儿方进去替他禀知西门庆。西门庆见了手本拜帖,上写着:“眷晚生王采顿首百拜。”一面先叫进文嫂,问了回话,然后才开大厅槅子门,使小厮请王三官进去。西门庆头戴忠靖巾,便衣出来迎接,见王三衣巾进来,故意说道:“文嫂怎不早说?我亵衣在此。”便令左右:“取我衣服来。”慌的王三官向前拦住道:“尊伯尊便,小侄敢来拜渎,岂敢动劳!”至厅内,王三官务请西门庆转上行礼。西门庆笑道:“此是舍下。”再三不肯。西门庆居先拜下去,王三官说道:“小侄有罪在身,久仰,欠拜。”西门庆道:“彼此少礼。”王三官因请西门庆受礼,说道:“小侄人家,老伯当得受礼,以恕拜迟之罪。”务让起来,受了两礼。西门庆让坐,王三官又让了一回,然后挪座儿斜佥坐的。

少顷,吃了茶,王三官向西门庆说道:“小侄有事,不敢奉渎尊严。”因向袖中取出揭帖递上,随即离座跪下。被西门庆一手拉住,说道:“贤契有甚话,但说何害!”王三官就说:“小侄不才,诚为得罪,望乞老伯念先父武弁一殿之臣,宽恕小侄无知之罪,完其廉耻,免令出官,则小侄垂死之日,实再生之幸也。衔结图报,惶恐,惶恐!”西门庆展开揭帖,上面有小张闲等五人名字,说道:“这起光棍,我今日衙门里,已各重责发落,饶恕了他,怎的又央你去?”王三官道:“他说老伯衙门中责罚了他,押出他来,还要小侄见官。在家百般辱骂喧嚷,索诈银两,不得安生,无处控诉,特来老伯这里请罪。”又把礼帖递上。西门庆一见,便道:“岂有此理!这起光棍可恶。我倒饶了他,如何倒往那里去搅扰!”把礼帖还与王三官收了,道:“贤契请回,我且不留你坐。如今就差人拿这起光棍去。容日奉招。”王三官道:“岂敢!蒙老伯不弃,小侄容当叩谢。”千恩万谢出门。西门庆送至二门首,说:“我亵服不好送的。”那王三官自出门来,还带上眼纱,小厮跟随去了。文嫂还讨了西门庆话。西门庆吩咐:“休要惊动他,我这里差人拿去。”

这文嫂同王三官暗暗到家。不想西门庆随即差了一名节级、四个排军,走到王招宣宅内。那起人正在那里饮酒喧闹,被公人进去不由分说都拿了,带上镯子。唬得众人面如土色,说道:“王三官干的好事,把俺每稳住在家,倒把锄头反弄俺每来了。”那个节级排军骂道:“你这厮还胡说,当的甚么?各人到老爹跟前哀告,讨你那命是正经。”小张闲道:“大爷教导的是。”

不一时,都拿到西门庆宅门首,门上排军并平安儿都张着手儿要钱,才替他禀。众人不免脱下褶儿,并拿头上簪圈下来,打发停当,方才说进去。半日,西门庆出来坐厅,节级带进去跪在厅下。西门庆骂道:“我把你这起光棍,我倒将就了你,你如何指称我衙门往他家讹诈去?实说诈了多少钱?若不说,令左右拿拶子与我着实拶起来!”当下只说了声,那左右排军登时拿了五六把新拶子来伺候。小张闲等只顾叩头哀告道:“小的每并没讹诈分文财物,只说衙门中打出来,对他说声。他家拿出些酒食来管待小的们,小的每并没需索他的。”西门庆道:“你也不该往他家去。你这些光棍,设骗良家子弟,白手要钱,深为可恨!既不肯实供,都与我带了衙门里收监,明日严审取供,枷号示众!”众人一齐哀告,哭道:“天官爷,超生小的每罢,小的再不敢上他门缠扰了。休说枷号,这一送到监里去,冬寒时月,小的每都是死数。”西门庆道:“我把你这起光棍,饶出你去,都要洗心改过,务要生理。不许你挨坊靠院,引诱人家子弟,诈骗财物。再拿到我衙门里来,都活打死了。”喝令:“叉出去!”众人得了个性命,往外飞跑。正是:

\[
敲碎玉笼飞彩凤,顿开金锁走蛟龙。
\]

西门庆发了众人去,回至后房,月娘问道:“这是那个王三官儿?”西门庆道:“此是王招宣府中三公子,前日李桂儿为那场事就是他。今日贼小淫妇儿不改,又和他缠,每月三十两银子教他包着。嗔道一向只哄着我!不想有个底脚里人儿又告我说,教我差干事的拿了这干人,到衙门里都夹打了。不想这干人又到他家里嚷赖,指望要诈他几两银子,只说衙门中要他。他从没见官,慌了,央文嫂儿拿了五十两礼帖来求我说人情。我刚才把那起人又拿了来,扎发了一顿,替他杜绝了。人家倒运,偏生这样不肖子弟出来。——你家祖父何等根基,又做招宣,你又见入武学,放着那名儿不干,家中丢着花枝般媳妇儿不去理论,白日黑夜只跟着这伙光棍在院里嫖弄。今年不上二十岁,年小小儿的,通不成器!”月娘道:“你乳老鸦笑话猪儿足,原来灯台不照自。你自道成器的?你也吃这井里水,无所不为,清洁了些甚么儿?还要禁人!”几句说的西门庆不言语了。

正摆上饭来吃,来安来报:“应二爹来了。”西门庆吩咐:“请书房里坐,我就来。”王经连忙开了厅上书房门,伯爵进里面坐了。良久,西门庆出来。声喏毕,就坐在炕上,两个说话。伯爵道:“哥,你前日在谢二哥家,怎老早就起身?”西门庆道:“我连日有勾当,又考察在迩,差人东京打听消息。我比你每闲人儿?”伯爵又问:“哥,连日衙门中有事没有?”西门庆道:“事,那日没有!”伯爵又道:“王三官儿说,哥衙门中把小张闲他每五个,初八日晚夕,在李桂姐屋里都拿的去了,只走了老孙、祝麻子两个。今早解到衙门里,都打出来了,众人都往招宣府缠王三官去了。怎的还瞒着我不说?”西门庆道:“傻狗才,谁对你说来?你敢错听了。敢不是我衙门里,敢是周守备府里?”伯爵道:“守备府中那里管这闲事!”西门庆道:“只怕是京中提人?”伯爵道:“也不是。今早李铭对我说,那日把他一家子唬的魂也没了,李桂儿至今唬的睡倒了,还没曾起炕儿。怕又是东京下来拿人,今早打听,方知是提刑院拿人。”西门庆道:“我连日不进衙门,并没知道。李桂儿既赌过誓不接他,随他拿乱去,又害怕睡倒怎的?”伯爵见西门庆迸着脸儿待笑,说道:“哥,你是个人,连我也瞒着起来。今日他告我说,我就知道哥的情。怎的祝麻子、老孙走了?一个缉捕衙门,有个走脱了人的?此是哥打着绵羊驹\textuni{29A07}战,使李桂儿家中害怕,知道哥的手段。若都拿到衙门去,彼此绝了情意,都没趣了。事情许一不许二。如今就是老孙、祝麻子见哥也有几分惭愧。此是哥明修栈道,暗度陈仓的计策。休怪我说,哥这一着做的绝了。这一个叫做真人不露相,露相不真人。若明逞了脸,就不是乖人儿了。还是哥智谋大,见的多。”几句说的西门庆扑吃的笑了,说道:“我有甚么大智谋?”伯爵道:“我猜一定还有底脚里人儿对哥说,怎得知道这等切?端的有鬼神不测之机!”西门庆道:“傻狗才,若要人不知,除非己莫为。”伯爵道:“哥衙门中如今不要王三官儿罢了。”西门庆道:“谁要他做甚么?当初干事的打上事件,我就把王三官、祝麻子、老孙并李桂儿、秦玉芝名字都抹了,只拿几个光棍来打了。”伯爵道:“他如今怎的还缠他?”西门庆道:“我实和你说罢,他指望讹诈他几两银子。不想刚才王三官亲上门来拜见,与我磕了头,陪了不是。我又差人把那几个光棍拿了,要枷号,他众人再三哀告说,再不敢上门缠他了。王三官一口一声称我是老伯,拿了五十两礼帖儿,我不受他的。他到明日还要请我家中知谢我去。”伯爵失惊道:“真个他来和哥陪不是来了?”西门庆道:“我莫不哄你?”因唤王经:“拿王三官拜帖儿与应二爹瞧。”那王经向房子里取出拜帖,上面写着:“眷晚生王采顿首百拜。”伯爵见了,极口称赞道:“哥的所算,神妙不测。”西门庆吩咐伯爵:“你若看见他每,只说我不知道。”伯爵道:“我晓得。机不可泄,我怎肯和他说!”坐了一回,吃了茶,伯爵道:“哥,我去罢,只怕一时老孙和祝麻子摸将来。只说我没到这里。”西门庆道。“他就来,我也不见他。”一面叫将门上人来,都吩咐了:“但是他二人,只答应不在家。”西门庆从此不与李桂姐上门走动,家中摆酒也不叫李铭唱曲,就疏淡了。正是:

\[
昨夜浣花溪上雨,绿杨芳草为何人?
\]
