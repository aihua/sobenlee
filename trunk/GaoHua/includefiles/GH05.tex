%# -*- coding:utf-8 -*-
%%%%%%%%%%%%%%%%%%%%%%%%%%%%%%%%%%%%%%%%%%%%%%%%%%%%%%%%%%%%%%%%%%%%%%%%%%%%%%%%%%%%%

\chapter{夺取意识形态的“解释权”}

\section{毛泽东从斯大林《联共党史》中学到了什么?}

从~1935~—~1938~年,经过四年的艰辛努力,其间虽有曲折和暂时的失意,毛泽东毕竟在实现其政治理想的大道上一路凯歌行进,到了~1938~年底,毛已将中共军权、党权牢牢控制在自己手里,然而,仍有一件事使毛如骨刺在喉,须臾不得安宁——这就是毛还未获得中共意识形态的解释权。

解释权——给词语下定义的权力,这是人类最重要的权力之一。在共产党内,解释权则尤其重要,谁获得对马列经典的解释权,谁就控制了党的意识。换言之,即使拥有军权和党权,若无意识形态解释权的支持,对党和军权的控制也难持久。解释之重要,不纯取决词语本身的内容及其意义,更在于词语与现实的联系,以及词语概念在社会生活中的作用。长期以来,在留苏派的经营下,俄化概念在中共党内早已形成一种特有的精神气质和浓厚的亲苏气氛,成为笼罩在党之上、阻遏一切创新精神的沉重低气压。王明、张闻天等不仅凭籍这种氛围扶摇直上,且沾沾自喜,俨然以圣杯看守人和护法大师自居,将一切独创思想均视为旁门左道而必欲除之而后快。在一个相当长的时期里,毛对此除了愤慨而毫无办法,彼等出自莫斯科正宗嫡传,在他们眼中,毛泽东的那一套岂只是离经叛道的“狭隘经验主义”,简直就是难登马列之堂奥的“野路子”。

自尊心受到极大伤害的毛泽东决意要发出自己的声音,且一鸣必求一言九鼎之效,使其政治对手就此噤口。还在~1910~年,当毛还是湘乡东山学堂学生时,他就借一首“咏蛙”诗而明其心志:

\begin{quoting}
独坐池塘如虎距,

绿杨树下养精神,

春来我不先开口,

哪个虫儿敢作声\footnote{转引自陈晋:《毛泽东的文化性格》(北京:中国青年出版社~1991~年),页~325。}!
\end{quoting}

而在毛泽东已初建其大业之后,他就更不能容忍中共党内还存有的那种精神指导系统。紧怀“传教”之志、兼有办事之才的毛,对自己及对手之特长均有极其清晰的了悟,他深知目前自己之优势非在于此——马列经典读得毕竟比那些洋学生少;但毛又极具自信,他之基于对中国历史及其文化传统深刻洞悉和体认,而对马列几个重要概念的融汇,在其实际功用价值上远胜于那些食洋不化的迂腐书生的纸上谈兵。在毛看来,留苏派攻“如秋潦无源,浮萍无根”,“其胸中茫然无有”,仅是凭籍莫斯科的栽培,先控制党的意识形态,继而夺取了党权和军权。毛却要反其道而行之,凭其个人的意志和智能先掌握军权和党权,最后攻占意识形态阵地。1938~年~10~月,当毛泽东已先读了一批马列著作和斯大林主义的解释课本——米丁、爱森堡的辩证唯物主义与历史唯物主义教科书之后,\footnote{参见中共中央文献研究室编:《毛泽东哲学批注集》(北京:中央文献出版社,1988~年)。该书收录经毛泽东批注的哲学读本共十一种,其中抗战初期阅读的达九种。}毛在中共六届六中全会讲台上向全党发出开展“学习运动”的号召。

学什么?一言以蔽之,学习马克思主义与中国实际相结合之产物——毛的新概念以及毛的态度和工作方法。可是当时既无“毛泽东思想”这一正式概念,又不便在斯大林远距离观察下直接鼓吹毛的新贡献,况且,毛也难于将其内心的真实想法和盘托出。毛真正陷入到欲语又止的境地。

然而这一切对于经历过无数风浪的毛泽东,又实在算不了什么。1938~年末,来自莫斯科任弼时的一份报告给了毛重要启示,帮助毛摆脱了这种困窘。任弼时向延安通报,莫斯科刚刚出版了斯大林编著的《联共党史》,建议中共中央立即翻译。\footnote{参见《中共党史人物传》,第~8~卷,页~48;另参见中共中央文献研究室编:《任弼时传》,页~436。}几个月后,当毛读了这本《联共党史》后,如获至宝,迅速向延安的各级干部发出号召:学习《联共党史》,做斯大林的学生!

毛泽东对《联共党史》给予了极高的评价,据当时担任毛军事秘书的郭化若回忆,毛在一个干部会上说:“《联共党史》是本好书,我已读了十遍。奉劝各位也多读几遍。”\footnote{郭化若:《在毛主席身边工作的片断》,载《毛泽东的哲学活动——回忆与评述》(北京:中共中央党校科研办公室,1985~年),页~157。}

从~1939~年至五十年代末,毛泽东不下十数次号召全党学习《联共党史》。1941~年~5~月~19~日,毛泽东在延安干部大会上作《改造我们的学习》的著名演讲,提议应“以联共党史为中心材料”,研究马列主义,“其它一切为辅助材料”。毛称赞道:

\begin{quoting}
《联共党史》是百年来全世界共产主义运动的最高综合和总结,是理论和实际结合的典型,在全世界还只有这个完全的典型\footnote{毛泽东~1941~年~5~月~19~日的演讲在~1942~年~3~月~27~日《解放日报》公开发表时,已作了修改和补充。1953~年收入《毛泽东选集》第~3~卷时又再次作少量修改。1953~年版本将毛发表演讲的时间模糊为~1941~年~5~月,把《联共党史》改为《联共产党(布)历史简要读本》,并删去毛原文中“其它一切为辅助材料”一段话。参见《整风文献》(订正本)(上海:新华书店,1950~年),页~56。}。
\end{quoting}

1942~年,毛泽东更将该书称之为“马列主义百科全书”,把其列入《整风文献》,以后又将其列入《干部必读》一类的学习丛书。1949~—~1956~年,《联共党史》甚至成为中国所有高等院校学生必修的政治课目,直到六十年代初,其至尊至荣的地位才被《毛泽东选集》所取代。

毛泽东如此推崇斯大林的《联共党史》,究竟是出于策略手段,抑或是发乎真心?毛泽东对斯大林有着很深的不满,眼下正在加紧赶制反对王明的炮弹,为何却对王明的精神导师——斯大林的大作这般推崇备至?乍看似乎令人费解,毛泽东和王明这两个政治上的对手,在歌颂斯大林的《联共党史》方面竟完全一致。1939~年~5~月~20~日,王明当着毛泽东的面,在延安召开的中央干部教育部学习动员大会上作的主报告就是“学习联共党史的重要意义”\footnote{延安《新中华报》,1939~年~5~月~26~日。},王明偏爱《联共党史》尚在情理之中,那么毛泽东呢?

毛泽东高度评价《联共党史》一定有他意欲迎合斯大林的目的,但是更重要的是,《联共党史》为毛提供了大量他所急需的开展党内斗争有用的经验与策略,完全可以用来为毛的政治目标服务,为毛夺取中共意识形态的解释权输送弹药。

《联共党史》全称为《联共(布)党史简明教程》,是斯大林为在政治上彻底消灭异己、巩固自己的独裁地位而下令编纂的。这部书经斯大林亲自修改、审核,于大镇压达到顶点的~1938~年正式出版。由于斯大林直接参与,精心组织了这部书的写作和出版,斯大林同意将这部书归于他的名下\footnote{前苏联国防部军史研究所所长德里·特里安东诺维奇·沃尔里戈诺夫在其所著《胜利与悲剧——斯大林政治肖像》一书中,利用苏共中央档案,详细披露了斯大林主持撰写《联共党史》的背景及其经过。参见《胜利与悲剧——斯大林政治肖像》,第~2~卷(北京:世界知识出版社,1990~年),页~576。}。

对于这本堪称斯大林主义标本的《联共党史》,毛泽东立时发现了它的巨大价值。毛泽东高度欣赏斯大林对马克思主义的灵活态度。和斯大林善于根据自己的政治需要,在《联共党史》中有选择地利用马克思主义原理一样,毛只专注于马克思主义的阶级斗争和无产阶级专政的理论。在把马克思主义通俗化方面,毛泽东和斯大林都堪称一流大师。斯大林擅长于将马克思主义化为几个概念,毛泽东则精于对马克思主义作简化处理,尤其善于把中国民间俗语、俚语引入到马克思主义。斯大林发明“干部决定一切”的口号,毛泽东则有名言:“枪杆子里面出政权”,“造反有理”!从《联共党史》简明、程序化的叙述中,毛泽东获益非浅。不久,他就依据《联共党史》的结束语部分,在中共西北局高干会议上作起《布尔什维克化十二条》的报告了。《联共党史》为了维护斯大林一贯正确的形象,任意剪裁历史,这也适应了毛泽东急欲重写以自己为中心的中共党史的政治需要。《联共党史》把苏联共产党历史描绘成正确路线与错误路线生死搏斗的历史,尤其突出斯大林的个人作用。全书提到斯大林、列宁的名字共有六百五十余次,介绍斯大林、列宁著作以及引用他们语录的篇幅占了全书的四分之一:共有一百页左右。这一切都吸引着自认为是党的正确路线化身的毛泽东的强烈兴趣,给了他丰富的启示。在毛的亲自主持下,1943~至~1945~年,任弼时、胡乔木等以《联共党史》为蓝本。以中共党内两条路线的斗争为纲,编写出《联共党史》的中国版:以毛泽东为中心的中共第一个历史总结文件——《关于党的若干历史问题的决议》。在这个决议中,只有毛泽东才是中共正确路线的代表(在~1945~年~8~月修改稿中,将刘少奇也增添为正确路线的代表),其他党的领导人,非“左”即右,至于广大党员则成了党的领袖的陪衬物。

《联共党史》以斯大林的个人意志作为评判是非的唯一标准,竭力丑化、贬低党的其他领导人,这又和毛泽东自大、专断的性格相吻合。《联共党史》把所有与斯大林意见相左的党的元老,一概斥之为“机会主义”、“人民公敌”、“叛徒”、“卖国贼”、“特务”、“暗害者”,为斯大林大规模屠杀老布尔什维克,实行社会恐怖,制造了理论依据。抗战阶段的毛泽东尚处于“打天下”的创业时期,革命成功还要依赖于全党齐心协力,因而只能有选择的学习斯大林的党内镇压经验。但是,毛泽东还是从《联共党史》学到了他所需要的东西。不久他就用“机会主义”、“经验主义”、“教条主义”,指控党的其他领导人。在毛的高压下,绝大多数党的领导人出于对党内团结的考虑,都自觉或违心地接受了这些指控,从此,他们的“小辫子”被毛牢牢拽在手心,毛可以随时根据需要任意处置他们,自己却永远立于不败之地。

毛泽东从《联共党史》中寻找到了他所需要的东西:以领袖为中心的革命战略与策略,以这个中心为基础,加强党的思想和建设的具体途径,展开党内斗争的方针与方法等。从这本书中,毛彻底了解了做一个斯大林式的领袖所必须具备的特殊素质,这就是:以铁腕统治全党,用意识形态为这种统治服务。对于这类问题,毛本来就不生疏,现在又有《联共党史》作依据,毛更加充满自信。

如果说《联共党史》从政治功利性的角度丰富了毛泽东的革命战略和策略思想,那么斯大林的哲学狙击手米丁、尤金,则为毛主义的哲学化助了一臂之力。

在中共领导人中,毛泽东是较少受马列教义束缚,思维最自由、最活跃的一位。毛青年时代深受中国传统哲学影响,尤其重视陆王心学,正由于毛与中国传统哲学始终保持一条沟通的渠道,他的有异于莫斯科正统的新思想、新概念才不时涌现。但是在三十年代前期,毛的一系列富有创见的观点和主张却遭到党内留苏派的贬损。心高气傲的毛泽东决不甘心仅仅做一个“实践家”,他也要进入马列主义理论家的殿堂。既为共产党员,就不能另辟新径,于是毛不得不求助于马克思主义教科书,使自己的观点系统化和理论化。

1937~—~1939~年,毛泽东在延安如饥似渴地钻研由博古、沈志远等从俄文翻译的唯物辩证法一类的马克思主义哲学教程。\footnote{郭化若:《在毛主席身边工作的片断》,载《毛泽东的哲学活动——回忆与评述》(北京:中共中央党校科研办公室,1985~年),页~576。}然而,毛所读的几乎都是当时苏联官方哲学家——米丁、尤金、西洛可夫等为斯大林著作作注解的教义问答式的“解释学”。

这样就不可避免造成互为矛盾的结果。一方面,由于毛泽东主动的迎合,他的活泼的思维被无形地压入进米丁、尤金等教条的框框,斯大林的直线性和独断性的思维方式对毛产生了巨大的影响,“一分为二”、“无限可分”等毛氏箴言,简直成了放之四海而皆准、应对世界万物的准则;另一方面,由于毛泽东研读米丁、尤金、西洛可夫乃是迫于外在压力,并非完全出于个人喜好;加之,毛的自由意志毕竟难以被斯大林式的“神学大全”所完全限制,因此,毛的思想经常“越轨”,不时跳出斯大林的紧箍咒而呈汪洋恣肆状,正是在这个时期,毛写出了他最重要的哲学论著《实践论》和《矛盾论》。总之,在抱定“为我所用”的宗旨下,毛或者能够从“本本”的片言只话中,获取灵感,将其引申开来,借洋调翻出中国曲;或者公然藐视“本本”而自谱新调。于是,创新开拓与僵硬保守并存,两者互为补充,互相融合,就成了毛思想的显著特征。在这个过程中,毛思想开始具有马克思主义的一般外观形式,而米丁、尤金、西洛可夫则在毛主义哲学化的早期工程中起到了脚手架的作用。

毛主义与斯大林主义的关系颇为复杂,将两者完全等同,或将两者完全割裂,皆有悖事实。然而就~1939~—~1941~年毛在延安推行的学习运动而言,毛之思想与斯大林主义有着千丝万缕的联系,乃是证据确凿。在斯大林的堂奥中,毛不仅发现了打击王明等的武器,而且也为自己理论的系统化找到了凝固剂。在毛精心的操作下,他竟可以一边用斯大林打击王明,另一边又用斯大林完善自己的思想体系。一旦斯大林也能为毛所用,那么斯大林的教条主义,毛也就食之如甘饴了。确实,毛的移花接木术已达到炉火纯青、造化无穷的境地。由此看来,毛之所为斯大林的学生,果真做到了“青出于蓝而胜于蓝”。

\section{“挖墙角、掺沙子”:陈伯达、胡乔木等的擢升}

经过政治上、理论上的艰苦努力,1940~年毛泽东通过发表《新民主主义的政治与新民主主义的文化》(后易名为《新民主主义论》),在中共党内已赢得首席马克思主义理论家的地位。现在毛要实施他酝酿已久的计划,这就是逐步削弱王明等对中共理论宣传部门的控制。

如果对王明等在中共党内发迹的历史稍作回顾,不难发现,这一批人主要是依靠意识形态起家的。他们首先因为熟读马列和斯大林本本,而得到共产国际的重视,从留苏学生中脱颖而出;继而依靠莫斯科支持,被扶植为中共领袖;复以意识形态开道,得以巩固在中共核心层中的地位。因此,意识形态成为王明、博古、张闻天等唯一真正具有优势的领域,也是他们看家护院仅有的一块阵地,中共理论、宣传部门长期由王明等留苏派把持也就不足为奇了。

中共六届六中全会后,王明、张闻天等虽然在政治上开始走下坡路,但是,王明等的失势并非是一天就完成的。直至四十年代初,中共意识形态主要部门仍由留苏派或与留苏派关系密切的人所掌握:

中共中央宣传部:部长张闻天、副部长凯丰

中共中央干部教育部:部长张闻天

中共中央党报委员会:主任博古

中共中央党校委员会:主任王明

中共中央党校:校长邓发

中央马列学院:院长张闻天

中国女子大学:校长王明

中共中央机关刊物《解放周刊》:主编张闻天\footnote{参见王建英编:《中国共产党组织史资料汇编——领导机构沿革和成员名录》,页~331~—~34。}

对于王明、张闻天等控制中共意识形态部门的现象,毛泽东一直予以相当的容忍。毛作为“策略大师”,十分明白区分在掌握实际权力与精神指导权力之间的轻重缓急关系。1935~—~1938~年,毛既是顺其自然,又是自觉促成,对张闻天领导党的意识形态工作不表示任何异议,毛的当务之急是将留苏派从权力核心地带引开,先巩固军权,进而夺取党权。毛深知,一旦有了军权和党权,再获取意识形态解释权乃水到渠成。1938~年六届六中全会,终于使毛如愿以偿地成为党的领袖,张闻天正式被剥去了党的总负责人的头衔,转而负责党的理论宣传、教育工作。不久,王明也被毛泽东召回延安。为了将王明置放于自己的监督之下又不让他掌握实权,毛让王明担任了中央统战部部长的闲职并兼任了几个中央文宣方面的职务,从表面现象上看,六中全会后,留苏派在党的意识形态领域的影响力反而得到增强。

面对这种复杂局面,毛泽东胸有成竹。毛的策略是,继续拉住张闻天,竭力分化张闻天与王明的关系;同时对张闻天领导的部门,“挖墙角,掺沙子”,提拔一些在党内根底较浅的青年理论新手,为日后取代张闻天等留苏派,储存干部队伍。

张闻天自六中全会后,在党的核心层内的影响已明显下降,他转而将全付精力转入到意识形态领域。此时正值毛泽东号召全党开展“学习运动”,在张闻天的主持下,延安编译出版了《马恩丛书》十册,和《列宁选集》二十卷。延安青年知识分子如饥似渴地研读原典,一时间,学习马列理论在延安蔚为风潮。具有讽刺意味的是,“学习运动”之开展,竟使得六中全会之后颇感失落、压抑、且被毛圈在延安中国女子大学和中央统战部几间窑洞里度日子的王明有枯木逢春之感。王明似乎感到施展自己马列才华的机会再次来临,竟然四处报告,居然受到延安各机关、学校广大青年知识分子的热烈欢迎。王明之风头仍健,其实并无任何意外。六中全会之后,王明仍是书记处和政治局成员。更重要的是,毛泽东在六中全会的政治报告《论新阶段》中包含了王明大量的政治观点,毛为了向斯大林显示其忠诚以及为了扩大中共在国内政治生活中的影响,通权达变,比王明更积极地主张加强与国民党的统一战线。六中全会的政治决议案也是由王明代表政治局起草。六中全会后,王明只是感到在政治上的失势,而无意识形态受挫之感。毛泽东对王明等的大出风头一时也无可奈何,站在共产党的角度,“学习运动”不学马列又学什么呢?于是毛泽东眼看着马列著作在延安一本本翻译出版,对张闻天的不满更加强烈。站在毛泽东的立场,张闻天旧错未改,又添新错,十足就是一个不可救药的教条主义分子。“学习运动”开展以后,张闻天不仅没有运用他所擅长的理论知识鼓吹毛的新贡献,也没有将其对王明的不满上升到理论批判的高度。更有甚者,张闻天一手掀起延安学习马列原著的热潮,言不及义,纸上谈兵,竟给王明等提供了表演的舞台,究其实质,纯粹是对“学习运动”别有用心的误导,目的是让张闻天、王明等这批“学阀”、“党阀”继续霸占党的文宣阵地。其实,1938~年后毛泽东在中共意识形态所占空间已大幅增进,掌管党的理论和宣传工作的张闻天主动给毛让出了最重要的权力:据当时张闻天的副手吴黎平(即吴亮平)回忆,六中全会后,中央内部已有规定,凡在延安《解放周刊》、《共产党人》等刊物发表重要文章,一概须经毛事先审阅批准\footnote{参见吴黎平:《〈论共产党员的修养〉出版的前前后后》,载《怀念刘少奇同志》(长沙:湖南人民出版社,1980~年),页~291。从《毛泽东书信选集》亦可看出中共中央已形成这种制度,1939~年初,张闻天为发表陈伯达的《论孔子的哲学思想》致信毛,请毛审阅陈文。2~月~20~日和~22~日,毛在审阅了陈伯达文后,两次写信给张闻天,谈对陈伯达文章的修改意见。参见《毛泽东书信选集》(北京:中国人民解放军出版杜,1984~年),页~144~—~51。},而毛的讲话、文稿无不刊登在党刊之首要位置。

然而毛泽东所要求意识形态部门的决不仅是这些,他的终极目标是改变中共气质。他要实现对意识形态的全部占领——包括控制和超越于控制之上的完全占有。将毛文章放在头版头条,或使毛拥有审稿权,并不能立时改变中共党内早已固定化的那种以俄为师的精神气氛,毛所要打破的正是在他个人与由张闻天等营造的气氛之间所存在的那种隔离状态,这种隔离状态与毛毫无亲和性,却能制造无数“又臭又长”、充满腐气的“党八股”,且使留苏派从容操纵全党的精神信仰系统。在这堵巨大的精神壁垒面前,毛氏新概念和新文体根本无从普及和推广,更遑论取其地位而代之!

毛泽东对于张闻天的个性和为人都了解很深。尽管在毛内心中,一直将张闻天与王明等量齐观,均视为是莫斯科教条主义集团的一路货色。但是,张闻天毕竟与王明分手较早,多年来对自己言听计从,骤然对张闻天下手,必定会引起党内高层的震动,反而对自己不利。因此,毛只能继续留用张闻天,再徐图良策。

1939~年~3~月~22~日,经过毛泽东的倡议,中共中央决定将中央党报委员会中的出版科(解放社)从党报委员会中划出来,另设立中央出版发行部,主管延安及各根据地党的出版物的政治审查和管理事务,由当时担任中央组织部副部长的李富春兼任部长一职。李富春是毛泽东多年的密友,任命李富春兼任此职,是削弱分散张闻天权限的一个重要举动。

然而,李富春长期从事党务工作,并不擅长马列理论,他所领导的出版发行部,其主要功能是“堵截”有害思想,虽然李富春在政治上完全效忠于毛,但由于他不熟悉理论,在宣传毛的贡献方面,一时乏善可陈。

毛泽东似乎早已预料到会出现这种“脱节”的局面,他并没有显出任何紧张和焦虑。因为此时在他的周围已经聚集了陈伯达、艾思奇、胡乔木、何思敬、何干之等一批青年理论家,只是这些人历练尚浅,还不能将彼等马上擢升到文宣部门的领导岗位。

被毛泽东慧眼发现,日后成为他的理论班子中坚的“秀才”由两类人组成,第一类:有留苏背景,但又与王明、博古等无历史渊源的陈伯达等;第二类:上海左翼文化运动中的活跃分子艾思奇、胡乔木,何干之等。这两类人的共同特点是,都未参加过苏区的军事斗争,在党内地位不高。

在毛泽东身边的这批“青年学士”中,较受毛欣赏和重视的是陈伯达和胡乔木。陈、胡二人因对毛助力最大而备受毛的信赖,分别于~1939~年和~1941~年被毛调入身边,担任中央政治局秘书和毛泽东个人的政治秘书,从而在中共的政治舞台上扮演了重要角色。

陈伯达对毛泽东的最大价值,在于他从理论方面丰富了毛的“马克思主义的中国化”的命题,为毛泽东新理论的完善立下了汗马功劳。

被毛泽东所吸引的正是陈伯达著述中所显示出的民族化共产主义的色彩。在三十年代北平、上海等一些大城市,李达、陈翰笙、王亚南、胡风等一批马克思主义哲学家、经济学家和文艺理论家积极从事马克思主义著作的译述活动,吸引了广大青年。但是陈伯达与李达等有所不同,陈伯达不是一般性做些马列著作的译介工作,他的著述有着明显的独创性。陈伯达师承著名的文史专家吴承仕教授,他较早地运用马克思主义理论解释中国哲学的基本概念,使他在左翼理论家中独树一帜。

陈伯达在~1933~年秋撰写的《论谭嗣同》小册子里,竭力用马克思主义的唯物辩证法原理去解释谭嗣同思想,提出在谭嗣同思想中“有初级唯物主义和不完美辩证法的痕迹”\footnote{Raymond Wylie: \textit{The Emergence of Maoism: Mao Tse-tung, Chen Pota, and the Search for Chinese Theory 1935--1945} (Cambridge: Harvard University Press, 1971), P15, 26--27.}。陈伯达甚至还提出中国的马克思主义者应成为中国伟大思想继承者的看法,\footnote{Raymond Wylie: \textit{The Emergence of Maoism: Mao Tse-tung, Chen Pota, and the Search for Chinese Theory 1935--1945} (Cambridge: Harvard University Press, 1971), P15, 26--27.}实际已接近“马克思主义的中国化”的命题。这在三十年代初中期,无疑是空谷足音。

陈伯达著述中所体现的民族化共产主义的倾向,在由他参与发起的“新启蒙”运动中得到进一步体现\footnote{1936~年~9~月~10~日,陈伯达在上海左翼刊物《读书生活》四卷九期发表《哲学的国防动员——新哲学者的自己批判和关于新启蒙运动的建议》,10~月~1~日,又在《新世纪》一卷二期发表《论新启蒙运动》;陈伯达的这两篇文章正式掀起了新启蒙运动的讨论。}。“新启蒙”运动是~1936~年~9~月至~1937~年夏,由陈伯达、艾思奇、周扬、何干之、胡乔木等左翼人士,为反对国民党倡导的“民族复兴运动”,在北平、上海等地,开展的一场旨在宣传马克思主义的思想文化运动。陈伯达虽然高度评价了“新哲学”(指马克思主义哲学)对中国的意义,但他同时又尖锐批评了中国左翼文化运动的严重不足。陈伯达认为,左翼人士应进行自我批判,因为他们“不能用辩证法来解释中国的现实生活”,没能对中国传统思想作出深入、系统的分析和批判\footnote{Raymond Wylie: \textit{The Emergence of Maoism: Mao Tse-tung, Chen Pota, and the Search for Chinese Theory 1935--1945} (Cambridge: Harvard University Press, 1971), P15, 30--31. 另参见何干之:《近代中国启蒙运动史》载,《何干之文集》(北京:中国人民大学出版社,1989~年),页~401~—~402。},以致国民党仍然可以将中国传统思想作为维护统治的有力工具,甚至日本帝国主义也在利用中国传统思想愚弄中国人民。陈伯达强调,中国现代文化一方面应从中国传统思想中汲取优秀美好的成份;另一方面,应吸收世界伟大的文化传统和成就,在马列主义的框架下,把中国传统哲学中的辩证法与先进的外国文化结合起来。\footnote{Raymond Wylie: \textit{The Emergence of Maoism: Mao Tse-tung, Chen Pota, and the Search for Chinese Theory 1935--1945} (Cambridge: Harvard University Press, 1971), P15, 30--31. 另参见何干之:《近代中国启蒙运动史》载,《何干之文集》(北京:中国人民大学出版社,1989~年),页~401~—~402。}

陈伯达所具有的中国古典哲学的学养和对民族化共产主义问题别具一格的看法,终于被毛泽东所发现。在民族化共产主义方面,毛泽东和陈伯达有着太多的共鸣。毛一经识得陈伯达,颇有相见恨晚之感。一时,毛、陈书信频频往来,纵论孔孟、墨子思想。\footnote{参见《毛泽东致陈伯达》(1939~年~2~月~1~日),载《毛泽东书信选集》。页~140~—~42。}此时的毛泽东正急欲把自己有关“马克思主义中国化”的观点理论化、系统化,只是苦于没有助手,陈伯达的出现,恰好填补了这一急需。1939~年春,在到达延安一年半后,陈伯达被毛泽东从马列学院教书的岗位上解放出来,一步跃升为中央军委主席办公室副秘书长,从此正式成为毛的首席理论助手。当陈伯达跨入权力中枢后,他的个人品质迅速被他周围的权力至上的气氛所毒化。陈伯达在这种“中国化”的环境中,也从一位朴素的教书先生,很快堕落成为一个利欲熏心的权力崇拜狂\footnote{陈伯达在~1940~年延安有关“民族形式”的讨论中,对与己观点不同的王实味无限上纲,暗指王实味是异己分子。陈伯达在与朋友相谈时,提到“最要紧的是跟人,跟准一个人”参见戴晴:《梁漱溟、王实味、储安平》(南京:江苏文艺出版社,1989~年),页~69~—~19;另参见于光远:《初识陈伯达》,载《读书》,1998~年第~6~期。}。

与陈伯达以理论学养获知于毛泽东不同,胡乔木主要是凭其走笔成章的能力及简练的文字功夫被毛录用为政治秘书的。胡乔木是三十年代中共领导上海左翼文化运动的中坚分子,因长期身居幕后,其文名远逊于当时的周扬、艾思奇、陈伯达。当胡乔木于~1937~年秋初抵延安后,中央组织部对他并没有特别重视,而是将其分配到远离延安、位于陕西泾阳县的安吴堡青年干部训练班作冯文彬的助手,担任青训班副主任和西北青年抗日联合会(共青团解散后,中共领导青年运动的组织,代行共青团职能)宣传部长\footnote{由西北青年救国会出面创办,实则由中共中央青年部(中央青委)领导的青年干部训练班,原设陕西泾阳县云阳镇,1938~年~1~月迁到云阳北的安吴堡,史称“安吴堡青训班”。有关胡乔木在安吴堡青训班的活动,参阅共青团中央青运史研究室、共青团陕西省委青运史研究室编:《安吴古堡的钟声——安吴青训班史料集》(北京:中央党史资料出版社,1987~年),页~2、28、240、247。}。

在安吴堡的两年,是胡乔木政治上失意的两年,胡乔木几乎没有和毛泽东接触的机会。青训班的主要干部大多为经历过长征的老红军,与胡乔木的气质、兴趣相去甚远。1938~年~5~月,中共中央决定成立中央青委,以取代原来的中央青年部,由陈云任书记,冯文彬任副书记,胡乔木也被吸收进中央青委\footnote{中共中央青委由陈云、冯文彬、李昌、刘光、乔木、高朗山、黄华、宋一平组成。参见《冯文彬给黄华、宋一平的信》(1938~年~5~月~16~日),载《安吴古堡的钟声——安吴青训班史料集》,页~23~—~24。},从而和陈云结上了工作关系。1939~年~7~月,胡乔木在安吴堡闲置了两年后\footnote{胡乔木自述,他是~1938~年~8~月返回延安的,但根据依档案编写的《安吴古堡的钟声——安吴青训班史料集》一书记载,胡乔木返回延安的时间应是~1939~年~7~月,胡记忆有误。另参见胡乔木:《胡乔木回忆毛泽东》(北京:人民出版社,1994~年),页~27。},终于回到了延安。此时陈伯达已担任了毛泽东的政治秘书,而胡乔木仍默默无闻。但是有了陈云的扶持,胡乔木渐渐有了崭露头角的机会。1939~年~4~月~16~日,《中国青年》杂志在延安复刊,胡乔木参加了该刊的编辑领导工作,逐渐引起毛的注意。1940~年~5~月,在安吴堡青训班基础上,正式建立了由陈云任校长的“泽东青年干部学校”,胡乔木成为该校的实际负责人,从此与毛泽东有了更多的接触机会。

1941~年,胡乔木正式被调入杨家岭,担任毛泽东的政治秘书,并兼任中央政治局秘书。胡乔木主要为毛的演讲稿做文字修辞方面的加工整理,根快成为毛须臾不可离开的助手。刚入中枢的胡乔木言谈举止极为谨慎,较少抛头露面,直到~1942~年整风高潮中,毛把胡乔木派往中央宣传部。代行原由凯丰代理的部长一职\footnote{参见胡乔木:《我所知道的田家英》,载董边、镡德山、曾自编:《毛泽东和他的秘书田家英》(北京:中央文献出版社,1989~年)页~121。},使胡乔木一夜之间成为延安的知名人物。

与陈伯达、胡乔木相比,同胡乔木一道于~1937~年秋奔赴延安的艾思奇、何干之、王学文等人,在政治上的前途就大为逊色了。艾思奇、何干之、王学文在三十年代上海左翼文化界享有盛名,他们刚抵延安的头几年,也蒙毛泽东的热情款待,被安排在陕北公学、中央党校、马列学院任教,享受每月津贴二十元和配备一名警卫员的待遇\footnote{参见成仿吾:《何干之文集序》,载《何干之文集》,页~3。},毛泽东并不时移尊与他们就一些哲学、理论问题进行讨论。在一段时期内,毛对艾思奇特别重视,除经常书信往来外,还不时邀艾思奇在毛的窑洞挑灯夜谈。但是时间一长,毛对艾思奇等的兴趣就大减了。毛虽然欣赏艾思奇对马克思主义的通俗化解释,但是艾思奇的通俗化没有丰厚的底蕴,其概念工具仍是俄式教条主义,“通俗化”几乎成了“教科书化”的代名词。何干之虽然著述颇丰,毛也一度很看重他,曾经想请何干之做自己的理论秘书。但何干之一则书生气较重,向毛表示自己有意著述,婉拒了毛的好意;\footnote{胡华、刘炼(何干之遗孀):《何干之》,载《中共党史人物传》,第~21~卷,页~266。}二则何干之为文过快,有粗疏之弊,毛也就顺其意愿,不再强邀何干之入幕。王学文资格最老,革命历史和理论学养都较深厚,但为文为人都过于学究气,“教条主义”色彩较为浓厚,也不适合放在身边工作\footnote{24~据刘雪苇回忆,王学文在马列学院用苏联列昂节夫《政治经济学》做课本,“讲得很拘谨,论点、要义的地方差不多就是照原书念”。虽然刘雪苇听王学文课“津津有味”,但同学中却有人“打瞌睡”。参见刘雪苇:《在延安马列学院三班听课的回忆》,载吴介民主编:《延安马列学院回忆录》(北京:中国社会科学出版社,1991~年),页~124。}。毛泽东虽然对艾思奇、何干之、王学文等有所失望,但在当时,毛对彼等并不求全责备,相反,毛积极吸取他们著述中一切有用之内容。例如,何干之~1936~年~11~月在《中国过去、现在和未来》(后易名为《转变期的中国》)小册子中,提出中国是“半殖民地半封建社会”,中国现阶段革命的性质是“新的民主革命”等观点,虽来源于共产国际,但其表述简洁、明了,对毛以后写作《新民主主义论》有一定的启示作用。对于这些不属留苏派、来自于上海亭子间的理论家,毛泽东用其长,弃其短,1938~年后,他们陆续都被委以延安各文宣部门的领导职务。艾思奇甫抵延安即被任命为边区文化界协会主席,马列学院成立后,艾思奇又担任了该院的哲学研究室主任,1939~年又担任了中宣部文化工作委员会秘书长,1940~年他又被任命为新创刊的《中国文化》主编。何干之被任命为中央文化工作委员会委员。王学文则在~1938~年秋担任了马列学院副院长兼教务主任。

1939~—~1941~年,毛泽东采用“挖墙角、掺沙子”的策略,步步为营,将自己的影响力逐步渗入到由留苏派控制约中共理论宣传教育部门。毛将陈伯达、胡乔木网罗在身边,犹如猛虎添翼,现在毛泽东要将“学习运动”引向纵深地带——为开展党的历史问题的讨论,直接向王明等发起挑战制造炮弹。

\section{“甩石头”:毛泽东编“党书”}

1940~年末,毛泽东决定扭转学习运动的方向,他再不能容忍充斥于延安各机关“空对空”的纯学理论的空气。毛的策略是,“射人先射马,擒贼先抢王”,暂时放过党的中下层干部,先将党的高级干部从单纯读书的氛围中解脱出来,把他们引人到对敏感的党的十年(1927~—~1937)历史的讨论中去,进而把野火烧向王明、张闻天。

毛泽东此举带有“玩火”的性质。1938~年共产国际在对中共的指示中明确告诫中共领导层,应慎重对待党的过去历史问题的讨论,以避免发生不必要的争论,从而影响党的团结。三年前共产国际的指示,虽言犹在耳,毛泽东却毫不在意,经过近几年的经营,毛的权力基础又有新的加强,他要先行一步,试探一下其他领导人的反应。

1940~年~12~月,毛泽东正式提出了隐藏在胸中多年的观点,12~月~4~日,毛在政治局会议上首次对苏维埃后期极左的政策作出判断,认为这实际上是路线上的错误,“所以遵义会议决议须有些修改”。会上马上产生争论,张闻天等不同意苏维埃后期错误是路线错误的判断\footnote{参见《毛泽东年谱》,中卷,页~235~—~36。}。面对张闻天等的反对,毛稍稍调整了自己的言论,12~月~25~日,毛在以后以《论政策》之名发表的党内指示中,第一次宣布,中共在苏维埃运动后期犯了左倾机会主义错误,并提及其在十一个方面的表现。在这里,毛放了一只观测气球,他使用的是比较笼统、含混的“苏维埃运动后期”的概念,而没有明确指明其时间段是从~1931~年中共六届四中全会至~1935~年遵义会议召开之前;他用“左倾机会主义错误”来代替“左倾机会主义路线错误”的正式判断。

毛泽东之所以选择在此时提出自己的观点,是基于他对其政治对手内部分裂状况之准确把握。毛十分清楚,在中共领导层内顽强坚持对原政治路线评价的人,并不是王明,而是与毛长期合作共事、且在~1940~年仍与毛关系密切的张闻天。至于王明,则在~1940~年~11~月就提出中共在苏维埃运动后期犯了严重错误的看法。\footnote{王明:《论马列主义决定策略的几个基本原则》,原载延安《共产党人》,1940~年第~12~期,引自蔡尚思主编、姜义华编:《中国现代思想史资料简编》,第~4~卷(杭州:浙江人民出版社,1983~年),页~488。}尽管王明只是重复其~1933~—~1934~年在莫斯科即曾表明的观点,但王明此时旧话重提,却完全是为了与在国内的原同事博古、张闻天撇清关系。对于王明、博古、张闻天等互相攻讦、竞相推卸责任,毛看在眼里,却丝毫没有表示他个人对王明的欣赏和支持,毕竟使毛感到威胁的是在国际共运中声名远扬的留苏派之精神领袖王明,而非王明昔日之朋友、且早已在一系列重大问题上与自己站在一边的张闻天。眼下形势早已变化,王明正不时向毛示好,且将毛从“中国革命的伟大政治家和战略家”升格为“伟大的理论家”\footnote{王明:《学习毛泽东》,延安《新中华报》,1940~年~5~月~7~日。}。只有张闻天一人还在顽强抵拒毛为修正历史结论所作的努力,这种情况迫使毛不得不另图他策。1940~年冬至~1941~年~6~月,毛泽东用了半年时间精心准备了一块砸向王明等留苏派的“石头”,这就是党的秘密文件集《六大以来》。

《六大以来》是一本深受斯大林《联共党史》影响、经毛泽东精心编排的中共历史文献汇编。全书分上、下两册,上册完成于~1941~年~6~月,全书编成于~1941~年~12~月\footnote{《六大以来——党内秘密文件》有全集本和选集本两种,选集本由八十六篇材料组成,均分为上、下两册,十六开本,由延安新华印刷厂印刷。1941~年~12~月,《六大以来》全集本仅印了五百套,发行到几个中共中央局、军委等少数单位,不对个人发放。选集本逐一编号、登记、分发给党的高级干部。1947~年,中共中央在国民党军队进攻下,撤离延安,“全集本”只在中央办公厅保留几部,其余全部销毁。1952、1980~年,中共中央办公厅在对该书作若干调整、补充后,两次予以重印,由内部控制发行。参见裴淑英:《关于〈六大以来〉一书的若干情况》,载《党的文献》,1989~年第~1~期。}。《六大以来》共收入各种文献凡五百一十九篇,约二百八十万字,收入文件的时间跨度,从~1928~年~6~月中共六大召开到~1941~年~11~月。

这套文献集的最大特点是具有鲜明的倾向性。在胡乔木、王首道的帮助下,毛泽东以中共党内所谓的“两条路线”为经纬,以此观点来编排史料。该书将毛泽东树为党的正确路线的代表,将王明、博古等列为错误路线的代表。全书共收入毛的文章、讲演、报告共五十五篇,占全书十分之一的比重。毛对王明、博古、张闻天起草的大量文稿作了细致的取舍,择其一部分,作为“反面教材”收入该书。毛泽东对周恩来在抗战初期支持王明的活动,也丝毫未忘却,在《六大以来》中将周恩来顺带捎上。该书全文收录了~1938~年~6~月~15~日周恩来和王明、博古联名发表的《我们对于保卫武汉与第三期抗战问题底意见》一文,作为周恩来对“错误路线”附和、妥协的证据。相比之下,刘少奇的政治行情看涨,在《六大以来》里也得到了充分的反映。该书收录了刘少奇四篇文章,被毛编排入“正确路线”一方。周恩来单独发表的文章仅收录一篇。毛通过此举向全党高级干部明确表示,只有刘少奇才是毛“正确路线”的真正拥护者。

毛泽东为表明自己一贯正确,对于收入《六大以来》的自己的文章,作了精心的选择,将一切有碍于自己“形象”的文字尽情作了删除。毛剔除了表明自己支持~1931~—~1935~年党的政治路线的文字,对自己在三十年代前、中期发表的一些文章还作了细心的剪裁。毛只选录了他在~1934~年~1~月全国第二次苏维埃大会上的报告(只有苏维埃才能救中国)的一小部分,易名为《我们的经济政策》和《关心群众生活,注意工作方法》,收入《六大以来》。毛对《论新阶段》更是大动手术,他只截取报告中谈“马克思主义的中国化”和“独立自主”的若干节,改名为《中国共产党在民族战争中的地位》,将其收入《六大以来》。

有确凿的证据证明,在编辑《大大以来》的过程中,毛泽东出于自己的政治目的,在个别文件的日期上作了手脚。《六大以来》收入的由任弼时主持通过的《中央苏区第一次党代表大会政治决议案》,该文件原来的形成时期是~1931~年~11~月~1~至~5~日,但是,被收入《六大以来》时,却被改为~1931~年~3~月\footnote{《六大以来——党内秘密文件》(上),页~129。1931~年~3~月~18~日至~21~日,项英以苏区中央局书记的身份主持召开了苏区中央局扩大会议,因等待中共中央指示的到来,没有形成会议决议就宣布暂停会议。4~月~17~日,任弼时等到达宁都县与项英、毛泽东等会合,苏区中央局再举行一天的会议,作为前一月扩大会议的继续。此次会议形成五个文件:一、(接受国际来信及四中全会决议的决议);二、《土地问题决议》;三、《关于一、三军团工作总结的决议》;四、《关于富田事变的决议》;五、《共青团工作的决议》——没有《中央苏区第一次党代表大会政治决议案》。中央苏区第一次党代表大会召开于~1931~年~11~月~1~日至~5~日,此次会议的政治决议案由王稼祥起草。在编辑《六大以来》时,毛对这份决议案也作了剪裁,毛只将该文件的第一部分收入《六大以来》,而略去了第二、三部分,因为该决议案第二部分“中央苏区过去工作的检阅”明确提到“苏区党代表大会完全同意中央九月指示信”,如果将这些内容收入《六大以来》,就不能把该文件的日期写成~1931~年~3~月。参见中央档案馆编:《中共中央文件选集》(1931),第~7~册,页~451。}。这个改动非同寻常,它掩盖了~1931~年~4~月至~10~月,由任弼时为首的中央代表团支持毛泽东、与毛联手共同反对项英的历史真相。毛之所以选中这份决议案,乃是该文件批评了毛所主张的“抽多补少”、“抽肥补瘦”的土地政策,正好从反面说明毛与六届四中全会后的中央存在深刻的路线分歧。把通过该决议案的时间改为~1931~年~3~月,是为了强调六届四中全会后的中央派往江西的代表团,下车伊始,就反对毛的正确主张,而毛长期遭受中央的错误压制。毛将这个决议案收入《六大以来》,也是给任弼时一个警戒;使其明白,他在历史上也和错误路线沾过边。

毛泽东编辑《六大以来》,其理由是为召开七大作资料准备,\footnote{《胡乔木回忆毛泽东》,页~176。}毛泽东的这个大行动得到了任弼时的全力支持。任弼时自~1938~年~3~月赴莫斯科后,在共产国际总部居留两年,直至~1940~年~3~月~26~日才返回延安。任弼时回国后,立即进入中共最高权力核心中央书记处(相当于政治局常委会),到当年的~7、8~月,又被任命为中央秘书长,负责中央书记处的常务工作,成为毛泽东最重要的助手。

任弼时是一个具有理想主义色彩的共产党人。从三十年代后期始,任弼时对毛泽东的智能和谋略愈益钦佩,认为在中共党内唯有毛泽东堪负领袖重责,并且相信只要有毛掌舵,中共就一定能走向成功。因此,任弼时在毛泽东与王明之间较早就作出了选择,在毛泽东比较困难的~1938~年初春,任弼时在“三月政治局会议”上明确支持毛泽东的意见,从而获得毛的信任。任弼时也有其难言之隐。尽管在~1931~年~4~至~10~月,任弼时曾与毛泽东真诚合作,一同联手反对项英,但是在~1931~年~10~月后,却是由他具体贯彻六届四中全会后的政治局路线,给毛扣上了“狭隘经验主义”的帽子。任弼时对三十年代初期曾经给毛造成的伤害一直深感内疚,多年来主动与毛靠拢,试图争取毛的谅解。1938~年~3~月,任弼时赴莫斯科后,为突出毛在共产国际的影响,竭心尽力,成效显著。返回延安后,任弼时更是全力辅佐毛,处处从中共的长远利益着眼,对毛的活动都从好的方面去理解。任弼时出于维护党的利益而支持毛泽东,这并不意味他就欣赏毛泽东的一切。任为人正派,有时颇看不惯毛的所作所为\footnote{王明:《中共五十年》,页~54。},只是出于各种复杂因素的考虑,对毛的一些过份之举,任弼时多佯装不知而已。作为六届四中全会后派往江西苏区的中央代表团团长,任弼时绝不会把他主持的苏区第一次党代表大会的日期忘记,正是由于任弼时的妥协态度,毛才敢于公开修改那次会议的日期。1940~年底到~1941~年上半年,毛泽东反击王明的活动已临近决战的前夜。毛一方面加紧编辑《六大以来》,同时,在为中共中央起草的文件中,他不断地亮出自己批评前中央政治路线的观点,为正式向王明等摊牌制造舆论。1941~年~1~月,毛以中央的名义,下令组成由一百二十人参加的党的高级干部学习组,其中有重要干部四十多人。开始讨论党的历史经验问题,将对王明等的包围圈进一步收紧。一旦观测到留苏派没有动静,尤其是留苏派的后台斯大林竟也毫无动作,毛迅速将阵地从党的十年历史问题移到当前。这次毛不再将王明、张闻天分开,而是要让这些“理论大师”立时呈现原形。毛决定彻底摧毁王明等赖以在党内坐大的基础——他们所拥有的马克思主义理论家兼圣杯看守人的名号!1941~年~5~月~19~日,毛泽东当着王明等人的面,向王明发起新的一轮攻击。在《改造我们的学习》的报告中,他要求彻底扭转~1938~年后开展的学习运动的方向,“废止孤立地、静止地研究马克思列宁主义的方法”,而代之以学习当代最高综合的马列主义——斯大林的《联共党史》和中国化的马克思主义!在向王明发起的最新挑战中,一组组最具隐喻性和挑战性的新词汇被毛创造出来——“言必称希腊”、“希腊和外国的故事”、“教条”、“留声机”,尽管皆有其针对意涵,却并不明确所指,这就更加容易在词语与现实之间引发疑问和联想,从而猛烈动摇王明等的老词汇的神圣地位,为毛通过改变词语、夺取意识形态解释权扫清障碍。紧接春,1941~年~6~月,毛泽东将《六大以来》的第一部分编辑完毕,该书立刻成为毛砸向王明等的一块石头。两年后,毛泽东在~1943~年~10~月政治局扩大会议的讲话中,回忆起《六大以来》的出版所造成的巨大效应,他说:

\begin{quoting}
1941~年~6~月编了党书,党书一出,许多同志解除武装,故可能召开九月会议,大家才承认错误\footnote{逄先知(1950~—~1966~年曾负责管理毛泽东的图书):《关于党的文献编辑工作的几个问题》,载《文献和研究》,1987~年第~3~期。}。
\end{quoting}

“许多同志解除武装”,固然和《六大以来》一书对党的核心层造成巨大的精神冲击有关,但是,毛泽东为配合该书的出版而精心策划的一系列活动,也极大地削弱了政治局大多数成员的抵抗能力。政治局在毛的进攻前,除了束手向毛输诚,别无其它出路。

从~1941~年春开始,毛泽东连续推出几个重大举措。3~月~26~日,毛以中共中央的名义,作出《关于调整刊物问题的决定》,一举端掉王明、张闻天等最后几个舆论阵地。该《决定》声明,由于“技术条件的限制”和“急于出刊”某些书籍和小册子,停止出版《中国妇女》、《中国青年》和《中国工人》三家刊物\footnote{《中央关于调整刊物问题的决定》(1941~年~3~月~26~日),载团中央青运史研究室、中央档案馆编:《中共中央青年运动文件选编》(1921~年~7~月—~1949~年~9~月)(北京:中国青年出版社,1988~年),页~539。}。毛为了显得“一碗水端平”,将胡乔木负责的《中国青年》与另两个刊物一并停刊,使领导《中国妇女》、《中国工人》的王明、博古、邓发等人,明知其中有诈,也无言可说。

1941~年~9~月~1~日,毛泽东又将上述谋略如法炮制一遍,宣布撤消由王明担任校长的中国女子大学,将女大与陕北公学、泽东青年干部学校合并为延安大学,把王明担任的最后一个可以抛头露面的职务巧妙地剥夺掉。

毛泽东且对手无缚鸡之力的王明如此处心积虑,对于那些手握兵符、统兵一方的八路军将领就更不放心了。为了防止彭德怀领导的八路军在前方可能会滋生的“自主性”和“分散性”,打击任何“将在外,君命有所不受”的企图,毛责成王稼祥、王若飞于~1941~年~7~月~1~日以中共中央的名义,起草了《关于增强党性的决定》。该文件不指名地警告彭德怀和各根据地领导人以及周恩来领导的中共南方局,必须一切听命于延安,不得“在政治上自由行动”,“在组织上搞独立王国”。毛并威胁彼等应吸取张国焘“身败名裂的历史教训”。毛泽东十分了解这些长期献身革命事业、无比珍惜自己革命历史的高级干部的心理特点,以党和革命的名义,将他们牢牢控制在手里。

现在,一切都已准备就绪,一场大规模的党内整肃风暴即将来临!