%# -*- coding:utf-8 -*-
%%%%%%%%%%%%%%%%%%%%%%%%%%%%%%%%%%%%%%%%%%%%%%%%%%%%%%%%%%%%%%%%%%%%%%%%%%%%%%%%%%%%%


\chapter{陈敬济弄一得双\KG 潘金莲热心冷面}


词曰:

\[
闻道双衔凤带,不妨单着鲛绡。夜香知为阿谁烧?怅望水沉烟枭。云鬓风前绿卷,玉颜想处红潮,莫交空负可怜宵,月下双湾步俏。\named{右调《西江月》}
\]

话说潘金莲与陈敬济,自从在厢房里得手之后,两个人尝着甜头儿,日逐白日偷寒,黄昏送暖。或倚肩嘲笑,或并坐调情,掐打揪撏,通无忌惮。或有人跟前不得说话,将心事写了,搓成纸条儿,丢在地下,你有话传与我,我有话传与你。一日,四月天气,潘金莲将自己袖的一方银丝汗贴儿,裹着一个纱香袋儿,里面装一缕头发并些松柏儿,封的停当,要与敬济。不想敬济不在厢房内,遂打窗眼内投进去。后敬济进房,看见弥封甚厚,打开却是汗巾香袋儿,纸上写一词,名《寄生草》:

\[
将奴这银丝帕,并香囊寄与他。当初结下青丝发。松柏儿要你常牵挂,泪珠儿滴写相思话。夜深灯照的奴影儿孤,休负了夜深潜等荼縻架。
\]
敬济见词上约他在荼縻架下等候,私会佳期。随即封了一柄湘妃笔金扇儿,亦写了一词在上回答他,袖入花园内。不想月娘正在金莲房中坐着,这敬济三不知,走进角门就叫:“可意人在家不在?”这金莲听见是他语音,恐怕月娘听见决撒了,连忙掀帘子走出来。看着他摆手儿,佯说:“我道是谁,原来是陈姐夫来寻大姐。大姐刚才在这里,和他每往花园亭子上摘花儿去了。”这敬济见有月娘在房里,就把物事暗暗递与妇人袖了,他就出去了。月娘便问:“陈姐夫来做甚么?”金莲道:“他来寻大姐,我回他往花园中去了。”以此瞒过月娘。少顷,月娘起身回后边去了。金莲向袖中取出拆开,却是湘妃竹金扇儿一柄,上面一种青蒲,半溪流水,有《水仙子》一首词儿:

\[
紫竹白纱甚逍遥,绿囗青蒲巧制成,金铰银钱十分妙。美人儿堪用着,遮炎天少把风招。有人处常常袖着,无人处慢慢轻摇,休教那俗人见偷了。
\]

妇人看见其词,到于晚夕月上时,早把春梅、秋菊两个丫头打发些酒与他吃,关在那边炕屋睡。然后自在房中,绿半启,绛烛高烧,收拾床铺衾枕,薰香澡牝,独立木香棚下,专等敬济来赴佳期。西门大姐那夜恰好被月娘请去后边,听王姑子宣卷去了,只有元宵儿在屋里。敬济梯己与了他一方手帕,分付他:“看守房中,我往你五娘那边下棋去。等大姑娘进来,你快来。”元宵儿应诺了。敬济得手,走来花园中,只见花筛月影,参差提成映。走到荼縻架下,远望见妇人摘去冠儿,乱挽乌云,悄悄在木香棚下独立。这敬济猛然从荼縻架下突出,双手把妇人抱住。把妇人唬了一跳,说:“呸,小短命!猛然外事出来,唬了我一跳。早是我,你搂便将就罢了,若是别人,你也恁胆大搂起来?”敬济吃得半酣儿,笑道:“早是搂了你,就错搂了红娘,也是没奈何。”两个于是相搂相抱,携手进入房中。房中荧煌煌掌着灯烛,桌上设着酒肴,一面顶了角门,并肩而坐饮酒。妇人便问:“你来,大姐在那里?”敬济道:“大姐后边听宣卷去了,我分付下元宵儿,有事来这里叫,我只说在这里下棋。”说毕,上欢笑做一处。饮酒多时,常言“风流茶说合,酒是色媒人”,不觉竹叶穿心,桃花上脸,一个嘴儿相亲,一个腮儿厮揾,罩了灯,上床交接。有《六娘子》小词为证:

\[
入门来,奴搂抱在怀。奴把锦被儿伸开,俏冤家顽的十分怪。嗏,将奴脚儿抬。脚儿抬,揉乱了乌云,\textuni{4BFC}髻儿歪。
\]

两人云雨才毕,只听得元宵叫门说:“大姑娘进房中来了。”这敬济慌的穿衣去了。正是:

\[
狂蜂浪蝶有时见,飞入梨花无处寻。
\]

原来潘金莲那边三间楼上,中间供养佛像,两边稍间堆放生药香料。两个自此以后,情沾肺腑,意密如漆,无日不相会做一处。一日也是合当有事,潘金莲早辰梳妆打扮,走来楼上观音菩萨前烧香。不想陈敬济正拿钥匙上楼,开库房门拿药材香料,撞遇在一处。这妇人且不烧香,见楼上无人,两个搂抱着亲嘴咂舌,一个叫“亲亲五娘”,一个呼“心肝短命”,因说:“趁无人,咱在这里干了罢。”一面解褪衣裤,就在一张春凳上双凫飞肩,灵根半入,不胜绸缪。当初没巧不成话,两个正干得好,不防春梅正上楼来,拿盒子取茶叶看见。两个凑手脚不迭,都吃了一惊。春梅恐怕羞了他,连忙倒退回身子,走下胡梯。慌的敬济兜小衣不迭,妇人穿上裙子,忙叫春梅:“我的好姐姐,你上来,我和你说话。”那春梅于是走上楼来。金莲道:“我的好姐姐,你姐夫不是别人,我今叫你知道了罢。俺两个情孚意合,拆散不开。你千万休对人说,只放在你心里。”春梅便说:“好娘,说那里话。奴伏侍娘这几年,岂不知娘心腹,肯对人说!”妇人道:“你若肯遮盖俺们,趁你姐夫在这里,你也过来和你姐夫睡一睡,我方信你。你若不肯,只是不可怜见俺每了。”那春梅把脸羞的一红一白,只得依他。卸下湘裙,解开裤带,仰在凳上,尽着这小伙儿受用。有这等事!正是:明珠两颗皆无价,可奈檀郎尽得钻。有《红绣鞋》为证:

\[
假认做女婿亲厚,往来和丈母歪偷。人情里包藏鬼胡油。明讲做儿女礼,暗结下燕莺俦,他两个见今有。
\]

当下尽着敬济与春梅耍完,大家方才走散。自此以后,潘金莲便与春梅打成一家,与这小伙儿暗约偷期,非只一日,只背着秋菊。

六月初一日,潘姥姥老病没了,有人来说。吴月娘买一张插桌,三牲冥纸,教金莲坐轿子往门外探丧祭祀,去了一遭回来。到次日,六月初三日,金莲起来得早,在月娘房里坐着,说了半日话出来,走在大厅院子里墙根下,急了溺尿。正撩起裙子,蹲踞溺尿。原来西门庆死了,没人客来往,等闲大厅仪门只是关闭不开。敬济在东厢房住,才起来,忽听见有人在墙根溺的尿刷刷的响,悄悄向窗眼里张看,却不想是他,便道:“是那个撒野,在这里溺尿?撩起衣服,看溅湿了裙子?”这妇人连忙系上裙子,走到窗下问道:“原来你在屋里,这咱才起来,好自在。大姐没在房里么?”敬济道:“在后边,几时出来!昨夜三更才睡,大娘后边拉着我听宣《红罗宝卷》,坐到那咱晚,险些儿没把腰累■■了,今日白扒不起来。”金莲道:“贼牢成的,就休捣谎哄我!昨日我不在家,你几时在上房内听宣卷来?丫鬟说你昨日在孟三儿房里吃饭来。”敬济道:“早是大姐看着,俺每都在上房内,几时在他屋里去来!”说着,这小伙儿站在炕上,把那话弄得硬硬的,直竖的一条棍,隔窗眼里舒过来。妇人一见,笑的要不得,骂道:“怪贼牢拉的短命,猛可舒出你老子头来,唬了我一跳。你趁早好好抽进去,我好不好拿针刺与你一下子,教你忍痛哩!”敬济笑道:“你老人家这回儿又不待见他起来,你好歹打发他个好处去,也是你一点阴骘。”妇人骂道:“好个怪牢成久惯的囚根子!”一面向腰里摸出面青铜小镜来,放在窗棂上,假做匀脸照镜,一面用朱唇吞裹吮咂他那话,吮咂的这小郎君一点灵犀灌顶,满腔春意融心。正咂在热闹处,忽听得有人走的脚步儿响,这妇人连忙摘下镜子,走过一边。敬济便把那话抽回去。却不想是来安儿小厮走来,说:“傅大郎前边请姐夫吃饭哩。”敬济道:“教你傅大郎且吃着,我梳头哩,就来。”来安儿回去了。妇人便悄悄向敬济说:“晚夕你休往那里去了,在屋里,我使春梅叫你。好歹等我,有话和你说。”敬济道:“谨依来命。”妇人说毕,回房去了。敬济梳洗毕,往铺中自做买卖。不题。

不一时,天色晚来。那日,月黑星密,天气十分炎热。妇人令春梅烧汤热水,要在房中洗澡,修剪足甲。床上收拾衾枕,赶了蚊子,放下纱帐子,小篆内炷了香。春梅便叫:“娘不,今日是头伏,你不要些凤仙花染指甲?我替你寻些来。”妇人道:“你那里寻去?”春梅道:“我直往那边大院子里才有,我去拔几根来。娘教秋菊寻下杵臼,捣下蒜。”妇人附耳低言,悄悄分付春梅:“你就厢房中请你姐夫晚夕来,我和他说话。”春梅去了,这妇人在房中,比及洗了香肌,修了足甲,也有好一回。只见春梅拔了几颗凤仙花来,整叫秋菊捣了半日。妇人又与他他几钟酒吃,打发他厨下先睡了。妇人灯光下染了十指春葱,令春梅拿凳子放在天井内,铺着凉簟衾枕纳凉。约有更阑时分,但见朱户无声,玉绳低转,牵牛、织女二星隔在天河两岸。又忽闻一阵花香,几点萤火。妇人手拈纨扇,伏枕而待。春梅把角门虚掩。正是:

\[
待月西厢下,迎风户半开。
隔墙花影动,疑是玉人来。
\]

原来敬济约定摇木瑾花树为号,就知他来了。妇人见花枝摇影,知是他来,便在院内咳嗽接应。他推开门进来,两个并肩而坐。妇人便问:“你来,房中有谁?”敬济道:“大姐今日没出来,我已分付元宵儿在房里,有事先来叫我。”因问:“秋菊睡了?”妇人道:“已睡熟了。”说毕,相搂相抱,二人就在院内凳上,赤身露体,席上交欢。不胜缱绻。但见:

\[
情兴两和谐,搂定香肩脸揾腮。手捻香乳绵似软,实奇哉!掀起脚儿脱绣鞋,玉体着郎怀。舌送丁香口便开,倒凤填鸾云雨罢,嘱多才:明朝千万早些来。
\]

两个云雨毕,妇人拿出五两碎银子来,递与敬济说:“门外你潘姥姥死了,棺材已是你爹在日与了他。三日入殓时,你大娘教我去探丧烧纸来了。明日出殡,你大娘不放我去,说你爹热孝在身,只见出门。这五两银子交与你,明早央你蚤去门外发送发送你潘姥姥,打发抬钱,看着下入土内,你来家。就同我去一般。”这敬济一手接了银子,说:“这个不打紧。我明日绝早就出门,干毕事,来回你老人家。”说毕,恐大姐进房,老早归厢房中去了。

一宿晚景休题。到次日,到饭时就来家。金莲才起来,在房中梳头。敬济走来回话,就门外昭化寺里,拿了两枝茉莉花儿来妇人戴。妇人问:“棺材下了葬了?”敬济道:“我管何事,不打发他老人家黄金入了柜,我敢来回话!还剩了二两六七钱银子,交付与你妹子收了,盘缠度日。千恩万谢,多多上覆你。”妇人听见他娘入土,落下泪来。便叫春梅:“把花儿浸在盏内,看茶来与你姐夫吃。”不一时,两盒儿蒸酥,四碟小菜,打发敬济吃了茶,往前边去了。由是越发与这小伙儿日亲日近。

一日,七月天气,妇人早辰约下他:“你今日休往那里去,在房中等着,我往你房里,和你顽耍。”这敬济答应了,不料那日被崔本邀了他,和几个朋友往门外耍子。去了一日,吃的大醉来家,倒在床上就睡着了,不知天高地下。黄昏时分,金莲蓦地到他房中,见他挺在床上,推他推不醒,就知他在那里吃了酒来。可霎作怪,不想妇人摸到他袖子里,吊下一根金头莲瓣簪儿来,上面趿着两溜字儿:“金勒马嘶芳草地,玉楼人醉杏花天。”迎亮一看,认的是孟玉楼簪子:“怎生落在他袖中?想必他也和玉楼有些首尾。不然,他的簪子如何他袖着?怪道这短命,几次在我面上无情无绪。我若不留几个字儿与他,只说我没来。等我写四句诗在壁上,使他知道。待我见了,慢慢追问他下落。”于是取笔在壁上写了四句。诗曰:

\[
独步书斋睡未醒,空劳神女下巫云。
襄王自是无情绪,辜负朝朝暮暮情。
\]

写毕,妇人回房去了。却说敬济一觉酒醒起来,房中掌上灯,因想起今日妇人来相会,我却醉了。回头见壁上写了四句诗在壁上,墨迹犹新,念了一遍,就知他来到,空回去了。心中懊悔不已。“这咱已是起更时分,大姐、元宵儿都在后边未出来,我若往他那边去,角门又关了。”走来木槿花下,摇花枝为号,不听见里面动静,不免踩着太湖石扒过粉墙去。那妇人见他有酒,醉了挺觉,大恨归房,闷闷在心,就浑衣上床歪睡。不料半夜他扒过墙来,见院内无人,想丫鬟都睡了,悄悄蹑足潜踪走到房门首,见门虚掩,就挨身进来。窗间月色照见床上妇人独自朝里歪着,低声叫“可意人”,数声不应,说道:“你休怪我,今日崔大哥众朋友,邀了我往门外五里原庄上射箭耍子了一日,来家就醉了。不知你到,有负你之约,恕罪恕罪。”那妇人也不理他。敬济见他不理,慌了,一面跪在地下,说了一遍又重复一遍。被妇人反手望脸上挝了一下,骂道:“贼牢拉负心短命,还不悄悄的,丫头听见!我知道你有了人,把我不放到心上。你今日端的那去来?”敬济道:“我本被崔大哥拉了门外射箭去,灌醉了来,就睡着了,失误你约,你休恼。我看见你留诗在壁上,就知恼了你。”妇人道:“怪捣鬼牢拉的,别要说嘴,与我禁声!你捣的鬼如泥弹儿圆,我手内放不过。你今日便是崔本叫了你吃酒,醉了来家,你袖子里这根簪子,却是那里的?”敬济道:“是那日花园中拾的,今两三日了。”妇人道:“你还\textuni{34B2}神捣鬼,是那花园里拾的?你再拾一根来,我才信你。这簪子是孟碱儿那麻淫妇的头上簪子,我认的千真万真,上面还趿着他名字,你还哄我。嗔道前日我不在,他叫你房里吃饭,原来你和他七个八个。我问你,还不肯认。你不和他两个有首尾,他的簪子缘何到你手里?原来把我的事都透露与他,怪道他前日见了我笑,原来有你的话在里头。自今以后,你是你,我是我,绿豆皮儿——请退了。”敬济听了,急的赌神发咒,继之以哭,道:“我敬济若与他有一字丝麻皂线,灵的是东岳城隍,活不到三十岁,生来碗大疔疮,害三五年黄病,要汤不汤,要水不水。”那妇人终是不信,说道:“你这贼才料,说来的牙疼誓,亏你口内不害碜!”两个絮聒了一回,见夜深了,不免解卸衣衫,挨身上床躺下。那妇人把身子扭过,倒背着他,使个性儿不理他,由着他姐姐长、姐姐短,只是反手望脸上挝过去。唬的敬济气也不敢出一口儿来,干霍乱了一夜。将天明,敬济恐怕丫头起身,依旧越墙而过,往前边厢房中去了。正是:

\[
三光有影遣谁系?万事无根只自生。
\]
