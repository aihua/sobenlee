%# -*- coding:utf-8 -*-
%%%%%%%%%%%%%%%%%%%%%%%%%%%%%%%%%%%%%%%%%%%%%%%%%%%%%%%%%%%%%%%%%%%%%%%%%%%%%%%%%%%%%


\chapter*{序}

\begin{fzliukai}
\zihao{-4}\setlength\parskip{\baselineskip-\ccwd}
《金瓶梅》,秽书也。袁石公亟称之,亦自寄其牢骚耳,非有取于《金瓶梅》也。然作者亦自有意,盖为世戒,非为世劝也。如诸妇多矣,而独以潘金莲、李瓶儿、春梅命名者,亦楚《梼杌》之意也。盖金莲以奸死,瓶儿以孽死,春梅以淫死,较诸妇为更惨耳。借西门庆以描画世之大净,应伯爵以描绘世之小丑,诸淫妇以描画世之丑婆、净婆,令人读之汗下。盖为世戒,非为世劝也。

余尝曰:“读《金瓶梅》而生怜悯心者,菩萨也;生畏惧心者,君子也;生欢喜心者,小人也;生效法心者,乃禽兽耳。”余友人褚孝秀偕一少年同赴歌舞之筵,衍至霸王夜宴,少年垂涎曰:“男儿何可不如此!”褚孝秀曰:“也只为这乌江设此一着耳。”同座闻之,叹为有道之言。若有人识得此意,方许他读《金瓶梅》也。不然,石公几为导淫宣欲之尤矣。奉劝世人,勿为西门之后车可也。

\bigskip\mbox{}\fzqiti\large\hfill 东吴弄珠客题\KG
\end{fzliukai}

\cleardoublepage

\tableofcontents
