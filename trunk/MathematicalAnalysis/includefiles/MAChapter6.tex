%# -*- coding:utf-8 -*-
%%%%%%%%%%%%%%%%%%%%%%%%%%%%%%%%%%%%%%%%%%%%%%%%%%%%%%%%%%%%%%%%%%%%%%%%%%%%%%%%%%%%%
%%  MAChapter6.tex'


\chapter{不定积分}\label{ch:6}
\section{不定积分概念}
\begin{exercise}
\item
\end{exercise}
\section{积分表与线性性质}
\begin{exercise}
\item
\end{exercise}
\section{换元法}
\subsection{第一换元法}
\subsection{第二换元法}
\begin{exercise}
\item
\end{exercise}
\section{分部积分法}
\begin{exercise}
\item
\end{exercise}
\section{有理函数的积分}
\begin{exercise}
\item
\end{exercise}
\section{三角函数有理式的积分}
\begin{exercise}
\item
\end{exercise}
\section{无理函数的积分}
\subsection(m 次根号积分){$\dps\int R\mparenbb{x,\kern-1ex\LEFTROOT{-5}\UPROOT{12}\SQRT[m]{\frac{ax+b}{\smash[b]{cx+d}}}\enspace}\diff x$~型积分}
\subsection{二项式微分式积分}
\subsection(二次根号积分){$\dps\int R\mparenb{x,\sqrt{ax^2+bx+c}}\diff x$~型积分}
\begin{exercise}
\item
\end{exercise}
\begin{exercise*}
\item
\end{exercise*}


\endinput
%%
%% End of file `MAChapter6.tex'.