%# -*- coding:utf-8 -*-
%%%%%%%%%%%%%%%%%%%%%%%%%%%%%%%%%%%%%%%%%%%%%%%%%%%%%%%%%%%%%%%%%%%%%%%%%%%%%%%%%%%%%


\chapter{西门庆踏雪访爱月\KG 贲四嫂带水战情郎}


词曰:

\[
梅其雪,岁暮斗新妆。月底素华同弄色,风前轻片半含香,不比柳花狂。双雀影,堪比雪衣娘。六出光中曾结伴,百花头上解寻芳,争似两鸳鸯。
\]

话说温秀才求见西门庆不得,自知惭愧,随移家小,搬过旧家去了。西门庆收拾书院,做了客坐,不在话下。

一日,尚举人来拜辞,上京会试,问西门庆借皮箱毡衫。西门庆陪坐待茶,因说起乔大户、云理守:“两位舍亲,一受义官,一受祖职,见任管事,欲求两篇轴文奉贺。不知老翁可有相知否?借重一言,学生具币礼相求。”尚举人笑道:“老翁何用礼,学生敝同窗聂两湖,见在武库肄业,与小儿为师,本领杂作极富。学生就与他说,老翁差盛使持轴来就是了。”西门庆连忙致谢。茶毕起身。西门庆随即封了两方手帕、五钱白金,差琴童送轴子并毡衫、皮箱,到尚举人处放下。那消两日,写成轴文差人送来。西门庆挂在壁上,但见金字辉粕,文不加点,心中大喜。只见应伯爵来问:“乔大户与云二哥的事,几时举行?轴文做了不曾?温老先儿怎的连日不见?”西门庆道:“又题什么温老先儿,通是个狗类之人!”如此这般,告诉一遍。伯爵道:“哥,我说此人言过其实,虚浮之甚,早时你有后眼,不然,教他调坏了咱家小儿每了。”又问他:“二公贺轴,何人写了?”西门庆道:“昨日尚小塘来拜我,说他朋友聂两湖善于词藻,央求聂两湖作了。文章已写了来,你瞧!”于是引伯爵到厅上观看,喝采不已,又说道:“人情都全了,哥,你早送与人家,好预备。”西门庆道:“明日好日期,早差人送去。”

正说着,忽报:“夏老爹儿来拜辞,说初六日起身去。小的回爹不在家。他说教对何老爹那里说声,差人那边看守去。”西门太看见贴儿上写着“寅家晚生夏承恩顿首拜,谢辞”。西门庆道:“连尚举人搭他家,就是两分程仪香绢。”分付琴童:“连忙买了,教你姐夫封了,写贴子送去。”正在书房中留伯爵吃饭,忽见平安儿慌慌张张拿进三个贴儿来报:“参议汪老爹、兵备雷老爹、郎中安老爹来拜。”西门庆看贴儿:“汪伯彦、雷启元、安忱拜。”连忙穿衣系带。伯爵道:“哥,你有事,我去罢。”西门庆道:“我明日会你哩。”一面整衣出迎。三官员皆相让而入。进入大厅,叙礼,道及向日叨扰之事。少顷茶罢,坐话间,安郎中便道:“雷东谷、汪少华并学生,又来干渎:有浙江本府赵大尹,新升大理寺正,学生三人借尊府奉请,已发柬,定初九日。主家共五席。戏子学生那里叫来。未知肯允诺否?”西门庆道:“老先生分付,学生扫门拱候。”安郎中令吏取分资三两递上,西门庆令左右收了,相送出门。雷东谷向西门庆道:“前日钱云野书到,说那孙文相乃是贵伙计,学生已并他除开了,曾来相告不曾?”西门庆道:“正是,多承老先生费心,容当叩拜。”雷兵备道:“你我相爱间,何为多数。”言毕,相揖上轿而去。

原来潘金莲自从当家管理银钱,另定了一把新等子。每日小厮买进菜蔬来,拿到跟前与他瞧过,方数钱与他。他又不数,只教春梅数钱,提等子。小厮被春鸿骂的狗血淋头,行动就说落,教西门庆打。以此众小厮互相抱怨,都说在三娘手儿里使钱好。

却说次日,西门庆衙门中散了,对何千户说:“夏龙溪家小已是起身去了,长官可曾委人那里看守门户去?”何千户道:“正是,昨日那边着人来说,学生已令小价去了。”西门庆道:“今日同长官那边看看去。”于是出衙门,并马到了夏家宅内。家小已是去尽了,伴当在门首伺候。两位官府下马,进到厅上。西门庆引着何千户前后观看了,又到前边花亭上,见一片空地,无甚花草。西门庆道:“长官到明日还收拾个耍子所在,栽些花柳,把这座亭子修理修理。”何千户道:“这个已定。学生开春从新修整修整,盖三间卷棚,早晚请长官来消闲散闷。”看了一回,分付家人收拾打扫,关闭门户。不日写书往东京回老公公话,赶年里搬取家眷。西门庆作别回家。何千户还归衙门去了。到次日才搬行李来住,不在言表。

西门庆刚到家下马,见何九买了一匹尺头、四样下饭、一坛酒来谢。又是刘内相差人送了一食盒蜡烛,二十张桌围,八十股官香,一盒沉速料香,一坛自造内酒,一口鲜猪。西门庆进门,刘公公家人就磕头,说道:“家公多多上履,这些微礼,与老爹赏人。”西门庆道:“前日空过老公公,怎又送这厚礼来?”便令左右:“快收了,请管家等等儿。”少顷,画童儿拿出一钟茶来,打发吃了。西门庆封了五钱银子赏钱,拿回贴,打发去了。一面请何九进去。西门庆见何九,一把手扯在厅上来。何九连忙倒身磕下头去,道:“多蒙老爹天心,超生小人兄弟,感恩不浅。”请西门庆受礼,西门庆不肯受磕头,拉起来,说道:“老九,你我旧人,快休如此。”就让他坐。何九说道:“小人微末之人,岂敢僭坐。”只说立在旁边。西门庆也站着,陪吃了一盏茶,说道:“老九,你如何又费心送礼来?我断然不受,若有甚么人欺负你,只顾来说,我替你出气。倘县中派你甚差事,我拿贴儿与你李老爹说。”何九道:“蒙老爹恩典,小人知道。小人如今也老了,差事已告与小人何钦顶替了。”西门庆道:“也罢,也罢,你清闲些好。”又说道:“既你不肯,我把这酒礼收了,那尺头你还拿去,我也不留你坐了。”那何九千恩万谢,拜辞去了。

西门庆就坐在厅上,看看打点礼物果盒、花红羊酒、轴文并各人分资。先差玳安送往乔大户家去,后叫王经送往云理守家去。玳安回来,乔家与了五钱银子。王经到云理守家,管待了茶食,与了一匹真青大布、一双琴鞋,回“门下辱爱生”双贴儿:“多上覆老爹,改日奉请。”西门庆满心欢喜,到后边月娘房中摆饭吃,因向月娘说:“贲四去了,吴二舅在狮子街卖货,我今日倒闲,往那里看看去。”月娘道:“你去不是,若是要酒菜儿,蚤使小厮来家说。”西门庆道:“我知道。”一面分付备马,就戴着毡忠靖巾,貂鼠暖耳,绿绒补子氅褶,粉底皂靴,琴童、玳安跟随,径往狮子街来。到房子内,吴二舅与来昭正挂着花拷拷儿,发买绸绢、绒线、丝绵,挤一铺子人做买卖,打发不开。西门庆下马,看了看,走到后边暖房内坐下。吴二舅走来作揖,因说:“一日也攒银二三十两。”西门庆又分付来昭妻一丈青:“二舅每日茶饭休要误了。”来昭妻道:“逐日伺候酒饭,不敢有误。”

西门庆见天色阴晦,彤云密布,冷气侵人,将有作雪的模样。忽然想起要往郑月儿家去,即令琴童:“骑马家中取我的皮袄来,问你大娘,有酒菜儿稍一盒与你二舅吃。”琴童应诺。到家,不一时,取了貂鼠皮袄,并一盒酒菜来。西门庆陪二舅在房中吃了三杯,分付:“二舅,你晚夕在此上宿,慢慢再用。我家去罢。”于是带上眼纱,骑马,玳安、琴童跟随,径进构栏,往郑爱月儿家来。转过东街口,只见天上纷纷扬扬,飘起一天瑞雪来。但见:

\[
漠漠严寒匝地,这雪儿下得正好。扯絮撏绵,裁成片片,大如拷拷。见林间竹笋茆茨,争些被他压倒。富豪侠却言:消灾障犹嫌少。围向那红炉兽炭,穿的是貂裘绣袄。手拈梅花,唱道是国家祥瑞,不念贫民些小。高卧有幽人,吟咏多诗草。
\]

西门庆踏着那乱琼碎玉,进入构栏,到于郑爱月儿家门首下马。只见丫鬟飞报进来,说:“老爹来了。”郑妈妈看见,出来,至于中堂见礼,说道:“前日多谢老爹重礼,姐儿又在宅内打搅,又教他大娘、三娘赏他花翠汗巾。”西门庆道:“那日空了他来。”一面坐下。西门庆令玳安:“把马牵进来,后边院落安放。”老妈道:“请爹后边明间坐罢。月姐才起来梳头,只说老爹昨日来,到伺候了一日,今日他心中有些不快,起来的迟些。”这西门庆一面进入他后边明间内,但见绿穿半启、毡幕低张,地平上黄铜大盆生着炭火。西门庆坐在正面椅上。先是郑爱香儿出来相见了,递了茶。然后爱月儿才出来,头挽一窝丝杭州缵,翠梅花钮儿,金趿钗梳,海獭卧兔儿。打扮的雾霭云鬟,粉妆玉琢。笑嘻嘻向西门庆道了万福,说道:“爹,我那一日来晚了。紧自前边散的迟,到后边,大娘又只顾不放俺每,留着吃饭,来家有三更天了。”西门庆笑道:“小油嘴儿,你倒和李桂姐两个把应花子打的好响瓜儿。”郑爱月儿道:“谁教他怪叨唠,在酒席上屎口儿伤俺每来!那一日祝麻子也醉了,哄我,要送俺每来。我便说:‘没爹这里灯笼送俺每,蒋胖子吊在阴沟里——缺臭了你了。’”西门庆道:“我昨日听见洪四儿说,祝麻子又会着王三官儿,大街上请了荣娇儿。”郑月儿道:“只在荣娇儿家歇了一夜,烧了一炷香,不去了。如今还在秦玉芝儿走着哩。”说了一回话,道:“爹,只怕你冷,往房里坐。”

这西门庆到于房中,脱去貂裘,和粉头围炉共坐,房中香气袭人。须臾,丫头拿了三瓯儿黄芽韭菜肉包、一寸大的水角儿来。姊妹二人陪西门庆,每人吃了一瓯儿。爱月儿又拨上半瓯儿,添与西门庆。西门庆道:“我勾了,才吃了两个点心来了。心里要来你这里走走,不想恰好天气又落下雪来了。”爱月儿道:“爹前日不会下我?我昨日等了一日不见爹,不想爹今日才来。”西门庆道:“昨日家中有两位士夫来望,乱着就不曾来得。”爱月儿道:“我要问爹,有貂鼠买个儿与我,我要做了围脖儿戴。”西门庆道:“不打紧,昨日韩伙计打辽东来,送了我几个好貂鼠。你娘们都没围脖儿,到明日一总做了,送两个一家一个。”于是爱香、爱月儿连忙起身道了万福。西门庆分付:“休见了桂姐、银姐说。”郑月儿道:“我知道。”因说:“前日李桂姐见吴银儿在那里过夜,问我他几时来的,我没瞒他,教我说:‘昨日请周爷,俺每四个都在这里唱了一日。爹说有王三官儿在这里,不好请你的。今日是亲朋会中人吃酒,才请你来唱。’他一声儿也没言语。”西门庆道:“你这个回的他好。前日李铭,我也不要他唱来,再三央及你应二爹来说。落后你三娘生日,桂姐买了一分礼来,再一与我陪不是。你娘们说着,我不理他。昨日我竟留下银姐,使他知道。”爱月儿道:“不知三娘生日,我失误了人情。”西门庆道:“明日你云老爹摆酒,你再和银姐来唱一日。”爱月儿道:“爹分付,我去。”说了回话,粉头取出三十二扇象牙牌来,和西门庆在炕毡条上抹牌顽耍。爱香儿也坐在旁边同抹。三人抹了回牌,须臾,摆上酒来,爱香与爱月儿一边一个捧酒,不免筝排雁柱,款跨鲛绡,姊妹两个弹唱。唱了一套,姐妹两个又拿上骰盆儿来,和西门庆抢红顽笑。杯来盏去,各添春色。西门庆忽看见郑爱月儿房中,床旁侧锦屏风上,挂着一轴《爱月美人图》,题诗一首:

\[
有美人兮迥出群,轻风斜拂石榴裙。
花开金谷春三月,月转花阴夜十分。
玉雪精神联仲琰,琼林才貌过文君。
少年情思应须慕,莫使无心托白云。
\]

西门庆看了,便问:“三泉主人是王三官儿的号?”慌的郑爱月儿连忙摭说道:“这还是他旧时写下的。他如今不号三泉了,号小轩了。他告人说,学爹说:‘我号四泉,他怎的号三泉?’他恐怕爹恼,因此改了号小轩。”一面走向前,取笔过来,把那“三”字就涂抹了。西门庆满心欢喜,说道:“我并不知他改号一节。”粉头道:“我听见他对一个人说来,我才晓的。说他去世的父亲号逸轩,他故此改号小轩。”说毕,郑爱香儿往下边去了,独有爱月儿陪西门庆在房内。两个并肩叠股,抢红饮酒,因说起林太太来,怎的大量,好风月:“我在他家吃酒,那日王三官请我到后边拜见。还是他主意,教三官拜认我做义父,教我受他礼,委托我指教他成人。”粉头拍手大笑道:“还亏我指与爹这条路儿,到明日,连三官儿娘子不怕不属了爹。”西门庆道:“我到明日,我先烧与他一炷香。到正月里,请他和三官娘子往我家看灯吃酒,看他去不去。”粉头道:“爹,你还不知三官娘子生的怎样标致,就是个灯人儿也没他那一段风流妖艳。今年十九岁儿,只在家中守寡,王三官儿通不着家。爹,你肯用些工夫儿,不愁不是你的人。”两个说话之间,相挨相凑。只见丫鬟又拿上许多细果碟儿来,粉头亲手奉与西门庆下酒。又用舌头噙凤香蜜饼送入他口中,又用纤手解开西门庆裤带,露出那话来,教他弄。那话狰狞跳脑,紫强光鲜,西门庆令他品之。这粉头真个低垂粉项,轻启朱唇,半吞半吐,或进或出,呜咂有声,品弄了一回。灵犀已透,淫心似火,便欲交欢。粉头便往后边去了。西门庆出房更衣,见雪越下得甚紧。回到房中,丫鬟向前打发脱靴解带,先上牙床。粉头澡牝回来,掩上双扉,共入鸳帐。正是:得多少动人春色娇还媚,惹蝶芳心软欲浓。有诗为证:

\[
聚散无凭在梦中,起来残烛映纱红。
钟情自古多神合,谁道阳台路不通。
\]

两个云雨欢娱,到一更时分起来。整衣理鬓,丫鬟复酾美酒,重整佳肴,又饮勾几杯。问玳安:“有灯笼、伞没有?”玳安道:“琴童家去取灯笼、伞来了。”这西门庆方才作别,鸨子、粉头相送出门,看着上马。郑月儿扬声叫道:“爹若叫我,蚤些来说。”西门庆道:“我知道。”一面上马,打着伞出院门,一路踏雪到家中。对着吴月娘,只说在狮子街和吴二舅饮酒,不在话下。一宿晚景题过。

到次日,却是初八日,打听何千户行李,都搬过夏家房子内去了,西门庆送了四盒细茶食、五钱折帕贺仪过去。只见应伯爵蓦地走来。西门庆见雪晴,风色甚冷,留他前边书房中向火,叫小厮拿菜儿,留他吃粥,因说道:“昨日乔亲家、云二哥礼并折帕,都送去了。你的人情,我也替你封了二钱出上了。你不消与他罢,只等发柬请吃酒。”应伯爵举手谢了,因问:“昨日安大人三位来做甚么?那两位是何人?”西门庆道:“那两个,一个是雷兵备,一个是汪参议,都是浙江人,要在我这里摆酒。明日请杭州赵霆知府,新升京堂大理寺丞,是他每本府父母官,相处分上,又不可回他的。通身只三两分资。”伯爵道:“大凡文职好细,三两银子勾做甚么!哥少不得赔些儿。”西门庆道:“这雷兵备,就是问黄四小舅子孙文相的,昨日还对我题起开除他罪名哩。”伯爵道:“你说他不仔细,如今还记着,折准摆这席酒才罢了。”

说话之间,伯爵叫:“应宝,你叫那个人来见你大爹。”西门庆便问:“是何人?”伯爵道:“一个小后生,倒也是旧人家出身。父母都没了,自幼在王皇亲宅内答应。已有了媳妇儿,因在庄子上和一般家人不和,出来了。如今闲着,做不的甚么。他与应宝是朋友,央及应宝要投个人家。今早应宝对我说:‘爹倒好举荐与大爹宅内答应。’我便说:‘不知你大爹用不用?’”因问应宝:“他叫甚么名字?你叫他进来。”应宝道:“他姓来,叫来友儿。”只见那来友儿,扒在地上磕了个头起来,帘外站立。伯爵道:“若论他这身材膂力尽有,掇轻负重却去的。”因问:“你多少年纪了?”来友儿道:“小的二十岁了。”又问:“你媳妇没子女?”那人道:“只光两口儿。”应宝道:“不瞒爹说,他媳妇才十九岁儿,厨灶针线,大小衣裳都会做。”西门庆见那人低头并足,为人朴实,便道:“既是你应二爹来说,用心在我这里答应。”分付:“拣个好日期,写纸文书,两口儿搬进来罢。”那来友儿磕了个头。西门庆就叫琴童儿领到后边,见月娘众人磕头去。月娘就把来旺儿原住的那一间房与他居住。伯爵坐了回,家去了。应宝同他写了一纸投身文书,交与西门庆收了,改名来爵,不在话下。

却说贲四娘子,自从他家长儿与了夏家,每日买东买西,只央及平安儿和来安、画童儿。西门庆家中这些大官儿,常在他屋里打平和儿吃酒。贲四娘子和气,就定出菜儿来,或要茶水,应手而至。就是贲四一时铺中归来撞见,亦不见怪。以此今日他不在家,使着那个不替他动?玳安儿与平安儿,在他屋里坐的更多。

初九日,西门庆与安郎中、汪参议、雷兵备摆酒,请赵知府,俱不必细说。那日蚤辰,来爵两口儿就搬进来。他媳妇儿后边见月娘众人磕头。月娘见他穿着紫绸袄,青布披袄,绿布裙子,生的五短身材,瓜子面皮儿,搽脂抹粉,缠的两只脚翘翘的,问起来,诸般针指都会做。取了他个名字,叫做惠元,与惠秀、惠祥一递三日上灶,不题。

一日,门外杨姑娘没了。安童儿来报丧。西门庆整治了一张插桌,三牲汤饭,又封了五两香仪。吴月娘、李娇儿、孟玉楼、潘金莲四顶轿子,都往北边与他烧纸吊孝,琴童儿、棋童儿、来爵儿、来安儿四个,都跟轿子,不在家。西门庆在对过段铺子书房内,看着毛袄匠与月娘做貂鼠围脖,先攒出一个围脖儿,使玳安送与院中郑月儿去,封了十两银子与他过节。郑家管待酒馔,与了他三钱银子。玳安走来,回西门庆话,说:“月姨多上覆,多谢了,前日空过了爹来。与了小的三钱银子。”西门庆道:“你收了罢。”因问他:“贲四不在家,你头里从他屋里出来做甚么?”玳安道:“贲四娘子从他女孩儿嫁了,没人使,常央及小的每替他买买甚么儿。”西门庆道:“他既没人使,你每替他勤勤儿也罢。”又悄悄向玳安道:“你慢慢和他说,如此这般,爹要来看你看儿,你心下如何?看他怎的说。他若肯了,你问他讨个汗巾儿来与我。”玳安道:“小的知道了。”领了西门庆言语,应诺下去。西门庆就走到家中来。只见王经向顾银铺内取了金赤虎,并四对金头银簪儿,交与西门庆。西门庆留下两对在书房内,余者袖进李瓶儿房内,与了如意儿那赤虎,又是一对簪儿。把那一对簪儿就与了迎春。二人接了,连忙磕头。西门庆就令迎春取饭去。须臾,拿饭来吃了,出来又到书房内坐下。只见玳安慢慢走到跟前,见王经在旁,不言语。西门庆使王经后边取茶去。那玳安方说:“小的将爹言语对他说了,他笑了。约会晚上些伺候,等爹进去。叫小的拿了这汗巾儿来。”西门庆见红绵纸儿,包着一方红绫织锦回纹汗巾儿,闻了闻喷鼻香,满心欢喜,连忙袖了。只见王经拿茶来,吃了,又走过对门,看匠人做生活去。

忽报:“花大舅来了。”西门庆道:“请过来这边坐。”花子繇走到书房暖阁儿里,作揖坐下。致谢外日相扰。叙话间,画童儿拿过茶来吃了。花子繇道:“门外一个客人,有五百包无锡米,冻了河,紧等要卖了回家去。我想着姐夫,倒好买下等价钱。”西门庆道:“我平白要他做甚么?冻河还没人要,到开河船来了,越发价钱跌了。如今家中也没银子。”即分付玳安:“收拾放桌儿,家中说,看菜儿来。”一面使画童儿:“请你应二爹来,陪你花爹坐。”不一时,伯爵来到。三人共在一处,围炉饮酒。又叫烙了两炷饼吃,良久,只见吴道官徒弟应春,送节礼疏诰来。西门庆请来同坐吃酒。就揽李瓶儿百日经,与他银子去。吃至日落时分,花子繇和应春二人先起身去了。次后甘伙计收了铺子,又请来坐,与伯爵掷骰猜枚谈话,不觉到掌灯已后。吴月娘众人轿子到了,来安走来回话。伯爵道:“嫂子们今日都往那里去来?”西门庆道:“杨姑娘没了,今日三日念经,我这里备了张祭卓,又封了香仪儿,都去吊问。”伯爵道:“他老人家也高寿了。”西门庆道:“敢也有七十五六。男花女花都没有,只靠侄儿那里养活,材儿也是我替他备下这几年了。”伯爵道:“好好,老人家有了黄金入柜,就是一场事了,哥的大阴骘。”说毕,酒过数巡,伯爵与甘伙计作辞去了。西门庆就起身走过来,分付后生王显:“仔细火烛。”王显道:“小的知道。”看着把门关上了。

这西门庆见没人,两天步就走入贲四家来。只见卉四娘子儿在门首独自站立已久,见对门关的门响,西门庆从黑影中走至跟前。这妇人连忙把封门一开,西门庆钻入里面。妇人还扯上封门,说道:“爹请里边纸门内坐罢。”原来里间槅扇厢着后半间,纸门内又有个小炕儿,笼着旺旺的火。桌上点着灯,两边护炕糊的雪白。妇人勒着翠蓝销金箍儿,上穿紫绸袄,青绡丝披袄,玉色绡裙子,向前与西门庆道了万福,连忙递了一盏茶与西门庆吃,因悄悄说:“只怕隔壁韩嫂儿知道。”西门庆道:“不妨事。黑影子里他那里晓的。”于是不由分说,把妇人搂到怀中就亲嘴。拉过枕头来,解衣按在炕沿子上,扛起腿来就耸。那话上已束着托子,刚插入牝中,就拽了几拽,妇人下边淫水直流,把一条蓝布裤子都湿了。西门庆拽出那话来,向顺袋内取出包儿颤声娇来,蘸了些在龟头上,攮进去,方才涩住淫津,肆行抽拽。妇人双手扳着西门庆肩膊,两厢迎凑,在下扬声颤语,呻吟不绝。这西门庆乘着酒兴,架起两腿在胳膊上,只顾没棱露脑,锐进长驱,肆行扇蹦,何止二三百度。须臾,弄的妇人云髻蓬松,舌尖冰冷,口不能言。西门庆则气喘吁吁,灵龟畅美,一泄如注。良久,拽出那话来,淫水随出,用帕搽之。两个整衣系带,复理残妆。西门庆向袖中掏出五六两一包碎银子,又是两对金头簪儿,递与妇人节间买花翠带。妇人拜谢了,悄悄打发出来。那边玳安在铺子里,专心只听这边门环儿响,便开大门,放西门庆进来。自知更无一人晓的。后次朝来暮往,也入港一二次。正是:若要人不知,除非己莫为。不想被韩嫂儿冷眼睃见,传的后边金莲知道了。这金莲亦不说破他。

一日,腊月十五日,乔大户家请吃酒。西门庆会同应伯爵、吴大舅一齐起身。那日有许多亲朋看戏饮酒,至二更方散。第二日,每家一张卓面,俱不必细说。

单表崔本治了二千两湖州绸绢货物,腊月初旬起身,雇船装载,赶至临清马头。教后生荣海看守货物,便雇头口来家,取车锐银两,到门首下头口。琴童道:“崔大哥来了,请厅上坐。爹在对门房子里,等我请去。”一面走到对门,不见西门庆,因问平安儿,平安儿道:“爹敢进后边去了。”这琴童走到上房问月娘,月娘道:“见鬼的,你爹从蚤辰出去,再几时进来?”又到各房里,并花园、书房都瞧遍了,没有。琴童在大门首扬声道:“省恐杀人,不知爹往那里去了,白寻不着!大白日里把爹来不见了。崔大哥来了这一日,只顾教他坐着。”那玳安分明知道,只不做声。不想西门庆忽从前边进来,把众人唬了一惊。原来西门庆在贲四屋里入港,才出来。那平安打发西门庆进去了,望着琴童儿吐舌头,都替他捏两把汗道:“管情崔大哥去了,有几下子打。”不想西门庆走到厅上,崔本见了,磕头毕,交了书帐,说:“船到马头,少车税银两。我从腊月初一日起身,在扬州与他两个分路。他每往杭州去了,俺每都到苗青家住了两日。”因说:“苗青替老爹使了十两银子,抬了扬州卫一个千户家女子,十六岁了,名唤楚云。说不尽生的花如脸,玉如肌,星如眼,月如眉,腰如柳,袜如钩,两只脚儿,恰刚三寸。端的有沉鱼落雁之容,闭月羞花之豹。腹中有三千小曲,八百大曲。苗青如此还养在家,替他打妆奁,治衣服。待开春,韩伙计、保官儿船上带来,伏侍老爹,消愁解闷。”西门庆听了,满心欢喜,说道:“你船上稍了来也罢。又费烦他治甚衣服,打甚妆砹,愁我家没有?”于是恨不的腾云展翅,飞上扬州,搬取娇姿,赏心乐事。正是:鹿分郑相应难辨,蝶化庄周未可。有诗为证:

\[
闻道扬州一楚云,偶凭青鸟语来真。
不知好物都离隔,试把梅花问主人。
\]

西门庆陪崔本吃了饭,兑了五十两银子做车税钱,又写书与钱主事,烦他青目。崔本言讫,作辞,往乔大户家回话去了。平安见西门庆不寻琴童儿,都说:“我儿,你不知有多少造化。爹今日不知有甚事喜欢,若不是,绑着鬼有几下打。”琴童笑道:“只你知爹性儿。”

比及起了货,来到狮子街卸下,就是下旬时分。西门庆正在家打发送节礼,忽见荆都监差人拿贴儿来,问:“宋大巡题本已上京数日,未知旨意下来不曾?伏惟老翁差人察院衙门一打听为妙。”西门庆即差答应节级,拿了五钱银子,往巡按公衙打听。果然昨日东京邸报下来,写抄得一纸,全报来与西门庆观看。上面写着:

\[
山东巡按监察御史宋乔年一本:循例举劾地方文武官员,以励人心,以隆圣治事。窃惟吏以抚民,武以御乱,所以保障地方,以司民命者也。苟非其人,则处置乖方,民受其害,国何赖焉!臣奉命按临山东等处,吏政民瘼,监司守御,无不留心咨访。复命按抚大臣,详加鉴别,各官贤否,颇得其实。兹当差满之期,敢不一一陈之。访得山东左布政陈四箴操履忠贞,抚民有方;廉使赵讷,纲纪肃清,士民服习;兵备副使雷启元,军民咸服其恩威,僚幕悉推其练达;济南府知府张叔夜,经济可观,才堪司牧;东平府知府胡师父,居任清慎,视民如伤。此数臣者,皆当荐奖而优擢者也。又访得左参议冯廷鹄,伛偻之形,桑榆之景,形若木偶,尚肆贪婪;东昌府知府徐松,纵父妾而通贿,毁谤腾于公堂,慕羡余而诛求,詈言遍于间里。此二臣者,所当亟赐置斥者也。再访得左军院佥书守备周秀,器宇恢弘,操持老练,军心允服,贼盗潜消;济州兵马都监荆忠,年力精强,才犹练达,冠武科而称为儒将,胜算可以临戎,号令而极其严明,长策卒能御侮。此二臣者,所当亟赐迁擢者也。清河县千户吴铠,以练达之才,得卫守之法,驱兵以\textuni{22B4F}中坚,靡攻不克;储食以资粮饷,无人不饱。推心置腹,人思效命。实一方之保障,为国家之屏藩。宜特加超擢,鼓舞臣寮。陛下如以臣言可采,举而行之,庶几官爵不滥而人思奋,守牧得人而圣治有赖矣。等因。
奉饮依:该部知道。续该吏、兵二部题前事:看得御史宋乔年所奏内,劾举地方文武官员,无非体国之忠,出于公论,询访事实,以裨圣治之事。优乞圣明俯赐施行,天下幸甚,生民幸甚。奉钦依:拟行。
\]

西门庆一见,满心欢喜。拿着邸报,走到后边,对月娘说:“宋道长本下来了。已是保举你哥升指挥佥事,见任管屯。周守备与荆大人都有奖励,转副参、统制之任。如今快使小厮请他来,对他说声。”月娘道:“你使人请去,我交丫鬟看下酒菜儿。我愁他这一上任,也要银子使。”西门庆道:“不打紧,我借与他几两银子也罢了。”不一时,请得吴大舅到了。西门庆送那题奏旨意与他瞧。吴大舅连忙拜谢西门庆与月娘,说道:“多累姐夫、姐姐扶持,恩当重报,不敢有忘。”西门庆道:“大舅,你若上任摆酒没银子,我这里兑些去使。”那大舅又作揖谢了。于是就在月娘房中,安排上酒来吃酒。月娘也在旁边陪坐。西门庆即令陈敬济把全抄写了一本,与大舅拿着。即差玳安拿贴送邸报往荆都监、周守御两家报喜去。正是:

\[
劝君不费镌研石,路上行人口似碑。
\]
