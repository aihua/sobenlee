%# -*- coding:utf-8 -*-
%%%%%%%%%%%%%%%%%%%%%%%%%%%%%%%%%%%%%%%%%%%%%%%%%%%%%%%%%%%%%%%%%%%%%%%%%%%%%%%%%%%%%


\chapter{翟管家寄书寻女子\KG 蔡状元留饮借盘缠}


诗曰:

\[
既伤千里目,还惊远去魂。岂不惮跋涉?深怀国士恩。
季布无一诺,侯嬴重一言。人生感意气,黄金何足论。
\]

话说次日,西门庆早与夏提刑接了新巡按,又到庄上犒劳做活的匠人。至晚来家,平安进门就禀:“今日有东昌府下文书快手,往京里顺便捎了一封书帕来,说是太师爷府里翟大爹寄来与爹的。小的接了,交进大娘房里去了。那人明日午后来讨回书。”西门庆听了,走到上房,取书拆开观看,上面写着:

\[
京都侍生翟谦顿首书拜即擢大锦堂西门大人门下:久仰山斗,未接丰标,屡辱厚情,感愧何尽!前蒙驰谕,生铭刻在心。凡百于老爷左右,无不尽力扶持。所有小事,曾托盛价烦渎,想已为我处之矣。今日鸿便,薄具帖金十两奉贺,兼候起居。伏望俯赐回音,生不胜感激之至。外新状元蔡一泉,乃老爷之假子,奉敕回籍省视,道经贵处,仍望留之一饭,彼亦不敢有忘也。至祝至祝!秋后一日信。
\]

西门庆看毕,只顾咨嗟不已,说道:“快叫小厮叫媒人去。我什么营生,就忘死了。”吴月娘问:“甚么勾当?”西门庆道:“东京太师老爷府里翟管家,前日有书来,说无子,央及我这里替他寻个女子。不拘贫富,不限财礼,只要好的,他要图生长。妆奁财礼,该使多少,教我开了去,他一一还我,往后他在老爷面前,一力扶持我做官。我一向乱着上任,七事八事,就把这事忘死了。来保又日逐往铺子里去了,又不题我。今日他老远的教人捎书来,问寻的亲事怎样了。又寄了十两折礼银子贺我。明日差人就来讨回书,你教我怎样回答他?教他就怪死了!叫了媒人,你分咐他,好歹上紧替他寻着,不拘大小人家,只要好女儿,或十五六、十七八的也罢,该多少财礼,我这里与他。再不,把李大姐房里绣春,倒好模样儿,与他去罢。”月娘道:“我说你是个火燎腿行货子!这两三个月,你早做什么来?人家央你一场,替他看个真正女子去也好。那丫头你又收过他,怎好打发去的!你替他当个事干,他到明日也替你用的力。如今急水发,怎么下得浆?比不得买什么儿,拿了银子到市上就买的来了。一个人家闺门女子,好歹不同,也等着媒人慢慢踏看将来。你倒说的好自在话儿!”西门庆道:“明日他来要回书,怎么回答他?”月娘道:“亏你还断事!这些勾当儿,便不会打发人?等那人明日来,你多与他些盘缠,写书回复他,只说女子寻下了,只是衣服妆奁未办,还待几时完毕,这里差人送去。打发去了,你这里教人替他寻也不迟。此一举两得其便,才干出好事来,也是人家托你一场。”西门庆笑道:“说的有理!”一面叫将陈敬济来,隔夜修了回书。

次日,下书人来到,西门庆亲自出来,问了备细。又问蔡状元几时船到,好预备接他。那人道:“小人来时蔡老爹才辞朝,京中起身。翟爹说:只怕蔡老爹回乡,一时缺少盘缠,烦老爹这里多少只顾借与他。写书去,翟老爹那里如数补还。”西门庆道:“你多上复翟爹,随他要多少,我这里无不奉命。”说毕,命陈敬济让去厢房内管待酒饭。临去交割回书,又与了他五两路费。那人拜谢,欢喜出门,长行去了。看官听说:当初安忱取中头甲,被言官论他是先朝宰相安惇之弟,系党人子孙,不可以魁多士。徽宗不得已,把蔡蕴擢为第一,做了状元。投在蔡京门下,做了假子。升秘书省正事,给假省亲。且说月娘家中使小厮叫了老冯、薛嫂儿并别的媒人来,分咐各处打听人家有好女子,拿帖儿来说,不在话下。

一日,西门庆使来保往新河口,打听蔡状元船只,原来就和同榜进士安忱同船。这安进士亦因家贫未续亲,东也不成,西也不就,辞朝还家续亲,因此二人同船来到新河口。来保拿着西门庆拜帖来到船上见,就送了一分下程,酒面、鸡鹅、下饭、盐酱之类。蔡状元在东京,翟谦已预先和他说了:“清河县有老爷门下一个西门千户,乃是大巨家,富而好礼。亦是老爷抬举,见做理刑官。你到那里,他必然厚待。”这蔡状元牢记在心,见面门庆差人远来迎接,又馈送如此大礼,心中甚喜。次日就同安进士进城来拜。西门庆已是预备下酒席。因在李知县衙内吃酒,看见有一起苏州戏子唱的好,旋叫了四个来答应。蔡状元那日封了一端绢帕、一部书、一双云履。安进士亦是书帕二事、四袋芽茶、四柄杭扇。各具宫袍乌纱,先投拜帖进去。西门庆冠冕迎接至厅上,叙礼交拜。献毕贽仪,然后分宾主而坐。先是蔡状元举手欠身说道:“京师翟云峰,甚是称道贤公阀阅名家,清河巨族。久仰德望,未能识荆,今得晋拜堂下,为幸多矣!”西门庆答道:“不敢!昨日云峰书来,具道二位老先生华辀下临,理当迎接,奈公事所羁,望乞宽恕。”因问:“二位老先生仙乡、尊号?”蔡状元道:“学生本贯滁州之匡庐人也。贱号一泉,侥幸状元,官拜秘书正字,给假省亲。”安进士道:“学生乃浙江钱塘县人氏。贱号凤山。见除工部观政,亦给假还乡续亲。敢问贤公尊号?”西门庆道:“在下卑官武职,何得号称。”询之再三,方言:“贱号四泉,累蒙蔡老爷抬举,云峰扶持,袭锦衣千户之职。见任理刑,实为不称。”蔡状元道:“贤公抱负不凡,雅望素著,休得自谦。”叙毕礼话,请去花园卷棚内宽衣。蔡状元辞道:“学生归心匆匆,行舟在岸,就要回去。既见尊颜,又不遽舍,奈何奈何!”西门庆道:“蒙二公不弃蜗居,伏乞暂住文旆,少留一饭,以尽芹献之情。”蔡状元道:“既是雅情,学生领命。”一面脱去衣服,二人坐下。左右又换了一道茶上来。蔡状元以目瞻顾因池台馆,花木深秀,一望无际,心中大喜,极口称羡道:“诚乃蓬瀛也!”于是抬过棋桌来下棋。西门庆道:“今日有两个戏子在此伺候,以供宴赏。”安进士道:“在那里?何不令来一见?”不一时,四个戏子跪下磕头。蔡状元问道:“那两个是生旦?叫甚名字?”内中一个答道:“小的妆生,叫苟子孝。那一个装旦的叫周顺。一个贴旦叫袁琰。那一个装小生的叫胡慥。”安进士问:“你们是那里子弟?”苟子孝道:“小的都是苏州人。”安进士道:“你等先妆扮了来,唱个我们听。”四个戏子下边妆扮去了。西门庆令后边取女衣钗梳与他,教书童也妆扮起来。共三个旦、两个生,在席上先唱《香囊记》。大厅正面设两席,蔡状元、安进士居上,西门庆下边主位相陪。饮酒中间,唱了一折下来,安进士看见书童儿装小旦,便道:“这个戏子是那里的?”西门庆道:“此是小价书童。”安进士叫上去,赏他酒吃,说道:“此子绝妙而无以加矣!”蔡状元又叫别的生旦过来,亦赏酒与他吃。因分咐:“你唱个《朝元歌》‘花边柳边’。”苟子孝答应,在旁拍手道:

\[
花边柳边,檐外晴丝卷。山前水前,马上东风软。自叹行踪,有如蓬转,盼望家乡留恋。雁杳鱼沉,离愁满怀谁与传?日短北堂萱,空劳魂梦牵。洛阳遥远,几时得上九重金殿?
\]
唱完了,安进士问书童道:“你们可记的《玉环记》‘恩德浩无边’?”书童答道:“此是《画眉序》,小的记得。”随唱道:

\[
恩德浩无边,父母重逢感非浅。幸终身托与,又与姻缘。风云会异日飞腾,鸾凤配今谐缱绻。料应夫妇非今世,前生种玉蓝田。
\]

原来安进士杭州人,喜尚男风,见书童儿唱的好,拉着他手儿,两个一递一口吃酒。良久,酒阑上来,西门庆陪他复游花园,向卷棚内下棋。令小厮拿两个桌盒,三十样都是细巧果菜、鲜物下酒。蔡状元道:“学生们初会,不当深扰潭府,天色晚了,告辞罢。”西门庆道:“岂有此理。”因问:“二公此回去,还到船上?”蔡状元道:“暂借门外永福寺寄居。”西门庆道:“如今就门外去也晚了。不如老先生把手下从者止留一二人答应,其余都分咐回去,明日来接,庶可两尽其情。”蔡状元道:“贤公虽是爱客之意,其如过扰何!”当下二人一面分咐手下,都回门外寺里歇去,明日早拿马来接。众人应诺去了,不在话下。

二人在卷棚内下了两盘棋,子弟唱了两折,恐天晚,西门庆与了赏钱,打发去了。止是书童一人,席前递酒伏侍。看看吃至掌灯,二人出来更衣,蔡状元拉西门庆说话:“学生此去回乡省亲,路费缺少。”西门庆道:“不劳老先生分咐。云峰尊命,一定谨领。”良久,让二人到花园:“还有一处小亭请看。”把二人一引,转过粉墙,来到藏春坞雪洞内。里面暖腾腾掌着灯烛,小琴桌上早已陈设果酌之类,床榻依然,琴书潇洒。从新复饮,书童在旁歌唱。蔡状元问道:“大官,你会唱‘红入仙桃’?”书童道:“此是《锦堂月》,小的记得。”于是把酒都斟,拿住南腔,拍手唱了一个。安进士听了,喜之下胜,向西门庆道:“此子可爱。”将杯中之酒一吸而饮之。那书童在席间穿着翠袖红裙,勒着销金箍儿,高擎玉斝,捧上酒,又唱了一个。当日直饮至夜分,方才歇息。西门庆藏春坞、翡翠轩两处俱设床帐,铺陈绩锦被褥,就派书童、玳安两个小厮答应。西门庆道了安置,方回后边去了。

到次日,蔡状元、安进士跟从人夫轿马来接。西门庆厅上摆酒伺候,馔饮下饭与脚下人吃。教两个小厮,方盒捧出礼物。蔡状元是金缎一端,领绢二端,合香五百,白金一百两。安进士是色缎一端,领绢一端,合香三百,白金三十两。蔡状元固辞再三,说道:“但假十数金足矣,何劳如此太多,又蒙厚腆!”安进士道:“蔡年兄领受,学生不当。”西门庆笑道:“些须微赆,表情而已。老先生荣归续亲,在下少助一茶之需。”于是两人俱出席谢道:“此情此德,何日忘之!”一面令家人各收下去,一面与西门庆相别,说道:“生辈此去,暂违台教。不日旋京,倘得寸进,自当图报。”安进士道:“今日相别,何年再得奉接尊颜?”西门庆道:“学生蜗居屈尊,多有亵慢,幸惟情恕!本当远送,奈官守在身,先此告过。”送二人到门首,看着上马而去。正是:

\[
博得锦衣归故里,功名方信是男儿。
\]
