%# -*- coding:utf-8 -*-
%%%%%%%%%%%%%%%%%%%%%%%%%%%%%%%%%%%%%%%%%%%%%%%%%%%%%%%%%%%%%%%%%%%%%%%%%%%%%%%%%%%%%


\chapter{西门庆贪欲丧命\KG 吴月娘失偶生儿}


词曰:

\[
人生南北如岐路,世事悠悠等风絮,造化弄人无定据。翻来覆去,倒横直竖,眼见都如许。到如今空嗟前事,功名富贵何须慕,坎止流行随所寓。玉堂金马,竹篱茅舍,总是伤心处。
\]

话说西门庆,奸耍了来爵老婆,复走到卷棚内,陪吴大舅、应伯爵、谢希大、常峙节饮酒。荆统制娘子、张团练娘子、乔亲家母、崔亲家母、吴大妗子、段大姐,坐了好一会,上罢元宵圆子,方才起身去了。大妗子那日同吴舜臣媳妇都家去了。陈敬济打发王皇亲戏子二两银子唱钱,酒食管待出门。只四个唱的并小优儿,还在卷棚内弹唱递酒。伯爵向西门庆说道:“明日花大哥生日,哥,你送了礼去不曾?”西门庆说道:“我早辰送过去了。”玳安道:“花大舅头里使来定儿送请贴儿来了。”伯爵道:“哥,你明日去不去?我好来会你。”西门庆道:“到明日看。再不,你先去罢。”少顷,四个唱的后边去了,李铭等上来弹唱,那西门庆不住只在椅子上打睡。吴大舅道:“姐夫连日辛苦了,罢罢,咱每告辞罢。”于是起身。那西门庆又不肯,只顾拦着,留坐到二更时分才散。西门庆先打发四个唱的轿子去了,拿大钟赏李铭等三人每人两钟酒,与了六钱唱钱,临出门,叫回李铭分付:“我十五日要请你周爷和你荆爷、何老爹众位,你早替我叫下四个唱的,休要误了。”李铭跪下禀问:“爹叫那四个?”西门庆道:“樊百家奴儿,秦玉芝儿,前日何老爹那里唱的一个冯金宝儿,并吕赛儿,好歹叫了来。”李铭应诺:“小的知道了。”磕了头去了。

西门庆归后边月娘房里来。月娘告诉:“今日林太太与荆大人娘子好不喜欢,坐到那咱晚才去了。酒席上再三谢我说:蒙老爹扶持,但得好处,不敢有忘。在出月往淮上催攒粮运去也。”又说:“何大娘子今日也吃了好些酒,喜欢六姐,又引到那边花园山子上瞧了瞧。今日各项也赏了许多东西。”说毕,西门庆就在上房歇了。到半夜,月娘做了一梦,天明告诉西门庆说道:“敢是我日里看着他王太太穿着大红绒袍儿,我黑夜就梦见你李大姐箱子内寻出一件大红绒袍儿,与我穿在身上,被潘六姐匹手夺了去,披在他身上,教我就恼了,说道:‘他的皮袄,你要的去穿了罢了,这件袍儿你又来夺。’他使性儿把袍儿上身扯了一道大口子,吃我大吆喝,和他骂嚷,嚷着就醒了。不想是南柯一梦。”西门庆道:“不打紧,我到明日替你寻一件穿就是了。自古梦是心头想。”

到次日起来,头沉,懒待往衙门中去,梳头净面,穿上衣裳,走来前边书房中坐的。只见玉箫问如意儿挤了半瓯子奶,径到书房与西门庆吃药。西门庆正倚靠床上,叫王经替他打腿。王经见玉箫来,就出去了。玉箫打发他吃了药,西门庆就使他拿了一对金镶头簪儿,四个乌银戒指儿,送到来爵媳妇子屋里去。那玉箫明见主子使他干此营生,又似来旺媳妇子那一本帐,连忙钻头觅缝,袖的去了。送到了物事,还走来回西门庆话,说道:“收了,改日与爹磕头。”就拿回空瓯子儿到上房去了。月娘叫小玉熬下粥,约莫等到饭时前后,还不见进来。

原来王经稍带了他姐姐王六儿一包儿物事,递与西门庆瞧,就请西门庆往他家去。西门庆打开纸包儿,却是老婆剪下的一柳黑臻臻、光油油的青丝,用五色绒缠就了一个同心结托儿,用两根锦带儿拴着,做的十分细巧。又一件是两个口的鸳鸯紫遍地金顺袋儿,里边盛着瓜穰儿。西门庆观玩良久,满心欢喜,遂把顺袋放在书厨内,锦托儿褪于袖中。正在凝思之际,忽见吴月娘蓦地走来,掀开帘子,见他躺在床上,王经扒着替他打腿,便说道:“你怎的只顾在前头,就不进去了,屋里摆下粥了。你告我说,你心里怎的,只是恁没精神?”西门庆道:“不知怎的,心中只是不耐烦,害腿疼。”月娘道:“想必是春气起了。你吃了药,也等慢慢来。”一面请到房中,打发他吃粥。因说道:“大节下,你也打起精神儿来,今日门外花大舅生日,请你往那里走走去。再不,叫将应二哥来,同你坐坐。”西门庆道:“他也不在,与花大舅做生日去了。你整治下酒菜儿,等我往灯市铺子内和他二舅坐坐罢。”月娘道:“你骑马去,我教丫鬟整理。”这西门庆一面分付玳安备马,王经跟随,穿上衣穿,径到狮子街灯市里来。但见灯市中车马轰雷,灯球灿彩,游人如蚁,十分热闹。

\[
太平时序好风催,罗绮争驰斗锦回。
鳌山高耸青云上,何处游人不看来。
\]

西门庆看了回灯,到房子门首下马,进入里面坐下。慌的吴二舅、贲四都来声喏。门首买卖,甚是兴盛。来昭妻一丈青又早书房内笼下火,拿茶吃了。不一时,吴月娘使琴童儿、来安儿拿了两方盒点心嗄饭菜蔬,铺内有南边带来豆酒,打开一坛,摆在楼上,请吴二舅与贲四轮番吃酒。楼窗外就看见灯市,来往人烟不断。

吃至饭后时分,西门庆使王经对王六儿说去。王六儿听见西门庆来,连忙整治下春台,果盒酒肴等候。西门庆分付来昭:“将这一桌酒菜,晚夕留着吴二舅、贲四在此上宿吃,不消拿回家去了。”又教琴童提送一坛酒,过王六儿这边来。西门庆于是骑马径到他家。妇人打扮迎接到明间内,插烛也似磕了四个头。西门庆道:“迭承你厚礼,怎的两次请你不去?”王六儿说道:“爹倒说的好,我家中再有谁来?不知怎的,这两日只是心里不好,茶饭儿也懒待吃,做事没入脚处。”西门庆道:“敢是想你家老公?”妇人道:“我那里想他!倒是见爹这一向不来,不知怎的怠慢着爹了,爹把我网巾圈儿打靠后了,只怕另有个心上人儿了。”西门庆笑道:“那里有这个理!倒因家中节间摆酒,忙了两日。”妇人道:“说昨日爹家中请堂客来。”西门庆道:“便是你大娘吃过人家两席节酒,须得请人回席。”妇人道:“请了那几位堂客?”西门庆便说某人某人,从头诉说一遍。妇人道:“看灯酒儿,只请要紧的,就不请俺每请儿。”西门庆道:“不打紧,到明日十六,还有一席酒,请你每众伙计娘子走走去。是必到跟前又推故不去了。”妇人道:“娘若赏个贴儿来,怎敢不去?”因前日他小大姐骂了申二姐,教他好不抱怨,说俺每。他那日原要不去来,倒是俺每撺掇了他去,落后骂了来,好不在这里哭。俺每倒没意思剌涑的。落后又教爹娘费心,送了盒子并一两银子来,安抚了他,才罢了。原来小大姐这等躁暴性子,就是打狗也看主人面。”西门庆道:“你不知这小油嘴,他好不兜达的性儿,着紧把我也擦刮的眼直直的。也没见,他叫你唱,你就唱个儿与他听罢了,谁教你不唱,又说他来?”妇人道:“耶嚛,耶嚛!他对我说,他几时说他来,说小大姐走来指着脸子就骂起来,在我这里好不三行鼻涕两行眼泪的哭。我留他住了一夜,才打发他去了。”说了一回,丫头拿茶吃了。老冯婆子又走来与西门庆磕头。西门庆与了他约三四钱一块银子,说道:“从你娘没了,就不往我那里走走去。”妇人道:“没他的主儿,那里着落?倒常时来我这里,和我做伴儿。”

不一时,请西门庆房中坐的,问:“爹和了午饭不曾?”西门庆道:“我早辰家中吃了些粥,刚才陪你二舅又吃了两个点心,且不吃甚么哩。”一面放桌儿,安排上酒来。妇人令王经打开豆酒,筛将上来,陪西门庆做一处饮酒。妇人问道:“我稍来的那物件儿,爹看见来?都是奴旋剪下顶中一溜头发,亲手做的。管情爹见了爱。”西门庆道:“多谢你厚情。”饮至半酣,见房内无人,西门庆袖中取出来,套在龟身下,两根锦带儿扎在腰间,用酒服下胡僧药去,那妇人用手搏弄,弄得那话登时奢棱跳脑,横筋皆现,色若紫肝,比银托子和白绫带子又不同。西门庆搂妇人坐在怀内,那话插进牝中,在上面两个一递一口饮酒,咂舌头顽笑。吃至掌灯,冯妈妈又做了些韭菜猪肉饼儿拿上来。妇人陪西门庆每人吃了两个,丫鬟收下去。两个就在里间暖炕上,撩开锦幔,解衣就寝。妇人知道西门庆好点着灯行房,把灯台移在里间炕边桌上,一面将纸门关上,澡牝干净,脱了裤儿,钻在被窝里,与西门庆做一处相搂相抱,睡了一回。原来西门庆心中只想着何千户娘子蓝氏,欲情如火,那话十分坚硬。先令妇人马伏在下,那话放入庭花内,极力扇蹦了约二三百度,扇蹦的屁股连声响亮,妇人用手在下揉着心子,口中叫达达如流水。西门庆还不美意,又起来披上白绫小袄,坐在一只枕头上,令妇人仰卧,寻出两条脚带,把妇人两只脚拴在两边护炕柱儿上,卖了个金龙探爪,将那话放入牝中,少时,没棱露脑,浅抽深送。恐妇人害冷,亦取红绫短襦,盖在他身上。这西门庆乘其酒兴,把灯光挪近跟前,垂首玩其出入之势。抽撤至首,复送至根,又数百回。妇人口中百般柔声颤语,都叫将出来。西门庆又取粉红膏子药,涂在龟头上攮进去,妇人阴中麻痒不能当,急令深入,两厢迎就。这西门庆故作逗留,戏将龟头濡晃其牝口,又操弄其花心,不肯深入,急的妇人淫津流出,如蜗之吐涎。灯光里,见他两只腿儿着红鞋,跷在两边,吊的高高的,一往一来,一冲一撞,其兴不可遏。因口呼道:“淫妇,你想我不想?”妇人道:“我怎么不想达达,只要你松柏儿冬夏长青便好。休要日远日疏,顽耍厌了,把奴来不理。奴就想死罢了,敢和谁说?有谁知道?就是俺那王八来家,我也不和他说。想他恁在外做买卖,有钱,他不会养老婆的?他肯挂念我?”西门庆道:“我的儿,你若一心在我身上,等他来家,我爽利替他另娶一个,你只长远等着我便了。”妇人道:“好达达,等他来家,好歹替他娶了一个罢,或把我放在外头,或是招我到家去,随你心里。淫妇爽利把不直钱的身子,拼与达达罢,无有个不依你的。”西门庆道:“我知道。”两个说话之间,又干勾两顿饭时,方才精泄。解御下妇人脚带来,搂在被窝内,并头交股,醉眼朦胧,一觉直睡到三更时分方起。西门庆起来,穿衣净手。妇人开了房门,叫丫鬟进来,再添美馔,复饮香醪,满斟暖酒,又陪西门庆吃了十数杯。不觉醉上来,才点茶漱口,向袖中掏出一纸贴儿递与妇人:“问甘伙计铺子里取一套衣服你穿,随你要甚花样。”那妇人万福谢了,方送出门。

王经打着灯笼,玳安、琴童笼着马,那时也有三更天气,阴云密布,月色朦胧,街市上人烟寂寞,闾巷内犬吠盈盈。打马刚走到西首那石桥儿跟前,忽然一阵旋风,只见个黑影子,从桥底下钻出来,向西门庆一扑。那马见了只一惊跳,西门庆在马上打了个冷战,醉中把马加了一鞭,那马摇了摇鬃,玳安、琴童两个用力拉着嚼环,收煞不住,云飞般望家奔将来,直跑到家门首方止。王经打着灯笼,后边跟不上。西门庆下马腿软了,被左右扶进,径往前边潘金莲房中来。此这一来,正是:

\[
失脱人家逢五道,滨冷饿鬼撞钟馗。
\]

原来金莲从后边来,还没睡,浑衣倒在炕上,等待西门庆。听见来了,连忙一骨碌扒起来,向前替他接衣服。见他吃的酩酊大醉,也不敢问他。西门太一只手搭伏着他肩膀上,搂在怀里,口中喃喃呐呐说道:“小淫妇儿,你达达今日醉了,收拾铺,我睡也。”那妇人持他上炕,打发他歇下。那西门庆丢倒头在枕上鼾睡如雷,再摇也摇他不醒。然后妇人脱了衣裳,钻在被窝内,慢慢用手腰里摸他那话,犹如绵软,再没硬朗气儿,更不知在谁家来。翻来覆去,怎禁那欲火烧身,淫心荡漾,不住用手只顾捏弄,蹲下身子,被窝内替他百计品咂,只是不起,急的妇人要不的。因问西门庆:“和尚药在那里放着哩?”推了半日推醒了。西门庆酩子里骂道:“怪小淫妇,只顾问怎的?你又教达达摆布你,你达今日懒待动弹。药在我袖中穿心盒儿内。你拿来吃了,有本事品弄的他起来,是你造化。”那妇人便去袖内摸出穿心盒来打开,里面只剩下三四丸药儿。这妇人取过烧酒壶来,斟了一钟酒,自己吃了一丸,还剩下三丸。恐怕力不效,千不合,万不合,拿烧酒都送到西门庆口内。醉了的人,晓的甚么?合着眼只顾吃下去。那消一盏热茶时,药力发作起来,妇人将白绫带子拴在根上,那话跃然而起,妇人见他只顾去睡,于是骑在他身上,又取膏子药安放在马眼内,顶入牝中,只顾揉搓,那话直抵苞花窝里,觉翕翕然,浑身酥麻,畅美不可言。又两手据按,举股一起一坐,那话坐棱露脑,一二百回。初时涩滞,次后淫水浸出,稍沾滑落,西门庆由着他掇弄,只是不理。妇人情不能当,以舌亲于西门庆口中,两手搂着他脖项,极力揉搓,左右偎擦,麈柄尽没至根,止剩二卵在外,用手摸之,美不可言,淫水随拭随出。比三鼓天,五换巾帕。妇人一连丢了两次,西门庆只是不泄。龟头越发胀的犹如炭火一般,害箍胀的慌,令妇人把根下带子去了,还发胀不已,令妇人用口吮之。这妇人扒伏在他身上,用朱唇吞裹龟头,只顾往来不已,又勒勾约一顿饭时,那管中之精猛然一股冒将出来,犹水银之淀筒中相似,忙用口接咽不及,只顾流将出来。初时还是精液,往后尽是血水出来,再无个收救。西门庆已昏迷去,四肢不收。妇人也慌了,急取红枣与他吃下去。精尽继之以血,血尽出其冷气而已。良久方止。妇人慌做一团,便搂着西门庆问道:“我的哥哥,你心里觉怎么的!”西门庆亦苏醒了一回,方言:“我头目森森然,莫知所以。”金莲问:“你今日怎的流出恁许多来?”更不说他用的药多了。看官听说,一己精神有限,天下色欲无穷。又曰“嗜欲深者生机浅”,西门庆只知贪淫乐色,更不知油枯灯灭,髓竭人亡。正是起头所说:

\[
二八佳人体似酥,腰间仗剑斩愚夫。
虽然不见人头落,暗里教君骨髓枯。
\]

一宿晚景题过。到次日清早辰,西门庆起来梳头,忽然一阵昏晕,望前一头抢将去。早被春梅双手扶住,不曾跌着磕伤了头脸。在椅上坐了半日,方才回过来。慌的金莲连忙问道:“只怕你空心虚弱,且坐着,吃些甚么儿着,出去也不迟。”一面使秋菊:“后边取粥来与你爹吃。”那秋菊走到后边厨下,问雪娥:“熬的粥怎么了?爹如此这般,今早起来害了头晕,跌了一交,如今要吃粥哩。”不想被月娘听见,叫了秋菊,问其端的。秋菊悉把西门庆梳头,头晕跌倒之事,告诉一遍。月娘不听便了,听了魂飞天外,魄散九霄,一面分付雪娥快熬粥,一面走来金莲房中看视。见西门庆坐在椅子上,问道:“你今日怎的头晕?”西门庆道:“我不知怎的,刚才就头晕起来。”金莲道:“早时我和春梅要跟前扶住了,不然好轻身子儿,这一交和你善哩!”月娘道:“敢是你昨日来家晚了,酒多了头沉。”金莲道:“昨日往谁家吃酒?那咱晚才来。”月娘道:“他昨日和他二舅在铺子里吃酒来。”不一时,雪娥熬了粥,教春梅拿着,打发西门庆吃。那西门庆拿起粥来,只吃了半瓯儿,懒待吃,就放下了。月娘道:“你心里觉怎的?”西门庆道:“我不怎么,只是身子虚飘飘的,懒待动旦。”月娘道:“你今日不往衙门中去罢。”西门庆道:“我不去了。消一回,我往前边看着姐夫写贴儿,十五日请周菊轩、荆南岗、何大人众官客吃酒。”月娘道:“你今日还没吃药,取奶来把那药再吃上一服。是你连日着辛苦忙碌了。”一面教春梅问如意儿挤了奶来,用盏儿盛着,教西门庆吃了药,起身往前边去。春梅扶着,刚走到花园角门首,觉眼便黑了,身子晃晃荡荡,做不的主儿,只要倒。春梅又扶回来了。月娘道:“依我且歇两日儿,请人也罢了,那里在乎这一时。且在屋里将息两日儿,不出去罢。”因说:“你心里要吃甚么,我往后边做来与你吃。”西门庆道:“我心里不想吃。”

月娘到后边,从新又审问金莲:“他昨日来家醉不醉?再没曾吃酒?与你行甚么事?”金莲听了,恨不的生出几个口来,说一千个没有:“姐姐,你没的说,他那咱晚来了,醉的行礼儿也没顾的,还问我要烧酒吃,教我拿茶当酒与他吃,只说没了酒,好好打发他睡了。自从姐姐那等说了,谁和他有甚事来,倒没的羞人子剌剌的。倒只怕别处外边有了事来,俺每不知道。若说家里,可是没丝毫事儿。”月娘和玉楼都坐在一处,一面叫了玳安、琴童两个到跟前审问他:“你爹昨日在那里吃酒来?你实说便罢,不然有一差二错,就在你这两个囚根子身上。”那玳安咬定牙,只说狮子街和二舅、贲四吃酒,再没往那里去。落后叫将吴二舅来,问他,二舅道:“姐夫只陪俺每吃了没多大回酒,就起身往别处去了。”这吴月娘听了,心中大怒,待二舅去了,把玳安、琴童尽力数骂了一遍,要打他二人。二人慌了,方才说出:“昨日在韩道国老婆家吃酒来。”那潘金莲得不的一声就来了,说道:“姐姐刚才就埋怨起俺每来,正是冤杀旁人笑杀贼。俺每人人有面,树树有皮,姐姐那等说来,莫不俺每成日把这件事放在头里?”又道:“姐姐,你再问这两个囚根子,前日你往何千户家吃酒,他爹也是那咱时分才来,不知在谁家来。谁家一个拜年,拜到那咱晚!”玳安又恐怕琴童说出来,隐瞒不住,遂把私通林太太之事,备说一遍。月娘方才信了,说道:“嗔道教我拿贴儿请他,我还说人生面不熟,他不肯来,怎知和他有连手。我说恁大年纪,描眉画鬓,搽的那脸倒像腻抹儿抹的一般,干净是个老浪货!”玉楼道:“姐姐,没见一个儿子也长恁大人儿,娘母还干这个营生。忍不住,嫁了个汉子,也休要出这个丑。”金莲道:“那老淫妇有甚么廉耻!”月娘道:“我只说他决不来,谁想他浪\textuni{22D5E}着来了。”金莲道:“这个,姐姐才显出个皂白来了!像韩道国家这个淫妇,姐姐还嗔我骂他!干净一家子都养汉,是个明王八,把个王八花子也裁派将来,早晚好做勾使鬼。”月娘道:“王三官儿娘,你还骂他老淫妇,他说你从小儿在他家使唤来。”那金莲不听便罢,听了把脸掣耳朵带脖子都红了,便骂道:“汗邪了那贼老淫妇!我平日在他家做甚么?还是我姨娘在他家紧隔壁住,他家有个花园,俺每小时在俺姨娘家住,常过去和他家伴姑儿耍子,就说我在他家来,我认的他是谁?也是个张眼露睛的老淫妇!”月娘道:“你看那嘴头子!人和你说话,你骂他。”那金莲一声儿就不言语了。

月娘主张叫雪娥做了些水角儿,拿了前边与西门庆吃。正走到仪门首,只见平安儿径直往花园中走。被月娘叫住问道:“你做甚么?”平安儿道:“李铭叫了四个唱的,十五日摆酒,因来回话。问摆的成摆不成。我说未发贴儿哩。他不信,教我进来禀爹。”月娘骂道:“怪贼奴才,还摆甚么酒,问甚么,还不回那王八去哩,还来禀爹娘哩。”把平安儿骂的往外金命水命去了。月娘走到金莲房中,看着西门庆只吃了三四个水角儿,就不吃了。因说道:“李铭来回唱的,教我回倒他,改日子了,他去了。”西门庆点头儿。

西门庆只望一两日好些出来,谁知过了一夜,到次日,内边虚阳肿胀,不便处发出红瘰来,连肾囊都肿得明滴溜如茄子大。但溺尿,尿管中犹如刀子犁的一般。溺一遭,疼一遭。外边排军、伴当备下马伺候,还等西门庆往衙门里大发放,不想又添出这样症候来。月娘道:“你依我拿贴儿回了何大人,在家调理两日儿,不去罢。你身子恁虚弱,趁早使小厮请了任医官,教瞧瞧。你吃他两贴药过来。休要只顾耽着,不是事。你偌大的身量,两日通没大好吃甚么儿,如何禁的?”那西门庆只是不肯吐口儿请太医,只说:“我不妨事,过两日好了,我还出去。”虽故差人拿贴儿送假牌往衙门里去,在床上睡着,只是急躁,没好气。西门庆只望一两日好些出来,谁知过了一夜,到次日,内边虚阳肿胀,不便处发出红瘰来,连肾囊都肿得明滴溜如茄子大。但溺尿,尿管中犹如刀子犁的一般。溺一遭,疼一遭。外边排军、伴当备下马伺候,还等西门庆往衙门里大发放,不想又添出这样症候来。月娘道:“你依我拿贴儿回了何大人,在家调理两日儿,不去罢。你身子恁虚弱,趁早使小厮请了任医官,教瞧瞧。你吃他两贴药过来。休要只顾耽着,不是事。你偌大的身量,两日通没大好吃甚么儿,如何禁的?”那西门庆只是不肯吐口儿请太医,只说:“我不妨事,过两日好了,我还出去。”虽故差人拿贴儿送假牌往衙门里去,在床上睡着,只是急躁,没好气。

应伯爵打听得知,走来看他。西门庆请至金莲房中坐的。伯爵声喏道:“前日打搅哥,不知哥心中不好,嗔道花大舅那里不去。”西门庆道:“我心中若好时,也去了。不知怎的懒待动旦。”伯爵道:“哥,你如今心内怎样的?”西门庆道:“不怎的,只是有些头晕,起来身子软,走不的。”伯爵道:“我见你面容发红色,只怕是火。教人看来不曾?”西门庆道:“房下说请任后溪来看我,我说又没甚大病,怎好请他的。”伯爵道:“哥,你这个就差了,还请他来看看,怎的说。吃两贴药,散开这火就好了。春气起,人都是这等痰火举发举发。昨日李铭撞见我,说你使他叫唱的,今日请人摆酒,说你心中不好,改了日子。把我唬了一跳,我今日才来看哥。”西门庆道:“我今日连衙门中拜牌也没去,送假牌去了。”伯爵道:“可知去不的,大调理两日儿出门。”吃毕茶道:“我去罢,再来看哥。李桂姐会了吴银儿,也要来看你哩。”西门庆道:“你吃了饭去。”伯爵道:“我一些不吃。”扬长出去了。

西门庆于是使琴童往门外请了任医官来,进房中诊了脉,说道:“老先生此贵恙,乃虚火上炎,肾水下竭,不能既济,此乃是脱阳之症。须是补其阴虚,方才好得。”说毕,作辞起身去了。一面封了五钱银子,讨将药来,吃了。止住了头晕,身子依旧还软,起不来。下边肾囊越发肿痛,溺尿甚难。西门庆于是使琴童往门外请了任医官来,进房中诊了脉,说道:“老先生此贵恙,乃虚火上炎,肾水下竭,不能既济,此乃是脱阳之症。须是补其阴虚,方才好得。”说毕,作辞起身去了。一面封了五钱银子,讨将药来,吃了。止住了头晕,身子依旧还软,起不来。下边肾囊越发肿痛,溺尿甚难。

到后晌时分,李桂姐、吴银儿坐轿子来看。每人两个盒子,进房与西门庆磕头,说道:“爹怎的心里不自在?”西门庆道:“你姐儿两个自恁来看看便了,如何又费心买礼儿。”因说道:“我今年不知怎的,痰火发的重些。”桂姐道:“还是爹这节间酒吃的多了,清洁他两日儿,就好了。”坐了一回,走到李瓶儿那边屋里,与月娘众人见节。请到后边,摆茶毕,又走来到前边,陪西门庆坐的说话儿。只见伯爵又陪了谢希大、常峙节来望。西门庆教玉箫搊扶他起来坐的,留他三人在房内,放桌儿吃酒。谢希大道:“哥,用了些粥不曾?”玉箫把头扭着不答应。西门庆道:“我还没吃粥,咽不下去。”希大道:“拿粥,等俺每陪哥吃些粥儿还好。”不一时,拿将粥来。西门庆拿起粥来,只扒了半盏儿,就吃不下了。月娘和李桂姐、吴银儿都在李瓶儿那边坐的。伯爵问道:“李桂姐与银姐来了,怎的不见?”西门庆道:“在那边坐的。”伯爵因令来安儿:“你请过来,唱一套儿与你爹听。”吴月娘恐西门庆不耐烦,拦着,只说吃酒哩,不教过来。众人吃了一回酒,说道:“哥,你陪着俺每坐,只怕劳碌着你。俺每去了,你自在侧侧儿罢。”西门庆道:“起动列位挂心。”三人于是作辞去了。

应伯爵走出小院门,叫玳安过来分付:“你对你大娘说,应二爹说来,你爹面上变色,有些滞气,不好,早寻人看他。大街上胡太医最治的好痰火,何不使人请他看看,休要耽迟了。”玳安不敢怠慢,走来告诉月娘。月娘慌进房来,对西门庆说:“方才应二哥对小厮说,大街上胡太医看的痰火好,你何不请他来看看你?”西门庆道:“胡太医前番看李大姐不济,又请他?”月娘道:“药医不死病,佛度有缘人。看他不济,只怕你有缘,吃了他的药儿好了是的。”西门庆道:“也罢,你请他去。”不一时,使棋童儿请了胡太医来。适有吴大舅来看,陪他到房中看了脉。对吴大舅、陈敬济说:“老爹是个下部蕴毒,若久而不治,卒成溺血之疾。乃是忍便行房。”又卦了五星药金,讨将药来吃下去,如石沉大海一般,反溺不出来。月娘慌了,打发桂姐、吴银儿去了,又请何老人儿子何春泉来看。又说:“是癃闭便毒,一团膀胱邪火,赶到这下边来。四肢经络中,又有湿痰流聚,以致心肾不交。”封了五钱药金,讨将药来,越发弄的虚阳举发,麈柄如铁,昼夜不倒。潘金莲晚夕不管好歹,还骑在他身上,倒浇蜡烛掇弄,死而复苏者数次。

到次日,何千户要来望,先使人来说。月娘便对西门庆道:“何大人要来看你,我扶你往后边去罢,这边隔二骗三,不是个待人的。”那西门庆点头儿。于是月娘替他穿上暖衣,和金莲肩搭搊扶着,方离了金莲房,往后边上房,铺下被褥高枕,安顿他在明间炕上坐的。房中收拾干净,焚下香。不一时,何千户来到,陈敬济请他到于后边卧房,看见西门庆坐在病榻上,说道:“长官,我不敢作揖。”因问:“贵恙觉好些?”西门庆告诉:“上边火倒退下了,只是下边肿毒,当不的。”何千户道:“此系便毒。我学生有一相识,在东昌府探亲,昨日新到舍下,乃是山西汾州人氏,姓刘号桔斋,年半百,极看的好疮毒。我就使人请他来看看长官贵恙。”西门庆道:“多承长官费心,我这里就差人请去。”何千户吃毕茶,说道:“长官,你耐烦保重。衙门中事,我每日委答应的递事件与你,不消挂意。”西门庆举手道:“只是有劳长官了。”作辞出门。西门庆这里随即差玳安拿贴儿,同何家人请了这刘桔斋来。看了脉,并不便处,连忙上了药,又封一贴煎药来。西门庆答贺了一匹杭州绢,一两银子。吃了他头一盏药,还不见动静。

那日不想郑月儿送了一盒鸽子雏儿,一盒果饼顶皮酥,坐轿子来看。进门与西门庆磕头,说道:“不知道爹不好,桂姐和银姐好人儿,不对我说声儿,两个就先来了。看的爹迟了,休怪。”西门庆道:“不迟,又起动你费心,又买礼来。”爱月儿笑道:“甚么大礼,惶恐。”因说:“爹清减的恁样的,每日饮馔也用些儿?”月娘道:“用的倒好了,吃不多儿。今日早辰,只吃了些粥汤儿,刚才太医看了去了。”爱月儿道:“娘,你分付姐把鸽子雏儿顿烂一个儿来,等我劝爹进些粥儿。你老人家不吃,恁偌大身量,一家子金山也似靠着你,却怎么样儿的。”月娘道:“他只害心口内拦着,吃不下去。”爱月儿道:“爹,你依我说,把这饮撰儿就懒待吃,须也强吃些儿,怕怎的?人无根本,水食为命。终须用的有柱■些儿。不然,越发淘渌的身子空虚了。”不一时,顿烂了鸽子雏儿,小玉拿粥上来,十香甜酱瓜茄,粳粟米粥儿。这郑月儿跳上炕去,用盏儿托着,跪在西门庆身边,一口口喂他。强打着精神,只吃了上半盏儿。拣两箸儿鸽子雏儿在口内,就摇头儿不吃了。爱月儿道:“一来也是药,二来还亏我劝爹,却怎的也进了些饮馔儿!”玉箫道:“爹每常也吃,不似今日月姐来,劝着吃的多些。”月娘一面摆茶与爱月儿吃,临晚管待酒馔,与了他五钱银子,打发他家去。爱月儿临出门,又与西门庆磕头,说道:“爹,你耐烦将息两日儿,我再来看你。”

比及到晚夕,西门庆又吃了刘桔斋第二贴药,遍身疼痛,叫了一夜。到五更时分,那不便处肾囊胀破了,流了一滩鲜血,龟头上又生出疳疮来,流黄水不止。西门庆不觉昏迷过去。月娘众人慌了,都守着看视,见吃药不效,一面请了刘婆子,在前边卷棚内与西门庆点人灯挑神,一面又使小厮往周守备家内访问吴神仙在那里,请他来看,因他原相西门庆今年有呕血流脓之灾,骨瘦形衰之病。贲四说:“也不消问周老爹宅内去,如今吴神仙见在门外土地庙前,出着个卦肆儿,又行医,又卖卦。人请他,不争利物,就去看治。”月娘连忙就使琴童把这吴神仙请将来。进房看了西门庆不似往时,形容消减,病体恹恹,勒着手帕,在于卧榻。先诊了脉息,说道:“官人乃是酒色过度,肾水竭虚,太极邪火聚于欲海,病在膏肓,难以治疗。吾有诗八句,说与你听。只因他:

\[
醉饱行房恋女娥,精神血脉暗消磨。
遗精溺血与白浊,灯尽油干肾水枯。
当时只恨欢娱少,今日翻为疾病多。
玉山自倒非人力,总是卢医怎奈何!”
\]

月娘见他说治不的了,道:“既下药不好,先生看他命运如何?”吴神仙掐指寻纹,打算西门庆八字,说道:“属虎的,丙寅年,戊申月,壬午日,丙辰时。今年戊戌,流年三十三年,算命,见行癸亥运。虽然是火土伤官,今年戊土来克壬水。正月又是戊寅月,三戊冲辰,怎么当的?虽发财发福,难保寿源。有四句断语不好。说道:

\[
命犯灾星必主低,身轻煞重有灾危。
时日若逢真太岁,就是神仙也皱眉。
\]
月娘道:“命不好,请问先生还有解么?”神仙道:“白虎当头,丧门坐命,神仙也无解,太岁也难推。造物已定,神鬼莫移。”月娘只得拿了一匹布,谢了神仙,打发出门。月娘见求神问卜皆有凶无吉,心中慌了。到晚夕,天井内焚香,对天发愿,许下“儿夫好了,要往泰安州顶上与娘娘进香挂袍三年”。孟玉楼又许下逢七拜斗,独金莲与李娇儿不许愿心。

西门庆自觉身体沉重,要便发昏过去,眼前看见花子虚、武大在他跟前站立,问他讨债,又不肯告人说,只教人厮守着他。见月娘不在跟前,一手拉着潘金莲,心中舍他不的,满眼落泪,说道:“我的冤家,我死后,你姐妹们好好守着我的灵,休要失散了。”那金莲亦悲不自胜,说道:“我的哥哥,只怕人不肯容我。”西门庆道:“等他来,等我和他说。”不一时,吴月娘进来,见他二人哭的眼红红的,便道:“我的哥哥,你有甚话,对奴说几句儿,也是我和你做夫妻一场。”西门庆听了,不觉哽咽哭不出声来,说道:“我觉自家好生不济,有两句遗言和你说:我死后,你若生下一男半女,你姊妹好好待着,一处居住,休要失散了,惹人家笑话。”指着金莲说:“六儿从前的事,你耽待他罢。”说毕,那月娘不觉桃花脸上滚下珍珠来,放声大哭,悲恸不止。西门庆嘱付了吴月娘,又把陈敬济叫到跟前,说道:“姐夫,我养儿靠儿,无儿靠婿。姐夫就是我的亲儿一般。我若有些山高水低,你发送了我入土。好歹一家一计,帮扶着你娘儿每过日子,休要教人笑话。”又分付:“我死后,段子铺里五万银子本钱,有你乔亲家爹那边,多少本利都找与他。教傅伙计把贷卖一宗交一宗,休要开了。贲四绒线铺,本银六千五百两,吴二舅绸绒铺是五千两,都卖尽了货物,收了来家。又李三讨了批来,也不消做了,教你应二叔拿了别人家做去罢。李三、黄四身上还欠五百两本钱,一百五十两利钱未算,讨来发送我。你只和傅伙计守着家门这两个铺子罢。印子铺占用银二万两,生药铺五千两,韩伙计、来保松江船上四千两。开了河,你早起身,往下边接船去。接了来家,卖了银子并进来,你娘儿每盘缠。前边刘学官还少我二百两,华主簿少我五十两,门外徐四铺内,还欠我本利三百四十两,都有合同见在,上紧使人摧去。到日后,对门并狮子街两处房子都卖了罢,只怕你娘儿们顾揽不过来。”说毕,哽哽咽咽的哭了。陈敬济道:“爹嘱咐,儿子都知道了。”不一时,傅伙计、甘伙计、吴二舅、贲四、崔本都进来看视问安。西门庆一一都分付了一遍。众人都道:“你老人家宽心,不妨事。”一日来问安看者,也有许多。见西门庆不好的沉重,皆嗟叹而去。

过了两日,月娘痴心,只指望西门庆还好,谁知天数造定,三十三岁而去。到于正月二十一日,五更时分,相火烧身,变出风来,声若牛吼一般,喘息了半夜。挨到巳牌时分,呜呼哀哉,断气身亡。正是:三寸气在千般用,一旦无常万事休。古人有几句格言,说得好:

\[
为人多积善,不可多积财。积善成好人,积财惹祸胎。
石崇当日富,难免杀身灾。邓通饥饿死,钱山何用哉!
今人非古比,心地不明白。只说积财好,反笑积善呆。
多少有钱者,临了没棺材。
\]

原来西门庆一倒头,棺材尚未曾预备。慌的吴月娘叫了吴二舅与贲四到跟前,开了箱子拿四四锭元宝,教他两个看材板去。刚才打发去了,不防忽一阵就害肚里疼,急扑进去床上倒下,就昏晕不省人事。孟玉楼与潘金莲、孙雪娥都在那边屋里,七手八脚,替西门庆戴唐巾,装柳穿衣服。忽听见小玉来说:“俺娘跌倒在床上。”慌的玉楼、李娇儿就来问视,月娘手按着害肚内疼,就知道决撒了。玉楼教李娇儿守着月娘,他就来使小厮快请蔡老娘去。李娇儿又使玉箫前边教如意儿来。比及玉楼回到上房里面,不见了李娇儿。原来李娇儿赶月娘昏沉,房内无人,箱子开着,暗暗拿了五锭元宝,往他屋里去了。手中拿将一搭纸,见了玉楼,只说:“寻不见草纸,我往房里寻草纸去来。”那玉楼也不留心,且守着月娘,拿杩子伺候,见月娘看看疼的紧了。

不一时,蔡老娘到了,登时生下一个孩儿来。这屋里装柳西门庆停当,口内才没气儿,合家大小放声号哭起来。蔡老娘收裹孩儿,剪去脐带,煎定心汤与月娘吃了。扶月娘暖炕上坐的。月娘与了蔡老娘三两银子,蔡老娘嫌少,说道:“养那位哥儿赏了我多少,还与我多少便了。休说这位哥儿是大娘生养的。”月娘道:“比不得当时,有当家的老爹在此,如今没了老爹,将就收了罢。待洗三来,再与你一两就是了。”那蔡老娘道:“还赏我一套衣服儿罢。”拜谢去了。

月娘苏醒过来,看见箱子大开着,便骂玉箫:“贼臭肉,我便昏了,你也昏了?箱子大开着,恁乱烘烘人走,就不说锁锁儿。”玉箫道:“我只说娘锁了箱子,就不曾看见。”于是取锁来锁。玉楼见月娘多心,就不肯在他屋里,走出对着金莲说:“原来大姐姐恁样的,死了汉子,头一日就防范起人来了。”殊不知李娇儿已偷了五锭元宝在屋里去了。

当下吴二舅、贲四往尚推官家买了一付棺材板来,教匠人解锯成椁。众小厮把西门庆抬出,停当在大厅上,请了阴阳徐先生来批书。不一时,吴大舅也来了。吴二舅、众伙计都在前厅热乱,收灯卷画,盖上纸被,设放香灯几席。来安儿专一打磨。徐先生看了手,说道:“正辰时断气,合家都不犯凶煞。”请问月娘:“三日大殓,择二月十六破土,三十出殡,有四七多日子。”一面管待徐先生去了,差人各处报丧,交牌印往何千户家去,家中披孝搭棚,俱不必细说。

到三日,请僧人念倒头经,挑出纸钱去。合家大小都披麻带孝。女婿陈敬济斩衰泣杖,灵前还礼。月娘在暗房中出不来。李娇儿与玉楼陪待堂客;潘金莲管理库房,收祭桌;孙雪娥率领家人媳妇,在厨下打发各项人茶饭。傅伙计、吴二舅管帐、贲四管孝帐;来兴管厨;吴大舅与甘伙计陪待人客。蔡老娘来洗了三,月娘与了一套绸绢衣裳打发去了。就把孩儿起名叫孝哥儿,未免送些喜面。亲邻与众街坊邻舍都说:“西门庆大官人正头娘子生了一个墓生儿子,就与老子同日同时,一头断气,一头生儿,世间有这等蹊跷古怪事。”

不说众人理乱这桩事。且说应伯爵闻知西门庆没了,走来吊孝哭泣,哭了一回。吴大舅、二舅正在卷棚内看着与西门庆传影,伯爵走来,与众人见礼,说道:“可伤,做梦不知哥没了。”要请月娘拜见,吴大舅便道:“舍妹暗房出不来,如此这般,就是同日添了个娃儿。”伯爵愕然道:“有这等事!也罢也罢,哥有了个后代,这家当有了主儿了。”落后陈敬济穿着一身重孝,走来与伯爵磕头。伯爵道:“姐夫姐夫,烦恼。你爹没了,你娘儿每是死水儿了,家中凡事要你仔细。有事不可自家专,请问你二位老舅主张。不该我说,你年幼,事体还不大十分历练。”吴大舅道:“二哥,你没的说。我自也有公事,不得闲,见有他娘在。”伯爵道:“好大舅,虽故有嫂子,外边事怎么理的?还是老舅主张。自古没舅不生,没舅不长。一个亲娘舅,比不的别人。你老人家就是个都根主儿,再有谁大?”因问道:“有了发引日期没有?”吴大舅道:“择二月十六日破土,三十日出殡,也在四七之外。”不一时,徐先生来到,祭告入殓,将西门庆装入棺材内,用长命丁钉了,安放停当,题了名旌:“诰封武略将军西门公之柩”。

那日何千户来吊孝。灵前拜毕,吴大舅与伯爵陪侍吃茶,问了发引的日期。何千户分付手下该班排军,原答应的,一个也不许动,都在这里伺候。直过发引之后,方许回衙门当差。又委两名节级管领,如有违误,呈来重治。又对吴大舅说:“如有外边人拖欠银两不还者,老舅只顾说来,学生即行追治。”吊老毕,到衙门里一面行文开缺,申报东京本卫去了。

话分两头。却说来爵、春鸿同李三,一日到兖州察院,投下了书礼,宋御史见西门庆书上要讨古器批文一节,说道:“你早来一步便好。昨日已都派下各府买办去了。”寻思间,又见西门庆书中封着金叶十两,又不好违阻了的。便留下春鸿、来爵、李三在公廨驻札。随即差快手拿牌,赶回东平府批文来,封回与春鸿书中,又与了一两路费,方取路回清河县。往返十日光景。走进城,就闻得路上人说:“西门大官人死了,今日三日,家中念经做斋哩。”这李三就心生奸计,路上说念来爵、春鸿:“将此批文按下,只说宋老爷没与来。咱每都投到大街张二老爹那里去罢。你二人不去,我每人与你十两银子,到家隐住,不拿出来就是了。”那来爵见财物倒也肯了,只春鸿不肯,口里含糊应诺。

到家,见门首挑着纸钱,僧人做道场,亲朋吊丧者不计其数,这李三就分路回家去了。来爵、春鸿见吴大舅、陈敬济磕了头,问:“讨批文如何?怎的李三不来?”那来爵欲说不肯,这春鸿把宋御史书连批都拿出来,递与大舅,悉把李三路上与的十两银子,说的言语,如此这般教他隐下,休拿出来,同他投往张二官家去:“小的怎敢忘恩负义?径奔家来。”吴大舅一面走到后边,告诉月娘:“这个小的儿,就是个知恩的。叵耐李三这厮短命,见姐夫没了几日,就这等坏心。”因把这件事就对应伯爵说:“李智、黄四借契上本利还欠六百五十两银子,趁着刚才何大人分付,把这件事写纸状子,呈到衙门里,教他替俺追追这银子来,发送姐夫。他同寮间自恁要做分上,这些事儿莫道不依。”伯爵慌了,说道:“李三却不该行此事。老舅快休动意,等我和他说罢。”于是走到李三家,请了黄四来,一处计较。说道:“你不该先把银子递与小厮,倒做了管手。狐狸打不成,倒惹了一屁股臊。如今恁般,要拿文书提刑所告你每哩。常言道官官相护,何况又同寮之间,你等怎抵斗的他过!依我,不如悄悄遂二十两银子与吴大舅,只当兖州府干了事来了。我听得说,这宗钱粮他家已是不做了,把这批文难得掣出来,咱投张二官那里去罢。你每二人再凑得二百两,少不也拿不出来,再备办一张祭桌,一者祭奠大官人,二者交这银子与他。另立一纸欠结,你往后有了买卖,慢慢还他就是了。这个一举两得,又不失了人情,有个始终。”黄四道:“你说的是。李三哥,你干事忒慌速了些。”真个到晚夕,黄四同伯爵送了二十两银子到吴大舅家,如此这般,“讨批文一节,累老舅张主张主。”这吴大舅已听见他妹子说不做钱粮,何况又黑眼见了白晃晃银子,如何不应承,于是收了银子。

到次日,李智、黄四备了一张插桌,猪首三牲,二百两银子,来与西门庆祭奠。吴大舅对月娘说了,拿出旧文书,从新另立了四百两一纸欠帖,饶了他五十两,余者教他做上买卖,陆续交还。把批文交付与伯爵手内,同往张二官处合伙,上纳钱粮去了,不在话下。正是:

\[
金逢火炼方知色,人与财交便见心。
\]
有诗为证:

\[
造物于人莫强求,劝君凡事把心收。
你今贪得收人业,还有收人在后头。
\]
