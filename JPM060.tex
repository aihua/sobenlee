%# -*- coding:utf-8 -*-
%%%%%%%%%%%%%%%%%%%%%%%%%%%%%%%%%%%%%%%%%%%%%%%%%%%%%%%%%%%%%%%%%%%%%%%%%%%%%%%%%%%%%


\chapter{李瓶儿病缠死孽\KG 西门庆官作生涯}


词曰:

\[
倦睡恹恹生怕起,如痴如醉如慵,半垂半卷旧帘栊。眼穿芳草绿,泪衬落花红。追忆当年魂梦断,为云为雨为风。凄凄楼上数归鸿。悲泪三两阵,哀绪万千重。
\]

话说潘金莲见孩子没了,每日抖擞精神,百般称快,指着丫头骂道:“贼淫妇!我只说你日头常响午,却怎的今日也有错了的时节?你斑鸠跌了蛋——也嘴答谷了。春凳折了靠背儿——没的椅了。王婆子卖了磨——推不的了。老鸨子死了粉头——没指望了。却怎的也和我一般!”李瓶儿这边屋里分明听见,不敢声言,背地里只是掉泪。着了这暗气暗恼,又加之烦恼忧戚,渐渐精神恍乱,梦魂颠倒,每日茶饭都减少了。自从葬了官哥儿第二日,吴银儿就家去了。老冯领了个十三岁的丫头来,五两银子卖与孙雪娥房中使唤,改名翠儿,不在话下。

这李瓶儿一者思念孩儿,二者着了重气,把旧病又发起来,照旧下边经水淋漓不止。西门庆请任医官来看,讨将药来吃下去,如水浇石一般,越吃越旺。那消半月之间,渐渐容颜顿减,肌肤消瘦,而精彩丰标无复昔时之态矣。正是:肌骨大都无一把,如何禁架许多愁!一日,九月初旬,天气凄凉,金风渐渐。李瓶儿夜间独宿房中,银床枕冷,纱窗月浸,不觉思想孩儿,唏嘘长叹,恍恍然恰似有人弹的窗棂响。李瓶儿呼唤丫鬓,都睡熟了不答,乃自下床来,倒靸弓鞋,翻披绣袄,开了房门。出户视之,仿佛见花子虚抱着官哥儿叫他,新寻了房儿,同去居住。李瓶儿还舍不的西门庆,不肯去,双手就抱那孩儿,被花子虚只一推,跌倒在地。撒手惊觉,却是南柯一梦。吓了一身冷汗,呜呜咽咽,只哭到天明。正是:有情岂不等,着相自家迷。有诗为证:

\[
纤纤新月照银屏,人在幽闺欲断魂。
益悔风流多不足,须知恩爱是愁根。
\]

那时,来保南京货船又到了,使了后生王显上来取车税银两。西门庆这里写书,差荣海拿一百两银子,又具羊酒金缎礼物谢主事:“就说此货过税,还望青目一二。”家中收拾铺面完备,又择九月初四日开张,就是那日卸货,连行李共装二十大车。那日,亲朋递果盒挂红者约有三十多人,夏提刑也差人送礼花红来。乔大户叫了十二名吹打的乐工、杂耍撮弄。西门庆这里,李铭、吴惠、郑春三个小优儿弹唱。甘伙计与韩伙计都在柜上发卖,一个看银子,一个讲说价钱,崔本专管收生活。西门庆穿大红,冠带着,烧罢纸,各亲友递果盒把盏毕,后边厅上安放十五张桌席,五果五菜、三汤五割,从新递酒上坐,鼓乐喧天。在坐者有乔大户、吴大舅、吴二舅、花大舅、沈姨夫、韩姨夫、吴道官、倪秀才、温葵轩、应伯爵、谢希大、常峙节,还有李智、黄四、傅自新等众伙计主管并街坊邻舍,都坐满了席面。三个小优儿在席前唱了一套《南吕·红衲袄》“混元初生太极”。须臾,酒过五巡,食割三道,下边乐工吹打弹唱,杂耍百戏过去,席上觥筹交错。应伯爵、谢希大飞起大钟来,杯来盏去。

饮至日落时分,把众人打发散了,西门庆只留下吴大舅、沈姨夫、韩姨夫、温葵轩、应伯爵、谢希大,从新摆上桌席留后坐。那日新开张,伙计攒帐,就卖了五百余两银子。西门庆满心欢喜,晚夕收了铺面,把甘伙计、韩伙计、傅伙计、崔本、贲四连陈敬济都邀来,到席上饮酒。吹打良久,把吹打乐工也打发去了,止留下三个小优儿在席前唱。

应伯爵吃的已醉上来,走出前边解手,叫过李铭问道:“那个扎包髻儿清俊的小优儿,是谁家的?”李铭道:“二爹原来不知道?”因说道:“他是郑奉的兄弟郑春。前日爹在他家吃酒,请了他姐姐爱月儿了。”伯爵道:“真个?怪道前日上纸送殡都有他。”于是归到酒席上,向西门庆道:“哥,你又恭喜,又抬了小舅子了。”西门庆笑道:“怪狗才,休要胡说。”一面叫过王经来:“斟与你应二爹一大杯酒。”伯爵向吴大舅说道:“老舅,你怎么说?这钟罚的我没名。”西门庆道:“我罚你这狗才一个出位妄言。”伯爵低头想了想儿,呵呵笑了,道:“不打紧处,等我吃,我吃死不了人。”又道:“我从来吃不得哑酒,你叫郑春上来唱个儿我听,我才罢了。”当下,三个小优一齐上来弹唱。伯爵令李铭、吴惠下去:“不要你两个。我只要郑春单弹着筝儿,只唱个小小曲儿我下酒罢。”谢希大叫道:“郑春你过来,依着你应二爹唱个罢。”西门庆道:“和花子讲过:有一个曲儿吃一钟酒。”叫玳安取了两个大银钟放在应二面前。那郑春款按银筝,低低唱《清江引》道:

\[
一个姐儿十六七,见一对蝴蝶戏。
香肩靠粉墙,春笋弹珠泪。
唤梅香赶他去别处飞。
\]
郑春唱了请酒,伯爵才饮讫,玳安又连忙斟上。郑春又唱:

\[
转过雕栏正见他,斜倚定荼蘼架;
佯羞整凤衩,不说昨宵话,笑吟吟掐将花片儿打。
\]
伯爵吃过,连忙推与谢希大,说道:“罢,我是成不的,成不的!这两大钟把我就打发了。”谢希大道:“傻花子,你吃不得推与我来,我是你家有\textuni{23B3C}的蛮子?”伯爵道:“傻花子,我明日就做了堂上官儿,少不的是你替。”西门庆道:“你这狗才,到明日只好做个韶武。”伯爵笑道:“傻孩儿,我做了韶武,把堂上让与你就是了。”西门庆笑令玳安儿:“拿磕瓜来打这贼花子!”谢希大悄悄向他头上打了一个响瓜儿,说道:“你这花子,温老先生在这里,你口里只恁胡说。”伯爵道:“温老先儿他斯文人,不管这闲事。”温秀才道:“二公与我这东君老先生,原来这等厚。酒席中间,诚然不如此也不乐。悦在心,乐主发散在外,自不觉手之舞之,足之蹈之如此。”

沈姨夫向西门庆说:“姨夫,不是这等。请大舅上席,还行个令儿——或掷骰,或猜枚,或看牌,不拘诗词歌赋、顶真续麻、急口令,说不过来吃酒。这个庶几均匀,彼此不乱。”西门庆道:“姨夫说的是。”先斟了一杯,与吴大舅起令。吴大舅拿起骰盆儿来说道:“列位,我行一令:顺着数去,遇点要个花名,花名下要顶真,不拘诗词歌赋说一句。说不来,罚一大杯。我就是一起——

\[
一掷一点红,红梅花对白梅花。”
\]
吴大舅掷了个二,多一杯。饮过酒,该沈姨夫接掷。沈姨夫说道:

\[
“二掷并头莲,莲漪戏彩鸳。”
\]
沈姨夫也掷了个二,饮过两杯,就过盆与韩姨夫行令。韩姨夫说道:

\[
“三掷三春李,李下不整冠。”
\]
韩姨夫掷完,吃了酒,送与温秀才。秀才道:“我学生奉令了——

\[
四掷状元红,红紫不以为亵服。”
\]
温秀才只遇了一杯酒,吃过,该应伯爵行令。伯爵道:“我在下一个字也不识,不会顶真,只说个急口令儿罢:

\[
一个急急脚脚的老小,左手拿着一个黄豆巴斗,右手拿着一条绵花叉口,望前只管跑走。一个黄白花狗,咬着那绵花叉口,那急急脚脚的老小,放下那左手提的那黄豆巴斗,走向前去打那黄白花狗。不知手斗过那狗,狗斗过那手。”
\]
西门庆笑骂道:“你这贼诌断肠子的天杀的,谁家一个手去逗狗来?一口不被那狗咬了?”伯爵道:“谁叫他不拿个棍儿来!我如今抄化子不见了拐棒儿——受狗的气了。”谢希大道:“大官人,你看花子自家倒了架,说他是花子。”西门庆道:“该罚他一钟,不成个令。谢子纯,你行罢!”谢希大道:“我也说一个,比他更妙:

\[
墙上一片破瓦,墙下一匹骡马。落下破瓦,打着骡马。不知是那破瓦打伤骡马,不知是那骡马踏碎了破瓦。”
\]
伯爵道:“你笑话我的令不好,你这破瓦倒好?你家娘子儿刘大姐就是个骡马,我就是个破瓦。——俺两个破磨对瘸驴。”谢希大道:“你家那杜蛮婆老淫妇,撒把黑豆只好喂猪哄狗,也不要他。”两个人斗了回嘴,每人斟了一钟,该韩伙计掷。韩道国道:“老爹在上,小人怎敢占先?”西门庆道:“顺着来,不要逊了。”于是韩道国说道:

\[
“五掷腊梅花,花里遇神仙。”
\]
掷毕,该西门庆掷,西门庆道:“我要掷个六:

\[
六掷满天星,星辰冷落碧潭水。”
\]
果然掷出个六来。应伯爵看见,说道:“哥今年上冬,管情加官进禄,主有庆事。”于是斟了一大杯酒与西门庆。一面李铭等三个上来弹唱,顽耍至更阑方散。西门庆打发小优儿出门,看收了家伙,派定韩道国、甘伙计、崔本、来保四人轮流上宿,吩咐仔细门户,就过那边去了。一宿晚景不题。

次日,应伯爵领了李智、黄四来交银子,说:“此遭只关了一千四百五六十两银子,不够还人,只挪了三百五十两银子与老爹。等下遭关出来再找完,不敢迟了。”伯爵在旁又替他说了两句美言。西门庆教陈敬济来,把银子兑收明白,打发去了。银子还摆在桌上,西门庆因问伯爵道:“常二哥说他房子寻下了,前后四间,只要三十五两银子。他来对我说,正值小儿病重,我心里乱,就打发他去了。不知他对你说来不曾?”伯爵道:“他对我说来,我说,你去的不是了,他乃郎不好,他自乱乱的,有甚么心绪和你说话?你且休回那房主儿,等我见哥,替你题就是了。”西门庆道:“也罢,你吃了饭,拿一封五十两银子,今日是个好日子,替他把房子成了来罢。剩下的,叫常二哥门面开个小铺儿,月间赚几钱银子儿,就够他两口儿盘搅了。”伯爵道:“此是哥下顾他了。”不一时,放桌儿摆上饭来,西门庆陪他吃了饭,道:“我不留你。你拿了这银子去,替他干干这勾当去罢。”伯爵道:“你这里还教个大官和我去。”西门庆道:“没的扯淡,你袖了去就是了。”伯爵道:“不是这等说,今日我还有小事。实和哥说,家表弟杜三哥生日,早晨我送了些礼儿去,他使小厮来请我后晌坐坐。我不得来回你话,教个大官儿跟了去,成了房子,好教他来回你话的。”西门庆道:“若是恁说,叫王经跟你去罢。”一面叫王经跟伯爵来到了常家。

常峙节正在家,见伯爵至,让进里面坐。伯爵拿出银子来与常峙节看,说:“大官人如此如此,教我同你今日成房子去,我又不得闲,杜三哥请我吃酒。我如今了毕你的事,我方才得去。”常峙节连忙叫浑家快看茶来,说道:“哥的盛情,谁肯!”一面吃茶毕,叫了房中人来,同到新市街,兑与卖主银子,写立房契。伯爵吩咐与王经,归家回西门庆话。剩的银子,叫与常峙节收了。他便与常峙节作别,往杜家吃酒去了。西门庆看了文契,还使王经送与常二收了,不在话下。正是:

\[
求人须求大丈夫,济人须济急时无。
一切万般皆下品,谁知恩德是良图。
\]
