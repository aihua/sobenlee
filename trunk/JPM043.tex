%# -*- coding:utf-8 -*-
%%%%%%%%%%%%%%%%%%%%%%%%%%%%%%%%%%%%%%%%%%%%%%%%%%%%%%%%%%%%%%%%%%%%%%%%%%%%%%%%%%%%%


\chapter{争宠爱金莲惹气\KG 卖富贵吴月攀亲}


词曰:

\[
情怀增怅望,新欢易失,往事难猜。问篱边黄菊,知为谁开?谩道愁须滞酒,酒未醒、愁已先回。凭栏久,金波渐转,白露点苍苔。
\]

话说西门庆归家,已有三更时分,吴月娘还未睡,正和吴大妗子众人说话,李瓶儿还伺候着与他递酒。大妗子见西门庆来家,就过那边去了。月娘见他有酒了,打发他脱了衣裳。只教李瓶儿与他磕了头,同坐下,问了回今日酒席上话。玉箫点茶来吃。因有大妗子在,就往孟玉楼房中歇了。

到次日,厨役早来收拾酒席。西门庆先到衙门中拜牌,大发放。夏提刑见了,致谢日昨房下厚扰之意。西门庆道:“日昨甚是简慢。恕罪,恕罪!”来家早有乔大户家使孔嫂儿引了乔五太太家人送礼来了。西门庆收了,家人管待酒饭。孔嫂儿进月娘房里坐的。吴舜臣媳妇儿郑三姐轿子也先来了,拜了月娘众人,都坐着吃茶。

正值李智、黄四关了一千两香蜡银子,贲四从东平府押了来家。应伯爵打听得知,亦走来帮扶交纳。西门庆令陈敬济拿天平在厅上兑明白,收了。黄四又拿出四锭金镯儿来,重三十两,算一百五十两利息之数,还欠五百两,就要捣换了合同。西门庆吩咐二人:“你等过灯节再来计较。我连日家中有事。”那李智、黄四,老爷长,老爷短,千恩万谢出门。应伯爵因记挂着二人许了他些业障儿,趁此机会好问他要,正要跟随同去,又被西门庆叫住说话。因问:“昨日你每三个,怎的三不知就走了?”伯爵道:“昨日甚是深扰哥,本等酒多了。我见哥也有酒了,今日嫂子家中摆酒,一定还等哥说话。俺每不走了,还只顾缠到多咱?我猜哥今日也没往衙门里去,本等连日辛苦。”西门庆道:“我昨日来家,已有三更天气。今日还早到衙门拜了牌,坐厅大发放,理了回公事。如今家中治料堂客之事。今日观里打上元醮,拈了香回来,还赶往周菊轩家吃酒去,不知到多咱才得到家。”伯爵道:“亏哥好神思,你的大福。不是面奖,若是第二个也成不的。”两个说了一回,西门庆要留伯爵吃饭,伯爵道:“我不吃饭,去罢。”西门庆又问:“嫂子怎的不来?”伯爵道:“房下轿子已叫下了,便来也。”举手作辞出门,一直赶黄四、李智去了。正是:

\[
假饶驾雾腾云术,取火钻冰只要钱。
\]

西门庆打发伯爵去了,手中拿着黄烘烘四锭金镯儿,心中甚是可爱,口中不言,心里暗道:“李大姐生的这孩子,甚是脚硬,一养下来,我平地就得些官。我今日与乔家结亲,又进这许多财。”于是用袖儿抱着那四锭金镯儿,也不到后边,径往李瓶儿房里来。正走到潘金莲角门首,只见金莲出来看见,叫他问道:“你手里托的是什么东西儿?过来我瞧瞧。”那西门庆道:“等我回来与你瞧。”托着一直往李瓶儿那边去了。金莲见叫不回他来,心中就有几分羞讪,说道:“什么罕稀货,忙的这等唬人子剌剌的!不与我瞧罢,贼跌折腿的三寸货强盗,进他门去,一齐的把那两条腿\textuni{22C49}折了,才现报了我的眼。”

却说西门庆拿着金子,走入李瓶儿房里,见李瓶儿才梳了头,奶子正抱着孩子顽耍。西门庆一径把四个金镯儿抱着,教他手儿挝弄。李瓶儿道:“是那里的?只怕冰了他手。”西门庆道:“是李智、黄四今日还银子准折利钱的。”李瓶儿生怕冰着他,取了一方通花汗巾儿,与他裹着耍子。只见玳安走来说道:“云伙计骑了两匹马来,在外边请爹出去瞧。”西门庆问道:“云伙计他是那里的马?”玳安道:“他说是他哥云参将边上捎来的。”正说着,只见后边李娇儿、孟玉楼陪着大妗子并他媳妇郑三姐,都来李瓶儿房里看官哥儿。西门庆丢了那四锭金子,就往外边看马去了。

李瓶儿见众人来到,只顾与众人见礼让坐,也就忘记了孩子拿着这金子,弄来弄去,少了一锭。只见奶子如意儿问李瓶儿道:“娘没曾收哥哥儿耍的那锭金子?怎只三锭,少了一锭了?”李瓶儿道:“我没曾收,我把汗个子替他裹着哩。”如意儿道:“汗巾子也落在地下了。那里得那锭金子?”屋里就乱起来。奶子问迎春,迎春就问老冯。老冯道:“耶嚛,耶嚛!我老身就瞎了眼,也没看见。老身在这里恁几年,莫说折针断线我不敢动,娘他老人家知道我,就是金子,我老身也不爱。你每守着哥儿,怎的冤枉起我来了!”李瓶儿笑道:“你看这妈妈子说混话,这里不见的,不是金子却是什么?”又骂迎春:“贼臭肉!平白乱的是些甚么?等你爹进来,等我问他,只怕是你爹收了。怎的只收一锭儿?”孟玉楼问道:“是那里金子?”李瓶儿道:“是他爹拿来的,与孩子耍。谁知道是那里的。”

且说西门庆在门首看马,众伙计家人都在跟前,叫小厮来回溜了两趟。西门庆道:“虽是东路来的马,鬃尾丑,不十分会行,论小行也罢了。”因问云伙计道:“此马你令兄那里要多少银子?”云离守道:“两匹只要七十两。”西门庆道:“也不多。只是不会行,你还牵了去,另有好马骑来,倒不说银子。”说毕,西门庆进来,只见琴童来说:“六娘房里请爹哩。”于是走入李瓶儿房里来。李瓶儿问他:“金子你收了一锭去了?如何只三锭在这里?”西门庆道:“我丢下,就外边去看马,谁收来!”李瓶儿道:“你没收,却往那里去了?寻了这一日没有。奶子推老冯,急的那老冯赌身罚咒,只是哭。”西门庆道:“端的是谁拿了,由他慢慢儿寻罢。”李瓶儿道:“头里因大妗子女儿两个来,乱着就忘记了。我只说你收了出去,谁知你也没收,就两耽了。才寻起来,唬的他们都走了。”于是把那三锭,还交与西门庆收了。正值贲四倾了一百两银子来交,西门庆就往后边收兑银子去了。

且说潘金莲听见李瓶儿这边嚷,不见了孩子耍的一锭金镯子,得不的风儿就是雨儿,就先走来房里,告月娘说:“姐姐,你看三寸货干的营生!随你家怎的有钱,也不该拿金子与孩子耍。”月娘道:“刚才他每告我说,他房里不见了金镯子,端的不知是那里的?”金莲道:“谁知他是那里的!你还没见,他头里从外边拿进来,用袄子袖儿裹着,恰似八蛮进宝的一般。我问他是什么,拿过来我瞧瞧。头儿也不回,一直奔命往屋里去了。迟了一回,反乱起来,说不见了一锭金子。干净就是他学三寸货,说不见了,由他慢慢儿寻罢。你家就是王十万也使不的。一锭金子,至少重十到两,也值五六十两银子,平白就罢了?瓮里走了鳖——左右是他家一窝子。再有谁进他屋里去?”正说着,只见西门庆进来,兑收贲四倾的银子,把剩的那三锭金子交与月娘收了。因告诉月娘:“此是李智、黄四还的四锭金子,拿了与孩子耍了耍,就不见了一锭。”吩咐月娘:“你与我把各房里丫头叫出来审问审问。我使小厮街上买狼筋去了,早拿出来便罢,不然,我就叫狼筋抽起来。”月娘道:“论起来,这金子也不该拿与孩子,沉甸甸冰着他,一时砸了他手脚怎了!”潘金莲在旁接过来说道:“不该拿与孩子耍?只恨拿不到他屋里。头里叫着,想回头也怎的,恰似红眼军抢将来的,不教一个人儿知道。这回不见了金子,亏你怎么有脸儿来对大姐姐说!叫大姐姐替你查考各房里丫头,叫各房里丫头口里不笑,\textuni{23B48}眼里也笑!”

几句说的西门庆急了,走向前把金莲按在月娘炕上,提起拳来,骂道:“狠杀我罢了!不看世界面上,把你这小\textuni{22C49}剌骨儿,就一顿拳头打死了!单管嘴尖舌快的,不管你事也来插一脚。”那潘金莲就假做乔妆,哭将起来,说道:“我晓的你倚官仗势,倚财为主,把心来横了,只欺负的是我,你说你这般威势,把一个半个人命儿打死了,不放在意里。那个拦着你手儿哩不成?你打不是的!我随你怎么打,难得只打得有这口气儿在着,若没了,愁我家那病妈妈子不问你要人!随你家怎么有钱有势,和你家一递一状。你说你是衙门里千户便怎的?无故只是个破纱帽债壳子——穷官罢了,能禁的几个人命?就不是教皇帝敢杀下人也怎么!”几句说的西门庆反呵呵笑了,说道:“你看这小\textuni{22C49}剌骨儿,这等刁嘴!我是破纱帽穷官?教丫头取我的纱帽来,我这纱帽那块儿破?这清河县问声,我少谁家银子?你说我是债壳子!”金莲道:“你怎的叫我是\textuni{22C49}剌骨来!”因跷起一只脚来,“你看老娘这脚,那些儿放着歪?你怎骂我是\textuni{22C49}剌骨?”月娘在旁笑道:“你两个铜盆撞了铁刷帚。常言:恶人自有恶人磨,见了恶人没奈何!自古嘴强的争一步。六姐,也亏你这个嘴头子,不然,嘴钝些儿也成不的。”

那西门庆见奈何不过他,穿了衣裳往外去了。迎见玳安来说:“周爷家差人邀来了。请问爹先往打醮处去,往周爷家去?”西门庆吩咐:“打醮处,教你姐夫去罢。伺候马,我往你周爷家吃酒去就是了。”只见王皇亲家扮戏两个师父率众过来,与西门庆叩头,西门庆教书童看饭与他吃,说:“今日你等用心伏侍众奶奶,我自有重赏,休要上边打箱去!”那师父跪下说道:“小的每若不用心答应,岂敢讨赏!”西门庆因吩咐书童:“他唱了两日,连赏赐封下五两银子赏他。”书童应诺。西门庆就上马往周守备家吃酒去了。

单表潘金莲在上房坐的,吴月娘便说:“你还不往屋里匀匀那脸去!揉的恁红红的。等住回人来看着甚么张致!谁叫你惹他来?我倒替你捏两把汗。若不是我在跟前劝着,绑着鬼,是也有几下子打在身上。汉子家脸上有狗毛,不知好歹,只顾下死手的和他缠起来了。不见了金子,随他不见去,寻不寻不在你,又不在你屋里不见了,平白扯着脖子和他强怎么!你也丢了这口气儿罢!”几句说的金莲闭口无言,往屋里匀脸去了。

不一时,李瓶儿和吴银儿都打扮出来,到月娘房里。月娘问他:“金子怎的不见了?刚才惹他爹和六姐两个,在这里好不辨了这回嘴,差些儿没曾辨恼了打起来!吃我劝开了。他爹就往人家吃酒去了。吩咐小厮买狼筋去了。等他晚上来家,要把各房丫头抽起来。你屋里丫头老婆管着那一门儿来?看着孩子耍,便不见了他一锭金子。是一个半个钱的东西儿也怎的?”李瓶儿道:“平白他爹拿进四锭金子来与孩子耍,我乱着陪大妗子和郑三姐并他二娘坐着说话,谁知就不见了一锭。如今丫头推奶子,奶子推老冯。急的冯妈妈哭哭啼啼,只要寻死。无眼难明勾当,如今冤谁的是?”吴银儿道:“天么,天么!每常我还和哥儿耍子,早是今日我在这边屋里梳头,没曾过去。不然怎了?虽然爹娘不言语,你我心上何安!谁人不爱钱?俺里边人家,最忌叫这个名声儿,传出去丑听!”

正说着,只见韩玉钏儿、董娇儿两个提着衣包儿进来,笑嘻嘻先向月娘、大妗子、李瓶儿磕了头,起来望着吴银儿拜了一拜,说道:“银姐昨日没家去?”吴银儿道:“你怎的晓得?”董娇儿道:“昨日,俺两个都在灯市街房子里唱来,大爹对俺们说,教俺今日来伏侍奶奶。”一面月娘让他两个坐下。须臾,小玉拿了两盏茶来。那韩玉钏儿、董娇儿连忙立起身来接茶,还望小玉拜了一拜。吴银儿因问:“你两个昨日唱多咱散了?”韩玉钏道:“俺们到家,也有二更多了,同你兄弟吴惠都一路去的。”说了一回话,月娘吩咐玉箫:“早些打发他们吃了茶罢。等住回只怕那边人来忙了。”一面放下桌儿,两方春槅、四盒茶食。月娘使小玉:“你二娘房里,请了桂姐来同吃了茶罢。”不一时,和他姑娘来到,两个各道了礼数坐下,同吃了茶,收过家活去。

忽见迎春打扮着,抱了官哥儿来,头上戴了金梁缎子八吉祥帽儿,身穿大红氅衣儿,下边白绫袜儿、缎子鞋儿,胸前项牌符索,手上小金镯儿。李瓶儿看见说道:“小大官儿,没人请你,来做什么?”一面接过来,放在膝盖上。看见一屋里人,把眼不住的看了这个,又看那个。桂姐坐在月娘炕上,笑引逗他耍子,道:“哥子只看着这里,想必要我抱他。”于是用手引了他引儿,那孩子就扑到怀里教他抱。吴大妗子笑道:“恁点小孩儿,他也晓的爱好!”月娘接过来说:“他老子是谁!到明日大了,管情也是小嫖头儿。”孟玉楼道:“若做了小嫖头儿,叫大妈妈就打死了。”李瓶儿道:“小厮,你姐姐抱,只休溺了你姐姐衣服,我就打死了!”桂姐道:“耶嚛!怕怎么?溺了也罢,不妨事。我心里要抱哥儿耍耍儿。”于是与他两个嘴揾嘴儿耍子。董娇儿、韩玉钏儿说道:“俺两个来了这一日,还没曾唱个儿与娘每听。”因取乐器,韩玉钏儿琵琶,董娇儿弹筝,吴银儿也在旁边陪唱。唱了一套“繁华满月开”《金索挂梧桐》。唱出一句来,端的有落尘绕梁之声,裂石流云之响,把官哥儿唬的在桂姐怀里只磕倒着,再不敢抬头出气儿。月娘看见,便叫:“李大姐,你接过孩子来,教迎春抱到屋里去罢。好个不长进的小厮,你看唬的那脸儿!”这李瓶儿连忙接过来,叫迎春掩着他耳朵,抱的往那边房里去了。

四个唱的正唱着,只见玳安进来,说道:“小的到乔亲家娘那边邀来,朱奶奶、尚举人娘子,都过乔亲家来了,只等着乔五太太到了就来了。大门前边、大厅上,都有鼓乐迎接。娘每都收拾伺候就是了。”月娘又吩咐后厅明间铺下锦毯,安放坐位。卷起帘来,金钩双控,兰麝香飘。春梅、迎春、玉箫、兰香,都打扮起来。家人媳妇都插金戴银,披红垂绿,准备迎接新亲。只见应伯爵娘子应二嫂先到了,应保跟着轿子。月娘等迎接进来。见了礼数,明间内坐下,向月娘拜了又拜,说:“俺家的常时打搅,多蒙看顾!”月娘道:“二娘,好说!常时累你二爹。”良久,只闻喝道之声渐近,前厅鼓乐响动。平安儿先进来报道:“乔太太轿子到了!”须臾,黑压压一群人,跟着五顶大轿落在门首。惟乔五太太轿子在头里,轿上是垂珠银顶、天青重沿、绡金走水轿衣,使藤棍喝路。后面家人媳妇坐小轿跟随,四名校尉抬衣箱、火炉,两个青衣家人骑着小马,后面随从。其余就是乔大户娘子、朱台官娘子、尚举人娘子、崔大官媳妇、段大姐,并乔通媳妇也坐着一顶小轿,跟来收叠衣裳。

吴月娘与李娇儿、孟玉楼、潘金莲、李瓶儿、孙雪娥,一个个打扮的似粉妆玉琢,锦绣耀目,都出二门迎接。众堂客簇拥着乔五太太进来。生的五短身材,约七旬年纪,戴着叠翠宝珠冠,身穿大红宫绣袍儿,近面视之,鬓发皆白。正是:眉分八道雪,髻绾一窝丝,眼如秋水微浑,鬓似楚山云淡。接入后厅,先与吴大妗子叙毕礼数,然后与月娘等厮见。月娘再三请太太受礼,太太不肯,让了半日,受了半礼。次与乔大户娘子,又叙其新亲家之礼,彼此道及款曲,谢其厚仪。已毕,然后向锦屏正面设放一张锦裀座位,坐了乔五太太,其次就让乔大户娘子。乔大户娘子再三辞说:“侄妇不敢与五太太上僭。”让朱台官、尚举人娘子,两个又不肯。彼此让了半日,乔五太太坐了首座,其余客东主西,两分头坐了。当中大方炉火厢笼起火来,堂中气暖如春。春梅、迎春、玉箫、兰香,一般儿四个丫头,都打扮起来,在跟前递茶。

良久,乔五太太对月娘说:“请西门大人出来拜见,叙叙亲情之礼。”月娘道:“拙夫今日衙门中去了,还未来家哩!”乔五太太道:“大人居于何官?”月娘道:“乃一介乡民,蒙朝廷恩例,实授千户之职,见掌刑名。寒家与亲家那边结亲,实是有玷。”乔五太太道:“娘子说那里话,似大人这等峥嵘也彀了。昨日老身听得舍侄妇与府上做亲,心中甚喜。今日我来会会,到明日好厮见。”月娘道:“只是有玷老太太名目。”乔五太太道:“娘子是甚说话,想朝廷不与庶民做亲哩!老身说起来话长,如今当今东宫贵妃娘娘,系老身亲侄女儿。他父母都没了,止有老身。老头儿在时,曾做世袭指挥使,不幸五十岁故了。身边又无儿孙,轮着别门侄另替了,手里没钱,如今倒是做了大户。我这个侄儿,虽是差役立身,颇得过的日子,庶不玷污了门户。”说了一回,吴大妗子对月娘说:“抱孩子出来与老太太看看,讨讨寿。”李瓶儿慌吩咐奶子,抱了官哥来与太太磕头。乔太太看了夸道:“好个端正的哥哥!”即叫过左右,连忙把毡包内打开,捧过一端宫中紫闪黄锦缎,并一副镀金手镯,与哥儿戴。月娘连忙下来拜谢了。请去房中换了衣裳。须臾,前边卷棚内安放四张桌席摆茶,每桌四十碟,都是各样茶果、细巧油酥之类。吃了茶,月娘就引去后边山子花园中,游玩了一回下来。

那时,陈敬济打醮去,吃了午斋回来了。和书童儿、玳安儿,又早在前厅摆放桌席齐整,请众奶奶每递酒上席。端的好筵席,但见:

\[
屏开孔雀,褥隐芙蓉。盘堆异果奇珍,瓶插金花翠叶。炉焚兽炭,香袅龙涎。白玉碟高堆麟脯,紫金壶满贮琼浆。梨园子弟,簇捧着凤管鸾箫;内院歌姬,紧按定银筝象板。进酒佳人双洛浦,分香侍女两姮娥。正是:两行珠翠列阶前,一派笙歌临坐上。
\]

吴月娘与李瓶儿同递酒,阶下戏子鼓乐响动。乔太太与众亲戚,又亲与李瓶儿把盏祝寿,方入席坐下。李桂姐、吴银儿、韩玉钏儿、董娇儿四个唱的,在席前唱了一套“寿比南山”。戏子呈上戏文手本,乔五太太吩咐下来,教做《王月英元夜留鞋记》。厨役上来献小割烧鹅,赏了五钱银子。比及割凡五道,汤陈三献,戏文四折下来,天色已晚。堂中画烛流光,各样花灯都点起来,锦带飘飘,彩绳低转。一轮明月从东而起,照射堂中灯光掩映。乐人又在阶下,琵琶筝\textuni{25C67},笙箫笛管,吹打了一套灯词《画眉序》“花月满香城”。吹打毕,乔太太和乔大户娘子叫上戏子,赏了两包一两银子,四个唱的,每人二钱。月娘又在后边明间内,摆设下许多果碟儿,留后坐。四张桌子都堆满了。唱的唱,弹的弹,又吃了一回酒。乔太太再三说晚了,要起身。月娘众人款留不住,送在大门首,又拦门递酒,看放烟火。两边街上,看的人鳞次蜂排一般。平安儿同众排军执棍拦挡再三,还涌挤上来。须臾,放了一架烟火,两边人散了。乔太大和众娘子方才拜辞月娘等,起身上轿去了。那时也有三更天气,然后又送应二嫂起身。月娘众姐妹归到后边来,吩咐陈敬济、来兴、书童、玳安儿,看着厅上收拾家活,管待戏子并两个师范酒饭,与了五两银子唱钱,打发去了。

月娘吩咐出来,剩攒下一桌肴馔、半罐酒,请傅伙计、贲四、陈姐夫,说:“他每管事辛苦,大家吃锺酒。就在大厅上安放一张桌儿,你爹不知多咱才回。”于是还有残灯未尽,当下傅伙计、贲四、敬济、来保上坐,来兴、书童、玳安、平安打横,把酒来斟。来保叫平安儿:“你还委个人大门首,怕一时爹回,没人看门。”平安道:“我叫画童看着哩,不妨事。”于是八个人猜枚饮酒。敬济道:“你每休猜枚,大惊小怪的,惹后边听见。咱不如悄悄行令儿耍子。每人要一句,说的出免罚,说不出罚一大杯。”该傅伙计先说:“堪笑元宵草物。”贲四道:“人生欢乐有数。”敬济道:“趁此月色灯光。”来保道:“咱且休要辜负。”来兴道:“才约娇儿不在。”书童道:“又学大娘吩咐。”玳安道:“虽然剩酒残灯。”平安道:“也是春风一度。”众人念毕,呵呵笑了。正是:

\[
饮罢酒阑人散后,不知明月转花梢。
\]
