%# -*- coding:utf-8 -*-
%%%%%%%%%%%%%%%%%%%%%%%%%%%%%%%%%%%%%%%%%%%%%%%%%%%%%%%%%%%%%%%%%%%%%%%%%%%%%%%%%%%%%


\chapter{花子虚因气丧身\KG 李瓶儿迎奸赴会}


诗曰:

\[
眼意心期未即休,不堪拈弄玉搔头。
春回笑脸花含媚,黛蹙娥眉柳带愁。
粉晕桃腮思伉俪,寒生兰室盼绸缪。
何如得遂相如意,不让文君咏白头。
\]

话说一日吴月娘心中不快,吴大妗子来看,月娘留他住两日。正陪在房中坐的,忽见小厮玳安抱进毡包来,说:“爹来家了。”吴大妗子便往李娇儿房里去了。西门庆进来,脱了衣服坐下。小玉拿茶来也不吃。月娘见他面色改常,便问:“你今日会茶,来家恁早?”西门庆道:“今该常二哥会,他家没地方,请俺们在城外永福寺去耍子。有花二哥邀了应二哥,俺们四五个,往院里郑爱香儿家吃酒。正吃着,忽见几个做公的进来,不由分说,把花二哥拿的去了。把众人吓了一惊。我便走到李桂姐躲了半日,不放心,使人打听。原来是花二哥内臣家房族中告家财,在东京开封府递了状子,批下来,着落本县拿人。俺们才放心,各人散归家来。”月娘闻言,便道:“这是正该的,你整日跟着这伙人,不着个家,只在外边胡撞;今日只当丢出事来,才是个了手。你如今还不心死。到明日不吃人挣锋厮打,群到那日是个烂羊头,你肯断绝了这条路儿!正经家里老婆的言语说着你肯听?只是院里淫妇在你跟前说句话儿,你到着个驴耳朵听他。正是:家人说着耳边风,外人说着金字经。”西门庆笑道:“谁人敢七个头八个胆打我!”月娘道:“你这行货子,只好家里嘴头子罢了。”

正说着,只见玳安走来说:“隔壁花二娘使天福儿来,请爹过去说话。”这西门庆听了,趔趄脚儿就往外走。月娘道:“明日没的教人讲你把。”西门庆道:“切邻间不防事。我去到那里,看他有甚么话说。”当下走过花子虚家来,李瓶儿使小厮请到后边说话,只见妇人罗衫不整,粉面慵妆,从房里出来,脸吓的蜡渣也似黄,跪着西门庆,再三哀告道:“大官人没奈何,不看僧面看佛面,常言道:家有患难,邻里相助。因他不听人言,把着正经家事儿不理,只在外边胡行。今日吃人暗算,弄出这等事来。这时节方对小厮说将来,教我寻人情救他。我一个妇人家没脚的,那里寻那人情去。发狠起来,想着他恁不依说,拿到东京,打的他烂烂的,也不亏他。只是难为过世老公公的姓字。奴没奈何,请将大官人过来,央及大官人,把他不要提起罢,千万看奴薄面,有人情好歹寻一个儿,只不教他吃凌逼便了。”西门庆见妇人下礼,连忙道:“嫂子请起来,不妨,我还不知为了甚勾当。”妇人道:“正是一言难尽。俺过世老公公有四个侄儿,大侄儿唤做花子由,第三个唤花子光,第四个叫花子华,俺这个名花子虚,都是老公公嫡亲的。虽然老公公挣下这一分钱财,见我这个儿不成器,从广南回来,把东西只交付与我手里收着。着紧还打倘棍儿,那三个越发打的不敢上前。去年老公公死了,这花大、花三、花四,也分了些床帐家伙去了,只现一分银子儿没曾分得。我常说,多少与他些也罢了,他通不理一理儿。今日手暗不通风,却教人弄下来了。”说毕,放声大哭。西门庆道:“嫂子放心,我只道是甚么事来,原来是房分中告家财事,这个不打紧。既是嫂子分付,哥的事就是我的事一般,随问怎的,我在下谨领。”妇人说道:“官人若肯时又好了。请问寻分上,要用多少礼儿,奴好预备。”西门庆道:“也用不多,闻得东京开封府杨府尹,乃蔡太师门生。蔡太师与我这四门亲家杨提督,都是当朝天子面前说得话的人。拿两个分上,齐对杨府尹说,有个不依的!不拘多大事情也了了。如今倒是蔡太师用些礼物。那提督杨爷与我舍下有亲,他肯受礼?”妇人便往房中开箱子,搬出六十锭大元宝,共计三千两,教西门庆收去寻人情,上下使用。西门庆道:“只一半足矣,何消用得许多!”妇人道:“多的大官人收了去。奴床后还有四箱柜蟒衣玉带,帽顶绦环,都是值钱珍宝之物,亦发大官人替我收去,放在大官人那里,奴用时来取。趁这时,奴不思个防身之计,信着他,往后过不出好日子来。眼见得三拳敌不得四手,到明日,没的把这些东西儿吃人暗算了去,坑闪得奴三不归!”西门庆道:“只怕花二哥来家寻问怎了?”妇人道:“这都是老公公在时,梯己交与奴收着之物,他一字不知。大官人只顾收去。”西门庆说道:“既是嫂子恁说,我到家教人来取。”于是一直来家,与月娘商议。月娘说:“银子便用食盒叫小厮抬来。那箱笼东西,若从大门里来,教两边街坊看着不惹眼?必须夜晚打墙上过来方隐密些。”西门庆听言大喜,即令玳安、来旺、来兴、平安四个小厮,两架食盒,把三千两银子先抬来家。然后到晚夕月上时分,李瓶儿那边同迎春、绣春放桌凳,把箱柜挨到墙上。西门庆这边,止是月娘、金莲、春梅,用梯子接着。墙头上铺衬毡条,一个个打发过来,都送到月娘房中去了。正是:

\[
富贵自是福来投,利名还有利名忧。
命里有时终须有,命里无时莫强求。
\]
西庆收下他许多细软金银宝物,邻舍街坊俱不知道。连夜打点驮装停当,求了他亲家陈宅一封书,差家人来保上东京。送上杨提督书礼,转求内阁蔡太师柬帖下与开封府杨府尹。这府尹名唤杨时,别号龟山,乃陕西弘农县人氏,由癸未进士升大理寺卿,今推开封府尹,极是清廉。况蔡太师是他旧时座主,杨戬又是当道时臣,如何不做分上!当日杨府尹升厅,监中提出花子虚来,一干人上厅跪下,审问他家财下落。此时花子虚已有西门庆捎书知会了,口口只说:“自从老公公死了,发送念经,都花费了。止有宅舍两所、庄田一处见在,其余床帐家火物件,俱被族人分散一空。”杨府尹道:“你们内官家财,无可稽考,得之易,失之易。既是花费无存,批仰清河县委官将花太监住宅二所、庄田一处,估价变卖,分给花子由等三人回缴。”花子由等又上前跪禀,还要监追子虚,要别项银两。被杨府尹大怒,都喝下来,说道:“你这厮少打!当初你那内相一死之时,你每不告做甚么来?如今事情已往,又来骚扰。”于是把花子虚一下儿也没打,批了一道公文,押发清河县前来估计庄宅,不在话下。

来保打听这消息,星夜回来,报知西门庆。西门庆听见分上准了,放出花子虚来家,满心欢喜。这里李瓶儿请过西门庆去计议,要叫西门庆拿几两银子,买了这所住的宅子:“到明日,奴不久也是你的人了。”西门庆归家与吴月娘商议。月娘道:“你若要他这房子,恐怕他汉子一时生起疑心来,怎了?”西门庆听记在心。那消几日,花子虚来家,清河县委下乐县丞丈估:太监大宅一所,坐落大街安庆坊,值银七百两,卖与王皇亲为业;南门外庄田一处,值银六百五十两,卖与守备周秀为业。止有住居小宅,值银五百四十两,因在西门庆紧隔壁,没人敢买。花子虚再三使人来说,西门庆只推没银子,不肯上帐。县中紧等要回文书,李瓶儿急了,暗暗使冯妈妈来对西门庆说,教拿他寄放的银子兑五百四十两买了罢。这西门庆方才依允。当官交兑了银两,花子由都画了字。连夜做文书回了上司,共该银一千八百九十五两,三人均分讫。

花子虚打了一场官司出来,没分的丝毫,把银两、房舍、庄田又没了,两箱内三千两大元宝又不见踪影,心中甚是焦躁。因问李瓶儿查算西门庆使用银两下落,今还剩多少,好凑着买房子。反吃妇人整骂了四五日,骂道:“呸!魉魉混沌,你成日放着正事儿不理,在外边眠花卧柳,只当被人弄成圈套,拿在牢里,使将人来教我寻人情。奴是个女妇人家,大门边儿也没走,晓得甚么?认得何人?那里寻人情?浑身是铁打得多少钉儿?替你添羞脸,到处求爹爹告奶奶。多亏了隔壁西门大官人,看日前相交之情,大冷天,刮得那黄风黑风,使了家下人往东京去,替你把事儿干得停停当当的。你今日了毕官司,两脚站在平川地,得命思财,疮好忘痛,来家到问老婆找起后帐儿来了,还说有也没有。你写来的帖子现在,没你的手字儿,我擅自拿出你的银子寻人情,抵盗与人便难了!”花子虚道:“可知是我的帖子来说,实指望还剩下些,咱凑着买房子过日子。”妇人道:“呸!浊蠢才!我不好骂你的。你早仔细好来,咊头儿上不算计,圈底儿下却算计。千也说使多了,万也说使多了,你那三千两银子能到的那里?蔡太师、杨提督好小食肠儿!不是恁大人情,平白拿了你一场,当官蒿条儿也没曾打在你这忘八身上,好好儿放出来,教你在家里恁说嘴!人家不属你管辖,你是他甚么着疼的亲?平白怎替你南上北下走跳,使钱教你!你来家也该摆席酒儿,请过人来,知谢人一知谢儿,还一扫帚扫得人光光的,到问人找起后帐儿来了!”几句连搽带骂,骂的子虚闭口无言。

到次日,西门庆使玳安送了一分礼来与子虚压惊。子虚这里安排了一席,请西门庆来知谢,就要问他银两下落。依着西门庆,还要找过几百两银子与他凑买房子。到是李瓶儿不肯,暗地使冯妈妈过来对西门庆说:“休要来吃酒,只开送一篇花帐与他,说银子上下打点都使没了。”花子虚不识时,还使小厮再三邀请。西门庆躲的一径往院里去了,只回不在家。花子虚气的发昏,只是跌脚。看观听说:大凡妇人更变,不与男子汉一心,随你咬折铁钉般刚毅之夫,也难测其暗地之事。自古男治外而女治内,往往男子之名都被妇人坏了者为何?皆由御之不得其道。要之在乎容德相感,缘分相投,夫唱妇随,庶可保其无咎。若似花子虚落魄飘风,谩无纪律,而欲其内人不生他意,岂可得乎!正是:

\[
自意得其垫,无风可动摇。
\]

话休饶舌。后来子虚只摈凑了二百五十两银子,买了狮子街一所房屋居住。得了这口重气,刚搬到那里,又不幸害了一场伤寒,从十一月初旬,睡倒在床上,就不曾起来。初时还请太医来看,后来怕使钱,只挨着。一日两,两日三,挨到二十头,呜呼哀哉,断气身亡,亡年二十四岁。那手下的大小厮天喜儿,从子虚病倒之时,就拐了五两银子走的无踪。子虚一倒了头,李瓶儿就使冯妈妈请了西门庆过去,与他商议买棺入殓,念经发送,到坟上安葬。那花大、花三、花四一般儿男妇,也都来吊孝送殡。西门庆那日也教吴月娘办了一张桌席,与他山头祭奠。当日妇人轿子归家,也设了一个灵位,供养在房中。虽是守灵,一心只想着西门庆。从子虚在日,就把两个丫头教西门庆耍了,子虚死后,越发通家往还。

一日,正值正月初九,李瓶儿打听是潘金莲生日,未曾过子虚五七,李瓶儿就买礼物坐轿子,穿白绫袄儿,蓝织金裙,白纻布鬒髻,珠子箍儿,来与金莲做生日。冯妈妈抱毡包,天福儿跟轿。进门先与月娘磕了四个头,说道:“前日山头多劳动大娘受饿,又多谢重礼。”拜了月娘,又请李娇儿、孟玉楼拜见了。然后潘金莲来到,说道:“这位就是五娘?”又要磕下头去,一口一声称呼:“姐姐,请受奴一礼儿。”金莲那里肯受,相让了半日,两个还平磕了头。金莲又谢了他寿礼。又有吴大妗子、潘姥姥一同见了。李瓶儿便请西门庆拜见。月娘道:“他今日往门外玉皇庙打醮去了。”一面让坐了,唤茶来吃了。良久,只见孙雪娥走过来。李瓶儿见他妆饰少次于众人,便起身来问道:“此位是何人?奴不知,不曾请见得。”月娘道:“此是他姑娘哩。”李瓶儿就要行礼。月娘道:“不劳起动二娘,只是平拜拜儿罢。”于是彼此拜毕,月娘就让到房中,换了衣裳,分付丫鬟,明间内放桌儿摆茶。须臾,围炉添炭,酒泛羊羔,安排上酒来。让吴大妗子、潘姥姥、李瓶儿上坐,月娘和李娇儿主席,孟玉楼和潘金莲打横。孙雪娥回厨下照管,不敢久坐。月娘见李瓶儿锺锺酒都不辞,于是亲自递了一遍酒,又令李娇儿众人各递酒一遍,因嘲问他话儿道:“花二娘搬的远了,俺姊妹们离多会少,好不思想。二娘狠心,就不说来看俺们看见?”孟玉楼便道:“二娘今日不是因与六姐做生日还不来哩!”李瓶儿道:“好大娘,三娘,蒙众娘抬举,奴心里也要来,一者热孝在身,二者家下没人。昨日才过了他五七,不是怕五娘怪,还不敢来。”因问:“大娘贵降在几时?”月娘道:“贱日早哩。”潘金莲接过来道:“大娘生日是八月十五,二娘好歹来走走。”李瓶儿道:“不消说,一定都来。”孟玉楼道:“二娘今日与俺姊妹相伴一夜儿,不往家去罢了。”李瓶儿道:“奴可知也要和众位娘叙些话儿。不瞒众位娘说,小家儿人家,初搬到那里,自从他没了,家下没人,奴那房子后墙紧靠着乔皇亲花园,好不空!晚夕常有狐狸抛砖掠瓦,奴又害怕。原是两个小厮,那个大小厮又走了,止是这个天福儿小厮看守前门,后半截通空落落的。倒亏了这个老冯,是奴旧时人,常来与奴浆洗些衣裳。”月娘因问:“老冯多少年纪?且是好个恩实妈妈儿,高大言也没句儿。”李瓶儿道:“他今年五十六岁,男花女花都没,只靠说媒度日。我这里常管他些衣裳。昨日拙夫死了,叫过他来与奴做伴儿,晚夕同丫头一炕睡。”潘金莲嘴快,说道:“既有老冯在家里看家,二娘在这里过一夜也不妨,左右你花爹没了,有谁管着你!”玉楼道:“二娘只依我,叫老冯回了轿子,不去罢。”那李瓶儿只是笑,不做声。话说中间,酒过数巡。潘姥姥先起身往前边去了。潘金莲随跟着他娘往房里去了。李瓶儿再三辞道:“奴的酒勾了。”李娇儿道:“花二娘怎的,在他大娘、三娘手里肯吃酒,偏我递酒,二娘不肯吃?显的有厚薄。”遂拿个大杯斟上。李瓶儿道:“好二娘,奴委的吃不去了,岂敢做假!”月娘道:“二娘,你吃过此杯,略歇歇儿罢。”那李瓶儿方才接了,放在面前,只顾与众人说话。孟玉楼见春梅立在旁边,便问春梅:“你娘在前边做甚么哩?你去连你娘、潘姥姥快请来,就说大娘请来陪你花二娘吃酒哩。”春梅去不多时,回来道:“姥姥害身上疼,睡哩。俺娘在房里匀脸,就来。”月娘道:“我倒也没见,他倒是个主人家,把客人丢了,三不知往房里去了。诸般都好,只是有这些孩子气。”有诗为证:

\[
倦来汗湿罗衣彻,楼上人扶上玉梯。
归到院中重洗面,金盆水里发红泥。
\]

正说着,只见潘金莲走来。玉楼在席上看见他艳抹浓妆,从外边摇摆将来,戏道:“五丫头,你好人儿!今日是你个驴马畜,把客人丢在这里,你躲到房里去了,你可成人养的!”那金莲笑嘻嘻向他身上打了一下。玉楼道:“好大胆的五丫头!你还来递一锺儿。”李瓶儿道:“奴在三娘手里吃了好少酒儿,也都勾了。”金莲道:“他手里是他手里帐,我也敢奉二娘一锺儿。”于是满斟一大锺递与李瓶儿。李瓶儿只顾放着不肯吃。月娘因看见金莲鬓上撇着一根金寿字簪儿,便问:“二娘,你与六姐这对寿字簪儿,是那里打造的?倒好样儿。到明日俺每人照样也配恁一对儿戴。”李瓶儿道:“大娘既要,奴还有几对,到明日每位娘都补奉上一对儿。此是过世老公公御前带出来的,外边那里有这样范!”月娘道:“奴取笑斗二娘耍子。俺姐妹们人多,那里有这些相送!”众女眷饮酒欢笑。

看看日西时分,冯妈妈在后边雪娥房里管待酒,吃的脸红红的出来,催逼李瓶儿道:“起身不起身?好打发轿子回去。”月娘道:“二娘不去罢,叫老冯回了轿子家去罢。”李瓶儿说:“家里无人,改日再奉看众位娘,有日子住哩。”孟玉楼道:“二娘好执古,俺众人就没些儿分上?如今不打发轿子,等住回他爹来,少不的也要留二娘。”自这说话,逼迫的李瓶儿就把房门钥匙递与冯妈妈,说道:“既是他众位娘再三留我,显的奴不识敬重。分付轿子回去,教他明日来接罢。你和小厮家去,仔细门户。”又教冯妈妈附耳低言:“教大丫头迎春,拿钥匙开我床房里头一个箱子,小描金头面匣儿里,拿四对金寿字簪儿。你明日早送来,我要送四位娘。”那冯妈妈得了话,拜辞了月娘,一面出门,不在话下。

少顷,李瓶儿不肯吃酒,月娘请到上房,同大妗子一处吃茶坐的。忽见玳安抱进毡包,西门庆来家,掀开帘子进来,说道:“花二娘在这里!”慌的李瓶儿跳起身来,两个见了礼,坐下。月娘叫玉箫与西门庆接了衣裳。西门庆便对吴大妗子、李瓶儿说道:“今日门外玉皇庙圣诞打醮,该我年例做会首,与众人在吴道官房里算帐。七担八柳缠到这咱晚。”因问:“二娘今日不家去罢了?”玉楼道:“二娘再三不肯,要去,被俺众姐妹强着留下。”李瓶儿道:“家里没人,奴不放心。”西门庆道:“没的扯淡,这两日好不巡夜的甚紧,怕怎的!但有些风吹草动,拿我个帖儿送与周大人,点到奉行。”又道:“二娘怎的冷清清坐着?用了些酒儿不曾?”孟玉楼道:“俺众人再三劝二娘,二娘只是推不肯吃。”西门庆道:“你们不济,等我劝二娘。二娘好小量儿!”李瓶儿口里虽说:“奴吃不去了。”只不动身。一面分付丫鬟,从新房中放桌儿,都是留下伺候西门庆的嗄饭菜蔬、细巧果仁,摆了一张桌子。吴大妗子知局,推不用酒,因往李娇儿房里去了。当下李瓶儿上坐,西门庆关席,吴月娘在炕上跐着炉壶儿。孟玉楼、潘金莲两边打横。五人坐定,把酒来斟,也不用小锺儿,都是大银衢花锺子,你一杯,我一盏。常言:风流茶说合,酒是色媒人。吃来吃去,吃的妇人眉黛低横,秋波斜视。正是:

\[
两朵桃花上脸来,眉眼施开真色相。
\]

月娘见他二人吃得饧成一块,言颇涉邪,看不上,往那边房里陪吴大妗子坐去了,由着他四个吃到三更时分。李瓶儿星眼乜斜,立身不住,拉金莲往后边净手。西门庆走到月娘房里,亦东倒西歪,问月娘打发他那里歇。月娘道:“他来与那个做生日,就在那个房儿里歇。”西门庆道:“我在那里歇?”月娘道:“随你那里歇,再不你也跟了他一处去歇罢。”西门庆忍不住笑道:“岂有此理!”因叫小玉来脱衣:“我在这房里睡了。”月娘道:“就别要汗邪,休要惹我那没好口的骂出来!你在这里,他大妗子那里歇?”西门庆道:“罢,罢!我往孟三儿房里歇去罢于是往玉楼房中歇了。

潘金莲引着李瓶儿净了手,同往他前边来,就和姥姥一处歇卧。到次日起来,临镜梳妆,春梅伏侍。他因见春梅灵变,知是西门庆用过的丫头,与了他一副金三事儿。那春梅连忙就对金莲说了。金莲谢了又谢,说道:“又劳二娘赏赐他。”李瓶儿道:“不枉了五娘有福,好个姐姐!”梳妆毕,金莲领着他同潘姥姥,叫春梅开了花园门,各处游看。李瓶儿看见他那边墙头开了个便门,通着他那壁,便问:“西门爹几时起盖这房子?”金莲道:“前者阴阳看来,说到这二月间兴工动土,要把二娘那房子打开,通做一处,前面盖山子卷棚,展一个大花园;后面还盖三间玩花楼,与奴这三间楼做一条边。”这李瓶儿听了在心。只见月娘使了小玉来请后边吃茶。三人同来到上房。吴月娘、李娇儿、孟玉楼陪着吴大妗子,摆下茶等着哩。众人正吃点心,只见冯妈妈进来,向袖中取出一方旧汗巾,包着四对金寿字簪儿,递与李瓶儿。李瓶儿先奉了一对与月娘,然后李娇儿、孟玉楼、孙雪娥每人都是一对。月娘道:“多有破费二娘,这个却使不得。”李瓶儿笑道:“好大娘,甚么稀罕之物,胡乱与娘们赏人便了。”月娘众人拜谢了,方才各人插在头上。月娘道:“闻说二娘家门首就是灯市,好不热闹。到明日我们看灯,就往二娘府上望望,休要推不在家。”李瓶儿道:“奴到那日,奉请众位娘。”金莲道:“姐姐还不知,奴打听来,这十五日是二娘生日。”月娘道:“今日说过,若是二娘贵降的日子,俺姊妹一个也不少,来与二娘祝寿。”李瓶儿笑道:“蜗居小室,娘们肯下降,奴一定奉请。”不一时吃罢早饭,摆上酒来饮酒。看看留连到日西时分,轿子来接,李瓶儿告辞归家。众姐妹款留不住。临出门,请西门庆拜见。月娘道:“他今日早起身,出门与人家送行去了。”妇人千恩万谢,方才上轿来家。正是:

\[
合欢核桃真堪爱,里面原来别有仁。
\]
