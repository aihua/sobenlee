%# -*- coding:utf-8 -*-
%%%%%%%%%%%%%%%%%%%%%%%%%%%%%%%%%%%%%%%%%%%%%%%%%%%%%%%%%%%%%%%%%%%%%%%%%%%%%%%%%%%%%
%%  MAChapter20.tex'


\chapter{重积分}\label{ch:20}

\section{引\emspace 言}

\section{$\mR{m}$~空间图形的~Jordan~测度}
\subsection{$\mR{m}$~中图形的容积}
\subsection{点集为可测图形的充要条件}
\subsection{点集为可测图形的充分条件}
\begin{exercise}
\item 试求~$\mR{2}$~中点集
\[
  E=\mathsetb{(x,y)}{(x,y)\in\mintc01\times\mintc01,\text{且~$x$~与~$y$~至少有一个是无理数}}
\]
的内容积与外容积。并讨论~$E$~是否是可测图形。
\item 设~$A,B,C$~是~$\mR{m}$~上的可测图形。证明,
\begin{exlistcols}
  \item $V\mparen{A\difset B}=V(A)-V(A\cap B)$;
  \item $V(A\cup B)=V(A)+V(B)-V(A\cap B)$;
  \item $V(A\cup B\cup C)=V(A)+V(B)+V(C)-V(A\cap B)-V(A\cap C)-V(B\cap C)-V(A\cap B\cap C)$。
\end{exlistcols}
\item 距离说明~$\mR{m}$~中两个点集~$E_1$~和~$E_2$~都不是可测图形,但~$E_1\cup E_2$~和~$E_1\cap E_2$~都是可测图形。并讨论
是否还可能有~$E_1\difset E_2$~也为可测图形。
\item 设有一点集
\[
  D=\mathsetb{(x,y)}{\mabs x\leq 2,~\mabs y\leq 2}。
\]
如果把~$D$~看成~$\mR{3}$~上的点集,它的容积~$V^{(3)}(D)$~是多少;如果把~$D$~看成~$\mR{2}$~上的点集,它的容积~$V^{(2)}(D)$~又
是多少。这是否有矛盾,试加以说明。
\item 设~$A$~为~$\mR{m}$~上一可测图形。证明~$\interior A$~和~$\closure A$~为可测图形,而且
~$V\mparen{\interior A}=V\mparen{\closure A}=V(A)$。
\item 设~$\Omega$~是~$\mR{m}$~上的点集。已知~$\closure\Omega$~是可测图形,证明或反驳~$\Omega$~是可测图形。
\item 举例说明,$\mR{m}$~中两个点集~$S_1$~和~$S_2$~都不是可测图形,但~$S_1\difset S_2$~为可测图形。
\item 设~$A$~和~$B$~是~$\mR{m}$~上的可测图形,而且~$B$~的容积为零。证明~$V(A\cup B)=V(A\difset B)=V(A)$。
\item 设~$A,B$~为~$\mR{m}$~上的两个可测图形,试给出~$V\mparen{A\difset B}=V(A)-V(B)$~成立的充分必要条件。
\end{exercise}

\section{在~$\mR{m}$~上的~Riemann~积分}
\subsection{$m$~重积分定义}
\subsection{可积的充要条件}
\subsection{可积函数类}
\subsection{可积函数的性质}
\begin{exercise}
\item 在~$\mR{2}$~的区域~$D=\mathsetb{(x,y)}{\mabs x\leq 1,~\mabs y\leq 1}$~上给定一函数
\[
  f(x,y)=\begin{Bdcases}
    1, & \text{当~$x,y$~都是有理数;}\\
    0, & \text{当~$x,y$~至少有一个是无理数 。}
  \end{Bdcases}
\]
讨论~$f(x,y)$~在区域~$D$~上的可积性。
\item 设~$\Omega$~和~$S$~是~$\mR{3}$~上的可测图形。函数
\[
  f(x,y,z)=\begin{Bdcases}
    1, & (x,y,z)\in S;\\
    0, & (x,y,z)\in\Omega\difset S 。
  \end{Bdcases}
\]
讨论~$f(x,y,z)$~在~$\Omega$~上的可积性。
\item 设函数~$\map f{A\subset\mR{m}}{\mR}$~在~$A$~上有界且可积,又设~$B\subset A$~为一可测图形。证明~$f$~的限制~$f\mrest{B}$~
在~$B$~上可积。
\item 在~$\mR{2}$~的点集
\[
  D=\mparenb{\mintc01\times\mintc01}\txts\bigcup\mparenb{\mbrace{0}\times\mintc12}
\]
上给定一函数
\[
  f(x,y)=\begin{cdcases}
    1, & (x,y)\in\mintc01\times\mintc01;\\
    \dfrac1{2-y} & (x,y)\in \mbrace{0}\times\minto12;\\
    0, & (x,y)=(0,2) 。
  \end{cdcases}
\]
证明,
\begin{exlistcols}[3]
  \item $D$~是~$\mR{2}$~上的可测图形;
  \item $f(x,y)$~在~$D$~上无界;
  \item $f(x,u)$~在~$D$~上可积。
\end{exlistcols}
\item 设开集~$A$~为~$\mR{m}$~上的可测图形,函数~$\map fA\mR$~在~$A$~上非负连续且不恒为零。证明
\[
  \int_Af\dif V>0 。
\]
如果~$A$~不是开集,上述结论是否仍成立,试举例说明。
\item\label{exer-20.3.6}%
设开区域~$\Omega$~是~$\mR{m}$~上的可测图形,函数~$\map f{\closure\Omega}\mR$~连续。如果对于任一可测图形~$B\subset\Omega$,都有
\[
  \int_Bf\dif V=0,
\]
证明~$f$~在~$\closure\Omega$~上恒等于零。
\item 试举例说明\ref{exer-20.3.6}~去掉~$\Omega$~为开区域的条件是否,结论是否依然成立。
\item 设开区域~$\Omega$~是~$\mR{m}$~上的可测图形,函数~$\map f{\closure\Omega}\mR$~连续。如果对任意一个在~$\closure\Omega$~上连续
而且在~$\bound\Omega$~上取值为零的函数~$\eta$,都有
\[
  \int_\Omega f\eta\dif V=0,
\]
证明~$f$~在~$\closure\Omega$~上恒为零。如果去掉~$\Omega$~为开区域的条件,讨论结论是否仍成立。
\item 设~$A,B$~为~$\mR{m}$~上的可测图形,且~$A\cap B$~的容积为零,又设~$f$~在~$A\cup B$~上可积。证明,
\[
  \int_{A\cup B}f\dif V=\int_Af\dif V+\int_Bf\dif V 。
\]
\item 设定义在可测图形~$\Omega\subset\mR{m}$~上的两个函数~$f,g$~有界可积,而且~$g$~在~$\Omega$~上非负。令,
\[
  m\coloneq\inf_{\mvec x\in\Omega}\mbraceb{\vecfunc f{x}},\quad
  M\coloneq\sup_{\mvec x\in\Omega}\mbraceb{\vecfunc f{x}}。
\]
\begin{exlist}\FixExHead
  \item 函数
  \[
    F(t)\coloneq\int_\Omega\mparenb{\vecfunc f{x}-t}\vecfunc g{x}\dif V
  \]
  是~$\mintc mM$~上的连续函数;
  \item\label{exer-20.3.10-2}$F(t)$~在~$\mintc mM$~上至少有一个零点;
  \item 利用~\ref{exer-20.3.10-2}~的结论,证明存在~$\mu\in\mintc mM$,使得
  \[
    \int_\Omega fg\dif V=\mu\int_\Omega g\dif V 。
  \]
\end{exlist}
\end{exercise}

\section{化重积分为累次积分}
\subsection{化重积分为累次积分的公式}
\subsection{二重积分计算}
\subsection{三重积分计算}
\subsection{$n$~重积分计算}
\begin{exercise}
\item 设一元函数~$f(x)$~和~$g(y)$~分别在~$\mintc ab$~与~$\mintc cd$~上可积。证明,二元函数~$F(x,y)\coloneq f(x)g(y)$~
在~$D=\mintc ab\times\mintc cd$~上可积,而且
\[
  \iint_DF(x,y)\dif x\dif y=\int_a^bf(x)\dif x\int_c^dg(y)\dif y 。
\]
\item 设一元函数~$f_i(x_i)$~在~$\mintc{a_i}{b_i}$~上可积,$i=1,2,\dotsc,n$。证明,$n$~元函数
\[
  F(\mvec x{1,2,:,n})\coloneq \prod_{i=1}^nf_i(x_i)
\]
在~$\Omega=\mintc{a_1}{b_1}\times\dotsb\times\mintc{a_n}{b_n}$~上可积,且
\[
  \int_\Omega F\dif V=\prod_{i=1}^n\int_{a_i}^{b_i}f_i(x_i)\dif x_i 。
\]
\item 设~$\Omega=\mintc{a_1}{b_1}\times\dotsb\times\mintc{a_n}{b_n}$,且~$f$~在~$\Omega$~上连续。令
~$\Omega_{\mvec x}\coloneq\mintc{a_1}{x_1}\times\dotsb\times\mintc{a_n}{x_n}$,其中~$x_i\in\mintc{a_i}{b_i}$。定义
\[
  \vecfunc F{x}\coloneq\int_{\Omega_{\mvec x}}f\dif V 。
\]
证明~$\vecfunc F{x}$~在~$\interior\Omega$~上有连续的一阶偏导数,并求出~$F_{x_i}'\mcond{i=1,\dotsc,n}$。
\item 设在~$D=\mintc ab\times\mintc cd$~上定义的二元函数~$f(x,y)$~有二阶连续偏导数。
\begin{exlist}
  \item\label{exer-20.4.4-1}证明~$\iint_Df_{xy}''(x,y)\dif x\dif y=\iint_Df_{yx}''(x,y)\dif x\dif y$;
  \item 利用~\ref{exer-20.4.4-1}~证明~$f_{xy}''(x,y)=f_{yx}''(x,y)$~在~$D$~上恒成立。
\end{exlist}
\item 设一元函数~$f(x)$~在~$\mintc ab$~上可积。在~$D=\mintc ab\times\mintc ab$~上定义~$F(x,y)\coloneq\mparenb{f(x)-f(y)}^2$。
\begin{exlist}
  \item 将重积分~$\iint_DF(x,y)\dif x\dif y$~化为累次积分;
  \item 证明~$\mparenbb{\int_a^bf(x)\dif x}^{\msp2}\leq(b-a)\int_a^bf^2f(x)\dif x$。
\end{exlist}
\item 对下列区域依两种不同顺序将二重积分~$\iint_Df(x,y)\dif x\dif y$~化为累次积分。
\begin{exlist}
  \item $\Omega$~是以~$A_1\minto{a_1}{b_1}$,$A_2\minto{a_2}{b_2}$~和~$A_3\minto{a_3}{b_3}$~为顶点的
  三角形,其中~$a_1<a_2$,$b_1<b_2$;
  \item $\Omega$~是以~$A\minto00$,$B\minto10$~和~$C\minto{-1}1$~为顶点的三角形;
  \item $\Omega$~是以~$A\minto a0$,$B\minto{2a}{2a}$,$C\minto{2a}{4a}$~和~$D\minto a{2a}$~为定点的四边形,其中~$a>0$;
\begin{exlistcols*}
  \item $\Omega$~是圆域~$\mparen{x-a}^2+\mparen{y-b}^2\leq R^2$;
  \item $\Omega$~是圆域~$x^2+y^2\leq 2x$;
\end{exlistcols*}
  \item $\Omega$~是环域~$R_1^2\leq\mparen{x-a}^2+\mparen{y-b}^2\leq R_2^2$,其中~$R_2>R_1$;
  \item $\Omega$~是由~$x^2+y^2=R^2$~和~$x^2+y^2=Rx$~围成的区域,其中~$R>0$;
  \item $\Omega$~是由~$y=x^3$,$y=2x^3$,$y=1$~和~$y=2$~围成的区域;
  \item $\Omega$~是由~$y=\sqrt[5]x$,$y=\sqrt[5]{\sfrac x3}$,$x=-1$~和~$x=1$~围成的区域。
\end{exlist}
\item 改变下述二重积分化为累次积分的顺序。
\begin{exlistcols}
  \item $\int_0^1\dif y\int_0^yf(x,y)\dif x$;
  \item $\int_1^\me\dif x\int_0^{\ln x}f(x,y)\dif y$;
  \item $\int_0^2\dif y\int_{y^2}^{3y}f(x,y)\dif x$;
  \item $\int_{-1}^1\dif x\int_{-\smbsqrt{1-x^2}}^{1-x^2}f(x,y)\dif y$;
  \item $\int_1^2\dif x\int_{\sqrt x}^2f(x,y)\dif y$;
  \item $\int_1^2\dif x\int_{2-x}^{\sqrt{2x-a}}f(x,y)\dif y$,其中~$0<a<1$;
  \item $\int_0^\pi\dif y\int_{\sin\frac y2}^{\sin y}f(x,y)\dif x$。
\end{exlistcols}
\item 证明下述~Dirichlet~公式,其中~$a>0$。
\[
  \int_0^a\dif x\int_0^xf(x,y)\dif y=\int_0^a\dif y\int_y^af(x,y)\dif x 。
\]
\item 设一元函数~$g(x)$~在~$\mintc01$~上可积。证明
\[
  \int_0^1\dif x\int_x^1g(t)\dif t=\int_0^1tg(t)\dif t 。
\]
\item 设~$m,n\in\mN$,且至少其中之一为奇数,而~$\Omega\subset\mR{2}$~是椭圆域~$\dfrac{x^2}{a^2}+\dfrac{y^2}{b^2}\leq1$。证明
\[
  \iint_\Omega x^my^n\dif x\dif y=0 。
\]
\item 设~$m,n\in\mN$,且都是偶数,而~$\Omega\subset\mR{2}$~是椭圆域~$\dfrac{x^2}{a^2}+\dfrac{y^2}{b^2}\leq1$,%
$\Omega^\ast$~为~$\Omega$~位于第一象限的部分(包含坐标轴)。证明
\[
  \iint_\Omega x^my^n\dif x\dif y=4\iint_{\Omega^\ast}x^my^n\dif x\dif y 。
\]
\item\begin{exlist}
  \item 设~$\Omega$~是由~$y^2=2px\mcond{p>0}$~与~$x=\dfrac p2$~围成的区域,计算~$\iint_\Omega x^my^k\dif x\dif y$,其中~$m,k>0$;
  \item 设~$\Omega$~是由~$y=0$,$y=\sin x^2$,$x=0$~和~$x=\sqrt\pi$~围成的区域,计算~$\iint_\Omega x\dif x\dif y$;
  \item 设~$\Omega=\mathsetb{\minto xy}{0\leq x\leq y^2,0\leq y\leq 2,x\leq 2}$,计算~$\iint_\Omega x^2y^2\dif x\dif y$;
  \item 设~$\Omega$~是由~$y=\sqrt{1-x^2}$~与~$y=0$~围成的区域,计算~$\iint_\Omega\mparenb{x^2+3xy^2}\dif x\dif y$;
  \item 设~$\Omega$~是由~$y=\me^x$,$y=1$,$x=0$~与~$x=1$~围成的区域,计算~$\iint_\Omega\mparen{x+y}\dif x\dif y$;
  \item 设~$\Omega$~是以~$\minto22$,$\minto23$~和~$\minto31$~为顶点的三角形,计算~$\iint_\Omega\me^{x+y}\dif x\dif y$;
  \item 设~$\Omega$~是以~$\minto11$,$\minto23$,$\minto31$~和~$\minto43$~为顶点的四边形,计算
  ~$\iint_\Omega\mparenb{x^2+y^2}\dif x\dif y$;
  \item 设~$\Omega$~是由~$y=x^2$,$y=4x$~和~$y=4$~围成的区域,计算~$\iint_\Omega\sin nx\dif x\dif y$。
\end{exlist}
\item 设函数~$\map f{\mintc01\times\mintc01}\mR$~定义如下,
\[
  f(x,y)\coloneq\begin{cBdcases}
    1, & \text{$x=\dfrac{p_1}{q_2}$,$y=\dfrac{p_2}{q_2}$,且~$q_1\neq q_2$};\\
    0, & \text{其它情形},
  \end{cBdcases}
\]
其中~$\mgcd{p_1}{q_1}=1$,$\mgcd{p_2}{q_2}=1$。证明~$f(x,y)$~在~$\mintc01\times\mintc01$~上不可积,但两个累次积分都存在。
\item\begin{exlist}
  \item 设~$\Omega$~为椭球域,计算~$\iiint_\Omega \mparen{x+y+z}\dif x\dif y\dif z$;
  \item 设~$\Omega$~是由曲面~$x^2+y^2+z^2=1$,$x=0$,$y=0$~与~$z=0$~围成的位于第一卦限上的有界区域,计算
  ~$\iiint_\Omega x^3yz\dif x\dif y\dif z$;
  \item 设~$\Omega$~由曲面~$z=x^2+y^2$,$z=1$~与~$z=2$~围成,计算~$\iiint_\Omega z\dif x\dif y\dif z$;
  \item 设~$\Omega$~由曲面~$x^2=z^2+y^2$,$x=2$~与~$x=4$~围成,计算~$\iiint_\Omega\mparenb{1+x^4}\dif x\dif y\dif z$;
  \item 设~$\Omega$~由~$z=16\mparenb{x^2+y^2}$,$z=4\mparenb{x^2+y^2}$~与~$z=64$~围成,计算
  ~$\iiint_\Omega\mparenb{x^2+y^2}\dif x\dif y\dif z$。
\end{exlist}
\item 改变下列三重积分化为累次积分的顺序。
\begin{exlistcols}
  \item $\int_0^1\dif x\int_0^{1-x}\dif y\int_0^{x+y}f(x,y,z)\dif z$;
  \item $\int_0^1\dif x\int_0^1\dif y\int_0^{x^2+y^2}f(x,y,z)\dif z$;
  \item $\int_{-1}^1\dif x\int_{-\sqrt{1-x^2}}^{\sqrt{1-x^2}}\dif y\int_{\smbsqrt{x^2+y^2}}^1f(x,y,z)\dif z$;
  \item $\int_{-b}^b\dif z\int_{-\frac ab\sqrt{b^2-z^2}}^{\frac ab\sqrt{b^2-z^2}}\dif y
         \int_0^{\frac{y^2}{a^2}+\frac{z^2}{b^2}}f(x,y,z)\dif x$;
  \item $\int_{-a}^a\dif x\int_{-\frac ca\sqrt{a^2-x^2}}^{\frac ca\sqrt{a^2-x^2}}\dif z
         \int_0^{b\sqrt{1-(\sfrac xz)^2-(\sfrac zc)^2}}f(x,y,z)\dif y$;
  \item $\int_0^1\dif y\int_{-y}^y\dif z\int_{-\smbsqrt{y^2-z^2}}^{\smbsqrt{y^2-z^2}}f(x,y,z)\dif x$。
\end{exlistcols}
\item 平面区域~$D$~的面积~$A$~由公式
\[
  A=\int_0^a\dif x\int_{a-x}^{\sqrt{a^2-x^2}}\dif y
\]
给出,确定~$D$~是由哪些曲线围成的。
\item 空间一例题~$\Omega$~的体积由公式
\[
  V(\Omega)=\int_0^1\dif x\int_0^{\sqrt{1-x^2}}\dif y\int_{\smbsqrt{x^2+y^2}}^{2-\smbsqrt{x^2+y^2}}\dif z
\]
给出,确定~$\Omega$~是由哪些曲面围成的。
\item 设~$\Omega$~是椭球域~$\dfrac{x^2}{a^2}+\dfrac{y^2}{b^2}+\dfrac{z^2}{c^2}\leq1$。证明,
\[
  \iiint_\Omega f(x,y,z)\dif y\dif y\dif z=\int_{-c}^c\mparenbb{\iint_{D_z}f(x,y,z)\dif x\dif y}\dif z,
\]
其中~$D_z$~为~$\Omega$~的横截面,即
\[
  D_z=\mathsetBB{\minto xy}{\frac{x^2}{a^2\mparenB{1-\dfrac{z^2}{c^2}}}+\frac{y^2}{b^2\mparenB{1-\dfrac{z^2}{c^2}}}\leq 1} 。
\]
\item 求下列立体~$\Omega$~的体积。
\begin{exlist}
  \item $\Omega$~由曲面~$z=xy$,$x+y+z=1$~和~$z=a$~围成,其中~$a>0$;
  \item $\Omega$~由~$z=\cos x\cos y$,$z=0$,$\mabs{x+y}\leq\dfrac\pi2$~和~$\mabs{x-y}\leq\dfrac\pi2$~围成。
\end{exlist}
\item 设~$f(x,y,z)=F_{xyz}'''$~在~$D=\mintc ab\times\mintc cd\times\mintc hk$~上连续。求
\[
  I=\iiint_Df(x,y,z)\dif x\dif y\dif z 。
\]
并证明~$I\leq 4\mparen{M-m}$,其中~$M,m$~分别是~$F(x,y,z)$~在~$D$~上的最大值和最小值。
\item 证明,若~$a,b>0$,则
\begin{exlistcols}
  \item $\lim_{T\to\pinf}\int_0^T\dif x\int_a^b\me^{-xy}\dif y=\ln\dfrac ba$;
  \item $\int_0^\pinf\dfrac{\me^{-ax}-\me^{-bx}}x\dif x=\ln\dfrac ba$。
\end{exlistcols}
\item 设~$f(t)$~在~$\mintco0\pinf$~上连续可微,且~$\int_1^\pinf\dfrac{f(t)}t\dif t$~收敛。证明,若~$a,b>0$,则
\begin{exlistcols}
  \item $\lim_{T\to\pinf}\int_0^T\dif x\int_b^af'(xy)\dif y=f(0)\ln\dfrac ba$;
  \item $\int_0^\pinf\dfrac{f(ax)-f(bx)}x\dif x=f(0)\ln\dfrac ba$。
\end{exlistcols}
\end{exercise}

\section{重积分的变量替换}
\subsection{正则变换及其性质}
\subsection{重积分的变量替换}
\subsection{极坐标变换}
\subsection{二重积分的其它变换}
\begin{exercise}
\item 证明变换~$T\colon y_1=x_1$,$y_2=x_2+\phi_1(x_1)$,$y_3=x_3+\phi_2(x_1,x_2)$~是正则变换,其中~$\phi_1$~与~$\phi_2$~有连续
的一阶偏导数。
\item 已知变换~$T\colon y_1=x_1^2-x_2^2$,$y_2=2x_1x_2$。
\begin{exlist}
\item 判断~$T$~是否是~$\mR{2}$~上的正则变换;
\item 证明~$T$~是区域~$\Omega=\minto0\pinf\times\minto0\pinf$~上的正则变换。
\end{exlist}
\item 已知变换~$T\colon u=\me^x\sin y$,$v=\me^x\sin y$。
\begin{exlist}
  \item 设~$D=\mintc01\times\mintc0{4\pi}$,求~$T(D)$;
  \item\label{exer-20.5.3-2}计算~$\iint_D\mabsb{\det\Dif T(x,y)}\dif x\dif y$~和~$\iint_{T(D)}\!\!\dif u\dif v$,并判断
  两者是否相等;
  \item 证明~$\det\Dif T(x,y)$~在~$\mR{2}$~上恒正,并考虑~\ref{exer-20.5.3-2}~的结论的意义。
\end{exlist}
\item 已知变换~$T\colon x=u-\dfrac14\mparen{u+v}^2$,$y=\dfrac12\mparen{u+v}$。又设~$B$~是~$xy$~平面上由曲线
~$x=-y^2$,$x=2y-y^2$~和~$x=2-y^2-2y$~围成的区域。
\begin{exlistcols}
\item 判断~$T$~是否是~$uv$~平面上的正则变换;
\item 求~$T^{-1}(B)$;
\item 计算~$\iint_Bx\dif x\dif y$~和~$\iint_{T^{-1}(B)}x\minto uv\mabsbb{\pdiff{\minto xy}{{\minto uv}}}\dif u\dif v$,并判断
两者是否相等。
\end{exlistcols}
\item 已知变换~$T\colon\xi_i=a_{i1}x_1+\dotsb+a_{in}x_n$,其中~$i=1,\dotsc,n$,且~$\det\mparenb{a_{ij}}\neq0$。
\begin{exlist}
  \item 证明~$T$~是~$\mR{n}$~上的正则变换;
  \item 设~$\Omega$~是有超平面~$a_{i1}x_1+\dotsb+a_{in}x_n=\pm h_i\mcond{h_i>0,i=1,\dotsc,n}$~围成的~$2n$~面体,求~$T(\Omega)$。
\end{exlist}
\item 证明,$\mR{n}$~中由~$n$~个向量~$\mvec a_1,\dotsc,\mvec a_n$~生成的平行多面体的体积为
~$\sqrt{\mabsb{\det\mparenb{A_{ij}}}}$,其中~$A_{ij}$~为向量~$\mvec a_i$~与~$\mvec a_j$~的内积~$\minp{\mvec a_i}{\mvec a_j}$。
\item 设~$\map {\funcvec f}{A\subset\mR{m}}{\mR{m}}$~是~$C^{(1)}$,而~$A\subset\mR{m}$~为开集。又设~$E$~是~$\interior A$~上
容积为零的点集。证明~$\funcvec f{E}$~的容积为零。
\item 设~$A\subset\mR{m}$~为开集,$\map{\funcvec g}A{E\subset\mR{m}}$~为~$C^{(1)}$~映射。在此假定下,$A$~上任意可测图形~$D$~的
象~$\funcvec g{D}$~也是可测图形。如果
\[
  V(D)=V\mparenb{\funcvec g{D}}
\]
对任一可测图形~$D$~成立,则称~$\funcvec g$~是\emph{保容积映射}。假定~$\funcvec g$~是一一映射且在~$A$~上
~$\det\Dif\funcvec g{x}\neq0$。证明~$\funcvec g$~是保容积映射当且仅当~$\mabsb{\det\Dif\funcvec g{x}}=1$~在~$A$~上恒成立。
\item 证明变换~$x_1=u_1$,$x_2=u_1+u_2$,$\dotsc$,$x_n=u_1+u_2+\dotsb+u_n$~是保容积映射。
\item 设~$S\subset\mR{n}$~为一可测图形,并且~$t>0$,令
\[
  R\coloneq\mathsetb{\mparen{\mvec {tx}{1,2,:,n}}}{\mparen{\mvec x{1,2,:,n}}\in S}。
\]
证明~$R$~是一可测图形,并且~$V(R)=t^nV(S)$。
\item 作极坐标变换,计算下列积分。
\begin{exlist}
  \item 设~$\Omega$~由双纽线~$\mparenb{x^2+y^2}^2=a^2\mparenb{x^2-y^2}\mcond{x\geq0}$~围成,计算
  ~$\iint_\Omega\mparenb{x^2+y^2}\dif x\dif y$;
  \item 设~$\Omega$~由~Archimedes~螺线~$r=\theta$~和半射线~$\theta=\pi$~围成,计算~$\iint_\Omega x\dif x\dif y$;
  \item 设~$\Omega$~由对数螺线~$r=\me^\theta$~和半射线~$\theta=0$,$\theta=\dfrac\pi2$~围成,计算~$\iint_\Omega xy\dif x\dif y$。
\end{exlist}
\item 对~$\iint_\Omega f(x,y)\dif x\dif y$~作极坐标变换,并用两种不同的顺序化为对~$r$~和~$\theta$~的累次积分。
\begin{exlistcols}
  \item $\Omega=\mintc01\times\mintc01$;
  \item $\Omega$~由~$y=x^2$,$y=\sqrt 3x^2$~和~$x=1$~围成;
  \item $\Omega$~由~$x^2+y^2=1$~和~$x^2+y^2-2x-2y+1=0$~围成;
  \item $\Omega$~由~$x^2+y^2=1$,$y\geq0$~和~$y=1-x$~围成。
\end{exlistcols}
\item 作极坐标变换,将下列二重积分化为定积分。
\begin{exlist}
  \item $\iint_\Omega f\txts\mparenb{\smbsqrt{x^2+y^2}}\dif x\dif y$,其中
  \begin{exlistcols}[label=\Ding*]
    \item $\Omega$~为圆域~$x^2+y^2\leq1$;
    \item $\Omega=\mathsetb{\minto xy}{\mabs y\leq\mabs x,\mabs x\leq 1}$;
  \end{exlistcols}
  \item $\iint_\Omega f\mparenB{\dfrac yx}\dif x\dif y$,其中~$\Omega=\mathsetb{\minto xy}{x^2+y^2\leq 4x+2y-4}$。
\end{exlist}
\item 利用二重积分,并作极坐标变换,求由下列曲面围成立体的体积。
\begin{exlist}
  \item $z=xy$,~$x^2+y^2=a^2$,~$z=0$;
  \item $z=x^2+y^2$,~$x^2+y^2=x$,~$x^2+y^2=2x$,~$z=0$;
  \item $x^2+y^2+z^2=a^2$,~$x^2+y^2\leq\mabs x a$,其中~$a>0$;
  \item $x^2+y^2-z=0$,~$\mparenb{x^2+y^2}^2=a^2\mparenb{x^2-y^2}$,~$z=0$。
\end{exlist}
\item 利用二重积分求下列曲面围成立体的体积。
\begin{exlist}
  \item $\dfrac{x^2}{a^2}+\dfrac{y^2}{b^2}+\dfrac{z^2}{c^2}=1$,~$\dfrac{x^2}{a^2}+\dfrac{y^2}{b^2}=\dfrac{z^2}{c^2}\mcond{z>0}$;
  \item $\dfrac{x^2}{a^2}+\dfrac{y^2}{b^2}=\dfrac{z^2}{c^2}$,~$\dfrac{x^2}{a^2}+\dfrac{y^2}{b^2}=\dfrac xa+\dfrac yb$,~$z=0$;
  \item $z^2=xy$,~$x+y=a$,~$x+y=b$,其中~$0<a<b$;
  \item $z=\dfrac{x^2}{a^2}+\dfrac{y^2}{b^2}$,~$xy=a^2$,~$xy=2a^2$,~$y=b^2x$,~$y=2b^2x$,~$z=0$,其中~$0<a<b$;
  \item $\dfrac{x^2}{a^2}+\dfrac{y^2}{b^2}+\dfrac{z^2}{c^2}=1$,~
  $\mparenbb{\dfrac xa}^{\msp\frac23}+\mparenbb{\dfrac xa}^{\msp\frac23}=1$,~$z=0$;
  \item $\mparenbb{\dfrac xa}^{\msp p}+\mparenbb{\dfrac yb}^{\msp p}+\mparenbb{\dfrac zc}^{\msp p}=1$,%
  ~$x=0$,~$y=0$,~$z=0$,其中~$p>0$;
  \item $\mparenbb{\dfrac xa+\dfrac yb}^{\msp p}+\mparenbb{\dfrac zc}^{\msp q}=1$,~$x=1$,~$y=0$,~$z=0$,其中~$p,q>0$。
\end{exlist}
\item 作适当的变量替换,求下列积分,其中~$a,b\in\mR$。
\begin{exlist}
  \item 设~$\Omega$~为椭圆域~$\dfrac{x^2}{a^2}+\dfrac{y^2}{b^2}\leq1$,计算
  ~$\iint_\Omega\txts\sqrt{1-\dfrac{x^2}{a^2}-\dfrac{y^2}{b^2}}\dif x\dif y$;
  \item 设~$\Omega$~是由~$x^4+y^4=1$~围成的区域,计算~$\iint_\Omega\mparenb{x^2+y^2}\dif x\dif y$;
  \item 设~$\Omega$~为圆域~$x^2+y^2\leq 1$,计算~$\iint_\Omega\mabsbb{\dfrac{x+y}{\sqrt2}-x^2-y^2}\dif x\dif y$;
  \item 设~$\Omega$~是由~$a^2x^2+b^2y^2=1$~围成的区域,计算
  ~$\iint_\Omega\mabsbb{\dfrac{ax+by}{\sqrt 2}-a^2x^2-b^2y^2}\dif x\dif y$;
  \item 设~$\Omega$~是由~$y=4x^2$,$y=9x^2$,$x=4y^2$,$x=9y^2$~围成的区域,计算~$\iint_\Omega(x+y)\dif x\dif y$;
  \item 设~$\Omega$~是由~$xy=2$,$xy=4$,$y=x$,$y=2x$~围成的区域,计算~$\iint_\Omega xy\dif x\dif y$。
\end{exlist}
\item 设~$f(x,y)$~在单位圆域~$\Omega$~上有连续的偏导数,且在边界~$\bound\Omega$~上取值为零。对任意~$\e\in\minto01$,记
~$\Omega_\e$~为环域~$\e^2\leq x^2+y^2\leq 1$。证明
\[
  f(0,0)=-\lim_{\e\to0+0}\dfrac1{2\pi}\iint_{\Omega_\e}\dfrac{xf_x'+yf_y'}{x^2+y^2}\dif x\dif y 。
\]
\item 设积分
\[
  I=\iint_D\mparenB{\mparenb{f_x'}^2+\mparenb{f_y'}^2}\dif x\dif y 。
\]
作正则变换~$T\colon x=x(u,v)$,$y=y(u,v)$,且~$\Omega\coloneq T(D)$。若此变换满足~$x_u'=y_v'$~与~$x_v'=y_u'$,证明
\[
  I=\iint_\Omega\mparenB{\mparenb{f_u'}^2+\mparenb{f_v'}^2}\dif x\dif y 。
\]
\end{exercise}

\section{重积分的变量替换(续)}
\subsection{柱坐标变换}
\subsection{球坐标变换}
\subsection{三重积分的其它变换}
\subsection{$n$~重积分的变量替换}
\begin{exercise}
\item 设~$\det\mparenb{a_{ij}}\neq0$,$i,j=1,2,3$,选取适当变换求下列曲面围成的立体的体积。
\begin{exlist}
  \item $\mparenbb{\sum_{i=1}^3a_{i1}x_i}^{\msp2}+\mparenbb{\sum_{i=1}^3a_{i2}x_i}^{\msp2}+
         \mparenbb{\sum_{i=1}^3a_{i3}x_i}^{\msp2}=R^2$;
  \item $\mparenbb{\sum_{i=1}^3a_{i1}x_i}^{\msp2}+\mparenbb{\sum_{i=1}^3a_{i2}x_i}^{\msp2}=
         \mparenbb{\sum_{i=1}^3a_{i3}x_i}^{\msp2}$,~$\sum_{i=1}^3a_{i3}x_i=\pm h$。
\end{exlist}
\item 计算椭球~$\dfrac{x^2}{a^2}+\dfrac{y^2}{b^2}+\dfrac{z^2}{c^2}\leq 1$~的体积。
\item 作柱坐标变换计算下列三重积分。
\begin{exlist}
  \item 设~$\Omega$~是由曲面~$z=x^2+y^2$,$z=4$,$z=16$~围成的立体,计算~$\iiint_\Omega\mparenb{x^2+y^2}^2\dif x\dif y\dif z$;
  \item 设~$\Omega$~是由曲面~$x^2+y^2=9$,$x^2+y^2=16$,$z^2=x^2+y^2$,$z\geq0$~围成的立体,计算
  ~$\iiint_\Omega\mparenb{x^2+y^2}^{\frac32}\dif x\dif y\dif z$。
\end{exlist}
\item 作球坐标变换计算下列三重积分。
\begin{exlist}
  \item 设~$\Omega$~为球体~$x^2+y^2+z^2\leq R^2$,计算~$\iiint_\Omega\mparen{x+y+z}\dif x\dif y\dif z$;
  \item 设~$\Omega$~为球体~$x^2+y^2+z^2\leq 2z$,计算~$\iiint_\Omega\mparenb{x^2+y^2+z^2}^{\frac52}\dif x\dif y\dif z$;
  \item 设~$\Omega$~是由~$x^2+y^2=z^2$~与~$x^2+y^2+z^2=8$~围成的立体,计算~$\iiint_\Omega x^2\dif x\dif y\dif z$。
\end{exlist}
\item 作适当变量替换,求下列三重积分。
\begin{exlist}
  \item\label{exer-20.6.5-1}设~$\Omega$~是由~$z=\dfrac1a\mparenb{x^2+y^2}$,$z=\dfrac1b\mparenb{x^2+y^2}$,$xy=c$,$xy=d$,%
  $y=\alpha x$~与~$y=\beta x$~围成的立体,其中~$0<a<b$,$0<c<d$~而~$0<\alpha<\beta$,计算
  ~$\iiint_\Omega x^2y^2z\dif x\dif y\dif z$;
  \item 设~$\Omega$~同~\ref{exer-20.6.5-1},计算~$\iiint_\Omega x^2yz\dif x\dif y\dif z$;
  \item 设~$\Omega$~是由~$x=az^2$,$x=bz^2$,$x=\alpha y$,$x=\beta y$~以及~$x=h$~围成的立体,其中~$z,h>0$,$0<a<b$~
  且~$0<\alpha<\beta$,计算~$\iiint_\Omega y^4\dif x\dif y\dif z$;
  \item 设~$\Omega$~为椭球体~$\dfrac{x^2}{a^2}+\dfrac{y^2}{b^2}+\dfrac{z^2}{c^2}\leq 1$,计算
  ~$\iiint_\Omega\exp\mbracebb{\sqrt{\dfrac{x^2}{a^2}+\dfrac{y^2}{b^2}+\dfrac{z^2}{c^2}}}\dif x\dif y\dif z$。
\end{exlist}
\item 设一元函数~$f(t)\in C^{(1)}\minto0\pinf$。令
\[
  \Omega_t\coloneq\mathsetbb{\mparen{x,y,z}}{\frac{x^2}{a^2}+\frac{y^2}{b^2}+\frac{z^2}{c^2}\leq t^2},\quad
  F(t)\coloneq \iiint_{\Omega_t}f\mparenB{\frac{x^2}{a^2}+\frac{y^2}{b^2}+\frac{z^2}{c^2}}\dif y\dif y\dif z 。
\]
\begin{exlistcols}
  \item 证明~$F(t)\in C^{(1)}\minto0\pinf$;
  \item 确定~$F'(t)$~的表达式。
\end{exlistcols}
\item 设有一光滑的封闭曲面~$S$,其球坐标方程为~$\rho=f\minto\phi\psi$,其中~$\phi\in\mintc0{2\pi}$~而
~$\psi\in\mintc{-\sfrac\pi2}{\sfrac\pi2}$。证明~$S$~围成立体之体积为
\[
  V=\frac13\int_0^{2\pi}\dif\phi\int_{-\frac\pi2}^{\frac\pi2}f3\minto\phi\psi\cos\psi\dif\psi,
\]
并求出~$f\minto\phi\psi=\sin^4\phi\cdot\sin^4\psi$~时~$V$~的值。
\item 求函数~$f(x,y,z)$~在区域~$\Omega$~上的平均值~$\mbar f$,
\[
  \mbar f\coloneq\frac1{V(\Omega)}\iiint_\Omega f(x,y,z)\dif x\dif y\dif z 。
\]
\begin{exlist}
  \item $f(x,y,z)=\dfrac{x^2}{a^2}+\dfrac{y^2}{b^2}+\dfrac{z^2}{c^2}$,而~$\Omega$~是由曲面
  ~$\dfrac{x^2}{a^2}+\dfrac{y^2}{b^2}+\dfrac{z^2}{c^2}=\dfrac xa+\dfrac yb+\dfrac zc$~围成的立体;
  \item $\Omega$~是由曲面~$\mparenbb{\dfrac xa}^{\msp p}+\mparenbb{\dfrac yb}^{\msp q}+\mparenbb{\dfrac zc}^{\msp r}=1$~围成的
  立体位于第一卦限上的部分区域,而函数~$f(x,y,z)=x^p+y^q$,其中~$p,q,r>0$。
\end{exlist}
\item 在~$\mR{n}$~上求椭球~$\sum_{i=1}^n\dfrac{x_i^2}{a_i^2}\leq1$~的容积。
\item 在~$\mR{n}$~上求曲面~$\sum_{i=1}^n\mparenbb{\sum_{j=1}^na_{ij}x_j}^{\msp2}=h^2$~围成的区域的
容积,其中~$\det\mparenb{a_{ij}}\neq0$,且~$h>0$。
\item 在~$\mR{n}$~上求曲面~$\sum_{j=1}^na_{ij}x_j=\pm h_i$,$i=1,\dotsc,n$~围成的区域的容积,其中
~$\det\mparenb{a_{ij}}\neq0$,且~$h_i>0$。
\item 证明,广义柱坐标变换~$T\colon$
\begin{align*}
  x_1 & = a_1r\cos\theta_1;\\
  x_2 & = a_2r\sin\theta_1\cos\theta_2;\\*
      & \shortvdotswithin{=}
  x_{n-2} & = a_{n-2}r\sin\theta_1\dotsm\sin\theta_{n-3}\cos\theta_{n-2};\\
  x_{n-1} & = a_{n-1}r\sin\theta_1\dotsm\sin\theta_{n-3}\sin\theta_{n-2};\\
  x_n & = a_n z
\end{align*}
是~$G$~上的正则变换,其中众~$a_i>0$,而
\[
  G\coloneq\mathsetB{\mparenb{r,\theta_1,\dotsc,\theta_{n-2},z}}
  {r>0,~\theta_i\in\minto0\pi(i=1,\dotsc,n-3),~\theta_{n-2}\in\minto0{2\pi},~z\in\mR}。
\]
并证明
\[
  \Dif T\mparenb{r,\theta_1,\dotsc,\theta_{n-2},z}=
  \mparenbb{\prod_{i=1}^n a_i}r^{n-2}\sin^{n-3}\sin\theta_1\sin^{n-4}\theta_2\dotsm\sin\theta_{n-3}。
\]
\item 证明~$\mR{n}$~中的球坐标变换~$T\colon$
\begin{align*}
  x_1 & = r\cos\theta_1;\\
  x_2 & = r\sin\theta_1\cos\theta_2;\\*
      & \shortvdotswithin{=}
  x_{n-2} & = r\sin\theta_1\dotsm\sin\theta_{n-3}\cos\theta_{n-2};\\
  x_{n-1} & = r\sin\theta_1\dotsm\sin\theta_{n-3}\sin\theta_{n-2}\cos_{n-1};\\
  x_n & = r\sin\theta_1\dotsm\sin\theta_{n-3}\sin\theta_{n-2}\sin_{n-1}
\end{align*}
是~$G$~上的正则变换,其中
\[
  G\coloneq\mathsetB{\mparenb{r,\theta_1,\dotsc,\theta_{n-2}}}
  {r>0,~\theta_i\in\minto0\pi(i=1,\dotsc,n-2),~\theta_{n-1}\in\minto0{2\pi}}。
\]
并证明
\[
  \Dif T\mparenb{r,\theta_1,\dotsc,\theta_{n-1}}=
  r^{n-1}\sin^{n-2}\sin\theta_1\sin^{n-3}\theta_2\dotsm\sin\theta_{n-2}。
\]
\item 利用~$\mR{n}$~中球坐标变换求半径为~$R$~的球体的容积。
\end{exercise}

\section{重积分在力学上的应用}
\subsection{物体的重心}
\subsection{转动惯量}
\subsection{引力场的位势}
\begin{exercise}
\item 求由圆锥曲面~$z^2=x^2+y^2\mcond{z\geq0}$~和平面~$z=h$~围成的立体的形心(即~$\rho=1$~时立体的重心)。
\item 求抛物面~$x^2+y^2=px$~和平面~$z=h$~所围成的立体的形心,其中~$p,h>0$。
\item 求由三个坐标平面和平面~$\dfrac xz+\dfrac yb+\dfrac zc=1$~围成的立体的形心。
\item 求半球壳~$a^2\leq x^2+y^2+z^2\leq b^2$,$z\geq0$~的形心。
\item 求~$\mR{3}$~质量为~$m$~的均匀长方体~$S=\mintc0a\times\mintc0b\times\mintc0c$~关于~$z$~轴的转动惯量。
\item 一物体~$\Omega$~的密度是均匀的,它关于~$x$~轴,$y$~轴和~$z$~轴的转动惯量分别记为~$I_x,I_y$~和~$I_z$。证明这三个转动惯量
满足“三角不等式”
\[
  I_x+I_y> I_z,\quad I_y+I_z> I_x,\quad I_z+I_x> I_y 。
\]
\item 求质量均匀的圆筒~$\Omega\mathsetb{(x,y,z)}{a^x\leq x^2+y^2\leq b^2,\mabs z\leq h}$~关于~$x$~轴和~$y$~轴的转动惯量。
\item 有一非均匀的球,球的半径为~$R$,求得密度为~$\rho(x,y,z)=\rho(r)$,其中~$r=\smbsqrt{x^2+y^2+z^2}$。$\rho(r)$~随~$r$~的增大
而线性地减少,从球心的~$\rho_0$~减少至球面的~$\rho_1$。
\begin{exlistcols}
  \item 求此球的重心;
  \item 求此球关于一直径的转动惯量。
\end{exlistcols}
\item 求密度均匀的椭球~$\dfrac{x^2}{a^2}+\dfrac{y^2}{b^2}+\dfrac{z^2}{c^2}\leq 1$~的转动惯量:
\begin{exlist}
  \item 关于~$z$~轴的;
  \item 关于过原点而由~$x:y:z=\alpha:\beta:\gamma$~给出的任一轴,其中~$\alpha^2+\beta^2+\gamma^2=1$。
\end{exlist}
\item 求旋转体~$\smbsqrt{x^2+y^2}\leq f(z)\mcond{0<a\leq z\leq b}$~在原点的位势,其中函数~$f(z)$~连续,在~$\mintc ab$~上有正值。
\item 有一半径为~$a$~的半球,球的密度为~$\rho$~(常数)。将球心置于原点并使它整个位于~$xy$~平面的上侧。证明,在点
~$(0,0,z)\mcond{z>a}$~的位势为
\[
  \frac{2\pi\rho}z\mparenB{\mparenb{a^2+z^2}^{\frac32}+a^3-\frac32a^2z}-\frac23\pi\rho z^2。
\]
\end{exercise}

\begin{exercise*}
\item 证明~$\Omega\subset\mR{m}$~为一可测图形当且仅当对任意~$\e>0$,存在两个可测图形
~$W_\e\supset\Omega$,$U_\e\subset\Omega$,使得
\[
  V\mparenb{W_\e\difset U_\e}=V(W_\e)-V(U_\e)<\e 。
\]
如果此条件满足,则有
\[
  V(\Omega)=\inf_{\e>0}\mbraceb{V(W_\e)}=\sup_{\e>0}\mbraceb{V(U_\e)}。
\]
\item 以~$\Omega\subset\mR{m}$~为一可测图形。设~$\mbrace{S_n}$~为开矩形的集合,若~$\bigcup_{n=1}^\sinf S_n\supset\Omega$,则称
~$\mbrace{S_n}$~为~$\Omega$~的一个开矩形覆盖。证明,
\[
  V(\Omega)=\inf\mathsetbb{\sum_{i=1}^\sinf V(S_n)}{\mbraceb{S_n}\text{~为~$\Omega$~的开矩形覆盖}}。
\]
\item 设~$A$~为~$\mR{m}$~的可测图形,并且~$A$~可以表为~$A=\bigcup_{i=1}^\sinf A_i$,其中诸~$A_i$~都为可测图形,且
~$\interior{A_i}\cap\interior{A_j}=\eset\mcond{i\neq j}$。证明~$V(A)=\sum_{i=1}^\sinf V\mparen{A_i}$。
\item 设~$A$~为~$\mR{m}$~的可测图形,并且~$A$~可以表为~$A=\bigcup_{i=1}^\sinf A_i$,其中诸~$A_i$~都为可测图形,且
~$A_{i+1}\supset A_i$。证明~$V(A)=\lim_\ntoinf V\mparen{A_n}$。
\item 设~$\vecfunc{f_k}{x}$~在~$\mR{m}$~的可测图形~$\Omega$~上有界且可积,并且~$\vecfunc{f_k}{x}$~在~$\Omega$~
上一致收敛到~$\vecfunc f{x}$。证明~$\vecfunc f{x}$~在~$\Omega$~上有界且可积,并且
\[
  \int_\Omega f\dif V=\lim_\ntoinf\int_\Omega f_n\dif V=\int_\Omega\mparenB{\lim_\ntoinf f_n}\dif V 。
\]
\item 证明,在球~$B\colon x^2+y^2+z^2\leq R^2$~上的积分
\[
  \iiint_B\frac{\dif x\dif y\dif z}{\smbsqrt{(x-a)^2+(y-b)^2+(z-c)^2}}=
  \frac{4\pi}3\cdot\frac1{\sqrt{a^2+b^2+c^2+\theta R}},
\]
其中~$a^2+b^2+c^2>R^2$,而~$0<\mabs\theta<1$。
\item 设区域~$\Omega$~是~$\mR{2}$~上一可测图形,且~$f(x,y)$~在~$\closure D$~上连续。证明,
\[
  \lim_{\substack{\xi\to\pinf\\ \eta\to\pinf}}
  \iint_\Omega f(x,y)\txts\ln\mparenb{\smbsqrt{(x-\xi)^2+(y-\eta)^2}}\dif x\dif y=0
\]
当且仅当~$\iint_\Omega f(x,y)\dif x\dif y=0$。
\item 设区域~$\Omega$~是~$\mR{3}$~上一可测图形,且~$f(x,y,z)$~在~$\closure \Omega$~上连续。证明,
\[
  \lim_{\substack{\xi\to\pinf\\ \eta\to\pinf\\ \zeta\to\pinf}}
  \iiint_\Omega f(x,y,z)\txts\ln\mparenb{\smbsqrt{(x-\xi)^2+(y-\eta)^2+(z-\zeta)^2}}\dif x\dif y\dif z=0
\]
当且仅当~$\iiint_\Omega f(x,y,z)\dif x\dif y\dif z=0$。
\item 设~$\map f{\mintc01\times\mintc01}\mR$~定义如下
\[
  f(x,y)\coloneq \begin{cdcases}
    0, & x\in\mQ\cap\mintc01,~y\in\mintcob0{\sfrac12};\\
    1, & x\in\mQ\cap\mintc01,~y\in\mintcb{\sfrac12}1;\\
    1, & x\in(\mR\difset\mQ)\cap\mintc01,~y\in\mintcob0{\sfrac12};\\
    0, & x\in(\mR\difset\mQ)\cap\mintc01,~y\in\mintcb{\sfrac12}1。
  \end{cdcases}
\]
讨论~$f(x,y)$~的两种二次定积分存在情形。
\item 证明~Cavaliort~原则,设~$\Omega_1,\Omega_2$~是~$\mR^3$~上的两个可测图形,而且任一平面~$z=z_0$~截出~$\Omega_1$~与~$\Omega_2$~
的平面图形有相同的面积,则~$\Omega_1$~与~$\Omega_2$~有相同的体积。
\item 设~$\map f{A\subset\mR{m}}\mR$~非负且可积。证明~$\mR^{m+1}$~上的点集
\[
  S=\mathsetb{(\mvec x,y)}{\mvec x\in A,~0\leq y\leq\vecfunc f{x}}
\]
是一可测图形,且容积为~$\int_A\vecfunc f{x}\dif V^{(m)}$。
\item 求函数~$f(x,y)\sin^{2n}x\sin^{2n}y\mcond{n\geq 1}$~在区域~$D=\mintc0\pi\times\mintc0\pi$~上的平均值~$I$,
\[
  I\coloneq\frac1{V(D)}\iint_Df(x,y)\dif x\dif y 。
\]
\item 证明
\[
  \int_0^a\dif y\int_0^y\me^{m(a-x)}f(x)\dif x=\int_0^a(a-x)\me^{m(a-x)}f(x)\dif x 。
\]
\item 令~$A=\int_0^1\me^{-t^2}\dif t$,而~$B=\int_0^{\frac12}\me^{-t^2}\dif t$。证明,对于积分
\[
  I=2\int_{-\frac12}^{\frac12}\dif x\int_0^x\me^{-y^2}\dif y,
\]
存在正数~$p$~和~$q$,使得~$I=pA-qB+\me^{-1}-\me^{-\frac14}$。
\item 证明~$f(x,y)=\sgn\mparenb{a^2x^3-y^2+2}$~在~$a^2x^2+y^2\leq 4\mcond{a>0}$~上可积,并求出积分值。
\item 设~$f(x,y)\in C^{(1)}$,其等高线~$f(x,y)=v$~($v$~为参数)是一简单封闭曲线,它围成的区域的面积为~$F(v)$,并设
~$F(v)\in C^{(1)}$。令~$\Omega_{v_1v_2}$~是由曲线~$f(x,y)=v_1$~和~$f(x,y)=v_2$~围成的区域,其中~$v_1<v_2$。证明
\[
  \iint_{\Omega_{v_1v_2}}f(x,y)\dif x\dif y=\int_{v_1}^{v_2}vF'(v)\dif v 。
\]
利用此结论计算积分~$\iint_\Omega\mparenb{x^2+y^2}\dif x\dif y$,其中~$\Omega$~为环域~$a^2\leq x^2+y^2\leq b^2$。
\item 设~$f(x,y,z)\in C^{(1)}$,其等值面~$f(x,y,z)=v$~($v$~为参数)是一简单封闭曲面,它围成的立体体积为~$F(v)$,并设
~$F(v)\in C^{(1)}$。令~$\Omega_{v_1v_2}$~是由曲面~$f(x,y,z)=v_1$~和~$f(x,y,z)=v_2$~围成的立体,其中~$v_1<v_2$。证明
\[
  \iiint_{\Omega_{v_1v_2}}f(x,y,z)\dif x\dif y\dif z=\int_{v_1}^{v_2}vF'(v)\dif v 。
\]
利用上述结果计算~$\Omega\colon a_1\dfrac{x^2}{a^2}+\dfrac{y^2}{b^2}+\dfrac{z^2}{c^2}\leq a_2$~上的积分
\[
  \iiint_\Omega\mparenB{\frac{x^2}{a^2}+\frac{y^2}{b^2}+\frac{z^2}{c^2}}\dif x\dif y\dif z 。
\]
\item 设~$f(x)\in C^{(2)}(\mR)$,且
\[
  M_0=\int_\minf^\pinf\mabsb{f(x)}\dif x<\pinf,\quad
  M_2=\int_\minf^\pinf\mabsb{f''(x)}\dif x<\pinf 。
\]
证明,
\begin{exlistcols}
  \item $M_1=\int_\minf^\pinf\mabsb{f'(x)}\dif x<\pinf$;
  \item $M_2^2\leq4M_0M_1$。
\end{exlistcols}
\item 设~$E$~是~$\mR{2}$~上的单位圆域~$x^2+y^2\leq 1$。令
\[
  F(h)=\iint_\Omega f\txts\mparen{\alpha x+\beta y}\dif x\dif y\mcond*{h=\smbsqrt{\alpha^2+\beta^2}>0}。
\]
\begin{exlist}
  \item 证明~$F(h)=2\int_{-1}^1\txts f(h\xi)\smbsqrt{1-\xi^2}\dif\xi$;
  \item 令~$f(u)=\cos u$,证明~$F(h)=\dfrac{2\pi}hJ_1(h)\mcond{h>0}$,其中~$J_1(h)$~为~Bessel~函数:
  \[
    J_1(h)\coloneq\frac{2h}\pi\int_0^1\txts\mparen{\cos h\xi}\smbsqrt{1-\xi^2}\dif\xi 。
  \]
\end{exlist}
\item 设~$\Omega$~为~$\mR{3}$~上的单位球~$x^2+y^2+z^2\leq 1$。给定积分
\[
  F(h)=\iiint_\Omega\cos\mparenb{\alpha x+\beta y+\gamma z}\dif x\dif y\dif z
  \mcond*{\txts h=\smbsqrt{\alpha^2+\beta^2+\gamma^2}>0}。
\]
证明,
\begin{exlistcols}
  \item $F(h)=\dfrac{4\pi}{h^2}\mparenB{\dfrac{\sin h}h-\cos h}\mcond{h>0}$;
  \item $F(h)$~在~$\mintco0\pinf$~上连续;
  \item $F(h)$~有无穷多个零点。
\end{exlistcols}
\item 设~$\Omega$~是由~$y-x=1$,$y+x=2$~和~$x=0$~围成的区域。将二重积分
\[
  \iint_\Omega\exp\mbracebb{\frac{y-x}{y+x}}\dif x\dif y
\]
化为定积分,并求出积分的近似值,误差不超过~$\num{.01}$。
\item 设~$\Omega$~是由曲面~$x+y+z=1$,$y=0$,$z=0$~与~$z=0$~围成的区域。证明
\[
  \iiint_\Omega x^py^qz^r\mparen{1-x-y-z}^s\dif x\dif y\dif z=
  \frac{\GammaF(p+1)\GammaF(q+1)\GammaF(r+1)\GammaF(s+1)}{\GammaF(p+q+r+s+4)},
\]
其中~$p,q,r,s>0$。
\item 证明,由超曲面~$\sum_{i=1}^{n-1}\dfrac{x_i^2}{a_i^2}=\dfrac{x_n^2}{a_n^2}\mcond{0\leq x_n\leq a_n}$~围成的~$n$~维锥体
的体积为
\[
  V=\frac{\pi^{\frac{n-1}2}}{n\GammaF\mparenB{1+\dfrac1n}}a_1a_2\dotsm a_n 。
\]
\item 设~$0<a<b$,且~$\mR{n}$~上的区域~$\Omega$~为
\[
  \Omega=\mathsetb{(\mvec x{1,:,n})}{a\leq x_1+\dotsb+x_n<b,~x_i\geq0\mcond{i=1,\dotsc,n}}。
\]
证明
\[
  \idotsint_\Omega f\mparen{x_1+\dotsb+x_n}\dif x_1\dotsm\dif x_n=
  \int_a^b f(y)\frac{y^{n-1}}{(n-1)!}\dif y 。
\]
\item 设~$f(x)\in C^{(2)}$。证明二阶差分
\[
  f(x+2h)-2f(x+h)+f(x)=\int_0^h\dif u_2\int_0^hf''\mparen{u_1+u_2+u_3}\dif u_1 。
\]
\item 设~$f(t)$~连续。证明
\[
  \int_0^t\dif t_1\int_0^{t_1}\dif t_2\dotsi\int_0^{t_{n-1}}f(t_1)f(t_2)\dotsm f(t_n)\dif t_n=
  \frac1{n!}\mparenBB{\int_0^tf(\tau)\dif\tau}^{\msp n}。
\]
\item 设~$f(\mvec x{1,2,:,n})$~是在区域~$0\leq x_i\leq x\mcond{i=1,2,\dotsc,n}$~上的连续函数。证明
\[
  \int_0^x\dif x_1\int_0^{x_1}\dif x_2\dotsi\int_0^{x_{n-1}}f\dif x_n=
  \int_0^x\dif x_n\int_{x_n}^x\dif x_{n-1}\dotsi\int_{x_2}^xf\dif x_1,
\]
其中~$n\geq2$。
\item 证明
\[
  \int_0^x\dif x_1\int_0^{x_1}\dif x_2\dotsi\int_0^{x_{n-1}}f(x_n)\dif x_n=
  \int_0^xf(u)\frac{(x-u)^{n-1}}{(n-1)!}\dif u 。
\]
\item 设~$\Omega=\mathsetb{(x_1,\dotsc,x_n)}{x_1^2+\dotsb+x_n^2\leq1}$,且~$m_j\in\mN\mcond{j=1,\dotsc,n}$,令
\[
  I=\idotsint_\Omega\mparenbb{\prod_{j=1}^nx_j^{m_j}}\dif x_1\dotsm\dif x_n 。
\]
\begin{exlist}
  \item 证明,若~$m_1,\dotsc,m_n$~中至少有一个为奇数,则~$I=0$;
  \item 设~$m_j$~都为偶数,计算出~$I$~的值。
\end{exlist}
\item 设~$\Omega$~为~$\mR{n}$~上的单位球体~$x_1^2+\dotsb+x_n^2\leq 1$,令
~$r\coloneq\mparenbb{\sum_{i=1}^nx_i^2}^{\msp\frac 12}$。将~$n$~重积分
\[
  \idotsint_\Omega f(r)\dif x_1\dotsm\dif x_n
\]
化为定积分,其中~$f(u)$~为一元连续函数。
\item 令~$\minto{x_1}{y_1}$,$\minto{x_2}{y_2}$~和~$\minto{x_3}{y_3}$~是面积为~$A$~的三角形的三个顶点,顶点坐标的下标是按
逆时针方向给定的。证明此三角形关于~$x$~轴的转动惯量为
\[
  \frac A6\mparenb{y_1^2+y_2^2+y_3^2+y_1y_2+y_2y_3+y_3y_1}。
\]
\item 椭球的八分之一为
\[
  \Omega\colon\frac{x^2}{a^2}+\frac{y^2}{b^2}+\frac{z^2}{c^2}\leq 1,\enspace x,y,z\geq0 。
\]
求~$\Omega$~的形心的~$x$~坐标。
\item 椭球~$\dfrac{x^2}{a^2}+\dfrac{y^2}{b^2}+\dfrac{z^2}{c^2}\leq 1$~被平面~$\ell x+my+nz=p$~剖分为两块立体,计算这两块立体
的体积。
\item 证明,物体~$\Omega$~在足够远处的位势近似于将其质量集中于它的重心的已质点的位势,其误差小于某一常数除以该点到
重心的距离的平方。
\item 设~$O$~是~$\mR{3}$~上任一点,而~$S$~是任一物体,在所有从~$O$~出发的半射线上取一点~$P$,使得~$P$~与~$O$~点的距离为
~$\dfrac1{\sqrt I}$,其中~$I$~表示~$S$~关于与该射线相重合的直线的转动惯量。证明这些~$P$~点作成一个椭球,称为\emph{惯量椭球}。
\item 求椭球~$\dfrac{x^2}{a^2}+\dfrac{y^2}{b^2}+\dfrac{z^2}{c^2}\leq 1$~在点~$\mparen{\xi,\eta,\zeta}$~处的惯量椭球。
\item 证明
\[
  \int_0^{\frac\pi2}\dif\phi\int_0^{\frac\pi2}\cos\mparenb{2z\sin\phi\sin\theta}\dif\theta=
  \mparenBB{\int_0^{\frac\pi2}\cos\mparenb{z\sin\xi}\dif\xi}^{\msp2}。
\]
\end{exercise*}




\endinput
%%
%% End of file `MAChapter20.tex'.