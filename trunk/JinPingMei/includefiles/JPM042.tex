%# -*- coding:utf-8 -*-
%%%%%%%%%%%%%%%%%%%%%%%%%%%%%%%%%%%%%%%%%%%%%%%%%%%%%%%%%%%%%%%%%%%%%%%%%%%%%%%%%%%%%


\chapter{逞豪华门前放烟火\KG 赏元宵楼上醉花灯}


诗曰:

\[
星月当空万烛烧,人间天上两元宵。
乐和春奏声偏好,人蹈衣归马亦娇。
易老韶光休浪度,最公白发不相饶。
千金博得斯须刻,吩咐谯更仔细敲。
\]

话说西门庆打发乔家去了,走来上房,和月娘、大妗子、李瓶儿商议。月娘道:“他家既先来与咱孩子送节,咱少不得也买礼过去,与他家长姐送节。就权为插定一般,庶不差了礼数。”大妗子道:“咱这里,少不的立上个媒人,往来方便些。”月娘道:“他家是孔嫂儿,咱家安上谁好?”西门庆道:“一客不烦二主,就安上老冯罢。”于是,连忙写了请帖八个,就叫了老冯来,同玳安拿请帖盒儿,十五日请乔老亲家母、乔五太太并尚举人娘子、朱序班娘子、崔亲家母、段大姐、郑三姐来赴席,与李瓶儿做生日,并吃看灯酒。一面吩咐来兴儿,拿银子早定下蒸酥点心并羹果食物。又是两套遍地锦罗缎衣服,一件大红小袍儿、一顶金丝绉纱冠儿、两盏云南羊角珠灯、一盒衣翠、一对小金手镯、四个金宝石戒指儿。十四日早装盒担,叫女婿陈敬济和贲四穿青衣服押送过去。乔大户那边,酒筵管待,重加答贺。回盒中,又回了许多生活鞋脚,俱不必细说。正乱着,应伯爵来讲李智、黄四官银子事,看见,问其所以。西门庆告诉与乔大户结亲之事:“十五日好歹请令正来陪亲家坐坐。”伯爵道:“嫂子呼唤,房下必定来。”西门庆道:“今日请众堂官娘子吃酒,咱每往狮子街房子内看灯去罢。”伯爵应诺去了,不题。

且说那日院中吴银儿先送了四盒礼来,又是两方销金汗巾,一双女鞋,送与李瓶儿上寿,就拜干女儿。月娘收了礼物,打发轿子回去。李桂姐只到次日才来,见吴银儿在这里,便悄悄问月娘:“他多咱来的?”月娘如此这般告他说:“昨日送了礼来,拜认你六娘做干女儿了。”李桂姐听了,一声儿没言语。一日只和吴银儿使性子,两个不说话。

却说前厅王皇亲家二十名小厮,两个师父领着,挑了箱子来,先与西门庆磕头。西门庆吩咐西厢房做戏房,管待酒饭。不一时,周守备娘子、荆都监母亲荆太太与张团练娘子,都先到了。俱是大轿,排军喝道,家人媳妇跟随。月娘与众姊妹,都穿着袍出来迎接,至后厅叙礼。与众亲相见毕,让坐递茶,等着夏提刑娘子到才摆茶。不料等到日中,还不见来。小厮邀了两三遍,约午后才喝了道来,抬着衣匣,家人媳妇跟随,许多仆从拥护。鼓乐接进后厅,与众堂客见毕礼数,依次序坐下。先在卷棚内摆茶,然后大厅上坐。春梅、玉箫、迎春、兰香,都是齐整妆束,席上捧茶斟酒。那日扮的是《西厢记》。

不说画堂深处,珠围翠绕,歌舞吹弹饮酒。单表西门庆打发堂客上了茶,就骑马约下应伯爵、谢希大,往狮子街房里去了。吩咐四架烟火,拿一架那里去。晚夕,堂客跟前放两架。旋叫了个厨子,家下抬了两食盒下饭菜蔬,两坛金华酒去。又叫了两个唱的——董娇儿、韩玉钏儿。原来西门庆已先使玳安雇轿子,请王六儿同往狮子街房里去。玳安见妇人道:“爹说请韩大婶,那里晚夕看放烟火。”妇人笑道:“我羞剌剌,怎么好去的,你韩大叔知道不嗔?”玳安道:“爹对韩大叔说了,教你老人家快收拾哩。因叫了两个唱的,没人陪他。”那妇人听了,还不动身。一回,只见韩道国来家。玳安道:“这不是韩大叔来了。韩大婶这里,不信我说哩。”妇人向他汉子说,“真个叫我去?”韩道国道:“老爹再三说,两个唱的没人陪他,请你过去,晚夕就看放烟火。你还不收拾哩!刚才教我把铺子也收了,就晚夕一搭儿里坐坐。保官儿也往家去了,晚夕该他上宿哩。”妇人道:“不知多咱才散,你到那里坐回就来罢,家里没人,你又不该上宿。”说毕,打扮穿了衣服,玳安跟随,迳到狮子街房里来。来昭妻一丈青早在房里收拾下床炕、帐幔、褥被,安息沉香薰的喷鼻香。房里吊着一对纱灯,笼着一盆炭火。妇人走到里面炕上坐下。一丈青走出来,道了万福,拿茶吃了。西门庆与应伯爵看了回灯,才到房子里。两个在楼上打双陆。楼上除了六扇窗户,挂着帘子,下边就是灯市,十分闹热。打了回双陆,收拾摆饭吃了,二人在帘里观看灯市。但见:

\[
万井人烟锦绣围,香车宝马闹如雷。
鳌山耸出青云上,何处游人不看来?
\]

二人看了一回,西门庆忽见人丛里谢希大、祝实念,同一个戴方巾的在灯棚下看灯,指与伯爵瞧。因问:“那戴方巾的,你可认的他?”伯爵道:“此人眼熟,不认的他。”西门庆便叫玳安:“你去下边,悄悄请了谢爹来。休教祝麻子和那人看见。”玳安小厮贼,一直走下楼来,挨到人闹里,待祝实念和那人先过去了,从旁边出来,把谢希大拉了一把。慌的希大回身观看,却是玳安。玳安道:“爹和应二爹在这楼上,请谢爹说话。”希大道:“你去,我知道了。等我陪他两个到粘梅花处,就来见你爹。”玳安便一道烟去了。希大到了粘梅花处,向人闹处,就叉过一边,由着祝实念和那一个人只顾寻。他便走来楼上,见西门庆、应伯爵两个作揖,因说道:“哥来此看灯,早晨就不呼唤兄弟一声?”西门庆道:“我早晨对众人,不好邀你每的。已托应二哥到你家请你去,说你不在家。刚才,祝麻子没看见么?”因问:“那戴方巾的是谁?”希大道:“那戴方巾的,是王昭宣府里王三官儿。今日和祝麻子到我家,要问许不与先生那里借三百两银子。央我和老孙、祝麻子作保。要干前程,入武学肄业。我那里管他这闲帐!刚才陪他灯市里走了走,听见哥呼唤,我只伴他到粘梅花处,交我乘人乱,就叉开了走来见哥。”因问伯爵:“你来多大回了?”伯爵道:“哥使我先到你家,你不在,我就来了,和哥在这里打了这回双陆。”西门庆问道:“你吃了饭不曾?”谢希大道:“早晨从哥那里出来,和他两个搭了这一日,谁吃饭来!”西门庆吩咐玳安:“厨下安排饭来,与你谢爹吃。”不一时,就是春盘小菜、两碗稀烂下饭、一碗\textuni{24191}肉粉汤、两碗白米饭。希大独自一个,吃的里外干净,剩下些汁汤儿,还泡了碗吃了。玳安收下家活去。希大在旁看着两个打双陆。

只见两个唱的门首下了轿子,抬轿的提着衣裳包儿,笑进来。伯爵在窗里看见,说道:“两个小淫妇儿,这咱才来。”吩咐玳安:“且别教他往后边去,先叫他楼上来见我。”希大道:“今日叫的是那两个?”玳安道:“是董娇儿、韩玉钏儿。”忙下楼说道:“应二爹叫你说话。”两个那里肯来,一直往后走了。见了一丈青,拜了,引他入房中。看见王六儿头上戴着时样扭心\textuni{4BFC}髻儿,身上穿紫潞绸袄儿,玄色披袄儿、白挑线绢裙子,下边露两只金莲,拖的水鬓长长的,紫膛色,不十分搽铅粉,学个中人打扮,耳边带着丁香儿。进门只望着他拜了一拜,都在炕边头坐了。小铁棍拿茶来,王六儿陪着吃了。两个唱的,上上下下把眼只看他身上。看一回,两个笑一回,更不知是什么人。落后,玳安进来,两个悄悄问他道:“房里那一位是谁?”玳安没的回答,只说是:“俺爹大姨人家,接来看灯的。”两个听的,从新到房中说道:“俺每头里不知是大姨,没曾见的礼,休怪。”于是插烛磕了两个头。慌的王六儿连忙还下半礼。落后,摆上汤饭来,陪着同吃。两个拿乐器,又唱与王六儿听。

伯爵打了双陆,下楼来小解净手,听见后边唱,点手儿叫玳安,问道:“你告我说,两个唱的在后边唱与谁听?”玳安只是笑,不做声,说道:“你老人家曹州兵备——管事宽。唱不唱,管他怎的?”伯爵道:“好贼小油嘴,你不说,愁我不知道?”玳安笑道:“你老人家知道罢了,又问怎的?”说毕,一直往后走了。伯爵上的楼来,西门庆又与谢希大打了三贴双陆。只见李铭、吴惠两个蓦地上楼来磕头。伯爵道:“好呀!你两个来的正好,怎知道俺每在这里?”李铭跪下说道:“小的和吴惠先到宅里来,宅里说爹在这边摆酒。特来伏侍爹每。”西门庆道:“也罢,你起来伺候。玳安,快往对门请你韩大叔去。”不一时,韩道国到了,作了揖,坐下。一面放桌儿,摆上春盘案酒来,琴童在旁边筛酒。伯爵与希大居上,西门庆主位,韩道国打横,坐下把酒来筛;一面使玳安后边请唱的去。

少顷,韩玉钏儿、董娇儿两个,慢条斯礼上楼来。望上不当不正磕下头去。伯爵骂道:“我道是谁来,原来是这两个小淫妇儿。头里我叫着,怎的不先来见我?这等大胆!到明日,不与你个功德,你也不怕。”董娇儿笑道:“哥儿那里隔墙掠个鬼脸儿,可不把我唬杀!”韩玉钏儿道:“你知道,爱奴儿掇着兽头城往里掠——好个丢丑儿的孩儿!”伯爵道:“哥,你今日忒多余了。有了李铭、吴惠在这里唱罢了,又要这两个小淫妇做什么?还不趁早打发他去。大节夜,还赶几个钱儿,等住回晚了,越发没人要了。”韩玉钏儿道:“哥儿,你怎么没羞?大爹叫了俺每来答应,又不伏侍你,你怎的闲出气?”伯爵道:“傻小歪剌骨儿,你见在这里,不伏侍我,你说伏侍谁?”韩玉钏道:“唐胖子吊在醋缸里——把你撅酸了。”伯爵道:“贼小淫妇儿,是撅酸了我。等住回散了家去时,我和你答话。我左右有两个法儿,你原出得我手!”董娇儿问道:“哥儿,那两个法儿?说来我听。”伯爵道:“我头一个,是对巡捕说了,拿你犯夜,教他拿了去,拶你一顿好拶子。十分不巧,只消三分银子烧酒,把抬轿的灌醉了,随你这小淫妇儿去,天晚到家没钱,不怕鸨子不打。”韩玉钏道:“十分晚了,俺每不去,在爹这房子里睡。再不,叫爹差人送俺每,王妈妈支钱一百文,不在于你。好淡嘴女又十撇儿。”伯爵道:“我是奴才,如今年程反了,拿三道三。”说笑回,两个唱的在旁弹唱春景之词。

众人才拿起汤饭来吃,只见玳安儿走来,报道:“祝爹来了。”众人都不言语。不一时,祝实念上的楼来,看见伯爵和谢希大在上面,说道:“你两个好吃,可成个人。”因说:“谢子纯,哥这里请你,也对我说一声儿,三不知就走的来了,叫我只顾在粘梅花处寻你。”希大道:“我也是误行,才撞见哥在楼上和应二哥打双陆。走上来作揖,被哥留住了。”西门庆因令玳安儿:“拿椅儿来,我和祝兄弟在下边坐罢。”于是安放锺箸,在下席坐了。厨下拿了汤饭上来,一齐同吃。西门庆只吃了一个包儿,呷了一口汤,因见李铭在旁,都递与李铭下去吃了。那应伯爵、谢希大、祝实念、韩道国,每人吃一大深碗八宝攒汤,三个大包子,还零四个桃花烧卖,只留了一个包儿压碟儿。左右收下汤碗去,斟上酒来饮酒。希大因问祝实念道:“你陪他到那里才拆开了?怎知道我在这里?”祝实念如此这般告说:“我因寻了你一回寻不着,就同王三官到老孙家会了,往许不与先生那里,借三百两银子去,吃孙寡嘴老油嘴把借契写差了。”希大道:“你每休写上我,我不管。左右是你与老孙作保,讨保头钱使。”因问:“怎的写差了?”祝实念道:“我那等吩咐他,文书写滑着些,立与他三限才还。他不依我,教我从新把文书又改了。”希大道:“你立的是那三限?”祝实念道:“头一限,风吹辘轴打孤雁;第二限,水底鱼儿跳上岸;第三限,水里石头泡得烂。这三限交还他。”谢希大道:“你这等写着,还说不滑哩。”祝实念道:“你到说的好,倘或一朝天旱水浅,朝廷挑河,把石头吃做工的两三镢头砍得稀烂,怎了?那时少不的还他银子。”众人说笑了一回。

看看天晚,西门庆吩咐楼上点灯,又楼檐前一边一盏羊角玲灯,甚是奇巧。家中,月娘又使棋童儿和排军,抬送了四个攒盒,都是美口糖食、细巧果品。西门庆叫棋童儿问道:“家中众奶奶们散了不曾?谁使你送来?”棋童道:“大娘使小的来,与爹这边下酒。众奶奶们还未散哩。戏文扮了四折,大娘留在大门首吃酒,看放烟火哩。”西门庆问:“有人看没有?”棋童道:“挤围着满街人看。”西门庆道:“我吩咐留下四名青衣排军,拿杆栏拦人伺候,休放闲杂人挨挤。”棋童道:“小的与平安儿两个,同排军都看放了烟火,并没闲杂人搅扰。”西门庆听了,吩咐把桌上饮馔都搬下去,将攒盒摆上,厨下又拿上一道果馅元宵来。两个唱的在席前递酒。西门庆吩咐棋童回家看去。一面重筛美酒,再设珍羞,叫李铭、吴惠席前弹唱了一套灯词。唱毕,吃了元宵,韩道国先往家去了。少顷,西门庆吩咐来昭将楼下开下两间,吊挂上帘子,把烟火架抬出去。西门庆与众人在楼上看,教王六儿陪两个粉头和一丈青在楼下观看。玳安和来昭将烟火安放在街心里。须臾,点着。那两边围看的,挨肩擦膀,不知其数。都说西门大官府在此放烟火,谁人不来观看?果然扎得停当好烟火。但见:

\[
一丈五高花桩,四周下山棚热闹。最高处一只仙鹤,口里衔着一封丹书,乃是一枝起火,一道寒光,直钻透斗牛边。然后,正当中一个西瓜炮迸开,四下里人物皆着,觱剥剥万个轰雷皆燎彻。彩莲舫,赛月明,一个赶一个,犹如金灯冲散碧天星;紫葡萄,万架千株,好似骊珠倒挂水晶帘。霸玉鞭,到处响亮;地老鼠,串绕人衣。琼盏玉台,端的旋转得好看;银蛾金弹,施逞巧妙难移。八仙捧寿,名显中通;七圣降妖,通身是火。黄烟儿,绿烟儿,氤氲笼罩万堆霞;紧吐莲,慢吐莲,灿烂争开十段锦。一丈菊与烟兰相对,火梨花共落地桃争春。楼台殿阁,顷刻不见巍峨之势;村坊社鼓,仿佛难闻欢闹之声。货郎担儿,上下光焰齐明;鲍老车儿,首尾迸得粉碎。五鬼闹判,焦头烂额见狰狞;十面埋伏,马到人驰无胜负。总然费却万般心,只落得火灭烟消成煨烬。
\]

应伯爵见西门庆有酒了,刚看罢烟火下楼来,因见王六儿在这里,推小净手,拉着谢希大、祝实念,也不辞西门庆就走了。玳安便道:“二爹那里去?”伯爵向他耳边说道:“傻孩子,我头里说的那本帐,我若不起身,别人也只顾坐着,显的就不趣了。等你爹问,你只说俺每都跑了。”落后,西门庆见烟火放了,问伯爵等那里去了,玳安道:“应二爹和谢爹都一路去了。小的拦不回来,多上覆爹。”西门庆就不再问了。因叫过李铭、吴惠来,每人赏了一大巨杯酒与他吃。吩咐:“我且不与你唱钱,你两个到十六日早来答应。还是应二爹三个并众伙计当家儿,晚夕在门首吃酒。”李铭跪下道:“小的告禀爹:十六日和吴惠、左顺、郑奉三个,都往东平府,新升的胡爷那里到任,官身去,只到后晌才得来。”西门庆道:“左右俺每晚夕才吃酒哩。你只休误了就是了。”二人道:“小的并不敢误。”两个唱的也就来拜辞出门。西门庆吩咐:“明日,家中堂客摆酒,李桂姐、吴银姐都在这里,你两个好歹来走一走。”二人应诺了,一同出门,不在话下。西门庆吩咐来昭、玳安、琴童收家活。灭息了灯烛,就往后边房里去了。

且说来昭儿子小铁棍儿,正在外边看放了烟火,见西门庆进去了,就来楼上。见他爹老子收了一盘子杂合的肉菜、一瓯子酒和些元宵,拿到屋里,就问他娘一丈青讨,被他娘打了两下。不防他走在后边院子里顽耍,只听正面房子里笑声,只说唱的还没去哩,见房门关着,就在门缝里张看,见房里掌着灯烛。原来西门庆和王六儿两个,在床沿子上行房。西门庆已有酒的人,把老婆倒按在床沿上,褪去小衣,那话上使着托子干后庭花。一进一退往来\textuni{22D5E}打,何止数百回,\textuni{22D5E}打的连声响亮,其喘息之声,往来之势,犹赛折床一般,无处不听见。这小孩子正在那里张看,不防他娘一丈青走来看见,揪着头角儿拖到前边,凿了两个栗爆,骂道:“贼祸根子,小奴才儿,你还少第二遭死?又往那里张他去!”于是,与了他几个元宵吃了,不放他出来,就唬住他上炕睡了。西门庆和老婆足干捣有两顿饭时才了事。玳安打发抬轿的酒饭吃了,跟送他到家,然后才来同琴童两个打着灯儿跟西门庆家去。正是:

\[
不愁明月尽,自有夜珠来。
\]
